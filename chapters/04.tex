\chapter[Reduplication]{Reduplication}
\label{Para_4}
Reduplication refers to “the morphological operation in which a new word (form) is created by copying a word or a part thereof, and affixing that copy to the base” \citep[321]{Booij.2007}. In Papuan Malay, as in other \ili{Austronesian} languages \citep[121–125]{Himmelmann.2005}, \isi{reduplication} is a very productive morphological device to derive new words.



With respect to \isi{lexeme formation}, Papuan Malay makes use of three different types of \isi{reduplication}: (1) full \isi{reduplication}, (2) partial \isi{reduplication}, and (3) imitative \isi{reduplication}. Alternatively, \citet[558]{Wiltshire.1978} refer to these \isi{reduplication} types as “exact total \isi{reduplication}”, “exact partial \isi{reduplication}”, and “inexact partial \isi{reduplication}”, respectively. In terms of \isi{lexeme interpretation}, a variety of meanings can be attributed to the reduplicated lexemes, such as plurality and diversity, \isi{intensity}, or \isi{continuation} and \isi{repetition}.



Reduplication in terms of \isi{lexeme formation} is described in §\ref{Para_4.1}  while \isi{lexeme interpretation} is discussed in §\ref{Para_4.2}. This discussion is followed by a comparison of \isi{reduplication} across different \ili{eastern Malay varieties} in §\ref{Para_4.3}. The main points of this chapter are summarized in §\ref{Para_4.4}.


\section{Lexeme formation}
\largerpage 
\label{Para_4.1}
A phonological approach to \isi{reduplication} is \citegen[436]{Marantz.1982} pros\-odic template model which views \isi{reduplication} as “normal \isi{affixation}” with “one uniq\-ue feature”, i.e. “the resemblance of the added material to the stem being reduplicated”. More specifically, “every \isi{reduplication} process may be characterized by a ‘skeleton’ of some sort”, either a phonemic melody, “a C-V skeleton, a syllabic skeleton, or a skeleton of morpheme symbols” (\citeyear*[439]{Marantz.1982}). The four-tiered representation in (\ref{Example_4.1}), taken from {\citet[437]{Marantz.1982}}, illustrates how the segments of the four skeleta are linked to each other.

   
\ea
\label{Example_4.1}
\parbox[t]{.8\textwidth}{
\vspace{-.7\baselineskip}
\begin{forest}
for tree={grow=north}
%[μ [σ [C [p1]][V [p2]] [C [p3]]] [σ [C [p4]] [V [p5]]] [σ [C [p6] ] [V [p7]]]]
[{}, phantom [{}, phantom] [μ [σ [V [p7]] [C [p6]]] [σ [V [p5]] [C [p4]]] [σ [C [p3]] [V [p2]] [C [p1]]]]  [\parbox{3cm}{morpheme symbol}, no edge  [\parbox{3cm}{syllabic skeleton }, no edge   [\parbox{3cm}{C-V skeleton}, no edge  [\parbox{3cm}{phonemic melody}, no edge ]]]]]
\end{forest} \\
 p1 = phoneme\\
 σ = syllable\\ 
}
\z 


During \isi{reduplication}, an affixed skeleton receives its phonemic content by “the copying of the stem’s phonemic melody on the same tier as the melody and on the same side of the stem melody to which the affix is attached [\ldots] along with some specific constraints on the autosegmental association of the phonemes of the copied melody with the Cs and Vs of reduplicating morphemes” \citep[445]{Marantz.1982}.



Full and partial \isi{reduplication} use two different types of skeleta. In full \isi{reduplication}, the affix is a morphemic skeleton or, more specifically, the morphological word. In partial \isi{reduplication}, the added material is a syllabic skeleton. In Papuan Malay, this syllabic skeleton is a closed, heavy syllable which gets prefixed to the base. This shows, that in Papuan Malay \isi{reduplication} in general is prefixal rather suffixal.



Both types of \isi{lexeme formation} are described in §\ref{Para_4.1.1.1} and §\ref{Para_4.1.1.2}, respectively. Imitative \isi{reduplication} is discussed in §\ref{Para_4.1.1.3}.


\subsection{Full reduplication}\label{Para_4.1.1}

Full \isi{reduplication} is very common in Papuan Malay. Cross-linguistically, in full \isi{reduplication} “the reduplicant matches the base from which it is copied without phoneme changes or additions” {\citep[2]{Rubino.2013}}. That is, in terms of {Marantz’s (1982) }prosodic template model, full morpheme \isi{reduplication} involves “the addition of a morphemic skeleton to a stem. The morphemic skeleton, lacking a syllabic skeleton, a C-V skeleton, and a phonemic melody, borrows all three from the stem to which it attaches” (\citeyear*[456]{Marantz.1982}).



Full \isi{reduplication} of morphological words in Papuan Malay is illustrated with the two examples in (\ref{Example_4.2}): \isi{reduplication} of the root \textitbf{dorang} ‘\textsc{3pl}’, resulting in \textitbf{dorang{\Tilde}dorang} ‘\textsc{rdp}{\Tilde}\textsc{3pl}’ in (\ref{Example_4.2a}), and \isi{reduplication} of the derived word \textitbf{tingka\-tang} ‘level’ (\textitbf{tingkat}\-\textitbf{ang} ‘floor-\textsc{pat}’), resulting in \textitbf{tingkatang{\Tilde}tingkat\-ang}  
{‘\textsc{rdp\-}{\Tilde}lev\-el’} in (\ref{Example_4.2b}). In each case, the content of the reduplicative affix is obtained by copying the phonemic melody of the base over the morphemic skeleton of the reduplicating affix. This applies to roots as in (\ref{Example_4.2a}) as well as to derived words as in (\ref{Example_4.2b}).
 

\ea
\label{Example_4.2}
\ea
\parbox[t]{3.5cm}{
\vspace{-.5\baselineskip}
\label{Example_4.2a} 
[\textstyleChCharisSIL{ˌdɔ.rɐŋ.ˈdɔ.rɐŋ}]
}\\

% \parbox[t]{\textwidth}{
% \vspace{-.6\baselineskip}
% \resizebox{\textwidth}{!}{
% \vspace{0pt}
\begin{forest}
for tree={grow=north}
[, phantom [{μ} [{σ} [{C} [{ŋ}]] [{V} [{ɐ}]] [{C} [{r}]]] [{σ} [{V} [{ɔ}]] [{C} [{d}]]]]
[, phantom [,phantom [{+}] ]]
[{μ} [{σ} [{C} [{ŋ}]] [{V} [{ɐ}]] [{C} [{r}]]] [{σ} [{V} [{ɔ}]] [{C} [{d}]]]]
 ]
\end{forest} 
% }
% }
\vspace{10pt}
\bigskip

\newpage 
\parbox[t]{3.5cm}{
\ex
\label{Example_4.2b} 
  \vspace{-.7\baselineskip}
  \textstyleChCharisSIL{[tɪŋ.ˌka.tɐŋ.tɪŋ.ˈka.tɐŋ]}
} \\

% \parbox[t]{\textwidth}{
% \vspace{-.7\baselineskip}
% \resizebox{7cm}{!}{
% \vspace{0pt}
\begin{forest}
for tree={grow=north, s sep=0}
[, phantom [{μ},   [{σ} [{C}  [{ŋ}]] [{V} [{ɐ}]] [{C} [{t}]]] [{σ} [{V} [{a}]] [{C} [{k}]]] [{σ} [{C} [{ŋ}]] [{V} [ {ɪ}]] [{C} [{t}]]]]
[, phantom [,phantom [{+}] ]]
[{μ} [{σ} [{C} [{ŋ}]] [{V} [{ɐ}]] [{C} [{t}]]] [{σ} [{V} [{a}]] [{C} [{k}]]] [{σ} [{C} [{ŋ}]] [{V} [{ ɪ}]] [{C} [{t}]]]]]
\end{forest} 
% } 
% }
\z
\z


In Papuan Malay, only words are reduplicated; bound morphemes such as prefixes are never reduplicated (see \tabref{Table_4.1} in §\ref{Para_4.1.1.1}). Full \isi{reduplication} is attested for content words (§\ref{Para_4.1.1.1}) and some function words (§\ref{Para_4.1.1.2}). The corpus also includes a few reduplicated items that do not have an unreduplicated single base (§\ref{Para_4.1.1.3}). Reduplication of reduplicated bases is unattested.


\subsubsection[Reduplication of content words]{Reduplication of content words}\label{Para_4.1.1.1}
Full \isi{reduplication} most commonly applies to content words. Attested are reduplicated nouns, verbs, adverbs, numerals, and quantifiers, as shown in \tabref{Table_4.1}.


\begin{table} 

\caption [Reduplication of content words]{Reduplication of content words\footnote{As discussed in §\ref{Para_4.2}, \isi{reduplication} conveys a variety of different meanings. Hence, a reduplicated item can receive different interpretations, depending on the context. The reduplicated \isi{noun} \textitbf{tulang{\Tilde}tulang}, for instance, can receive the following readings: ‘any one of the bones’, ‘different kinds of bones’, ‘all of the bones’. Hence, no translation is given for the reduplicated items in \tabref{Table_4.1}. The same applies to \tabref{Table_4.2} in §\ref{Para_4.1.1.2}.}}\label{Table_4.1}
\begin{tabular}{llll}
\lsptoprule

 Word class & \multicolumn{1}{c}{Base} & \multicolumn{1}{c}{Gloss} &  \multicolumn{1}{c}{Reduplicated item}\\
\midrule
Nouns & \textitbf{ade} & ‘younger sibling’ & \textitbf{ade{\Tilde}ade}\\
& \textitbf{bua} & ‘fruit’ & \textitbf{bua{\Tilde}buaang}\\
& \textitbf{tingkatang} & ‘level’ & \textitbf{tingkatang{\Tilde}tingkatang}\\
& \textitbf{tulang} & ‘bone’ & \textitbf{tulang{\Tilde}tulang}\\
Verbs & \textitbf{baik} & ‘be good’ & \textitbf{baik{\Tilde}baik}\\
& \textitbf{ceritra} & ‘tell’ & \textitbf{ceritra{\Tilde}ceritra}\\
& \textitbf{talipat} & ‘fold’ & \textitbf{talipat{\Tilde}talipat}\\
& \textitbf{tumpuk} & ‘pile’ & \textitbf{bertumpuk{\Tilde}tumpuk}\\
Adverbs & \textitbf{baru} & ‘recently’ & \textitbf{baru{\Tilde}baru}\\
& \textitbf{skarang} & ‘now’ & \textitbf{skarang{\Tilde} skarang}\\
& \textitbf{sring} & ‘often’ & \textitbf{sring{\Tilde}sring}\\
Numerals & \textitbf{satu} & ‘one’ & \textitbf{satu{\Tilde}satu}\\
& \textitbf{dua} & ‘two’ & \textitbf{dua{\Tilde}dua}\\
& \textitbf{lima} & ‘five’ & \textitbf{lima{\Tilde}lima}\\
Quantifiers & \textitbf{banyak} & ‘many’ & \textitbf{banyak{\Tilde}banyak}\\
& \textitbf{sedikit} & ‘few’ & \textitbf{sedikit{\Tilde}sedikit}\\
& \textitbf{sembarang} & ‘any (kind of)’ & \textitbf{sembarang{\Tilde}sembarang}\\
\lspbottomrule
\end{tabular}

\end{table}

Four of the content words listed in \tabref{Table_4.1} involve \isi{affixation}: \textitbf{bua} ‘fruit’ and reduplicated \textitbf{bua{\Tilde}bua\-ang} (suffix \textitbf{-ang} ‘\textsc{pat}’), \textitbf{tumpuk} ‘pile’ and reduplicated \textitbf{ber\-tumpuk{\Tilde}tumpuk} (prefix \textscItal{ber-} ‘\textsc{vblz}’), \textitbf{tingkat\-ang} ‘level’ and reduplicated \textitbf{tingkat}\textitbf{{}-}\textitbf{ang{\Tilde}tingkat}\textitbf{{}-}\textitbf{ang} (suffix \textitbf{-ang} ‘\textsc{pat}’), and \textitbf{ta-lipat} ‘be folded’ and reduplicated \textitbf{ta-lipat{\Tilde}ta-lipat} (prefix \textscItal{ter-} ‘\textsc{acl}’). The four lexeme pairs illustrate that \isi{reduplication} may precede \isi{affixation} as with \textitbf{bua} ‘fruit’ or \textitbf{tumpuk} ‘pile’ or may follow \isi{affixation} as with \textitbf{tingkatang} ‘level’ or \textitbf{talipat} ‘be folded’. These examples also show that \isi{reduplication} only affects free morphemes while affixes are never reduplicated.



Reduplication of content words is demonstrated with the three examples in (\ref{Example_4.3}) to (\ref{Example_4.5}). Reduplication of a \isi{noun} is illustrated in (\ref{Example_4.3}); in this context reduplicated \textitbf{ade} ‘younger sibling’ conveys plurality. The utterance in (\ref{Example_4.4}) includes a reduplicated \isi{verb}; in this context, \textitbf{lari} ‘run’ expresses \isi{continuation}. And the elicited example in (\ref{Example_4.5}) illustrates \isi{reduplication} of an ad\isi{verb}; in this context \isi{prohibitive} \textitbf{jangang} ‘\textsc{neg.imp}, don’t’ denotes \isi{intensity}. The three examples illustrate only three of the different meanings expressed with \isi{reduplication}. Depending on the context, a reduplicated \isi{noun} can also signal \isi{repetition}, to name just one other meaning aspect. Along similar lines, a reduplicated \isi{verb} can also express \isi{aimlessness}, among other meanings. This variety of different meanings is discussed in detail in §\ref{Para_4.2}.


\ea
\label{Example_4.3}

\gll {jadi} {saya}, {saya} {deng} {sa} {pu} {\bluebold{ade{\Tilde}ade}} {tinggal} {di} {ruma}\\ %
 so  \textsc{1sg}  \textsc{1sg}  with  \textsc{1sg}  \textsc{poss}  \textsc{rdp}{\Tilde}ySb  stay  at  house\\
\glt 
Plurality: ‘so I, I and my \bluebold{younger siblings} stayed at the house’ \textstyleExampleSource{[081014-014-NP.0002]}
\z

\ea
\label{Example_4.4}
\gll {kitong} {dua} {\bluebold{lari{\Tilde}lari}} {sampe} {di} {Martewar}\\ %
 \textsc{1pl}  two  \textsc{rdp}{\Tilde}run  reach  at  Martewar\\
\glt 
Continuation: ‘the two of us \bluebold{kept running} all the way to Martewar’ \textstyleExampleSource{[080923-010-CvNP.0009]}
\z

\ea
\label{Example_4.5}
\gll {\ldots} {tapi} {\bluebold{jangang{\Tilde}jangang}} {hujang} {di} {tenga} {jalang}\\ %
 { }   but  \textsc{rdp}{\Tilde}\textsc{neg.imp}  {rain}  {at}  {middle}  {walk}\\

\glt
Intensity: ‘[I want to go to (my) gardens,] but \bluebold{let’s hope} it \bluebold{won’t} rain in the middle of the way’ \textstyleExampleSource{[Elicited BR120813.031]}
\z


\newpage 
\subsubsection[Reduplication of function words]{Reduplication of function words}\label{Para_4.1.1.2}
\label{bkm:Ref339284928}
Some Papuan Malay functions words can also be reduplicated. Attested are reduplicated personal pronouns, demonstratives, locatives,\footnote{While \isi{reduplication} of \textitbf{sana} ‘\textsc{l.dist}’ is unattested in the corpus, it does occur, following one consultant.} interrogatives, the \isi{causative} \isi{verb} \textitbf{kasi} ‘give’, and the \isi{reciprocity marker} \textitbf{baku} ‘\textsc{recp}’, as listed in \tabref{Table_4.2}.



\begin{table}
\caption{ Reduplication of function words}\label{Table_4.2}
\begin{tabular}{llll}
\lsptoprule
\multicolumn{1}{c}{Word class} & \multicolumn{1}{c}{Base} & \multicolumn{1}{c}{Gloss} &  \multicolumn{1}{c}{Reduplicated item}\\
\midrule

Personal pronouns & \textitbf{saya} & ‘\textsc{1sg}’ & \textitbf{saya{\Tilde}saya}\\
& \textitbf{kamu} & ‘\textsc{2pl}’ & \textitbf{kamu{\Tilde}kamu}\\
& \textitbf{dorang} & ‘\textsc{3pl}’ & \textitbf{dorang{\Tilde}dorang}\\
Demonstratives & \textitbf{ini} & ‘\textsc{d.prox}’ & \textitbf{ini{\Tilde}ini}\\
& \textitbf{itu} & ‘\textsc{d.dist}’ & \textitbf{itu{\Tilde}itu}\\
Locatives & \textitbf{sini} & ‘\textsc{l.prox}’ & \textitbf{sini{\Tilde}sini}\\
& \textitbf{situ} & ‘\textsc{l.med}’ & \textitbf{situ{\Tilde}situ}\\
& \textitbf{sana} & ‘\textsc{l.dist}’ & \textitbf{sana{\Tilde}sana}\\
Interrogatives & \textitbf{siapa} & ‘who’ & \textitbf{siapa{\Tilde}siapa}\\
& \textitbf{apa} & ‘what’ & \textitbf{apa{\Tilde}apa}\\
& \textitbf{kapang} & ‘when’ & \textitbf{kapang{\Tilde}kapang}\\
Causative \isi{verb} & \textitbf{kasi} & ‘give’ & \textitbf{kas{\Tilde}kas}\\
Reciprocity marker & \textitbf{baku} & ‘\textsc{recp}’ & \textitbf{baku{\Tilde}baku}\\
\lspbottomrule
\end{tabular}
\end{table}

Reduplication of three different types of functions words and the different meaning aspects conveyed is illustrated in (\ref{Example_4.6}) to (\ref{Example_4.8}): personal pronouns in (\ref{Example_4.6}), locatives in the elicited example in (\ref{Example_4.7}), and interrogatives in (\ref{Example_4.8}).

\ea
\label{Example_4.6}
\gll {\bluebold{kamu{\Tilde}kamu}} {ini} {bangung}, {bangung}\\ %
 \textsc{rdp}{\Tilde}\textsc{2pl}  \textsc{d.prox}  wake.up  wake.up\\
\glt 
Collectivity: ‘\bluebold{you all} here wake-up!, wake-up!’ \textstyleExampleSource{[081115-001a-Cv.0330]}
\z

\ea
\label{Example_4.7}
\gll {ko} {lari} {suda} {ke} {\bluebold{sana{\Tilde}sana}}\\ %
 \textsc{2sg}  run  already  to  \textsc{rdp}{\Tilde}\textsc{loc.dist}\\
\glt 
Diversity: ‘you run to \bluebold{somewhere} over there!’ \textstyleExampleSource{[Elicited BR120813.016]}
\z

\ea
\label{Example_4.8}
\gll {\ldots} {sa} {tra} {perna} {lari} {ke} {\bluebold{siapa{\Tilde}siapa}}\\ %
  { } \textsc{1sg}  \textsc{neg}  once  run  to  \textsc{rdp}{\Tilde}who\\
\glt
Intensity: ‘[even (when) my children were already sick,] I’ve never run to \bluebold{anyone} (for black-magic help)’ \textstyleExampleSource{[081006-034-CvEx.0028]}

\z

\subsubsection[Reduplication without corresponding single base]{Reduplication without corresponding single base}\label{Para_4.1.1.3}
Across word classes, some reduplicated forms do not have an unreduplicated single base. Attested are four nouns, three verbs, one \isi{quantifier}, and one \isi{conjunction}, as listed in \tabref{Table_4.3}.


\begin{table}
\caption{ Reduplication without corresponding single base}\label{Table_4.3}

\begin{tabular}{llll}
\lsptoprule

 Word class & Base & Reduplicated item &  Gloss\\
\midrule
Nouns & \textitbf{*alang} & \textitbf{alang-alang} & ‘cogongrass’\\
& \textitbf{*kura} & \textitbf{kura-kura} & ‘turtle’\\
& \textitbf{*pori} & \textitbf{pori-pori} & ‘pore’\\
& \textitbf{*soa} & \textitbf{soa-soa} & ‘monitor lizard’\\
Verbs & \textitbf{*belit} & \textitbf{belit-belit} & ‘curve’\\
& \textitbf{*gong} & \textitbf{gong-gong} & ‘bark (at)’\\
& \textitbf{*tele} & \textitbf{tele-tele} & ‘talk excessively’\\
Quantifier & \textitbf{*masing} & \textitbf{masing-masing} & ‘each’\\
Conjunction & \textitbf{*gara} & \textitbf{gara-gara} & ‘because’\\
\lspbottomrule
\end{tabular}
\end{table}
\subsection{Partial reduplication}\label{Para_4.1.2}
Partial \isi{reduplication} is rare in Papuan Malay. Generally speaking, this type of \isi{reduplication} “involves the reiteration of only part of the semantic-syntactic or phonetic-phonological constituent whose meaning is accordingly modified” \citep[304]{Moravcsik.1978}.



That is, the added material is not a morphemic skeleton as in the case of full \isi{reduplication} but the reduplicant is a C-V skeleton or a syllabic skeleton which gets prefixed to the base. If the reduplicant is a C-V skeleton, the “entire phonemic melody of the stem is copied over the affixed C-V skeleton and linked to C and V ‘slots’ in the skeleton” \citep[437]{Marantz.1982} (concerning the principles involved in this linking, see \citealt[446–447]{Marantz.1982}). A “syllabic skeleton, lacking a phonemic melody and a C-V skeleton, borrows both from the stem to which it attaches” (\citeyear*[437]{Marantz.1982}).



In Papuan Malay, the reduplicant is a closed heavy syllable which is prefixed to the stem from which it borrows the phonemic melody and C-V skeleton, as shown in (\ref{Example_4.9}). In (\ref{Example_4.9a}), for example, the initial closed syllable [\textstyleChCharisSIL{bɐp}] is copied over the reduplicating syllabic skeleton. With vowel-initial stems, the initial VC is copied over the reduplicating syllabic skeleton. This is shown in (\ref{Example_4.9c}) with the initial VC [\textstyleChCharisSIL{an}] which is copied over the prefixed CVC syllable. These examples also show that the prefixed syllable does not take into account the \isi{syllable structure} of the base.


\ea
\label{Example_4.9}

\ea
~~~
\label{Example_4.9a} 
\parbox[t]{10cm}{
\vspace{-.85\baselineskip}
\begin{forest} for tree={grow=north}
 [{}, 
  phantom   [ { = [bɐp.ˈba.pa]} ]
  [, phantom]  
  [σ [V [a]] [C [p]]] 
  [σ [V [a]] [C [b]]] 
  [, phantom [{+}]]  
  [σ [, phantom [a]] [C [p]]   [V [a]] [C [b]] [{}, phantom]]
  ]
\end{forest}
} 

\medskip

 \ex
~~~
 \label{Example_4.9b} 
\parbox[t]{10cm}{
\vspace{-.85\baselineskip}
\begin{forest} for tree={grow=north}
[, phantom    [ { = [bɐr.ˈba.ru]}]
[, phantom] [σ [V [u]] [C [r]]] [σ [V [a]] [C [b]]] [,phantom]  [, phantom [{+}]] [σ [,phantom  [u]] [C [r]] [V [a]] [C [b]] [{}, phantom]]]
\end{forest} 
}

\medskip

\ex
~~~
\label{Example_4.9c} 
\parbox[t]{10cm}{
\vspace{-.7\baselineskip}
\begin{forest} for tree={grow=north}
[, phantom   [{\hspace{5mm} = [a.ˈna.na]}] [σ [V [a]] [C [n]] ] [σ [V [a]] [C]]  [, phantom [{+}]] [σ [,phantom  [a]] [C [n]] [V [a]][C] [,phantom]]]
\end{forest} 
}

\medskip

\ex
~~~
\label{Example_4.9d} 
\parbox[t]{10cm}{
\vspace{-.7\baselineskip}
\begin{forest} for tree={grow=north}
[, phantom   [{\hspace{5mm} = [a.ˈpa.pa]}] [σ [V [a]] [C [p]] ] [σ [V [a]] [C]]  [, phantom [{+}]] [σ [,phantom  [a]] [C [p]] [V [a]][C] [,phantom]]]
\end{forest} 
}
\z
\z  

In Papuan Malay, partial \isi{reduplication} is only attested for disyllabic lexical roots with penultimate stress. It always involves the partial \isi{reduplication} of the stressed penultimate syllable of the base, as shown in \tabref{Table_4.4}. The results are trisyllabic words with penultimate stress. If the base has a CV.CV(C) \isi{syllable structure}, stress in the reduplicated word remains on the penultimate syllable of the base, as in \textitbf{bapa{\Tilde}bapa} [\textstyleChCharisSIL{bɐp.ˈba.pa}] ‘fathers’. With vowel-initial stems, Papuan Malay copies the initial VC sequence, as in \textitbf{ana{\Tilde}ana} [\textstyleChCharisSIL{a.ˈ}\textstyleChCharisSIL{na.na}] ‘children’. In this case, the reduplicant’s segments do not originate from one and the same syllable of the base. Across languages this phenomenon is rather common. That is, as \citet[562]{Wiltshire.1978} note, partial \isi{reduplication} can be “oblivious to the prosodic structure of the base from which it copies a melody”.



The partially reduplicated forms are alternants of fully reduplicated ones and have the same semantics; [\textstyleChCharisSIL{a.ˈna.na}] ‘children’, for instance, is an alternant of [\textstyleChCharisSIL{ˌa.na.ˈa.na}] ‘children’.


\begin{table}
\caption{ Partial reduplication}\label{Table_4.4}


\begin{tabular}{llll}
\lsptoprule
 Base & Gloss & \multicolumn{2}{c}{ Reduplicated item}\\
\midrule
\textitbf{ana} & ‘child’ & \textitbf{ana{\Tilde}ana} & [\textstyleChCharisSIL{a.ˈna.na}]\\
\textitbf{apa} & ‘what’ & \textitbf{apa{\Tilde}apa} & [\textstyleChCharisSIL{a.ˈpa.pa}]\\
\textitbf{bapa} & ‘father’ & \textitbf{bapa{\Tilde}bapa} & [\textstyleChCharisSIL{bɐp.ˈba.pa}]\\
\textitbf{baru} & ‘recently’ & \textitbf{baru{\Tilde}baru} & [\textstyleChCharisSIL{bɐr.ˈba.ru}]\\
\lspbottomrule
\end{tabular}
\end{table}

\newpage 
\subsection{Imitative reduplication}\label{Para_4.1.3}

The third attested type of \isi{reduplication} is imitative or rhyming \isi{reduplication}. Cross-linguistically, this \isi{reduplication} type is also being referred to as “echo construction”; it “involves \isi{reduplication} with some different phonological material, such as a vowel or consonant change or addition, or morpheme order reversal” {\citep[2]{Rubino.2013}}.



Imitative \isi{reduplication} in Papuan Malay is unproductive and rare; attested are only the three lexemes listed in \tabref{Table_4.5}. The reduplicated component resembles the base in part but also differs from it, in that imitative \isi{reduplication} involves a vowel change. For one of the attested lexemes, the bare base is also inexistent: \textitbf{*ngyaung}.


\begin{table}
\caption{ Imitative \isi{reduplication} with vowel change}\label{Table_4.5}
\resizebox{\textwidth}{!}{
\begin{tabular}{llll}
\lsptoprule
 Reduplicated item & \multicolumn{1}{c}{Gloss} & \multicolumn{1}{c}{Base} &  \multicolumn{1}{c}{Gloss}\\
\midrule

\textitbf{ngying{\Tilde}ngyaung} & ideophone: cockatoo call & \textitbf{*ngyaung} & -- \\
\textitbf{tuk{\Tilde}tak} & ideophone: bang! & \textitbf{tak} & ideophone: bang!\\
\textitbf{bola{\Tilde}balik} & ‘move back and forth’ & \textitbf{balik} & ‘return’\\
\lspbottomrule
\end{tabular}}
\end{table}
\section{Lexeme interpretation}\label{Para_4.2}
In Papuan Malay, as in other languages, \isi{reduplication} conveys a variety of different meanings, such as plurality and diversity, \isi{intensity}, or \isi{continuation} and \isi{repetition}. Some of these meaning aspects tend to be limited to certain word classes, while others are conveyed by a variety of different word classes.



The meaning aspects of reduplicated Papuan Malay content words are examined in §\ref{Para_4.2.1} to §\ref{Para_4.2.4}, those of reduplicated function words in §\ref{Para_4.2.5}. The underlying general meaning or \isi{gesamtbedeutung} of \isi{reduplication} is explored in §\ref{Para_4.2.6}.


\subsection{Reduplication of nouns}\label{Para_4.2.1}
Across languages, \isi{reduplication} of nouns has been found to express a variety of meanings such as “number [\ldots], case, distributivity, \isi{indefiniteness}, reciprocity, size (diminutive or augmentative), and \isi{associative} qualities” {\citep{Rubino.2013}}. In Papuan Malay, the following meaning aspects are attested: \isi{collectivity} and diversity (§\ref{Para_4.2.1.1}), \isi{repetition} (§\ref{Para_4.2.1.2}), and \isi{indefiniteness} (§\ref{Para_4.2.1.3}). Reduplicated nouns can also undergo an \isi{interpretational shift} and receive a verbal or adverbial reading (§\ref{Para_4.2.1.4}).


\subsubsection[Collectivity and diversity]{Collectivity and diversity}
\label{Para_4.2.1.1}
A major function of \isi{noun} \isi{reduplication} is to signal plurality, given that in Papuan Malay bare nouns are not marked for number. Instead, speakers express plurality as deemed necessary. Depending on the context, the lexical item \textitbf{ana} ‘child’, for instance, could also be read as ‘children’. One strategy to express plurality overtly is the \isi{reduplication} of nouns. Overall, however, speakers use \isi{reduplication} only when an unambiguous plural reading is important to them and when the context does not allow such an unambiguous interpretation. (Alternative strategies to indicate plurality are \isi{modification} with a \isi{numeral} or \isi{quantifier}, or with a plural personal \isi{pronoun}; see §\ref{Para_8.2.3} and §\ref{Para_6.2.2}, respectively.)



Cross-linguistically, three types of plurality have been identified which are encoded by \isi{noun} \isi{reduplication} \citep[561]{Wiltshire.1978}: \isi{collectivity}, diversity (or variety), and distributivity.\footnote{\citet[561]{Wiltshire.1978} refer to “\isi{collectivity}” as “simple plurality”.} Of these three types, Papuan Malay uses two, namely \isi{collectivity} as in (\ref{Example_4.10}) and (\ref{Example_4.11}), and diversity as in (\ref{Example_4.12}) and (\ref{Example_4.13}). Another type of plurality is \isi{indefiniteness} \citep{Rubino.2013}, which is also found in Papuan Malay, as demonstrated in (\ref{Example_4.16}) and (\ref{Example_4.17}) in §\ref{Para_4.2.1.3} (p. \pageref{Example_4.16}).


Reduplication of nouns most often indicates \isi{collectivity} in the sense of ``all \textsc{base}'', as shown with \textitbf{ana{\Tilde}ana} ‘children’ in (\ref{Example_4.10}) and \textitbf{orang{\Tilde}orang} ‘people’ in (\ref{Example_4.11}).


\begin{styleExampleTitle}
Reduplicated nouns: Collectivity
\end{styleExampleTitle}

\ea
\label{Example_4.10}
\gll {\bluebold{ana{\Tilde}ana}} {su} {pergi} {kerja}, {\bluebold{ana{\Tilde}ana}} {su} {kawing}\\ %
 \textsc{rdp}{\Tilde}child  already  go  work  \textsc{rdp}{\Tilde}child  already  marry.inofficially\\

\glt 
[Complaint of a lonely couple:] ‘\bluebold{all (our) children} already went to work (elsewhere), \bluebold{all (our) children} are already married’ \textstyleExampleSource{[080917-010-CvEx.0071]}
\z

\ea
\label{Example_4.11}
\gll {e}, {\bluebold{orang{\Tilde}orang}} {itu} {dong} {mara{\Tilde}mara}\\ %
 hey!  \textsc{rdp}{\Tilde}person  \textsc{d.dist}  \textsc{3pl}  \textsc{rdp}{\Tilde}feel.angry(.about)\\

\glt 
‘hey, \bluebold{all those people}, they’ll be really angry (with you)’ \textstyleExampleSource{[080917-008-NP.0053]}
\z


Less often, reduplicated nouns signal diversity such as \textitbf{bua{\Tilde}bua} ‘various fruit (trees)’ in (\ref{Example_4.12}), or \textitbf{pohong{\Tilde}pohong} ‘various trees’ in (\ref{Example_4.13}).


\newpage
\begin{styleExampleTitle}
Reduplicated nouns: Diversity
\end{styleExampleTitle}

\ea
\label{Example_4.12}
\gll {\bluebold{bua{\Tilde}bua}} {di} {sini} {banyak}\\ %
 \textsc{rdp}{\Tilde}fruit  at  \textsc{l.prox}  many\\

\glt 
‘there are a many \bluebold{different kinds of fruit (trees)} here’ (Lit. ‘\bluebold{the various fruit (trees)} here are many’) \textstyleExampleSource{[080922-001a-CvPh.0425]}
\z
\ea
\label{Example_4.13}
\gll {\ldots} {ini} {suda} {tida} {begini} {lagi}, {suda} {ada}\\ %
  { } \textsc{d.prox}  already  \textsc{neg}  like.this  again  already  exist\\
 \gll {\bluebold{pohong{\Tilde}pohong}}\\
 {\textsc{rdp}{\Tilde}tree}\\

\glt
‘[in five years, yes,] this (garden) won’t be same (as) there will be already \bluebold{various trees} (here)’ \textstyleExampleSource{[081029-001-Cv.0007]}

\z

\subsubsection[Repetition]{Repetition}
\label{Para_4.2.1.2}
Reduplication of nouns denoting periods of the day can indicate \isi{repetition}. This is illustrated with \textitbf{pagi{\Tilde}pagi} ‘every morning’ in (\ref{Example_4.14}), and \textitbf{malam{\Tilde}malam} ‘every evening’ in (\ref{Example_4.15}). (For alternative readings of reduplicated nouns expressing time divisions, see (\ref{Example_4.17}) in §\ref{Para_4.2.1.3}, p. \pageref{Example_4.17}, and (\ref{Example_4.23}) and (\ref{Example_4.24}) in §\ref{Para_4.2.1.4}, p. \pageref{Example_4.23}.)


\ea
\label{Example_4.14}
\gll {\bluebold{pagi{\Tilde}pagi}} {biking} {te}\\ %
 \textsc{rdp}{\Tilde}morning  make  tea\\
\glt 
‘\bluebold{every morning} (they) made tea’ \textstyleExampleSource{[081025-009a-Cv.0023]}
\z
\ea
\label{Example_4.15}
\gll {ko} {jangang} {ikut{\Tilde}ikut} {orang} {tua} {\bluebold{malam{\Tilde}malam}}\\ %
 \textsc{2sg}  \textsc{neg.imp}  \textsc{rdp}{\Tilde}follow  person  be.old  \textsc{rdp}{\Tilde}night\\
\glt
‘don’t keep hanging out with the grown-ups \bluebold{every evening}’ \textstyleExampleSource{[081013-002-Cv.0005]}

\z

\subsubsection[Indefiniteness]{Indefiniteness}
\label{Para_4.2.1.3}
Depending on the context, reduplicated nouns may signal \isi{indefiniteness} by referring to an unspecified group member, in the sense of ``any'' or "some''. This is illustrated with \textitbf{om{\Tilde}om} ‘any one of the uncles’ in (\ref{Example_4.16}), and \textitbf{malam{\Tilde}malam} ‘at some point in the evening’ in (\ref{Example_4.17}). (For alternative interpretations of reduplicated nouns signaling time divisions, see (\ref{Example_4.14}) and (\ref{Example_4.15}) in §\ref{Para_4.2.1.2}, p. \pageref{Example_4.14}, and (\ref{Example_4.23}) and (\ref{Example_4.24}) in §\ref{Para_4.2.1.4}, p. \pageref{Example_4.23}.)


\ea
\label{Example_4.16}
\gll {baru} {titip} {di}, {ini}, {\bluebold{om{\Tilde}om}} {dorang}\\ %
 and.then  deposit  at  \textsc{d.prox}  \textsc{rdp}{\Tilde}uncle  \textsc{3pl}\\
\glt 
‘leave (the letter) with, what’s-his-name, \bluebold{any one of the uncles} and his family’ \textstyleExampleSource{[080922-001a-CvPh.0602]}
\z
\ea
\label{Example_4.17}
\gll {dia} {lewat} {pante} {\bluebold{malam{\Tilde}malam}}\\ %
 \textsc{3sg}  pass.by  coast  \textsc{rdp}{\Tilde}night\\
\glt
‘he drove along the beach \bluebold{at some (point in) the evening}’ \textstyleExampleSource{[081006-020-Cv.0016]}
\z


\subsubsection[Interpretational shift]{Interpretational shift}
\label{Para_4.2.1.4}
Reduplicated nouns can also undergo, what generally-speaking \citet[212]{Booij.2007} calls, an “\isi{interpretational shift}” or “type coercion”. In Papuan Malay, such a shift can result in a stative verbal reading of reduplicated nouns as in (\ref{Example_4.18}) to (\ref{Example_4.20}), or in an adverbial reading as in (\ref{Example_4.21}) to (\ref{Example_4.24}), depending on the larger linguistic context.



Interpretational shift resulting in a stative verbal reading of reduplicated nouns usually applies to reduplicated kinship terms, taking the predicate slot in nonverbal clauses. This is illustrated with \textitbf{ana} ‘child’ in (\ref{Example_4.18}) and \textitbf{tete} ‘grandfather’ in the elicited example in (\ref{Example_4.19}). In this context, the reduplicated nouns receive a stative verbal rather than a nominal reading. That is, referring to specific age groups, they designate pertinent attributes of their base words, as in \textitbf{ana{\Tilde}ana} ‘be quite small’ (literally ‘\textsc{rdp}{\Tilde}child’) in (\ref{Example_4.18}), or \textitbf{tete{\Tilde}tete} ‘be quite old’ (literally ‘\textsc{rdp}{\Tilde}grandfather’) in (\ref{Example_4.19}). In addition, the corpus contains one example in which a non-\isi{kinship term}, namely the common \isi{noun} \textitbf{rawa} ‘swamp’, undergoes a similar \isi{interpretational shift}, receiving the stative verbal reading in \textitbf{rawa{\Tilde}rawa} ‘be swampy’, as shown in (\ref{Example_4.20}).


\begin{styleExampleTitle}
Reduplicated nouns: Stative verbal reading
\end{styleExampleTitle}

\ea
\label{Example_4.18}
\gll {waktu} {itu} {sa} {masi} {\bluebold{ana{\Tilde}ana}}\\ %
 time  \textsc{d.dist}  \textsc{1sg}  still  \textsc{rdp}{\Tilde}child\\
\glt 
‘at that time I was still \bluebold{quite small}’ \textstyleExampleSource{[080922-008-CvNP.0004]}
\z

\ea
\label{Example_4.19}
\gll {pace} {{ni}} {de} {{su}} {\bluebold{tete{\Tilde}tete}} {tapi} {masi}\\ %
 man  {\textsc{d.prox}}  \textsc{3sg}  {already}  \textsc{rdp}{\Tilde}grandfather  but  still\\
 \gll  {maing}  deng  {ana{\Tilde}ana}  {muda}\\
 {play}  with  {\textsc{rdp}{\Tilde}child}  {be.young}\\
\glt 
‘this guy, he’s already \bluebold{quite old} but he still hangs out with the young people’ \textstyleExampleSource{[Elicited BR120813.003]}
\z

\ea
\label{Example_4.20}
\gll {masi} {\bluebold{rawa{\Tilde}rawa}}\\ %
 still  \textsc{rdp}{\Tilde}swamp\\
\glt 
[About a road building project:] ‘(the area is) still \bluebold{swampy}’ \textstyleExampleSource{[081006-033-Cv.0027]}
\z


Interpretational shift can also affect reduplicated location nouns or nouns denoting periods of the day, with the reduplicated nouns receiving an intensified or emphatic adverbial reading. This is illustrated with the location nouns \textitbf{depang} ‘front’ in (\ref{Example_4.21}) and \textitbf{samping} ‘side’ in (\ref{Example_4.22}), and the temporal nouns \textitbf{pagi} ‘morning’ in (\ref{Example_4.23}) and \textitbf{malam} ‘night’ in (\ref{Example_4.24}). (For alternative readings of reduplicated nouns designating time divisions, see (\ref{Example_4.14}) and (\ref{Example_4.15}) in §\ref{Para_4.2.1.2}, p. \pageref{Example_4.14}, and (\ref{Example_4.17}) in §\ref{Para_4.2.1.3}, p. \pageref{Example_4.17}.)
 
\begin{styleExampleTitle}
Reduplicated nouns: Adverbial reading
\end{styleExampleTitle}

\ea
\label{Example_4.21}
\gll {sa} {tunjuk} {\bluebold{depang{\Tilde}depang}} {muka}, {blajar} {untuk} {mandiri}\\ %
 \textsc{1sg}  show  \textsc{rdp}{\Tilde}front  front  study  for  stand.alone\\
\glt 
‘I point \bluebold{right into} (their) faces (and tell them), ``study to become independent''' \textstyleExampleSource{[081115-001a-Cv.0054]}
\z

\ea
\label{Example_4.22}
\gll {jalang} {di} {\bluebold{samping{\Tilde}samping}} {itu} {pagar} {itu}\\ %
 walk  at  \textsc{rdp}{\Tilde}side  \textsc{d.dist}  fence  \textsc{d.dist}\\
\glt 
‘(he/she) walked \bluebold{right next to}, what’s-its-name, that fence’ \textstyleExampleSource{[081025-006-Cv.0094]}
\z

\ea
\label{Example_4.23}
\gll {\ldots} {\bluebold{pagi{\Tilde}pagi}} {jam} {lima} {sa} {su} {masuk} {di} {kamar}\\ %
  { }  \textsc{rdp}{\Tilde}morning  hour  five  \textsc{1sg}  already  enter  at  room\\
\glt 
[About disciplining ill-behaved teenagers:] ‘[tonight I’ll still sleep,] (but tomorrow) \bluebold{early in the morning} at five o’clock I will already have gone into (their) room’ \textstyleExampleSource{[081115-001a-Cv.0325]}
\z
\ea
\label{Example_4.24}
\gll {{\bluebold{malam{\Tilde}malam}}} {Ise} {bawa} {pulang} {dia} {pi} {tidor} {dengang}\\ %
 {\textsc{rdp}{\Tilde}night}  Ise  bring  go.home  \textsc{3sg}  go  sleep  with\\
 \gll  de  punya  {mama}\\
 \textsc{3sg}  \textsc{poss}  {mother}\\
\glt
[About a crying child:] ‘\bluebold{late at night} Ise brought (her) home so that she (would) go and sleep with her mother’ \textstyleExampleSource{[081006-025-CvEx.0007]}

\z

\subsection{Reduplication of verbs}
\label{Para_4.2.2}
Cross-linguistically, \isi{reduplication} of verbs tends to encode meaning aspects such as “distribution of an argument, tense, aspect (continued or repeated occurrence; completion; inchoativity), \isi{attenuation}, \isi{intensity}, transitivity (valence, object defocusing), or reciprocity” (\citealt{Rubino.2013}; see also \citealt[561]{Wiltshire.1978}). In Papuan Malay, the following meaning aspects are attested: \isi{continuation}, \isi{repetition}, and habit (§\ref{Para_4.2.2.1}), \isi{collectivity} and diversity (§\ref{Para_4.2.2.2}), \isi{intensity} (§\ref{Para_4.2.2.3}), \isi{immediacy} (§\ref{Para_4.2.2.4}), \isi{aimlessness} (§\ref{Para_4.2.2.5}), \isi{attenuation} (4.2.2.6), and \isi{imitation} (§\ref{Para_4.2.2.7}). Reduplicated verbs can also undergo \isi{interpretational shift}, in that they can receive a nominal or adverbial reading (§\ref{Para_4.2.2.8}).


\subsubsection[Continuation, {repetition}, and habit]{Continuation, {repetition}, and habit}
\label{Para_4.2.2.1}
A major function of \isi{verb} \isi{reduplication} is to indicate \isi{continuation}, \isi{repetition}, or habit. The function of signaling \isi{continuation} is demonstrated with a dynamic \isi{verb} in (\ref{Example_4.25}) and a stative \isi{verb} in the elicited example in (\ref{Example_4.26}). The function of signaling \isi{repetition} of an action is shown in (\ref{Example_4.27}).


\begin{styleExampleTitle}
Reduplicated verbs: Continuation and \isi{repetition}
\end{styleExampleTitle}
\ea
\label{Example_4.25}
\gll {\ldots} {ada} {setang} {datang} {\bluebold{ganggu{\Tilde}ganggu}} {kitorang}\\ %
 {}  exist  evil.spirit  come  \textsc{rdp}{\Tilde}disturb  \textsc{1pl}\\
\glt 
‘[when (we) sleep at night,] there is an evil spirit (who) comes and \bluebold{continuously bothers} (us)’ \textstyleExampleSource{[081006-022-CvEx.0168]}
\z

\ea
\label{Example_4.26}
\gll {sa} {pu} {temang} {de} {\bluebold{sakit{\Tilde}sakit}} {di} {Dok-Dua}\\ %
 \textsc{1sg}  \textsc{poss}  friend  \textsc{3sg}  \textsc{rdp}{\Tilde}be.sick  at  Dok-Dua\\
\glt 
‘my friend is being \bluebold{sick continuously} in the Dok-Dua (hospital)’ \textstyleExampleSource{[Elicited BR120813.036]}
\z

\ea
\label{Example_4.27}
\gll {baru} {de} {pi} {\bluebold{bicara{\Tilde}bicara}} {sa} {begini}\\ %
 and.then  \textsc{3sg}  go  \textsc{rdp}{\Tilde}speak  \textsc{1sg}  like.this\\
\glt 
‘but then he went to \bluebold{talk} about me like this \bluebold{again and again}’ \textstyleExampleSource{[081025-009b-Cv.0006]}
\z


As an extension of marking \isi{continuation} or \isi{repetition}, reduplicated verbs may also signal habit, as shown in (\ref{Example_4.28}).


\begin{styleExampleTitle}
Reduplicated verbs: Habit
\end{styleExampleTitle}

\ea
\label{Example_4.28}
\gll {begitu} {de} {besar} {baru} {de} {\bluebold{nakal{\Tilde}nakal}} {begini}\\ %
 like.that  \textsc{3sg}  be.big  and.then  \textsc{3sg}  \textsc{rdp}{\Tilde}be.mischievous  like.this\\

\glt
‘he grew up like that, and now he’s \bluebold{mischievous} like this \bluebold{all the time}’ \textstyleExampleSource{[080917-010-CvEx.0044]}
\z

\subsubsection[Collectivity and diversity]{Collectivity and diversity}
\label{Para_4.2.2.2}
Verb \isi{reduplication} may also indicate \isi{collectivity} or diversity of the clausal subject. The function of signaling \isi{collectivity} is illustrated with the examples in (\ref{Example_4.29}) and (\ref{Example_4.30}), while the diversity-marking function of reduplicated verbs is shown in the elicited examples in (\ref{Example_4.31}) and (\ref{Example_4.32}).


\ea
\label{Example_4.29}
\gll {dong} {taru} {piring{\Tilde}piring} {kaleng} {yang} {\bluebold{piring}} {yang} {\bluebold{bagus{\Tilde}bagus}}\\ %
 \textsc{3pl}  put  \textsc{rdp}{\Tilde}plate  tin.can  \textsc{rel}  plate  \textsc{rel}  \textsc{rdp}{\Tilde}be.good\\

\glt 
[About honoring guests:] ‘they place tin plates (in front of them) that are \bluebold{plates} that are \bluebold{all good}’ \textstyleExampleSource{[081014-010-CvEx.0015]}
\z
\ea
\label{Example_4.30}
\gll {\bluebold{pisang}} {\bluebold{Sorong}} {sana} {tu}, {iii}, {\bluebold{besar{\Tilde}besar}} {manis}\\ %
 banana  Sorong  \textsc{l.dist}  \textsc{d.dist}  oh!  \textsc{rdp}{\Tilde}be.big  be.sweet\\
\glt 
‘those \bluebold{bananas (from) Sorong} over there, oooh, (they) are \bluebold{all big} (and) sweet’ \textstyleExampleSource{[081011-003-Cv.0017]}
\z

\ea
\label{Example_4.31}
\gll {ko} {pu} {\bluebold{kwe}} {\bluebold{kras{\Tilde}kras}}\\ %
 \textsc{2sg}  \textsc{poss}  cake  \textsc{rdp}{\Tilde}be.harsh\\
\glt 
‘your \bluebold{various cakes} are \bluebold{hard}’ \textstyleExampleSource{[Elicited BR120813.034]}
\z

\ea
\label{Example_4.32}
\gll {\bluebold{mobil}} {di} {jalang} {\bluebold{rusak{\Tilde}rusak}} {karna} {banjir}\\ %
 car  at  walk  \textsc{rdp}{\Tilde}be.damaged  because  flooding\\
\glt
‘the \bluebold{various cars} in the street were \bluebold{broken} because of the flooding’ \textstyleExampleSource{[Elicited BR120813.035]}
\z


\subsubsection[Intensity]{Intensity}\label{Para_4.2.2.3}

Also, quite commonly reduplicated verbs signal \isi{intensity}. In such cases, reduplicated dynamic verbs receive the reading ``\textsc{base} intensely'', as in (\ref{Example_4.33}) and (\ref{Example_4.34}), while \isi{reduplication} of stative verbs translates with ``very \textsc{base}'', as in (\ref{Example_4.35}) and (\ref{Example_4.36}).


\begin{styleExampleTitle}
Reduplicated verbs: Intensity
\end{styleExampleTitle}

\ea
\label{Example_4.33}
\gll {baru} {dia} {tertawa}, {de} {\bluebold{tertawa{\Tilde}tertawa}}\\ %
 and.then  \textsc{3sg}  laugh  \textsc{3sg}  \textsc{rdp}{\Tilde}laugh\\
\glt 
‘but then he laughed, he \bluebold{laughed intensely}’ \textstyleExampleSource{[080916-001-CvNP.0004]}
\z

\ea
\label{Example_4.34}
\gll {orang} {\bluebold{bertriak{\Tilde}triak}} {tu}\\ %
 person  \textsc{rdp}{\Tilde}scream  \textsc{d.dist}\\

\glt 
‘the people were really \bluebold{screaming} \bluebold{intensely}’ \textstyleExampleSource{[081006-022-CvEx.0007]}
\z

\ea
\label{Example_4.35}
\gll {sa} {jalang} {sampe} {sa} {su} {\bluebold{swak{\Tilde}swak}}\\ %
 \textsc{1sg}  walk  until  \textsc{1sg}  already  \textsc{rdp}{\Tilde}be.exhausted\\
\glt 
‘I walked until I was already \bluebold{very exhausted}’ \textstyleExampleSource{[081025-008-Cv.0038]}
\z

\ea
\label{Example_4.36}
\gll {\ldots} {dong} {tu} {\bluebold{pintar{\Tilde}pintar}}\\ %
  { } \textsc{3pl}  \textsc{d.dist}  \textsc{rdp}{\Tilde}be.clever\\
\glt 
‘{\ldots} they (\textsc{emph}) are \bluebold{very clever}’ \textstyleExampleSource{[081109-001-Cv.0117]}
\z


When reduplicated verbs are negated with \textitbf{tra} ‘\textsc{neg}’ or \textitbf{jangang} ‘don’t’, they express an intensified negative in the sense of ``not \textsc{base} at all'', as shown in (\ref{Example_4.37}) and (\ref{Example_4.38}).


\begin{styleExampleTitle}
Negation of reduplicated verbs
\end{styleExampleTitle}
\ea
\label{Example_4.37}
\gll {sa} {\bluebold{tra}} {\bluebold{takut{\Tilde}takut}} {siapa} {pun}\\ %
 \textsc{1sg}  \textsc{neg}  \textsc{rdp}{\Tilde}feel.afraid(.of)  who  even\\

\glt 
‘I’m \bluebold{not afraid at all} of anybody’ \textstyleExampleSource{[081006-034-CvEx.0026]}
\z

\ea
\label{Example_4.38}
\gll {\bluebold{jangang}} {\bluebold{bli{\Tilde}bli}} {di} {sini}, {ini} {su} {malam}\\ %
 \textsc{neg.imp}  \textsc{rdp}{\Tilde}buy  at  \textsc{l.prox}  \textsc{d.prox}  already  night\\

\glt
‘(you) \bluebold{shouldn’t buy} (your sweets at the kiosk) here \bluebold{at all} (because) it is already night’ \textstyleExampleSource{[080917-008-NP.0061]}
\z


\subsubsection[Immediacy]{Immediacy}\label{Para_4.2.2.4}

Reduplicated verbs can indicate \isi{immediacy} in the sense of ``as soon as \textsc{base}''. This is illustrated with the reduplicated dynamic verbs in the elicited examples in (\ref{Example_4.39}) and (\ref{Example_4.40}).


\ea
\label{Example_4.39}
\gll {\bluebold{pulang{\Tilde}pulang}} {dari} {kantor} {pace} {de} {tidor}\\ %
 \textsc{rdp}{\Tilde}go.home  from  office  man  \textsc{3sg}  sleep\\
\glt 
‘\bluebold{as soon as} (he) \bluebold{came home} from the office, the man slept’ \textstyleExampleSource{[Elicited BR120813.007]}
\z

\ea
\label{Example_4.40}
\gll {mace} {ni} {\bluebold{datang{\Tilde}datang}} {trus} {de} {makang}\\ %
 woman  \textsc{d.prox}  \textsc{rdp}{\Tilde}come  next  \textsc{3sg}  eat\\
\glt
‘\bluebold{as soon as} this woman \bluebold{arrived}, she ate’ \textstyleExampleSource{[Elicited BR120813.008]}
\z


\subsubsection[Aimlessness]{Aimlessness}\label{Para_4.2.2.5}

Quite often, \isi{reduplication} adds the connotation of \isi{aimlessness} or casualness. That is, the reduplicated \isi{verb} may signal that an activity is done repeatedly without a specific goal, as in (\ref{Example_4.41}) and (\ref{Example_4.42}).


\ea
\label{Example_4.41}
\gll {sa} {itu} {sa} {pegang} {sagu} {sa} {makang} {\bluebold{jalang{\Tilde}jalang}}\\ %
 \textsc{1sg}  \textsc{d.dist}  \textsc{1sg}  hold  sago  \textsc{1sg}  eat  \textsc{rdp}{\Tilde}walk\\
\glt 
‘as for me, I was holding (some) sago, I ate (it) while \bluebold{strolling around}’ \textstyleExampleSource{[081025-009a-Cv.0073]}
\z
\ea
\label{Example_4.42}
\gll {malam} {kitong} {\bluebold{duduk{\Tilde}duduk}} {kitong} {\bluebold{menyanyi{\Tilde}menyanyi}}\\ %
 night  \textsc{1pl}  \textsc{rdp}{\Tilde}sit  \textsc{1pl}  \textsc{rdp}{\Tilde}sing\\
\glt
‘in the evening we were \bluebold{sitting around}, we were \bluebold{singing casually}’ \textstyleExampleSource{[081025-009a-Cv.0001]}
\z


\subsubsection[Attenuation]{Attenuation}\label{Para_4.2.2.6}

Depending on the context, reduplicated stative verbs may signal \isi{attenuation} in the sense of ``rather \textsc{base}'', as demonstrated in (\ref{Example_4.43}) and (\ref{Example_4.44}).



\ea
\label{Example_4.43}
\gll {\ldots} {{biking}} {{macang}} {kam} {{pu}} {Jayapura} {pu} {sayur}\\ %
  { }  {make}  {variety}  \textsc{2pl}  {\textsc{poss}}  Jayapura  \textsc{poss}  vegetable\\
\gll  {gnemo}  {yang}  {\bluebold{pahit{\Tilde}pahit}}  {itu}\\
 {melinjo}  {\textsc{rel}}  {\textsc{rdp}{\Tilde}be.bitter}  {\textsc{d.dist}}\\
\glt 
‘[then she asked, ‘you don’t fear the bitter (taste of melinjos)?, then mama Pawla said,] ``do you think this (melinjo) is like your Jayapura melinjo vegetable which is \bluebold{somewhat bitter}?''' \textstyleExampleSource{[080923-004-Cv.0016]}
\z
\ea
\label{Example_4.44}
\gll {badang} {{\bluebold{kurus{\Tilde}kurus}},} {rambut} {{ini}} {{tebal},} {de} {pu}\\ %
 body  {\textsc{rdp}{\Tilde}be.thin}  hair  {\textsc{d.prox}}  {be.thick}  \textsc{3sg}  \textsc{poss}\\
 \gll  {kuku}  ini  {\bluebold{panjang{\Tilde}panjang}},  {kaki}  {\bluebold{kurus{\Tilde}kurus}}\\
 {digit.nail}  \textsc{d.prox}  {\textsc{rdp}{\Tilde}be.long}  {foot}  {\textsc{rdp}{\Tilde}be.thin}\\
\glt
‘(his) body was \bluebold{somewhat thin}, (his) hair was thick, his fingernails were \bluebold{rather long}, (and his) legs were \bluebold{rather thin}’ \textstyleExampleSource{[081006-035-CvEx.0077]}
\z


\subsubsection[Imitation]{Imitation}
\label{Para_4.2.2.7}
Reduplicated verbs may also mark \isi{imitation} in the sense of ``something is an \isi{imitation} of X'' or ``something is similar to X''. This is illustrated with the dynamic verbs in (\ref{Example_4.45}) and (\ref{Example_4.46}), and the stative verbs in the elicited examples in (\ref{Example_4.47}) and (\ref{Example_4.48}).


\ea
\label{Example_4.45}
\gll {sa} {tendang} {dia} {di} {kaki} {sampe} {de} {\bluebold{lari{\Tilde}lari}} {\bluebold{babi}}\\ %
 \textsc{1sg}  kick  \textsc{3sg}  at  leg  until  \textsc{3sg}  \textsc{rdp}{\Tilde}run  pig\\
\glt 
‘I kicked him against (his) lower leg with the result that he \bluebold{staggered}’ (Lit. ‘he \bluebold{ran{\Tilde}ran} (like) \bluebold{a pig} (which has been shot)’) \textstyleExampleSource{[Elicited BR120813.004]}
\z

\ea
\label{Example_4.46}
\gll {dia} {\bluebold{mati{\Tilde}mati}} {\bluebold{ayam}}\\ %
 \textsc{3sg}  \textsc{rdp}{\Tilde}die  chicken\\
\glt 
‘he had an \bluebold{epileptic seizure}’ (Lit. ‘he \bluebold{died{\Tilde}died} (like) \bluebold{a chicken}’; that is, he was shaking like a chicken with its head cut off) \textstyleExampleSource{[Elicited BR120813.006]}
\z

\ea
\label{Example_4.47}
\gll {pace} {{ni}} {{de}} {su} {\bluebold{tua{\Tilde}tua}} {\bluebold{kladi}} {tapi} {suka}\\ %
 man  {\textsc{d.prox}}  {\textsc{3sg}}  already  \textsc{rdp}{\Tilde}be.old  taro.root  but  enjoy\\
\gll  {cari}  {prempuang}  {muda}\\
 {search}  {woman}  {be.young}\\
\glt 
‘this guy, he’s already \bluebold{very old} but (he) likes to have young women’ (Lit. ‘he’s \bluebold{old{\Tilde}old} (like) \bluebold{a taro root})’ \textstyleExampleSource{[Elicited BR120813.038]}
\z
\ea
\label{Example_4.48}
\gll {prempuang} {itu} {de} {pu} {kulit} {\bluebold{hitam{\Tilde}hitam}} {\bluebold{panta}} {\bluebold{blanga}}\\ %
 woman  \textsc{d.dist}  \textsc{3sg}  \textsc{poss}  skin  \textsc{rdp}{\Tilde}be.black  buttock  cooking.pot\\
\glt
‘that woman, her skin is \bluebold{black} (like) the bottom of a frying pan’ (Lit. ‘her skin is \bluebold{black{\Tilde}black} (like) \bluebold{the bottom} \ldots’) \textstyleExampleSource{[Elicited BR120813.046]}
\z


\subsubsection[Interpretational shift]{Interpretational shift}
\label{Para_4.2.2.8}
Reduplicated verbs can also undergo an \isi{interpretational shift}. Such a shift can result in a nominal reading of reduplicated verbs, as in the elicited examples in (\ref{Example_4.49}) and (\ref{Example_4.50}), or an adverbial reading, as in (\ref{Example_4.51}) to (\ref{Example_4.53}).



Reduplicated verbs with a nominal reading typically denote the instrument of the action specified by the \isi{verbal base}, such as \textitbf{garo{\Tilde}garo} ‘rake’ (literally ‘\textsc{rdp}{\Tilde}scratch’) in (\ref{Example_4.49}) or \textitbf{gait{\Tilde}gait} ‘pole’ (literally ‘\textsc{rdp}{\Tilde}hook’) in (\ref{Example_4.50}).


\begin{styleExampleTitle}
Reduplicated verbs: Nominal reading
\end{styleExampleTitle}
\ea
\label{Example_4.49}
\gll {tadi} {de} {pake} {\bluebold{garo{\Tilde}garo}} {buat} {garo} {rumput}\\ %
 earlier  \textsc{3sg}  use  \textsc{rdp}{\Tilde}scratch  for  scratch  grass\\
\glt 
‘earlier he took a \bluebold{rake} to rake the grass’ \textstyleExampleSource{[Elicited BR120813.010]}
\z
\ea
\label{Example_4.50}
\label{bkm:Ref359941330}
\gll {sa} {gait} {mangga} {deng} {\bluebold{gait{\Tilde}gait}}\\ %
 \textsc{1sg}  hook  mango  with  \textsc{rdp}{\Tilde}hook\\
\glt 
‘I plucked mangoes with a \bluebold{pole}’ \textstyleExampleSource{[Elicited BR120813.033]}
\z


Reduplicated verbs can also receive an adverbial reading, as in (\ref{Example_4.51}) to (\ref{Example_4.53}). Certain reduplicated dynamic verbs may take on the function as modal adverbs, such as \textitbf{taw{\Tilde}taw} ‘suddenly’ (literally ‘\textsc{rdp}{\Tilde}know’) in (\ref{Example_4.51}). Some reduplicated stative verbs are used as temporal adverbs such as \textitbf{lama{\Tilde}lama} ‘gradually’ (literally ‘\textsc{rdp}{\Tilde}be.long (of.duration)’) in (\ref{Example_4.52}), while others are used as manner adverbs, such as \textitbf{cepat{\Tilde}cepat} ‘quickly’ (literally ‘\textsc{rdp}{\Tilde}be.fast’) in (\ref{Example_4.53}).


\begin{styleExampleTitle}
Reduplicated verbs: Adverbial reading
\end{styleExampleTitle}

\ea
\label{Example_4.51}
\gll {\bluebold{taw{\Tilde}taw}} {orang} {itu} {tida} {keliatang}\\ %
 \textsc{rdp}{\Tilde}know  person  \textsc{d.dist}  \textsc{neg}  be.visible\\
\glt 
‘\bluebold{suddenly}, that person wasn’t visible (any longer)’ \textstyleExampleSource{[080922-002-Cv.0123]}
\z

\ea
\label{Example_4.52}
\gll {\bluebold{lama{\Tilde}lama}} {de} {padat} {itu} {macang} {aspal}\\ %
 \textsc{rdp}{\Tilde}be.long  \textsc{3sg}  be.solid  \textsc{d.dist}  variety  asphalt\\
\glt 
‘\bluebold{gradually}, it (the lime stone) becomes solid like asphalt’ \textstyleExampleSource{[081011-001-Cv.0304]}
\z

\ea
\label{Example_4.53}
\gll {yo}, {pak} {Hendrik} {ini} {de} {bilang}, {mandi} {\bluebold{cepat{\Tilde}cepat}}\\ %
 yes  father  Hendrik  \textsc{d.prox}  \textsc{3sg}  say  bathe  \textsc{rdp}{\Tilde}be.fast\\
\glt
‘yes, Mr. Hendrik here, he said, ``bathe \bluebold{quickly}''' \textstyleExampleSource{[080917-008-NP.0133]}
\z


\subsection{Reduplication of adverbs}\label{Para_4.2.3}

Reduplication of adverbs typically signals \isi{intensity}, similar to the \isi{reduplication} of verbs, discussed in §\ref{Para_4.2.2} (concerning the similarities between adverbs and verbs, see also §\ref{Para_5.4}). This is illustrated with the \isi{grading} ad\isi{verb} \textitbf{paling} ‘most’, the temporal ad\isi{verb} \textitbf{skarang} ‘now’, and the frequency ad\isi{verb} \textitbf{sring} ‘often’ in the three elicited examples in (\ref{Example_4.54}) to (\ref{Example_4.56}).



\ea
\label{Example_4.54}
\gll {de} {bilang} {de} {mo} {kerja} {tapi} {\bluebold{paling{\Tilde}paling}} {de} {tidor}\\ %
 \textsc{3sg}  say  \textsc{3sg}  want  work  but  \textsc{rdp}{\Tilde}most  \textsc{3sg}  sleep\\
\glt 
‘he says, he wants to work but \bluebold{most likely} he’ll sleep’ \textstyleExampleSource{[Elicited BR120813.015]}
\z
\ea
\label{Example_4.55}
  \bluebold{skarang{\Tilde}skarang}  de  ada  di  polisi\\
 \textsc{rdp}{\Tilde}now  \textsc{3sg}  exist  at  police\\
\glt 
‘\bluebold{right now} he/she is at the police (station)’ \textstyleExampleSource{[Elicited BR131231.002]}
\z

\ea
\label{Example_4.56}
\gll {sa} {pu} {kaka} {\bluebold{sring{\Tilde}sring}} {ke} {Jayapura}\\ %
 \textsc{1sg}  \textsc{poss}  oSb  \textsc{rdp}{\Tilde}often  to  Jayapura\\
\glt
‘my older sibling (travels) to Jayapura \bluebold{very often}’ \textstyleExampleSource{[Elicited BR131231.001]}
\z


\subsection{Reduplication of numerals and quantifiers}
\label{Para_4.2.4}
Across languages, \isi{reduplication} of numerals “has been found to express various categories including collectives, distributives, multiplicatives, and limitatives” {\citep{Rubino.2013}. In Papuan Malay, reduplicated numerals typically express \isi{collectivity} or distributiveness, while quantifiers signal distributiveness.

Reduplicated numerals have two meaning aspects. They may express \isi{collectivity} in the sense of ``all \textsc{base}' as in (\ref{Example_4.57}) and the elicited example in (\ref{Example_4.58}), or signal distributiveness in the sense of ``\textsc{base} by \textsc{base}'' as in (\ref{Example_4.59}) and (\ref{Example_4.60}).


\begin{styleExampleTitle}
Reduplication of numerals: Collectivity and distributiveness
\end{styleExampleTitle}

\ea
\label{Example_4.57}
\gll {yo}, {kas} {tinggal} {\bluebold{dua{\Tilde}dua}}\\ %
 yes  give  stay  \textsc{rdp}{\Tilde}two\\
\glt 
‘yes, let \bluebold{both of them} stay’ \textstyleExampleSource{[080919-006-CvNP.0018]}\\
\z

\ea
\label{Example_4.58}
\gll {\ldots} {karna} {pesta} {tu} {de} {su} {kasi} {mati} {\bluebold{tiga{\Tilde}tiga}}\\ %
 { }  because  party  \textsc{d.dist}  \textsc{3sg}  already  give  die  \textsc{rdp}{\Tilde}three\\
\glt 
‘[he/she has three pigs (but)] because of that festivity he/she already killed \bluebold{all three of them}’ \textstyleExampleSource{[Elicited BR120813.043]}\\
\z

\ea
\label{Example_4.59}
\gll {\ldots} {jadi} {lega} {ada} {lepas{\Tilde}lepas} {\bluebold{satu{\Tilde}satu}}\\ %
 { }  so  be.relieved  exist  \textsc{rdp}{\Tilde}free  \textsc{rdp}{\Tilde}one\\
\glt 
‘[fortunately, (the people) over there have already received Jesus,] so (you can feel) relieved, they were freed \bluebold{one-by-one}’ \textstyleExampleSource{[081025-007-Cv.0017]}\\
\z

\ea
\label{Example_4.60}
\gll  {sa} {minum} {\bluebold{lima{\Tilde}lima}} {mangkok} \\
\textsc{1sg} drink \textsc{rdp}{\Tilde}five cup\\
\\
\glt [About the lack of water during a retreat:] ‘I drank \bluebold{five} cups (\bluebold{every morning})’ (Lit. ‘\bluebold{five-by-five} cups’) \textstyleExampleSource{[081025-009a-Cv.0070]} \\
\z


Reduplicated quantifiers signal distributiveness, as in (\ref{Example_4.61}) and (\ref{Example_4.62}).


\begin{styleExampleTitle}
Reduplication of quantifiers: Distributiveness
\end{styleExampleTitle}

\ea
\label{Example_4.61}
\gll {\ldots} {kariawang} {dong} {\bluebold{banyak{\Tilde}banyak}} {dong} {baru} {turung} {ini}\\ %
 { }  employee  \textsc{3pl}  \textsc{rdp}{\Tilde}many  \textsc{3pl}  recently  descend  \textsc{d.prox}\\
\glt 
[Waiting for other boat passengers:] ‘the employees came recently down(stream) \bluebold{in groups of numerous people}’ (Lit. ‘\bluebold{many-by-many}’) \textstyleExampleSource{[080922-001a-CvPh.0812]}
\z
\ea
\label{Example_4.62}
\gll {dong} {blum} {isi}, {selaing} {dong} {isi} {\bluebold{sedikit{\Tilde}sedikit}} {to?}\\ %
 \textsc{3pl}  not.yet  fill  besides  \textsc{3pl}  fill  \textsc{rdp}{\Tilde}few  right?\\
\glt
[About how best to distribute food during a retreat:] ‘they haven’t yet filled (their plates), moreover \bluebold{each one of them} (should) fill (their plates with) \bluebold{little} (food), right?’ (Lit. ‘\bluebold{little by little}’) \textstyleExampleSource{[081025-009a-Cv.0081]}
\z


\subsection{Reduplication of function words}
\label{Para_4.2.5}
Reduplication of function words occurs considerably less often than that of content words. This section describes \isi{reduplication} of the following function words: personal pronouns (§\ref{Para_4.2.5.1}), demonstratives and locatives (§\ref{Para_4.2.5.2}), interrogatives (§\ref{Para_4.2.5.3}), and the \isi{causative} \isi{verb} \textitbf{kasi} ‘give’ and the \isi{reciprocity marker} \textitbf{baku} ‘\textsc{recp}’ (§\ref{Para_4.2.5.4}).


\subsubsection[Personal pronouns]{Personal pronouns}
\label{Para_4.2.5.1}
Personal pronouns can be reduplicated when used pronominally (for details on their different uses, see §\ref{Para_5.5} and §\ref{Para_6.1}). Reduplicated personal pronouns have three meaning aspects. Depending on the context, they signal \isi{collectivity}, disparagement, or \isi{imitation},\footnote{As mentioned in §\ref{Para_4.2.2.7}, the term ``\isi{imitation}'' includes meanings such as ``something is an \isi{imitation} of X'' or ``something is similar to X''.} as in (\ref{Example_4.63}) and in the elicited examples in (\ref{Example_4.64}) and (\ref{Example_4.65}), respectively.


\ea
\label{Example_4.63}
\gll {\bluebold{kamu{\Tilde}kamu}} {ini} {bangung} {bangung} {bangung} {bangung}\\ %
 \textsc{rdp}{\Tilde}\textsc{2pl}  \textsc{d.prox}  wake.up  wake.up  wake.up  wake.up\\
\glt 
‘\bluebold{all of you} here wake-up, wake-up, wake-up, wake-up’ \textstyleExampleSource{[081115-001a-Cv.0329]}
\z
\ea
\label{Example_4.64}
\gll {knapa} {\bluebold{saya{\Tilde}saya}} {saja} {yang} {bapa} {kasi} {tugas}\\ %
 why  \textsc{rdp}{\Tilde}\textsc{1sg}  just  \textsc{rel}  father  give  duty\\
\glt 
‘why is it (always) \bluebold{poor me} whom father gives chores’ \textstyleExampleSource{[Elicited BR120813.025]}
\z
\ea
\label{Example_4.65}
\gll {\bluebold{dorang{\Tilde}dorang}} {tra} {perna} {kasi} {bersi} {halamang}\\ %
 \textsc{rdp}{\Tilde}\textsc{3pl}  \textsc{neg}  ever  give  be.clean  yard\\
\glt
‘\bluebold{people like them} never clean (their) yard’ \textstyleExampleSource{[Elicited BR120813.024]}
\z


\subsubsection[Demonstratives and locatives]{Demonstratives and locatives}
\label{Para_4.2.5.2}
Demonstratives and locatives can also be reduplicated when used pronominally (for details on their different uses, see §\ref{Para_5.6} and §\ref{Para_5.7}, respectively). Reduplicated demonstratives express diversity as in (\ref{Example_4.66}) and (\ref{Example_4.67}).\footnote{Demonstrative sequences such as \textitbf{itu tu} ‘\textsc{d.dist} \textsc{d.dist}’ also convey \isi{intensity} or emphasis, as discussed in detail in §\ref{Para_7.1.2.3}. Given its phonological properties, however, juxtaposed \textitbf{itu tu} ‘\textsc{d.dist} \textsc{d.dist}’ is not taken as an instance of partial \isi{reduplication}. As discussed in §\ref{Para_4.1.2}, partial \isi{reduplication} of the stem \textitbf{itu} ‘\textsc{d.dist}’ should result in the reduplicated form \textitbf{it{\Tilde}itu} ‘\textsc{d.dist}{\Tilde}\textsc{d.dist}’. Therefore, \textitbf{itu tu} ‘\textsc{d.dist} \textsc{d.dist}’ is taken as an instance of \isi{demonstrative} stacking (see §\ref{Para_5.6.4}).} Depending on the context, reduplicated locatives may signal diversity as in (\ref{Example_4.68}), or emphasize the core meaning of the respective \isi{locative}, as in (\ref{Example_4.69}).


\ea
\label{Example_4.66}

\gll {setela} {itu} {nanti} {buat} {\bluebold{ini{\Tilde}ini}}\\ %
 after  \textsc{d.dist}  very.soon  make  \textsc{rdp}{\Tilde}\textsc{d.prox}\\
\glt 
‘soon after that (they) did \bluebold{these various} (things)’ \textstyleExampleSource{[080923-013-CvEx.0030]}
\z

\ea
\label{Example_4.67}
\gll {\ldots} {yang} {laing} {\bluebold{itu{\Tilde}itu}} {honorer} {smua} {itu}\\ %
 { }  \textsc{rel}  be.different  \textsc{rdp}{\Tilde}\textsc{d.dist}  be.honorary  all  \textsc{d.dist}\\
\glt 
‘[there are no school teachers, only him and Markus,] (as for) the others, \bluebold{those various} (teachers) are all honorary (teachers)’ \textstyleExampleSource{[081011-024-Cv.0054]}
\z
\ea
\label{Example_4.68}
\gll {jadi} {de} {bapa} {ke} {Jayapura} {tinggal} {\bluebold{situ{\Tilde}situ}}\\ %
 so  \textsc{3sg}  father  to  Jayapura  stay  \textsc{rdp}{\Tilde}\textsc{l.med}\\
\glt 
‘so her father (went) to Jayapura and lived \bluebold{there in a number of different places}’ \textstyleExampleSource{[081011-023-Cv.0163]}
\z
\ea
\label{Example_4.69}
\gll {\ldots} {di} {sini} {ada} {air}, {mari} {\bluebold{sini{\Tilde}sini}}\\ %
{ }   at  \textsc{l.prox}  exist  water  hither  \textsc{rdp}{\Tilde}\textsc{l.prox}\\
\glt
‘[(you) may fish from up here,] here is water, (come) here, \bluebold{right here}’ \textstyleExampleSource{[081025-003-Cv.0093]}
\z


\subsubsection[Interrogatives]{Interrogatives}
\label{Para_4.2.5.3}
Likewise, interrogatives can be reduplicated when used pronominally (for details on their different uses, see §\ref{Para_5.8}). Reduplicated interrogatives signal \isi{indefiniteness} by referring to an unspecified group member, in the sense of ``any'' or ``Wh-ever''. This is illustrated with the examples in (\ref{Example_4.70}) to (\ref{Example_4.73}); the example in (\ref{Example_4.73}) is elicited.


\ea
\label{Example_4.70}
\gll {yo}, {tida} {bole} {kas} {taw} {\bluebold{siapa{\Tilde}siapa}}\\ %
 yes  \textsc{neg}  may  give  know  \textsc{rdp}{\Tilde}who\\
\glt 
‘yes, (you) must not tell \bluebold{anybody}’ \textstyleExampleSource{[080922-001a-CvPh.0288]}
\z

\ea
\label{Example_4.71}
 \gll saya  tida  biking  \bluebold{apa{\Tilde}apa}  karna  babi  suda  mati\\
 \textsc{1sg}  \textsc{neg}  make  \textsc{rdp}{\Tilde}what  because  pig  already  die\\
\glt 
[About hunting a wild pig:] ‘I didn’t do \bluebold{anything} because the pig was already dead’ \textstyleExampleSource{[080919-004-NP.0023]}
\z

\ea
\label{Example_4.72}
\gll {nanti} {\bluebold{kapang{\Tilde}kapang}} {ka} {ko} {jalang{\Tilde}jalang} {ke} {mari}\\ %
 very.soon  \textsc{rdp}{\Tilde}when  or  \textsc{2sg}  \textsc{rdp}{\Tilde}walk  to  hither\\
\glt 
‘later \bluebold{whenever} (you have time) you come here’ \textstyleExampleSource{[Elicited BR120813.029]}
\z

\ea
\label{Example_4.73}
\gll {di} {\bluebold{mana{\Tilde}mana}} {smua} {pake} {ini}, {tajam} {besi} {ini}\\ %
 at  \textsc{rdp}{\Tilde}where  all  use  \textsc{d.prox}  be.sharp  metal  \textsc{d.prox}\\
\glt
[About sagu production:] ‘\bluebold{wherever} everybody uses it, this sharp metal’ \textstyleExampleSource{[081014-006-CvPr.0059]}
\z


Alternatively, speakers may use the bare \isi{interrogative} followed by the focus ad\isi{verb} \textitbf{saja} ‘just’ to encode an \isi{indefinite} referent, as discussed in §\ref{Para_5.8.8}.


\subsubsection[Causative \isi{verb} kasi ‘give’ and {reciprocity marker} baku ‘recp’]{Causative {verb} \textitbf{kasi} ‘give’ and {reciprocity marker} \textitbf{baku} ‘\textsc{recp}’}
\label{Para_4.2.5.4}
Reduplication of the \isi{causative} \isi{verb} \textitbf{kasi} ‘give’ and \isi{reciprocity marker} \textitbf{baku} ‘\textsc{recp}’, as in (\ref{Example_4.74}) and (\ref{Example_4.75}) respectively, signals \isi{repetition} or \isi{continuation}. (For more details on \isi{causative} and reciprocal constructions, see §\ref{Para_11.2} and §\ref{Para_11.3}, respectively.)


\ea
\label{Example_4.74}
\gll {knapa} {kam} {\bluebold{kas{\Tilde}kas}} {bangung} {dia}, {de} {masi} {mo} {tidor}\\ %
 why  \textsc{2pl}  \textsc{rdp}{\Tilde}give  wake.up  \textsc{3sg}  \textsc{3sg}  still  want  sleep\\
\glt 
‘why do you \bluebold{keep }waking him up?, he still wants to sleep’ (Lit. ‘\bluebold{give{\Tilde}give} to wake up’) \textstyleExampleSource{[080918-001-CvNP.0039]}
\z
\ea
\label{Example_4.75}
\gll {itu} {sampe} {tong} {\bluebold{baku{\Tilde}baku}} {tawar} {ini} {deng} {doseng}\\ %
 \textsc{d.dist}  until  \textsc{1pl}  \textsc{rdp}{\Tilde}\textsc{recp}  bargain  \textsc{d.prox}  with  lecturer\\
\glt
‘it got to the point that we and the lecturer were arguing \bluebold{constantly with each other}’ \textstyleExampleSource{[080917-010-CvEx.0177]}
\z


\subsection{Gesamtbedeutung of reduplication}
\label{Para_4.2.6}
Reduplication in Papuan Malay conveys a number of different meaning aspects ranging from \isi{continuation} and diversity to disparagement and \isi{imitation}. This variety in meaning raises two questions: first, does \isi{reduplication} have a general meaning or \isi{gesamtbedeutung}, and second, is there a specific relation between the meaning and the syntactic class of the base word.



\tabref{Table_4.6} lists the Papuan Malay word classes which attract \isi{reduplication} and the meaning aspects they convey.


\begin{table}
\caption{ Word classes and meaning aspects in reduplication}\label{Table_4.6}
\begin{tabular}{llll}
\lsptoprule

\multicolumn{2}{c}{Dimension} & \multicolumn{1}{c}{Meaning aspects} &  Word class of base\\
\midrule
\textsc{aug} & \textsc{quant} & Continuation/\isi{repetition}/habit & \textsc{n}, \textsc{v}, \textsc{caus}, \textsc{recp}\\
& \textsc{quant} & Collectivity & \textsc{n}, \textsc{v}, \textsc{num}, \textsc{pro}\\
& \textsc{quant} & Diversity & \textsc{n}, \textsc{v}, \textsc{dem}, \textsc{loc}\\
& \textsc{quant} & Distributiveness & \textsc{num}, \textsc{qt}\\
& \textsc{intens} & Intensity & \textsc{v}, \textsc{adv}, \textsc{loc}\\
& \textsc{intens} & Immediacy & \textsc{v}\\
\textsc{dim} &  & Disparagement & \textsc{pro}\\
&  & Indefiniteness & \textsc{n}, \textsc{int}\\
&  & Aimlessness & \textsc{v}\\
&  & Attenuation & \textsc{v}\\
&  & Imitation & \textsc{v}, \textsc{pro}\\
\lspbottomrule
\end{tabular}
\end{table}

Some of the meaning aspects which \isi{reduplication} in Papuan Malay conveys include, what cross-linguistically \citet[130]{Moravcsik.2013} refers to as “contradictory senses”. The aspect ``\isi{immediacy}'', for instance, represents an increase in \isi{intensity}, while the aspect ``\isi{aimlessness}'' refers to a decrease in \isi{intensity}. This phenomenon that \isi{reduplication} brings together a variety of meanings, some of which are opposite, is quite common cross-linguistically (see, for instance, \citealt{Regier.1994}; \citealt[124–125]{Mattes.2007}; \citealt[1151]{Kiyomi.2009}; \citealt[129–133]{Moravcsik.2013}).



Examining the “crosslinguistically recurrent semantic properties of \isi{reduplication}”, {\citet[131]{Moravcsik.2013}} comes to the conclusion that


\begin{quote}
\isi{reduplication} may be viewed as a marking device to indicate that the word is to be understood in an out-of-the-ordinary sense: the meaning deviates from the normal sense of the base either by being “more” or by being “less”.
\end{quote}


These contradictory meaning aspects of augmentation and diminution have also been noted for \ili{Malayo-Polynesian} languages. In her study on \isi{reduplication} in 30 of these languages, {\citet[1151]{Kiyomi.2009}} considers these two meanings of \isi{reduplication} to be


\begin{quote}
two manifestations of the same semantic principle of ‘a {\ldots}er degree of \ldots’, which is projected in the opposite directions. Then one can postulate that the prototypical meanings of \isi{reduplication} represent the semantic principle ‘\textsc{a} \textsc{higher}/\textsc{lower} \textsc{degree} \textsc{of} \ldots’
\end{quote}


The overview presented in \tabref{Table_4.6} indicates that this semantic principle of “a {\ldots}er degree of \ldots” in terms of augmentation or diminution also accounts for the different meaning aspects of \isi{reduplication} in Papuan Malay.



In Papuan Malay, the notion of ``higher degree of \ldots'' involves augmentation in terms of quantity or \isi{intensity}. Cross-linguistically, \citet[317, 321]{Moravcsik.2013} specifies that in the context of \isi{reduplication} quantity can refer to the “participants of [an] event or the events themselves”, while \isi{intensity} refers to the   amount of “energy investment or size of effect”. In Papuan Malay, the augmentation of quantity includes meaning aspects such as \isi{collectivity} or \isi{repetition}, while the increase in \isi{intensity} includes the meaning aspects of \isi{intensity} and \isi{immediacy}, as listed in \tabref{Table_4.6}.



The notion of ``lesser degree of \ldots'' involves, generally speaking, diminution which typically “adds the meaning of smallness to the stem meaning” {\citep[1153]{Kiyomi.2009}}. As {\citet[424]{Jurafsky.1993}} points out, however, the diminutive exhibits a variety of “metaphorical extensions” which involve “meaning shifts from the physical world to the social domain, and from the physical world to the conceptual or category domain”. Such semantic extensions of the diminutive are also found in Papuan Malay, in that the semantic effect of diminution brings together the meaning aspects of disparagement, \isi{indefiniteness}, \isi{aimlessness}, \isi{attenuation}, and \isi{imitation}.



The ``disparagement'' sense is linked to the notion of diminution metaphorically in that it has to do with social importance or power. The ``\isi{indefiniteness}'' sense is also a metaphorical extension in that it conveys toned-down reference. Likewise, the ``\isi{aimlessness}'' sense is linked to the notion of diminution in that it denotes actions which are done with less \isi{intensity}. The ``\isi{attenuation}'' sense is a metaphorical extension of the core sense ``size'' in that it denotes properties which are weaker, or activities which are carried out less intensely. The ``\isi{imitation}'' sense refers to objects or activities which copy or imitate what the base denotes. This sense is linked to the core sense of diminution in that the objects and activities are not identical with their models but merely resemble them a little bit. (See \citealt[426, 430]{Jurafsky.1993}; \citealt[125]{Mattes.2007}; V. Mattes p.c. 2013; \citealt[129--130]{Moravcsik.2013}.)

In summarizing the above and in applying Kiyomi’s \citeyearpar[1151]{Kiyomi.2009} terminology, it is concluded that in Papuan Malay, the \isi{gesamtbedeutung} of \isi{reduplication} is ``a \textsc{higher}/\textsc{lower} \textsc{degree} \textsc{of} \ldots''. \tabref{Table_4.7} gives examples for the two dimensions of augmentation and diminution conveyed by \isi{reduplication}.


\begin{table}
\caption{ Gesamtbedeutung of reduplication}\label{Table_4.7}
\resizebox{\textwidth}{!}{
\begin{tabular}{lllll}
\lsptoprule\multicolumn{2}{c}{ Dimensions} & \multicolumn{1}{c}{Item} & \multicolumn{2}{c}{ Gloss}\\
\midrule

Augmentation & Quantity & \textitbf{ana{\Tilde}ana} & \textsc{rdp}{\Tilde}child & ‘children’\\
& Intensity & \textitbf{pintar{\Tilde}pintar} & \textsc{rdp}{\Tilde}be.clever & ‘be very clever’\\
Diminution & Attenuation & \textitbf{kurus{\Tilde}kurus} & \textsc{rdp}{\Tilde}be.thin & ‘be rather thin’\\
& Imitation & \textitbf{mati{\Tilde}mati} & \textsc{rdp}{\Tilde}die & ‘die like \ldots’\\
\lspbottomrule
\end{tabular}}
\end{table}

With respect to the relation between the meaning and the syntactic class of the base word, two major observations are made. First, across word classes, reduplicated lexemes differ in terms of the meaning aspects which they convey. Second, meaning aspects differ as regards the range of word classes they attract for \isi{reduplication}. (See \tabref{Table_4.6}.)



First, concerning the reduplicated lexemes and the meaning aspects they convey, the gathered data indicates that within certain word classes \isi{reduplication} tends to convey more than one specific meaning. Reduplication in certain verbs, for example, can express \isi{immediacy} while in other verbs it signals \isi{continuation} or \isi{repetition}. It is notably content words which carry this variety of different meanings, that is, nouns, verbs, and numerals. In addition, \isi{reduplication} within two classes of function words also conveys more than one meaning aspect, namely in the classes of personal pronouns and locatives. Reduplication within the other three classes of function words, by contrast, tends to carry specific meanings: reduplicated demonstratives express diversity, interrogatives indicate \isi{indefiniteness}, and the \isi{causative} and reciprocity markers signal \isi{continuation} or \isi{repetition}. In relating the word classes which attract \isi{reduplication} to certain meaning aspects it is noted, however, that the meaning of a given reduplicated form is more than the meaning of its constituents. The fact that the entire reduplicated form and not its individual constituents carry this meaning, indicates, what \citet[260–261]{Booij.2013} calls, a “holistic” or constructional meaning of the reduplicated forms.



Second, regarding the meaning aspects and the range of word classes they attract for \isi{reduplication}, three meaning aspects bring together the largest number of different word classes, namely four each. The \isi{continuation}/\isi{repetition}/habit meaning aspect brings together nouns, verbs, and the \isi{causative} and reciprocity markers. The \isi{collectivity} meaning aspect brings together nouns, verbs, numerals, and personal pronouns. And the diversity meaning aspect brings together nouns, verbs, demonstratives, and locatives. Another pertinent meaning aspect is \isi{intensity}, which attracts three different word classes for \isi{reduplication}, namely verbs, adverbs, and locatives. Three more meaning aspects, which attract two word classes each for \isi{reduplication}, are distributiveness, \isi{indefiniteness}, and \isi{imitation}. The remaining meaning aspects attract only one word class each for \isi{reduplication}, that is, verbs for \isi{immediacy}, \isi{aimlessness}, and \isi{attenuation}, and personal pronouns for disparagement. These observations suggest that there is not a specific, one-to-one relation between the meaning and the syntactic class of the base word.


\section{Reduplication across eastern Malay varieties}
\label{Para_4.3}
Reduplication is also very common in other \ili{eastern Malay varieties}, such as \ili{Ambon Malay} (AM) \citep[112–140]{vanMinde.1997}, \ili{Banda Malay} (BM) \citep[160, 206]{Paauw.2009}, \ili{Kupang Malay} (KM) \citep[160, 171–173, 206, 252–253]{Paauw.2009}, \ili{Larantuka Malay} (LM) \citep[161, 171–173, 206, 256–258]{Paauw.2009}, \ili{Manado Malay} (MM) \citep[25–28]{Stoel.2005}, and \ili{Ternate Malay} (TM) \cite[136–139]{Litamahuputty.2012}. This section compares \isi{reduplication} across these Malay varieties in terms of \isi{lexeme formation} (§\ref{Para_4.3.1}), \isi{lexeme interpretation} (§\ref{Para_4.3.2}), and \isi{interpretational shift} (§\ref{Para_4.3.3}), as far as mentioned in the literature. For comparison, \isi{reduplication} in Papuan Malay (PM) is also included. Also included for comparison is \ili{Standard Indonesian} (SI) (\citealt{MacDonald.1976, Mintz.2002, Sneddon.2010}).


\subsection{Lexeme formation}
\label{Para_4.3.1}
Similar to Papuan Malay, the above-mentioned six Malay varieties also employ full \isi{reduplication}, as shown in \tabref{Table_4.8}. Typically, \isi{reduplication} affects content words, while \isi{reduplication} of function words is rarer. Manado and \ili{Ternate Malay} also employ \isi{reduplication} of bound morphemes. The data in \tabref{Table_4.8} also shows which varieties use a combination of \isi{reduplication} and \isi{affixation}, and in which varieties reduplicated forms without corresponding base words are found. Besides Papuan Malay, only two of the six other \ili{eastern Malay varieties} use partial and imitative \isi{reduplication}, namely Ambon and \ili{Larantuka Malay}.


\begin{table}
\caption{ Lexeme formation across eastern Malay varieties and Standard Indonesian}\label{Table_4.8}

\begin{tabular}{lllllllllll}
\lsptoprule
 \stepcounter{InTableCounter0} \arabic{InTableCounter0}. & \multicolumn{10}{l}{Full reduplication}\\
\midrule
 & \multicolumn{10}{l}{\stepcounter{InTableCounter1} \alph{InTableCounter1}) Content words (productive)}\\
\midrule
& & \textsc{n} & {PM} & {AM} & BM & KM & LM & MM & TM & SI\\
& & \textsc{v} & {PM} & {AM} &  & KM & LM & MM & TM & SI\\
& & \textsc{adv} & {PM} & {AM} &  &  &  & MM &  & SI\\
& & \textsc{num} & {PM} & {AM} &  & KM &  & MM &  & SI\\
& & \textsc{qt} & {PM} &  &  & LM &  &  & & \\
\midrule
& \multicolumn{10}{l}{ \stepcounter{InTableCounter1}\alph{InTableCounter1}) Function words (unproductive)}\\
 \midrule
& & \textsc{pro} & {PM} & {AM} &  & KM & LM & MM &  & SI\\
& & \textsc{dem} & {PM} & {AM} &  &  &  &  &  & \\
& & \textsc{loc} & {PM} & {AM} &  &  &  &  &  & \\
& & \textsc{int} & {PM} & {AM} &  & KM & LM &  &  & \\
& & \textsc{caus} & {PM} & {} &  &  &  &  &  & \\
& & \textsc{recp} & {PM} & {} &  &  &  & MM & TM & \\
\midrule
&  \multicolumn{10}{l}{ \stepcounter{InTableCounter1} \alph{InTableCounter1}) Bound morphemes (unproductive)}\\
\midrule
& & \textsc{pfx} &  &  &  &  &  & MM & TM & \\
\midrule
&  \multicolumn{10}{l}{ \stepcounter{InTableCounter1} \alph{InTableCounter1}) Reduplication and \isi{affixation} (productive)}\\
\midrule
& & {\textsc{rdp} prec. \textsc{affx}} & {PM} & AM &  &  & LM & MM &  & SI\\
& & {\textsc{affx} prec. \textsc{rdp}} & {PM} & AM &  & KM &  &  &  & SI\\
\midrule
&  \multicolumn{10}{l}{ \stepcounter{InTableCounter1} \alph{InTableCounter1}) No corresponding base words (unproductive)}\\
\midrule
& & \textsc{n} & {PM} & {AM} &  &  &  &  &  & SI\\
& & \textsc{v} & {PM} & {AM} &  &  &  &  &  & SI\\
& & \textsc{qt} & {PM} & {} &  &  &  &  &  & \\
& & \textsc{adv} & {} & {AM} &  &  &  &  &  & SI\\
& & \textsc{cnj} & {PM} & {AM} &  &  &  &  &  & SI\\
\midrule
\stepcounter{InTableCounter0} \arabic{InTableCounter0}. & \multicolumn{10}{l}{ Partial reduplication}\\
\midrule
& & productive & {PM} & {AM} &  &  & LM &  &  & \\
& & unproductive & {} & {} &  &  &  &  &  & SI\\
\midrule
\stepcounter{InTableCounter0} \arabic{InTableCounter0}. & \multicolumn{10}{l}{ Imitative \isi{reduplication} (unproductive)}\\
\midrule
& &  & {PM} & {AM} &  &  & LM &  &  & SI\\
\lspbottomrule

\end{tabular}
\end{table}

The data given in \tabref{Table_4.8} shows that \isi{reduplication} in \ili{Ambon Malay} is about as pervasive as in Papuan Malay, with both varieties sharing many features. This applies to the attested \isi{reduplication} types (full, partial, and imitative), as well as to the attested morpheme types which can be reduplicated. For the five other \ili{eastern Malay varieties} and \ili{Standard Indonesian}, \isi{reduplication} seems to play a much lesser role, as shown by the gaps in \tabref{Table_4.8}. For the \ili{eastern Malay varieties}, this applies especially to the \isi{reduplication} of function words; furthermore, these varieties appear not to have reduplicated forms which lack a corresponding unreduplicated base.



Two explanations present themselves for these observations. One explanation is that the commonalities between Papuan Malay and \ili{Ambon Malay}, together with the lack of overlap with the five other \ili{eastern Malay varieties}, are due to the distinct history of Papuan Malay, argued for in §\ref{Para_1.8}. An alternative explanation is that the differences among the \ili{eastern Malay varieties} are due to differing degrees of depth with which the different authors describe \isi{reduplication} in the Malay varieties presented in \tabref{Table_4.8}. This grammar on Papuan Malay, as well as that of \ili{Ambon Malay}, and also those of \ili{Standard Indonesian}, describe \isi{reduplication} as a \isi{word-formation} process rather thoroughly, while the descriptions of the five other \ili{eastern Malay varieties} mention only the most salient features of \isi{reduplication} in these varieties; hence, the rather large number of gaps in \tabref{Table_4.8}.


\subsection{Lexeme interpretation}
\label{Para_4.3.2}
As in Papuan Malay, the \isi{gesamtbedeutung} of \isi{reduplication} in the six other \ili{eastern Malay varieties} is ``a \textsc{higher}/\textsc{lower} \textsc{degree} \textsc{of} \ldots''. \tabref{Table_4.9} gives examples for this \isi{gesamtbedeutung} across the seven Malay varieties.


\begin{table}[b]
\caption{ Gesamtbedeutung of \isi{reduplication} across eastern Malay varieties}\label{Table_4.9}

\begin{tabularx}{\textwidth}{lllXX}
\lsptoprule
 Dimensions & Malay & \multicolumn{1}{c}{Item} & \multicolumn{2}{c}{ Gloss}\\
\midrule
\textsc{aug.quant} & PM & \textitbf{bua{\Tilde}bua} & \textsc{rdp}{\Tilde}fruit & ‘various fruits’\\
& AM & \textitbf{kata{\Tilde}kata} & \textsc{rdp}{\Tilde}word & ‘words’\\
& BM & \textitbf{orang{\Tilde}orang} & \textsc{rdp}{\Tilde}person & ‘people’\\
& KM & \textitbf{buku{\Tilde}buku} & \textsc{rdp}{\Tilde}book & ‘books’\\
& LM & \textitbf{ana{\Tilde}ana} & \textsc{rdp}{\Tilde}child & ‘children’\\
& MM & \textitbf{dua{\Tilde}dua} & \textsc{rdp}{\Tilde}two & ‘all two, both’\\
& TM & \textitbf{ular{\Tilde}ular} & \textsc{rdp}{\Tilde}snake & ‘snakes’\\
\textsc{aug.intens} & PM & \textitbf{pintar{\Tilde}pintar} & \textsc{rdp}{\Tilde}be.clever & ‘be very clever’\\
& AM & \textitbf{biru{\Tilde}biru} & \textsc{rdp}{\Tilde}blue & ‘be very blue’\\
& LM & \textitbf{uma{\Tilde}ame} & \textsc{rdp}{\Tilde}chew & ‘chew wildly’\\
& MM & \textitbf{kita{\Tilde}kita} & \textsc{rdp}{\Tilde}\textsc{1sg} & ‘constantly me’\\
& TM & \textitbf{ba{\Tilde}ba}–\textitbf{diang} & \textsc{rdp}{\Tilde}\textsc{int}–be.quiet & ‘be very quiet’\\
\midrule
\textsc{dim} & PM & \textitbf{kurus{\Tilde}kurus} & \textsc{rdp}{\Tilde}be.thin & ‘rather thin’\\
& AM & \textitbf{malu{\Tilde}malu} & \textsc{rdp}{\Tilde}ashamed & ‘shy as’\\
& KM & \textitbf{apa{\Tilde}apa} & \textsc{rdp}{\Tilde}what & ‘anything’\\
& LM & \textitbf{apa{\Tilde}apa} & \textsc{rdp}{\Tilde}what & ‘anything’\\
& MM & \textitbf{saki{\Tilde}saki} & \textsc{rdp}{\Tilde}be.sick & ‘sickly’\\
\lspbottomrule
\end{tabularx}
\end{table}

\tabref{Table_4.10} demonstrates in more detail which word classes attract \isi{reduplication} and which meaning aspects they convey in all seven Malay varieties.



First, the data in \tabref{Table_4.10} shows that across the seven Malay varieties, \isi{reduplication} of content words tends to convey more than one meaning aspect, while reduplicated function words tend to carry specific meaning aspects, such as \isi{indefiniteness} for interrogatives. The exception is \ili{Manado Malay}, where \isi{reduplication} of content words tends to carry a specific meaning, such as plurality for nouns.



Second, the data in \tabref{Table_4.10} illustrates that in the other \ili{eastern Malay varieties} some meaning aspects also attract a wider range of word classes for \isi{reduplication} than other meaning aspects. This applies to the plurality/diversity, the \isi{intensity}, the \isi{continuation}/\isi{repetition}/habit, and the \isi{indefiniteness} meaning aspects.


\begin{table}[b]

\caption[Word classes and meaning aspects in \isi{reduplication}]{Word classes and meaning aspects in \isi{reduplication} across eastern Malay varieties\footnote{In \tabref{Table_4.10}, the category of prefixes in Manado Malay \citep[27]{Stoel.2005} and Ternate Malay \citep[139]{Litamahuputty.2012} includes the reciprocal marker \textitbf{baku} ‘\textsc{rcp}’. Reduplicated Ternate Malay “activity words” \citep[136–138]{Litamahuputty.2012} are included in the word class of verbs.}}\label{Table_4.10}
{\setlength{\tabcolsep}{3pt}
\centering
\begin{tabularx}{\textwidth}{p{3cm}p{1.5cm}llllll}
%\begin{tabularx}{\textwidth}{llllllll}
\lsptoprule
 & PM & AM & BM & KM & LM & MM &  TM\\
\midrule
\multicolumn{8}{l}{Augmentation (quantity)}\\
\midrule
Continuation/ \isi{repetition}/habit & \textsc{n}, \textsc{v}, \textsc{adv}, \textsc{recp}, \textsc{caus} & \textsc{v} &  & \textsc{v} & \textsc{v} & \textsc{v}, \textsc{pfx} & \textsc{v}, \textsc{pfx}\\
Plurality/ diversity & \textsc{n}, \textsc{v}, \textsc{dem}, \textsc{loc} & \textsc{n}, \textsc{v} & \textsc{n} & \textsc{n} & \textsc{n} & \textsc{n} & \textsc{n}, \textsc{v}\\
Collectivity & \textsc{num}, \textsc{pro} & \textsc{num} &  &  &  & \textsc{num} & \\
Distributiveness & \textsc{num}, \textsc{qt} &  &  &  &  &  & \\
Involvement &  &  &  &  &  & \textsc{pro} & \\
Totality &  & \textsc{n} & \textsc{n} &  &  &  & \\
Aimlessness & \textsc{v} &  &  &  & \textsc{v} &  & \textsc{v}\\
\midrule
\multicolumn{8}{l}{Augmentation (\isi{intensity})}\\
\midrule
Intensity & \textsc{v}, \textsc{adv}, \textsc{loc} & \textsc{v}, \textsc{adv} &  &  & \textsc{v} & \textsc{adv} & \textsc{v}, \textsc{pfx}\\
Immediacy & \textsc{v} &  &  &  &  &  & \\
\midrule
\multicolumn{8}{l}{Diminution}\\
\midrule
Disparagement & \textsc{pro} &  &  &  &  &  & \\
Indefiniteness & \textsc{n}, \textsc{int} & \textsc{dem}, \textsc{int} &  & \textsc{pro}, \textsc{int} & \textsc{pro}, \textsc{int} &  & \\
Attenuation & \textsc{v} & \textsc{v} &  &  &  &  & \\
Vagueness &  & \textsc{adv} &  &  &  &  & \\
Imitation & \textsc{v}, \textsc{pro} & \textsc{v}, \textsc{pro} &  &  &  &  & \\
\lspbottomrule
\end{tabularx}
}

\end{table}



Of all the \ili{eastern Malay varieties}, the different meaning aspects attested in Papuan Malay attract the widest range of different word classes. For \ili{Ambon Malay}, the range of attracted word classes is also rather large. In the other \ili{eastern Malay varieties}, however, the attracted range of word classes is much smaller. At this point, it remains unclear, though, whether these smaller ranges are due to inherent properties of these varieties or due to an incomplete documentation in the respective literature.


Overall, there is not a specific, one-to-one relation between the meaning aspects of the reduplicated lexemes and the syntactic class of the corresponding base words in any of the Malay varieties discussed here.


\subsection{Interpretational shift}
\label{Para_4.3.3}
Interpretational shift of reduplicated lexemes, as described for Papuan Malay (see §\ref{Para_4.2.1.4} and §\ref{Para_4.2.2.8}), is also attested for \ili{Ambon Malay} \citep[118, 123, 125]{vanMinde.1997}, \ili{Larantuka Malay} \citep[126, 270]{Paauw.2009}, \ili{Manado Malay} {\citep[26]{Stoel.2005}}, and \ili{Ternate Malay} {\citep[220]{Litamahuputty.2012}}.



With respect to the patterns of \isi{interpretational shift}, two observations are made which are summarized in \tabref{Table_4.11}. First, in each of the varieties for which \isi{interpretational shift} is mentioned, it is content words that may undergo such a shift. Second, the Malay varieties differ in terms of the syntactic categories of the base and the readings which the reduplicated forms can receive. In Papuan and \ili{Ambon Malay}, nouns and verbs can undergo \isi{interpretational shift}, while in \ili{Manado Malay} only nouns and in Larantuka and \ili{Ternate Malay} only verbs are affected. Most often, such a shift results in the reduplicated form receiving an adverbial reading. Such is the case in Papuan, Ambon, Larantuka, and \ili{Ternate Malay}; the exception is \ili{Manado Malay}. Considerably less often the shift results in a nominal reading (Papuan and \ili{Ambon Malay}) or a verbal reading (Papuan Malay) of reduplicated lexemes.


\begin{table} 
\caption{Patterning of \isi{interpretational shift} across eastern Malay varieties}\label{Table_4.11}

\begin{tabularx}{\textwidth}{llp{3cm}Xp{2cm}}
\lsptoprule
\multicolumn{2}{c}{ Syntactic category} & \multicolumn{2}{c}{ Reduplicated forms and their meanings} &  Received reading\\
\midrule
Nouns & PM & \textitbf{rawa{\Tilde}rawa} & ‘be swampy’ & verbal\\
&  & \textsc{rdp}{\Tilde}swamp &  & \\
&  & \textitbf{malam{\Tilde}malam} & ‘late at night’ & adverbial\\
&  & \textsc{rdp}{\Tilde}night &  & \\
& AM & \textitbf{malang{\Tilde}malang} & ‘during the night’ & adverbial\\
&  & \textsc{rdp}{\Tilde}night &  & \\
& MM & \textitbf{opa{\Tilde}opa} & ‘quite old’ & verbal\\
&  & \textsc{rdp}{\Tilde}grandfather &  & \\
\midrule
Verbs\footnote{The ``\isi{verb}'' category includes \ili{Manado Malay} adjectives and \ili{Ternate Malay} “quality words” {(Litamahuputty 2012: 136–138)}.}
 & PM & \textitbf{gait{\Tilde}gait} & ‘pole’ & nominal\\
&  & \textsc{rdp}{\Tilde}hook &  & \\
&  & \textitbf{taw{\Tilde}taw} & ‘suddenly’ & adverbial\\
&  & \textsc{rdp}{\Tilde}know &  & \\
& AM & \textitbf{gai{\Tilde}gai} & ‘pole’ & nominal\\
&  & \textsc{rdp}{\Tilde}hook &  & \\
&  & \textitbf{kamuka{\Tilde}kamuka} & ‘formerly, earlier’ & adverbial\\
&  & \textsc{rdp}{\Tilde}go.first &  & \\
& LM & \textitbf{tiba{\Tilde}tiba} & ‘suddenly’ & adverbial\\
&  & \textsc{rdp}{\Tilde}arrive &  & \\
& TM & \textitbf{asik{\Tilde}asik} & ‘busily’ & adverbial\\
&  & \textsc{rdp}{\Tilde}busy &  & \\
\lspbottomrule
\end{tabularx}

\end{table}

The ability of reduplicated lexemes to undergo \isi{interpretational shift} seems to be best explained in terms of a slot filling-function of \isi{reduplication}. Cross-linguistically, temporal \isi{noun} phrases, for instance, are prone to fill adverbial slots; an example is the English sentence ``she came home late at night''.



Hence, in this grammar of Papuan Malay, the interpretational shifts described in §\ref{Para_4.2.1.4} and §\ref{Para_4.2.2.8} are taken to result from a slot filling-function of \isi{reduplication}. That is, \isi{reduplication} enables base words to fill different syntactic slots, such as an adverbial or a nominal slot.

 
In \ili{Ternate Malay}, interpretational shifts also seem to be the results of a slot-filling function of \isi{reduplication}, with {\citet[220]{Litamahuputty.2012} }noting that “both reduplicated quality words and activity words may serve to express manner when they immediately follow an activity”. For \ili{Ambon Malay}, by contrast, \citet[118, 123, 125]{vanMinde.1997} considers the observed interpretational shifts as “transpositions” which result from “derivational” processes. For \ili{Manado Malay}, {\citet[26]{Stoel.2005}} notes that when kinship terms or similar words are reduplicated “then the reduplicated form is an adjective referring to a certain age group”. This statement suggests that {\citet{Stoel.2005}} considers interpretational shifts to result from a category-changing function of \isi{reduplication}. For \ili{Larantuka Malay}, \citet[126, 270]{Paauw.2009} does not discuss the attested interpretational shifts.


\section{Summary}
\label{Para_4.4}
Reduplication in Papuan Malay is a very productive morphological device for deriving new words. In terms of \isi{lexeme formation}, three different types of \isi{reduplication} are attested: full, partial, and imitative \isi{reduplication}. The most common type is full \isi{reduplication}, which involves the \isi{repetition} of an entire root, stem, or word; bound morphemes are not reduplicated. Full \isi{reduplication} usually applies to content words, although some function words can also be reduplicated. Partial and imitative \isi{reduplication} are rare. The \isi{gesamtbedeutung} of \isi{reduplication} is ``a \textsc{higher}/\textsc{lower} \textsc{degree} \textsc{of} \ldots'' in the sense of augmentation and diminution, applying \citegen[1151]{Kiyomi.2009} terminology. There is, however, no specific, one-to-one relation between the meaning aspects of the reduplicated lexemes and the syntactic class of the corresponding base words.


\newpage
A comparison of \isi{reduplication} in Papuan Malay and six  other \ili{eastern Malay varieties} shows that Papuan Malay shares many features with \ili{Ambon Malay}. In both varieties, \isi{reduplication} plays an important role. In Banda, Kupang, Manado, Larantuka, and \ili{Ternate Malay}, by contrast, \isi{reduplication} seems to be much less pervasive. These commonalities and differences may well point to the particular history of Papuan Malay, argued for in §\ref{Para_1.8}. The observed differences could, however, also result from gaps in the descriptions of Banda, Kupang, Manado, Larantuka, and \ili{Ternate Malay}.

