\chapter[Demonstratives and locatives]{Demonstratives and locatives}
\label{Para_7}
This chapter discusses the Papuan Malay demonstratives and locatives, focusing on their different functions and domains of use. Demonstratives and locatives are deictic expressions that provide orientation to the hearer in the outside world and in the speech situation. Papuan Malay employs a two-term \isi{demonstrative} system and a three-term \isi{locative} system as shown in \tabref{Table_7.1}.



\begin{table}
\caption{Papuan Malay demonstratives and locatives}\label{Table_7.1}

\begin{tabular}{llll}
\lsptoprule

\multicolumn{2}{c}{ Papuan Malay \textsc{dem}} & \multicolumn{2}{c}{ Gloss}\\\midrule
Proximal & \textitbf{ini} / \textitbf{ni} & \textsc{\textup{‘}}\textsc{d.prox}\textsc{\textup{’}} & ‘this’\\
Distal & \textitbf{itu} / \textitbf{tu} & ‘\textsc{d.dist}’ & ‘that’\\
\midrule
\multicolumn{2}{c}{ Papuan Malay \textsc{loc}} & \multicolumn{2}{c}{ Gloss}\\\midrule
Proximal & \textitbf{sini} & ‘\textsc{l.prox}’ & ‘here’\\
Medial & \textitbf{situ} & ‘\textsc{l.med}’ & ‘there’\\
Distal & \textitbf{sana} & ‘\textsc{l.dist}’ & ‘over there’\\
\lspbottomrule
\end{tabular}
\end{table}

Both systems are distance oriented. Cross-linguistically, such systems “indicate the relative distance of a referent in the speech situation vis-à-vis the deictic center” which “is defined by the speaker’s location at the time of the utterance” \citep[430]{Diessel.2006}. The unmarked deictic center is defined by the speaker in the ``here'' and ``now''. In reported \isi{direct speech} and narratives, however, the deictic center is readjusted to the \isi{reported speech} situation and defined by the quoted speakers and the location and time of their speaking.



While their major domain of use is to provide spatial orientation, the demonstratives and locatives also signal distance in metaphorical terms. Their main domains of use are presented in \tabref{Table_7.2}.



\begin{table}
\caption{Demonstratives (\textsc{dem}) and locatives (\textsc{loc}) and their of domains of use}\label{Table_7.2}
\begin{tabularx}{\textwidth}{p{3.25cm}Xcc}
\lsptoprule
 \multicolumn{1}{c}{Domains of use} &  \multicolumn{1}{c}{Function} &  \textsc{dem} &  \textsc{loc}\\
\midrule
Spatial & to provide spatial orientation to the hearer & X &  X\\
Figurative \isi{locational} & to signal a figurative \isi{locational} endpoint &  &  X\\
Temporal & to indicate the temporal setting of the situation or event talked about & X &  X\\
Psychological & to signal the speaker’s psychological involvement with the situation or event talked about & X &  X\\
Identificational & to aid in the identification of a referent & X & \\
Textual anaphoric & to keep track of a discourse antecedent & X &  X\\
Textual discourse & to establish an overt link between two & X & \\
\hspace{10pt}deictic &  propositions\\
Placeholder & to substitute for specific lexical items in the context of word-formulation trouble & X & \\
\lspbottomrule
\end{tabularx}
\end{table}

In the following sections the demonstratives (§\ref{Para_7.1}) and locatives (§\ref{Para_7.2}) are investigated in more detail. The ways in which the demonstratives and locatives can be combined are discussed in §\ref{Para_7.3}. The different findings for the demonstratives and locatives are summarized and compared in §\ref{Para_7.4}.


\section{Demonstratives}
\label{Para_7.1}
In the following sections, the syntactic properties and forms of the Papuan Malay demonstratives are reviewed and discussed (§\ref{Para_7.1.1}), followed by an in-depth discussion of their different functions and domains of use (§\ref{Para_7.1.2}).


\subsection{Syntax and forms of demonstratives}
\label{Para_7.1.1}
The distributional properties of the demonstratives are briefly reviewed in §\ref{Para_7.1.1.1}. This review is followed in §\ref{Para_7.1.1.2} by a discussion of the distribution and frequencies of the long and short \isi{demonstrative} forms.


\subsubsection[Distributional properties of demonstratives]{Distributional properties of demonstratives}
\label{Para_7.1.1.1}
The Papuan Malay demonstratives have the following distributional properties (for more details see §\ref{Para_5.6}):


\begin{enumerate}
\item 
Co-occurrence with \isi{noun} phrases (adnominal uses): \textsc{n}/\textsc{np} \textsc{dem} (§\ref{Para_5.6.1})
\item 
Substitution for \isi{noun} phrases (pronominal uses) (§\ref{Para_5.6.2})
\item 
Modification with relative clauses (pronominal uses): \textsc{dem rel} (§\ref{Para_5.6.2}).
\item 
Co-occurrence with verbs or adverbs (adverbial uses): \textsc{v dem} and \textsc{adv dem} (§\ref{Para_5.6.3})
\item 
Stacking of demonstratives: \textsc{dem} \textsc{dem} and \textsc{n} \textsc{dem} \textsc{dem} (§\ref{Para_5.6.4})

\end{enumerate}
\subsubsection[Distribution of the long versus short {demonstrative} forms]{Distribution of the long versus short {demonstrative} forms}
\label{Para_7.1.1.2}
This section investigates the distribution and frequencies of the long versus the short \isi{demonstrative} forms and explores the factors that contribute to this distribution. The data shows that the reduced \isi{demonstrative} forms are fast-speech phenomena that fulfill the same syntactic functions as the long forms. With two exceptions they are also employed in the same domains of use.



The corpus contains a total of 2,304 \textitbf{ini} ‘\textsc{d.prox}’ tokens of which 2,046 (88.8\%) are the long form and 258 (11.2\%) the short form, as shown in \tabref{Table_7.3}. The number of \textitbf{itu} ‘\textsc{d.dist}’ token is considerably larger with a total of 4,159 token of which 3,491 (83.9\%) are the long form and 668 (16.1\%) the short form.



\begin{table}
\caption{Demonstratives according to their phonological environment}\label{Table_7.3}

\begin{tabular}{l*{8}{r}} 
\lsptoprule
& \multicolumn{2}{c}{ \textitbf{ini}} & \multicolumn{2}{c}{ \textitbf{itu}} & \multicolumn{2}{c}{ \textitbf{ni}} & \multicolumn{2}{c}{ \textitbf{tu}}\\\midrule

Clause-initial &  86 &  4\% &  279 &  8\% &  7 &  3\% &  28 &  4\%\\
Post-vowel &  1,156 &  57\% &  1,671 &  48\% &  222 &  86\% &  531 &  80\%\\
Post-nasal &  432 &  21\% &  833 &  24\% &  18 &  7\% &  67 &  10\%\\
Post-consonant &  372 &  18\% &  708 &  20\% &  11 &  4\% &  42 &  6\%\\\midrule
Total &  2,046 &  100\% &  3,491 &  100\% &  258 &  100\% &  668 &  100\%\\
\lspbottomrule
\end{tabular}
\end{table}

The long forms occur in all phonological environments with \tabref{Table_7.3} indicating some differences in their distribution, however. First, the high number of long demonstratives (about 50\%) that follow lexical items with word-final vowels is due to the fact that in Papuan Malay more lexical items have a word-final vowel than a word-final nasal or consonant. Second, the low number of long demonstratives ({\textless}10\%) in clause-initial position is due to the fact that the number of pronominally used demonstratives is much lower than that of adnominally or adverbially used ones.



The short forms also occur in all phonological environments. \tabref{Table_7.3} shows, however, that most of them (${\geq}$80\%) occur after lexical items with a word-final vowel. The percentage of short demonstratives following lexical items with a word-final nasal or consonant is considerably lower compared to the long forms.



Interestingly, the short demonstratives also occur in clause-initial position: of the seven short \textitbf{ini} ‘\textsc{d.prox}’ tokens occurring clause-initially, five occur at the beginning of an utterance. The remaining two tokens occur clause-initially in the middle of an utterance. In both cases the preceding clause-final lexical item has a word-final vowel which appears to \isi{condition} these two short forms. Of the 28 short \textitbf{itu} ‘\textsc{d.dist}’ tokens occurring clause-initially, eleven occur at the beginning of an utterance. The remaining 17 tokens occur clause-initially in the middle of an utterance. Of these, 14 tokens are conditioned by the preceding word-final phoneme: in eleven cases the preceding clause-final lexical item has a word-final vowel while in the remaining three cases the preceding clause-final lexical item has coda /t/.



These findings suggest that for the most part the short \isi{demonstrative} forms are conditioned by the environment of their occurrence and constitute fast-speech phenomena. The listed exceptions require further investigation.



The short demonstratives fulfill the same syntactic functions as the long ones. With two exceptions they are also employed in the same domains of use. The data does, however, suggest some preferences. \tabref{Table_7.4} presents the short demonstratives according to their syntactic functions. The data shows a clear preference for their adnominal uses. Of the 209 adnominally used short \textitbf{ini} ‘\textsc{d.prox}’ tokens, 170 (81\%) modify \isi{noun} phrases with nominal heads, while 34 (16\%) modify \isi{noun} phrases with pronominal heads; the remaining five tokens modify interrogatives. Likewise, of the 482 adnominally used short \textitbf{itu} ‘\textsc{d.dist}’ tokens, 345 (72\%) modify \isi{noun} phrases with nominal heads, while 105 (22\%) modify \isi{noun} phrases with pronominal heads, and 30 (6\%) modify locatives; the remaining two tokens modify interrogatives. Considerably less frequently, the short demonstratives have pronominal uses (${\leq}$8\%) or adverbial uses (${\leq}$13\%). (Given their large numbers, the long demonstratives have not been quantified according to their syntactic functions.)



\begin{table}
\caption{Reduced demonstratives according to their syntactic functions}\label{Table_7.4}

\begin{tabular}{l*{4}{r}}
\lsptoprule

 \multicolumn{1}{c}{Syntactic functions} & \multicolumn{2}{c}{ \textitbf{ni}} & \multicolumn{2}{c}{ \textitbf{tu}}\\\midrule
Adnominal uses &  209 &  81\% &  482 &  72\%\\
Pronominal uses &  20 &  8\% &  44 &  7\%\\
Adverbial uses (verbal modifier) &  20 &  8\% &  86 &  13\%\\
Adverbial uses (adverbial modifier) &  9 &  3\% &  56 &  8\%\\\midrule
Total &  258 &  100\% &  668 &  100\%\\
\lspbottomrule
\end{tabular}
\end{table}

The short demonstratives are also employed in the same domains of use as the long ones, except for the identificational and \isi{placeholder} uses. Given their rather frequent alternative readings (§\ref{Para_7.1.2}), the demonstratives have not been quantified according to their domains of use.



Examples (\ref{Table_Example_7.5.1}) to (\ref{Table_Example_7.6.8})  present an overview of the short demonstratives according to their syntactic functions and domains of use. For \textitbf{ini} ‘\textsc{d.prox}’ in examples (\ref{Table_Example_7.5.1}) to (\ref{Table_Example_7.5.7}) , its syntactic functions are exemplified as follows: adnominal in (\ref{Table_Example_7.5.1}) to (\ref{Table_Example_7.5.4}); pronominal in (\ref{Table_Example_7.5.5}); and adverbial in (\ref{Table_Example_7.5.6}) and (\ref{Table_Example_7.5.7}). Its domains of use are given as follows: spatial in (\ref{Table_Example_7.5.1a}) and (\ref{Table_Example_7.5.5a}); temporal in (\ref{Table_Example_7.5.1b}); psychological in (\ref{Table_Example_7.5.1c}) (\isi{emotional involvement}), (\ref{Table_Example_7.5.3}), ({\ref{Table_Example_7.5.4}), (\ref{Table_Example_7.5.6}), (\ref{Table_Example_7.5.7}) (\isi{vividness}), and (\ref{Table_Example_7.5.2}) (contrast); and textual in (\ref{Table_Example_7.5.5b}) (anaphoric) and (\ref{Table_Example_7.5.5c}) (discourse deictic).\\

\newpage


\begin{styleExampleTitle}
Short \textitbf{ini} ‘\textsc{d.prox}’: Syntactic functions and domains of use]\footnote{Documentation: (1) 081025-003-Cv.0042, 080918-001-CvNP.0055, 081025-003-Cv.0135; (2) 081014-007-Pr.0053; (3) 080922-010a-NF.0101; (4) 080922-004-Cv.0017; (5) 080922-010a-NF.0081, 080917-006-CvHtEx.0005, 080917-010-CvEx.0116; (6) 080919-005-Cv.0015; (7) 080922-001a-CvPh.0735.}\\

\end{styleExampleTitle}
\ea
\label{Table_Example_7.5.1}
Adnominal uses (modifies nouns) / Spatial uses in (\ref{Table_Example_7.5.1a}), temporal uses in (\ref{Table_Example_7.5.1b}), psychological uses (\isi{emotional involvement}) in (\ref{Table_Example_7.5.1c})\\
\ea
\label{Table_Example_7.5.1a} 
\gll {bawa} \bluebold{mace} \bluebold{ni} {ke} {ruma-sakit}\\
 {bring} {woman} {\textsc{d.prox}} {to} {hospital }\\
\glt ‘(I) brought \bluebold{(my) wife} \bluebold{here} to the hospital’ \\
\vspace{5pt}

\ex\label{Table_Example_7.5.1b}
\gll  {sekertaria} {ni} \bluebold{pagi} \bluebold{ni} {sedi} {\ldots}  \\
{secretariat} \textsc{d.prox} {morning} \textsc{d.prox} {be.sad}   \\
\glt ‘\bluebold{this morning} the secretary was (very) sad’\footnote{The speaker made a mistake; instead of saying \textitbf{sekertaris} ‘secretary’, he produced \textitbf{sekertaria} ‘secretariat’.}\\
\vspace{5pt}

\ex\label{Table_Example_7.5.1c}
\gll  {kalo} \bluebold{Ise} \bluebold{ni} \bluebold{selesay \ldots} { } \\
 {if} {Ise} \textsc{d.prox} {finish}\\
\glt ‘when \bluebold{this (my daughter) Ise} has finished (school) \ldots’\footnote{The referent \textitbf{Ise} was not present when the speaker talked about his daughter; hence proximal \textitbf{ini} ‘\textsc{d.prox}’ does not have spatial uses in this context.} \\
\z
\z
\ea
\label{Table_Example_7.5.2}
Adnominal uses (modifies personal pronouns) / Psychological uses (contrast)\\
\gll  \bluebold{kitong} \bluebold{ni} {tra} {bisa} \\
\textsc{1pl} \textsc{d.prox} \textsc{neg} be.able  \\
\glt  ‘\bluebold{we, by contrast}, can’t (work like   that)’\\
\z
\ea\label{Table_Example_7.5.3}
Adnominal uses (modifies demonstratives) / Psychological uses (\isi{vividness})\\
\gll  {dia} {buang} \bluebold{ini} \bluebold{ni}  \\
 \textsc{3sg} throw(.away) \textsc{d.prox} \textsc{d.prox}  \\
\glt ‘he threw away \bluebold{these very (ones)}’\\
\z
\ea\label{Table_Example_7.5.4}
Adnominal uses (modifies interrogatives) / Psychological uses (\isi{vividness})\\
\gll  {de} {mo} {ke} \bluebold{mana} \bluebold{ni}?\\
 \textsc{3sg} want to where \textsc{d.prox}  \\
\glt ‘\bluebold{where (}\blueboldSmallCaps{emph}\bluebold{)} does he want to go?’\\
\z
\ea\label{Table_Example_7.5.5}
Pronominal uses / Spatial uses in (\ref{Table_Example_7.5.5a}), textual uses (anaphoric) in (\ref{Table_Example_7.5.5b}), textual uses (discourse deictic) in (\ref{Table_Example_7.5.5c})\\
\ea\label{Table_Example_7.5.5a}
\gll  {ada} \bluebold{ni}  \\
 exist \textsc{d.prox}  \\
\glt ‘(the fish) are \bluebold{here}’\\
\vspace{5pt}

\ex\label{Table_Example_7.5.5b}
\gll   {de} {menyala} \bluebold{ni}  \\
 \textsc{3sg} put.fire.to \textsc{d.prox}  \\
\glt ‘he puts fire to \bluebold{this}’\\
\vspace{5pt}

\ex\label{Table_Example_7.5.5c}
\gll  \bluebold{ni} {usul} {saja}  \\
 \textsc{d.prox} proposal just  \\
\glt ‘\bluebold{this} is just a proposal’\\
\z
\z
\ea\label{Table_Example_7.5.6}
Adverbial uses (modifies verbs) / Psychological uses (\isi{vividness})\\
\gll  {sa} {masi} \bluebold{hidup} \bluebold{ni}  \\
 \textsc{1sg} still live \textsc{d.prox}  \\
\glt ‘I’m still \bluebold{very} \bluebold{much} \bluebold{alive}’\\
\z
\ea\label{Table_Example_7.5.7} Adverbial uses (modifies adverbs) / Psychological uses (\isi{vividness})\\
\gll {\ldots} {tapi} \bluebold{skarang} \bluebold{ni} {ada} {libur}  \\
{} but now \textsc{d.prox} exist vacation  \\
\glt ‘{\ldots} but \bluebold{right now} (we) are on vacation’ \\
\z

Examples  (\ref{Table_Example_7.6.1}) to (\ref{Table_Example_7.6.8}) list the syntactic functions and domains of use of \textitbf{itu} ‘\textsc{d.dist}’. Its syntactic functions are listed as follows: adnominal in (\ref{Table_Example_7.6.1}) to (\ref{Table_Example_7.6.5}); pronominal in (\ref{Table_Example_7.6.6}); and adverbial in (\ref{Table_Example_7.6.7}) and (\ref{Table_Example_7.6.8}). Its domains of use are given as follows: spatial in (\ref{Table_Example_7.6.6a}); temporal in (\ref{Table_Example_7.6.1a}); psychological in (\ref{Table_Example_7.6.1b}), (\ref{Table_Example_7.6.4}) (\isi{emotional involvement}), (\ref{Table_Example_7.6.3}) (\ref{Table_Example_7.6.7}) (\isi{vividness}), and (\ref{Table_Example_7.6.2}), (\ref{Table_Example_7.6.8}) (contrast); textual in (\ref{Table_Example_7.6.1c}) (anaphoric), and (\ref{Table_Example_7.6.6b}), (\ref{Table_Example_7.6.6b}) (discourse deictic).\\


\begin{styleExampleTitle}
Short \bluebold{itu} ‘\textsc{d.dist}’: Syntactic functions and domains of use\footnote{Documentation: (1) 081011-005-Cv.0001, 081025-009b-Cv.0016, 081014-004-Cv.0019; (2) 080922-001a-CvPh.0455; (3) 081115-001a-Cv.0145; (4) 081109-001-Cv.0092; (5) 081006-022-CvEx.0150; (6) 081025-009b-Cv.0006, 081006-022-CvEx.0113, 081013-002-Cv.0003; (7) 081023-001-Cv.0020; (8) 081115-001a-Cv.0058.}
\end{styleExampleTitle}

\ea
\label{Table_Example_7.6.1} Adnominal uses (modifies nouns) / Temporal uses in (\ref{Table_Example_7.6.1a}), psychological uses (\isi{emotional involvement}) in (\ref{Table_Example_7.6.1b}), discourse uses (anaphoric) in (\ref{Table_Example_7.6.1c})\\
\ea
\label{Table_Example_7.6.1a}
\gll  \bluebold{rabu} \bluebold{tu} {\ldots} {ko} {datang}  \\
Wednesday \textsc{d.dist} { } \textsc{2sg} come   \\
\glt {‘\bluebold{next Wednesday }you’ll come’}\\
\vspace{5pt}
\ex
\label{Table_Example_7.6.1b}
\gll  \bluebold{ko} \bluebold{pu} \bluebold{swara} \bluebold{tu} \bluebold{bahaya}  \\
 \textsc{2sg} \textsc{poss} voice \textsc{d.dist} be.dangerous \\
\glt {‘\bluebold{that voice of yours} is dangerous’}
\vspace{5pt}
\ex
\label{Table_Example_7.6.1c}
 \gll  \bluebold{Herman} \bluebold{tu} {biasa} {tida} {\ldots}  \\
 Herman \textsc{d.dist} be.usual \textsc{neg}  \\
\glt ‘\bluebold{that} \bluebold{Herman} usually (can)not {\ldots}’\\
\z
\z
\ea
\label{Table_Example_7.6.2} Adnominal uses (modifies personal pronouns) / Psychological uses (contrast)\\
\gll  \bluebold{sa} \bluebold{tu} {rajing} {skola}   \\
 \textsc{1sg} \textsc{d.dist} be.diligent go.to.school \\
\glt ‘\bluebold{I,} \bluebold{nonetheless}, go to school  diligently’ \\
\z
\ea
\label{Table_Example_7.6.3}
Adnominal uses (modifies demonstratives) / Psychological uses (\isi{vividness})\\
\gll  \bluebold{itu} \bluebold{tu} {kata}{\Tilde}{kata} {dasar} {\ldots}  \\
\textsc{d.dist} \textsc{d.dist} \textsc{rdp}{\Tilde}word base   \\
\glt ‘\bluebold{that very} (word belongs to) the  basic words \ldots’ \\
\z
\ea
\label{Table_Example_7.6.4} Adnominal uses (modifies interrogatives) / Psychological uses (\isi{emotional involvement})\\
\gll  {itu} \bluebold{apa} \bluebold{tu}{?}  \\
 \textsc{d.dist} what \textsc{d.dist}  \\
\glt {‘\bluebold{what (}\blueboldSmallCaps{emph}\bluebold{)} was that?}\\
\z
\ea
\label{Table_Example_7.6.5}
Adnominal uses (modifies locatives) / Psychological uses (\isi{vividness})\\
\gll  {di} \bluebold{sini} \bluebold{tu} {ada} {orang} {swanggi} {satu}  \\
 at \textsc{l.prox} \textsc{d.dist} exist person nocturnal.evil.spirit one \\
\glt ‘\bluebold{here (}\blueboldSmallCaps{emph}\bluebold{)} there is a certain evil  sorcerer’ \\
\z
\ea
\label{Table_Example_7.6.6}
Pronominal uses / Spatial uses in (\ref{Table_Example_7.6.6a}), textual uses (discourse deictic) in (\ref{Table_Example_7.6.6b}) and (\ref{Table_Example_7.6.6c})\\
\ea
\label{Table_Example_7.6.6a}
\gll  {de} {ada} \bluebold{tu}{,} {de} {ada} \bluebold{tu}  \\
 \textsc{3sg} exist \textsc{d.dist} \textsc{3sg} exist \textsc{d.dist}  \\
\glt ‘she is \bluebold{over there}, she is \bluebold{over there}’\\
\vspace{5pt}
\ex
\label{Table_Example_7.6.6b}
\gll   {dorang} {liat} {kitorang,} \bluebold{tu} {herang} \\
 \textsc{3pl} see \textsc{1pl} \textsc{d.dist} feel.surprised(.about)  \\
\glt  ‘they see us, \bluebold{that’s} surprising’\\
\vspace{5pt}
\ex
\label{Table_Example_7.6.6c}
\gll  \bluebold{tu} \bluebold{yang} {sa} {tampeleng} {Aleks}  \\
   \textsc{d.dist} \textsc{rel} \textsc{1sg} slap.on.face/ears Aleks  \\
\glt  ‘\bluebold{that’s why} I slapped Aleks in the face’ \\
\z
\z
\ea
\label{Table_Example_7.6.7}
Adverbial uses (modifies verbs) / Psychological uses (\isi{vividness})\\
\gll  {tong} \bluebold{maing} \bluebold{tu} {hancur}\\
 \textsc{1pl} play \textsc{d.dist} be.shattered \\
\glt  ‘we did our \bluebold{very playing} poorly’\\
\z
\ea
\label{Table_Example_7.6.8}Adverbial uses (modifies adverbs) / Psychological uses (contrast)\\
\gll  {de} \bluebold{skarang} \bluebold{tu} {tida} {terlalu} {\ldots} \\
\textsc{3sg} now \textsc{d.dist} \textsc{neg} too \\
\glt  ‘he’s \bluebold{now (as opposed to the past)} not too \ldots’\\
\z

In summary, the data in the corpus suggests that the short \isi{demonstrative} forms are fast-speech phenomena that for the most part are conditioned by their phonological environment. The data also shows that the long and short \isi{demonstrative} forms fulfill the same syntactic functions. Moreover, they are employed in the same domains of use with two exceptions (the identificational and \isi{placeholder} uses).


\subsection{Functions of demonstratives}
\label{Para_7.1.2}
The Papuan Malay demonstratives have a range of different functions and uses which are discussed in more detail in the following sections: spatial uses in §\ref{Para_7.1.2.1}, temporal uses in §\ref{Para_7.1.2.2}, psychological uses in §\ref{Para_7.1.2.3}, identificational uses in §\ref{Para_7.1.2.4}, textual uses in §\ref{Para_7.1.2.5}, and \isi{placeholder} uses in §\ref{Para_7.1.2.6}. Unless the context of an utterance is clear and explicit, the specific domain of use of the \isi{demonstrative} may have multiple possible readings


\subsubsection[Spatial uses of demonstratives]{Spatial uses of demonstratives}
\label{Para_7.1.2.1}
The major domain of use for the demonstratives is to provide spatial orientation. This is achieved by drawing the hearer’s attention to specific entities in the discourse or surrounding situation. Proximal \textitbf{ini} ‘\textsc{d.prox}’ indicates that the referent is conceived as spatially close to the speaker, whereas \textitbf{itu} ‘\textsc{d.dist}’ signals that the referent is conceived as being located further away. This distinction is shown in three sets of contrastive examples.



In the first set of examples in (\ref{Example_7.1}) the contrast is illustrated for the adnominally used demonstratives, each of them modifying the common \isi{noun} \textitbf{ruma} ‘house’. This example is part of a conversation that took place at the speaker’s house. Employing \textitbf{ini} ‘\textsc{d.prox}’, the speaker relates her plans to move from her current house, \textitbf{ruma ini} ‘this house’, to a different house in a neighboring village. Because the new house is smaller than the older one, the speaker’s husband is going to enlarge \textitbf{ruma itu} ‘that house’, with \textitbf{itu} ‘\textsc{d.dist}’ indicating that the new house is located at some distance.


\begin{styleExampleTitle}
Spatial uses: Examples set \#1
\end{styleExampleTitle}

\ea
\label{Example_7.1}
\gll {{ini}} {{kasi}} {{tinggal,}} {{ana{\Tilde}ana}} {{dong}} {{tinggal,}} {tong} {pi}\\ %
 {\textsc{d.prox}}  {give}  {stay}  {\textsc{rdp}{\Tilde}child}  {\textsc{3pl}}  {stay}  \textsc{1pl}  go\\
\gll tinggal  {di}  {Sawar}  {sana}  {\ldots}  {\bluebold{ruma}}  {\bluebold{ini}}  {tinggal}  {\ldots}\\
 stay  {at}  {Sawar}  {\textsc{l.dist}}   { }  {house}  {\textsc{d.prox}}  {stay}  \\
\gll {baru}  {\bluebold{ruma}}  {\bluebold{itu}}  {biking}  {besar}\\
 {and.then}  {house}  {\textsc{d.dist}}  {make}  {be.big}\\
\glt 
‘(we’ll) leave this (house) behind, the children will stay (here), (and) we’ll move to Sawar over there \ldots, (we’ll) leave \bluebold{this house} behind {\ldots} and then (we’ll) make \bluebold{that house} (in Sawar) bigger’ \textstyleExampleSource{[081110-001-Cv.0012/0022/0025/0027]}
\z


The second set of examples in (\ref{Example_7.2}) and (\ref{Example_7.3}) illustrates how the pronominally used demonstratives signal spatial distance. The example in (\ref{Example_7.2}) occurred when the speaker and his brother were fishing. When asked where he had put the fish they had just caught, the speaker employs \textitbf{ini} ‘\textsc{d.prox}’ to convey that the fish \textitbf{ada ini} ‘are here’, in the bucket right next to him. In (\ref{Example_7.3}), the speaker replies to the question where a certain other person was. Employing \textitbf{itu} ‘\textsc{d.dist}’ the speaker states that \textitbf{de ada tu} ‘she is over there’.


\begin{styleExampleTitle}
Spatial uses: Examples set \#2
\end{styleExampleTitle}

\ea
\label{Example_7.2}
\gll {ada} {\bluebold{ni}}\\ %
 exist  \textsc{d.prox}\\
\glt 
[Reply to a question:] ‘(the fish) are \bluebold{here}’ \textstyleExampleSource{[080917-006-CvHt.0005]}
\z

\ea
\label{Example_7.3}
\gll {de} {ada} {\bluebold{tu},} {de} {ada} {\bluebold{tu}}\\ %
 \textsc{3sg}  exist  \textsc{d.dist}  \textsc{3sg}  exist  \textsc{d.dist}\\
\glt 
[Reply to a question:] ‘she’s \bluebold{over there}, she’s \bluebold{over there}’ \textstyleExampleSource{[081025-009b-Cv.0006]}
\z



In the third set of examples in (\ref{Example_7.4}) and (\ref{Example_7.5}), the demonstratives are used adverbially. The utterance in (\ref{Example_7.4}) occurred during a discussion about the teenagers living in the house. Noting that they are ill-behaved, the speaker uses \textitbf{ini} ‘\textsc{d.prox}’ to assert that they \textitbf{tinggal ini} ‘live here’ in this house. In (\ref{Example_7.5}) the speaker relates that she used to live in a different part of the regency, namely in Takar. Employing \textitbf{itu} ‘\textsc{d.dist}’, the speaker maintains that she used to \textitbf{tinggal itu} ‘live there’.


\begin{styleExampleTitle}
Spatial uses: Examples set \#3
\end{styleExampleTitle}

\ea
\label{Example_7.4}
\gll {ko} {\bluebold{tinggal}} {\bluebold{ini}}\\ %
 \textsc{2sg}  stay  \textsc{d.prox}\\
\glt 
‘you \bluebold{live here}’ \textstyleExampleSource{[081115-001b-Cv.0030]}
\z

\ea
\label{Example_7.5}
\gll {{waktu}} {{kitong}} {dari} {{Jayapura}} {{baru}} {{pulang}} {ke} {kampung}\\ %
 {when}  {\textsc{1pl}}  from  {Jayapura}  {and.then}  {go.home}  to  village\\
\gll di  {Takar}  {Pante-Timur,}  {baru}  {kitong}  \bluebold{tinggal}  {\bluebold{itu}}\\
 at  {Takar}  {Pante-Timur}  {and.then}  {\textsc{1pl}}  stay  {\textsc{d.dist}}\\
\glt
‘when we (were back) from Jayapura, then (we) returned home to the village at Takar at Pante-Timur, and then we \bluebold{lived there}’ \textstyleExampleSource{[081006-022-CvEx.0159]}
\z



\subsubsection[Temporal uses of demonstratives]{Temporal uses of demonstratives}
\label{Para_7.1.2.2}
In their temporal uses, the demonstratives signal the temporal setting of the situation or event talked about in terms of some temporal reference point. This function is attested for the adnominally, pronominally, and adverbially used demonstratives.



The (near) contrastive examples in (\ref{Example_7.6}) to (\ref{Example_7.8}) demonstrate the temporal uses of the adnominally used demonstratives. Proximal \textitbf{ini} ‘\textsc{d.prox}’ signals that the event is temporally close to the current speech situation, as in \textitbf{hari ni} ‘today’ in (\ref{Example_7.6}). By contrast, \textitbf{itu} ‘\textsc{d.dist}’ indicates that the temporal reference point is located at some distance from the current speech situation, either in the past as in \textitbf{hari itu} ‘that day’ in (\ref{Example_7.7}) or in the future as in \textitbf{rabu tu, hari kamis itu} ‘next Wednesday, next Thursday’ in (\ref{Example_7.8}).


\begin{styleExampleTitle}
Temporal uses of the adnominally used demonstratives
\end{styleExampleTitle}

\ea
\label{Example_7.6}
\gll {\bluebold{hari}} {\bluebold{ni}} {ko} {kasi} {makang,} {nanti} {\ldots}\\ %
 day  \textsc{d.prox}  \textsc{2sg}  give  food  very.soon  \\
\glt 
[About helping each other:] ‘\bluebold{today} you feed (others), at some point in the future [they’ll feed your children]’ \textstyleExampleSource{[081110-008-CvNP.0254]}
\z

\ea
\label{Example_7.7}
\gll {yo,} {{dong}} {{dua}} {pergi} {ke} {skola} {lagi,} {\bluebold{hari}} {\bluebold{itu}} {dong}\\ %
 yes  {\textsc{3pl}}  {two}  go  to  school  again  day  \textsc{d.dist}  \textsc{3pl}\\
\gll {ada}  {meter}  {sedikit}\\
 {exist}  {meter}  {few}\\
\glt 
‘yes, they both went to school, \bluebold{that day} they were a little drunk’ \textstyleExampleSource{[081115-001a-Cv.0038]}
\z

\ea
\label{Example_7.8}
\gll {\bluebold{rabu}} {\bluebold{tu},} {\bluebold{hari}} {\bluebold{kamis}} {\bluebold{itu},} {ko} {datang} {\ldots}\\ %
 Wednesday  \textsc{d.dist}  day  Thursday  \textsc{d.dist}  \textsc{2sg}  come  \\
\glt 
‘\bluebold{next Wednesday, next Thursday}, you’ll come \ldots’ \textstyleExampleSource{[081011-005-Cv.0001]}
\z



Pronominally used demonstratives also have temporal uses as shown in (\ref{Example_7.9}) and (\ref{Example_7.10}). Again, \textitbf{ini} ‘\textsc{d.prox}’ in (\ref{Example_7.9}) indicates that the event is temporally close to the current speech situation: \textitbf{ini} ‘right now’ (literally ‘this (is when)’). Distal \textitbf{itu} ‘\textsc{d.dist}’ in (\ref{Example_7.10}), by contrast, signals temporal distance: \textitbf{itu} ‘at that time’ (literally ‘that (is when)’).


\begin{styleExampleTitle}
Temporal uses of the pronominally used demonstratives
\end{styleExampleTitle}

\ea
\label{Example_7.9}
\gll {mandi} {cepat{\Tilde}cepat,} {\bluebold{ini}} {tong} {mo} {lanjut} {lagi}\\ %
 bathe  \textsc{rdp}{\Tilde}be.fast  \textsc{d.prox}  \textsc{1pl}  want  continue  again\\
\glt 
‘bathe very quickly, \bluebold{right now} we want to continue further’ \textstyleExampleSource{[080917-008-NP.0134]}
\z

\ea
\label{Example_7.10}
\gll {satu} {kali} {tong} {pergi} {berdoa} {\ldots} {\bluebold{itu}} {de} {ikut}\\ %
 one  time  \textsc{1pl}  go  pray { }   \textsc{d.dist}  \textsc{3sg}  follow\\
\glt 
‘one time we went to pray \ldots, \bluebold{at that time} she (my daughter) also followed (us)’ \textstyleExampleSource{[080917-008-NP.0175]}
\z



The temporal uses of the adverbially used demonstratives are illustrated in (\ref{Example_7.11}) and (\ref{Example_7.12}). Again, \textitbf{ini} ‘\textsc{d.prox}’ signals temporal proximity as in \textitbf{ada datang ini} ‘is coming right now’ in (\ref{Example_7.11}), while \textitbf{itu} ‘\textsc{d.dist}’ indicates temporal distance as in \textitbf{bangung itu} ‘woke up at that time’ in (\ref{Example_7.12}).


\begin{styleExampleTitle}
Temporal uses of the adverbially used demonstratives
\end{styleExampleTitle}

\ea
\label{Example_7.11}
\gll {\ldots} {o,} {betul,} {Papua-Satu} {ini} {\bluebold{ada}} {\bluebold{datang}} {\bluebold{ini}}\\ %
 { }   oh!  be.true  Papua-Satu  \textsc{d.prox}  exist  come  \textsc{d.prox}\\
\glt 
‘[and then we saw,] ``oh!, (it’s) true, this Papua-Satu (ship) is \bluebold{coming right now}''' (Lit. ‘\bluebold{this coming}’) \textstyleExampleSource{[080917-008-NP.0130]}
\z

\ea
\label{Example_7.12}
\gll {sa} {bawa} {{pulang}} {\ldots} {{mace}} {{\bluebold{bangung}}} {\bluebold{itu}} {dia} {suda}\\ %
 \textsc{1sg}  bring  {go.home}  { }  {woman}  {wake.up}  \textsc{d.dist}  \textsc{3sg}  already\\
\gll {snang}  {karna}  liat  {ada}  {makangang}\\
 {feel.happy(.about)}  {because}  see  {exist}  {food}\\
\glt
‘I brought home (the game that I had shot) {\ldots} (when my) wife \bluebold{got up at that time}, already she was glad because (she) saw there was food’ (Lit. ‘\bluebold{that waking up}’) \textstyleExampleSource{[080919-004-NP.0030/0032]}
\z


\subsubsection[Psychological uses of demonstratives]{Psychological uses of demonstratives}
\label{Para_7.1.2.3}
In their psychological uses, the demonstratives signal the speakers’ psychological involvement with the situation or event talked about \citep[347]{Lakoff.1974}. Three major domains of \isi{psychological use} are attested: \isi{emotional involvement}, \isi{vividness}, and contrast.


\subsubsubsection{\textit{Demonstratives signaling emotional involvement}\label{Para_7.1.2.3.1}}

Speakers employ the de\-mon\-stratives to signal their \isi{emotional involvement}, close association, and/or attitudes concerning the subject matter. Proximal \textitbf{ini} ‘\textsc{d.prox}’ indicates emotional proximity or positive attitudes, while \textitbf{itu} ‘\textsc{d.dist}’ signals emotional distance or negative attitudes, as illustrated in three sets of examples.



In the first set of examples in (\ref{Example_7.13}) and (\ref{Example_7.14}), the demonstratives modify the personal \isi{pronoun} \textitbf{ko} ‘\textsc{2sg}’. In (\ref{Example_7.13}), a mother scolds her daughter for having ripped off the blossoms of the garden’s flowers. In shouting at her youngest, she uses \textitbf{ini} ‘\textsc{d.prox}’, thereby signaling her nevertheless close \isi{emotional involvement} with her daughter. By contrast, in (\ref{Example_7.14}) a teacher is exasperated with one of his students who does not know the English word ``please''. In voicing his frustration, the speaker uses \textitbf{itu} ‘\textsc{d.dist}’ and thereby signals his momentary emotional distance from the referent.


\begin{styleExampleTitle}
Demonstratives signaling \isi{emotional involvement}: Examples set \#1
\end{styleExampleTitle}

\ea
\label{Example_7.13}
\gll {ko} {liat} {{Luisa}} {{pu}} {{bagus,}} {suda} {kembang} {banyak,}\\ %
 \textsc{2sg}  see  {Luisa}  {\textsc{poss}}  {be.good}  already  f\isi{lowering}  many\\
\gll  \bluebold{ko}  {\bluebold{ini},}  {bunga}  {tida}  {slamat}\\
 \textsc{2sg}  {\textsc{d.prox}}  {flower}  {\textsc{neg}}  {safe}\\
\glt 
[After the speaker’s daughter had ripped off blossoms:] ‘you see Luisa’s (flowers) are good, (they are) already f\isi{lowering} a lot, \bluebold{you (}\blueboldSmallCaps{emph}\bluebold{)}, the flowers (you picked) can’t be saved’ \textstyleExampleSource{[081006-021-CvHt.0002]}
\z

\ea
\label{Example_7.14}
\gll {Dodo} {{kipas}} {{de}} {{suda}} {mo,} {{\bluebold{ko}}} {{\bluebold{tu},}} {{ora}} {orang}\\ %
 Dodo  {beat}  {\textsc{3sg}}  {already}  want  {\textsc{2sg}}  {\textsc{d.dist}}  {\textsc{tru}{}-person}  person\\
\gll {bilang}  {please,}  {kata}  {pis}  {saja}  tida  {taw,}  {goblok}\\
 {say}  {please[E]}  {word}  {please[E]}  {just}  \textsc{neg}  {know}  {be.stupid}\\
\glt 
‘Dodo reprimanded her immediately, ``\bluebold{you there}, people[\textsc{tru}], people say ``please'', don’t (you) know the word ``please''?!, (you’re) stupid!''' \textstyleExampleSource{[081115-001a-Cv.0140]}
\z



In the second set of examples in (\ref{Example_7.15}) and (\ref{Example_7.16}), each \isi{demonstrative} modifies the common \isi{noun} \textitbf{ana} ‘child’. The utterance in (\ref{Example_7.15}) is part of a story about a motorbike accident that the speaker had in a remote area of the regency. The speaker relates how her nephew came and picked her up and took her all the way to the next hospital in the regency capital. In choosing \textitbf{ini} ‘\textsc{d.prox}’, the speaker signals her emotional closeness to the referent, who was not present when the speaker related her story; in fact, at that time the nephew was living about 300 km away. The utterance (\ref{Example_7.16}) occurred during a conversation about the speaker’s youngest brother. Her interlocutors relate several complaints about the referent, who was present during the conversation. Finally, the speaker joins her interlocutors and comments that \textitbf{ana kecil itu} ‘that small child’ constantly changes his opinion. By employing \textitbf{itu} ‘\textsc{d.dist}’, the speaker, who often criticizes her brother publicly, signals that she wishes to dissociate herself from her brother.

\begin{styleExampleTitle}
Demonstratives signaling \isi{emotional involvement}: Examples set \#2
\end{styleExampleTitle}

\ea
\label{Example_7.15}
\gll {baru} {\bluebold{sa}} {\bluebold{punya}} {\bluebold{ana}} {\bluebold{ini}} {mantri} {de} {pi} {ambil} {saya}\\ %
 and.then  \textsc{1sg}  \textsc{poss}  child  \textsc{d.prox}  male.nurse  \textsc{3sg}  go  fetch  \textsc{1sg}\\
\glt 
‘and then \bluebold{this child of mine}, the male nurse came to get me’ \textstyleExampleSource{[081015-005-NP.0044]}
\z

\ea
\label{Example_7.16}
\gll {putar} {putar} {\bluebold{ana}} {\bluebold{kecil}} {\bluebold{itu}}\\ %
 turn.around  turn.around  child  be.small  \textsc{d.dist}\\
\glt 
‘(he’s) constantly changing (his opinion), \bluebold{that small child}’ \textstyleExampleSource{[081011-003-Cv.0016]}
\z



In the third set of examples in (\ref{Example_7.17}) and (\ref{Example_7.19}), each \isi{demonstrative} modifies an \isi{adnominal possessive construction} with the possessum \textitbf{swara} ‘voice’. The utterance in (\ref{Example_7.17}) is part of a conversation about the young people living in the house, none of whom is present at this conversation. The speaker relates that the teenagers enjoy singing. Using \isi{direct speech}, the speaker conveys her positive attitudes about the teenagers’ singing: they should sing more in public because \textitbf{kamu pu swara ini} ‘these voices of yours’ are good. The utterance in (\ref{Example_7.18}) occurred during a conversation outside at night. When one of the teenagers laughs out loudly, the others reprimand her. Employing \textitbf{itu} ‘\textsc{d.dist}’ in \textitbf{ko pu swara tu} ‘that voice of yours’, the speaker conveys her negative attitudes about this behavior.


\begin{styleExampleTitle}
Demonstratives signaling \isi{emotional involvement}: Examples set \#3
\end{styleExampleTitle}

\ea
\label{Example_7.17}
\gll {\ldots} {dang} {menyanyi} {\ldots} {\bluebold{kamu}} {\bluebold{pu}} {\bluebold{swara}} {\bluebold{ini}} {bagus}\\ %
  { } and  sing  { } \textsc{2pl}  \textsc{poss}  voice  \textsc{d.prox}  be.good\\
\glt 
‘[come in front] and sing {\ldots} \bluebold{these voices of yours} are good’ \textstyleExampleSource{[081014-015-Cv.0026/0028]}\\
\z

\ea
\label{Example_7.18}
\gll {\bluebold{ko}} {\bluebold{pu}} {\bluebold{swara}} {\bluebold{tu}} {bahaya,} {ko} {stop}\\ %
 \textsc{2sg}  \textsc{poss}  voice  \textsc{d.dist}  be.dangerous  \textsc{2sg}  stop\\
\glt
‘\bluebold{that voice of yours} is dangerous, stop (it)!’ \textstyleExampleSource{[081025-009b-Cv.0016]}\\
\z


\subsubsubsection{\textit{Demonstratives signaling vividness}\label{Para_7.1.2.3.2}}

The \isi{emotional involvement} does not need to be as substantial as described in ``Demonstratives signaling \isi{emotional involvement}'' (§\ref{Para_7.1.2.3.1}). The demonstratives are also used in more general terms to indicate that the subject matter is vivid “to the mind of the speaker”, adopting \citegen[278]{Anderson.1985} terminology. To signal that an event or situation is of special interest to them, the speakers use the demonstratives adnominally or adverbially, or employ \isi{demonstrative} stacking. This section discusses both of these strategies.



The first strategy to signal \isi{vividness} is to employ the demonstratives adnominally or adverbially to modify and thereby intensify nominal and pronominal constituents as in (\ref{Example_7.19}) and (\ref{Example_7.20}), or verbs as in (\ref{Example_7.21}) and (\ref{Example_7.22}).



In the first set of examples in (\ref{Example_7.19}) and (\ref{Example_7.20}), the short \isi{demonstrative} forms modify nominal and pronominal constituents.



The utterance in (\ref{Example_7.19}) occurred after the speaker had been provoked verbally by an older relative. In her reaction, the speaker modifies the constituents \textitbf{bapa-tua} ‘uncle’ and \textitbf{emosi} ‘feel angry (about)’ with \textitbf{ini} ‘\textsc{d.prox}’, thereby emphasizing them. In choosing the proximal rather than the distal \isi{demonstrative} to modify the constituent \textitbf{bapa-tua} ‘uncle’, the speaker also signals that the referent is still nearby.



The utterance in (\ref{Example_7.20}) is part of a conversation about the work stamina of a wife from the Pante-Barat area. When she and her husband lived in a different area, the women from that area were surprised how hard the Pante-Barat woman worked. The utterance in (\ref{Example_7.20}) relates the husband’s response to these women. Having referred to his wife twice with the personal \isi{pronoun} \textitbf{de} ‘\textsc{3sg}’, the speaker refers to her again. This time he modifies the personal \isi{pronoun} with short \textitbf{itu} ‘\textsc{d.dist}’, thereby emphasizing it. This example again illustrates the at times overlapping functions of the demonstratives. In addition to signaling \isi{vividness}, the distal \isi{demonstrative} also signals that the referent was not present in the speech situation.


\begin{styleExampleTitle}
Adnominal uses to signal \isi{vividness}
\end{styleExampleTitle}

\ea
\label{Example_7.19}
\gll {sa} {bilang,} {{adu,}} {\bluebold{bapa-tua}} {\bluebold{ni}} {mancing}\\ %
 \textsc{1sg}  say  {oh.no!}  uncle  \textsc{d.prox}  fish.with.rod\\
\gll {\bluebold{emosi}}  {\bluebold{ni}}\\
 {feel.angry(.about)}  {\textsc{d.prox}}\\
\glt 
[After having been provoked:] ‘I said, ``oh no, \bluebold{uncle (}\blueboldSmallCaps{emph}\bluebold{)} is provoking \bluebold{(our) emotions (}\blueboldSmallCaps{emph}\bluebold{)}''' \textstyleExampleSource{[081025-008-Cv.0124]}
\z

\ea
\label{Example_7.20}
\gll {\ldots} {{de}} {{bilang,}} {\ldots} {de} {{suda}} {biasa} {de} {bisa} {kerja,}\\ %
  {}  {\textsc{3sg}} {say}  { }   \textsc{3sg}  {already}  be.usual  \textsc{3sg}  be.able  work\\
\gll {\bluebold{de}}  {\bluebold{tu}}  {kerja}  {kaya}  {laki{\Tilde}laki}\\
 {\textsc{3sg}}  {\textsc{d.dist}}  {work}  {like}  {\textsc{rdp}{\Tilde}husband}\\
\glt 
‘[and then my husband told (them),] he said, ``{\ldots} she’s already used (to working like this), she can work (hard), \bluebold{she (}\blueboldSmallCaps{emph}\bluebold{)} works like a man''' \textstyleExampleSource{[081014-007-CvEx.0049-0050]}
\z


In the second set of examples in (\ref{Example_7.22}) and (\ref{Example_7.22}), the demonstratives are used adverbially to signal \isi{vividness}. In (\ref{Example_7.21}), \textitbf{ini} ‘\textsc{d.prox}’ modifies the \isi{verb} \textitbf{hidup} ‘live’, resulting in the emphatic reading \textitbf{hidup ini} ‘to be very much alive’. Along similar lines, in (\ref{Example_7.22}), \textitbf{itu} ‘\textsc{d.dist}’ modifies the \isi{verb} \textitbf{lompat} ‘jump’, giving the emphatic reading \textitbf{lompat itu} ‘really jumped’. Again, these examples illustrate the overlapping functions of the demonstratives: while indicating \isi{vividness}, they also have temporal uses. They signal temporal proximity indicating present tense in (\ref{Example_7.21}), and temporal distance indicating past tense in (\ref{Example_7.22}).


\begin{styleExampleTitle}
Adverbial uses to signal \isi{vividness}
\end{styleExampleTitle}

\ea
\label{Example_7.21}
\gll {{wa,}} {sa} {{masi}} {\bluebold{hidup}} {\bluebold{ni},} {kam} {suda} {hinggap} {di}\\ %
 {wow!}  \textsc{1sg}  {still}  live  \textsc{d.prox}  \textsc{2pl}  already  perch  at\\
\gll  sa  {punya}  {badang}\\
\textsc{1sg}  {\textsc{poss}}  {body}\\
\glt 
[After having been pestered by flies:] ‘wow!, I’m still \bluebold{very much alive}, you (blue flies) had already perched upon my body’ \textstyleExampleSource{[080919-005-Cv.0015]}
\z

\ea
\label{Example_7.22}
\gll {sunggu}, {sa} {\bluebold{lompat}} {\bluebold{itu}} {dengang} {tenaga}\\ %
 be.true  \textsc{1sg}  jump  \textsc{d.dist}  with  energy\\
\glt 
‘truly, I \bluebold{really jumped} with energy’ \textstyleExampleSource{[081025-006-Cv.0218]}
\z


The second, although less common, strategy to signal \isi{vividness} is the stacking of demonstratives. In the corpus the first \isi{demonstrative} is always a long one, while the second is always the corresponding short one. In these constructions, the first \isi{demonstrative} may be used adnominally as in (\ref{Example_7.23}) and (\ref{Example_7.24}), or pronominally as in (\ref{Example_7.25}) and (\ref{Example_7.26}). In each case, the result of the stacking is an emphatic reading of the entire \isi{noun} phrase.



In (\ref{Example_7.23}) and (\ref{Example_7.24}) the second \isi{demonstrative} modifies a nested \isi{noun} phrase with an adnominal \isi{demonstrative} such that ``[[\textsc{n} \textsc{dem}] \textsc{dem}]''. The result of the stacking is an emphatic reading in the sense of ``this/that very \textsc{n}'': \textitbf{orang ini ni} ‘this very person’ in (\ref{Example_7.23}) and \textitbf{ruma itu tu} ‘that very house’ in (\ref{Example_7.24}).


\begin{styleExampleTitle}
Adnominal uses of stacked demonstratives to signal \isi{vividness}
\end{styleExampleTitle}
\ea
\label{Example_7.23}
\gll {[[\bluebold{orang}} {\bluebold{ini}]} {\bluebold{ni}]} {percaya} {sama} {Tuhang} {Yesus}\\ %
 person  \textsc{d.prox}  \textsc{d.prox}  trust  to  God  Jesus\\
\glt 
‘\bluebold{this very person }believes in God Jesus’ \textstyleExampleSource{[081006-022-CvEx.0177]}
\z

\ea
\label{Example_7.24}
\gll {waktu} {kitorang} {masuk} {di} {[[\bluebold{ruma}} {\bluebold{itu}]} {\bluebold{tu}]} {\ldots}\\ %
 when  \textsc{1pl}  enter  at  house  \textsc{d.dist}  \textsc{d.dist}  \\
\glt 
‘when we moved into \bluebold{that very house}, \ldots’ \textstyleExampleSource{[081006-022-CvEx.0167]}
\z



In (\ref{Example_7.25}) and (\ref{Example_7.26}) the second \isi{demonstrative} modifies a pronominally used first one. The result is an emphatic reading in the sense of ‘this/that very (one)’: \textitbf{ini ni} ‘these very (ones)’ in (\ref{Example_7.25}), and \textitbf{itu tu} ‘those very (ones)’ in (\ref{Example_7.26}).


\begin{styleExampleTitle}
Pronominal uses of stacked demonstratives to signal \isi{vividness}
\end{styleExampleTitle}

\ea
\label{Example_7.25}
\gll {ada} {segala} {macang} {tulang,} {dia} {buang} {[\bluebold{ini}} {\bluebold{ni}]}\\ %
 exist  all  variety  bone  \textsc{3sg}  throw(.away)  \textsc{d.prox}  \textsc{d.prox}\\
\glt 
‘there were all kinds of bones, he threw away \bluebold{these very (ones)}’ \textstyleExampleSource{[080922-010a-CvNF.0101]}
\z

\ea
\label{Example_7.26}
\gll {ko} {{taw}} {{kata}} {{pis}} {ka} {{tida,}} {[\bluebold{itu}} {\bluebold{tu}]} {kata{\Tilde}kata}\\ %
 \textsc{2sg}  {know}  {word}  {please[E]}  or  {\textsc{neg}}  \textsc{d.dist}  \textsc{d.dist}  \textsc{rdp}{\Tilde}word\\
\gll {dasar}  {yang}  {harusnya}  {kamu}  {taw}\\
 {base}  {\textsc{rel}}  {appropriately}  {\textsc{2pl}}  {know}\\
\glt
[Addressing a school student:] ‘do you know the (English) word ``please'' or not?, \bluebold{that very} (word belongs to) the basic words that you should know’ \textstyleExampleSource{[081115-001a-Cv.0145]}
\z


\subsubsubsection{\textit{Demonstratives signaling contrast between two entities}\label{Para_7.1.2.3.3}}

In their contrast\-ive uses, the demonstratives signal contrast between a discourse referent and another entity, thereby conveying the speakers’ attitudes about the subject matter. This contrastive use is illustrated with three sets of examples.



In the first set of examples in (\ref{Example_7.27}) and (\ref{Example_7.28}), the demonstratives modify the personal \isi{pronoun} \textitbf{saya} ‘\textsc{1sg}’, each time indicating an explicit contrast.



In (\ref{Example_7.27}), the speaker compares the ill-behaved young people living in the house to himself. While they have the privilege of staying with relatives in the regional city to complete their secondary schooling, he had to stay with strangers when he was young. This contrast is indicated with \textitbf{ini} ‘\textsc{d.prox}’.


\begin{styleExampleTitle}
Demonstratives signaling contrast: \textitbf{saya ini} ‘\textsc{1sg} \textsc{d.prox}’
\end{styleExampleTitle}

\ea
\label{Example_7.27}
\gll {kamu} {{ana{\Tilde}ana}} {skarang} {ini} {susa} {\ldots} {\bluebold{saya}} {\bluebold{ini}}\\ %
 \textsc{2pl}  {\textsc{rdp}{\Tilde}child}  now  \textsc{d.prox}  be.difficult   { }  \textsc{1sg}  \textsc{d.prox}\\
\gll {tinggal}  dengang  {orang}\\
 {stay}  with  {person}\\
\glt 
‘you, the young people, nowadays are difficult {\ldots} \bluebold{I, by contrast}, stayed with (other) people’ (Lit. ‘\bluebold{this I}’) \textstyleExampleSource{[081115-001b-Cv.0038/0040]}
\z



The exchange in (\ref{Example_7.28}) occurred during a phone conversation when a daughter asked her father to buy her a cell-phone. In (\ref{Example_7.28a}) her father suggests that a cell-phone would distract her from her studies. The daughter responds with the contrastive statement in (\ref{Example_7.28b}) in which \textitbf{itu} ‘\textsc{d.dist}’ modifies \textitbf{saya} ‘\textsc{1sg}’, resulting in the contrastive reading \textitbf{sa tu} ‘I, nevertheless’. The exact semantic distinctions between \textitbf{ini} ‘\textsc{d.prox}’ and \textitbf{itu} ‘\textsc{d.dist}’ need further investigation, though. The use of \textitbf{itu} ‘\textsc{d.dist}’ with \textitbf{saya} ‘\textsc{1sg}’ is especially surprising given that a first person singular \isi{pronoun} is inherently proximal. A temporal non-contemporaneous interpretation is not likely since the speaker talks about her behavior in general.


\begin{styleExampleTitle}
Demonstratives signaling contrast: \textitbf{saya itu} ‘\textsc{1sg} \textsc{d.dist}’
\end{styleExampleTitle}

\ea
\label{Example_7.28}

\ea
\label{Example_7.28a}
\gll  {Father:} {kalo} {bli} {{HP}} {di} {{situ}} {{nanti}} {{su}}\\ %
 { }     if  buy  {cell.phone}  at  {\textsc{l.med}}  {very.soon}  {already}\\
 \gll    tra  {bisa}  {skola,}  {maing}  {HP}  saja\\
     \textsc{neg}  {be.able}  {go.to.school}  {play}  {cell.phone}  just\\
\glt Father: ‘if (you) buy a cell-phone there then (you) won’t be able to do (any) schooling, (you’ll) just play (with your) cell-phone’\\
\vspace{5pt}
\ex
\label{Example_7.28b}
\gll  Daughter:  \bluebold{sa}  \bluebold{tu}  rajing  skola\\
 { }      \textsc{1sg}  \textsc{d.dist}  be.diligent  go.to.school\\
\glt Daughter: ‘\bluebold{I, nonetheless}, go to school diligently’ (Lit. ‘\bluebold{that I}’) \textstyleExampleSource{[080922-001a-CvPh.0448/0455]}
\z
\z



In the second set of examples in (\ref{Example_7.29}) and (\ref{Example_7.30}), the demonstratives modify the personal \isi{pronoun} \textitbf{ko} ‘\textsc{2sg}’: the contrast is implicit in (\ref{Example_7.29}), while it is explicit in (\ref{Example_7.30}).



The example in (\ref{Example_7.29}) is part of joke about a boy who chooses to attend a choir rather than a karate club together with his friends. The father is upset about his son’s choice. Finally, he vents his anger with a contrastive statement in which \textitbf{ini} ‘\textsc{d.prox}’ modifies \textitbf{ko} ‘\textsc{2sg}’. Thereby, the father contrasts his son implicitly with his friends: \textitbf{ko ni} ‘and what about you’.


\begin{styleExampleTitle}
Demonstratives signaling contrast: \textitbf{ko ini} ‘\textsc{2sg} \textsc{d.prox}’
\end{styleExampleTitle}

\ea
\label{Example_7.29}
\gll {\ldots} {{sampe}} {{dep}} {{bapa}} {{su}} {{mara,}} {\bluebold{ko}} {{\bluebold{ni}}}\\ %
 { }   {until}  {\textsc{3sg}:\textsc{poss}}  {father}  {already}  {feel.angry(.about)}  \textsc{2sg}  {\textsc{d.prox}}\\
\gll {setiap}  hari  {ko}  {ikut}  {latiang}  {paduang-swara}  {trus,}  kalo\\
 {every}  day  {\textsc{2sg}}  {follow}  {practice}  {choir}  {be.continuous}  if\\
\gll {dong}  {pukul}  {ko}  {ko}  {bisa}  {tangkis}  ka  {tida}\\
 {\textsc{3pl}}  {hit}  {\textsc{2sg}}  {\textsc{2sg}}  {be.able}  {ward.off}  or  {\textsc{neg}}\\
\glt 
‘[his father sees him (practicing in a choir) while his other friends practice self-defense] until his father gets angry (with his son), ``\bluebold{and what about you}, every day you attend the choir practice, (but) if someone hits you, can you defend (yourself) or not?''' (Lit. ‘\bluebold{this you}’) \textstyleExampleSource{[081109-006-JR.0001-0003]}
\z



In (\ref{Example_7.30}), an aunt gives advice to her niece who had been insulted by her younger cousin. Agreeing that the younger cousin has lighter skin and longer hair than the referent, the speaker continues her advice with a contrastive statement in which \textitbf{itu} ‘\textsc{d.dist}’ modifies \textitbf{ko} ‘\textsc{2sg}’: \textitbf{ko itu} ‘you, however’.


\begin{styleExampleTitle}
Demonstrative signaling contrast: \textitbf{ko itu} ‘\textsc{2sg} \textsc{d.dist}’
\end{styleExampleTitle}

\ea
\label{Example_7.30}
\gll {ade} {{tu}} {{biar}} {{puti,}} {rambut} {mayang} {tinggal} {rambut}\\ %
 ySb  {\textsc{d.dist}}  {let}  {be.white}  hair  palm.blossom  stay  hair\\
\gll {panjang,}  {\bluebold{ko}}  {\bluebold{itu}}  {jalang}\\
 {be.long}  {\textsc{2sg}}  {\textsc{d.dist}}  {walk}\\
\glt 
‘let that younger sister have light skin, (let her have) hair that’s long down to her bottom, \bluebold{you, however}, go (your own way)’ (Lit. ‘\bluebold{that you}’) \textstyleExampleSource{[081115-001a-Cv.0244]}
\z



In the third set in (\ref{Example_7.31}), the demonstratives modify temporal adverbs, thereby signaling temporal contrasts. In (\ref{Example_7.31}), a wife and her husband recount how a young man damaged his leg during a motorbike accident. In (\ref{Example_7.31a}) the wife relates that \textitbf{skarang} ‘now’ the referent walks crookedly. Her husband continues the narrative in (\ref{Example_7.31b}) with a contrastive statement in which \textitbf{itu} ‘\textsc{d.dist}’ modifies the temporal ad\isi{verb} \textitbf{dulu} ‘first, in the past’, thereby signaling a temporal contrast: \textitbf{dulu itu} ‘in the past, however’. Subsequently, the wife further elaborates on the referent’s \isi{condition}. She concludes the exchange with yet another contrastive statement in (\ref{Example_7.31c}) in which \textitbf{ini} ‘\textsc{d.prox}’ modifies the temporal ad\isi{verb} \textitbf{skarang} ‘now’, again signaling a temporal contrast: \textitbf{skarang ini} ‘(it’s) just now’.


\begin{styleExampleTitle}
Demonstrative signaling contrast: Modifying temporal adverbs
\end{styleExampleTitle}

\ea
\label{Example_7.31}
\ea
\label{Example_7.31a}
\gll  {Wife:} {skarang} {ada} {jalang} {bengkok} {sedikit}\\ %
 { }      now  exist  walk  be.crooked  few\\
\glt
Wife: ‘now he’s walking a little crookedly (because of his motorbike accident)’
\vspace{5pt}

\ex
\label{Example_7.31b}
\gll  Husband:  \bluebold{dulu}  \bluebold{itu}  de  jalang  lurus\\
 { }      first  \textsc{d.dist}  \textsc{3sg}  walk  be.straight\\
\glt
Husband: ‘\bluebold{in the past, however}, he walked straight’
\vspace{5pt}

\ex
\label{Example_7.31c}
\gll  Wife:  {\ldots}  {ini}  {bengkok}  {ini,}  {kaki}  ini\\
  { }   { }    {\textsc{d.prox}}  {be.crooked}  {\textsc{d.prox}}  {foot}  \textsc{d.prox}\\
\gll     {\bluebold{skarang}}  {\bluebold{ini}}  {baru}  {ada}  {baik{\Tilde}baik}\\
     {now}  {\textsc{d.prox}}  {recently}  {exist}  {\textsc{rdp}{\Tilde}be.good}\\
\glt
Wife: ‘this (foot) was crooked here, this foot, (\bluebold{it’s) just now} that (it got) well’ \textstyleExampleSource{[081006-020-Cv.0006-0007/0013]}
\z
\z


\subsubsection[Identificational uses of demonstratives]{Identificational uses of demonstratives}
\label{Para_7.1.2.4}
The demonstratives have identificational uses when they appear in the subject slot of a \isi{nominal predicate clause} (§\ref{Para_12.2}). In this context, the demonstratives aid in the identification of a \isi{definite} or identifiable referent encoded by the predicate. For instance, \textitbf{ini} ‘\textsc{d.prox}’ takes the subject slot in (\ref{Example_7.32}) and \textitbf{itu} ‘\textsc{d.dist}’ in (\ref{Example_7.33}). In this domain of use only the long demonstratives are attested.

\ea
\label{Example_7.32}
\gll {\bluebold{ini}} {daging} {yang} {saya} {bawa} {antar} {buat} {sodara} {dorang}\\ %
 \textsc{d.prox}  meat  \textsc{rel}  \textsc{1sg}  bring  deliver  for  sibling  \textsc{3pl}\\
\glt 
‘\bluebold{this} is the (wild pig) meat that I brought (and) delivered for (my) relatives’ \textstyleExampleSource{[080919-003-NP.0021]}
\z

\ea
\label{Example_7.33}
\gll {\bluebold{itu}} {kali} {Biri}\\ %
 \textsc{d.dist}  river  Biri\\
\glt
‘\bluebold{that} is the Biri river’ \textstyleExampleSource{[081025-008-Cv.0006]}
\z


\subsubsection[Textual uses of demonstratives]{Textual uses of demonstratives}
\label{Para_7.1.2.5}
In their textual uses, the demonstratives provide discourse orientation. Across languages, two major discourse uses of demonstratives can be distinguished, according to \citet[95–105]{Diessel.1999}: “anaphoric” and “discourse deictic” uses. In their anaphoric uses, the demonstratives “are coreferential with a prior \textsc{np}” and thereby “keep track of discourse participants” (\citeyear*[93]{Diessel.1999}). In their discourse deictic uses, the demonstratives are not coreferent with the referent of a previously established \isi{noun} phrase. Instead, they are coreferential with a preceding or following proposition. That is, they “establish an overt link between two propositions: the one in which they are embedded and the one to which they refer” (\citeyear*[101]{Diessel.1999}). \citep[See also][]{Himmelmann.1996}



Both discourse functions also apply to the Papuan Malay demonstratives.


\subsubsubsection{\textit{Anaphoric uses}\label{Para_7.1.2.5.1}}

The anaphoric uses of the Papuan Malay demonstratives are demonstrated in (\ref{Example_7.34}) to (\ref{Example_7.39}). Being coreferential with a preceding \textsc{np}, they keep track of different discourse participants. In this use the demonstratives may be employed adnominally as in (\ref{Example_7.34}) and (\ref{Example_7.35}), pronominally as in (\ref{Example_7.36}) and (\ref{Example_7.37}), or adverbially as in (\ref{Example_7.38}) and (\ref{Example_7.39}). The exact semantic distinctions between \textitbf{ini} ‘\textsc{d.prox}’ and \textitbf{itu} ‘\textsc{d.dist}’ as participant tracking devices, however, are yet to be investigated in more detail.



The examples in (\ref{Example_7.34}) and (\ref{Example_7.35}) demonstrate the adnominal anaphoric uses of the demonstratives. The utterance in (\ref{Example_7.34}) is part of a joke about a school student who does not know to draw. The teacher orders the students to \textitbf{gambar monyet} ‘draw a monkey’. When the \textitbf{monyet} ‘monkey’ is mentioned the next time, it is marked with \textitbf{ini} ‘\textsc{d.prox}’ thereby indicating coreference with this specific monkey. The example in (\ref{Example_7.35}) is part of a narrative that describes how the speaker’s ancestor first came down to the coast where he finds a \textitbf{bua mera} ‘red fruit’. At its next mention, the \isi{noun} phrase \textitbf{bua mera} ‘red fruit’ is marked with \textitbf{itu} ‘\textsc{d.dist}’ to signal coreference with that specific fruit.


\begin{styleExampleTitle}
Adnominal anaphoric uses
\end{styleExampleTitle}

\ea
\label{Example_7.34}
\gll {{ibu}} {{mulay}} {{suru}} {{ana{\Tilde}ana}} {{murit}} {{mulay}} {{gambar}} {\bluebold{monyet}} {di}\\ %
 {woman}  {start}  {order}  {\textsc{rdp}{\Tilde}child}  {pupil}  {start}  {draw}  monkey  at\\
\gll {atas}  {pohong}  {pisang}  {\ldots}  trus  {de}  {gambar}  {\bluebold{monyet}}  {\bluebold{ini}}\\
 {top}  {tree}  {banana}  { }  next  {\textsc{3sg}}  {draw}  {monkey}  {\textsc{d.prox}}\\
\gll di  {bawa}  {pohong}  {pisang}\\
 at  {bottom}  {tree}  {banana}\\
\glt 
‘Ms. (Teacher) starts ordering the school kids to start drawing \bluebold{a monkey} on a banana tree {\ldots} and then he draws \bluebold{this monkey} under the banana tree’ \textstyleExampleSource{[081109-002-JR.0001-0002]}
\z

\ea
\label{Example_7.35}
\gll {trus} {di} {{situ}} {\ldots} {ada} {{\bluebold{bua}}} {\bluebold{mera}} {\ldots} {de} {pegang}\\ %
 next  at  {\textsc{l.med}}  { }  exist  {fruit}  be.red  { }   \textsc{3sg}  hold\\
\gll \bluebold{bua}  {\bluebold{mera}}  {\bluebold{itu}}  dang  de  {jalang}\\
 fruit  {be.red}  {\textsc{d.dist}}  and  \textsc{3sg}  {walk}\\
\glt 
‘and then there {\ldots} was \bluebold{a red fruit} {\ldots} he took \bluebold{that red fruit} and he walked (further)’ \textstyleExampleSource{[080922-010a-CvNF.0218-219]}
\z



The examples in (\ref{Example_7.36}) and (\ref{Example_7.37}) illustrate the pronominal anaphoric uses of the demonstratives. The remark in (\ref{Example_7.36}) is part of a description of sagu production. After having introduced the main tool, \textitbf{penokok kayu} ‘wooden pounder’, the speaker replaces it at its next mention with \textitbf{ini} ‘\textsc{d.prox}’. In (\ref{Example_7.37}) the speaker talks about a female weight lifter. Noting that she is talking about weights in kilogram, she employs short \textitbf{itu} ‘\textsc{d.dist}’ which is coreferent with \textitbf{de pu brat} ‘her weights’.


\begin{styleExampleTitle}
Pronominal anaphoric uses
\end{styleExampleTitle}

\ea
\label{Example_7.36}
\gll {ada} {\bluebold{penokok}} {\bluebold{kayu}} {\ldots} {smua} {orang} {tokok} {dengang} {\bluebold{ini}}\\ %
 exist  pounder  wood   {}  all  person  tap  with  \textsc{d.prox}\\
\glt 
‘there is \bluebold{a wooden pounder} {\ldots} all people pound (sagu) with \bluebold{this}’ \textstyleExampleSource{[081014-006-CvPr.0011/0057]}
\z

\ea
\label{Example_7.37}
\gll {{prempuang}} {{Bandung}} {{itu}} {{\bluebold{de}}} {{\bluebold{pu}}} {{\bluebold{brat}}} {yang} {itu}\\ %
 {woman}  {Bandung}  {\textsc{d.dist}}  {\textsc{3sg}}  {\textsc{poss}}  {be.heavy}  \textsc{rel}  \textsc{d.dist}\\
\gll  sa  {angkat,}  {\bluebold{tu}}  {kilo}  {\ldots}  {dlapang}  pulu  {tiga}\\
 \textsc{1sg}  {lift}  {\textsc{d.dist}}  {kilogram}  { }  {eight}  tens  {three}\\
\glt 
‘that woman from Bandung, \bluebold{her weights} which I lifted, \bluebold{that} (is in) kilogram {\ldots} eighty three (kilogram)’ \textstyleExampleSource{[081023-003-Cv.0003]}
\z



The examples in (\ref{Example_7.38}) and (\ref{Example_7.39}) illustrate the adverbial anaphoric uses of both demonstratives. The utterance in (\ref{Example_7.38}) is part of a narrative about a youth retreat. During their journey to the retreat, the teenagers meet an old woman who gives them advice for the retreat. The woman mentions the \isi{verb} \textitbf{jalang} ‘walk’ three times while advising the teenagers where to walk and how to behave. When she mentions \textitbf{jalang} ‘walk’ again, she marks it with \textitbf{ini} ‘\textsc{d.prox}’.


\begin{styleExampleTitle}
Adverbial anaphoric uses of \textitbf{ini} ‘\textsc{d.prox}’
\end{styleExampleTitle}

\ea
\label{Example_7.38}
\gll {{kamu}} {{\bluebold{jalang},}} {{\bluebold{jalang}}} {{baik{\Tilde}baik}} {saja,} {{kamu}} {{tinggal,}} {{kamu}} {\bluebold{jalang},}\\ %
 {\textsc{2pl}}  {walk}  {walk}  {\textsc{rdp}{\Tilde}be.good}  just  {\textsc{2pl}}  {stay}  {\textsc{2pl}}  walk\\
\gll tida  {bole}  {ini}  ini  {\ldots}  {kamu}  {\bluebold{jalang}}  {\bluebold{ini}}  {untuk}\\
 \textsc{neg}  {may}  {\textsc{d.prox}}  \textsc{d.prox}  { }   {\textsc{2pl}}  {walk}  {\textsc{d.prox}}  {for}\\
\gll apa  {pekerjaang}  {Tuhang}\\
 what  {work}  {God}\\
\glt 
‘you \bluebold{travel}, (just) \bluebold{travel} well, (when) you stay (at Takar and when you) \bluebold{walk around} (in Takar), (you) shouldn’t (do) this (and) this, {\ldots} you (do) \bluebold{this traveling} for, what-is-it, God’s work’ \textstyleExampleSource{[081025-008-Cv.0142/0144]}
\z



The exchange (\ref{Example_7.39}) occurred between two sisters just before the youth retreat. In (\ref{Example_7.39a}) the younger one states that she wants to \textitbf{jalang} ‘travel’ to the youth retreat without, however, attending the services; instead she plans to stay at the guesthouse. Her older sister responds in (\ref{Example_7.39b}) that in that case it were better if she stayed home. Being upset about this reaction, the younger one asks her older sister in (\ref{Example_7.39c}) why she said so. In her reply in (\ref{Example_7.39d}), the older sister mentions \textitbf{jalang} ‘walk’ again, this time modifying it with \textitbf{itu} ‘\textsc{d.dist}’.


\begin{styleExampleTitle}
Adverbial anaphoric uses of \textitbf{itu} ‘\textsc{d.dist}’
\end{styleExampleTitle}

\ea
\label{Example_7.39}
\ea
\label{Example_7.39a}
\gll  {Younger sister:} {sa} {\bluebold{jalang}} {tra} {sembayang} {tinggal} {di} {ruma}\\ %
  { }    \textsc{1sg}  walk  \textsc{neg}  worship  stay  at  house\\
\glt
Young sister: ‘I’ll \bluebold{go} (to the youth retreat, but) I won’t worship, (I’ll) stay at the house’
\vspace{10pt}
\ex
\label{Example_7.39b}
\gll  {Older sister:}  kalo  {mo}  {tinggal}  di  ruma  tinggal  di\\
 { }      if  {want}  {stay}  at  house  stay  at\\
\gll  { }      {ruma}  {sini}  {\ldots}\\
 { } {house} \textsc{l.prox}\\
\glt Older sister: ‘if (you) want to stay at the house, stay home \ldots’
\vspace{10pt}
\ex
\label{Example_7.39c}
\gll  {Younger sister:}  knapa?\\
    { }   why\\
\glt
Young sister: ‘why?’
\vspace{10pt}
\ex
\label{Example_7.39d}
\gll  {Older sister:}  orang  \bluebold{jalang}  \bluebold{itu}  mo  pergi  sembayang\\
   { }  person  walk  \textsc{d.dist}  want  go  worship\\
\glt
Older sister: ‘people (doing) \bluebold{that traveling} want to go worship’ \textstyleExampleSource{[081006-016-Cv.0012-0015]}
\z
\z


Alternatively, however, one might argue that in (\ref{Example_7.38}) and (\ref{Example_7.39}) the demonstratives do not function as participant tracking devices, but rather signal \isi{emotional involvement}.


\subsubsubsection{\textit{Discourse deictic uses}\label{Para_7.1.2.5.2}}

The discourse deictic uses of the Papuan Malay de\-mon\-stratives are demonstrated in (\ref{Example_7.40}) to (\ref{Example_7.44}). In this use, they are coreferential with a preceding or following proposition. As shown in (\ref{Example_7.40}) to (\ref{Example_7.44}), though, only the pronominally used demonstratives have discourse deictic uses.



Proximal \textitbf{ini} ‘\textsc{d.prox}’ may refer to a preceding statement as in (\ref{Example_7.40}) or to a following statement as in (\ref{Example_7.41}). The example in (\ref{Example_7.40}) is part of a conversation about difficult children. Maintaining that children should be disciplined, the speaker makes a number of suggestions how to do so. Employing short \textitbf{ini} ‘\textsc{d.prox}’, the speaker summarizes her previous statements. Thereby she creates a link to her closing statement that her interlocutor should decide for herself what to make of these suggestions. In (\ref{Example_7.41}), \textitbf{ini} ‘\textsc{d.prox}’ creates a link to the following direct quote.


\begin{styleExampleTitle}
Discourse deictic uses of \textitbf{ini} ‘\textsc{d.prox}’
\end{styleExampleTitle}

\ea
\label{Example_7.40}
\gll {\ldots} {\bluebold{ni}} {usul} {saja} {jadi} {kaka} {sendiri} {\ldots}\\ %
  { }   \textsc{d.prox}  proposal  just  so  oSb  be.alone  \\
\glt ‘\bluebold{this} is just a proposal, so you (‘older sister’) (have to decide for) yourself \ldots’ \textstyleExampleSource{[080917-010-CvEx.0116]}
\z

\ea
\label{Example_7.41}
\gll {pace} {de} {bilang} {\bluebold{ini},} {mace} {ko} {sendiri} {yang} {ikut} {\ldots}\\ %
 man  \textsc{3sg}  say  \textsc{d.prox}  wife  \textsc{2sg}  be.alone  \textsc{rel}  follow  \\
\glt 
‘(my) husband said \bluebold{this}, ``you wife yourself (should) go (with them) \ldots''' (Lit. ‘(it’s) you wife yourself who \ldots’) \textstyleExampleSource{[081025-009a-Cv.0032]}
\z



Distal \textitbf{itu} ‘\textsc{d.dist}’ is used only to create a link to a preceding statement, as in (\ref{Example_7.42}). This example is part of joke about an uneducated person who notes that \textitbf{di kalender dua blas} ‘in the calendar are twelve (moons)’ while \textitbf{di langit ini cuma satu} ‘in this sky is only one’. Distal \textitbf{itu} ‘\textsc{d.dist}’ summarizes these remarks, creating an overt link to the speaker’s conclusion that this state of affairs is \textitbf{tipu skali} ‘very deceptive’.


\begin{styleExampleTitle}
Discourse deictic uses of \textitbf{itu} ‘\textsc{d.dist}’
\end{styleExampleTitle}

\ea
\label{Example_7.42}
\gll {{masa}} {{di}} {{kalender}} {{dua}} {{blas,}} {baru} {di} {langit}\\ %
 {be.impossible}  {at}  {calendar}  {two}  {teens}  and.then  at  sky\\
\gll ini  {cuma}  {satu}  {\ldots,}  {\bluebold{itu}}  {tipu}  {skali}\\
 \textsc{d.prox}  {just}  {one}  { }   {\textsc{d.dist}}  {cheat}  {very}\\
\glt [Joke:] ‘(it’s) impossible, in a calendar are twelve (moons), but in the sky here is only one (moon) {\ldots} \bluebold{that’s} very deceptive’ \textstyleExampleSource{[081109-007-JR.0003]}
\z



The discourse deictic uses of \textitbf{itu} ‘\textsc{d.dist}’ are very commonly extended to that of a “sentence connective” that signals “a \isi{causal} link between two propositions”, employing \citegen[125]{Diessel.1999} terminology. This is illustrated in (\ref{Example_7.43}) and (\ref{Example_7.44}). Standing alone, \textitbf{itu} ‘\textsc{d.dist}’ introduces a reason relation as in (\ref{Example_7.43}). When co-occurring with the \isi{relativizer} \textitbf{yang} ‘\textsc{rel}’, \textitbf{itu} ‘\textsc{d.dist}’ marks a result relation as in (\ref{Example_7.44}).



In (\ref{Example_7.40}), the speaker recounts a conversation with a local doctor after a motorbike accident. In using \textitbf{itu} ‘\textsc{d.dist}’ the doctor summarizes the speaker’s comments concerning her health and creates an overt link to his explanation why she is in pain. In this context \textitbf{itu} ‘\textsc{d.dist}’ functions as a \isi{causal} link that marks a reason relation.


\begin{styleExampleTitle}
Discourse deictic uses of \textitbf{itu} ‘\textsc{d.dist}’: Marker of a reason relation
\end{styleExampleTitle}

\ea
\label{Example_7.43}
\gll {sa} {{bilang,}} {{tulang}} {{bahu}} {{yang}} {pata,} {tulang} {rusuk,} {o,} {a,}\\ %
 \textsc{1sg}  {say}  {bone}  {shoulder}  {\textsc{rel}}  break  bone  rib  oh!  ah!\\
\gll {mama}  {\bluebold{itu}}  {hanya}  ko  {jatu}  {kaget}\\
 {mother}  {\textsc{d.dist}}  {only}  \textsc{2sg}  {fall}  {feel.startled(.by)}\\
\glt 
‘I said, ``(it’s my) shoulder bone that is broken, (my) ribs'', (the doctor said,) ``oh!, ah, Mother \bluebold{that is} just \bluebold{because} you’re in shock''' \textstyleExampleSource{[081015-005-NP.0048]}
\z



The utterance in (\ref{Example_7.44}) is part of a conversation about the speaker’s husband who had fallen sick after a straining journey. Recounting some details about the journey, the speaker relates that her husband had returned home hungry. At the beginning of the next clause \textitbf{itu} ‘\textsc{d.dist}’ summarizes this account and, combined with the \isi{relativizer} \textitbf{yang} ‘\textsc{rel}’, signals a result relation: \textitbf{itu yang de sakit} ‘that’s why he’s sick’.


\begin{styleExampleTitle}
Discourse deictic uses of \textitbf{itu} ‘\textsc{d.dist}’: Marker of a result relation
\end{styleExampleTitle}

\ea
\label{Example_7.44}
\gll {pace} {de} {tida} {makang} {\ldots} {lapar,} {\bluebold{itu}} {\bluebold{yang}} {de} {sakit}\\ %
 man  \textsc{3sg}  \textsc{neg}  eat  { }   be.hungry  \textsc{d.dist}  \textsc{rel}  \textsc{3sg}  be.sick\\
\glt
‘he (my) husband hadn’t eaten {\ldots} (he was) hungry, \bluebold{that’s why} he’s sick’ \textstyleExampleSource{[080921-004b-CvNP.0003/0007]}
\z


\subsubsection[Placeholder uses of demonstratives]{Placeholder uses of demonstratives}
\label{Para_7.1.2.6}
Demonstratives are also rather commonly employed pronominally as “placeholders” in the context of “word-formulation trouble”, as \citet{Hayashi.2006} show in their cross-linguistic study. In this function, they serve “as temporary substitutes for specific lexical items that have eluded the speaker” (\citeyear*[499]{Hayashi.2006}). Besides, demonstratives are also used as “interjective hesitators” that signal “the speaker’s hesitation in utterance production” (\citeyear*[512–513]{Hayashi.2006}). \citep[See also][36.]{Dooley.2001}



The \isi{placeholder} and hesitator uses also apply to the Papuan Malay demonstratives, as shown in (\ref{Example_7.45}) to (\ref{Example_7.47}) and in (\ref{Example_7.48}), respectively. In this function, however, only the long \isi{demonstrative} forms are attested, while the short forms are unattested.



As placeholders, the demonstratives can substitute for any lexical item, such as nouns as in (\ref{Example_7.45}), personal pronouns as in (\ref{Example_7.46}), or verbs as in (\ref{Example_7.47}). More investigation is needed, though, to account for the alternation of \textitbf{ini} ‘\textsc{d.prox}’ and \textitbf{itu} ‘\textsc{d.dist}’ in this context. In most cases, as in (\ref{Example_7.46}) and (\ref{Example_7.47}), the \isi{demonstrative} is set off from the subsequently produced target word by a \isi{comma intonation} (``{\textbar}''). Often, however, there is no audible pause between the \isi{placeholder} and the target word as in (\ref{Example_7.45}).



\begin{styleExampleTitle}
Placeholder for a \isi{proper noun}
\end{styleExampleTitle}

\ea
\label{Example_7.45}
\gll {\ldots} {saya} {ingat} {\bluebold{ini}} {\bluebold{Ise}}\\ %
   { }  \textsc{1sg}  remember  \textsc{d.prox}  Ise\\
\glt 
‘(at that particular time) I remembered, \bluebold{what’s-her-name, Ise}’ \textstyleExampleSource{[080917-008-NP.0102]}
\z

\begin{styleExampleTitle}
Placeholder for a personal \isi{pronoun}
\end{styleExampleTitle}

\ea
\label{Example_7.46}
\gll {wa,} {\bluebold{ini}} {} \textup{\textbar} {} {\bluebold{kitong}} {lari{\Tilde}lari} {kemaring} {sampe} {\ldots}\\ %
 wow  \textsc{d.prox} {}  {}  {}  \textsc{1pl}  \textsc{rdp}{\Tilde}run  yesterday  reach  \\
\glt 
‘wow!, \bluebold{what’s-their-name}, \bluebold{we} drove yesterday all the way to \ldots’ \textstyleExampleSource{[081006-033-Cv.0007]}
\z

\begin{styleExampleTitle}
Placeholder for a \isi{verb}
\end{styleExampleTitle}

\ea
\label{Example_7.47}
\gll {skarang} {sa} {\bluebold{itu}} {} \textup{\textbar} {} {\bluebold{simpang}} {sratus} {ribu}\\ %
 now  \textsc{1sg}  \textsc{d.dist} {} {} {}  store  one.hundred  thousand\\
\glt 
‘now I (already), \bluebold{what’s-its-name}, \bluebold{set aside} one hundred thousand (rupiah)’ \textstyleExampleSource{[081110-002-Cv.0039]}
\z



While in (\ref{Example_7.45}) to (\ref{Example_7.47}), the demonstratives are used referentially to substitute for a lexical item, this is not the case in (\ref{Example_7.48}). In this example, \textitbf{itu} ‘\textsc{d.dist}’ is used as a nonreferential interjective hesitator. This is evidenced by the fact that \textitbf{itu} ‘\textsc{d.dist}’ does not agree with adnominally used \textitbf{ini} ‘\textsc{d.prox}’, which modifies the head nominal \textitbf{pace} ‘man’. Overall, however, the hesitator uses of the demonstratives are rare; most commonly, Papuan Malay speakers use the hesitator \textitbf{e(m)} ‘uh’ (see §\ref{Para_5.13.2}).



\begin{styleExampleTitle}
Interjective hesitator
\end{styleExampleTitle}

\ea
\label{Example_7.48}
\gll {yo} {\bluebold{itu}} {\bluebold{itu}} {\bluebold{pace}} {\bluebold{ini}} {de} {baru} {ambil} {\ldots}\\ %
 oh!  \textsc{d.dist}  \textsc{d.dist}  man  \textsc{d.prox}  \textsc{3sg}  recently  fetch  \\
\glt 
‘oh, \bluebold{umh}, \bluebold{umh}, \bluebold{this man}, he recently took \ldots’ \textstyleExampleSource{[081011-009-Cv.0007]}
\z



Further investigation is required, to explore whether and in which ways Papuan Malay makes a distinction between the \isi{placeholder} and nonreferential hesitator uses of its demonstratives and whether it may in fact be using the right-displacement attested in (\ref{Example_7.48}) as a deliberate construction for emphasis in some contexts.


\section{Locatives}
\label{Para_7.2}
In the following sections, the syntactic properties and forms of the Papuan Malay locatives are reviewed and discussed (§\ref{Para_7.2.1}), followed by an in-depth discussion of their different functions and domains of use (§\ref{Para_7.2.2}).


\subsection{Syntax and forms of locatives}
\label{Para_7.2.1}
The distributional properties of the locatives are briefly reviewed in §\ref{Para_7.2.1.1}. This review is followed in §\ref{Para_7.2.1.2} by a discussion of the distribution and frequencies of the pronominally versus the adnominally used locatives.


\subsubsection[Distributional properties of locatives]{Distributional properties of locatives}
\label{Para_7.2.1.1}
The Papuan Malay locatives have the following distributional properties (for more details see §\ref{Para_5.7}):



\begin{enumerate}
\item 
Substitution for \isi{noun} phrases that occur in prepositional phrases (pronominal uses) (§\ref{Para_5.7.1})
\item 
Modification with demonstratives or relative clauses (pronominal uses) (§\ref{Para_5.7.1})
\item 
Co-occurrence with \isi{noun} phrases(adnominal uses): \textsc{n}/\textsc{np} \textsc{loc} (§\ref{Para_5.7.2})

\end{enumerate}

\subsubsection[Distribution of the pronominally versus the adnominally used locatives]{Distribution of the pronominally versus the adnominally used locatives}
\label{Para_7.2.1.2}
This section describes the distribution and frequencies of the pronominally versus the adnominally used locatives (their semantic distinctions are discussed in §\ref{Para_7.2.2}).



The corpus includes a total of 1,366 \isi{locative} tokens: 494 \textitbf{sini} ‘\textsc{l.prox}’ (36\%), 411 \textitbf{situ} ‘\textsc{l.med}’ (30\%), and 461 \textitbf{sana} ‘\textsc{l.dist}’ (34\%) tokens. Most commonly the locatives are employed pronominally (1,106/1,366 tokens – 91\%), while their adnominal uses are considerably less common (260/1,366 tokens – 19\%), as shown in \tabref{Table_7.7}.



\begin{table}
\caption{Locatives according to their syntactic functions}\label{Table_7.7}

\begin{tabular}{l*{6}{r}}
\lsptoprule
 & \multicolumn{2}{c}{ \textitbf{sini} ‘\textsc{l.prox}’} & \multicolumn{2}{c}{ \textitbf{situ} ‘\textsc{l.med}’} & \multicolumn{2}{c}{ \textitbf{sana} ‘\textsc{l.dist}}\\
\midrule
Pronominal uses &  416 &  84\% &  345 &  84\% &  345 &  75\%\\
Adnominal uses &  78 &  16\% &  66 &  16\% &  116 &  25\%\\
\midrule
Total &  494 &  100\% &  411 &  100\% &  461 &  100\%\\
\lspbottomrule
\end{tabular}
\end{table}

The distribution of the pronominally used locatives is presented in \tabref{Table_7.8}. In their pronominal uses, as mentioned in §\ref{Para_5.7.1}, the locatives always occur in prepositional phrases. Most often they are introduced with an overt \isi{preposition}: 384/416 \textitbf{sini} ‘\textsc{l.prox}’ (92\%), 302/345 \textitbf{situ} ‘\textsc{l.med}’ (87\%), and 311/345 \textitbf{sana} ‘\textsc{l.dist}’ (90\%) tokens. When the context allows the disambiguation of the semantic role or relationship of the \isi{locative}, however, the \isi{preposition} can also be deleted (the \isi{elision} of prepositions is discussed in §\ref{Para_10.1.5}): 32/416 \textitbf{sini} ‘\textsc{l.prox}’ (8\%), 45/345 \textitbf{situ} ‘\textsc{l.med}’ (13\%), and 34/345 \textitbf{sana} ‘\textsc{l.dist}’ (10\%) tokens. Overall, however, pronominally used locatives with zero-\isi{preposition} are rather rare (109/1,106 tokens – 10\%), as shown in \tabref{Table_7.8}.



\begin{table}
\caption{Pronominally used locatives in prepositional phrases (\textsc{pp}) with or without \isi{preposition} (\textsc{prep})}\label{Table_7.8}

\begin{tabular}{l*{6}{r}}
\lsptoprule
 & \multicolumn{2}{l}{ \textitbf{sini} ‘\textsc{l.prox}’} & \multicolumn{2}{l}{ \textitbf{situ} ‘\textsc{l.med}’} & \multicolumn{2}{l}{ \textitbf{sana} ‘\textsc{l.dist}}\\
\midrule
\textsc{pp} with \textsc{prep} &  384 &  92\% &  302 &  88\% &  311 &  90\%\\
\textsc{pp} with zero \textsc{prep} &  32 &  8\% &  43 &  12\% &  34 &  10\%\\
\midrule
Total &  416 &  100\% &  345 &  100\% &  345 &  100\%\\
\lspbottomrule
\end{tabular}
\end{table}

The distribution of the adnominally used locatives is presented in \tabref{Table_7.9}. In their adnominal uses, as mentioned in §\ref{Para_5.7.2}, the locatives most commonly co-occur with \isi{noun} phrases that occur in prepositional phrases (225/260 tokens – 87\%). Like the pronominally used locatives, the vast majority of adnominally used locatives occur in prepositional phrases with an overt \isi{preposition}: 46/78 \textitbf{sini} ‘\textsc{l.prox}’ (59\%), 55/66 \textitbf{situ} ‘\textsc{l.med}’ (84\%), and 86/116 \textitbf{sana} ‘\textsc{l.dist}’ (74\%) tokens. Far fewer \isi{locative} tokens occur in prepositional phrases with zero \isi{preposition}: 16/78 \textitbf{sini} ‘\textsc{l.prox}’ (21\%), 6/66 \textitbf{situ} ‘\textsc{l.med}’ (9\%), and 16/116 \textitbf{sana} ‘\textsc{l.dist}’ (14\%) tokens. The number of \isi{locative} tokens occurring in unembedded \isi{noun} phrases is equally low or still lower: 16/78 \textitbf{sini} ‘\textsc{l.prox}’ (21\%), 5/66 \textitbf{situ} ‘\textsc{l.med}’ (7\%), and 14/116 \textitbf{sana} ‘\textsc{l.dist}’ (12\%) tokens.



\begin{table}
\caption{Adnominally used locatives in prepositional phrases and unembedded \isi{noun} phrases}\label{Table_7.9}

\begin{tabular}{l*{6}{r}}
\lsptoprule
 & \multicolumn{2}{c}{ \textitbf{sini} ‘\textsc{l.prox}’} & \multicolumn{2}{c}{ \textitbf{situ} ‘\textsc{l.med}’} & \multicolumn{2}{c}{ \textitbf{sana} ‘\textsc{l.dist}}\\
\midrule
\textsc{pp} with \textsc{prep} &  46 &  59\% &  55 &  84\% &  86 &  74\%\\
\textsc{pp} with zero \textsc{prep} &  16 &  21\% &  6 &  9\% &  16 &  14\%\\
Unembedded \textsc{np} &  16 &  21\% &  5 &  7\% &  14 &  12\%\\
\midrule
Total &  78 &  100\% &  66 &  100\% &  116 &  100\%\\
\lspbottomrule
\end{tabular}
\end{table}
\subsection{Functions of locatives}
\label{Para_7.2.2}
The locatives have a number of different functions and uses which are discussed in more detail in the following sections: spatial uses in §\ref{Para_7.2.2.1}, figurative \isi{locational} uses in §\ref{Para_7.2.2.2}, temporal uses in §\ref{Para_7.2.2.3}, psychological uses in §\ref{Para_7.2.2.4}, and textual uses in §\ref{Para_7.2.2.5}.


\subsubsection[Spatial uses of locatives]{Spatial uses of locatives}
\label{Para_7.2.2.1}
In their spatial uses, the Papuan Malay locatives designate the location of an entity relative to that of the speaker and focus the hearer’s attention to the specific location of these entities. In the following, two issues are explored in more detail: the semantic distinctions between the three locatives, and the semantic distinctions between the pronominally and adnominally used locatives.


\subsubsubsection{\textit{Semantic distinctions between the three locatives}\label{Para_7.2.2.1.1}}

Generally speaking, proximal \textitbf{sini} ‘\textsc{l.prox}’ signals proximity to a deictic center, while distal \textitbf{sana} ‘\textsc{l.dist}’ expresses distance from this center. Medial \textitbf{situ} ‘\textsc{l.med}’ indicates mid distance; that is, the referent is more remote from the speaker than the referent of \textitbf{sini} ‘\textsc{l.prox}’ but not as far as the referent of \textitbf{sana} ‘\textsc{l.dist}’. The actual distances signaled with the locatives are relative, however, and depend on the speakers’ perceptions. The data also shows that the locatives are very commonly used independently of the parameter of visibility. Although \textitbf{sini} ‘\textsc{l.prox}’ most commonly denotes visible locations, it can also refer to invisible ones; likewise, the non-proximal locatives can refer to visible or invisible locations.



The spatial uses of \textitbf{sini} ‘\textsc{l.prox}’ are illustrated in (\ref{Example_7.49}) and (\ref{Example_7.50}). The semantic distinctions between \textitbf{sini} ‘\textsc{l.prox}’ and \textitbf{sana} ‘\textsc{l.dist}’ are shown in (\ref{Example_7.51}). The spatial uses of \textitbf{situ} ‘\textsc{l.med}’ and its semantic distinctions from \textitbf{sini} ‘\textsc{l.prox}’ and \textitbf{sana} ‘\textsc{l.dist}’ are illustrated in (\ref{Example_7.52}) and (\ref{Example_7.53}).



The examples in (\ref{Example_7.49}) and (\ref{Example_7.50}) illustrate the spatial uses of \textitbf{sini} ‘\textsc{l.prox}’. In both cases, adnominally used \textitbf{sini} ‘\textsc{l.prox}’ indicates the location of an entity close to the speaker: \textitbf{ember sini} ‘the bucket here’ in (\ref{Example_7.49}) and \textitbf{Sawar sini} ‘Sawar here’ in (\ref{Example_7.50}). The actual distances signaled with \textitbf{sini} ‘\textsc{l.prox}’ differ, however, depending on the speakers’ perceptions. In (\ref{Example_7.49}), \textitbf{ember sini} ‘the bucket here’ is standing right next to the speaker. By contrast in (\ref{Example_7.50}), \textitbf{Sawar sini} ‘Sawar here’ denotes a location that is situated about ten kilometers away from the speaker’s location. The speaker, however, perceives \textitbf{Sawar} as being close to his own location given that \textitbf{Apawer} is situated still further away. The examples in (\ref{Example_7.49}) and (\ref{Example_7.50}) also illustrate that \textitbf{sini} ‘\textsc{l.prox}’ is used independently of the parameter of visibility: the \isi{locative} is used for a visible location in (\ref{Example_7.49}) and for an invisible one in (\ref{Example_7.50}).


\begin{styleExampleTitle}
Spatial uses of \textitbf{sini} ‘\textsc{l.prox}’
\end{styleExampleTitle}

\ea
\label{Example_7.49}
\gll {sa} {su} {taru} {di} {\bluebold{ember}} {\bluebold{sini}}\\ %
 \textsc{1sg}  already  put  at  bucket  \textsc{l.prox}\\
\glt 
‘I already put (the fish) in \bluebold{the bucket here}’ \textstyleExampleSource{[080917-006-CvHt.0003]}
\z

\ea
\label{Example_7.50}
\gll {de} {mulay} {turung} {dari} {Apawer} {\ldots} {sampe} {di} {\bluebold{Sawar}} {\bluebold{sini}}\\ %
 \textsc{3sg}  start  descend  from  Apawer   { }  reach  at  Sawar  \textsc{l.prox}\\
\glt 
‘he (the ancestor) started coming down from Apawer {\ldots} (and) reached \bluebold{Sawar here}’ \textstyleExampleSource{[080922-010a-CvNF.0149]}
\z



The example in (\ref{Example_7.51}) illustrates the semantic distinctions between \textitbf{sini} ‘\textsc{l.prox}’ and \textitbf{sana} ‘\textsc{l.dist}’. The utterance occurred during a conversation outside at night. Noting that their neighbors are also sitting outside, the speaker employs the distal \isi{locative} to refer to the neighbors’ location \textitbf{sana} ‘over there’ and the proximal \isi{locative} to refer to their own location \textitbf{sini} ‘here’.


\begin{styleExampleTitle}
Spatial uses of \textitbf{sini} ‘\textsc{l.prox}’ and \textitbf{sana} ‘\textsc{l.dist}’
\end{styleExampleTitle}

\ea
\label{Example_7.51}
\gll {dong} {juga} {duduk} {di} {\bluebold{sana}} {tong} {juga} {duduk} {di} {\bluebold{sini}}\\ %
 \textsc{3pl}  also  sit  at  \textsc{l.dist}  \textsc{1pl}  also  sit  at  \textsc{l.prox}\\
\glt 
‘they also sit (outside) \bluebold{over there}, we also sit (outside) \bluebold{here}’ \textstyleExampleSource{[081025-009b-Cv.0075]}
\z



The examples in (\ref{Example_7.52}) and (\ref{Example_7.53}) show the spatial uses of \textitbf{situ} ‘\textsc{l.med}’ and its semantic distinctions from \textitbf{sini} ‘\textsc{l.prox}’ and \textitbf{sana} ‘\textsc{l.dist}’.



The exchange in (\ref{Example_7.52}) took place at night in front of the house while a meeting took place inside in the living room where the teenagers usually sleep. The young people were waiting for the guests to leave so that they could go to sleep. Employing \textitbf{situ} ‘\textsc{l.med}’, the first teenager wonders what the adults are doing \textitbf{situ} ‘there’ in the living room. Finally, the second teenager suggests they do not wait any longer: using \textitbf{sini} ‘\textsc{l.prox}’ she proposes that they sleep \textitbf{luar sini} ‘outside here’. The utterance in (\ref{Example_7.53}) occurred during a conversation about a road construction project. The speaker informs his interlocutor that the construction work has already reached the village of \textitbf{Warmer}, located to the east of the interlocutors’ location. Employing \textitbf{situ} ‘\textsc{l.med}’ and \textitbf{sana} ‘\textsc{l.dist}’, the speaker maintains that the construction work would continue \textitbf{dari situ} ‘from there (Warmer)’ further eastwards \textitbf{ke sana} ‘to over there’.


\begin{styleExampleTitle}
Spatial uses of \textitbf{situ} ‘\textsc{l.med}’
\end{styleExampleTitle}

\ea
\label{Example_7.52}
\ea
\label{Example_7.52a}
\gll {Teenager-1:} {dong} dong  biking  apa  \bluebold{situ}  \ldots\\ %
  { }     \textsc{3pl}  \textsc{3pl}  make  what  \textsc{l.med}  \\

\glt Teenager-1: ‘what are they they doing \bluebold{there}? \ldots’

\vspace{10pt}
\ex
\label{Example_7.52b}
 \gll  {Teenager-2:}  {yo,}  kitong  tidor  \bluebold{luar}  \bluebold{sini}\\
  { }    {yes}  \textsc{1pl}  sleep  outside  \textsc{l.prox}\\
\glt Teenager-2: ‘yes, we sleep \bluebold{outside here}’ \textstyleExampleSource{[080921-009-Cv.0001/0013]}\\
\z
\z
\ea
\label{Example_7.53}
\gll {{yo,}} {{mulay}} {{menuju}} {{jembatang}} {Warmer} {\ldots} {kalo} {dari} {\bluebold{situ}}\\ %
 {yes}  {start}  {aim.at}  {bridge}  Warmer {\ldots}   if  from  \textsc{l.med}\\
\gll ke  {\bluebold{sana},}  {o},  {itu}  dia  {\ldots}\\
 to  {\textsc{l.dist}}  {oh!}  {\textsc{d.dist}}  \textsc{3sg}  {}\\
\glt 
‘yes, (they) started working toward the Warmer bridge {\ldots} when (they’ll work the stretch of the road) from \bluebold{there} to \bluebold{over there}, oh, what’s-its-name, it \ldots’ \textstyleExampleSource{[081006-033-Cv.0013/0015/0017]}
\z



The examples in (\ref{Example_7.51}) and (\ref{Example_7.53}) again show that distances signaled with the non-prox\-i\-mal locatives are relative. In (\ref{Example_7.51}) \textitbf{sana} ‘\textsc{l.dist}’ refers to the neighbor’s house, situated about fifty meters away from where the speakers are sitting. By contrast, \textitbf{situ} ‘\textsc{l.med}’ in (\ref{Example_7.53}) denotes the village of Warmer which is located several kilometers away from the speaker’s location, while \textitbf{sana} ‘\textsc{l.dist}’ refers to the area beyond Warmer. Besides, these examples show that the non-proximal locatives are also used independently of the parameter of visibility. Distal \textitbf{sana} ‘\textsc{l.dist}’ is used for a visible location in (\ref{Example_7.51}) and an invisible one in (\ref{Example_7.53}). Medial \textitbf{situ} ‘\textsc{l.med}’ refers to a visible location in (\ref{Example_7.52}) and an invisible one in (\ref{Example_7.53}).


\subsubsubsection{\textit{Semantic distinctions between the pronominally and adnominally used locatives}\label{Para_7.2.2.1.2}}

In designating the location of a referent relative to that of the speaker, Papuan Malay makes a distinction between the pronominally and the adnominally used locatives.



Pronominally used locatives provide additional information about the location of an entity or referent without restricting its referential scope. Adnominally used locatives, by contrast, have a restrictive function, thereby assisting the hearer in the identification of the referent. That is, by directing the hearer’s attention to the referent’s location, adnominal locatives indicate that the referent is precisely the one situated in the location designated by the \isi{locative}. This distinction is illustrated with the (near) contrastive examples in (\ref{Example_7.54}) and (\ref{Example_7.55}).



The prepositional phrases with pronominally used \textitbf{sini} ‘\textsc{l.prox}’ in (\ref{Example_7.54a}) and (\ref{Example_7.55a}) provide additional information about the location of the referents, information that is nonessential for their identification: \textitbf{orang di sini} ‘the people here’ in (\ref{Example_7.54a}) and \textitbf{dorang di sini} ‘them here’ in (\ref{Example_7.55a}). By contrast, in (\ref{Example_7.54b}) and (\ref{Example_7.55b}) the respective head nominals \textitbf{orang} ‘person’ and \textitbf{dorang} ‘\textsc{3pl}’ are modified with \textitbf{sini} ‘\textsc{l.prox}’. In both cases, the locatives indicate that the referents of \textitbf{orang} ‘person’ and \textitbf{dorang} ‘\textsc{3pl}’ are precisely the ones located \textitbf{sini} ‘here’ as opposed to other locations: \textitbf{orang sini} ‘the people that are here’ in \ref{Example_7.54b}) and \textitbf{dorang sini} ‘they that are here’ in (\ref{Example_7.55b}).


\begin{styleExampleTitle}
Adnominally versus pronominally used locatives
\end{styleExampleTitle}

\ea
\label{Example_7.54}
\ea
\label{Example_7.54a}
\gll  {\bluebold{orang}} {\bluebold{di}} {\bluebold{sini}} {bilang} {pake-pake}\\ %
   person  at  \textsc{l.prox}  say  practice.black.magic\\
\glt ‘\bluebold{the people here} say ``black magic''' \textstyleExampleSource{[081006-022-CvEx.0028]}
\vspace{10pt}
\ex
\label{Example_7.54b}
\gll  jadi  \bluebold{orang}  \bluebold{sini}  bilang,  kemaring  dulu\\
   so  person  \textsc{l.prox}  say  yesterday  be.prior\\
\glt ‘so \bluebold{the people (that are) here} say ``the day before yesterday''' \textstyleExampleSource{[081006-019-Cv.0015]}
\z
\z

\ea
\label{Example_7.55}
\ea
\label{Example_7.55a}
\gll  {Lodia} {datang} {ke} {mari,} {de} {kas} {bodo} {\bluebold{dorang}} {\bluebold{di}} {\bluebold{sini}}\\ %
   Lodia  come  to  hither  \textsc{3sg}  give  be.stupid  \textsc{3pl}  at  \textsc{l.prox}\\
\glt ‘(when) Lodia came here, she told \bluebold{them here} how stupid they were’ (Lit. ‘made \bluebold{them here} stupid’) \textstyleExampleSource{[081115-001a-Cv.0136]}
\vspace{10pt}
\ex
\label{Example_7.55b}
\gll  baru  sa  liat  \bluebold{dorang}  \bluebold{sini}  su  terlalu  enak\\
   and.then  \textsc{1sg}  see  \textsc{3pl}  \textsc{l.prox}  already  too  be.pleasant\\
\glt [Comment about ill-behaved teenagers:] ‘and then I see \bluebold{(that) they (that are) here} already (have) too pleasant (lives)’ \textstyleExampleSource{[081115-001a-Cv.0311]}\\
\z
\z

\subsubsection[Figurative {locational} uses of locatives]{Figurative {locational} uses of locatives}
\label{Para_7.2.2.2}
The spatial uses of the locatives can be expanded to figurative \isi{locational} uses in narratives. Employing a \isi{locative} preceded by \textitbf{sampe di} ‘reach at’, the narrators bring their stories to a figurative \isi{locational} endpoint. Such uses are attested for \textitbf{sini} ‘\textsc{l.prox}’ as in (\ref{Example_7.56}) and \textitbf{situ} ‘\textsc{l.med}’ as in (\ref{Example_7.57}), but not for \textitbf{sana} ‘\textsc{l.dist}’.

\ea
\label{Example_7.56}
\gll {\ldots} {sa} {su} {sembu,} {trima-kasi} {sampe} {di} {\bluebold{sini}}\\ %
   { }  \textsc{1sg}  already  be.healed  thank.you  reach  at  \textsc{l.prox}\\
\glt 
‘[after this accident] I already recovered, thank you! \bluebold{this is all}’ (Lit. ‘reach \bluebold{here}’) \textstyleExampleSource{[081015-005-NP.0051]}
\z

\ea
\label{Example_7.57}
\gll {sa} {pikir} {mungking} {sampe} {di} {\bluebold{situ}} {dulu}\\ %
 \textsc{1sg}  think  maybe  reach  at  \textsc{l.med}  first\\
\glt
‘I think maybe \bluebold{that’s all} for now’ (Lit. ‘reach \bluebold{there}’) \textstyleExampleSource{[080919-004-NP.0083]}
\z


\subsubsection[Temporal uses of locatives]{Temporal uses of locatives}
\label{Para_7.2.2.3}
The \isi{locative} \textitbf{situ} ‘\textsc{l.med}’ also has temporal uses. Preceded by the \isi{preposition} \textitbf{dari} ‘from’, \textitbf{situ} ‘\textsc{l.med}’ signals the temporal setting of the event talked about with respect to some temporal reference point in the past, as illustrated in (\ref{Example_7.58}). Overall, however, this domain of use is not very common, with the corpus containing only two such occurrences. The proximal and distal locatives are not attested to have temporal uses.

\ea
\label{Example_7.58}
\gll {\bluebold{dari}} {{\bluebold{situ}}} {sa} {punya} {mama} {tida} {maw} {jualang} {pagi} {lagi}\\ %
 from  {\textsc{l.med}}  \textsc{1sg}  \textsc{poss}  mother  \textsc{neg}  want  merchandise/sell  {morning} {again}\\
\glt ‘\bluebold{from that moment on} my mother didn’t want to do any more vending in the morning’ (Lit. ‘\bluebold{from there}’) \textstyleExampleSource{[081014-014-NP.0006]}
\z


\subsubsection[Psychological uses of locatives]{Psychological uses of locatives}
\label{Para_7.2.2.4}
The locatives also have limited psychological uses to signal the speakers’ \isi{emotional involvement} and attitudes. The corpus contains a fair number of utterances in which the speakers switch from \textitbf{sana} ‘\textsc{l.dist}’ to \textitbf{situ} ‘\textsc{l.med}’ to refer to the same location. With this switch the speakers indicate that the location has become vivid to their minds and psychologically closer than \textitbf{sana} ‘\textsc{l.dist}’ would signal, as illustrated in (\ref{Example_7.59}) and (\ref{Example_7.60}).



In (\ref{Example_7.59}), a father relates that he will bring his two oldest children to the provincial capital Jayapura for further schooling once the younger one has finished high school. Anaphorically used \textitbf{sana} ‘\textsc{l.dist}’ signals that Jayapura is at considerable distance from the speaker’s current location (ca. 300 km). The subsequent use of \textitbf{situ} ‘\textsc{l.med}’ indicates that with his two children going to live there, distant Jayapura has become psychologically much closer.


\begin{styleExampleTitle}
Psychological uses: Example \#1
\end{styleExampleTitle}

\ea
\label{Example_7.59}
\gll {{kalo}} {Ise} {{ni}} {{selesay}} {saya} {mo} {bawa} {dong} {dua}\\ %
 {if}  Ise  {\textsc{d.prox}}  {finish}  \textsc{1sg}  want  bring  \textsc{3pl}  two\\
\gll ke  {\bluebold{sana}}  {tinggal}  di  {\bluebold{situ}}\\
 to  {\textsc{l.dist}}  {stay}  at  {\textsc{l.med}}\\
\glt 
‘when Ise here has finished (her schooling) I want to bring the two of them to (Jayapura) \bluebold{over there} to live \bluebold{there}’ \textstyleExampleSource{[081025-003-Cv.0135]}
\z



Likewise in (\ref{Example_7.60}), the speaker switches from \textitbf{sana} ‘\textsc{l.dist}’ to \textitbf{situ} ‘\textsc{l.med}’ to refer to the Mambramo area \textitbf{sana} ‘over there’, situated about 100 km to the west. The switch occurs at the moment when the speaker considers his own involvement with the Mambramo area, namely that he has never been \textitbf{situ} ‘there’. Again, this switch indicates that the location talked about has become more vivid and psychologically closer to the speaker’s mind.


\begin{styleExampleTitle}
Psychological uses: Example \#2
\end{styleExampleTitle}

\ea
\label{Example_7.60}
\gll {kaka} {{dong}} {{di}} {{\bluebold{sana}}} {sodara} {{banyak}} {skali} {\ldots} {sa} {juga}\\ %
 oSb  {\textsc{3pl}}  {at}  {\textsc{l.dist}}  sibling  {many}  very { }   \textsc{1sg}  also\\
\gll {blum}  {perna}  {sa}  {kunjungang}  ke  {\bluebold{situ}}\\
 {not.yet}  {once}  {\textsc{1sg}}  {visit}  to  {\textsc{l.med}}\\
\glt
‘the older relatives \bluebold{over there}, (the) relatives are very many in (the Mambramo area), {\ldots} me too, I have never been \bluebold{there}’ \textstyleExampleSource{[080922-010a-CvNF.0158]}
\z



\subsubsection[Textual anaphoric uses of locatives]{Textual anaphoric uses of locatives}
\label{Para_7.2.2.5}
In their textual uses, locatives are used anaphorically; that is, they are coreferen\-tial with a discourse antecedent that denotes a location. In (\ref{Example_7.61}) \textitbf{sini} ‘\textsc{l.prox}’ corefers with the place where the speaker was standing, namely where there were \textitbf{daung klapa {\ldots} itu} ‘those coconut leaves’. Medial \textitbf{situ} ‘\textsc{l.med}’ in (\ref{Example_7.62}) corefers with \textitbf{laut} ‘sea’, and \textitbf{sana} ‘\textsc{l.dist}’ in (\ref{Example_7.63}) with \textitbf{sa pu temang} ‘my friend’. The three examples also show that in their anaphoric uses the locatives may be employed pronominally as in (\ref{Example_7.61}) and (\ref{Example_7.62}), or adnominally as in (\ref{Example_7.63}).

\ea
\label{Example_7.61}
\gll {{baru}} {{daung}} {{klapa}} {itu} {{\bluebold{daung}}} {{\bluebold{klapa}}} {{\bluebold{yang}}} {{\bluebold{saya}}} {{\bluebold{ada}}}\\ %
 {and.then}  {leaf}  {coconut}  \textsc{d.dist}  {leaf}  {coconut}  {\textsc{rel}}  {\textsc{1sg}}  {exist}\\
\gll \bluebold{berdiri}  {\bluebold{itu}}  {\ldots}  sa  {bilang}  {\ldots}  {dari}  {\bluebold{sini}}  {sa}  {kutuk}  dia\\
 stand  {\textsc{d.dist}}  {}  \textsc{1sg}  {say} {}   {from}  {\textsc{l.prox}}  {\textsc{1sg}}  {curse}  \textsc{3sg}\\
\glt 
‘and then those coconut leaves, \bluebold{those coconut leaves where I was standing} {\ldots} I said, ``{\ldots} from \bluebold{here} I curse him (the evil spirit)''' \textstyleExampleSource{[080917-008-NP.0101/0103]}
\z

\ea
\label{Example_7.62}
\gll {{ey,}} {kam} {dua} {{pi}} {{mandi}} {di} {\bluebold{laut}} {suda!,} {trus} {kam}\\ %
 {hey!}  \textsc{2pl}  two  {go}  {bathe}  at  sea  already  next  \textsc{2pl}\\
\gll dua  {cuci}  {celana}  {di}  {\bluebold{situ}}\\
 two  {wash}  {trouser}  {at}  {\textsc{l.med}}\\
\glt 
[A mother addressing her young sons:] ‘hey, you two go bathe in the \bluebold{sea} already!, then you two wash (your) trousers \bluebold{there}!’ \textstyleExampleSource{[080917-006-CvHt.0007]}
\z


\ea
\label{Example_7.63}
\gll {{tong}} {{dari}} {\bluebold{sa}} {{\bluebold{pu}}} {{\bluebold{temang}}} {{pinjam}} {{trening}} {untuk} {besok}\\ %
 {\textsc{1pl}}  {from}  \textsc{1sg}  {\textsc{poss}}  {friend}  {borrow}  {tracksuit}  for  tomorrow\\
\gll {\ldots}  {tu}  {yang}  {tadi}  sa  {ke}  {\bluebold{temang}}  {\bluebold{sana}}\\
  { }  {\textsc{d.dist}}  {\textsc{rel}}  {earlier}  \textsc{1sg}  {to}  {friend}  {\textsc{l.dist}}\\
\glt
‘we (are back) from \bluebold{my friend} (from whom we) borrowed a tracksuit for tomorrow {\ldots} that’s why a short while ago I (went) to (my) \bluebold{friend (who is) over there}’ \textstyleExampleSource{[081011-020-Cv.0052/0056]}
\z


\section{Combining demonstratives and locatives}
\label{Para_7.3}
Demonstratives and locatives can be combined with an adnominally used \isi{demonstrative} modifying a pronominally used \isi{locative} as in (\ref{Example_7.64}) and (\ref{Example_7.65}), or an adnominally used \isi{locative} as in (\ref{Example_7.66}). In these constructions, the \isi{demonstrative} serves to intensify the \isi{locative}, resulting in an emphatic reading that conveys \isi{vividness}.



Short distal \textitbf{itu} ‘\textsc{d.dist}’ modifies proximal \textitbf{sini} ‘\textsc{l.prox}’ in (\ref{Example_7.64}) and medial \textitbf{situ} ‘\textsc{l.med}’ in (\ref{Example_7.65}). In (\ref{Example_7.66}) long distal \textitbf{itu} ‘\textsc{d.dist}’ modifies distal \textitbf{sana} ‘\textsc{l.dist}’.

\ea
\label{Example_7.64}
\gll {{dorang}} {tida} {{bisa}} {dekat} {{sama}} {{dorang}} {\ldots} {di} {{\bluebold{sini}}} {\bluebold{tu}}\\ %
 {\textsc{3pl}}  \textsc{neg}  {be.able}  near  {with}  {\textsc{3pl}}   { }  at  {\textsc{l.prox}}  \textsc{d.dist}\\
\gll ada  {orang}  {swanggi}  {satu}  {de}  {bertobat}  {\ldots}\\
 exist  {person}  {nocturnal.evil.spirit}  {one}  {\textsc{3sg}}  {repent}  { }\\
\glt 
‘they (the evil spirits) can’t be close to them (God’s children) {\ldots} \bluebold{here (}\blueboldSmallCaps{emph}\bluebold{)} is one evil sorcerer, he has become a Christian’ \textstyleExampleSource{[081006-022-CvEx.0146/0150]}
\z

\ea
\label{Example_7.65}
\gll {tida} {{bisa}} {{kamu}} {{tinggal}} {{di}} {situ,} {di} {\bluebold{situ}} {\bluebold{tu}}\\ %
 \textsc{neg}  {be.able}  {\textsc{2pl}}  {stay}  {at}  \textsc{l.med}  at  \textsc{l.med}  \textsc{d.dist}\\
\gll {ruma}  {tu}  {ada}  {setang}  {banyak}\\
 {house}  {\textsc{d.dist}}  {exist}  {evil.spirit}  {many}\\
\glt 
‘you can’t live there, \bluebold{there (}\blueboldSmallCaps{emph}\bluebold{)}, (in) that house are many evil spirits’ \textstyleExampleSource{[081006-022-CvEx.0164]}
\z

\ea
\label{Example_7.66}
\gll {sana,} {te} {ada} {di} {\bluebold{sana}} {\bluebold{itu}}\\ %
 \textsc{l.dist}  tea  exist  at  \textsc{l.dist}  \textsc{d.dist}\\
\glt 
‘there, the tea (is) \bluebold{over there (}\blueboldSmallCaps{emph}\bluebold{)}’ \textstyleExampleSource{[081014-011-CvEx.0010]}
\z


In all attested combinations, it is the distal \isi{demonstrative} that modifies a \isi{locative}. Modification of a \isi{locative} with proximal \textitbf{ini} ‘\textsc{d.prox}’ is also possible although unattested, as discussed in §\ref{Para_5.7.1}).


\section{Summary}
\label{Para_7.4}
The Papuan Malay demonstratives and locatives are deictic expressions. They provide orientation to the hearer in the outside world and in the speech situation, in spatial as well as in nonspatial domains. Both deictic systems are distance oriented, in that they signal the relative distance of an entity vis-à-vis a deictic center. At the same time, the two systems differ in a number of respects. They are distinct both in terms of their syntactic characteristics and forms and in terms of their functions.



The differences between the demonstratives and the locatives with respect to their syntactic characteristics and forms are summarized in \tabref{Table_7.10}.

The main distinctions between the demonstratives and the locatives in terms of their various functions are summarized in \tabref{Table_7.11}.

In summary, with respect to their syntactic properties, the demonstratives have a wider range of uses (adnominal, pronominal, and adverbial uses) than the locatives. Likewise, in terms of their functions, the demonstratives have a wider range of uses than the locatives. The \isi{locative} system, by contrast, allows finer semantic distinctions to be made than the \isi{demonstrative} system, given that the former expresses a three-way deictic contrast, whereas the latter expresses a two-way deictic contrast.




\begin{table}
\caption{Syntax and forms of the demonstratives (\textsc{dem}) and locatives (\textsc{loc})}\label{Table_7.10}

\begin{tabularx}{\textwidth}{p{2.2 cm}p{4.4 cm}p{4.4 cm}}
%\begin{tabularx}{\textwidth}{lll}
\lsptoprule
 \multicolumn{1}{c}{Syntax and} & \multicolumn{1}{c}{\textsc{dem}} &  \multicolumn{1}{c}{\textsc{loc}}\\
 \multicolumn{1}{c}{forms}\\
\midrule
Deictic forms & Two term system:

\begin{itemize}[noitemsep,nolistsep]
\item proximal \textitbf{ini} ‘\textsc{d.prox}’
\item distal \textitbf{itu} ‘\textsc{d.dist}’
 \end{itemize}
  &
  Three-term system:
\begin{itemize}[noitemsep,nolistsep]
\item proximal \textitbf{sini} ‘\textsc{l.prox}’\item medial \textitbf{situ} ‘\textsc{l.med}’
\item distal \textitbf{sana} ‘\textsc{l.dist}’
\end{itemize}
\\
\tablevspace
Distributional properties & 
\begin{itemize}[noitemsep,nolistsep, before*={\mbox{}\vspace{-\baselineskip}}]
\item adnominal uses
\item pronominal uses
\item adverbial uses
\end{itemize}
 & 
\begin{itemize}[noitemsep,nolistsep, before*={\mbox{}\vspace{-\baselineskip}}]
\item adnominal uses
\item pronominal uses
\end{itemize}\\
\tablevspace

Pronominal uses & 

\begin{itemize}[noitemsep,nolistsep, before*={\mbox{}\vspace{-\baselineskip}}]
\item in unembedded \textsc{np}s
\item in \textsc{pp}s
\item in adnominal possessive constructions
\end{itemize}
 & 
\begin{itemize}[noitemsep,nolistsep, before*={\mbox{}\vspace{-\baselineskip}}]
\item in \textsc{pp}s
\end{itemize}\\
\tablevspace

Adnominal uses & 
 \begin{itemize}[noitemsep,nolistsep, before*={\mbox{}\vspace{-\baselineskip}}]
\item can be stacked
\end{itemize}
 & 
\begin{itemize}[noitemsep,nolistsep, before*={\mbox{}\vspace{-\baselineskip}}]
\item unattested
\end{itemize}\\

\lspbottomrule
\end{tabularx}
\end{table}



\begin{table}
\caption{Functions of the demonstratives (\textsc{dem}) and locatives (\textsc{loc})}\label{Table_7.11}

\begin{tabularx}{\textwidth}{p{3 cm}p{4 cm}p{4 cm}}
\lsptoprule
 \multicolumn{1}{c}{Domains of use} & \multicolumn{1}{c}{\textsc{dem}} &  \multicolumn{1}{c}{\textsc{loc}}\\
\midrule
Spatial & provide spatial orientation by drawing the hearer’s at\-tention to specific entities in the discourse or surround\-ing situation & provide spatial orientation by designating the location of an entity and focusing the hearer’s attention to its specific location\\
\tablevspace
Figurative\newline \isi{locational} & unattested  & signal a figurative \isi{locational} endpoint\\
\tablevspace
Temporal & indicate the temporal setting of an event/situation & indicate the temporal setting of an event/situation (medial \isi{locative} only)\\
\tablevspace
Psychological & indicate the speaker’s emo\-tional involvement with an event/situation

signal \isi{vividness}

indicate contrast & indicate the speakers’ emo\-tional involvement with an event/situation

signal \isi{vividness}\\
\tablevspace
Identificational & aid in the identification of referents (long forms) &  unattested \\
\tablevspace
Textual anaphoric & keep track of discourse participants & keep track of the location of an entity\\
\tablevspace
Textual discourse deictic & establish an overt link be\-tween two propositions &  unattested \\
\tablevspace
Placeholder & substitute for specific lexi\-cal items in the context of word-formulation trouble (long forms) & unattested \\
\lspbottomrule
\end{tabularx}
\end{table}
