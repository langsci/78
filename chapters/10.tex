\chapter[Prepositions and the prepositional phrase]{Prepositions and the prepositional phrase}
\label{Para_10}
This chapter describes prepositional phrases in Papuan Malay, that is, constructions which consist of a \isi{preposition} followed by a \isi{noun} phrase, such that ``\textsc{prep} \textsc{np}''.



Papuan Malay employs eleven prepositions that can be grouped semantically into (1) prepositions encoding location in space and time, (2) prepositions encoding \isi{accompaniment}/instruments, goals, and \isi{benefaction}, and (3) prepositions encoding comparisons. The defining characteristics of prepositions are discussed in §\ref{Para_5.11}.



Prepositional phrases have the following defining characteristics:


\begin{enumerate}
\item 
All prepositional phrases function as peripheral adjuncts; as such they do not have a grammatically restricted position within the clause but can be moved to different positions.
\item Most prepositional phrases also function as nonverbal predicates and/\-or oblique arguments (see §\ref{Para_12.4} and §\ref{Para_11.1.3.2}, respectively).

Some prepositional phrases also function as modifiers within \isi{noun} phrases (§\ref{Para_8.2.7})

\item 
Prepositional phrases that function as nonverbal predicates can be modified by aspectual adverbs (§\ref{Para_5.4.1}), while such \isi{modification} is unattested for prepositional phrases having other functions.

\end{enumerate}

In the following, Papuan Malay prepositional phrases are discussed according to the semantics of their prepositional head: location in space and time in §\ref{Para_10.1}, \isi{accompaniment}/instruments, goals, and \isi{benefaction} in §\ref{Para_10.2}, and comparisons in §\ref{Para_10.3}. The main points of this chapter are summarized in §\ref{Para_10.4}.


\section{Prepositions encoding location in space and time}
\label{Para_10.1}
Papuan Malay employs four prepositions that express location in space and time: \isi{locative} \textitbf{di} ‘at, in’ designates static location (§\ref{Para_10.1.1}), allative \textitbf{ke} ‘to’ denotes direction toward a location (§\ref{Para_10.1.2}), elative \textitbf{dari} ‘from’ expresses direction away from or out of a location (§\ref{Para_10.1.3}), and lative \textitbf{sampe} ‘until’ designates direction up to a nonspatial temporal location (§\ref{Para_10.1.4}).


\subsection{\textitbf{di} ‘at, in’}
\label{Para_10.1.1}
Prepositional phrases introduced with \isi{locative} \textitbf{di} ‘at, in’ indicate static location in spatial and nonspatial figurative terms. Most often the \isi{preposition} denotes location ‘at’ or ‘in’ a referent; depending on its context, though, it is also translatable as ‘on’.

Very commonly, \textitbf{di} ‘at, in’ introduces a peripheral location as in \textitbf{di kampung} ‘in the village’ in (\ref{Example_10.1}), or \textitbf{di dia} ‘at hers’ in (\ref{Example_10.2}). When following placement verbs such as \textitbf{taru} ‘put’ in (\ref{Example_10.3}), \textitbf{di} ‘at, in’ introduces oblique \isi{locative} arguments that indicate the location of the referent, as in \textitbf{di sini} ‘here’. Frequently, \textitbf{di} ‘at, in’ also introduces nonverbal predicates, as in (\ref{Example_10.4}) (see §\ref{Para_12.4}). Only rarely, \textitbf{di} ‘at, in’ introduces locations encoded by adnominal prepositional phrases, as in \textitbf{pasar di bawa tu} ‘the market down there’ in (\ref{Example_10.5}). The examples in (\ref{Example_10.1}) to (\ref{Example_10.5}) also show that \textitbf{di} ‘at, in’ introduces animate and inanimate, as well as nominal and pronominal referents.\footnote{In the corpus only the following pronominal complements of \textitbf{di} ‘at, in’ are attested: \textsc{2sg}, \textsc{1pl}, and \textsc{3pl}.}


\ea
\label{Example_10.1}
\gll {waktu} {{saya}} {dengang} {bapa} {tinggal} {\bluebold{di}} {\bluebold{kampung}} {saya} {kerja}\\ %
 time  {\textsc{1sg}}  with  father  stay  at  village  \textsc{1sg}  work\\
\gll {sperti}  {laki{\Tilde}laki}\\
 {similar.to}  {\textsc{rdp}{\Tilde}husband}\\
\glt 
‘when I and (my) husband (‘father’) were living \bluebold{in the village}, I worked like a man’ \textstyleExampleSource{[081014-007-CvEx.0048]}
\z

\ea
\label{Example_10.2}
\gll {{jadi}} {{saya}} {{besar}} {di} {Ida} {dengang} {de} {punya} {laki} {tu}\\ %
 {so}  {\textsc{1sg}}  {be.big}  at  Ida  with  \textsc{3sg}  \textsc{poss}  husband  \textsc{d.dist}\\
\gll {\ldots}  {besar}  {\bluebold{di}}  {\bluebold{dia}}\\
  { } {be.big}  {at}  {\textsc{3sg}}\\
\glt 
‘so I grew up with Ida and that husband of hers {\ldots}, (I) grew up \bluebold{at hers}’ \textstyleExampleSource{[080927-007-CvNP.0017/0019]}
\z

\ea
\label{Example_10.3}
\gll {skarang} {kamu} {kasi} {terpol{\Tilde}terpol,} {taru} {\bluebold{di}} {\bluebold{sini}}\\ %
 now  \textsc{2pl}  give  \textsc{rdp}{\Tilde}jerry.can  put  at  \textsc{l.prox}\\
\glt 
‘now you give (me) the jerry cans, put (them) \bluebold{here}’ \textstyleExampleSource{[081110-002-Cv.0065]}
\z

\ea
\label{Example_10.4}
\gll {sa} {\bluebold{di}} {\bluebold{IPS}} {\bluebold{satu}}\\ %
 \textsc{1sg}  at  social.sciences  one\\
\glt 
[About course tracks in high school:] ‘I (am) \bluebold{in Social Sciences I}’ \textstyleExampleSource{[081023-004-Cv.0020]}
\z

\ea
\label{Example_10.5}
\gll {\bluebold{pasar}} {\bluebold{di}} {\bluebold{bawa}} {\bluebold{tu}} {raaame}\\ %
 market  at  bottom  \textsc{d.dist}  be.bustling\\
\glt
‘\bluebold{the market down there} is very bustling’ \textstyleExampleSource{[081109-005-JR.0008]}
\z

\subsection{\textitbf{ke} ‘to’}
\label{Para_10.1.2}
Prepositional phrases introduced with allative \textitbf{ke} ‘to’ denote direction toward a referent. Following motion verbs such as \textitbf{lari} ‘run’ in (\ref{Example_10.6}) or \textitbf{datang} ‘come’ in (\ref{Example_10.7}), \textitbf{ke} ‘to’ introduces oblique \isi{locative} arguments which indicate the goal of the motion, as in \textitbf{ke pante} ‘to the beach’ or \textitbf{ke kitong} ‘to us’, respectively. Allative \textitbf{ke} ‘to’ also very often introduces nonverbal predicates as in (\ref{Example_10.8}). The three examples also show that \textitbf{ke} ‘to’ introduces animate and inanimate, as well as nominal and pronominal referents.\footnote{In the corpus only the following pronominal complements of \textitbf{ke} ‘to’ are attested: \textsc{1sg}, \textsc{3sg}, \textsc{2pl}.}


\ea
\label{Example_10.6}
\gll {dong} {lari} {\bluebold{ke}} {\bluebold{pante}}\\ %
 \textsc{3pl}  run  to  coast\\
\glt 
‘they ran \bluebold{to the beach}’ \textstyleExampleSource{[081115-001a-Cv.0008]}
\z

\ea
\label{Example_10.7}
\gll {\ldots} {dia} {punya} {aroa} {datang} {\bluebold{ke}} {\bluebold{kitong}} {kasi} {tanda}\\ %
  {} \textsc{3sg}  \textsc{poss}  departed.spirit  come  to  \textsc{1pl}  give  sign\\
\glt 
‘[so when there is another person (who) dies in a different village,] (then) his/her departed spirit comes \bluebold{to us} (and) gives (us) a sign’ \textstyleExampleSource{[081014-014-NP.0048]}
\z

\ea
\label{Example_10.8}
\gll {sa} {\bluebold{ke}} {\bluebold{ruma-sakit}}\\ %
 \textsc{1sg}  to  hospital\\
\glt
‘I (went) \bluebold{to the hospital}’ \textstyleExampleSource{[081015-005-NP.0047]}
\z

\subsection{\textitbf{dari} ‘from’}
\label{Para_10.1.3}
Prepositional phrases introduced with elative \textitbf{dari} ‘from’ designate direction away from or out of a source location; depending on its context, though, \textitbf{dari} also translates with ‘of’. Most commonly, the source location is spatial. In addition, \textitbf{dari} ‘from’ expresses nonspatial figurative sources, temporal starting points, and the notions of superiority and dis\isi{similarity} in comparison constructions.



Elative \textitbf{dari} ‘from’ forms peripheral adjuncts, as in \textitbf{dari blakang} ‘from the back’ in (\ref{Example_10.9}). When following motion verbs such as \textitbf{kluar} ‘go out’, it expresses the source of the motion in an oblique argument, as in (\ref{Example_10.10}). Besides, elative \textitbf{dari} ‘from’ expresses spatial source locations in nonverbal predicates, as in (\ref{Example_10.11}). Much less often, \textitbf{dari} ‘from’ introduces sources encoded by adnominal prepositional phrases, as in (\ref{Example_10.12}).


\begin{styleExampleTitle}
Introducing spatial source locations
\end{styleExampleTitle}

\ea
\label{Example_10.9}
\gll {de} {tutup} {itu} {spit} {itu} {\bluebold{dari}} {\bluebold{blakang}} {\ldots}\\ %
 \textsc{3sg}  close  \textsc{d.dist}  speedboat  \textsc{d.dist}  from  backside  \\
\glt 
‘(this wave,) it totally covered, what’s-its-name, that speedboat \bluebold{from the back} [to the front]’ \textstyleExampleSource{[080923-015-CvEx.0021]}
\z

\ea
\label{Example_10.10}
\gll {\ldots} {sa} {harus} {kluar} {\bluebold{dari}} {\bluebold{kam}} {\bluebold{pu}} {\bluebold{kluarga}}\\ %
  {}  \textsc{1sg}  have.to  go.out  from  \textsc{2pl}  \textsc{poss}  family\\
\glt 
‘[I hadn’t thought that] I would have to depart \bluebold{from your family}’ \textstyleExampleSource{[080919-006-CvNP.0012]}
\z

\ea
\label{Example_10.11}
\gll {tong} {smua} {\bluebold{dari}} {\bluebold{kampung}}\\ %
 \textsc{1pl}  all  from  village\\
\glt 
‘we all are \bluebold{from the village}’ \textstyleExampleSource{[081010-001-Cv.0084]}
\z

\ea
\label{Example_10.12}
\gll {satu} {kali} {ini} {{\bluebold{de}}} {{\bluebold{pu}}} {{\bluebold{bapa}}} {\bluebold{pu}} {\bluebold{temang}} {\bluebold{dari}} {\bluebold{skola},}\\ %
 one  time  \textsc{d.prox}  {\textsc{3sg}}  {\textsc{poss}}  {father}  \textsc{poss}  friend  from  school\\
\gll \textsc{STT}  {dorang}  {pergi}  {\ldots}\\
 {theological.seminary}  {\textsc{3pl}}  {go}  {}\\
\glt 
‘this one time \bluebold{her father’s friends from school}, theological seminary, they went {\ldots}’ \textstyleExampleSource{[081006-023-CvEx.0062]}
\z


The source location indicated with \textitbf{dari} ‘from’ can also be nonspatial figurative as in the prepositional predicate clauses \textitbf{dari uang} ‘up to the money’ in (\ref{Example_10.13}), or \textitbf{dari ko} ‘up to you’ in (\ref{Example_10.14}).


\begin{styleExampleTitle}
Introducing nonspatial figurative source locations
\end{styleExampleTitle}

\ea
\label{Example_10.13}
\gll {yo,} {tong} {mo} {biking} {cepat,} {smua} {itu} {\bluebold{dari}} {\bluebold{uang}}\\ %
 yes  \textsc{1pl}  want  make  be.fast  all  \textsc{d.dist}  from  money\\
\glt 
‘yes, we want to do (it) quickly, all that (is) \bluebold{up to the money}’ (Lit. ‘\bluebold{from money}’) \textstyleExampleSource{[080927-006-CvNP.0034]}
\z

\ea
\label{Example_10.14}
\gll {pinda} {ke} {IPA} {itu} {\bluebold{dari}} {\bluebold{ko}} {saja}\\ %
 move  to  natural.sciences  \textsc{d.dist}  from  \textsc{2sg}  just\\
\glt 
‘switching (from Social Sciences) to Natural Sciences, that (is) \bluebold{up to you} alone’ (Lit. ‘\bluebold{from you}’) \textstyleExampleSource{[081023-004-Cv.0023]}
\z


The examples in (\ref{Example_10.9}) to (\ref{Example_10.14}) also illustrate that \textitbf{dari} ‘from’ introduces animate and inanimate, as well as nominal and pronominal referents.\footnote{In the corpus one pronominal complement of \textitbf{dari} ‘from’ is unattested, namely \textsc{2pl}.}



Derived from its spatial semantics, \textitbf{dari} ‘from’ also very commonly introduces nonspatial temporal source locations, which are always encoded by peripheral adjuncts. The temporal starting point can be encoded by a \isi{noun} that indicates time as in \textitbf{dari pagi} ‘from the morning’ in (\ref{Example_10.15}), or by a temporal ad\isi{verb} as in \textitbf{dari dulu} ‘from the past’ in (\ref{Example_10.16}).


\begin{styleExampleTitle}
Introducing temporal starting points
\end{styleExampleTitle}

\ea
\label{Example_10.15}
\gll {tra} {{bole}} {tutup} {pintu,} {\bluebold{dari}} {\bluebold{pagi}} {buka} {pintu}\\ %
 \textsc{neg}  {permitted}  close  door  from  morning  open  door\\
\gll {sampe}  {malam}\\
 {until}  {night}\\
\glt 
‘you shouldn’t close the door, (you should keep it) open \bluebold{from morning} until night’ \textstyleExampleSource{[081110-008-CvNP.0108]}
\z

\ea
\label{Example_10.16}
\gll {jadi} {itu} {suda} {kebiasaang} {\bluebold{dari}} {\bluebold{dulu}}\\ %
 so  \textsc{d.dist}  already  habit  from  first\\
\glt 
‘so that (tradition) has already become a custom \bluebold{from the past}’ \textstyleExampleSource{[081014-007-CvEx.0063]}
\z


Finally, elative \textitbf{dari} ‘from’ is also used in comparative constructions marking degree or identity. In such constructions, \textitbf{dari} ‘from’ functions as the mark of comparison which introduces the standard. In (\ref{Example_10.17}), for instance, \textitbf{dari} ‘from’ serves as the mark in a comparative construction marking degree, namely superiority, while in (\ref{Example_10.18}) it serves as the mark in a comparative construction marking identity, namely dis\isi{similarity} (for details on comparative constructions, see §\ref{Para_11.5}).


\begin{styleExampleTitle}
Introducing standards of comparison
\end{styleExampleTitle}

\ea
\label{Example_10.17}
\gll {\ldots} {dia} {lebi} {besar} {\bluebold{dari}} {\bluebold{smua}} {\bluebold{ana{\Tilde}ana}} {\ldots}\\ %
 {}   \textsc{3sg}  more  be.big  from  all  \textsc{rdp}{\Tilde}child  \\
\glt 
‘[in that class] he’s bigger \bluebold{than all the kids} [in it]’ \textstyleExampleSource{[081109-003-JR.0001]}\footnote{The original recording says \textitbf{dari smuat} rather than \textitbf{dari smua} ‘than all’. Most likely the speaker wanted to say \textitbf{dari smua temang} ‘than all friends’ but cut himself off to replace \textitbf{temang} ‘friend’ with \textitbf{ana{\Tilde}ana} ‘children’.}
\z

\ea
\label{Example_10.18}
\gll {sifat} {ini} {\bluebold{laing}} {\bluebold{dari}} {\bluebold{ko}}\\ %
 nature  \textsc{d.prox}  be.different  from  \textsc{2sg}\\
\glt
‘this disposition is different \bluebold{from you}’ \textstyleExampleSource{[081110-008-CvNP.0089]}
\z

\subsection{\textitbf{sampe} ‘until’}
\label{Para_10.1.4}
The \isi{preposition} \textitbf{sampe} ‘until’ introduces nonspatial temporal endpoints which are always encoded by peripheral adjuncts. Given these semantics, \textitbf{sampe} ‘until’ typically introduces nouns that indicate time, as in \textitbf{sampe sore} ‘until the afternoon’ in (\ref{Example_10.19}); that is, animate or pronominal referents of \textitbf{sampe} ‘until’ are unattested.


\begin{styleExampleTitle}
Introducing time-denoting nouns
\end{styleExampleTitle}

\ea
\label{Example_10.19}
\gll {saya} {tidor} {\bluebold{sampe}} {\bluebold{sore}}\\ %
 \textsc{1sg}  sleep  until  afternoon\\
\glt 
‘I slept \bluebold{until the afternoon}’ \textstyleExampleSource{[081015-005-NP.0033]}
\z


Typically, peripheral prepositional phrases can be moved to other positions within the clause with no change in meaning. This does not, however, apply to the example in (\ref{Example_10.19}). When the \isi{prepositional phrase} is moved to the front it denotes the temporal starting rather than the temporal endpoint of \textitbf{tidor} ‘sleep’, as in (\ref{Example_10.20}). Hence, the meaning changes to ‘come afternoon’ (literally ‘reaches the afternoon’). One initial explanation for this change in meaning is that the utterance in (\ref{Example_10.20}) expresses a sequence of two events, namely the \textitbf{sampe} ‘reaching’ of the afternoon and subsequently the \textitbf{tidor} ‘sleeping’. In that case, \textitbf{sampe sore} does not express the \isi{prepositional phrase} ‘until afternoon’ but the \isi{verbal clause} ‘reached the afternoon’ or ‘come afternoon’. This explanation, however, requires further investigation.


\begin{styleExampleTitle}
Clause-initial position
\end{styleExampleTitle}

\ea
\label{Example_10.20}
\gll {\bluebold{sampe}} {\bluebold{sore}} {saya} {tidor}\\ %
 reach  afternoon  \textsc{1sg}  sleep\\
\glt 
‘\bluebold{come afternoon} I slept’ (Lit. ‘\bluebold{reach the afternoon}’) \textstyleExampleSource{[Elicited BR120817.008]}
\z


Temporal \textitbf{sampe} ‘until’ also introduces temporal adverbs that denote a temporal endpoint as in \textitbf{sampe skarang} ‘until now’ in (\ref{Example_10.21}). Overall, however, these constructions are very rare in the corpus.


\begin{styleExampleTitle}
Introducing temporal adverbs
\end{styleExampleTitle}

\ea
\label{Example_10.21}
\gll {\ldots} {tapi} {\bluebold{sampe}} {\bluebold{skarang}} {blum} {brangkat}\\ %
{}   but  until  now  not.yet  leave\\
\glt 
‘{\ldots} but \bluebold{until now} (the team) hasn’t yet left’ \textstyleExampleSource{[081023-002-Cv.0001]}
\z


The \isi{preposition} \textitbf{sampe} ‘until’ has trial \isi{word class membership}. Besides introducing prepositional phrases, \textitbf{sampe} is also used as the \isi{bivalent} \isi{verb} ‘reach’, or as an anteriority-marking \isi{conjunction} that introduces temporal clauses (see §\ref{Para_14.2.3.3} for its uses as a \isi{conjunction}; see also §\ref{Para_5.14}).


\subsection{Elision of prepositions encoding location}
\label{Para_10.1.5}
Two of the prepositions of location may be omitted if the semantic relationship between the complement and the predicate can be deduced from the context. The prepositions are \isi{locative} \textitbf{di} ‘at, in’, as illustrated with the contrastive examples in (\ref{Example_10.22}) and (\ref{Example_10.23}), and allative \textitbf{ke} ‘to’, as shown in (\ref{Example_10.24}) and (\ref{Example_10.25}).



When \isi{locative} \textitbf{di} ‘at, in’ introduces a spatial location and combines with a position \isi{verb} such as \textitbf{tidor} ‘sleep’, as in (\ref{Example_10.22}) and (\ref{Example_10.23}), the \isi{preposition} can be elided. Both the preceding \isi{verb} and the complement of \textitbf{di} ‘at, in’ are already deictic and therefore allow the \isi{elision} of \textitbf{di} ‘at, in’: the position \isi{verb} \textitbf{tidor} ‘sleep’ implies the notion of static location, while the complement \textitbf{sana} ‘over there’ signals the position of the location.


\begin{styleExampleTitle}
Prepositional phrases with elided \isi{locative} \textitbf{di} ‘at, in’
\end{styleExampleTitle}
%\todo{This should be in langinfo - please clarify?}
\ea
\label{Example_10.22}
\gll {ko} {punya} {mama} {ada} {tidor} {\bluebold{di}} {\bluebold{sana}}\\ %
 \textsc{2sg}  \textsc{poss}  mother  exist  sleep  at  \textsc{l.dist}\\
\glt 
‘your mother is sleeping \bluebold{over there}’ \textstyleExampleSource{[081006-025-CvEx.0007]}
\z

\ea
\label{Example_10.23}
\gll {a,} {omong} {kosong,} {ko} {masuk} {tidor} {\bluebold{Ø}} {\bluebold{sana}} {suda}\\ %
 ah!  way.of.talking  be.empty  \textsc{2sg}  enter  sleep  {}  \textsc{l.dist}  already\\
\glt 
‘ah, nonsense, you just go inside (and) \bluebold{sleep over there}’ \textstyleExampleSource{[081023-001-Cv.0057]}
\z


Along similar lines allative \textitbf{ke} ‘to’ can be omitted, when the \isi{preposition} introduces a location and combines with a motion \isi{verb} that also expresses direction such as \textitbf{masuk} ‘enter’ in (\ref{Example_10.24}) and (\ref{Example_10.25}). Again, both the \isi{verb} and the complement of \textitbf{ke} ‘to’ are deictic, thereby allowing the \isi{elision} of \textitbf{ke} ‘to’: the \isi{verb} \textitbf{masuk} ‘enter’ implies the notion of motion and direction, while the complement \textitbf{hutang} ‘forest’ denotes the location toward which the motion is directed.


\begin{styleExampleTitle}
Prepositional phrases with elided allative \textitbf{ke} ‘to’
\end{styleExampleTitle}

\ea
\label{Example_10.24}
\gll {smua} {masarakat} {masuk} {\bluebold{ke}} {\bluebold{hutang}}\\ %
 all  community  enter  to  forest\\
\glt 
‘the entire community went \bluebold{into the forest}’ \textstyleExampleSource{[081029-005-Cv.0012]}
\z

\ea
\label{Example_10.25}
\gll {smua} {masuk} {\bluebold{Ø}} {\bluebold{hutang}}\\ %
 all  enter {}   forest\\
\glt 
‘all went \bluebold{(into) the forest}’ \textstyleExampleSource{[081029-005-Cv.0111]}
\z


The \isi{elision} typically affects prepositional phrases with common nouns denoting locations as in (\ref{Example_10.25}), or locatives as in (\ref{Example_10.23}). In addition, the \isi{elision} can also affect prepositional phrases with location nouns as in (\ref{Example_10.26}) and (\ref{Example_10.27}): in (\ref{Example_10.26}) the omitted \isi{preposition} is \isi{locative} \textitbf{di} ‘at, in’, whereas in (\ref{Example_10.27}) it is allative \textitbf{ke} ‘to’.


\begin{styleExampleTitle}
Prepositional phrases with elided \isi{preposition} and location \isi{noun} complement
\end{styleExampleTitle}

\ea
\label{Example_10.26}
\gll {baru} {kitong} {taru} {\bluebold{Ø}} {\bluebold{depang}} {to?}\\ %
 and.then  \textsc{1pl}  put  {}  front  right?\\
\glt 
‘and then we put (the cake down) \bluebold{(in) front}, right?’ \textstyleExampleSource{[081011-005-Cv.0031]}
\z

\ea
\label{Example_10.27}
\gll {itu} {yang} {sa} {bilang,} {kalo} {dong} {pinda} {\bluebold{Ø}} {\bluebold{sebla}} {bole}\\ %
 \textsc{d.dist}  \textsc{rel}  \textsc{1sg}  say  if  \textsc{3pl}  move {}   side  may\\
\glt 
‘that’s why I said, ``if they move \bluebold{(to) the (other) side} (that’s) alright''' \textstyleExampleSource{[081011-001-Cv.0144]}
\z


Elision of \textitbf{di} ‘at, in’ and \textitbf{ke} ‘to’ is not possible, though, in nonverbal prepositional predicate clauses as this would create nominal clauses with unacceptable semantics. This is illustrated with elided \textitbf{di} ‘at, in’ in (\ref{Example_10.28}), which is based on the example in (\ref{Example_10.4}), and with elided \textitbf{ke} ‘to’ in (\ref{Example_10.29}), which is based on the example in (\ref{Example_10.8}).


\begin{styleExampleTitle}
Nonverbal prepositional predicate clauses with elided \isi{locative} \textitbf{di} ‘at, in’ and allative \textitbf{ke} ‘to’
\end{styleExampleTitle}

\ea
\label{Example_10.28}
\gll {*} {sa} {\bluebold{Ø}} {\bluebold{IPS}} {\bluebold{satu}}\\ %
 {}  \textsc{1sg} {}   social.sciences  one\\
\glt 
[About course tracks in high school:] (‘I (am) \bluebold{Social Sciences I}’) \textstyleExampleSource{[based on 081023-004-Cv.0020]}
\z

\ea
\label{Example_10.29}
\gll {*} {sa} {\bluebold{Ø}} {\bluebold{ruma-sakit}}\\ %
 {}   \textsc{1sg} {}   hospital\\
\glt 
(‘I (am) \bluebold{the hospital}’) \textstyleExampleSource{[based on 081015-005-NP.0047]}
\z


Elision of elative \textitbf{dari} ‘from’ and temporal \textitbf{sampe} ‘until’ is also not possible, as illustrated in (\ref{Example_10.30}) and (\ref{Example_10.31}). In the example in (\ref{Example_10.30}), which is based on (\ref{Example_10.10}), elative \textitbf{dari} ‘from’ is omitted, resulting in an ungrammatical utterance. In the example in (\ref{Example_10.31}), which is based on (\ref{Example_10.19}), temporal \textitbf{sampe} ‘until’ is elided. The result is a change in meaning of the entire utterance: ‘I slept (the entire) afternoon’.


\begin{styleExampleTitle}
Prepositional phrases with elided elative \textitbf{dari} ‘from’ and temporal \textitbf{sampe} ‘until’
\end{styleExampleTitle}

\ea
\label{Example_10.30}
\gll {*} {\ldots} {sa} {harus} {kluar} {\bluebold{Ø}} {\bluebold{kam}} {\bluebold{pu}} {\bluebold{kluarga}}\\ %
  {} {}   \textsc{1sg}  have.to  go.out {}   \textsc{2pl}  \textsc{poss}  family\\
\glt 
(‘[I hadn’t thought that] I would have to depart \bluebold{your family}’) \textstyleExampleSource{[Elicited BR120817.009]}
\z

\ea
\label{Example_10.31}
\gll {saya} {tidor} {\bluebold{Ø}} {\bluebold{sore}}\\ %
 \textsc{1sg}  sleep  {}   afternoon\\
\glt
‘I slept (the entire) \bluebold{afternoon}’ \textstyleExampleSource{[Elicited BR120817.010]}
\z

\section{Prepositions encoding {accompaniment}/instruments, goals, and benefaction}
\label{Para_10.2}
Papuan Malay employs four prepositions encoding \isi{accompaniment}/instruments, goals, and \isi{benefaction}: \isi{comitative} \textitbf{dengang} ‘with’ (§\ref{Para_10.2.1}), \isi{goal-oriented} \textitbf{sama} ‘to’ (§\ref{Para_10.2.2}), benefactive \textitbf{untuk} ‘for’ (§\ref{Para_10.2.3}), and \textitbf{buat} ‘for’ (§\ref{Para_10.2.4}).


\subsection{\textitbf{dengang} ‘with’}
\label{Para_10.2.1}
Prepositional phrases introduced with \isi{comitative} \textitbf{dengang} ‘with’, with its short form \textitbf{deng}, typically express \isi{accompaniment} with animate or inanimate associates. Also very often, \textitbf{dengang} ‘with’ introduces instruments. In addition, \textitbf{dengang} ‘with’ introduces objects of mental verbs and the notion of identity in comparison constructions.



The associates introduced with \textitbf{dengang} ‘with’ are most commonly animate human as in \textitbf{deng mama-tua} ‘with aunt’ in (\ref{Example_10.32}), \textitbf{deng de pu temang{\Tilde}temang} ‘with his friends’ in (\ref{Example_10.33}), or in \textitbf{deng kamu} ‘with you’ in (\ref{Example_10.34}). These examples also show that the complements of \textitbf{dengang} ‘with’ can be nouns or personal pronouns. Besides animate associates, \textitbf{dengang} ‘with’ also introduces inanimate associates, as in \textitbf{deng motor} ‘with (his) motorbike’ in (\ref{Example_10.35}), or in \textitbf{deng} \textitbf{itu} ‘with those (spices)’ in (\ref{Example_10.36}). The associates introduced with \textitbf{dengang} ‘with’ are either encoded in peripheral adjuncts as in (\ref{Example_10.32}), or (\ref{Example_10.34}) to (\ref{Example_10.36}), or in nonverbal predicates as in (\ref{Example_10.33}). The example in (\ref{Example_10.33}) also illustrates that prepositional phrases functioning as nonverbal predicates can be modified by adverbs, such as prospective \textitbf{masi} ‘still’; such \isi{modification} is unattested for prepositional phrases having other functions.


\begin{styleExampleTitle}
Introducing associates
\end{styleExampleTitle}

\ea
\label{Example_10.32}
\gll {sebentar} {Hurki} {datang} {ko} {pulang} {\bluebold{deng}} {\bluebold{mama-tua}}\\ %
 in.a.moment  Hurki  come  \textsc{2sg}  go.home  with  aunt\\
\glt 
‘in a moment (when) Hurki comes, you’ll go home \bluebold{with me} (‘\bluebold{aunt}’)’ \textstyleExampleSource{[081011-006-Cv.0003]}
\z

\ea
\label{Example_10.33}
\gll {Roni} {masi} {\bluebold{deng}} {\bluebold{de}} {\bluebold{pu}} {\bluebold{temang{\Tilde}temang}}\\ %
 Roni  still  with  \textsc{3sg}  \textsc{poss}  \textsc{rdp}{\Tilde}friend\\
\glt 
‘Roni is still \bluebold{with his friends}’ \textstyleExampleSource{[081006-031-Cv.0011]}
\z

\ea
\label{Example_10.34}
\gll {slama} {sa} {tinggal} {\bluebold{deng}} {\bluebold{kamu}} {sa} {kerja}\\ %
 as.long.as  \textsc{1sg}  stay  with  \textsc{2pl}  \textsc{1sg}  work\\
\glt 
‘as long as I stayed \bluebold{with you} I worked’ \textstyleExampleSource{[080919-006-CvNP.0014]}
\z

\ea
\label{Example_10.35}
\gll {de} {jatu} {\bluebold{deng}} {\bluebold{motor}}\\ %
 \textsc{3sg}  fall  with  motorbike\\
\glt 
‘he fell \bluebold{with (his) motorbike}’ \textstyleExampleSource{[081006-020-Cv.0008]}
\z

\ea
\label{Example_10.36}
\gll {itu} {nanti} {kitong} {tumbuk} {baru} {masak} {\bluebold{deng}} {\bluebold{itu}}\\ %
 \textsc{d.dist}  very.soon  \textsc{1pl}  pound  and.then  cook  with  \textsc{d.dist}\\
\glt 
‘later we’ll pound those (spices and) and then cook \bluebold{with them}’ \textstyleExampleSource{[081010-001-Cv.0196]}
\z


Instruments introduced with \isi{comitative} \textitbf{dengang} ‘with’ are expressed in peripheral adjuncts as in \textitbf{deng pisow} ‘with a knife’ in (\ref{Example_10.37}).


\begin{styleExampleTitle}
Introducing instruments
\end{styleExampleTitle}

\ea
\label{Example_10.37}
\gll {bapa} {de} {pukul} {sa} {\bluebold{deng}} {\bluebold{pisow}}\\ %
 father  \textsc{3sg}  hit  \textsc{1sg}  with  knife\\
\glt 
‘(my) husband stabbed me \bluebold{with a knife}’ \textstyleExampleSource{[081011-023-Cv.0167]}
\z


In addition, \isi{comitative} \textitbf{dengang} ‘with’ introduces oblique arguments for mental verbs such as \textitbf{mara} ‘feel angry (about)’ in (\ref{Example_10.38}), \textitbf{takut} ‘feel afraid (of)’ in (\ref{Example_10.39}), or \textitbf{perlu} ‘need’ in (\ref{Example_10.40}).\footnote{Bivalent verbs such as \textitbf{mara} ‘feel angry (about)’ or \textitbf{takut} ‘feel afraid (of)’ do not require but allow two syntactic arguments (see §\ref{Para_5.3.1} and §\ref{Para_11.1}). That is, speakers quite commonly encode patients, such as \textitbf{orang} in (\ref{Example_10.38}) or \textitbf{setang} ‘evil spirit’ in (\ref{Example_10.39}), as oblique arguments rather than as direct objects.}


\begin{styleExampleTitle}
Introducing objects of mental verbs
\end{styleExampleTitle}

\ea
\label{Example_10.38}
\gll {kalo} {saya} {mara} {\bluebold{dengang}} {\bluebold{orang}} {begitu} {sa} {takut}\\ %
 if  \textsc{1sg}  feel.angry(.about)  with  person  like.that  \textsc{1sg}  feel.afraid(.of)\\
\glt 
‘if I was angry \bluebold{with someone} like that I’d feel afraid’ \textstyleExampleSource{[081110-008-CvNP.0067]}
\z

\ea
\label{Example_10.39}
\gll {adu,} {kang} {dong} {terlalu} {takut} {\bluebold{dengang}} {\bluebold{setang}}\\ %
 oh.no!  you.know  \textsc{3pl}  too  feel.afraid(.of)  with  evil.spirit\\
\glt 
‘oh no!, you know, they feel too afraid \bluebold{of evil spirits}’ \textstyleExampleSource{[081025-006-Cv.0198]}
\z

\ea
\label{Example_10.40}
\gll {mama-ade} {sa} {perlu} {\bluebold{deng}} {\bluebold{mama-ade}}\\ %
 aunt  \textsc{1sg}  need  with  aunt\\
\glt 
‘aunt, I need \bluebold{you} (‘\bluebold{aunt}’)’ (Lit. ‘need \bluebold{with aunt}’) \textstyleExampleSource{[081014-004-Cv.0004]}
\z


Comitative \textitbf{dengang} ‘with’ is also used in comparative constructions. As the mark of comparison, \textitbf{dengang} ‘with’ introduces the standard of comparison in \isi{identity-marking} constructions. In (\ref{Example_10.41}), for example, \textitbf{dengang} ‘with’ serves as the mark in a \isi{similarity} construction, while in (\ref{Example_10.42}) it is the mark in a dis\isi{similarity} construction (for more details on comparative constructions, see §\ref{Para_11.5}).


\begin{styleExampleTitle}
Introducing standards of comparison
\end{styleExampleTitle}

\ea
\label{Example_10.41}
\gll {de} {sombong} {sama} {\bluebold{deng}} {\bluebold{ko}}\\ %
 \textsc{3sg}  be.arrogant  same  with  \textsc{2sg}\\
\glt 
‘she’ll be as arrogant \bluebold{as you} (are)’ \textstyleExampleSource{[081006-005-Cv.0002]}
\z

\ea
\label{Example_10.42}
\gll {orang} {Papua} {beda} {\bluebold{dengang}} {\bluebold{orang}} {\bluebold{Indonesia}}\\ %
 person  Papua  be.different  with  person  Indonesia\\
\glt 
‘Papuans are different \bluebold{from Indonesians}’ \textstyleExampleSource{[081029-002-Cv.0009]}
\z


The \isi{preposition} \textitbf{dengang} ‘with’ has dual \isi{word class membership}; it is also used as an addition-marking \isi{conjunction} (§\ref{Para_14.2.1.1}; see also §\ref{Para_5.14}).


\subsection{\textitbf{sama} ‘to’}
\label{Para_10.2.2}
The goal \isi{preposition} \textitbf{sama} ‘to’ is rather general in its meaning. Typically, it translates with ‘to’ but depending on its context it also translates with ‘of, from, with’. The complement always denotes an animate referent which can be encoded in a \isi{noun} or in a personal \isi{pronoun}.



As the exchange in (\ref{Example_10.43}) shows, \textitbf{sama} ‘to’ usually introduces oblique goal or recipient arguments of transfer verbs, such as \textitbf{bawa} ‘bring’ in (\ref{Example_10.43a}) or \textitbf{kasi} ‘give’ in (\ref{Example_10.43b}).


\begin{styleExampleTitle}
Introducing goals or recipients
\end{styleExampleTitle}

\ea
\label{Example_10.43}

\ea
\label{Example_10.43a}
\gll {Speaker-1:} {ko} {bawa} {ke} {sana} {ko} {\bluebold{bawa}} {\bluebold{sama}} {\bluebold{ade}}\\ %
  {}   \textsc{2sg}  take  to  \textsc{l.dist}  \textsc{2sg}  take  to  ySb\\
\glt Speaker-1: ‘bring (the ball) over there, \bluebold{bring} (it) \bluebold{to (your) younger cousin}’\\
\vspace{5pt}

\ex
\label{Example_10.43b}
\gll Speaker-2:  e,  \bluebold{kasi}  bola  \bluebold{sama}  \bluebold{ade}\\
{}     hey!  give  ball  to  ySb\\
\glt Speaker-2: ‘hey, \bluebold{give} the ball \bluebold{to (your) younger cousin}’ \textstyleExampleSource{[081011-009-Cv.0015-0016]}
\z
\z


Also very commonly, \textitbf{sama} ‘to’ introduces oblique addressee arguments for communication verbs, such as \textitbf{bicara} ‘speak’ in (\ref{Example_10.44}) or \textitbf{minta} ‘request’ in (\ref{Example_10.45}).


\begin{styleExampleTitle}
Introducing addressees
\end{styleExampleTitle}

\ea
\label{Example_10.44}
\gll {sa} {minta} {maaf,} {e} {tadi} {sa} {\bluebold{bicara}} {kasar} {\bluebold{sama}} {\bluebold{ko}}\\ %
 \textsc{1sg}  ask  pardon  uh  earlier  \textsc{1sg}  speak  be.coarse  to  \textsc{2sg}\\
\glt 
‘I apologize, uh, a short while ago I \bluebold{spoke to you} harshly’ \textstyleExampleSource{[081115-001a-Cv.0277]}
\z

\ea
\label{Example_10.45}
\gll {de} {\bluebold{minta}} {apa} {\bluebold{sama}} {\bluebold{kitorang}} {kitorang} {kasi}\\ %
 \textsc{3sg}  ask  what  to  \textsc{1pl}  \textsc{1pl}  give\\
\glt 
‘(whenever) she (our daughter) \bluebold{asks us} (for) something, we give (it to her)’ \textstyleExampleSource{[081006-025-CvEx.0022]}
\z


Goal \isi{preposition} \textitbf{sama} ‘to’ denotes the goal of a transfer or communication without concurrently marking this goal as the beneficiary of the event talked about. In this it contrasts with benefactive \textitbf{untuk} ‘for’ and \textitbf{buat} ‘for’; compare the examples in (\ref{Example_10.43b}) and (\ref{Example_10.44}) with \textitbf{kasi}/\textitbf{bicara untuk} ‘give/speak to and for’ in (\ref{Example_10.55}) and (\ref{Example_10.56}) in §\ref{Para_10.2.3} (p. \pageref{Example_10.56}) and with \textitbf{kasi}/\textitbf{bicara buat} ‘give/speak to and for’ in (\ref{Example_10.65}) and (\ref{Example_10.66}) in §\ref{Para_10.2.4} (p. \pageref{Example_10.65}).



In addition, \textitbf{sama} ‘to’ introduces oblique arguments of mental verbs, such as \textitbf{ingat} ‘remember’ in (\ref{Example_10.46}), \textitbf{mara} ‘feel angry (about)’ in (\ref{Example_10.47}), or \textitbf{takut} ‘feel afraid (of)’ in (\ref{Example_10.48}). Most of the objects of mental verbs introduced with \textitbf{sama} ‘to’ can also occur with \isi{comitative} \textitbf{dengang} ‘with’ (§\ref{Para_10.2.1}): compare \textitbf{mara sama} ‘feel angry about’ in (\ref{Example_10.47}) with \textitbf{mara dengang} ‘feel angry with’ in (\ref{Example_10.38}), or \textitbf{takut sama} ‘feel afraid of’ in (\ref{Example_10.48}) with \textitbf{takut dengang} ‘feel afraid of’ in (\ref{Example_10.39})\addtocounter{footnote}{-1}\footnotemark{}.

%\footnote{Bivalent verbs such as \textitbf{mara} ‘feel angry (about)’ or \textitbf{takut} ‘feel afraid (of)’ do not require but allow two syntactic arguments (see §\ref{Para_5.3.1} and §\ref{Para_11.1}). That is, speakers quite commonly encode patients, such as \textitbf{orang} in (\ref{Example_10.38}) or \textitbf{setang} ‘evil spirit’ in (\ref{Example_10.39}), as oblique arguments rather than as direct objects}\todo[inline]{This is footnote 223 in Word, and is referenced twice}.
 Overall, however, the range of verbs is smaller for \textitbf{sama} ‘to’ than for \isi{comitative} \textitbf{dengang} ‘with’.



The semantic distinctions between \textitbf{sama} ‘to’ and \textitbf{dengang} ‘with’ are subtle. When speakers want to emphasize the agent of the mental \isi{verb} they employ \textitbf{sama} ‘to’. If they want to signal that the object of the mental \isi{verb} is also involved in the mental process talked about, they use \isi{comitative} \textitbf{dengang} ‘with’. The contrastive examples in (\ref{Example_10.47}) and (\ref{Example_10.48}) illustrate this distinction. In (\ref{Example_10.47a}) \textitbf{sama} ‘to’ emphasizes the fact that the agent \textitbf{de} ‘3\textsc{sg}’ \textitbf{mara} ‘feels angry’ about the patient \textitbf{pak Bolikarfus} ‘Mr. Bolikarfus’ whereas the patient himself is not involved in this mental process. By contrast, in (\ref{Example_10.47b}) \textitbf{deng(ang)} ‘with’ signals that in some ways the patient \textitbf{pak Bolikarfus} ‘Mr. Bolikarfus’ has contributed to the agent’s anger. Likewise, in (\ref{Example_10.48a}) \textitbf{sama} ‘to’ focuses on the fact that the agent \textitbf{dia} ‘3\textsc{sg}’ \textitbf{takut} ‘feels afraid (of)’; again, the patient \textitbf{ana{\Tilde}ana Tuhang} ‘God’s children’ is not involved in this mental process. In (\ref{Example_10.48b}), by contrast, \textitbf{deng(ang)} ‘with’ signals that the patient \textitbf{ana{\Tilde}ana Tuhang} ‘God’s children’ has contributed in some ways to the agent’s fear.

\newpage

\begin{styleExampleTitle}
Introducing objects of mental verbs\footnote{The examples in (\ref{Example_10.47a}) and (\ref{Example_10.48a}) are taken from the corpus while the examples in (\ref{Example_10.47b}) and (\ref{Example_10.48b}) are elicited.}
\end{styleExampleTitle}

\ea
\label{Example_10.46}
\gll {biar} {dia} {masi} {muda} {tapi} {Fitri} {ingat} {\bluebold{sama}} {\bluebold{Roni}}\\ %
 although  \textsc{3sg}  still  be.young  but  Fitri  remember  to  Roni\\
\glt 
‘even though she was still young, Fitri was thinking \bluebold{of Roni}’ \textstyleExampleSource{[081006-024-CvEx.0067]}
\z

\ea
\label{Example_10.47}
\ea
\label{Example_10.47a}
\gll  {de} {mara} {\bluebold{sama}} {\bluebold{pak}} {\bluebold{Bolikarfus}}\\ %
   \textsc{3sg}  feel.angry(.about)  to  father  Bolikarfus\\
\glt ‘he was angry \bluebold{about Mr. Bolikarfus}’ \textstyleExampleSource{[081014-016-Cv.0042]}
\vspace{5pt}

\ex
\label{Example_10.47b}
\gll  de  mara  \bluebold{deng}  \bluebold{pak}  \bluebold{Bolikarfus}\\
   \textsc{3sg}  feel.angry(.about)  with  father  Bolikarfus\\
\glt ‘he was angry \bluebold{with Mr. Bolikarfus}’ \textstyleExampleSource{[Elicited BR120817.001]}
\z
\z


\ea
\label{Example_10.48}
\ea
\label{Example_10.48a}
\gll  {memang} {dia} {takut} {\bluebold{sama}} {\bluebold{ana{\Tilde}ana}} {\bluebold{Tuhang}}\\ %
   indeed  \textsc{3sg}  feel.afraid(.of)  to  \textsc{rdp}{\Tilde}child  God\\
\glt ‘(that evil spirit) indeed he/she feels afraid \bluebold{of God’s children}’ \textstyleExampleSource{[081006-022-CvEx.0175]}
\vspace{5pt}

\ex
\label{Example_10.48b}
\gll  memang  dia  takut  \bluebold{deng}  \bluebold{ana{\Tilde}ana}  \bluebold{Tuhang}\\
   indeed  \textsc{3sg}  feel.afraid(.of)  with  \textsc{rdp}{\Tilde}child  God\\
\glt ‘(that evil spirit) indeed he/she feels afraid \bluebold{of God’s children}’ \textstyleExampleSource{[Elicited BR120817.001]}
\z
\z



Furthermore, although not very frequently, \textitbf{sama} ‘to’ introduces animate associates. As with \isi{comitative} \textitbf{dengang} ‘with’ (§\ref{Para_10.2.1}), associates are expressed in peripheral adjuncts as in \textitbf{sama dorang} ‘with them’ in (\ref{Example_10.49}), or in nonverbal predicates as in \textitbf{sama saya} ‘with me’ in (\ref{Example_10.50}).


\begin{styleExampleTitle}
Introducing animate associates
\end{styleExampleTitle}

\ea
\label{Example_10.49}
\gll {Papeas} {maing{\Tilde}maing} {\bluebold{sama}} {\bluebold{dorang}}\\ %
 Papeas  \textsc{rdp}{\Tilde}play  to  \textsc{3pl}\\
\glt 
‘Papeas is going to play \bluebold{with them}’ \textstyleExampleSource{[080918-001-CvNP.0040]}
\z

\ea
\label{Example_10.50}
\gll {hanya} {tiga} {saja} {\bluebold{sama}} {\bluebold{saya}}\\ %
 only  three  just  to  \textsc{1sg}\\
\glt 
‘just only three (of my children) are \bluebold{with me}’ \textstyleExampleSource{[081006-024-CvEx.0001]}
\z


The goal \isi{preposition} \textitbf{sama} ‘to’ has trial \isi{word class membership}. That is, besides being used as a \isi{preposition}, it is also used as the stative \isi{verb} \textitbf{sama} ‘be same’ and, although not very frequently, as an addition-marking \isi{conjunction} (see §\ref{Para_5.14}; see also §\ref{Para_14.2.1.3} for its uses as a \isi{conjunction}).\footnote{\label{Footnote_10.225} In terms of its etymology, U. Tadmor (p.c. 2013) notes that “\textitbf{sama}\textit{ }was borrowed from \ili{Sanskrit} into Malay in ancient times with the meaning ‘same’. Much later it also came to mean ‘with’ in \ili{Bazaar Malay}”.}


\subsection{\textitbf{untuk} ‘for’}
\label{Para_10.2.3}
The benefactive \isi{preposition} \textitbf{untuk} usually translates with ‘for’; depending on its context, however, it also translates with ‘to, about’. The \isi{preposition} introduces animate and inanimate, as well as nominal and pronominal referents. In most cases, the referents are beneficiaries or recipients (148 tokens). In this regard, \textitbf{untuk} ‘for’ is similar to benefactive \textitbf{buat} ‘for’ (§\ref{Para_10.2.4}). Contrasting with \textitbf{buat} ‘for’, however, \textitbf{untuk} ‘for’ has a wider distribution and more functions in that it (1) combines with demonstratives, (2) introduces inanimate referents, and (3) introduces circumstance.



Beneficiaries introduced with \textitbf{untuk} ‘for’ are typically animate human, as in (\ref{Example_10.51}), (\ref{Example_10.53}) or (\ref{Example_10.54}). The beneficiary can, however, also be animate nonhuman, as in \textitbf{untuk anjing dorang} ‘for the dogs’ in (\ref{Example_10.52}).



Usually, \textitbf{untuk} ‘for’ follows \isi{bivalent} verbs such as \textitbf{buat} ‘make, do’ or \textitbf{biking} ‘make’, and introduces beneficiaries encoded by peripheral adjuncts, as in (\ref{Example_10.51}) or (\ref{Example_10.52}), respectively. Only rarely is the beneficiary encoded by a nonverbal prepositional predicate (2 tokens) as in \textitbf{untuk tamu} ‘for the guests’ in (\ref{Example_10.53}), or an adnominal \isi{prepositional phrase} (2 tokens) as in \textitbf{untuk kafir} ‘for unbelievers’ in (\ref{Example_10.54}). As for the low token frequencies of two each, one consultant suggested that these constructions are not native Papuan Malay but represent instances of code-switching with Indonesian. The low frequencies support this statement.


\begin{styleExampleTitle}
Introducing animate beneficiaries
\end{styleExampleTitle}

\ea
\label{Example_10.51}
\gll {Tuhang} {buat} {mujisat} {\bluebold{untuk}} {\bluebold{kita}}\\ %
 God  make  miracle  for  \textsc{1pl}\\
\glt 
‘God made a miracle \bluebold{for us}’ \textstyleExampleSource{[080917-008-NP.0163]}
\z

\ea
\label{Example_10.52}
\gll {\ldots} {yang} {sa} {pu} {bini} {biking} {malam} {\bluebold{untuk}} {\bluebold{anjing}} {\bluebold{dorang}}\\ %
{}    \textsc{rel}  \textsc{1sg}  \textsc{poss}  wife  make  night  for  dog  \textsc{3pl}\\
\glt 
‘[I fed the dogs with papeda] which my wife had made in the evening \bluebold{for the dogs}’ \textstyleExampleSource{[080919-003-NP.0002]}
\z

\ea
\label{Example_10.53}
\gll {ikang} {sedikit,} {itu} {\bluebold{untuk}} {\bluebold{tamu}}\\ %
 fish  few  \textsc{d.dist}  for  guest\\
\glt 
‘(as for) the few fish, those are \bluebold{for the guests}’ \textstyleExampleSource{[081014-011-CvEx.0008]}
\z

\ea
\label{Example_10.54}
\gll {di} {sana} {kang} {masi} {\bluebold{tempat}} {\bluebold{untuk}} {\bluebold{kafir}}\\ %
 at  \textsc{l.dist}  you.know  still  place  for  unbeliever\\
\glt 
‘(the area) over there, you know, is still a location \bluebold{for unbelievers}’ \textstyleExampleSource{[081011-022-Cv.0238]}
\z


With transfer verbs, \textitbf{untuk} ‘for’ introduces benefactive recipients, and with communication verbs it introduces benefactive addressees. That is, the referent is not merely a recipient or addressee. Benefactive \textitbf{untuk} ‘for’ indicates that the referent is also the beneficiary of the transfer or communication, hence ``benefactive recipient'' and ``benefactive addressee''. This is illustrated with \textitbf{kasi untuk} ‘give to and for’ in (\ref{Example_10.55}), and \textitbf{bicara untuk} ‘speak to and for’ in (\ref{Example_10.56}).


\begin{styleExampleTitle}
Introducing benefactive recipients and addressees
\end{styleExampleTitle}

\ea
\label{Example_10.55}
\gll {sa} {kasi} {hadia} {\bluebold{untuk}} {\bluebold{kamu}}\\ %
 \textsc{1sg}  give  gift  for  \textsc{2pl}\\
\glt 
‘I’ll give gifts \bluebold{to you for your benefit}’ \textstyleExampleSource{[080922-001a-CvPh.1332]}
\z

\ea
\label{Example_10.56}
\gll {jadi} {{sperti}} {itu,} {harus} {bicara} {\bluebold{untuk}} {\bluebold{dorang},} {ceritra}\\ %
 so  {similar.to}  \textsc{d.dist}  have.to  speak  for  \textsc{3pl}  tell\\
 \gll {\bluebold{untuk}}  {\bluebold{dorang}}\\
 {for}  {\textsc{3pl}}\\
\glt 
‘so it’s like that, (we) have to speak \bluebold{to them} (our children) \bluebold{for their benefit}, talk \bluebold{to them for their benefit}’ \textstyleExampleSource{[081014-007-CvEx.0136]}
\z


Besides introducing animate referents, \textitbf{untuk} ‘for’ also introduces inanimate beneficiaries that are concrete, abstract, or temporal. In (\ref{Example_10.57}), the beneficiary is inanimate concrete: \textitbf{kamar mandi} ‘the bathroom’. In (\ref{Example_10.58}), the beneficiary is inanimate abstract: distal \isi{demonstrative} \textitbf{itu} ‘\textsc{d.dist}’ summarizes the speaker’s previous statements about balanced birth rates across families related by marriage. In (\ref{Example_10.59}) and (\ref{Example_10.60}), the beneficiary is temporal: \textitbf{taung ini} ‘this year’ in (\ref{Example_10.59}) and \textitbf{besok} ‘tomorrow’ in (\ref{Example_10.60}). Overall, however, these uses of benefactive \textitbf{untuk} ‘for’ are quite rare, with the corpus including only very few examples.


\begin{styleExampleTitle}
Introducing inanimate beneficiaries
\end{styleExampleTitle}

\ea
\label{Example_10.57}
\gll {tong} {mo} {pake} {\bluebold{untuk}} {\bluebold{kamar}} {\bluebold{mandi}}\\ %
 \textsc{1pl}  want  use  for  room  bathe\\
\glt 
‘we want to use (the corrugated iron sheets) \bluebold{for the bathroom} (roof)’ \textstyleExampleSource{[080925-003-Cv.0005]}
\z

\ea
\label{Example_10.58}
\gll {\ldots} {{lahir}} {{ana}} {{suku}} {{A.,}} {a,} {{saya}} {{lahir}}\\ %
 {}  {give.birth}  {child}  {ethnic.group}  {A}.  ah!  {\textsc{1sg}}  {give.birth}\\
\gll {suku}  {Y}.  {\ldots}  {tujuangnya}  {hanya}  {\bluebold{untuk}}  \bluebold{itu}\\
 {ethnic.group}  {Y}.  {}  {purpose:\textsc{3possr}}  {only}  {for}  \textsc{d.dist}\\
\glt 
[About the exchange of bride-price children:] ‘(our daughter) will give birth to a child (for) the A. family, well, I give birth for the Y. family {\ldots} its purpose is only \bluebold{for that} (namely,\bluebold{ }a balanced birth rate across families)’ \textstyleExampleSource{[081006-024-CvEx.0079]}
\z

\ea
\label{Example_10.59}
\gll {\bluebold{untuk}} {\bluebold{taung}} {\bluebold{ini}} {kam} {kas} {los} {sa} {dulu}\\ %
 for  year  \textsc{d.prox}  \textsc{2pl}  give  loosen  \textsc{1sg}  first\\
\glt 
‘\bluebold{for (the rest of) this year} you release me (from my duties) for now’ \textstyleExampleSource{[080922-002-Cv.0084]}
\z

\ea
\label{Example_10.60}
\gll {tong} {dari} {sa} {pu} {temang} {pinjam} {trening} {\bluebold{untuk}} {\bluebold{besok}}\\ %
 \textsc{1pl}  from  \textsc{1sg}  \textsc{poss}  friend  borrow  tracksuit  for  tomorrow\\
\glt 
‘we (are back) from my friend (from whom we) borrowed a tracksuit \bluebold{for tomorrow}’ \textstyleExampleSource{[081011-020-Cv.0052]}
\z


In addition, \textitbf{untuk} ‘for’ introduces peripheral adjuncts that express the notion of circumstance, as in \textitbf{untuk seng itu} ‘about those corrugated iron sheets’ in (\ref{Example_10.61}), or \textitbf{untuk masala tahang lapar} ‘about the problem of enduring to be hungry’ in (\ref{Example_10.62}).


\begin{styleExampleTitle}
Introducing circumstance
\end{styleExampleTitle}

\ea
\label{Example_10.61}
\gll {tanya} {Sarles,} {bapa,} {\bluebold{untuk}} {\bluebold{seng}} {\bluebold{itu}}\\ %
 ask  Sarles  father  for  corrugated.iron  \textsc{d.dist}\\
\glt 
‘father, ask Sarles \bluebold{about/for those corrugated iron} (sheets)’ \textstyleExampleSource{[080925-003-Cv.0003]}
\z

\ea
\label{Example_10.62}
\gll {sa} {{bilang,}} {\bluebold{untuk}} {{\bluebold{masala}}} {{\bluebold{tahang}}} {\bluebold{lapar}} {kitong}\\ %
 \textsc{1sg}  {say}  for  {problem}  {hold.(out/back)}  be.hungry  \textsc{1pl}\\
\gll  {bisa}  {tahang}  {lapar}  juga  {e?}\\
 {be.able}  {hold (out/back)}  {be.hungry}  also  {eh}\\
\glt 
‘I say \bluebold{about the problem of enduring to be hungry}, we can also endure being hungry, eh?’ \textstyleExampleSource{[081025-009a-Cv.0118]}
\z


The \isi{preposition} \textitbf{untuk} ‘for’ has dual \isi{word class membership}; it is also used as a \isi{conjunction} that introduces purpose clauses (§\ref{Para_14.2.4.3}; see also §\ref{Para_5.14}).


\subsection{\textitbf{buat} ‘for’}
\label{Para_10.2.4}
The core semantics of the \isi{preposition} \textitbf{buat} ‘for’ are benefactive; that is, it introduces beneficiaries and benefactive recipients. In this, it is similar to benefactive \textitbf{untuk} ‘for’. Otherwise, as already mentioned in §\ref{Para_10.2.3}, \textitbf{buat} ‘for’ is more restricted in its distribution and functions: (1) it is not attested to combine with demonstratives, (2) it only rarely introduces inanimate referents, and (3) it is not attested to introduce other complements such as circumstance.



Most commonly, \textitbf{buat} ‘for’ follows \isi{bivalent} action verbs such as \textitbf{putar} ‘stir’, and introduces peripheral adjuncts denoting human beneficiaries as in \textitbf{buat de bapa} ‘for her father’ in (\ref{Example_10.63}). Considerably less frequently, \textitbf{buat} ‘for’ introduces beneficiaries encoded by adnominal prepositional phrases, as in \textitbf{buat torang} ‘for us’ in the exchange in (\ref{Example_10.64}).


\begin{styleExampleTitle}
Introducing animate beneficiaries
\end{styleExampleTitle}

\ea
\label{Example_10.63}
\gll {Ika} {biking} {papeda} {putar} {\bluebold{buat}} {\bluebold{de}} {\bluebold{bapa}}\\ %
 Ika  make  sagu.porridge  stir  for  \textsc{3sg}  father\\
\glt 
‘Ika made sagu porridge, she stirred (it) \bluebold{for her father}’ \textstyleExampleSource{[081006-032-Cv.0071]}
\z

\ea
\label{Example_10.64}
\ea
\label{Example_10.64a}
\gll Speaker-1:  sa  juga  dengang  ini  kaka  siapa  tu\\ %
   {}   \textsc{1sg}  also  with  \textsc{d.prox}  oSb  who  \textsc{d.dist}\\
\glt Speaker-1: ‘I was also with, what’s-his-name, that older brother, who-is-it?’
\vspace{5pt}

\ex
\label{Example_10.64b}
\gll    Speaker-2:  satpam  \bluebold{buat}  \bluebold{torang}\\
 {}     security.guard  for  \textsc{1pl}\\
\glt Speaker-2: ‘\bluebold{our} security guard’ (Lit. ‘the security guard \bluebold{for us})’ \textstyleExampleSource{[081025-006-Cv.0109]}
\z
\z



Benefactive \textitbf{buat} ‘for’ also introduces benefactive recipients and addressees encoded by oblique arguments, as shown in (\ref{Example_10.65}) and (\ref{Example_10.66}), respectively. Hence, like \textitbf{untuk} ‘for’ (§\ref{Para_10.2.3}), benefactive \textitbf{buat} ‘for’ contrasts with \isi{goal-oriented} \textitbf{sama} ‘to’ (§\ref{Para_10.2.2}), which expresses recipients and addressees, as in (\ref{Example_10.43}) to (\ref{Example_10.45}), without, however, signaling the concurrent notion of beneficiary.


\begin{styleExampleTitle}
Introducing benefactive recipients and addressees
\end{styleExampleTitle}

\ea
\label{Example_10.65}
\gll {slama} {ini} {de} {tida} {kasi} {uang} {\bluebold{buat}} {\bluebold{saya}}\\ %
 as.long.as  \textsc{d.prox}  \textsc{3sg}  \textsc{neg}  give  money  for  \textsc{1sg}\\
\glt 
‘so far he hasn’t given (any) money \bluebold{to me for my benefit}’ \textstyleExampleSource{[081014-003-Cv.0034]}
\z

\ea
\label{Example_10.66}
\gll {sa} {perna} {bicara} {\bluebold{buat}} {\bluebold{satu}} {\bluebold{ibu} \ldots}\\ %
 \textsc{1sg}  once  speak  for  one  woman  \\
\glt 
‘once I talked \bluebold{to a woman for her benefit} {\ldots}’ \textstyleExampleSource{[081011-024-Cv.0073]}
\z


Benefactive \textitbf{buat} ‘for’ also introduces inanimate beneficiaries, as in the adnominal \isi{prepositional phrase} \textitbf{buat natal} ‘for Christmas’ in (\ref{Example_10.67}). This use, however, is very rare with the corpus including only this one example.


\begin{styleExampleTitle}
Introducing inanimate beneficiaries
\end{styleExampleTitle}

\ea
\label{Example_10.67}
\gll {pi} {ambil} {kayu} {bakar,} {kayu} {bakar} {\bluebold{buat}} {\bluebold{Natal}}\\ %
 go  fetch  wood  burn  wood  burn  for  Christmas\\
\glt
‘(we) went to get firewood, firewood \bluebold{for Christmas}’ \textstyleExampleSource{[081006-017-Cv.0014]}
\z


The \isi{preposition} \textitbf{buat} ‘for’ has dual \isi{word class membership}; it is also used as the \isi{bivalent} \isi{verb} \textitbf{buat} ‘make’ (see §\ref{Para_5.14}).


\section{Prepositions encoding comparisons}
\label{Para_10.3}
Papuan Malay employs three prepositions of comparison: similative \textitbf{sperti} ‘similar to’ (§\ref{Para_10.3.1}) and \textitbf{kaya} ‘like’ (§\ref{Para_10.3.2}), and equative \textitbf{sebagey} ‘as’ (§\ref{Para_10.3.3}). All three introduce similes that express explicit resemblance or equatability between two bases of comparison.


\subsection{\textitbf{sperti} ‘similar to’}
\label{Para_10.3.1}
The \isi{preposition} \textitbf{sperti} ‘similar to’ introduces similes that highlight resemblance or likeness in some respect between the two bases of comparison. Hence, \textitbf{sperti} ‘like’ is similar to \textitbf{kaya} ‘like’; for the distinctions between both similative prepositions see the discussion in §\ref{Para_10.3.2}.



Very commonly, \textitbf{sperti} ‘similar to’ forms peripheral adjuncts, as in \textitbf{sperti klawar} ‘similar to a cave bat’ in (\ref{Example_10.68}). Also quite frequently, \textitbf{sperti} ‘similar to’ expresses resemblance in oblique arguments of some \isi{bivalent} verbs, as in (\ref{Example_10.69}): \textitbf{sperti manusia} ‘similar to a human’ is the oblique object of the change \isi{verb} \textitbf{jadi} ‘become’. In addition, \textitbf{sperti} ‘similar to’ introduces the simile in nonverbal predicates with the complement being a common \isi{noun}, a personal \isi{pronoun} as in \textitbf{sperti ko} ‘similar to you’ in (\ref{Example_10.70}), or a \isi{demonstrative} as in \textitbf{sperti itu}’ like that’ in (\ref{Example_10.71}). Finally, although rather infrequently, \textitbf{sperti} ‘similar to’ expresses resemblance in adnominal prepositional phrases, as in \textitbf{baju sperti ini} ‘clothes like these’ (\ref{Example_10.72}). The examples in (\ref{Example_10.68}) to (\ref{Example_10.72}) also illustrate that \textitbf{sperti} ‘similar to’ introduces animate and inanimate, as well as nominal and pronominal referents.\footnote{In the corpus only singular pronominal complements of \textitbf{sperti} ‘similar to’ are attested.}


\ea
\label{Example_10.68}
\gll {de} {bisa} {terbang} {\bluebold{sperti}} {\bluebold{klawar}}\\ %
 \textsc{3sg}  be.able  fly  similar.to  cave.bat\\
\glt 
‘he/she (the evil spirit) can fly \bluebold{similar to a cave bat}’ \textstyleExampleSource{[081006-022-CvEx.0137]}
\z

\ea
\label{Example_10.69}
\gll {setang} {itu} {de} {bisa} {jadi} {\bluebold{sperti}} {\bluebold{manusia}}\\ %
 evil.spirit  \textsc{d.dist}  \textsc{3sg}  be.able  become  similar.to  human.being\\
\glt 
‘that evil spirit, he/she can become \bluebold{similar to a human}’ \textstyleExampleSource{[081006-022-CvEx.0010]}
\z

\ea
\label{Example_10.70}
\gll {kalo} {kaka} {\bluebold{sperti}} {\bluebold{ko}} {kaka} {malu}\\ %
 if  oSb  similar.to  \textsc{2sg}  oSb  feel.embarrassed(.about)\\
\glt 
‘if I (‘older sibling’) were \bluebold{similar to you}, I (‘older sibling’) would feel ashamed’ \textstyleExampleSource{[081115-001a-Cv.0040]}
\z

\ea
\label{Example_10.71}
\gll {mama} {pu} {hidup} {\bluebold{sperti}} {\bluebold{itu}}\\ %
 mother  \textsc{poss}  life  similar.to  \textsc{d.dist}\\
\glt 
‘my (‘mother’s’) life is \bluebold{like that}’ \textstyleExampleSource{[080922-001a-CvPh.0932/0938]}
\z

\ea
\label{Example_10.72}
\gll {dorang} {tida} {pake} {\bluebold{baju}} {\bluebold{sperti}} {\bluebold{ini},} {pake} {daung{\Tilde}daung}\\ %
 \textsc{3pl}  \textsc{neg}  use  shirt  similar.to  \textsc{d.prox}  use  \textsc{rdp}{\Tilde}leaf\\
\glt 
‘they don’t wear \bluebold{clothes like these}, (they) wear leaves’ \textstyleExampleSource{[081006-023-CvEx.0007]}
\z


The \isi{preposition} \textitbf{sperti} ‘similar to’ has dual \isi{word class membership}; it is also used as a \isi{conjunction} that introduces \isi{similarity} clauses (§\ref{Para_14.2.6}; see also §\ref{Para_5.14}).


\subsection{\textitbf{kaya} ‘like’}
\label{Para_10.3.2}
The core semantics of the \isi{preposition} \textitbf{kaya} ‘like’ are similative: it indicates likeness between the two bases of comparison, similar to \textitbf{sperti} ‘similar to’.\footnote{In terms of its etymology, U. Tadmor (p.c. 2013) notes that the \isi{preposition} \textitbf{kaya} ‘like’ is distinct from the stative \isi{verb} \textitbf{kaya} ‘be rich’: stative “\textitbf{kaya} ‘be rich’ “was borrowed from \ili{Persian} into Classical Malay” while similative \textitbf{kaya} ‘like’ “was borrowed from \ili{Javanese} into colloquial varieties of Indonesian many centuries later. There is no etymological connection between the two”.}
%\todo[inline]{Check footnote reference - there are two Tadmor 2013 refs in BibTeX.}
Unlike \textitbf{sperti} ‘like’, however, \textitbf{kaya} ‘like’ is not attested to combine with demonstratives. Moreover, \textitbf{kaya} ‘like’ is semantically distinct from \textitbf{sperti} ‘similar to’, as discussed below.



Most commonly, \textitbf{kaya} ‘like’ forms peripheral adjuncts, as in \textitbf{kaya burung} ‘like a bird’ in (\ref{Example_10.73}). This example also illustrates that \textitbf{kaya} ‘like’ co-occurs with some of the same verbs as \textitbf{sperti} ‘similar to’, such as \textitbf{terbang} ‘fly’ in (\ref{Example_10.68}) (§\ref{Para_10.3.1}). Less frequently, \textitbf{kaya} ‘like’ introduces the simile in nonverbal predicates as in \textitbf{kaya buaya} ‘like a crocodile’ in (\ref{Example_10.74}). These examples also illustrate that typically the referent is animate and nominal; for an inanimate referent see the example in (\ref{Example_10.77}), and for a pronominal referent see (\ref{Example_10.76b}).


\begin{styleExampleTitle}
Signaling overall likeness or resemblance
\end{styleExampleTitle}

\ea
\label{Example_10.73}
\gll {bisa} {terbang} {\bluebold{kaya}} {\bluebold{burung},} {bisa} {merayap} {\bluebold{kaya}} {\bluebold{ular}}\\ %
 be.able  fly  like  bird  be.able  creep  like  snake\\
\glt 
‘(the evil spirit) can fly \bluebold{like a bird}, can creep \bluebold{like a snake}’ \textstyleExampleSource{[081006-022-CvEx.0031]}
\z

\ea
\label{Example_10.74}
\gll {dong} {bilang} {soa-soa} {kang,} {\bluebold{kaya}} {\bluebold{buaya}}\\ %
 \textsc{3pl}  say  monitor.lizard  you.know  like  crocodile\\
\glt 
‘they call (it) a monitor lizard, you know, (it’s) \bluebold{like a crocodile}’ \textstyleExampleSource{[080922-009-CvNP.0053]}
\z


The semantic distinctions between \textitbf{kaya} ‘like’ and \textitbf{sperti} ‘similar to’ are subtle. While both signal likeness in terms of appearance or behavior, they differ in terms of their semantic effect. Similative \textitbf{kaya} ‘like’ signals overall resemblance between the two bases of comparison. By contrast, the semantic effect of \textitbf{sperti} ‘similar to’ is more limited: it signals likeness or resemblance in some, most often implied, respect. This distinction is illustrated in the contrastive examples in (\ref{Example_10.75}) and (\ref{Example_10.76}).



In (\ref{Example_10.75a}), \textitbf{kaya} ‘like’ signals overall physical resemblance: the speaker’s brother has the same facial features as their father. By contrast, in the elicited example in (\ref{Example_10.75b}) \textitbf{sperti} ‘similar to’ signals limited or partial resemblance: that is, father and son share specific facial features. In (\ref{Example_10.70}), repeated as (\ref{Example_10.76a}), a teacher relates a conversation she had with a socially maladjusted student. Employing \textitbf{sperti} ‘similar to’, the teacher signals that she refers to some specific aspects of the student’s behavior: \textitbf{kalo kaka sperti ko} ‘if I (‘older sibling’) were similar to you (with respect to the behavior you’re displaying at school)’. If, by contrast, the teacher had used \textitbf{kaya} ‘like’, as in the elicited example in (\ref{Example_10.76b}), the semantic effect of the comparison would have been much wider, not only referring to the student’s behavior at school but signaling overall resemblance between the speaker and her student.


\begin{styleExampleTitle}
Semantic distinctions between \textitbf{kaya} ‘like’ and \textitbf{sperti} ‘similar to’
\end{styleExampleTitle}

\ea
\label{Example_10.75}
\ea
\label{Example_10.75a}
\gll  {de} {pu} {muka} {\bluebold{kaya}} {\bluebold{de}} {\bluebold{pu}} {\bluebold{bapa}}\\ %
   \textsc{3sg}  \textsc{poss}  face  like  \textsc{3sg}  \textsc{poss}  father\\
\glt ‘his (my brother’s) face is \bluebold{like his father’s (face)}’ \textstyleExampleSource{[080922-001a-CvPh.1445]}
\vspace{5pt}

\ex
\label{Example_10.75b}
\gll  de  pu  muka  \bluebold{sperti}  \bluebold{de}  \bluebold{pu}  \bluebold{bapa}\\
   \textsc{3sg}  \textsc{poss}  face  similar.to  \textsc{3sg}  \textsc{poss}  father\\
\glt ‘his (my brother’s) face is \bluebold{similar to his father’s (face)}’ \textstyleExampleSource{[Elicited BR120817.007]}
\z
\z

\ea
\label{Example_10.76}
\ea
\label{Example_10.76a}
\gll  {kalo} {kaka} {\bluebold{sperti}} {\bluebold{ko}} {kaka} {malu}\\ %
   if  oSb  similar.to  \textsc{2sg}  oSb  feel.embarrassed(.about)\\
\glt ‘if I (‘older sibling’) were \bluebold{similar to you}, I (‘older sibling’) would feel ashamed’ \textstyleExampleSource{[081115-001a-Cv.0040]}
\vspace{5pt}

\ex
\label{Example_10.76b}
\gll  kalo  kaka  \bluebold{kaya}  \bluebold{ko}  kaka  malu\\
   if  oSb  like  \textsc{2sg}  oSb  feel.embarrassed(.about)\\
\glt ‘if I (‘older sibling’) were \bluebold{like you}, I (‘older sibling’) would feel ashamed’ \textstyleExampleSource{[Elicited BR120817.006]}
\z
\z


Signaling overall resemblance, similative \textitbf{kaya} ‘like’ is also used when the speaker wants to make a more expressive, metaphorical comparison as in (\ref{Example_10.77}). This example also illustrates that the referent can be inanimate.


\begin{styleExampleTitle}
Introducing expressive similes
\end{styleExampleTitle}

\ea
\label{Example_10.77}
\gll {smua} {jalang} {\bluebold{kaya}} {\bluebold{kapal}} {\bluebold{kayu}}\\ %
 all  walk  like  ship  wood\\
\glt 
‘[because they were so hungry] (they) all were strolling around \bluebold{like wooden boats}’ \textstyleExampleSource{[081025-009a-Cv.0188]}
\z


The \isi{preposition} \textitbf{kaya} ‘like’ has dual \isi{word class membership}; it is also used as a \isi{conjunction} that introduces \isi{similarity} clauses (§\ref{Para_14.2.6}; see also §\ref{Para_5.14}).


\subsection{\textitbf{sebagey} ‘as’}
\label{Para_10.3.3}
The equative \isi{preposition} \textitbf{sebagey} ‘as’ introduces similes that express equatability between the two bases of comparison in terms of specific roles or capacities. Hence, \textitbf{sebagey} ‘as’ contrasts with the \isi{similarity} prepositions \textitbf{sperti} ‘similar to’ (§\ref{Para_10.3.1}) and \textitbf{kaya} ‘like’ (§\ref{Para_10.3.2}) which express resemblance and likeness.



Most commonly, the complement is expressed in an adnominal \isi{prepositional phrase}. In (\ref{Example_10.78}), for example, \textitbf{sebagey} ‘as’ links the head nominal \textitbf{torang} ‘\textsc{1pl}’ to the role-encoding adnominal constituent \textitbf{kepala kampung} ‘village heads’. Following mono- or \isi{bivalent} action verbs, \textitbf{sebagey} ‘as’ expresses equatability in peripheral adjuncts. In (\ref{Example_10.79}), for example, \textitbf{sebagey} ‘as’ follows the communication \isi{verb} \textitbf{bicara} ‘speak’ and relates the role-encoding complement \textitbf{ibu camat} ‘Ms. Subdistrict-Head’ to the clausal subject \textitbf{ko} ‘\textsc{2sg}’. The corpus also includes two examples in which \textitbf{sebagey} ‘as’ introduces nonverbal predicates to express equatability, as for example in (\ref{Example_10.80}) between the predicate \textitbf{kepala acara} ‘the head of the festivity’ and the clausal subject \textitbf{sa} ‘\textsc{1sg}’.


\ea
\label{Example_10.78}
\gll {torang} {\bluebold{sebagey}} {\bluebold{kepala}} {\bluebold{kampung}} {juga} {penanggung-jawap}\\ %
 \textsc{1pl}  as  head  village  also  responsibility\\
\glt 
‘we as \bluebold{village heads} are also bearers of responsibility’ \textstyleExampleSource{[081008-001-Cv.0035]}
\z

\ea
\label{Example_10.79}
\gll {{sebentar}} {di} {{Diklat}} {ko} {bicara}\\ %
 {a.moment}  at  {government.education.program}  \textsc{2sg}  speak\\
\gll \bluebold{sebagey}  {\bluebold{ibu}}  {\bluebold{camat}}\\
 as  {woman}  {subdistrict.head}\\
\glt 
‘a bit later at the government education and training (office) you’ll speak \bluebold{as Ms. Subdistrict-Head}’ \textstyleExampleSource{[081010-001-Cv.0099]}
\z

\ea
\label{Example_10.80}
\gll {paling} {sa} {tra} {kerja,} {sa} {\bluebold{sebagey}} {\bluebold{kepala}} {\bluebold{acara}}\\ %
 most  \textsc{1sg}  \textsc{neg}  work  \textsc{1sg}  as  head  festivity\\
\glt 
[About organizing a festivity:] ‘most likely I won’t (have to) work, I’ll be \bluebold{the head of the festivity}’ (Lit. ‘\bluebold{as the head {\ldots}}’) \textstyleExampleSource{[080919-004-NP.0068]}
\z


As for the syntactic properties of its complements, the examples in (\ref{Example_10.78}) to (\ref{Example_10.80}) show that equative \textitbf{sebagey} ‘as’ introduces common nouns, as similative \textitbf{sperti} ‘similar to’ (§\ref{Para_10.3.1}) and \textitbf{kaya} ‘like’ (§\ref{Para_10.3.2}) do. Unlike the findings for both similative prepositions, however, the corpus does not include prepositional phrases in which \textitbf{sebagey} ‘as’ introduces personal pronouns. Neither are examples attested in which \textitbf{sebagey} ‘as’ combines with demonstratives as \textitbf{sperti} ‘similar to’ does.


\section{Summary}
\label{Para_10.4}
Prepositional phrases consist of a \isi{preposition} and a \isi{noun} phrase complement which is obligatory and may not be fronted. The \isi{preposition} indicates the grammatical and semantic relationship of the complement to the predicate. Prepositional phrases in Papuan Malay are formed with eleven different prepositions:



\begin{enumerate}
\item 
Prepositions encoding location in space or time: \textitbf{di} ‘at, in’, \textitbf{ke} ‘to’, \textitbf{dari} ‘from’, and \textitbf{sampe} ‘until’

\item 
Prepositions encoding \isi{accompaniment}/instruments, goals, or \isi{benefaction}: \textitbf{dengang} ‘with’, \textitbf{sama} ‘to’, \textitbf{untuk} ‘for’, and \textitbf{buat} ‘for’

\item 
Prepositions encoding comparisons: \textitbf{sperti} ‘similar to’, \textitbf{kaya} ‘like’, and \textitbf{sebagey} ‘as’

\end{enumerate}

A substantial number of the prepositions have dual \isi{word class membership}, two have trial class membership. That is, three prepositions are also used as verbs, namely \textitbf{buat} ‘for’, \textitbf{sama} ‘to’, and \textitbf{sampe} ‘until’ (see §\ref{Para_5.3}). Six prepositions are also used as conjunctions, namely \textitbf{dengang} ‘with’, \textitbf{kaya} ‘like’, \textitbf{sama} ‘to’, \textitbf{sampe} ‘until’, \textitbf{sperti} ‘similar to’, and \textitbf{untuk} ‘for’ (see §\ref{Para_5.12} and \chapref{Para_14}). (Variation in \isi{word class membership} is discussed in §\ref{Para_5.14}.)



Prepositional phrases take on different functions within the clause and combine with different types of syntactic constituents. The complements of the prepositions take different semantic roles within the clause, depending on the prepositions they are introduced with. These findings are summarized in \tabref{Table_10.1} to \tabref{Table_10.3}; in these tables, the prepositions are listed according to the order in which they are discussed in this chapter, starting with \textitbf{di} ‘at, in’. Empty cells signal unattested constituent combinations.



\tabref{Table_10.1} lists the three syntactic functions that prepositional phrases can take within the clause according to the prepositions they are introduced with, that is, their functions as peripheral adjuncts, nonverbal predicates, and arguments. In addition, \tabref{Table_10.1} lists those prepositions that introduce modifying, adnominal prepositional phrases and those that are also used as conjunctions.


\begin{table}
\caption{Syntactic functions of prepositional phrases}\label{Table_10.1}
\begin{tabular}{llllllll}
\lsptoprule
 & \multicolumn{3}{c}{ Clausal functions}  &  & \multicolumn{3}{c}{ Additional functions}\\
& \textsc{adjct} & \textsc{pred} & \textsc{argt} &  & \textsc{mod} &  \textsc{cnj} & \textsc{adv}\\
\midrule

\textitbf{di} ‘at, in’ & X & X & X  & & X & \\
\textitbf{ke} ‘to’ & X & X & X &  & \\
\textitbf{dari} ‘from’ & X & X & X &  & X & \\
\textitbf{sampe} ‘until’ & X &  & &   &  &  X\\
\textitbf{dengang} ‘with’ & X & X & X &  &  &  X  &  X\\
\textitbf{sama} ‘to’ & X & X & X &  &  &  X\\
\textitbf{untuk} ‘for’ & X & X & X &  & X &  X\\
\textitbf{buat} ‘for’ & X &  & X & &  X & \\
\textitbf{sperti} ‘similar to’ & X & X & X  & & X &  X &  X\\
\textitbf{kaya} ‘like’ & X & X &  & &   &  X  &  X\\
\textitbf{sebagey} ‘as’ & X & X & &   & X & \\
\lspbottomrule
\end{tabular}
\end{table}

With respect to their complements, the data in \tabref{Table_10.2} shows that the prepositions combine with different constituents from different word classes, namely nouns, personal pronouns, demonstratives, locatives, and temporal adverbs.


\begin{table}
\caption{Word classes of complements}\label{Table_10.2}

\begin{tabular}{llllllll}
\lsptoprule
 & \textsc{n.com} & \textsc{n.loc} & \textsc{n.time} & \textsc{pro} & \textsc{dem} & \textsc{loc} &  \textsc{adv.t}\\
\midrule
\textitbf{di} ‘at, in’ & X & X &  & X &  & X & \\
\textitbf{ke} ‘to’ & X & X &  & X &  & X & \\
\textitbf{dari} ‘from’ & X & X & X & X &  & X &  X\\
\textitbf{sampe} ‘until’ &  &  & X &  &  &  &  X\\
\textitbf{dengang} ‘with’ & X &  &  & X & X &  & \\
\textitbf{sama} ‘to’ & X &  &  & X & X &  & \\
\textitbf{untuk} ‘for’ & X &  & X & X & X &  &  X\\
\textitbf{buat} ‘for’ & X &  &  & X &  &  & \\
\textitbf{sperti} ‘similar to’ & X &  &  & X & X &  & \\
\textitbf{kaya} ‘like’ & X &  &  & X &  &  & \\
\textitbf{sebagey} ‘as’ & X &  &  &  &  &  & \\
\lspbottomrule
\end{tabular}
\end{table}

Lastly, the complements of prepositions take different semantic roles within the clause, depending on the prepositions they are introduced with. These different semantic roles are summarized in \tabref{Table_10.3} with the primary role underlined.


\begin{table}
\caption{Semantic roles of complements}\label{Table_10.3}


\begin{tabular}{lllllllll}
\lsptoprule
 & \textsc{loct} & \textsc{assct} & \textsc{omv} & \textsc{ins} & \textsc{rec} & \textsc{ben} & \textsc{circ} &  \textsc{std}\\
\midrule

\textitbf{di} ‘at, in’ & \textstyleChUnderl{X} &  &  &  &  &  &  & \\
\textitbf{ke} ‘to’ & \textstyleChUnderl{X} &  &  &  &  &  &  & \\
\textitbf{dari} ‘from’ & \textstyleChUnderl{X} &  &  &  &  &  &  & \\
\textitbf{sampe} ‘until’ & \textstyleChUnderl{X} &  &  &  &  &  &  & \\
\textitbf{dengang} ‘with’ &  & \textstyleChUnderl{X} & X & X &  &  &  & \\
\textitbf{sama} ‘to’ &  & X & X &  & \textstyleChUnderl{X} &  &  & \\
\textitbf{untuk} ‘for’ &  &  &  &  & X & \textstyleChUnderl{X} & X & \\
\textitbf{buat} ‘for’ &  &  &  &  & X & \textstyleChUnderl{X} &  & \\
\textitbf{sperti} ‘similar to’ &  &  &  &  &  &  &  &  \textstyleChUnderl{X}\\
\textitbf{kaya} ‘like’ &  &  &  &  &  &  &  &  \textstyleChUnderl{X}\\
\textitbf{sebagey} ‘as’ &  &  &  &  &  &  &  &  \textstyleChUnderl{X}\\
\lspbottomrule
\end{tabular}
\end{table}

\largerpage
If the context allows the disambiguation of the semantic relationship of the complement to the predicate, two of the prepositions of location can be omitted: \isi{locative} \textitbf{di} ‘at, in’ and allative \textitbf{ke} ‘to’.

