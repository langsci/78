\chapter[Verbal clauses]{Verbal clauses}
\label{Para_11}
This chapter discusses different types of verbal predicate clauses in Papuan Malay, in which a \isi{verb} occupies the syntactic and semantic core of the clause. In Papuan Malay verbal clauses, the predicate typically follows the subject and, in transitive clauses, precedes the direct object. In negated verbal clauses, the negator precedes the predicate.



Papuan Malay verbal clauses can be distinguished into intransitive and transitive clauses; this distinction is discussed in §\ref{Para_11.1}. The subsequent sections describe special types of (in)transitive clauses: \isi{causative} clauses in §\ref{Para_11.2}, reciprocal clauses in §\ref{Para_11.3}, existential clauses in §\ref{Para_11.4}, and comparative clauses in §\ref{Para_11.5}. The main points of this chapter are summarized in §\ref{Para_11.6}. Negation is described in §\ref{Para_13.1}.


\section{Intransitive and transitive clauses}
\label{Para_11.1}
Papuan Malay verbal clauses can be intransitive, monotransitive, or ditransitive. Typically, intransitive clauses are formed with \isi{monovalent} verbs which take one core argument; as discussed below, though, bi- and \isi{trivalent} verbs also occur in intransitive or monotransitive clauses. Monotransitive clauses are usually formed with \isi{bivalent} verbs which take two core arguments, the subject and a direct object. These two types of verbs and verbal clauses are the most common ones in Papuan Malay. In addition, Papuan Malay has ditransitive clauses formed with a small number of \isi{trivalent} verbs which take three core arguments, a subject and two objects.



It is important to note, though, that in Papuan Malay the \isi{trivalent} verbs allow but do not require three syntactic arguments. Likewise, \isi{bivalent} verbs allow but do not require two arguments. That is, in clauses with tri- or \isi{bivalent} verbs, core arguments are often elided when they are understood from the context. (See also \citeauthor{Margetts.2007}'s \citeyear*{Margetts.2007} cross-linguistic typology for the rather common \isi{elision} of syntactic arguments.)



Given this syntactic mismatch between \isi{valency} and transitivity, this section on transitivity is not organized in terms of intransitive, monotransitive, and ditransitive clauses. Instead, it is organized in terms of the \isi{valency} of the verbs, and describes how the three \isi{verb} classes are used in transitive and/or intransitive clauses. Verbal clauses with \isi{monovalent} verbs are discussed in §\ref{Para_11.1.1}, with \isi{bivalent} verbs in §\ref{Para_11.1.2}, and with \isi{trivalent} verbs in §\ref{Para_11.1.3}. (The properties of verbs are described in §\ref{Para_5.3}. For details on optional linguistic expressions providing additional information about the setting of the events or states depicted by the verbs, see \chapref{Para_10}; see also §\ref{Para_5.2.5}.)


\subsection{Verbal clauses with {monovalent} verbs}
\label{Para_11.1.1}
Papuan Malay has a large open class of \isi{monovalent} verbs. Involving only one participant, they always occur in intransitive clauses (490 \isi{monovalent} verbs are attested in the corpus; for a list of examples see \tabref{Table_5.14} in §\ref{Para_5.3.1}).



Semantically, the attested 490 verbs can be divided into dynamic ones (139 verbs) and stative ones (351 verbs), as is typical of languages lacking a class of adjectives. The former denote actions, while the latter designate states or more time-stable properties. Syntactically, however, there are no distinctions between dynamic and stative verbs.



Typically, \isi{monovalent} verbs follow their clausal subjects, as shown with dynamic \textitbf{lari} ‘run’ in ({\ref{Example_11.1}), and with stative \textitbf{bagus} ‘be good’ in ({\ref{Example_11.2}).


\begin{styleExampleTitle}
Monovalent verbs with canonical subject-\isi{verb} word order
\end{styleExampleTitle}

\ea
\label{Example_11.1}
\gll {o,} {babi} {\bluebold{lari}}\\ %
 oh  pig  run\\
\glt 
‘o, the pig \bluebold{ran}’ \textstyleExampleSource{[080919-004-NP.0021]}
\z

\ea
\label{Example_11.2}
\gll {itu} {\bluebold{bagus}} {skali}\\ %
 \textsc{d.dist}  be.good  very\\
\glt 
‘that is very \bluebold{good}’ \textstyleExampleSource{[081025-003-Cv.0267]}
\z


If speakers want to emphasize the predicate with a \isi{monovalent} stative \isi{verb}, they can front it, such as stative \textitbf{bagus} ‘be good’ in ({\ref{Example_11.3}). In this case, the predicate is set-off by a boundary intonation, which is achieved by marking the stressed syllable of the \isi{verb} with a slight increase in pitch (“~\'{~}~”). Consultants disagree, however, whether \isi{monovalent} dynamic verbs can be fronted. While two consultants stated that dynamic \textitbf{jatu} ‘fall’ in the elicited example in ({\ref{Example_11.4}) can be fronted, a third one rejected the example as ungrammatical. Furthermore, one of the consultants who accepted the \isi{verbal clause} in ({\ref{Example_11.4}) suggested that the fronting of \isi{monovalent} dynamic verbs is a recent development and that older Papuan Malay speakers would not use such a construction.


\begin{styleExampleTitle}
Preposed \isi{monovalent} verbs
\end{styleExampleTitle}

\ea
\label{Example_11.3}
\gll {\bluebold{bágus}} {skali} {itu}\\ %
 be.good  very  \textsc{d.dist}\\
\glt 
‘very \bluebold{good} is that’ \textstyleExampleSource{[081025-003-Cv.0270]}
\z

\ea
\label{Example_11.4}
\gll {o,} {\bluebold{játu}} {dia!}\\ %
 oh  fall  \textsc{3sg}\\
\glt 
‘oh, he \bluebold{fell!}’ \textstyleExampleSource{[Elicited BR131227.001]}
\z


The subject can also be omitted if it can be inferred from the context. In ({\ref{Example_11.5}) the elided subject is \textitbf{sa} ‘\textsc{1sg}’, and in ({\ref{Example_11.6}) it is \textitbf{dia}/\textitbf{de} ‘\textsc{3sg}’.


\begin{styleExampleTitle}
Elision of the subject argument
\end{styleExampleTitle}

\ea
\label{Example_11.5}
\gll {siang} {Ø} {\bluebold{jalang},} {trus} {malam} {Ø} {\bluebold{duduk}} {\bluebold{menyanyi}} {sampe} {jam} {dua}\\ %
 day  {}  walk  next  night  {}   sit  sing  until  hour  two\\
\glt 
‘(during) the day (I) \bluebold{went} (over there), then in the evening (I) \bluebold{sat about} (and) \bluebold{sang} (songs) until two o’clock (in the morning)’ \textstyleExampleSource{[080923-003-CvNP.0002]}
\z

\ea
\label{Example_11.6}
\gll {Speaker-2:} {adu,} {Ø} {\bluebold{nakal}}\\ %
 {}   oh.no!    {} be.mischievous\\
\glt [Speaker 1: ah, that Petrus!]\\
Speaker-2: oh no!, (he’s) \bluebold{mischievous}’ \textstyleExampleSource{[081115-001a-Cv.0033]}
\z

\subsection{Verbal clauses with {bivalent} verbs}
\label{Para_11.1.2}
Papuan Malay has a large open class of \isi{bivalent} verbs (535 are attested in the corpus; for a set of examples see \tabref{Table_5.14} in §\ref{Para_5.3.1}). Bivalent verbs have two core arguments, a subject and an object. In terms of their semantic roles, “two-place predicates take an agent-like argument A, and a non-agent-like argument P”, adopting \citegen[396]{Margetts.2007} terminology. As mentioned, though, \isi{bivalent} verbs in Papuan Malay allow but do not require two syntactic arguments. Examples of \isi{bivalent} verbs are \textitbf{bunu} ‘kill’ in ({\ref{Example_11.7}) and \textitbf{potong} ‘cut’ in ({\ref{Example_11.8}).


\begin{styleExampleTitle}
Bivalent verbs with two arguments and canonical subject-verb-object order
\end{styleExampleTitle}

\ea
\label{Example_11.7}
\gll {kalo} {ko} {masi} {mo} {berjuang} {kitorang} {\bluebold{bunu}} {ko}\\ %
 if  \textsc{2sg}  still  want  struggle  \textsc{1pl}  kill  \textsc{2sg}\\
\glt 
‘if you still want to fight, we’ll \bluebold{kill} you’ \textstyleExampleSource{[081029-004-Cv.0072]}
\z

\ea
\label{Example_11.8}
\gll {jadi} {kamu} {\bluebold{potong}} {sapi}\\ %
 so  \textsc{2pl}  cut  cow\\
\glt 
‘so you \bluebold{cut up} the cow’ \textstyleExampleSource{[080925-005-CvPh.0007]}
\z


The monotransitive clauses in ({\ref{Example_11.7}) and ({\ref{Example_11.8}) illustrate the canonical subject-verb-object order for \isi{bivalent} verbs. If speakers want to emphasize the object, they can also front it. Unlike clauses with preposed \isi{monovalent} verbs, though, there is no clear boundary intonation to set-off the preposed object arguments from the rest of the clause. In ({\ref{Example_11.9}), the preposed object \textitbf{paylot} ‘pilot’ is marked with a slight increase in pitch of its stressed penultimate syllable (“~\'{~}~”) and it is separated from the rest of the clause with a \isi{comma intonation} (“{\textbar}”). Besides, the ultimate syllable of \textitbf{bunu} ‘kill’ receives final \isi{lengthening}, signaled with the vowel tripling. In ({\ref{Example_11.10}), the preposed object remains unmarked but the clause-final \isi{verb} \textitbf{potong} ‘cut’ is marked with a slight increase in pitch of its stressed penultimate syllable.


\begin{styleExampleTitle}
Bivalent verbs with preposed object arguments
\end{styleExampleTitle}

\ea
\label{Example_11.9}
\gll {páylot} {\textup{\textbar}} {dorang} {\bluebold{bunuuu}}\\ %
 pilot  {}  \textsc{3pl}  kill\\
\glt 
‘the pilot they \bluebold{killed}’ \textstyleExampleSource{[081025-004-Cv.0040]}
\z

\ea
\label{Example_11.10}
\gll {dong} {dua} {pu} {telefisi} {sidi} {dua} {dia} {\bluebold{pótong}}\\ %
 \textsc{3pl}  two  \textsc{poss}  television  CD.player  two  \textsc{3sg}  cut\\
\glt 
‘the television (and) both CDs of the two of them he \bluebold{destroyed}’ \textstyleExampleSource{[081011-009-Cv.0006]}
\z


When one or both of the core arguments are understood from the context, they can be omitted, as shown in ({\ref{Example_11.11}) to ({\ref{Example_11.16}).\footnote{
At this point in the research, the number of clauses with overt and elided core arguments has not been quantified to examine which strategy is the preferred one.} Elision of the object argument is illustrated for \textitbf{bunu} ‘kill’ in ({\ref{Example_11.11}), and \textitbf{potong} ‘cut (up) in ({\ref{Example_11.12}).


\begin{styleExampleTitle}
Elision of the object argument and retention of the subject argument
\end{styleExampleTitle}

\ea
\label{Example_11.11}
\gll {\ldots} {kalo} {prempuang} {melahirkang} {laki{\Tilde}laki} {dong} {\bluebold{bunu}} {Ø}\\ %
 {}   if  woman  give.birth  \textsc{rdp}{\Tilde}husband  \textsc{3pl}  kill  \\
\glt 
‘[indeed, these women can’t live with men,] when a woman gives birth to a boy, they \bluebold{kill} (him)’ \textstyleExampleSource{[081006-023-CvEx.0058]}
\z

\ea
\label{Example_11.12}
\gll {\ldots} {tong} {\bluebold{potong}} {Ø} {hari} {itu}\\ %
 {}  \textsc{1pl}  cut  {}  day  \textsc{d.dist}\\
\glt 
‘[we shouldered it, the pig, (and) carried (it) to the garden shelter,] we \bluebold{cut} (it) \bluebold{up} that day’ \textstyleExampleSource{[080919-003-NP.0013-0014]}
\z


Elision of the subject argument is demonstrated for \textitbf{bunu} ‘kill’ in ({\ref{Example_11.13}), and \textitbf{potong} ‘cut’ in ({\ref{Example_11.14}).


\begin{styleExampleTitle}
Elision of the subject argument and retention of the object argument
\end{styleExampleTitle}

\ea
\label{Example_11.13}
\gll {Ø} {\bluebold{bunu}} {dia,} {Ø} {\bluebold{bunu}} {dia}\\ %
 {}  kill  \textsc{3sg} {}   kill  \textsc{3sg}\\
\glt 
‘(they) \bluebold{kill} him, (they) \bluebold{kill} him’ \textstyleExampleSource{[081006-022-CvEx.0088]}
\z

\ea
\label{Example_11.14}
\gll {baru} {Ø} {\bluebold{potong}} {pisang} {di} {tenga{\Tilde}tenga} {to?}\\ %
 and.then {}  cut  banana  at  \textsc{rdp}{\Tilde}middle  right?\\
\glt 
‘and then (we) \bluebold{cut} the bananas in the middle, right?’ \textstyleExampleSource{[080922-009-CvNP.0041]}
\z


Finally, speakers can also omit both core arguments at the same time, as shown for \textitbf{bunu} ‘kill’ in ({\ref{Example_11.15}), and \textitbf{potong} ‘cut’ in ({\ref{Example_11.16}).


\begin{styleExampleTitle}
Elision of the subject and object arguments
\end{styleExampleTitle}

\ea
\label{Example_11.15}
\gll {Ø} {\bluebold{bunu}} {Ø} {tapi} {kasi} {hidup} {lagi}\\ %
 {}  kill {}  but  give  live  again\\
\glt 
[About sorcerers who can resurrect the dead:] ‘(they) \bluebold{kill} (him) but (they) make (him) live again’ \textstyleExampleSource{[081006-022-CvEx.0087]}
\z

\ea
\label{Example_11.16}
\gll {Ø} {\bluebold{potong}} {Ø} {kecil{\Tilde}kecil}\\ %
 {}  cut  {}   \textsc{rdp}{\Tilde}be.small\\
\glt
‘(I) \bluebold{cut} (the meat) very small’ \textstyleExampleSource{[080919-003-NP.0016]}
\z

\subsection{Verbal clauses with {trivalent} verbs}
\label{Para_11.1.3}
Papuan Malay has a small number of \isi{trivalent} verbs with three core arguments, that is, a subject, and two objects. In the corpus seven \isi{trivalent} verbs are attested: \textitbf{ambil} ‘fetch’, \textitbf{bawa} ‘bring’, \textitbf{bli} ‘buy’, \textitbf{ceritra} ‘tell’, \textitbf{kasi} ‘give’, \textitbf{kirim} ‘send’, and \textitbf{minta} ‘request’.



In terms of their semantic roles, three-place predicates “take an agent-like A, a participant that will label R on the basis of its most common role as recipient (but that may also be a beneficiary, goal, addressee, location, or source), and a T (typically some thing or information conveyed by A to R)”, applying \citegen[396]{Margetts.2007} terminology. As mentioned, though, \isi{trivalent} verbs in Papuan Malay allow but do not require three syntactic arguments.



Trivalent verbs exhibit dative alternation in that they appear in ditransitive clauses with \isi{double-object} constructions (§\ref{Para_11.1.3.1}), or in monotransitive clauses with oblique constructions (§\ref{Para_11.1.3.2}). Alternatively, the R and T arguments can be combined into one \isi{noun} phrase with an adnominal possessor (§\ref{Para_11.1.3.3}). Another option is to omit the R and/or T arguments (§\ref{Para_11.1.3.4}). The distributional frequencies for these strategies are discussed in §\ref{Para_11.1.3.5}.


\subsubsection[Double{}-object constructions]{Double-object constructions}
\label{Para_11.1.3.1}
In Papuan Malay ditransitive clauses with \isi{double-object} constructions, the R and T arguments are unflagged and occur in the order R-T. In this construction type, as \citet[173]{Payne.1997} puts it, the semantically peripheral R is brought “center-stage” while the T has “status as the ‘second object’”. Cross-linguistically, the R typically precedes the T which, as \citet[16]{Malchukov.2010} suggest, “probably derives from the fact that the R is generally human (and often \isi{definite}) and thus tends to be more topical than the T, which is typically inanimate (and often \isi{indefinite})”.



Papuan Malay double object constructions with R-T word order are presented in ({\ref{Example_11.17}) to ({\ref{Example_11.23}). Overall, however, \isi{double-object} constructions are not very common in Papuan Malay. The corpus contains only 30 constructions among a total of 1,160 verbal clauses formed with \isi{trivalent} verbs (2.6\%).\footnote{This total excludes serial \isi{verb} constructions formed with \textitbf{kasi} ‘give’ (see §\ref{Para_11.2.1.2}).}


\begin{styleExampleTitle}
Double-object constructions: R-T word order
\end{styleExampleTitle}

\ea
\label{Example_11.17}
\gll {mungking} {de} {suru} {dia,} {ko} {\bluebold{ambil}} {sa} {air!}\\ %
 maybe  \textsc{3sg}  order  \textsc{3sg}  \textsc{2sg}  fetch  \textsc{1sg}  water\\
\glt 
‘maybe he/she’ll order him/her, ``you \bluebold{fetch} me water!''' \textstyleExampleSource{[081006-024-CvEx.0092]}
\z

\ea
\label{Example_11.18}
\gll {tiga} {orang} {itu} {datang} {\ldots} {\bluebold{bawa}} {dong} {pakeang}\\ %
 three  person  \textsc{d.dist}  come {}   bring  \textsc{3pl}  clothes\\
\glt 
‘those three people came {\ldots} (and) \bluebold{brought} them clothes’ \textstyleExampleSource{[081006-023-CvEx.0074]}
\z

\ea
\label{Example_11.19}
\gll {paytua} {dia} {\bluebold{bli}} {Andi} {satu} {set}\\ %
 husband  \textsc{3sg}  buy  Andi  one  set\\
\glt 
‘the gentleman \bluebold{bought} Andi one (TV/CD) set’ \textstyleExampleSource{[081011-009-Cv.0055]}
\z
\ea
\label{Example_11.20}
\gll  nanti  {waktu}  tidor  de  bilang,  a,  bapa  \bluebold{ceritra}  ko\\
 very.soon  {time}  sleep  \textsc{3sg}  say  ah!  father  tell  \textsc{2sg}\\
\gll {dongeng{\Tilde}dongeng}  {dulu}\\
 {\textsc{rdp}{\Tilde}legend}  {first}\\
\glt 
‘later at bed-time he’ll say, ``ah, I (‘father’) \bluebold{tell} you some stories first''' \textstyleExampleSource{[081110-008-CvNP.0140]}
\z
\ea
\label{Example_11.21}
\gll  skarang  dong  \bluebold{kasi}  dia  senter\\
 now  \textsc{3pl}  give  \textsc{3sg}  flashlight\\
\glt 
‘now they \bluebold{give} him a flashlight’ \textstyleExampleSource{[081108-003-JR.0002]}
\z
\ea
\label{Example_11.22}
\gll  sa  baru{\Tilde}baru  bilang,  {\ldots}  kaka  \bluebold{kirim}  dong  uang!\\
 \textsc{1sg}  \textsc{rdp}{\Tilde}recently  say {}   oSb  send  \textsc{3pl}  money\\
\glt 
‘just now I said, ``older sibling \bluebold{send} them money!''' \textstyleExampleSource{[080922-001a-CvPh.0860]}
\z

\ea
\label{Example_11.23}
\gll {trus} {sa} {bukang} {orang} {miskin} {\bluebold{minta{\Tilde}minta}} {kamu} {uang}\\ %
 next  \textsc{1sg}  \textsc{neg}  person  be.poor  \textsc{rdp}{\Tilde}request  \textsc{2pl}  money\\
\glt 
‘and I’m not a poor person (who) \bluebold{keeps begging} you (for) money’ \textstyleExampleSource{[081011-020-Cv.0043/0045]}
\z


The T can also precede the R in \isi{double-object} constructions, as shown in ({\ref{Example_11.24}) and ({\ref{Example_11.25}). This T-R order “is relatively widespread in South-East Asia”, as \citet[17]{Malchukov.2010} point out. Building on \citegen[435–436]{Dik.1997} notion of “iconic sequencing”, \citet[17]{Malchukov.2010} suggest that “the order T-R is more iconic than the order R-T, because in the unfolding of the event the T is first involved in the action, which reaches the R only in a second step”.



In Papuan Malay, however, T-R constructions are even less common than \mbox{R-T} constructions; the corpus contains 17 constructions among the total of 1,160 verbal clauses formed with \isi{trivalent} verbs (1.5\%). All of them are formed with \textitbf{kasi} ‘give’, as in ({\ref{Example_11.24}) and ({\ref{Example_11.25}). In 12 of them, the T is \textitbf{nasihat} ‘advice’ as in ({\ref{Example_11.24}), in two it is \textitbf{ijing} or \textitbf{ijing{\Tilde}ijing} ‘permission’ as in ({\ref{Example_11.25}), and in the remaining three the Ts are \textitbf{ana} ‘child’, \textitbf{kemerdekaang} ‘independence’ and \textitbf{swara} ‘voice’.


\begin{styleExampleTitle}
Double-object constructions: T-R word order
\end{styleExampleTitle}

\ea
\label{Example_11.24}
\gll {sa} {bilang} {begini,} {sa} {\bluebold{kasi}} {nasihat} {kamu}\\ %
 \textsc{1sg}  say  like.this  \textsc{1sg}  give  advice  \textsc{2pl}\\
\glt 
‘I said like this, ``I \bluebold{give} you advice''' \textstyleExampleSource{[081115-001a-Cv.0332]}
\z

\ea
\label{Example_11.25}
\gll {adu,} {nene} {knapa} {\bluebold{kasi}} {ijing{\Tilde}ijing} {dia} {begitu}\\ %
 oh.no!  grandmother  why  give  \textsc{rdp}{\Tilde}permission  \textsc{3sg}  like.that\\
\glt 
‘oh no!, why did you (‘grandmother’) \bluebold{give} him permission like that?’ \textstyleExampleSource{[081014-008-CvNP.0026]}
\z


In \isi{double-object} constructions the R is most often encoded by a personal \isi{pronoun}, namely in 42/47 attested constructions (89\%), as in ({\ref{Example_11.17}) and ({\ref{Example_11.18}). In the remaining five constructions, the R is encoded by a nominal. Three nominals occur in R-T constructions, namely in \textitbf{bli Andi} ‘buy Andi’ in ({\ref{Example_11.19}), and in \textitbf{kirim bapa} ‘send father’, and \textitbf{minta Noferus} ‘request Noferus’. The remaining two occur in T-R constructions, namely in ‘\textitbf{kasi nasihat} R’ constructions. The respective Rs are \textitbf{pendeta} ‘pastor’ and \textitbf{ana{\Tilde}ana} ‘children’. These distributional frequencies are discussed in §\ref{Para_11.1.3.5}.


\subsubsection[R{}-type oblique constructions]{{R-type oblique} constructions}
\label{Para_11.1.3.2}
One common alternative to \isi{double-object} constructions is the “oblique strategy” \citep[411]{Margetts.2007} in which “the \isi{verb} takes only two direct arguments and the third participant is expressed as an oblique argument or an adjunct”. Very commonly, it is the R that is expressed with a \isi{prepositional phrase}; hence “\isi{R-type oblique}”. Alternatively, the T is encoded in this manner; hence, “T-type oblique”. (\citealt[413]{Margetts.2007}; see also \citealt[17]{Malchukov.2010}.)\footnote{Alternatively, the oblique strategy is also called “‘dative alternation’, earlier ‘dative shift’ or ‘dative movement’” \citep[18]{Malchukov.2010}; an alternative term for “R-type obliques” is “indirective alignment” \citep[3]{Malchukov.2010}.}



In Papuan Malay oblique constructions, it is always the R that is expressed as an oblique, with the R following the T, as shown in ({\ref{Example_11.26}) to ({\ref{Example_11.32}). Overall, however, \isi{R-type oblique} constructions are not very common. The corpus contains only 41 R-type obliques among the total of 1,160 verbal clauses formed with \isi{trivalent} verbs (3.5\%). Moreover, in the corpus, R-type obliques are not attested for all seven verbs (the examples for \textitbf{bawa} ‘bring’ in ({\ref{Example_11.27}), \textitbf{bli} ‘buy’ in ({\ref{Example_11.28}), and \textitbf{kirim} ‘send’ in ({\ref{Example_11.31}) are elicited). Most R-type obliques are introduced with the benefactive prepositions \textitbf{buat} ‘for’ or \textitbf{untuk} ‘for’ (26/41 tokens – 63\%), while the remaining 15 R-type obliques are formed with \isi{goal-oriented} \textitbf{sama} ‘to’. (The semantics of the three prepositions are discussed in §\ref{Para_10.2}.)


\begin{styleExampleTitle}
\isi{R-type oblique} constructions
\end{styleExampleTitle}

\ea
\label{Example_11.26}
\gll {pi} {\bluebold{ambil}} {bola} {sama} {ade}\\ %
 go  fetch  ball  to  ySb\\
\glt 
[Talking to a young boy:] ‘go (and) \bluebold{fetch} the ball for the younger sibling!’ \textstyleExampleSource{[081011-009-Cv.0022]}
\z

\ea
\label{Example_11.27}
\gll {kemaring} {Lukas} {de} {\bluebold{bawa}} {kayu} {bakar} {buat} {Dodo} {dorang}\\ %
 yesterday  Lukas  \textsc{3sg}  bring  wood  burn  for  Dodo  \textsc{3pl}\\
\glt 
‘yesterday Lukas \bluebold{brought} fire wood to Dodo and his associates for their benefit’ \textstyleExampleSource{[Elicited BR130221.035]}
\z

\ea
\label{Example_11.28}
\gll {bapa} {de} {su} {\bluebold{bli}} {baju} {natal} {buat} {sa} {pu} {ade}\\ %
 father  \textsc{3sg}  already  buy  shirt  Christmas  for  \textsc{1sg}  \textsc{poss}  ySb\\
\glt 
‘father already \bluebold{bought} a Christmas shirt for my younger sibling’ \textstyleExampleSource{[Elicited BR130221.002]}
\z

\ea
\label{Example_11.29}
\gll {\ldots} {nanti} {sa} {\bluebold{ceritra}} {ini} {sama} {dia}\\ %
 {}  very.soon  \textsc{1sg}  tell  \textsc{d.prox}  to  \textsc{3sg}\\
\glt 
‘[when he has returned home,] then I’ll \bluebold{tell} this to him’ \textstyleExampleSource{[080921-010-Cv.0004]}
\z

\ea
\label{Example_11.30}
\gll {sa} {\bluebold{kasi}} {hadia} {untuk} {kamu} {kalo} {kam} {kenal} {bapa}\\ %
 \textsc{1sg}  give  gift  for  \textsc{2pl}  if  \textsc{2pl}  know  father\\
\glt 
‘I’ll \bluebold{give} a gift to you for your benefit if you recognize me (‘father’)’ \textstyleExampleSource{[080922-001a-CvPh.1334]}
\z

\ea
\label{Example_11.31}
\gll {kaka} {dorang} {su} {\bluebold{kirim}} {uang} {banyak} {sama} {dong} {pu} {mama}\\ %
 oSb  \textsc{3pl}  already  send  money  many  to  \textsc{3pl}  \textsc{poss}  mother\\
\glt 
‘older sibling and his/her associates already \bluebold{sent} lots of money to their mother’ \textstyleExampleSource{[Elicited BR130221.003]}
\z

\ea
\label{Example_11.32}
\gll {de} {bilang,} {yo,} {sa} {\bluebold{minta}} {maaf} {sama} {paytua}\\ %
 \textsc{3sg}  say  yes  \textsc{1sg}  request  pardon  to  husband\\
\glt 
‘he said, ``yes, I \bluebold{beg} pardon of (your) husband''' \textstyleExampleSource{[081011-024-Cv.0140]}
\z


In the \isi{R-type oblique} constructions in the corpus, the R is most often encoded by a \isi{noun} or a \isi{noun} phrase, namely in 28/41 attested constructions (68\%), as for instance in ({\ref{Example_11.26}) and ({\ref{Example_11.27}). In the remaining 13 constructions (32\%), the R is encoded by a personal \isi{pronoun}, as in ({\ref{Example_11.29}) or ({\ref{Example_11.30}). The distributional frequencies and possible explanations for them are further discussed in §\ref{Para_11.1.3.5}.


\subsubsection[Adnominal possessive constructions]{Adnominal possessive constructions}
\label{Para_11.1.3.3}
Another, cross-linguistic alternative to encode the R and T arguments is to express them in an \isi{adnominal possessive construction}, in which “the agent and the theme are expressed as syntactic arguments of the \isi{verb}, while the R-type participant, which will be the beneficiary with transfer verbs [{\ldots}], is expressed as a grammatical dependent of the theme, namely as its possessor” \citep[426]{Margetts.2007}.



In Papuan Malay, speakers use adnominal possessive constructions when the T is \isi{definite}. The corpus includes 14 such constructions among the 1,160 clauses formed with \isi{trivalent} verbs (1.2\%). Examples are given for \textitbf{ambil} ‘fetch’ in ({\ref{Example_11.33}), \textitbf{bli} ‘buy’ in ({\ref{Example_11.34}), and \textitbf{kasi} ‘give’ in ({\ref{Example_11.35}). In each case, the possessor denotes the benefiting R of the event expressed by the \isi{verb}; the possessum denotes the T as the anticipated object of possession. In the corpus, the possessor is typically encoded by a personal \isi{pronoun} (13/14 tokens – 93\%), as in ({\ref{Example_11.34}) and ({\ref{Example_11.35}). Only in one construction, presented in ({\ref{Example_11.33}), the possessor is expressed with a \isi{noun}, namely the \isi{proper noun} \textitbf{Sofia}. (Adnominal possession is described in detail in \chapref{Para_9}.)


\begin{styleExampleTitle}
Adnominal possessive constructions
\end{styleExampleTitle}
\ea
\label{Example_11.33}
\gll {mama} {nanti} {\bluebold{ambil}} {[Sofia} {pu} {ijasa} {SD]}\\ %
 mother  very.soon  fetch  Sofia  \textsc{poss}  diploma  primary.school\\
\glt 
‘later you (‘mother’) \bluebold{fetch} the primary school diploma for Sofia’ (Lit. ‘Sofia’s primary school diploma’) \textstyleExampleSource{[081011-023-Cv.0065]}
\z

\ea
\label{Example_11.34}
\gll {dia} {punya} {ulang-taung} {kita} {\bluebold{bli}} {[de} {punya} {pakeang} {ulang-taung]}\\ %
 \textsc{3sg}  \textsc{poss}  birthday  \textsc{1pl}  buy  \textsc{3sg}  \textsc{poss}  clothes  birthday\\
\glt 
‘(for) her birthday we \bluebold{buy} birthday clothes for her’ (Lit. ‘her birthday clothes’) \textstyleExampleSource{[081006-025-CvEx.0022]}
\z

\ea
\label{Example_11.35}
\gll {ibu} {distrik} {de} {\bluebold{kasi}} {[kitong} {dua} {pu} {uang} {ojek]}\\ %
 woman  district  \textsc{3sg}  give  \textsc{1pl}  two  \textsc{poss}  money  motorbike.taxi\\
\glt
‘Ms. District \bluebold{gave} us two money for the motorbike taxis’ (Lit. ‘our two motorbike taxi money’) \textstyleExampleSource{[081110-002-Cv.0036]}
\z

\subsubsection[Elision]{Elision}
\label{Para_11.1.3.4}
Elision is a third alternative to \isi{double-object} constructions and used when the T and/or R are understood from the context. In this case, one or both of them can be omitted. In the corpus, this strategy is used in 1,058 of 1,160 verbal clauses formed with \isi{trivalent} verbs (91\%).



Most often the R is elided and the T retained (601/1,058 tokens – 57\%); these distributional frequencies are further discussed in §\ref{Para_11.1.3.5}. Examples are given for \textitbf{bli} ‘buy’ in ({\ref{Example_11.36}), \textitbf{ceritra} ‘tell’ in ({\ref{Example_11.37}), and \textitbf{kirim} ‘send’ in ({\ref{Example_11.38}).


\begin{styleExampleTitle}
Elision of R and retention of T
\end{styleExampleTitle}

\ea
\label{Example_11.36}
\gll {kalo} {besok} {ada} {berkat} {sa} {\bluebold{bli}} {Ø} {komputer} {baru}\\ %
 if  tomorrow  exist  blessing  \textsc{1sg}  buy {}   computer  be.new\\
\glt 
‘if there is a (financial) blessing in the near future, I’ll \bluebold{buy} (us) a new computer’ \textstyleExampleSource{[081025-003-Cv.0086]}
\z

\ea
\label{Example_11.37}
\gll {malam} {nanti} {Matias} {bilang,} {mama} {\bluebold{ceritra}} {Ø} {dongeng} {ka?}\\ %
 night  very.soon  Matias  say  mother  tell  {}  legend  or\\
\glt 
‘later tonight Matias will say, ``are you (‘mother’) going to \bluebold{tell} (me) a story?''' \textstyleExampleSource{[081110-008-CvNP.0142]}
\z

\ea
\label{Example_11.38}
\gll {bapa} {\bluebold{kirim}} {Ø} {uang} {banyak{\Tilde}banyak!}\\ %
 father  send {}   money  \textsc{rdp}{\Tilde}many\\
\glt 
‘[father I want to buy a cell-phone for myself,] father \bluebold{send} (me) lots of money!’ \textstyleExampleSource{[080922-001a-CvPh.0440]}
\z


Constructions with elided T and retained R 
occur 
% much 
less often 
in the corpus (75/1,058 tokens – 7\%). In most cases, the retained R is encoded as an oblique (49/75 tokens – 65\%). This is demonstrated for \textitbf{bawa} ‘bring’ in ({\ref{Example_11.39}), \textitbf{ceritra} ‘tell’ in ({\ref{Example_11.40}), and \textitbf{kasi} ‘give’ in ({\ref{Example_11.41}).


\begin{styleExampleTitle}
Elision of T and retention of oblique R
\end{styleExampleTitle}

\ea
\label{Example_11.39}
\gll {e,} {ko} {bawa} {Ø} {ke} {sana,} {ko} {\bluebold{bawa}} {Ø} {sama} {ade}\\ %
 hey!  \textsc{2sg}  bring {}    to  \textsc{l.dist}  \textsc{2sg}  bring  {}  to  ySb\\
\glt 
[Talking to a young boy:] ‘hey, \bluebold{bring} (the ball) over there, \bluebold{bring} (the ball) to the younger sibling’ \textstyleExampleSource{[081011-009-Cv.0015]}
\z

\ea
\label{Example_11.40}
\gll {\ldots} {baru} {dia} {yang} {\bluebold{ceritra}} {Ø} {sama} {saya}\\ %
 {}  and.then  \textsc{3sg}  \textsc{rel}  tell  {}  to  \textsc{1sg}\\
\glt 
‘[I’d already forgotten who this gentleman was,] and then (it was) him (who) \bluebold{told} (this story) to me’ \textstyleExampleSource{[080917-008-NP.0005]}
\z

\ea
\label{Example_11.41}
\gll {ko} {\bluebold{kasi}} {Ø} {sama} {kaka} {mantri,} {e?}\\ %
 \textsc{2sg}  give  {}  with  oSb  male.nurse  eh\\
\glt 
‘\bluebold{give} (the keys) to the older brother nurse, eh?’ \textstyleExampleSource{[080922-010a-CvNF.0167]}
\z


Less often (26/75 tokens – 35\%), the retained R is encoded as a direct object. This is illustrated for \textitbf{kasi} ‘give’ in ({\ref{Example_11.42}), and \textitbf{minta} ‘request’ in ({\ref{Example_11.43}).


\begin{styleExampleTitle}
Elision of T and retention of direct-object R
\end{styleExampleTitle}

\ea
\label{Example_11.42}
\gll {\ldots} {hari} {ini} {dorang} {bisa} {\bluebold{kasi}} {ko} {Ø}\\ %
 {}  day  \textsc{d.prox}  \textsc{3pl}  be.able  give  \textsc{2sg}  \\
\glt 
‘[if (you) say (you also want) a trillion (rupiah),] today they can \bluebold{give} you (the money)’ \textstyleExampleSource{[081029-004-Cv.0023]}
\z

\ea
\label{Example_11.43}
\gll {piring{\Tilde}piring} {kosong,} {sa} {\bluebold{minta}} {Ise} {Ø,} {sa} {bilang} {\ldots}\\ %
 \textsc{rdp}{\Tilde}plate  be.empty  \textsc{1sg}  request  Ise {}   \textsc{1sg}  say  \\
\glt 
‘the (cake) plates were empty, I \bluebold{asked} Ise (for a piece of cake), I said {\ldots}’ \textstyleExampleSource{[081011-005-Cv.0034]}
\z


In constructions with elided T and retained R, the R is most often encoded by a nominal (56/75 tokens – 75\%). This applies to oblique Rs (39/49 – 80\%), as in ({\ref{Example_11.39}), as well as to direct-object Rs (17/26 – 65\%), as in ({\ref{Example_11.43}). Retained pronominal Rs, by contrast, occur much less often (19/75 tokens – 25\%), be they oblique Rs as in ({\ref{Example_11.40}), or direct-object Rs as in ({\ref{Example_11.42}). These distributional frequencies are discussed in §\ref{Para_11.1.3.5}.



Finally, \isi{elision} can also affect the R and the T at the same time. That is, both can be omitted at once if they are understood from the context. In the corpus, this applies to a substantial number of verbal clauses formed with \isi{trivalent} verbs (382/1,160 tokens – 36\%). This type of \isi{elision} is illustrated for \textitbf{ambil} ‘fetch’ in ({\ref{Example_11.44}), \textitbf{bli} ‘buy’ in ({\ref{Example_11.45}), and \textitbf{kirim} ‘send’ in ({\ref{Example_11.46}).


\begin{styleExampleTitle}
Elision of R and T
\end{styleExampleTitle}

\ea
\label{Example_11.44}
\gll {\ldots} {Matias} {nanti} {anjing,} {cepat,} {ko} {\bluebold{ambil}} {Ø} {Ø} {dulu!}\\ %
 {}  Matias  very.soon  dog  be.fast  \textsc{2sg}  fetch {} {}     first\\
\glt 
‘[Matias, younger sister’s fish fell down,] Matias, very soon the dogs (will get it), quick, you \bluebold{fetch} (your sister the fish)!’ \textstyleExampleSource{[081006-019-Cv.0002]}
\z

\ea
\label{Example_11.45}
\gll {\ldots} {de} {pu} {tete} {tanya} {dia,} {ko} {\bluebold{bli}} {Ø} {Ø} {di} {mana?}\\ %
{}   \textsc{3sg}  \textsc{poss}  grandfather  ask  \textsc{3sg}  \textsc{2sg}  buy {} {}     at  where\\
\glt 
‘[when the grandchild emerged, he was holding a fried banana,] then his grandfather asked him, ``where did you \bluebold{buy} (yourself the fried banana)?''' \textstyleExampleSource{[081109-005-JR.0007]}
\z

\ea
\label{Example_11.46}
\gll {\ldots} {mama} {dong} {di} {kampung} {tra} {\bluebold{kirim}} {Ø} {Ø}\\ %
 {}  mother  \textsc{3pl}  at  village  \textsc{neg}  send    \\
\glt
‘[it’s difficult, there is no money,] mother and the others in the village don’t \bluebold{send} (us money)’ \textstyleExampleSource{[080922-001a-CvPh.0943/0945]}
\z

\subsubsection[Distributional frequencies]{Distributional frequencies}
\label{Para_11.1.3.5}
The above description of how Papuan Malay \isi{trivalent} verbs are used in verbal clauses shows three types of \isi{variation}, namely in word order, in encoding the R and T arguments, and in eliding one or both of these arguments. The data also indicates distributional preferences for these three \isi{variation} types. Summarizing this \isi{variation}, this section provides an explanation for the distributional frequencies and preferences in terms of salience.



Cross-linguistically, ditransitive alignment \isi{variation} is related to distinctions between the R and T arguments in terms of three “salience scales (animacy, definiteness, person)”, with \citet[84]{Haspelmath.2007b} presenting the following scale for “differential R marking”:\footnote{See also \citegen{Comrie.1989} animacy hierarchy, \citegen[85]{Dixon.1979} agency scale, and  \citegen{Silverstein.1976} hierarchy of features.}


\begin{styleIIndented}
1st/2nd {\textgreater} 3rd {\textgreater} \isi{proper noun} {\textgreater} human {\textgreater} nonhuman
\end{styleIIndented}


When the R is more salient than the T, speakers favor a \isi{double-object} construction. This preference applies especially to pronominal Rs, which are the most salient ones. Otherwise, as \citet[83]{Haspelmath.2007b} states, the oblique construction is the favored one:


\begin{styleIIndented}
Special (“indirective” or “dative”) R-marking is the more likely, the lower the R is on the animacy, definiteness, and person scales.
\end{styleIIndented}


The same distributional preferences apply to Papuan Malay, as shown in \tabref{Table_11.2}. Before discussing the distribution of nominal and pronominal Rs, however, \tabref{Table_11.1} gives an overview of the distributional frequencies for \isi{trivalent} verbs in the different constructions types discussed in the preceding sections.




\begin{table}
\caption{Distributional preferences for \isi{trivalent} verbs}\label{Table_11.1}
\begin{tabular}{lrr}
\lsptoprule
& Token \# &  \%\\
\midrule

DO &  47 &  4.1\\
Obl. &  41 &  3.5\\
AdPoss. &  14 &  1.2\\
Elision &  1,058 &  91.2\\
\midrule
Total &  1,160 &  100\\
\lspbottomrule
\end{tabular}
\end{table}



\tabref{Table_11.1} shows that Papuan Malay disfavors clauses in which both the R and T arguments are overtly mentioned. Double-object constructions are rare (4.1\%); the 47 clauses include 30 clauses with R-T order and 17 with T-R order. Likewise, \isi{R-type oblique} constructions are rare (3.5\%). Adnominal possessive constructions with an R possessor are even rarer (1.2\%). Instead, \isi{trivalent} verbs usually occur in clauses with elided R and/or T arguments (91\%). Details on \isi{elision} are presented in \tabref{Table_11.3}.



\begin{table}[b]
\setlength{\tabcolsep}{4mm}
\begin{tabular}{l*{5}{r}}
\lsptoprule
 & \multicolumn{1}{c}{DO} & \multicolumn{1}{c}{Obl.} & \multicolumn{1}{c}{AdPoss.} & \multicolumn{1}{c}{T Elision} &  \multicolumn{1}{c}{Total\footnote{As percentages are rounded to one decimal place, they do not always add up to 100\%.}}\\
\midrule
\textsc{nom}{}-R & 5 &  \numprint{28} &  1 &  56 &  90\\
&  5.6\% &  31.1\% &  1.1\% &  62.2\% &  100\%\\
\textsc{pro}{}-R &  42 &  13 &  13 &  19 &  87\\
&  48.3\% &  14.9\% &  14.9\% &  21.8\% &  100\%\\
\midrule
Total &  47 &  41 &  14 &  75 &  177\\
&  26.6\% &  23.2\% &  7.9\% &  42.4\% &  100\%\\
\lspbottomrule
\end{tabular}
\caption{\label{Table_11.2}Distribution of nominal and pronominal Rs}
\end{table}

As for the distribution of nominal and pronominal Rs, \tabref{Table_11.2} indicates clear preferences. Only five nominal Rs occur in \isi{double-object} constructions (6\%), and about one third in \isi{R-type oblique} constructions (28/90 tokens – 31\%). Besides, one nominal R is used in an \isi{adnominal possessive construction} (1\%). Instead, most nominal Rs occur in clauses with elided T arguments (56/90 tokens – 62\%; \tabref{Table_11.3} gives details on \isi{elision}). By contrast, about half of the pronominal Rs occur in \isi{double-object} constructions (42/87 tokens – 48\%), while 13 Rs are used in \isi{R-type oblique} constructions (15\%). Another 13 Rs occur in adnominal possessive constructions (15\%; compare with one token for nominal Rs). Yet another 19 Rs occur in clauses with elided T (22\%; compare with 56 nominal Rs).

\begin{table}[t]
\caption{Distributional preferences for argument \isi{elision} and retention}\label{Table_11.3}
\setlength{\tabcolsep}{4mm}
\begin{tabular}{l*{5}{r}}
\lsptoprule
 & R \textsc{els} & T \textsc{els} &  T \textsc{els} & T \textsc{els} &  Total\\
 & T \textsc{ret} & DO-R \textsc{ret} & Obl.-R \textsc{ret} & R \textsc{els} \\
\midrule

\multicolumn{6}{l}{Distribution of elided and retained arguments}\\
\midrule
Total &  601 &  26 &  49 &  382 &  1,058\\
&  57\% &  2\% &  5\% &  36\% &  100\%\\
\midrule
\multicolumn{6}{l}{Encoding of retained Rs}\\
\midrule
\textsc{nom}{}-R &   --  &  17 &  39 &  --  &  56\\
\textsc{pro}{}-R &   --  &  9 &  10 &   --  &  19\\
\midrule
Total &   --  &  26 &  49 &   --  &  75\\
&  &  25\% &  75\% &  &  100\%\\
\lspbottomrule
\end{tabular}
\end{table}

This tendency for pronominal Rs to occur in \isi{double-object} constructions, \linebreak while nominal Rs are more often used in \isi{R-type oblique} constructions is in line with \citegen[84]{Haspelmath.2007b} scale for differential R marking,\linebreak presented above. As mentioned, this scale suggests that speakers favor a \isi{double-object} construction when the R is more salient than the T, a preference that applies especially to pronominal Rs. Otherwise, speakers favor an oblique construction.



There is one exception, though. When speakers want to signal that a pronominal R is also the beneficiary of the transfer, they encode this R as an \isi{R-type oblique}, which is introduced with benefactive \textitbf{buat} ‘for’ or \textitbf{untuk} ‘for’ (both prepositions and their semantics are discussed in §\ref{Para_10.2}). This benefactive marking of the R is not possible in \isi{double-object} constructions. Hence, speakers have to use an \isi{R-type oblique} construction; this also applies to the 13 pronominal Rs in the corpus occurring in \isi{R-type oblique} constructions. In nine of them (70\%), the oblique is introduced with a benefactive \isi{preposition}.



As already discussed, however, Papuan Malay disfavors constructions in \linebreak which the R and T arguments are both overtly mentioned. Instead, \isi{trivalent} verbs usually occur in clauses in which the R and/or T arguments are elided (1,058/1,160 tokens – 91\%; see \tabref{Table_11.1}). Most often, the more salient R is omitted while the less salient T is retained (601/1,058 tokens – 57\%), as shown in \tabref{Table_11.3}. Clauses in which the R and the T are both elided at the same time are also rather common (382/1,058 tokens – 36\%). Only rarely, the T is omitted while the R is retained (75/1,058 tokens – 7\%).



Retention of the R most often affects nominal Rs (\textsc{nom}{}-R) (56/75 tokens – 75\%); most of them are encoded as R-type obliques (39/56 tokens – 70\%). Retention of pronominal Rs (\textsc{pro}{}-R), which are more salient than nominal ones, is much less frequent (19/75 tokens – 25\%). In light of the data given in \tabref{Table_11.2}, one would expect the 19 pronominal Rs to be encoded as direct objects rather than as R-type obliques. As shown in \tabref{Table_11.3}, however, ten of the 19 pronominal Rs are encoded as R-type obliques (53\%). Again, this has to do with their marking as benefactive Rs: seven of the ten pronominal Rs are introduced with a benefactive \isi{preposition}, similar to the 13 pronominal R-type obliques listed in \tabref{Table_11.2}.




An explanation for this preference to delete the R argument and to retain the T argument is given by \citet{Polinsky.1998} in her study on asymmetries in \isi{double-object} constructions (DOC) in English. The author explains the optional deletion of the R arguments “as sensitive to topic”, in that it applies “to those elements of [Information Structure {\ldots}] that have already been activated and are accessible to speaker and hearer. More topical information is easily backgrounded, which explains why the recipient is more easily deleted” (\citeyear*[416]{Polinsky.1998}). Hence, \citet[407]{Polinsky.1998} presents the following implication: “If the patient of DOC can undergo optional deletion, the recipient of DOC can undergo optional deletion, too”.



This observation that the more accessible argument can be deleted also provides an explanation for the preference of Papuan Malay to elide the more salient R argument and to retain the less salient T argument.



The observed tendency to omit the R and/or T arguments has also been noted for western \ili{Austronesian} languages in general. In these languages, as \citet[171]{Himmelmann.2005} points out, “there are few (if any) morphosyntactic constraints on the omission of coreferential arguments in clause sequences. That is, the possibility to omit a coreferential argument is not restricted to subject arguments”. This also applies to other \ili{eastern Malay varieties}, such as \ili{Ambon Malay} \citep[209]{vanMinde.1997}, and \ili{Manado Malay} \citep[133–154]{Stoel.2005}. Along similar lines, \citet{Mosel.2010} notes for the \ili{Oceanic} language \ili{Teop} that “[all] three arguments of ditransitive constructions can be elided in both topical and non-topical positions”. These studies, however, do not discuss whether the languages under investigation have a preference for omitting the R or the T arguments in ditransitive constructions, and what the reasons for such a preference might be. An exception is \citegen{Klamer.2013} study on ‘give’-constructions in heritage and baseline \ili{Ambon Malay}. Noting that \isi{elision} affects the R but not the T, the authors suggest that these distributional preferences are due to “a difference in the prominence of T and R” (\citeyear*[9]{Klamer.2013}).


\section{Causative clauses}
\label{Para_11.2}
Papuan Malay employs three types of \isi{causative} constructions: syntactic, lexical, and periphrastic causatives.



Generally speaking, \isi{causative} clauses are constructions which involve two events: “(1) the causing event in which the causer does something, and (2) the caused event in which the causee carries out an action or undergoes a change of \isi{condition} or state as a result of the causer’s action” \citep[265]{Song.2006}. Hence, \isi{causative} constructions are the result of a valency-increasing operation: in addition to the arguments of the cause event, or “non-\isi{causative} predicate”, there is also the “causer” \citep[175]{Comrie.1989}. This valency-increasing operation is possible with intransitive and transitive events.


\newpage
Cross-linguistically, four major strategies of encoding the notion of causation can be distinguished: lexical, morphological, syntactic, and periphrastic causatives. These constructions differ with respect to the degree of “structural integration” between the causing event, or the “predicate of cause”, and the caused event, or the “predicate of effect” \citep[159–160]{Payne.1997}. Lexical causatives show a maximal degree of structural integration in that the cause and effect are encoded in a single lexical item. Periphrastic \isi{causative} constructions, by contrast, show the least degree of structural integration in that the cause and effect are encoded in two separate clauses. According to \citet[888–889]{Kulikov.2001}, however, lexical causatives do not “qualify as \textstyleChItalic{causatives sensu stricto}” as they do not involve a morphological or syntactic change; neither do periphrastic constructions qualify as ``causatives sensu stricto'' given their biclausal structure.



Morphological and syntactic causatives differ from lexical and periphrastic causatives in that they integrate the cause with the caused event into a single predication. Hence, a causativized intransitive event yields a transitive \isi{causative} construction, while a caus\-a\-tiv\-ized transitive caused event yields a ditransitive construction. The integration of the causer is achieved by demoting the agent of the caused event, the causee. Cross-linguistically, \citet[176]{Comrie.1989} notes the following grammatical relation hierarchy for this process: “subject {\textgreater} direct object {\textgreater} indirect object {\textgreater} oblique object”; that is, “the causee occupies the highest (leftmost) position on this hierarchy that is not already filled”.



As mentioned, Papuan Malay uses three of the four types of \isi{causative} constructions: lexical, syntactic, and periphrastic causatives; morphological causatives are unattested. The main topic of this section is syntactic causatives (§\ref{Para_11.2.1}), since only they qualify as ``causatives sensu stricto'' \citep[888–889]{Kulikov.2001}. Lexical and periphrastic causatives are mentioned only briefly in §\ref{Para_11.2.2} and §\ref{Para_11.2.3}, respectively. The main points of this section are summarized in §\ref{Para_11.2.4}.


\subsection{Syntactic causatives}
\label{Para_11.2.1}
In syntactic causatives, or “\isi{compound}” causatives \citep[450]{Song.2013}, the notion of causation is encoded in a monoclausal construction which consists of two constituents, namely a \isi{causative} \isi{verb}, which expresses the notion of cause, and a second constituent that denotes the effect \citep[887]{Kulikov.2001}.



In Papuan Malay syntactic causatives, a serial \isi{verb} construction V\textsubscript{1}V\textsubscript{2} encodes the causation: the \isi{causative} \isi{verb} V\textsubscript{1} expresses the cause event and the V\textsubscript{2} the caused event. Two free \isi{verb} forms are used as \isi{causative} verbs: \isi{trivalent} \textitbf{kasi} ‘give’ and \isi{bivalent} \textitbf{biking} ‘make’. In \textitbf{kasi}{}-causatives the V\textsubscript{2} can be \isi{monovalent} or \isi{bivalent} while in \textitbf{biking}{}-causatives the V\textsubscript{2} is always \isi{monovalent}.



Semantically, causatives with \textitbf{kasi} ‘give’ focus on the outcome of the causation or manipulation. Causatives with \textitbf{biking} ‘make’, by contrast, focus on the manipulation of circumstances that ultimately leads to the caused event or effect. This is shown with the contrastive examples in ({\ref{Example_11.47}) and ({\ref{Example_11.48}) both of which are formed with \isi{monovalent} stative \textitbf{bersi} ‘be clean’. In ({\ref{Example_11.47}), \textitbf{kasi bersi} ‘cause to be clean’ stresses the outcome of the washing process, namely that the clothes are clean. In the elicited example in ({\ref{Example_11.48}), by contrast, \textitbf{biking bersi} ‘make clean’ focuses on the manipulation itself, which leads to the effect that the clothes are clean.


\begin{styleExampleTitle}
\textitbf{kasi} ‘give’ versus \textitbf{biking} ‘make’ causatives
\end{styleExampleTitle}

\ea
\label{Example_11.47}
\gll {malam} {cuci} {pakeang} {\bluebold{kasi}} {\bluebold{bersi}} {jemur}\\ %
 night  wash  clothes  give  be.clean  dry\\
\glt 
‘(if you have to do laundry at night time) wash (your clothes), \bluebold{clean} (them, and hang them up) to dry’ \textstyleExampleSource{[081011-019-Cv.0009]}
\z

\ea
\label{Example_11.48}
\gll {malam} {cuci} {pakeang} {\bluebold{biking}} {\bluebold{bersi}} {jemur}\\ %
 night  wash  clothes  make  be.clean  dry\\
\glt 
‘(if you have to do laundry at night time) wash (your clothes), \bluebold{clean} (them, and hang them up) to dry’ \textstyleExampleSource{[Elicited BR131103.001]}
\z


The following sections discuss the syntax and semantics of Papuan Malay syntactic causatives in more detail. The two verbs that qualify as \isi{causative} verbs are presented in §\ref{Para_11.2.1.1}, followed by a description of syntactic causatives with the \isi{causative} \isi{verb} \textitbf{kasi} ‘give’ in §\ref{Para_11.2.1.2}, and with \textitbf{biking} ‘make’ in §\ref{Para_11.2.1.3}.


\subsubsection[Causative verbs]{Causative verbs}
\label{Para_11.2.1.1}
The Papuan Malay verbs which express the notion of cause in syntactic \linebreak causatives, \textitbf{kasi} ‘give’ and \textitbf{biking} ‘make’, are used synchronically as full transitive verbs, as shown in ({\ref{Example_11.49}) to ({\ref{Example_11.51}). Trivalent \textitbf{kasi} ‘give’ exhibits dative alternation, as illustrated with the \isi{double-object} constructions in ({\ref{Example_11.49}) and the \isi{R-type oblique} construction in ({\ref{Example_11.50}) (see §\ref{Para_11.1.3} for more details on dative alternation). The transitive uses of \textitbf{biking} ‘make’ are illustrated in ({\ref{Example_11.51}).


\ea
\label{Example_11.49}
\gll {a,} {kam} {\bluebold{kasi}} {sa} {air} {ka}\\ %
 ah  \textsc{2pl}  give  \textsc{1sg}  water  or\\
\glt 
‘ah, you \bluebold{give} me water, please’ \textstyleExampleSource{[080919-008-CvNP.0005]}
\z


\ea
\label{Example_11.50}
\gll {de} {\bluebold{kasi}} {sratus} {ribu} {sama} {Madga}\\ %
 \textsc{3sg}  give  one.hundred  thousand  to  Madga\\
\glt 
‘he \bluebold{gave} one hundred thousand (rupiah) to Madga’ \textstyleExampleSource{[081014-003-Cv.0008]}
\z

\ea
\label{Example_11.51}
\gll {Ika} {\bluebold{biking}} {papeda}\\ %
 Ika  make  sagu.porridge\\
\glt
‘Ika \bluebold{made} sagu porridge’ \textstyleExampleSource{[081006-032-Cv.0071]}
\z

\subsubsection[Syntactic causatives with kasi ‘give’]{Syntactic causatives with \textitbf{kasi} ‘give’}
\label{Para_11.2.1.2}
As a \isi{causative}, \isi{trivalent} \textitbf{kasi} ‘give’, with its short form \textitbf{kas}, is used with two types of verbal bases: \isi{monovalent} ones (§\ref{Para_11.2.1.2.1}), or \isi{bivalent} ones (§\ref{Para_11.2.1.2.2}). Semantically, \isi{causative} \textitbf{kasi} ‘give’ highlights the outcome of a causation.

\subsubsubsection{Monovalent bases\label{Para_11.2.1.2.1}}

Cross-linguistically, in causatives with \isi{monovalent} bases, the agent of the caused event is demoted from its intransitive subject function (S) to the transitive object or \textsc{patient} (P) function, while the incoming causer takes the transitive subject or \textsc{agent} (A) function \citep[110–111]{Comrie.1989}. This strategy, which corresponds to \citegen[176]{Comrie.1989} mentioned \isi{causative} hierarchy, is also used in Papuan Malay causatives with \isi{monovalent} bases. This is illustrated with the monoclausal \isi{causative} constructions in ({\ref{Example_11.52}) to ({\ref{Example_11.59}): causatives with \isi{monovalent} non-agentive bases are presented in ({\ref{Example_11.52}) to ({\ref{Example_11.55}) and causatives with \isi{monovalent} agentive bases in ({\ref{Example_11.56}) to ({\ref{Example_11.59}). (Compare also with the biclausal \isi{causative} constructions in §\ref{Para_11.2.3}.)



In causatives with \isi{monovalent} non-agentive bases, the effect expression can be a stative \isi{verb} such as \textitbf{panjang} ‘be long’ in ({\ref{Example_11.52}), or a non-agentive dynamic \isi{verb} such as \textitbf{gugur} ‘fall (prematurely)’ in ({\ref{Example_11.54}). The resulting V\textsubscript{1}V\textsubscript{2} expressions function as transitive predicates.


\begin{styleExampleTitle}
Causatives with \isi{monovalent} non-agentive bases
\end{styleExampleTitle}

\ea
\label{Example_11.52}
\gll {\ldots} {mama} {harus} {\bluebold{kas}} {\bluebold{panjang}} {kaki}\\ %
 {}  mother  have.to  give  long  foot\\
\glt 
[Addressing someone with a bad knee:] ‘[you shouldn’t fold (your legs) under!,] you (‘mother’) have to \bluebold{stretch out} (your) legs’ \textstyleExampleSource{[080921-004a-CvNP.0069]}
\z
\ea
\label{Example_11.53}
\gll  ko  \bluebold{kasi}  \bluebold{sembu}  sa  punya  ana  ini!\\
 \textsc{2sg}  give  be.healed  \textsc{1sg}  \textsc{poss}  child  \textsc{d.prox}\\
\glt 
[Addressing an evil spirit:] ‘you \bluebold{heal} this child of mine!’ \textstyleExampleSource{[081006-023-CvEx.0031]}
\z

\ea
\label{Example_11.54}
\gll {perna} {dia} {{punya}} {pikirang} {untuk} {de} {mo} {\bluebold{kasi}}\\ %
 ever  \textsc{3sg}  {have}  thought  for  \textsc{3sg}  want  give\\
\gll {\bluebold{gugur}}  {Ø}\\
 {fall(.prematurely)}  {}\\
\glt 
‘once she had the thought that she wanted to \bluebold{abort} (the child)’ \textstyleExampleSource{[080917-010-CvEx.0097]}
\z

\ea
\label{Example_11.55}
\gll {banyak} {mati} {di} {lautang,} {\bluebold{kas}} {\bluebold{tenggelam}} {Ø}\\ %
 many  die  at  ocean  give  sink  \\
\glt 
[About people in a container who died in the ocean:] ‘many died in the (open) ocean, (the murderers) \bluebold{sank} (the containers)’ \textstyleExampleSource{[081029-002-Cv.0025]}
\z


In causatives with \isi{monovalent} agentive bases, the effect expression is encoded by a \isi{monovalent} dynamic \isi{verb}, as shown in ({\ref{Example_11.56}) to ({\ref{Example_11.59}).


\begin{styleExampleTitle}
Causatives with \isi{monovalent} agentive bases
\end{styleExampleTitle}

\ea
\label{Example_11.56}
\gll {sa} {di} {bawa,} {Roni} {\bluebold{kas}} {\bluebold{duduk}} {sa} {di} {atas}\\ %
 \textsc{1sg}  at  bottom  Roni  give  sit  \textsc{1sg}  at  top\\
\glt 
[A ten-year old boy on a truck-trip:] ‘I was down (in the cargo area, but) Roni \bluebold{enabled} me \bluebold{to sit} on top (of the cab)’ \textstyleExampleSource{[081022-002-CvNP.0012]}
\z

\ea
\label{Example_11.57}
\gll {\ldots} {tapi} {dong} {kasi} {bangkit} {dia} {lagi,} {\bluebold{kasi}} {\bluebold{hidup}} {dia}\\ %
 {}  but  \textsc{3pl}  give  be.resurrected  \textsc{3sg}  again  give  live  \textsc{3sg}\\
\glt 
[About sorcerers who can resurrect the dead:] ‘[he’s already (dead),] but they resurrect him again, \bluebold{make} him \bluebold{live}’ \textstyleExampleSource{[081006-022-CvEx.0095]}
\z

\ea
\label{Example_11.58}
\gll {kam} {\bluebold{kas}} {\bluebold{kluar}} {pasir} {dulu!}\\ %
 \textsc{2pl}  give  go.out  sand  first\\
\glt 
‘you \bluebold{remove} the sand first!’ \textstyleExampleSource{[080925-002-CvHt.0005]}
\z

\ea
\label{Example_11.59}
\gll {kam} {\bluebold{kas}} {\bluebold{kluar}} {Dodo} {dari} {dalam} {meja} {situ!}\\ %
 \textsc{2pl}  give  go.out  Dodo  from  inside  table  \textsc{l.med}\\
\glt 
[About a fearful person hiding under the table:] ‘you \bluebold{remove} Dodo / \bluebold{enable} Dodo \bluebold{to get out} from under the table there!’ \textstyleExampleSource{[081025-009b-Cv.0028]}
\z


Cross-linguistically, \isi{causative} constructions receive different readings, \linebreak depending on the causee’s level of agentivity \citep[891–893]{Kulikov.2001}. This also applies to Papuan Malay. When the causee has no control, the \isi{causative} receives a “manipulative or directive” reading, while it receives an “assistive or cooperative” reading, when the causee has some level of agentivity (\citeyear*[892]{Kulikov.2001}).



In causatives with \isi{monovalent} non-agentive bases, as in ({\ref{Example_11.52}) to ({\ref{Example_11.55}), the causer controls the event while the causee has no control. Hence, these causatives always receive a directive reading. Likewise, causatives with \isi{monovalent} agentive bases receive a directive reading when the causee is inanimate, or animate but helpless. This is the case in ({\ref{Example_11.57}) and ({\ref{Example_11.58}). When, by contrast, the causee has some level of control, as in ({\ref{Example_11.56}), the causation is less direct; hence, the \isi{causative} receives an assistive reading. Sometimes, however, the reading of a \isi{causative} is ambiguous, as in ({\ref{Example_11.59}). If the causee \textitbf{Dodo} is unconscious out of fear and thereby helpless, the \isi{causative} receives the directive reading ‘remove’. But if \textitbf{Dodo} is conscious and can move, the \isi{causative} receives the assistive reading ‘enable to come out’.

\subsubsubsection{Bivalent bases\label{Para_11.2.1.2.2}}

In causatives with \isi{bivalent} bases, the cross-linguistically expected operation is for the \textsc{patient} (P) of the caused event to retain its P function and for the \textsc{agent} (A) of the caused event to be demoted to the indirect object function \citep[176]{Comrie.1989}.



Papuan Malay, however, uses a different strategy, in that all the arguments involved shift their functions. That is, the A of the caused event, or causee, is demoted to the P function. In turn, the P of the caused event is moved out of the core into an oblique slot; as an oblique, P is encoded in a \isi{prepositional phrase} introduced with \isi{comitative} \textitbf{dengang} ‘with’, with its short form \textitbf{deng} (see also §\ref{Para_10.2.1}). This is shown with the examples in ({\ref{Example_11.60}) and ({\ref{Example_11.61}).



In ({\ref{Example_11.60}), for instance, the original A, or causee, \textitbf{anjing} ‘dog’, is demoted to the P function and juxtaposed to the V\textsubscript{1}V\textsubscript{2} construction. Semantically, the causee becomes the theme argument of the \isi{causative} expression \textitbf{kas makang} ‘give to eat’. With the P slot being taken, the original P \textitbf{papeda} ‘sagu porridge’ is moved out of the core into an oblique slot.


\begin{styleExampleTitle}
Causatives with \isi{bivalent} bases: Demoting the A and P functions
\end{styleExampleTitle}

\ea
\label{Example_11.60}
\gll {saya} {\bluebold{kas}} {\bluebold{makang}} {anjing} {deng} {papeda}\\ %
 \textsc{1sg}  give  eat  dog  with  sagu.porridge\\
\glt 
‘I \bluebold{fed} the dogs with papeda’ \textstyleExampleSource{[080919-003-NP.0002]}
\z

\ea
\label{Example_11.61}
\gll {dia} {\bluebold{kasi}} {\bluebold{minum}} {kitong} {dengang} {kopi} {air}\\ %
 \textsc{3sg}  give  drink  \textsc{1pl}  with  coffee  water\\
\glt 
‘he’ll \bluebold{give} us coffee and water \bluebold{to drink}’ \textstyleExampleSource{[080919-004-NP.0069]}
\z


In the attested causatives with \isi{bivalent} bases, the causees are able to control their own actions. Therefore, \textitbf{kasi} ‘give’ receives an assistive or cooperative reading, as in ({\ref{Example_11.60}) and ({\ref{Example_11.61}). Causative with \isi{bivalent} bases and inanimate, or animate but helpless referents are unattested.


\subsubsection[Syntactic causatives with biking ‘make’]{Syntactic causatives with \textitbf{biking} ‘make’}
\label{Para_11.2.1.3}
As a \isi{causative}, \isi{bivalent} \textitbf{biking} ‘make’ is used with \isi{monovalent} bases. Semantically, this \isi{causative} type stresses the causer’s manipulation of circumstances, which leads to the caused event or effect. That is, \textitbf{biking}{}-causatives are causer-controlled, with the causee having no control. Therefore, causatives with \textitbf{biking} ‘make’ are formed with \isi{monovalent} non-agentive bases, or with \isi{monovalent} agentive bases with inanimate or with animate but helpless causees. This is shown in ({\ref{Example_11.62}) to ({\ref{Example_11.67}). Overall, though, \textitbf{biking}{}-causatives are rare in the corpus.



The \isi{causative} in ({\ref{Example_11.62}), for example, is formed with non-agentive stative \textitbf{pusing} ‘be dizzy, be confused’. The use of \textitbf{biking} ‘make’ stresses the manipulating behavior of the causer \textitbf{ana{\Tilde}ana} ‘children’ which leads to the effect \textitbf{pusing} ‘be worried’; the causee \textitbf{mama} ‘mother’ has no control. The elicited examples in ({\ref{Example_11.63}) and ({\ref{Example_11.64}) contrast with the corresponding \textitbf{kasi}{}-causatives in ({\ref{Example_11.54}) and ({\ref{Example_11.55}). They show that \textitbf{biking}{}-causatives are also formed with \isi{monovalent} non-agentive dynamic bases, such as \textitbf{gugur} ‘abort’ or \textitbf{tenggelam} ‘sink’, respectively. Again, the manipulation itself is stressed. The base can also be agentive dynamic if the causee is animate but helpless. This is illustrated with the elicited example in ({\ref{Example_11.65}), which contrasts with the corresponding \textitbf{kasi}{}-\isi{causative} in ({\ref{Example_11.57}). The base is agentive dynamic \textitbf{hidup} ‘live’ but the animate causee is helpless and therefore has no control.


\begin{styleExampleTitle}
Causatives with \isi{monovalent} non-agentive bases
\end{styleExampleTitle}

\ea
\label{Example_11.62}
\gll {ana{\Tilde}ana} {\bluebold{biking}} {\bluebold{pusing}} {mama}\\ %
 \textsc{rdp}{\Tilde}child  make  be.dizzy  mother\\
\glt 
‘the kids \bluebold{worry} (their mother)’ (Lit. ‘\bluebold{make to be dizzy / confused}’) \textstyleExampleSource{[081014-007-CvEx.0047]}
\z

\ea
\label{Example_11.63}
\gll {perna} {dia} {punya} {pikirang} {untuk} {de} {mo} {\bluebold{biking}} {\bluebold{gugur}}\\ %
 ever  \textsc{3sg}  have  thought  for  \textsc{3sg}  want  make  fall(.prematurely)\\
\glt 
‘once she had the thought that she wanted to \bluebold{abort} (the child)’ \textstyleExampleSource{[Elicited BR131103.002]}
\z

\ea
\label{Example_11.64}
\gll {banyak} {mati} {di} {lautang,} {\bluebold{biking}} {\bluebold{tenggelam}}\\ %
 many  die  at  ocean  make  sink\\
\glt 
[About people in a container who died in the ocean:] ‘many died in the (open) ocean, (the murderers) \bluebold{sank} (the containers)’ \textstyleExampleSource{[Elicited BR131103.003]}
\z

\ea
\label{Example_11.65}
\gll {\ldots} {tapi} {dong} {\bluebold{biking}} {bangkit} {dia} {lagi,} {\bluebold{biking}} {\bluebold{hidup}} {dia}\\ %
 {}  but  \textsc{3pl}  make  be.resurrected  \textsc{3sg}  again  make  live  \textsc{3sg}\\
\glt 
[About sorcerers who can resurrect the dead:] ‘[he’s already (dead),] but they \bluebold{resurrect} him again, \bluebold{make} him \bluebold{live}’ \textstyleExampleSource{[Elicited BR131103.005]}
\z


Causatives with agentive bases are unacceptable. This is due to the fact that \textitbf{biking}{}-causatives focus on the causer’s manipulation of circumstances itself while the causee has no control. This is illustrated with the unacceptable \textitbf{biking}{}-causatives in ({\ref{Example_11.66}) and ({\ref{Example_11.67}), which are formed with \isi{monovalent} dynamic \textitbf{duduk} ‘sit’ and \isi{bivalent} \textitbf{makang} ‘eat’, respectively. The two elicited examples contrast with the corresponding \textitbf{kasi}{}-causatives in ({\ref{Example_11.56}) and ({\ref{Example_11.60}).


\begin{styleExampleTitle}
Causatives with \isi{monovalent} and \isi{bivalent} agentive bases
\end{styleExampleTitle}

\ea
\label{Example_11.66}
\gll {*} {sa} {di} {bawa,} {Roni} {\bluebold{biking}} {\bluebold{duduk}} {sa} {di} {atas}\\ %
 {}  \textsc{1sg}  at  bottom  Roni  make  sit  \textsc{1sg}  at  top\\
\glt 
Intended reading: ‘I was down (in the cargo area, but) Roni \bluebold{made} me \bluebold{sit} on top (of the cab)’ \textstyleExampleSource{[Elicited BR131103.006]}
\z

\ea
\label{Example_11.67}
\gll {*} {saya} {\bluebold{biking}} {\bluebold{makang}} {anjing} {deng} {papeda}\\ %
 {}  \textsc{1sg}  make  eat  dog  with  sagu.porridge\\
\glt
Intended reading: ‘I \bluebold{made} the dogs \bluebold{eat} papeda’ \textstyleExampleSource{[Elicited BR131103.009]}
\z

\subsection{Lexical causatives}
\label{Para_11.2.2}
Generally speaking, lexical causatives “are in a suppletive relation with their non-{caus\-a\-tive} counterparts” \citep[887]{Kulikov.2001}. That is, the notion of causation is encoded in the semantics of the \isi{causative} \isi{verb} itself and not in an additional morpheme as in \isi{syntactic causative} constructions.



For Papuan Malay, this suppletive relation is illustrated with the lexical causa\-tives \textitbf{bunu} ‘kill’ and \textitbf{tebang} ‘fell’ in ({\ref{Example_11.68}) and ({\ref{Example_11.70}), and their respective non-\isi{causative} counterparts \textitbf{mati} ‘die’ and \textitbf{jatu} ‘fall’ in ({\ref{Example_11.68}) and ({\ref{Example_11.69}), respectively.


\ea
\label{Example_11.68}
\gll {{de}} {{bisa}} {{jalang}} {gigit,} {{\bluebold{bunu}}} {manusia,} {sperti} {ular,}  de  {bisa}  {gigit,}  {orang}  {\bluebold{mati}}\\ %
 {\textsc{3sg}}  {be.able}  {walk}  bite  {kill}  human.being  similar.to  snake  \textsc{3sg}  {be.able}  {bite}  {person}  {die}\\
\glt 
[About an evil spirit:] ‘he/she can go (and) bite (and) \bluebold{kill} humans like a snake, he/she can bite (and) someone \bluebold{dies}’ \textstyleExampleSource{[081006-022-CvEx.0133]}
\z

\ea
\label{Example_11.69}
\gll {\ldots} {itu} {yang} {monyet} {\bluebold{jatu}} {dari} {atas}\\ %
 {}  \textsc{d.dist}  \textsc{rel}  monkey  fall  from  top\\
\glt 
‘{\ldots} that’s why the monkey \bluebold{fell} off from the top (of the banana plant)’ \textstyleExampleSource{[081109-002-JR.0005]}
\z

\ea
\label{Example_11.70}
\gll {mo} {\bluebold{tebang}} {sagu}\\ %
 want  fell  sago\\
\glt
‘(I) want to \bluebold{fell} a sago tree’ \textstyleExampleSource{[081014-006-CvPr.0069]}
\z

\subsection{Periphrastic {causative} constructions}
\label{Para_11.2.3}
Cross-linguistically, \isi{periphrastic causative} constructions are defined as constructions which involve two predicates: (1) a “matrix predicate” which “contains the notion of causation”, the “predicate of cause”, and (2) an embedded predicate which “expresses the effect of the \isi{causative} situation”, the “predicate of effect” \citep[159–160]{Payne.1997}.



Papuan Malay \isi{periphrastic causative} constructions are presented in ({\ref{Example_11.71}) to ({\ref{Example_11.74}). The matrix \isi{verb} is \textitbf{kasi} ‘give’ in ({\ref{Example_11.71}) and ({\ref{Example_11.72}), and \textitbf{biking} ‘make’ in ({\ref{Example_11.73}) and ({\ref{Example_11.74}). Besides, Papuan Malay forms periphrastic causatives with a wide range of speech verbs; they are not further discussed here.


\ea
\label{Example_11.71}
\gll {kalo} {de} {minta} {kesembuang,} {setang} {\bluebold{kasi}} {\bluebold{de}} {\bluebold{sembu}}\\ %
 if  \textsc{3sg}  ask  recovery  evil.spirit  give  \textsc{3sg}  be.healed\\
\glt 
‘when she asks for recovery, the evil spirit \bluebold{has her healed}’ \textstyleExampleSource{[081006-023-CvEx.0082]}
\z

\ea
\label{Example_11.72}
\gll {\ldots} {baru} {mo} {biking} {papeda} {\bluebold{kasi}} {\bluebold{ana{\Tilde}ana}} {\bluebold{makang}}\\ %
 {}  and.then  want  make  sagu.porridge  give  \textsc{rdp}{\Tilde}child  food\\
\glt 
‘[they said (they) wanted to catch chickens,] and then (they) wanted to make sagu porridge (and) \bluebold{have the children eat}’ \textstyleExampleSource{[081010-001-Cv.0191]}
\z

\ea
\label{Example_11.73}
\gll {de} {pu} {swami} {\bluebold{biking}} {\bluebold{de}} {\bluebold{sakit}} {\bluebold{hati}} {to?}\\ %
 \textsc{3sg}  \textsc{poss}  husband  make  \textsc{3sg}  be.sick  liver  right?\\
\glt 
‘her husband \bluebold{made her feel miserable}, right?’ \textstyleExampleSource{[081025-006-Cv.0161]}
\z

\ea
\label{Example_11.74}
\gll {kata} {itu} {tu} {yang} {\bluebold{biking}} {\bluebold{sa}} {\bluebold{bertahang}}\\ %
 word  \textsc{d.dist}  \textsc{d.dist}  \textsc{rel}  make  \textsc{1sg}  hold(.out/back)\\
\glt
‘(it was) those very words that \bluebold{made me hold out}’ \textstyleExampleSource{[081115-001a-Cv.0234]}
\z

\subsection{Summary}
\label{Para_11.2.4}
Papuan Malay employs three different strategies to express the notion of causation: syntactic, periphrastic, and lexical causatives. The description of causation focused on the syntax and semantics of syntactic causatives. Lexical and periphrastic causatives were discussed only briefly.



Papuan Malay syntactic causatives are monoclausal V\textsubscript{1}V\textsubscript{2} constructions in \linebreak which a \isi{causative} \isi{verb} V\textsubscript{1}, namely \isi{trivalent} \textitbf{kasi} ‘give’ or \isi{bivalent} \textitbf{biking} ‘make’, encodes the notion of cause while the V\textsubscript{2} denotes the notion of effect. Syntactic causatives have \isi{monovalent} or \isi{bivalent} bases. In causatives with \isi{monovalent} bases, the original A is demoted from its intransitive S function to the transitive P function, while the incoming causer takes the transitive A function. In causatives with \isi{bivalent} bases, the original A is demoted to the P function while the original P is moved out of the core into an oblique slot. Hence, in causatives with \isi{monovalent} bases the grammatical relations correspond to those established by \citet[176]{Comrie.1989}, whereas in causatives with \isi{bivalent} bases they do not correspond.



Semantically, causatives with \textitbf{kasi} ‘give’ focus on the outcome of the manipulation, whereas causatives with \textitbf{biking} ‘make’ focus on the manipulation of the circumstances itself, which results in the effect. Both \isi{causative} verbs typically generate causer-controlled causatives, in which the causer controls the event while the causee has no agentivity. This applies especially to \textitbf{biking}{}-causatives which stress the manipulation itself. Causatives with \textitbf{kasi} ‘give’, however, can also receive an assistive, rather than the typical directive, reading. This applies to agentive \isi{monovalent} or \isi{bivalent} bases when the causee has some level of agentivity.



Most \isi{causative} constructions in the corpus are formed with \textitbf{kasi} ‘give’ while causatives with \textitbf{biking} ‘make’ are much fewer. \tabref{Table_11.4} lists the type and token frequencies for both \isi{causative} verbs in the corpus.


\begin{table}
\caption{Frequencies of \isi{causative} constructions}\label{Table_11.4}

\begin{tabular}{l*{11}{r}}
\lsptoprule
 & \multicolumn{5}{c}{ \textitbf{kasi} ‘give’} & & \multicolumn{5}{c}{ \textitbf{biking} ‘make’}\\
Base & \multicolumn{2}{c}{Type \# / \%} & & \multicolumn{2}{c}{Token \# / \%} & & \multicolumn{2}{c}{Type \# / \%} & & \multicolumn{2}{c}{Token \# / \%}\\
\midrule

\textsc{v.mo(st)} &  24 &  30 &  &  36 &  8 &   & 16 &  100 &  &  25 &  100\\
\textsc{v.mo}(\textsc{dy}) &  18 &  22 &   & 115 &  24 &  &   --  &   --  &   &  --  &   -- \\
\textsc{v.bi} &  39 &  48 & &   327 &  68  & &   --  &   --  &  &   --  &   -- \\
\midrule
Total &  81 &  100 & &   478 &  100 &  &  16 &  100 &  &  25 &  100\\
\lspbottomrule
\end{tabular}
\end{table}

\newpage
In the corpus, \textitbf{kasi} ‘give’ is used most often with \isi{bivalent} bases. Less often, \textitbf{kasi} ‘give’ occurs with \isi{monovalent} bases, which can be agentive or non-agentive. Most \isi{monovalent} bases are dynamic, whereas stative bases, which are mostly non-agentive, are much rarer. Most \isi{monovalent} dynamic bases, in turn, are agentive, while non-agentive dynamic bases are rare. By contrast, \textitbf{biking} ‘make’ always takes \isi{monovalent} bases which are typically stative and non-agentive. Causatives with \isi{monovalent} non-agentive dynamic bases are also possible, although they are unattested in the corpus. Causatives with \isi{monovalent} agentive bases are only possible if the causee is inanimate or animate but helpless. \tabref{Table_11.5} shows these distributional patterns.


\begin{table}
\caption{Properties of \isi{causative} constructions}\label{Table_11.5}

\begin{tabularx}{\textwidth}{lllX}
\lsptoprule

 Base & Agentivity & \textitbf{kasi} ‘give’ &  \textitbf{biking} ‘make’\\
 \midrule
\textsc{v.mo}(\textsc{st}) & \textsc{non-agt} & Less often & Most often\\
\textsc{v.mo}(\textsc{dy}) & \textsc{non-agt} & Rarely & Possible although unattested\\
\textsc{v.mo}(\textsc{dy}) & \textsc{agt} & Less often & Possible with inanimate or with animate but helpless causees although unattested\\
\textsc{v.bi} & \textsc{agt} & Most often & Unacceptable\\
\lspbottomrule
\end{tabularx}
\end{table}

\section{Reciprocal clauses}
\label{Para_11.3}
Papuan Malay employs two different strategies to express reciprocal relations: a syntactic strategy with the dedicated \isi{reciprocity marker} \textitbf{baku} ‘\textsc{recp}’, and a lexical strategy.



Generally speaking, reciprocal clauses describe situations “in which two participants equally act upon each other” \citep[181]{Payne.1997}, with the two participants performing “two identical semantic roles” \citep[6]{Nedjalkov.2007b}. That is, in reciprocal clauses “two subevents are shown as one event or situation” by presenting two predications as one (\citeyear*[7]{Nedjalkov.2007b}).



Cross-linguistically, four major strategies of encoding the notion of reciprocity structurally are distinguished, according to \citet[9–16]{Nedjalkov.2007b}: syntactic, morphological, clitic, and lexical constructions.\footnote{\citet[10]{Nedjalkov.2007b} groups syntactic, morphological, and clitic reciprocal constructions together as grammatical or derived reciprocals.} Syntactic reciprocals are formed with reciprocal pronouns or reciprocal adverbs. Morphological reciprocals are formed by means of \isi{affixation}, \isi{reduplication}, \isi{compounding}, or periphrastic constructions involving an auxiliary.



Papuan Malay syntactic reciprocals are discussed in §\ref{Para_11.3.1}, and lexical reciprocals are briefly mentioned in §\ref{Para_11.3.2}. Morphological and clitic reciprocal constructions are unattested.


\subsection{Syntactic reciprocals}
\label{Para_11.3.1}
Papuan Malay forms syntactic reciprocals with the dedicated \isi{reciprocity marker} \textitbf{baku} ‘\textsc{recp}’. A typical example is given in ({\ref{Example_11.75}).


\begin{styleExampleTitle}
Papuan Malay \isi{reciprocity marker} \textitbf{baku} ‘\textsc{recp}’
\end{styleExampleTitle}

\ea
\label{Example_11.75}
\gll {kitong} {dua} {\bluebold{baku}} {\bluebold{melawang}} {gara-gara} {ikang}\\ %
 \textsc{1pl}  two  \textsc{recp}  oppose  because  fish\\
\glt 
‘the two of us are \bluebold{fighting each other} because of the fish’ \textstyleExampleSource{[081109-011-JR.0008]}
\z


The corpus contains 101 reciprocal clauses formed with 42 different verbs. The vast majority are \isi{bivalent}: 37 verbs (88\%) accounting for 95 tokens (94\%). One \isi{reciprocal clause} is formed with \isi{trivalent} \textitbf{ceritra} ‘tell’. The remaining four verbs are \isi{monovalent} dynamic (accounting for five tokens) (for details see §\ref{Para_11.3.1.1}).



Structurally, Papuan Malay uses two different types of syntactic reciprocals: (1) a “\isi{simple reciprocal} construction” (§\ref{Para_11.3.1.1}), and (2) a “discontinuous construction” (§\ref{Para_11.3.1.2}), using \citegen[27–30]{Nedjalkov.2007b} terminology. In simple reciprocals \textitbf{baku} ‘\textsc{recp}’ can receive a reciprocal or a sociative reading, while in discontinuous reciprocals the marker always receives a reciprocal reading.



Cross-linguistically, the \isi{reciprocity marker} is classified in different ways; in some languages it is classified as a \isi{pronoun} or an ad\isi{verb}, in others as an affix or an auxiliary (see \citeauthor{Nedjalkov.2007b}'s \citeyear*[9–16]{Nedjalkov.2007b} above-mentioned distinction of syntactic and morphological reciprocals). As for the Papuan Malay \isi{reciprocity marker}, this grammar analyzes \textitbf{baku} ‘\textsc{recp}’ as an independent word and not as an affix, without, however, further specifying its morphosyntactic status at this point. This analysis as a separate word is based on the fact that \textitbf{baku} ‘\textsc{recp}’ can be reduplicated, as shown in ({\ref{Example_11.76}). Affixes, by contrast, are not reduplicated in Papuan Malay, as discussed in §\ref{Para_4.1}.


\begin{styleExampleTitle}
Reduplication of \textitbf{baku} ‘\textsc{recp}’
\end{styleExampleTitle}

\ea
\label{Example_11.76}
\gll {itu} {sampe} {tong} {\bluebold{baku{\Tilde}baku}} {tawar} {ini} {deng} {doseng}\\ %
 \textsc{d.dist}  until  \textsc{1pl}  \textsc{rdp}{\Tilde}\textsc{recp}  bargain  \textsc{d.prox}  with  lecturer\\
\glt 
‘it got to the point that we and the lecturer were arguing \bluebold{constantly with each other}’ \textstyleExampleSource{[080917-010-CvEx.0177]}
\z


This analysis of \textitbf{baku} ‘\textsc{recp}’ as an independent word is also applied by \citet[24]{Donohue.2003}, while other researchers, such as \citet[324]{vanVelzen.1995}, treat the \isi{reciprocity marker} as a prefix. For most of the other \ili{eastern Malay varieties}, the \isi{reciprocity marker} is also treated as a prefix, namely for \ili{Ambon Malay} \citep[101–105]{vanMinde.1997}, \ili{Banda Malay} \citep[250]{Paauw.2009}, \ili{Kupang Malay} \citep[46]{Steinhauer.1983}, \ili{Manado Malay} \citep[23]{Stoel.2005}, and North Moluccan / \ili{Ternate Malay} (\citealt[19]{Taylor.1983}; \citealt[4]{Voorhoeve.1983}; \citealt[130–133]{Litamahuputty.2012}).


\subsubsection[Simple reciprocal constructions]{Simple reciprocal constructions}
\label{Para_11.3.1.1}
Most reciprocal constructions in the corpus (86/101 – 85\%) are “simple reciprocals”. In such a construction, both participants are encoded as the clausal subject, which is called the “reciprocator”, following \citegen[2092]{Haspelmath.2007} terminology.\footnote{\citet[6]{Nedjalkov.2007b} uses the term “reciprocant” rather than “reciprocator”.} Hence, the typical structure for simple reciprocals is ``\textsc{reciprocator} \textitbf{baku} \textsc{v}'', as shown in ({\ref{Example_11.77}) to ({\ref{Example_11.85}). The reciprocator can be a coordinate \isi{noun} phrase such as \textitbf{nona{\Tilde}nona ana laki{\Tilde}laki} ‘the girls (and) boys’ in ({\ref{Example_11.77}), or a plural personal \isi{pronoun} such as \textitbf{kamu} ‘\textsc{2pl}’ in ({\ref{Example_11.78}).



In ``\textsc{reciprocator} \textitbf{baku} \textsc{v}'' constructions, \textitbf{baku} ‘\textsc{recp}’ can receive a reciprocal reading in the sense of ``\textsc{reciprocator} \textsc{v} each other'', or a sociative reading in the sense of ``\textsc{reciprocator} \textsc{v} together''.



``\textsc{reciprocator} \textitbf{baku} \textsc{v}'' constructions with a reciprocal reading are characterized by a reduction in syntactic \isi{valency}, which corresponds to the reduction in semantic \isi{valency}: with both participants being encoded by the clausal subject, the object that typically encodes a second participant is deleted. This is shown in ({\ref{Example_11.77}) to ({\ref{Example_11.82}); reciprocals with a sociative reading are given in ({\ref{Example_11.83}) to ({\ref{Example_11.85}).



Typically, the \isi{verbal base} in a ``\textsc{reciprocator} \textitbf{baku} \textsc{v}'' construction is \isi{bivalent} (80/86 reciprocals – 93\%); the corpus also contains one \isi{reciprocal construction} formed with \isi{trivalent} \textitbf{ceritra} ‘tell’. Examples are given in ({\ref{Example_11.77}) to ({\ref{Example_11.79}). These examples show that the bases can have reciprocal/bidirectional semantics such as \textitbf{cium} ‘kiss’ in ({\ref{Example_11.77}), or nonreciprocal/unidirectional semantics such as \textitbf{benci} ‘hate’ in ({\ref{Example_11.78}). (Reciprocals with \isi{monovalent} bases are presented in ({\ref{Example_11.80}) and ({\ref{Example_11.81}).)


\begin{styleExampleTitle}
``\textsc{reciprocator} \textitbf{baku} \textsc{v}'' constructions with \isi{bivalent} verbs: Reciprocal reading
\end{styleExampleTitle}

\ea
\label{Example_11.77}
\gll {{nona{\Tilde}nona,}} {{ana}} {{laki{\Tilde}laki}} {\bluebold{baku}} {\bluebold{pacar}} {di} {pinggir}\\ %
 {\textsc{rdp}{\Tilde}girl}  {child}  {\textsc{rdp}{\Tilde}husband}  \textsc{recp}  date  at  edge\\
\gll  skola  {\ldots}  {\bluebold{baku}}  {\bluebold{cium}}  di  {pinggir{\Tilde}pinggir}\\
 school {}   {\textsc{recp}}  {kiss}  at  {\textsc{rdp}{\Tilde}edge}\\
\glt 
‘the girls (and) boys are \bluebold{courting each other} at the edge of the school (grounds), {\ldots} (they) are \bluebold{kissing each other} at the edges (of the school grounds)’ \textstyleExampleSource{[081115-001a-Cv.0017]}
\z

\ea
\label{Example_11.78}
\gll {kamu} {tida} {bole} {\bluebold{baku}} {\bluebold{benci},} {tida} {bole} {\bluebold{baku}} {\bluebold{mara}}\\ %
 \textsc{2pl}  \textsc{neg}  may  \textsc{recp}  hate  \textsc{neg}  may  \textsc{recp}  feel.angry(.about)\\
\glt 
‘you must not \bluebold{hate each other}, (you) must not \bluebold{feel angry with each other}’ \textstyleExampleSource{[081115-001a-Cv.0271]}
\z

\ea
\label{Example_11.79}
\gll {Markus} {deng} {Yan} {dong} {\bluebold{baku}} {\bluebold{ceritra}}\\ %
 Markus  with  Yan  \textsc{3sg}  \textsc{recp}  tell\\
\glt 
‘they Markus and Yan were \bluebold{talking to each other}’ \textstyleExampleSource{[Elicited BR130601.001]}\footnote{The corpus contains one \isi{reciprocal construction} formed with \isi{trivalent} \textitbf{ceritra} ‘tell’, similar to the elicited one in (\ref{Example_11.79}). For the most part, however, the original utterance it unclear, as the speaker mumbles.} \\
\z


``\textsc{reciprocator} \textitbf{baku} \textsc{v}'' constructions with \isi{monovalent} dynamic bases are also possible, but rare. Of the attested 86 simple reciprocals, only five are formed with \isi{monovalent} verbs (6\%), namely with \textitbf{bertengkar} ‘quarrel’ (1 token), \textitbf{saing} ‘compete’ (1 token), \textitbf{tampil} ‘perform’ (2 tokens), and \textitbf{tanding} ‘compete’ (1 token) (none of the four verbs occurs in a discontinuous \isi{reciprocal construction}). Examples are given for \textitbf{saing} ‘compete’ in ({\ref{Example_11.80}) and for \textitbf{tanding} ‘compete’ in ({\ref{Example_11.81}).


\begin{styleExampleTitle}
``\textsc{reciprocator} \textitbf{baku} \textsc{v}'' constructions with \isi{monovalent} dynamic verbs: Reciprocal reading
\end{styleExampleTitle}

\ea
\label{Example_11.80}
\glll {ade-kaka} {\bluebold{baku}} {\bluebold{saing}}\\ %
 {ySb-oSb}\\
{siblings}  {\textsc{recp}}  {compete}\\
\glt ‘the siblings were \bluebold{competing with each other}’ \textstyleExampleSource{[080919-006-CvNP.0001]}
\z

\ea
\label{Example_11.81}
\gll {dong} {ada} {brapa} {orang} {itu} {\bluebold{baku}} {\bluebold{tanding}} {rekam}\\ %
 \textsc{3pl}  exist  several  person  \textsc{d.dist}  \textsc{recp}  compete  record\\
\glt 
‘they were (indeed) several people (who) were \bluebold{competing with each other} to record (their songs)’ \textstyleExampleSource{[080923-016-CvNP.0006]}
\z


Most of the verbs used in reciprocal clauses in the corpus also occur in nonreciprocal transitive clauses (38/42 verbs). This is illustrated with \textitbf{gendong} ‘hold’ in ({\ref{Example_11.82}). The remaining four verbs are only used in reciprocal constructions: \isi{bivalent} \textitbf{ancam} ‘threaten’ (1 token) and \textitbf{cium} ‘kiss’ (2 tokens), and \isi{monovalent} \textitbf{bertengkar} ‘quarrel’ (1 token) and \textitbf{tanding} ‘compete’ (1 token). Whether these verbs can also occur in nonreciprocal transitive clauses requires further investigation.


\begin{styleExampleTitle}
Reciprocal and nonreciprocal uses of verbs
\end{styleExampleTitle}

\ea
\label{Example_11.82}
\gll {{Nofela}} {{\bluebold{gendong}}} {{\bluebold{bapa}}} {ato} {{bapa}} {{yang}} {{\bluebold{gendong}}} {\bluebold{Nofela}}\\ %
 {Nofela}  {hold}  {father}  or  {father}  {\textsc{rel}}  {hold}  Nofela\\
\gll  \bluebold{deng}  {\bluebold{Siduas}}  {ka}  {\ldots}  {kitong}  {\bluebold{baku}}  {\bluebold{gendong}}  {to?}\\
 with  {Siduas}  {or}   {} {\textsc{1pl}}  {\textsc{recp}}  {hold}  {right?}\\
\glt 
[During a phone conversation between a father and his children:] ‘you (‘Nofela’) will \bluebold{hold me (‘father’)} or I (‘father’) will \bluebold{hold you (‘Nofela’) and Siduas} {\ldots} we’ll \bluebold{hold each other}, right?’ \textstyleExampleSource{[080922-001a-CvPh.0687/0695]}
\z


In the simple reciprocals presented so far, \textitbf{baku} ‘\textsc{recp}’ denotes reciprocal relations. Alternatively, though, ``\textsc{reciprocator} \textitbf{baku} \textsc{v}'' clauses can signal sociative relations in the sense of ``\textsc{reciprocator} \textsc{v} together''.



Generally speaking, the “sociative meaning (also called \isi{associative}, collective, cooperative, etc.) suggests that an action is performed jointly and simultaneously by a group of people (at least two) named by the subject [{\ldots}] and engaged in the same activity”, as \citet[33]{Nedjalkov.2007b} notes in his typology of reciprocal constructions. Reciprocals with a sociative reading are characterized by \isi{valency} retention, in that “the number of the participants increases without changing the syntactic structure” (\citeyear*[22]{Nedjalkov.2007b}).



This observation also applies to ``\textsc{reciprocator} \textitbf{baku} \textsc{v}'' constructions, as \linebreak shown in ({\ref{Example_11.83}) and ({\ref{Example_11.85}). That is, reciprocal clauses with a sociative reading are characterized by \isi{valency} retention, although the number of participants \linebreak increases.

\begin{styleExampleTitle}
``\textsc{reciprocator} \textitbf{baku} \textsc{v}'' constructions: Sociative reading
\end{styleExampleTitle}
\ea
\label{Example_11.83}
\gll {baru} {kitong} {mulay} {\bluebold{baku}} {\bluebold{ojek}}\\ %
 and.then  \textsc{1pl}  start  \textsc{recp}  take.motorbike.taxi\\
\glt 
‘and then we\bluebold{ }started \bluebold{taking motorbike taxis together}’ \textstyleExampleSource{[081002-001-CvNP.0004]}
\z
\ea
\label{Example_11.84}
\gll  kitong  mo  \bluebold{baku}  \bluebold{bagi}  \bluebold{swara}  bagemana\textup{?}\\
 \textsc{1pl}  want  \textsc{recp}  divide  voice  how\\
\glt 
[About upcoming local elections:] ‘how do we want to \bluebold{share the votes together}?’ \textstyleExampleSource{[080919-001-Cv.0165]}
\z

\ea
\label{Example_11.85}
\gll {Aksamina} {deng} {Klara} {dong} {dua} {\bluebold{baku}} {\bluebold{rampas}} {\bluebold{bola}}\\ %
 Aksamina  with  Klara  \textsc{3pl}  two  \textsc{recp}  seize  ball\\
\glt 
‘both Aksamina and Klara \bluebold{tackled the ball together}’ \textstyleExampleSource{[081006-014-Cv.0007]}
\z


Overall, the corpus contains only few ``\textsc{reciprocator} \textitbf{baku} \textsc{v}'' constructions with a sociative reading. Further research is needed to determine whether there are any formal criteria that allow ``\textsc{reciprocator} \textitbf{baku} \textsc{v}'' constructions with a reciprocal reading to be distinguished from those with a sociative reading.


\subsubsection[Discontinuous reciprocal constructions]{Discontinuous reciprocal constructions}
\label{Para_11.3.1.2}
In \isi{discontinuous reciprocal} constructions, cross-linguistically, only one of the participants is expressed as the subject, while the second participant “is a \isi{comitative} phrase”, as \citet[29]{Nedjalkov.2007b} points out. Given that the second participant is not encoded as the direct object but as a \isi{prepositional phrase}, discontinuous reciprocals result in a reduction in syntactic \isi{valency}. Hence, pragmatically and syntactically, the second, non-subject participant is “a constituent of lower [{\ldots}] status” (\citeyear*[28]{Nedjalkov.2007b}); semantically, however, it is of the same status as the subject reciprocator.



These observations also apply to Papuan Malay. In \isi{discontinuous reciprocal} constructions, the second participant, or “reciprocee”, adopting {Haspelmath’s (2007c: 2092)} {terminology, }is encoded by a \isi{prepositional phrase}. This \isi{prepositional phrase} is introduced with the \isi{comitative} \isi{preposition} \textitbf{dengang} ‘with’, with its short form \textitbf{deng} (see also §\ref{Para_10.2.1}).\footnote{\citet[8]{Nedjalkov.2007b} refers to non-subject reciprocants as “co-participants”.} Hence, the structure for discontinuous reciprocals is ``\textsc{reciprocator} \textitbf{baku} \textsc{v} \textitbf{dengang} \textsc{reciprocee}''.



In the corpus, however, discontinuous constructions occur much less often than simple ones; only 15 of the 101 reciprocals are discontinuous (15\%). All of them designate reciprocal relations in the sense of ``\textsc{reciprocator} \textsc{v} with \textsc{reciprocee}'', literally ``\textsc{recipro\-cator} \textsc{v} each other with \textsc{reciprocee}''. Unlike the simple reciprocals in §\ref{Para_11.3.1.1}, discontinuous constructions do not express sociative relations.



In most of the discontinuous reciprocals (10/15 – 67\%), the reciprocee is mentioned overtly, as in ({\ref{Example_11.86}) to ({\ref{Example_11.88}). (For discontinuous constructions with omitted reciprocee see the examples in ({\ref{Example_11.89}) and ({\ref{Example_11.90}).)


\begin{styleExampleTitle}
``\textsc{reciprocator} \textitbf{baku} \textsc{v} \textitbf{deng(ang)} \textsc{reciprocee}'' constructions
\end{styleExampleTitle}

\ea
\label{Example_11.86}
\gll {\ldots} {ko} {laki{\Tilde}laki} {bisa} {\bluebold{baku}} {\bluebold{dapat}} {\bluebold{deng}} {bapa}\\ %
  {} \textsc{2sg}  \textsc{rdp}{\Tilde}husband  be.able  \textsc{recp}  get  with  father\\
\glt 
‘[I thought,] you, a man, can \bluebold{meet with} me (‘father’)’ (Lit. ‘can \bluebold{meet each other with} father’) \textstyleExampleSource{[080922-001a-CvPh.0234]}
\z

\ea
\label{Example_11.87}
\gll  sa  tida  perna  \bluebold{baku}  \bluebold{mara}  \bluebold{deng}  orang  laing\\
 \textsc{1sg}  \textsc{neg}  ever  \textsc{recp}  feel.angry(.about)  with  person  be.different\\
\glt 
‘I never \bluebold{get angry with} other people’ (Lit. ‘\bluebold{feel angry about each other with} another person’) \textstyleExampleSource{[081110-008-CvNP.0067]}
\z

\ea
\label{Example_11.88}
\gll {\ldots} {de} {\bluebold{baku}} {\bluebold{tabrak}} {\bluebold{deng}} {Sarles}\\ %
 {}  \textsc{3sg}  \textsc{recp}  hit.against  with  Sarles\\
\glt 
‘[right then Sarles was standing by the door,] he/she (the evil spirit) \bluebold{collided with} Sarles’ (Lit. ‘\bluebold{hit against each other with} Sarles’) \textstyleExampleSource{[081025-009b-Cv.0026]}
\z


Given the lower pragmatic status of the reciprocee, it can also remain “unspecified” \citep[42]{Nedjalkov.2007b}, as in ({\ref{Example_11.89}) and ({\ref{Example_11.90}). This applies to five of the 15 discontinuous constructions in the corpus (33\%). That is, if the second participant is understood from the context, or considered irrelevant, it can be omitted together with its \isi{preposition}. In ({\ref{Example_11.89}), the omitted reciprocee \textitbf{orang} ‘person’ was mentioned earlier. In ({\ref{Example_11.90}), the omitted reciprocee ‘community’ is understood from the context, as the topic of the narrative is communal life in the village.


\begin{styleExampleTitle}
``\textsc{reciprocator} \textitbf{baku} \textsc{v} Ø'' constructions
\end{styleExampleTitle}
\ea
\label{Example_11.89}
\gll {saya} {kalo} {macang} {\bluebold{baku}} {\bluebold{pukul}} {\bluebold{Ø}} {rasa} {takut}\\ %
 \textsc{1sg}  if  variety  \textsc{recp}  hit {}   feel  feel.afraid(.of)\\
\glt 
‘(as for) me, when (I) kind of \bluebold{fight (with another person)}, I feel afraid’ (Lit. ‘\bluebold{hit each other}’) \textstyleExampleSource{[081110-008-CvNP.0066]}
\z

\ea
\label{Example_11.90}
\gll {\ldots} {dia} {dapat} {babi,} {de} {biasa} {\bluebold{baku}} {\bluebold{bagi}} {\bluebold{Ø}}\\ %
 {} \textsc{3sg}  get  pig  \textsc{3sg}  be.usual  \textsc{recp}  divide  \\
\glt
[How to be a good villager:] ‘[when he catches fish,] (when) he catches a pig, he usually \bluebold{shares} (it \bluebold{with the community)}’ (Lit. ‘\bluebold{divide each other}’) \textstyleExampleSource{[080919-004-NP.0063]}
\z

\subsection{Lexical reciprocals}
\label{Para_11.3.2}
Lexical reciprocals are, generally speaking, “words with an inherent reciprocal meaning” \citep[14]{Nedjalkov.2007b}. Therefore, they do not need to be marked with a \isi{reciprocity marker}.



Papuan Malay \isi{lexical reciprocal} are presented in ({\ref{Example_11.91}) to ({\ref{Example_11.93}). All three examples denote, what \citet[102]{Kemmer.1993}{ calls}, “naturally reciprocal events”, such as \textitbf{ketemu} ‘meet’ in ({\ref{Example_11.91}), \textitbf{nika} ‘marry’ in ({\ref{Example_11.92}), or \textitbf{cocok} ‘be suitable’ in ({\ref{Example_11.93}).


\ea
\label{Example_11.91}
\gll {sa} {\bluebold{ketemu}} {de} {di} {kampus}\\ %
 \textsc{1sg}  meet  \textsc{3sg}  at  campus\\
\glt 
‘I \bluebold{met} him on the (university) campus’ \textstyleExampleSource{[080922-003-Cv.0102]}
\z

\ea
\label{Example_11.92}
\gll {dorang} {dua} {\bluebold{nika}}\\ %
 \textsc{3pl}  two  marry.officially\\
\glt 
‘the two of them \bluebold{married}’ \textstyleExampleSource{[081110-005-CvPr.0095]}
\z

\ea
\label{Example_11.93}
\gll {kam} {dua} {\bluebold{cocok}}\\ %
 \textsc{2pl}  two  be.suitable\\
\glt
‘the two of you \bluebold{match}’ \textstyleExampleSource{[080922-004-Cv.0033]}
\z

\subsection{Summary}
\label{Para_11.3.3}

In Papuan Malay, the dedicated \isi{reciprocity marker} \textitbf{baku} ‘\textsc{recp}’ signals recipro\-city. In reciprocity clauses two predications are presented with the two subjects of each predication equally acting upon each other. The present description focused on \isi{syntactic reciprocal} constructions; lexical reciprocals were mentioned only briefly.



Two types of reciprocal constructions are attested, simple and discontinuous ones. In simple reciprocals, both participants are encoded by the clausal subject. The base is most often a \isi{bivalent} \isi{verb}, although reciprocals with \isi{monovalent} verbs are also attested. Usually, these clauses are the result of a valency-reducing operation and receive the reciprocal reading ``\textsc{reciprocator} \textsc{v} each other''. Alternatively, these constructions can receive a sociative reading in which case the \isi{reciprocal clause} is characterized by \isi{valency} retention. Further investigation is needed to determine whether there are formal criteria to distinguish the reciprocal from the sociative readings. The basic scheme for simple reciprocals is given in ({\ref{Example_11.94}).


\begin{styleExampleTitle}
Scheme for simple reciprocals
\end{styleExampleTitle}

\ea
\label{Example_11.94}
{\textsc{reciprocator}} {\textitbf{baku}} {\textsc{V}}\\ %

\z

In discontinuous reciprocals, one participant is encoded by the clausal subject while the second one, the \textsc{reciprocee}, is expressed in a \isi{prepositional phrase} introduced with \isi{comitative} \textitbf{dengang} ‘with’. This type of reciprocal also results from a valency-reducing operation and receives the reading ``\textsc{reciprocator} \textsc{v} with \textsc{reciprocee}''. The second participant can also be omitted if it is understood from the context. The basic scheme for discontinuous reciprocals is given in ({\ref{Example_11.95}).


\begin{styleExampleTitle}
Scheme for discontinuous reciprocals
\end{styleExampleTitle}

\ea
\label{Example_11.95}
{\textsc{reciprocator}} {\textitbf{baku}} {\textsc{V}} \textup{(}{\textitbf{dengang}} {\textsc{reciprocee}}\textup{)}\\ %
 \z
\section{Existential clauses}
\label{Para_11.4}
In Papuan Malay, existential clauses are formed with the existential \isi{verb} \textitbf{ada} ‘exist’. Structurally, two types of existential clauses can be distinguished: (1) intransitive clauses with one core argument and (2) transitive clauses with two core arguments.



In one-argument clauses, \textitbf{ada} ‘exist’ precedes or follows the theme expression depending on the theme’s definiteness. This clause type asserts the existence of an entity, expresses its availability, or, with \isi{definite} themes, denotes possession. In two-argument clauses, \textitbf{ada} ‘links’ the subject with the direct object. This clause type signals possession of an \isi{indefinite} possessum. One-argument clauses are described in §\ref{Para_11.4.1} and two-argument clauses in §\ref{Para_11.4.2}; §\ref{Para_11.4.3} summarizes the main points of this section. (Negation of existential clauses is discussed in §\ref{Para_13.1.1.2}.)


\subsection{One-argument existential clauses}
\label{Para_11.4.1}
In one-argument existential clauses, \textitbf{ada} ‘exist’ precedes or follows the subject, or theme expression, such that ``S \textitbf{ada}'' or ``\textitbf{ada} S''. These differences in word order serve to distinguish nonidentifiable themes from identifiable ones, as shown with the near contrastive examples in ({\ref{Example_11.96}) and ({\ref{Example_11.97}). When the theme is pragmatically \isi{indefinite} or nonidentifiable, \textitbf{ada} ‘exist’ precedes it, such that ``\textitbf{ada} S'', as in ({\ref{Example_11.96}). When the theme is \isi{definite} or identifiable, \textitbf{ada} ‘exist’ follows it, such that ``S \textitbf{ada}'', as in ({\ref{Example_11.97}).


\begin{styleExampleTitle}
One-argument existential clauses: ``\textitbf{ada} S'' versus ``S \textitbf{ada}'' word order
\end{styleExampleTitle}

\ea
\label{Example_11.96}
\gll {ke} {mari,} {ada} {\bluebold{nasi}}\\ %
 to  hither  exist  cooked.rice\\
\glt 
‘(come) here, there’s \bluebold{cooked rice}’ \textstyleExampleSource{[081006-035-CvEx.0052]}
\z

\ea
\label{Example_11.97}
\gll {\bluebold{nasi}} {ada} {itu,} {timba} {suda!}\\ %
 cooked.rice  exist  \textsc{d.dist}  spoon  already\\
\glt 
‘\bluebold{the cooked rice} is there, just spoon (it)!’ \textstyleExampleSource{[081110-002-Cv.0051]}
\z


In existential clauses with \isi{indefinite} or nonidentifiable themes, fronted \textitbf{ada} ‘exist’ has two functions, as shown in ({\ref{Example_11.98}) and ({\ref{Example_11.99}). One is to convey the existence of an entity, such that ``a \textsc{theme} exists'', as in ({\ref{Example_11.98}), where \textitbf{ada} ‘exist’ signals the existence of \textitbf{babi} ‘pig’. A second function is to signal availability in the sense of ``a \textsc{theme} is available'', as in ({\ref{Example_11.99}), where \textitbf{ada} ‘exist’ asserts the availability of \textitbf{kuskus} ‘cuscus’ and other game; see also the example in ({\ref{Example_11.96}).


\begin{styleExampleTitle}
``\textitbf{ada} S'' word order: Existence or availability of an \isi{indefinite}/nonidentifiable theme
\end{styleExampleTitle}

\ea
\label{Example_11.98}
\gll {ada} {\bluebold{babi}} {di} {situ}\\ %
 exist  pig  at  \textsc{l.med}\\
\glt 
‘there is \bluebold{a pig} there’ \textstyleExampleSource{[081006-023-CvEx.0004]}
\z

\ea
\label{Example_11.99}
\gll {maytua} {liat,} {{wa,}} {{kantong}} {{itu}} {fol,} {ada} {\bluebold{kuskus},} {ada}\\ %
 wife  see  {wow!}  {bag}  {\textsc{d.dist}}  be.full  exist  cuscus  exist\\
\gll {\bluebold{tikus-tana},}  {ada}  {\bluebold{kepiting}}  e,  {\bluebold{ketang},}  ada  {\bluebold{ikang}}\\
 {spiny.bandicoot}  {exist}  {crab}  uh  {crab}  exist  {fish}\\
\glt 
[After a successful hunt:] ‘(my) wife saw, ``wow!, that bag is full'', there was \bluebold{cuscus}, there were \bluebold{bandicoots}, there were \bluebold{crabs}, uh, \bluebold{crabs}, there were \bluebold{fish}’ \textstyleExampleSource{[080919-004-NP.0031]}
\z


In existential clauses with \isi{definite} or identifiable themes, postposed \textitbf{ada} ‘exist’ also has two functions, as demonstrated in ({\ref{Example_11.100}) and ({\ref{Example_11.101}). One function is to assert the existence of an already established theme, such that ``the \textsc{theme} exists''. This is the case in the elicited example in ({\ref{Example_11.100}), which contrasts with the \isi{existential clause} in ({\ref{Example_11.98}). This reading also applies to the examples in ({\ref{Example_11.101}) and in ({\ref{Example_11.97}).


\begin{styleExampleTitle}
``S \textitbf{ada}'' word order: Existence of a \isi{definite}/identifiable theme
\end{styleExampleTitle}

\ea
\label{Example_11.100}
	\gll {} {\bluebold{babi}} {\hspace{6pt}} {ada} {di} {situ}\\
{} pig {\hspace{6pt}} exist  at  \textsc{l.med}\\
\glt 
‘\bluebold{the pig} is there’ \textstyleExampleSource{[Elicited MY131105.004]}
\z

\ea
\label{Example_11.101}
\gll {\bluebold{saya}} {ada}\\ %
 1\textsc{sg}  exist\\
\glt 
[About a motorbike accident:] ‘\bluebold{I} am alive’ \textstyleExampleSource{[081015-005-NP.0024]}
\z


A second function of postposed \textitbf{ada} ‘exist’ is to designate possession of a \isi{definite} or identifiable possessum, as shown in ({\ref{Example_11.102}) and ({\ref{Example_11.103}). To convey the notion of possession the theme is expressed in an \isi{adnominal possessive construction}, such that ``\textsc{possessive} \textsc{np} \textsc{exist}s'' or ``\textsc{possessor} has the \textsc{possessum}''. The clause in ({\ref{Example_11.102}) asserts the known existence of \textitbf{bapa pu motor} ‘father’s motorbike’. In this \isi{adnominal possessive construction}, the \isi{possessor \isi{noun} phrase} \textitbf{bapa} ‘father’ modifies the identifiable \isi{possessum \isi{noun} phrase} \textitbf{motor} ‘motorbike’; both constituents are linked with the possessive marker \textitbf{pu} ‘\textsc{poss}’. The same applies to the clause in ({\ref{Example_11.103}) which signals possession of the \isi{definite} \isi{possessum \isi{noun} phrase} \textitbf{dana} ‘funds’. (Adnominal possessive relations are discussed in detail in \chapref{Para_9}. Possession of an \isi{indefinite} possessum is expressed with a \isi{two-argument existential clause} or with a nominal clause, as described in §\ref{Para_11.4.2} and §\ref{Para_12.2}, respectively.)


\begin{styleExampleTitle}
``S \textitbf{ada}'' word order: Possession of a \isi{definite}/identifiable theme
\end{styleExampleTitle}
\ea
\label{Example_11.102}
\gll {\bluebold{bapa}} {\bluebold{pu}} {\bluebold{motor}} {ada}\\ %
 father  \textsc{poss}  motorbike  exist\\
\glt 
[Reply to a question:] ‘\bluebold{father} had \bluebold{a motorbike}’ (Lit. ‘\bluebold{father’s motorbike} exists’) \textstyleExampleSource{[080919-002-Cv.0012]}
\z

\ea
\label{Example_11.103}
\gll {kalo} {\bluebold{sa}} {\bluebold{pu}} {\bluebold{dana}} {suda} {ada} {brarti} {sa} {undang} {\ldots}\\ %
 if  1\textsc{sg}  \textsc{poss}  fund  already  exist  mean  \textsc{1sg}  invite  \\
\glt 
[About a planned meeting:] ‘if \bluebold{I} already had \bluebold{the funds}, that means, I would invite {\ldots}’ (Lit. ‘\bluebold{my funds} already exist’) \textstyleExampleSource{[081010-001-Cv.0131]}
\z


If the theme can be inferred from the context it can also be omitted as in ({\ref{Example_11.104}). In this example, the omitted theme is \textitbf{bagiang dana} ‘funding department’. Having been presented in the previous clause, it is now omitted, which leaves \textitbf{ada} ‘exist’ as the sole constituent of the \isi{existential clause}.


\begin{styleExampleTitle}
Omitted theme expression
\end{styleExampleTitle}

\ea
\label{Example_11.104}
\gll {\bluebold{Ø}} {\bluebold{ada},} {de} {punya} {dana} {sendiri}\\ %
 {}  exist  3\textsc{sg}  have  fund  be.alone\\
\glt 
‘\bluebold{(the funding department) exists}, it has its own funding’ \textstyleExampleSource{[081010-001-Cv.0174]}
\z


Definite or identifiable existential clauses also co-occur with prepositional phrases, such as the \isi{locational} phrase \textitbf{di situ} ‘there’ in ({\ref{Example_11.105}). This clause can be analyzed in two ways. One analysis is that of an \isi{existential clause} with a \isi{locational} adjunct which gives additional information about the theme’s current location. This analysis is substantiated by the contrastive example in ({\ref{Example_11.106}), in which \textitbf{situ} ‘\textsc{l.med}’ is fronted to the clause-initial position. This possibility of fronting the \isi{prepositional phrase} is typical for adjuncts. In ({\ref{Example_11.106}) the fronting serves to emphasize the location (concerning the rather common \isi{elision} of \isi{locative} \textitbf{di} ‘at’, see §\ref{Para_10.1.5}). An alternative analysis of ({\ref{Example_11.105}) is that of a \isi{prepositional predicate clause} with progressive reading. This analysis is substantiated with the (near) contrastive examples in ({\ref{Example_11.107}) to ({\ref{Example_11.109}). The example in ({\ref{Example_11.107}) presents a non\isi{verbal clause} in which \textitbf{di situ} ‘there’ serves as the predicate. The example in ({\ref{Example_11.108}) shows how a \isi{prepositional predicate clause} can undergo aspectual \isi{modification}, as for instance with the prospective ad\isi{verb} \textitbf{masi} ‘still’. The example in ({\ref{Example_11.109}) shows the progressive-marking function of existential \textitbf{ada} ‘exist’ in verbal clauses (see also §\ref{Para_5.4.1}). When presented with both analyses, however, one of the consultants rejected the first analysis. Instead this consultant maintained that \textitbf{ada} ‘exist’ in ({\ref{Example_11.105}) has the same function as \textitbf{masi} ‘still’ in ({\ref{Example_11.108}), namely to modify the prepositional predicate \textitbf{di situ} ‘there’. The two analyses and the reading chosen by one of the consultants for the clauses in ({\ref{Example_11.105}) require further investigation.


\begin{styleExampleTitle}
Alternative readings of clauses with \isi{definite}/identifiable themes and postposed prepositional phrases
\end{styleExampleTitle}

\ea
\label{Example_11.105}
\gll {de} {\bluebold{ada}} {di} {situ,} {Martina} {\bluebold{ada}} {di} {situ}\\ %
 \textsc{3sg}  exist  at  \textsc{l.med}  Martina  exist  at  \textsc{l.med}\\
\glt 
‘she \bluebold{was (being)} there, Martina \bluebold{was (being)} there’ \textstyleExampleSource{[081109-001-Cv.0087]}
\z

\ea
\label{Example_11.106}

\gll {\ldots} {pace} {de} {tulis} {di} {kertas,} {suda,} {situ} {de} \bluebold{ada}, {de}  {su}  {biking}  {daftar}\\
  {}  {man}  \textsc{3sg}  write  {at}   paper  already  \textsc{l.med}  \textsc{3sg}  exist \textsc{3sg}  {already}  {make}  {list}\\
\glt [Enrolling for a sports team:] ‘[Herman gave his name,] the man wrote (it) on a paper, that’s it, there it \bluebold{was}!, he (the man) had already made a list’ \textstyleExampleSource{[081023-001-Cv.0001]}
\z

\ea
\label{Example_11.107}
\gll {de} {\bluebold{di}} {\bluebold{situ}}\\ %
 \textsc{3sg}  at  \textsc{l.med}\\
\glt 
‘he \bluebold{(was) there}’ \textstyleExampleSource{[080922-010a-CvNF.0256]}
\z

\ea
\label{Example_11.108}
\gll {de} {\bluebold{masi}} {di} {situ}\\ %
 \textsc{3sg}  still  at  \textsc{l.med}\\
\glt 
‘he (was) \bluebold{still} there’ \textstyleExampleSource{[Elicited MY131105.002]}
\z

\ea
\label{Example_11.109}
\gll {de} {\bluebold{ada}} {tidor} {di} {situ}\\ %
 \textsc{3sg}  exist  sleep  at  \textsc{l.med}\\
\glt
‘he \bluebold{is} sleep\bluebold{ing} there’ \textstyleExampleSource{[Elicited MY131105.003]}
\z

\subsection{Two-argument existential clauses}
\label{Para_11.4.2}
In two-argument existential clauses, \textitbf{ada} ‘exist’ links both core arguments. This type of \isi{existential clause} expresses possession of an \isi{indefinite} possessum. As shown in ({\ref{Example_11.110}) and ({\ref{Example_11.111}), the \isi{possessor \isi{noun} phrase} takes the subject slot and the \isi{possessum \isi{noun} phrase} takes the direct object slot, such that ``\textsc{possessor} \textsc{exist}s \textsc{possessum}'' or ``\textsc{possessor} has a \textsc{possessum}''. In ({\ref{Example_11.110}) \textitbf{ada} ‘exist’ links the possessor \textitbf{sa} ‘\textsc{1sg}’ with the possessum \textitbf{ana} ‘child’ which gives the possessive reading ‘I have children’. The possessum can be encoded by a bare \isi{noun} as in ({\ref{Example_11.110}), or by a \isi{noun} phrase such as \textitbf{dia punya jing} ‘her genies’ in ({\ref{Example_11.111}). (Alternatively, possession of an \isi{indefinite} possessum can be expressed with a nominal predicate; for details see §\ref{Para_12.2}. Possession of a \isi{definite} possessum is encoded by an \isi{adnominal possessive construction}; for details see \chapref{Para_9}, and also §\ref{Para_11.4.1}.)


\ea
\label{Example_11.110}
\gll {sa} {\bluebold{ada}} {\bluebold{ana},} {jadi} {sa} {kasi} {untuk} {sa} {pu} {sodara}\\ %
 1\textsc{sg}  exist  child  so  \textsc{1sg}  give  for  \textsc{1sg}  \textsc{poss}  sibling\\
\glt 
‘I \bluebold{have children}, so I gave (one) to my relative’ \textstyleExampleSource{[081006-024-CvEx.0010]}
\z

\ea
\label{Example_11.111}
\gll {prempuang} {iblis} {itu} {\bluebold{ada}} {\bluebold{dia}} {\bluebold{punya}} {\bluebold{jing}}\\ %
 woman  devil  \textsc{d.dist}  exist  3\textsc{sg}  \textsc{poss}  genie\\
\glt 
[About evil spirits taking on the form of women:] ‘that woman spirit \bluebold{has her (own) genies}’ \textstyleExampleSource{[081006-022-CvEx.0053]}
\z


Cross-linguistically, \citet{Stassen.2013} identifies five major types of predicate possession: Have-Possessive, Oblique Possessive, Genitive Possessive, Topic Possessive, and Conjunctional Possessive. In terms of this classification, the existential possessive constructions in ({\ref{Example_11.110}) and ({\ref{Example_11.111}) are best explained as Topic Possessives.\footnote{As for the remaining four types of possessive constructions, the data in the corpus indicates the following: (1) the Have-Possessive is formed with the ditransitive \isi{verb} \textitbf{punya} ‘have’, as in (\ref{Example_9.1}) in §\ref{Para_9.1} (p. \pageref{Example_9.1}), and the Genitive Possessive is used to encode possessive relations in which the possessum has a \isi{definite} reading, as in (\ref{Example_11.102}) and (\ref{Example_11.103}) in §\ref{Para_11.4.1} (p. \pageref{Example_11.102}) (see also \chapref{Para_9}). The Oblique and Conjunctional Possessives are unattested.} According to \citet[219]{Stassen.2009},


\begin{styleIIndented}
[in] a standard Topic Possessive, the possessee is the subject of the be-\isi{verb}. [{\ldots}] The possessor is constructed as a sentential topic and may or may not be marked as such, for example by sentence-initial position [{\ldots}]
\end{styleIIndented}


Following this analysis, an alternative translation for the possessive construction \textitbf{sa ada ana} ‘I have children’ in ({\ref{Example_11.110}) would be: ‘(as for) me, children exist’.


\subsection{Summary}
\label{Para_11.4.3}
In Papuan Malay, existential clauses are formed with the existential \isi{verb} \textitbf{ada} ‘exist’. Syntactically, two clause types can be distinguished: intransitive clauses with one core argument, and transitive clauses with two core arguments. \tabref{Table_11.6} gives an overview of the different constructions and their functions, with one-argument clauses given in (I) and two-argument clauses in (II).



\begin{table}[b]
\caption{Overview of \isi{existential clause} constructions}\label{Table_11.6}

\begin{tabular}{llllll}
\lsptoprule

\multicolumn{6}{l}{%\setcounter{itemize}{0}
\stepcounter{InTableCounter0} \Roman{InTableCounter0}. One-argument existential clauses} \\
\midrule
& \multicolumn{5}{l}{\stepcounter{InTableCounter1} \Alph{InTableCounter1}.  \textitbf{ada} ‘exist’ precedes an \isi{indefinite}/nonidentifiable theme}\\
&  & \textitbf{ada} \textsc{theme} & \multicolumn{2}{l}{‘a \textsc{theme} exists’} & Existence\\
&  &  & \multicolumn{2}{l}{‘a \textsc{theme} is available’} & Availability\\
\tablevspace
& \multicolumn{5}{l}{\stepcounter{InTableCounter1} \Alph{InTableCounter1}. \textitbf{ada} ‘exist’ follows a \isi{definite}/identifiable theme}\\
&  & \textsc{theme} \textitbf{ada} & \multicolumn{2}{l}{‘the \textsc{theme} exists’} & \textitbf{\textmd{\textup{Existence}}}\\
&  &  & \multicolumn{2}{l}{‘\textsc{possessor} has the \textsc{possessum}’} & Possession\\
\midrule
\multicolumn{6}{l}{\stepcounter{InTableCounter0}
\Roman{InTableCounter0}. Two-argument existential clauses}\\
\midrule
& \multicolumn{5}{l}{Possession of an \isi{indefinite} possessum}\\
&  & \multicolumn{2}{l}{\textsc{subject} \textitbf{ada} \textsc{object}} & \multicolumn{2}{l}{‘\textsc{possessor} has a \textsc{possessum}’}\\
\lspbottomrule
\end{tabular}
\end{table}



In one-argument clauses, \textitbf{ada} ‘exist’ precedes the theme expression when this is pragmatically \isi{indefinite} or nonidentifiable, as in (I.A). This construction conveys the existence or availability of an entity. When the theme is \isi{definite} or identifiable, \textitbf{ada} ‘exist’ follows it, as in (I.A). This construction asserts the existence of an already established theme or denotes possession of a \isi{definite}/identifiable possessum. In two-argument clauses, \textitbf{ada} ‘exist’ links the subject and direct object arguments. This type of \isi{existential clause} indicates possession of an \isi{indefinite} possessum, as in (II).


\section{Comparative clauses}
\label{Para_11.5}
Papuan Malay employs two structurally distinct types of comparative constructions: \isi{degree-marking} clauses as shown in the elicited example in ({\ref{Example_11.112}), and \isi{identity-marking} clauses as illustrated in ({\ref{Example_11.113}).



Generally speaking, comparative clauses with gradable predicates involve \linebreak “two participants being compared, and the property in terms of which they are compared” \citep[788]{Dixon.2008}. The two participants being compared are the \textsc{comparee}, that is, the object of comparison, and the \textsc{standard} of comparison, in \citegen{Dixon.2008} terminology. When the standard is expressed in a \isi{prepositional phrase}, the \isi{preposition} serves as the \textsc{mark} of the comparison. The property attributed to the comparee and standard is the \textsc{parameter} of comparison. The parameter is marked with an \textsc{index} of comparison which signals the “ordering relation” between the comparee and the standard “to the degree or amount to which they possess some property” \citep[690–691]{Kennedy.2006}.


\begin{styleExampleTitle}
Degree-marking and \isi{identity-marking} comparative clauses
\end{styleExampleTitle}

\ea
\label{Example_11.112}
\glll {\textupsc{comparee}} {\textupsc{index}} {\textupsc{parameter}} {\textupsc{mark}} {\textupsc{standard}}\\ %
 {dia}  {\bluebold{lebi}}  {\bluebold{tinggi}}  {dari}  {saya}\\
 {\textsc{3sg}}  {more}  {be.high}  {from}  {\textsc{1sg}}\\
\glt 
‘he/she is \bluebold{taller} than me’ (Lit. ‘be \bluebold{more tall} from me’) \textstyleExampleSource{[Elicited BR111011.002]}
\z

\ea
\label{Example_11.113}
\glll {\textupsc{comparee}} {\textupsc{parameter}} {\textupsc{index}} {\textupsc{mark}} {\textupsc{standard}}\\ %
 de  \bluebold{sombong}  \bluebold{sama}  deng  ko\\
 \textsc{3sg}  be.arrogant  be.same  with  \textsc{2sg}\\
\glt 
‘she’ll be \bluebold{as arrogant as} you (are)’ (Lit. ‘be \bluebold{arrogant same} with you’) \textstyleExampleSource{[081006-005-Cv.0002]}
\z


Papuan Malay \isi{degree-marking} clauses, expressing the notions of superiority, as in the elicited example in ({\ref{Example_11.112}), inferiority, or superlative, are discussed in §\ref{Para_11.5.1}. Identity-marking clauses, signaling \isi{similarity}, as in ({\ref{Example_11.113}), or dis\isi{similarity}, are described in §\ref{Para_11.5.2}. Both clause types differ in terms of their word order. In \isi{degree-marking} clauses the parameter follows the index, while in \isi{identity-marking} clauses the parameter precedes the index or is omitted.


\subsection{Degree-marking comparative clauses}
\label{Para_11.5.1}
Degree-marking comparative clauses convey the notions of superiority, inferiority, and superlative in the sense of ``less than'', ``more than'' and ``most'', respectively, such that ``\textsc{comparee} is more/less/most \textsc{parameter} (than \textsc{standard})''. In this type of \isi{comparative clause}, the parameter follows the index, as illustrated in the elicited superiority clause in ({\ref{Example_11.112}). The following constituents serve as index: the \isi{grading} ad\isi{verb} \textitbf{lebi} ‘more’ signals superiority while \textitbf{paling} ‘most’ marks superlative; the \isi{bivalent} \isi{verb} \textitbf{kurang} ‘lack’ marks inferiority. The standard can be stated overtly, as in ({\ref{Example_11.114}) and ({\ref{Example_11.115}), or be omitted as in ({\ref{Example_11.116}) to ({\ref{Example_11.119}).



In clauses with an overt standard, the standard is expressed in a \isi{prepositional phrase} which is introduced with the elative \isi{preposition} \textitbf{dari} ‘from’, as illustrated in ({\ref{Example_11.114}) and in the elicited example in ({\ref{Example_11.115}). This \isi{preposition} serves as the mark of the comparison. In the corpus, however, \isi{degree-marking} clauses with an overt standard are rare. The corpus contains only two superiority clauses, one of which is given in ({\ref{Example_11.114}). Inferiority clauses with an overt standard are also possible, as shown with the elicited example in ({\ref{Example_11.115}). Superlative clauses with an overt standard are unattested.


\begin{styleExampleTitle}
Superiority and inferiority clauses with overt standard
\end{styleExampleTitle}

\ea
\label{Example_11.114}
\gll {di} {klas} {itu} {dia} {\bluebold{lebi}} {\bluebold{besar}} {dari} {smua} {ana{\Tilde}ana} {di} {dalam}\\ %
 at  class  \textsc{d.dist}  \textsc{3sg}  more  be.big  from  all  \textsc{rdp}{\Tilde}child  at  inside\\
\glt 
‘in that class he’s \bluebold{bigger} than all the (other) kids in it’ \textstyleExampleSource{[081109-003-JR.0001]}\footnote{The original recording says \textitbf{dari smuat} rather than \textitbf{dari smua} ‘than all’. Most likely the speaker wanted to say \textitbf{dari smua temang} ‘than all friends’ but cut himself off to replace \textitbf{temang} ‘friend’ with \textitbf{ana{\Tilde}ana} ‘children’.}
\z

\ea
\label{Example_11.115}
\gll {saya} {\bluebold{kurang}} {\bluebold{tinggi}} {dari} {dia}\\ %
 \textsc{1sg}  lack  be.high  from  \textsc{3sg}\\
\glt 
‘I am \bluebold{shorter} than him/her’ (Lit. ‘\bluebold{lack being tall}’) \textstyleExampleSource{[Elicited BR111011.001]}
\z


Most often, the standard is elided in \isi{degree-marking} clauses, as it is usually known from the discourse, as in the examples in ({\ref{Example_11.116}) to ({\ref{Example_11.119}). The superiority clause in ({\ref{Example_11.116}) is part of a conversation about a village mayors’ meeting which had been delayed several times. The speaker criticizes the fact that the mayors accepted this delay in spite of the fact that they had more authority than the elided standard ‘those who caused the delay’. Likewise, in ({\ref{Example_11.117}) to ({\ref{Example_11.119}) the standard of comparison is known from the preceding discourse. Besides, the example in ({\ref{Example_11.119}) shows that a superlative comparison can be reinforced with the degree ad\isi{verb} \textitbf{skali} ‘very’.


\begin{styleExampleTitle}
Degree-marking clauses with omitted standard
\end{styleExampleTitle}

\ea
\label{Example_11.116}
\gll {kam} {punya} {fungsi} {wewenang} {\bluebold{lebi}} {\bluebold{besar}} { Ø}\\ %
 \textsc{2pl}  \textsc{poss}  function  authority  more  be.big  \\
\glt 
[About a mayors’ meeting:] ‘your function (and) authority is \bluebold{bigger} (than that of those who caused the delay)’ \textstyleExampleSource{[081008-003-Cv.0056]}
\z

\ea
\label{Example_11.117}
\gll {\ldots} {karna} {itu} {\bluebold{kurang}} {\bluebold{bagus}} { Ø}\\ %
 {}   because  \textsc{d.dist}  lack  be.good  \\
\glt 
‘{\ldots} because those (old ways) are \bluebold{less good} (than our new ways)’ (Lit. ‘\bluebold{lack being good}’) \textstyleExampleSource{[080923-013-CvEx.0010]}
\z

\ea
\label{Example_11.118}
\gll {puri} {tu} {\bluebold{paling}} {\bluebold{besar}} { Ø}\\ %
 anchovy-like.fish  \textsc{d.dist}  most  be.big  \\
\glt 
‘that anchovy-like fish is \bluebold{the biggest} (among the larger pile of fish)’ \textstyleExampleSource{[080927-003-Cv.0002]}
\z

\ea
\label{Example_11.119}
\gll {Aris} {\bluebold{paling}} {\bluebold{tinggi}} {skali} { Ø}\\ %
 Aris  most  be.high  very  \\
\glt 
‘Aris is \bluebold{the very tallest} (among the two of you)’ \textstyleExampleSource{[080922-001b-CvPh.0026]}
\z


In the corpus, inferiority clauses formed with \textitbf{kurang} ‘lack’ occur much less often than superiority clauses with \textitbf{lebi} ‘more’. Instead of stating that the comparee is inferior to the standard in terms of a specific quality, as in the elicited example in ({\ref{Example_11.115}), repeated as ({\ref{Example_11.120}), speakers prefer to use a superiority clause which asserts that the comparee is superior to the standard, as in the elicited example in ({\ref{Example_11.112}), repeated as ({\ref{Example_11.121}).


\begin{styleExampleTitle}
Inferiority versus superiority clauses
\end{styleExampleTitle}

\ea
\label{Example_11.120}
\gll {saya} {\bluebold{kurang}} {\bluebold{tinggi}} {dari} {dia}\\ %
 \textsc{1sg}  lack  be.high  from  \textsc{3sg}\\
\glt 
‘I am \bluebold{shorter} than him/her’ (Lit. ‘\bluebold{lack being tall}’) \textstyleExampleSource{[Elicited BR111011.001]}
\z

\ea
\label{Example_11.121}
\gll {dia} {\bluebold{lebi}} {\bluebold{tinggi}} {dari} {saya}\\ %
 \textsc{3sg}  more  be.high  from  \textsc{1sg}\\
\glt 
‘he/she is \bluebold{taller} than I am’ \textstyleExampleSource{[Elicited BR111011.002]}
\z


Alternatively, the attested inferiority clauses could be interpreted as instances of mitigation used for politeness. This mitigating function is also illustrated with the inferiority clauses in ({\ref{Example_11.122}) and ({\ref{Example_11.123}): the speakers assert that the respective referents possess less of the positive qualities of being \textitbf{ajar} ‘taught, educated’ or \textitbf{hati{\Tilde}hati} ‘careful’, instead of stating that they are ‘impolite’ or ‘careless’.


\begin{styleExampleTitle}
Inferiority clauses: Mitigation function
\end{styleExampleTitle}

\ea
\label{Example_11.122}
\gll {Klara} {\bluebold{kurang}} {\bluebold{ajar}}\\ %
 Klara  lack  teach\\
\glt 
‘Klara was \bluebold{impolite}’ (Lit. ‘\bluebold{lack being educated}’) \textstyleExampleSource{[081025-009a-Cv.0045]}
\z

\ea
\label{Example_11.123}
\gll {itu} {karna} {\bluebold{kurang}} {\bluebold{hati{\Tilde}hati}}\\ %
 \textsc{d.dist}  because  lack  \textsc{rdp}{\Tilde}liver\\
\glt 
‘that (happened) because (I) was \bluebold{careless}’ (Lit. ‘\bluebold{lack being careful}’) \textstyleExampleSource{[081011-017-Cv.0009]}
\z


For the most part, mitigating inferiority constructions are fixed expressions, such as the \textitbf{kurang} ‘lack’ constructions presented in ({\ref{Example_11.117}), ({\ref{Example_11.122}) and ({\ref{Example_11.123}).



Superlative constructions have the additional function of expressing ``high degrees of parameter'', as illustrated in ({\ref{Example_11.124}) and ({\ref{Example_11.125}). In ({\ref{Example_11.124}), the superlative construction \textitbf{paling emosi} ‘feel most angry (about)’ conveys that the speaker was ‘very very angry’. Likewise in ({\ref{Example_11.125}), the superlative construction signals ``high degrees of parameter''. The superlative clauses in ({\ref{Example_11.124}) and ({\ref{Example_11.125}) do not involve a comparison, unlike the superlative constructions in ({\ref{Example_11.118}) and ({\ref{Example_11.119}).

\begin{styleExampleTitle}
Superlative clauses: ‘High degrees of parameter’
\end{styleExampleTitle}

\ea
\label{Example_11.124}
\gll {\bluebold{paling}} {\bluebold{emosi}}\\ %
 most  feel.angry(.about)\\
\glt 
‘(I) \bluebold{felt very very angry}’ (Lit. ‘\bluebold{most angry}’) \textstyleExampleSource{[081025-009a-Cv.0154]}
\z

\ea
\label{Example_11.125}
\gll {de} {\bluebold{paling}} {\bluebold{takut}}\\ %
 \textsc{3sg}  most  feel.afraid(.of)\\
\glt 
‘he \bluebold{felt very very afraid}’ (Lit. ‘\bluebold{feel most afraid}’) \textstyleExampleSource{[081115-001a-Cv.0060]}
\z


In summary, the scheme for \isi{degree-marking} comparative constructions in Papuan Malay is ``\textsc{comparee} – \textsc{index} – \textsc{parameter} (– \textsc{mark} – \textsc{standard})''.


\subsection{Identity-marking comparative clauses}
\label{Para_11.5.2}
Identity-marking comparative clauses express \isi{similarity} or dis\isi{similarity} between a comparee and a standard, in the sense of ``same as'' or ``different from'', respectively. In this type of \isi{comparative clause}, the index follows the parameter, as illustrated with the \isi{similarity} clause in ({\ref{Example_11.113}), repeated as ({\ref{Example_11.126}).


\begin{styleExampleTitle}
Identity-marking comparative clauses
\end{styleExampleTitle}

\ea
\label{Example_11.126}
\glll {\textupsc{comparee}} {\textupsc{parameter}} {\textupsc{index}} {\textupsc{mark}} {\textupsc{standard}}\\ %
 de  \bluebold{sombong}  \bluebold{sama}  deng  ko\\
 \textsc{3sg}  be.arrogant  be.same  with  \textsc{2sg}\\
\glt 
‘she’ll be \bluebold{as arrogant as} you (are)’ \textstyleExampleSource{[081006-005-Cv.0002]}
\z


Similarity comparisons are presented in ({\ref{Example_11.127}) to ({\ref{Example_11.132}) and dis\isi{similarity} comparisons in ({\ref{Example_11.133}) to ({\ref{Example_11.136}).



In \isi{similarity} clauses, the index is the stative \isi{verb} \textitbf{sama} ‘be same’, and the mark is the \isi{comitative} \isi{preposition} \textitbf{dengang} ‘with’, with its short form \textitbf{deng}. The standard can be encoded in two ways. One option is to express it in a \isi{prepositional phrase}, as in ({\ref{Example_11.126}) to ({\ref{Example_11.128}); the second possibility is illustrated in ({\ref{Example_11.129}) to ({\ref{Example_11.131}). In the \isi{similarity} comparison in ({\ref{Example_11.127}), the comparee and standard are considered to be similar in terms of a specific property, such that ``\textsc{comparee} is as \textsc{parameter} as \textsc{standard}''. If, however, the parameter is known from the context, it can be omitted, such that ``\textsc{comparee} is the same as \textsc{standard} (in terms of an understood \textsc{parameter})'', as in ({\ref{Example_11.128}) where \textitbf{de} ‘\textsc{3sg}’ is the \textsc{comparee} and \textitbf{kitong} ‘\textsc{2pl}’ is the \textsc{standard}.


\begin{styleExampleTitle}
Similarity clauses: Standard is expressed in a \isi{prepositional phrase}
\end{styleExampleTitle}

\ea
\label{Example_11.127}
\gll {{orang}} {{itu}} {\bluebold{ganas}} {\bluebold{sama}} {deng}\\ %
 {person}  {\textsc{d.dist}}  feel.furious(.about)  be.same  with\\
\gll dong  {pu}  {penunggu}\\
 \textsc{3pl}  {\textsc{poss}}  {tutelary.spirit}\\
\glt 
‘those people were \bluebold{as ferocious as} their tutelary spirits’ \textstyleExampleSource{[081025-006-Cv.0286]}
\z

\ea
\label{Example_11.128}
\gll {de} {\bluebold{Ø}} {\bluebold{sama}} {dengang} {kitong} {juga}\\ %
 \textsc{3sg}  {}  be.same  with  \textsc{1pl}  also\\
\glt 
‘she is also \bluebold{the same as} we are (in terms of \bluebold{being foreign})’ \textstyleExampleSource{[081010-001-Cv.0061]}
\z


Alternatively, the standard can be encoded as the clausal subject together with the comparee, such that ``\textsc{comparee} \& \textsc{standard} are equally \textsc{parameter}'', as in ({\ref{Example_11.129}) to ({\ref{Example_11.131}). The standard and comparee can be encoded by a coordinate \isi{noun} phrase, as in ({\ref{Example_11.129}), or a plural personal \isi{pronoun}, as in ({\ref{Example_11.130}). Again, the parameter can be omitted if it is understood from the context, such that ``\textsc{comparee} \& \textsc{standard} are the same (in terms of an understood \textsc{parameter})'', as in ({\ref{Example_11.131}).


\begin{styleExampleTitle}
Similarity clauses: Standard is encoded as the clausal subject together with the comparee
\end{styleExampleTitle}

\ea
\label{Example_11.129}
\gll {sa} {deng} {mace} {tu} {\bluebold{cocok}} {\bluebold{sama}}\\ %
 \textsc{1sg}  with  woman  \textsc{d.dist}  be.suitable  be.same\\
\glt 
‘I and that woman are \bluebold{equally well-matched}’ \textstyleExampleSource{[081011-022-Cv.0016]}
\z

\ea
\label{Example_11.130}
\gll {kam} {dua} {pu} {mulut} {\bluebold{besar}} {\bluebold{sama}}\\ %
 \textsc{2pl}  two  \textsc{poss}  mouth  be.big  be.same\\
\glt 
‘the two of yours mouth is \bluebold{equally big}’ \textstyleExampleSource{[080922-004-Cv.0033]}
\z

\ea
\label{Example_11.131}
\gll {prempuang} {laki{\Tilde}laki} {\bluebold{Ø}} {\bluebold{sama}}\\ %
 woman  \textsc{rdp}{\Tilde}husband {}   be.same\\
\glt 
‘women (and) men are \bluebold{the same} (in terms of \bluebold{having leadership qualities})’ \textstyleExampleSource{[081011-023-Cv.0244]}
\z


Not only the parameter, but also the standard can be omitted if it is understood from the context. In ({\ref{Example_11.132}), for instance, the omitted standard is ‘the Yali children’, while the omitted parameter has to do with the fact that both the comparee and standard are adventurous and would rather roam the forest than study.


\begin{styleExampleTitle}
Similarity clauses with omitted standard and parameter
\end{styleExampleTitle}

\ea
\label{Example_11.132}
\gll {{misionaris{\Tilde}misionaris}} {dong} {punya} {ana{\Tilde}ana} {juga}\\ %
 {\textsc{rdp}{\Tilde}missionary}  \textsc{3pl}  \textsc{poss}  \textsc{rdp}{\Tilde}child  also\\
\gll \bluebold{Ø}  \bluebold{sama}  saja  {\bluebold{Ø}}\\
  {} be.same  just  {}\\
\glt 
‘the missionaries’ children are just \bluebold{the same} (as the Yali children in terms of \bluebold{being adventurous})’ \textstyleExampleSource{[081011-022-Cv.0280]}
\z


Dis\isi{similarity} clauses are formed without an overt parameter. Instead, the comparee and standard are compared in terms of an understood attribute or quality, such that ``\textsc{comparee} is different from \textsc{standard} (in terms of an understood \textsc{parameter})'', as illustrated in ({\ref{Example_11.133}) to ({\ref{Example_11.136}).



The index is the stative \isi{verb} \textitbf{laing} ‘be different’ or \textitbf{beda} ‘be different’, and the mark is elative \textitbf{dari} ‘from’ or \isi{comitative} \textitbf{dengang} ‘with’. Dis\isi{similarity} comparisons are typically formed with \textitbf{laing dari} ‘be different from’ as in ({\ref{Example_11.133}). They signal that the two participants are dissimilar in terms of their overall nature. If speakers want to indicate that the two participants diverge from each other in terms of specific attributes or features rather than their overall nature, they use a dis\isi{similarity} clause formed with \textitbf{beda dengang} ‘be different with’. This is demonstrated with the elicited example in ({\ref{Example_11.134}), which contrasts with the clause in ({\ref{Example_11.133}). Another example is the dis\isi{similarity} clause in ({\ref{Example_11.135}). Clauses formed with \textitbf{beda dari} ‘be different from’ are also acceptable but considered to be Indonesian-like rather than typical Papuan Malay. Clauses formed with \textitbf{laing dengang} ‘be different from’ are unacceptable.

\begin{styleExampleTitle}
Dis\isi{similarity} clauses: ‘\textsc{comparee} is different from \textsc{standard}’
\end{styleExampleTitle}
\ea
\label{Example_11.133}
\gll {sifat} {ini} {\bluebold{laing}} {\bluebold{dari}} {ko}\\ %
 nature  \textsc{d.prox}  be.different  from  \textsc{2sg}\\
\glt 
‘this disposition (of mine) is \bluebold{different from} you (in every aspect)’ \textstyleExampleSource{[081110-008-CvNP.0089]}
\z

\ea
\label{Example_11.134}
\gll {sifat} {ini} {\bluebold{beda}} {\bluebold{dengang}} {ko}\\ %
 nature  \textsc{d.prox}  be.different  with  \textsc{2sg}\\
\glt 
‘this disposition (of mine) is \bluebold{different from} you (in terms of some specific aspect)’ \textstyleExampleSource{[Elicited BR111011.008]}
\z

\ea
\label{Example_11.135}
\gll {orang} {Papua} {\bluebold{beda}} {\bluebold{dengang}} {orang} {Indonesia}\\ %
 person  Papua  be.different  with  person  Indonesia\\
\glt 
‘Papuans are \bluebold{different from} Indonesians (in terms of their physical features)’ \textstyleExampleSource{[081029-002-Cv.0009]}
\z


If the comparee is understood from the context, it can be omitted, as shown in ({\ref{Example_11.136}).


\begin{styleExampleTitle}
Dis\isi{similarity} clauses with omitted comparee
\end{styleExampleTitle}

\ea
\label{Example_11.136}
\gll {banyak,} {tapi} {Ø} {\bluebold{beda}} {\bluebold{dengang}} {Jayapura} {punya}\\ %
 many  but {}   be.different  with  Jayapura  \textsc{poss}\\
\glt 
[Comparing different melinjo varieties:] ‘(there’re) lots (of melinjo), but (they’re) \bluebold{different from} Jayapura’s (melinjos in terms of \bluebold{being bitter})’ \textstyleExampleSource{[080923-004-Cv.0010]}
\z


In summary, the typical scheme for \isi{identity-marking} comparative constructions in Papuan Malay is ``(\textsc{comparee} – \textsc{parameter}) – \textsc{index} – \textsc{mark} – \textsc{standard}''. Alternatively, the standard can be encoded as the clausal subject together with the comparee, such that ``\textsc{comparee} \& \textsc{standard} are equally \textsc{parameter}''.


\subsection{Summary}
\label{Para_11.5.3}

Papuan Malay employs two structurally distinct types of comparative constructions: (1) \isi{degree-marking} clauses, and (2) \isi{identity-marking} clauses.



Degree-marking clauses signal superiority, inferiority, or superlative. The following constituents serve as index: \textitbf{lebi} ‘more’ (superiority), \textitbf{kurang} ‘lack’ (inferiority), and \textitbf{paling} ‘most’ (superlative). The
mark is elative \textitbf{dari} ‘from’. The index precedes the parameter. The standard together with its mark can be omitted. The basic scheme for this type of comparative clauses is given in ({\ref{Example_11.137}).


\begin{styleExampleTitle}
Scheme for \isi{degree-marking} clauses
\end{styleExampleTitle}

\ea
\label{Example_11.137}
{\textsc{comparee}} {\textsc{index}} {\textsc{parameter}} {(\textsc{mark}} {\textsc{standard})}\\ %

 \z

Identity-marking clauses express \isi{similarity} or dis\isi{similarity}. In \isi{similarity}\linebreak clauses the index is \textitbf{sama} ‘be same’ and the mark is \isi{comitative} \textitbf{dengang} ‘with’. In {dissimilarity} clauses, the index is \textitbf{laing} ‘be different’ in combination with the mark \textitbf{dari} ‘from’, or \textitbf{beda} ‘be different’ in combination with the mark \textitbf{dengang} ‘with’. Clauses formed with \textitbf{laing dari} ‘be different from’ indicate overall  {dissimilarity}, whereas clauses with \textitbf{beda dengang} ‘be different from’ signal  {dissimilarity} in terms of some specific features. In \isi{identity-marking} clauses the index follows the parameter, which is optional. The standard is typically encoded in a \isi{prepositional phrase}, with the \isi{preposition} serving as the mark of comparison. This scheme for \isi{identity-marking} clauses is illustrated in ({\ref{Example_11.138}). In \isi{similarity} clauses, the standard can also be encoded as the clausal subject together with the comparee, as shown in ({\ref{Example_11.139}).


\begin{styleExampleTitle}
Schemes for \isi{identity-marking} clauses
\end{styleExampleTitle}

\ea
\label{Example_11.138}
{(\textsc{comparee}} {{\textsc{parameter})}} {\textsc{index}} {{\textsc{mark}}} {\textsc{standard}}
\z
\ea
\label{Example_11.139}
{{\textsc{comparee} \ \textsc{standard}}  {(\textsc{parameter})}  {\textsc{index}}}\\ %
\z

\section{Summary}
\label{Para_11.6}
This chapter has described different types of verbal clauses. The most pertinent distinction is that between intransitive and transitive clauses. It is important to note, though, that Papuan Malay verbs allow but do not require core arguments. Trivalent verbs most often occur in monotransitive or intransitive clauses rather than in ditransitive clauses. Along similar lines, \isi{bivalent} verbs are very commonly used in intransitive clauses.



Also discussed are \isi{causative} clauses. They are the result of a valency-increa\-sing operation. Papuan Malay causatives are monoclausal V\textsubscript{1}V\textsubscript{2} constructions in which \isi{causative} V\textsubscript{1} encodes the notion of cause while V\textsubscript{2} expresses the notion of effect. Papuan Malay has two \isi{causative} verbs which usually produce causer-controlled causatives: \isi{trivalent} \textitbf{kasi} ‘give’, and \isi{bivalent} \textitbf{biking} ‘make’. While \textitbf{kasi}{}-causa\-tives stress the outcome of the manipulation, \textitbf{biking}{}-causatives focus on the manipulation of circumstances, which leads to the effect. Causatives with \textitbf{kasi} ‘give’ can have mono- or \isi{bivalent} bases, while \textitbf{biking}{}-causatives always have \isi{monovalent} bases.



Reciprocal clauses are a third type of clauses described in this chapter. They are formed with the \isi{reciprocity marker} \textitbf{baku} ‘\textsc{recp}’. In these clauses, two predications are presented as one, with two participants equivalently acting upon each other. In simple reciprocals, both participants are encoded as the clausal subject. In discontinuous reciprocals, the reciprocee is expressed with a \isi{comitative} phrase. Both clause types typically result in a reduction in syntactic \isi{valency}. The exception is simple constructions with a sociative reading which are characterized by \isi{valency} retention.



Also discussed are existential clauses formed with the existential \isi{verb} \textitbf{ada} ‘exist’. Two clause types can be distinguished: intransitive clauses with one core argument, and transitive clauses with two core arguments. In one-argument clauses, \textitbf{ada} ‘exist’ precedes or follows the subject, or theme, depending on its definiteness. Existential clauses express existence, availability, or possession.



A final type of verbal clauses discussed in this chapter are \isi{degree-marking} and \isi{iden\-ti\-ty-marking} comparative clauses. Degree-marking clauses denote superiority, inferiority, or superlative. In these clauses, the parameter follows the index, the comparee takes the subject slot, and the optional standard is expressed in a \isi{prepositional phrase}. Identity-marking clauses designate \isi{similarity} or dis\isi{similarity}. In these constructions, the parameter either precedes the index or is omitted. The comparee takes the subject slot while the standard is usually expressed with a \isi{prepositional phrase}. In \isi{similarity} clauses, the standard can also be encoded as the clausal subject together with comparee.

