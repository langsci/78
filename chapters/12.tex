\chapter[Nonverbal clauses]{Nonverbal clauses}
\label{Para_12}
This chapter discusses nonverbal predicate clauses in Papuan Malay, that is, clauses in which the main semantic content is not conveyed by a \isi{verb} or verbal phrase, but by some other predicate category.



Papuan Malay has three syntactically distinct types of nonverbal predicate clauses, namely, nominal, \isi{numeral}, \isi{quantifier}, and prepositional predicate  clauses. Nominal predicates have ascriptive or equative function and also encode possession. Numeral and \isi{quantifier} predicates denote quantities. Prepositional predicates encode \isi{locational} or non\isi{locational} relations between a figure and the ground. As in verbal clauses, the nonverbal predicate typically follows the subject; no copula intervenes (see \chapref{Para_11}).



Before discussing the three types of nonverbal clauses in more detail, §\ref{Para_12.1} explores which constituents can fill the subject slot in nonverbal clauses. Nominal predicate clauses are described in §\ref{Para_12.2}, \isi{numeral} and \isi{quantifier} clauses in §\ref{Para_12.3}, and prepositional clauses in §\ref{Para_12.4}. The main points of this chapter are summarized in §\ref{Para_12.5}. (Negation of nonverbal clauses is discussed in §\ref{Para_13.1}.)


\section{Non{verbal clause} subjects}
\label{Para_12.1}
In nonverbal clauses, the subject can be a \isi{noun} or \isi{noun} phrase, a personal \isi{pronoun}, or a \isi{demonstrative}, as shown in (\ref{Example_12.1}) to (\ref{Example_12.6}). Alternatively, the subject can be elided if it is understood from the context, as shown in (\ref{Example_12.7}) and (\ref{Example_12.8}).



In the nominal clause in (\ref{Example_12.1}) and the \isi{quantifier} clause in (\ref{Example_12.2}), the subject is a \isi{noun} phrase or a \isi{noun}, respectively. In the nominal clause in (\ref{Example_12.3}) and the prepositional clause in (\ref{Example_12.4}) the subjects are encoded as personal pronouns. And in the \isi{numeral} clause in (\ref{Example_12.5}) and the prepositional clause in (\ref{Example_12.6}), the subjects are expressed with demonstratives. (For a nominal clause with a \isi{demonstrative} subject see (\ref{Example_12.12}) in §\ref{Para_12.2}, p. \pageref{Example_12.12}, for a \isi{numeral} clause with a personal \isi{pronoun} subject see (\ref{Example_12.19}) in §\ref{Para_12.3}, p. \pageref{Example_12.19}, and for a \isi{prepositional phrase} with a \isi{noun} phrase subject see (\ref{Example_12.23}) in §\ref{Para_12.4.1}, p. \pageref{Example_12.23}.)


\begin{styleExampleTitle}
Subjects in nonverbal clauses
\end{styleExampleTitle}

\ea
\label{Example_12.1}
\gll {\bluebold{orang}} {\bluebold{ini}} {muka} {baru}\\ %
 person  \textsc{d.prox}  face  be.new\\
\glt 
‘\bluebold{this person} is a new person’ \textstyleExampleSource{[080919-004-NP.0079]}
\z

\ea
\label{Example_12.2}
\gll {\ldots} {\bluebold{picaang}} {juga} {banyak}\\ %
  {} splinter  also  many\\
\glt 
‘[at the beach] there are also lots of \bluebold{splinters}’ (Lit. ‘\bluebold{the splinters} (are) also many’) \textstyleExampleSource{[080917-006-CvHt.0008]}
\z

\ea
\label{Example_12.3}
\gll {\bluebold{ko}} {prempuang} {Jayapura,} {de} {bilang,} {\bluebold{ko}} {prempuang} {Demta}\\ %
 \textsc{2sg}  woman  Jayapura  \textsc{3sg}  say  \textsc{2sg}  woman  Demta\\
\glt 
``\bluebold{you}’re a Jayapura girl'', he says, ``\bluebold{you}’re a Demta girl''' \textstyleExampleSource{[081006-025-CvEx.0014]}
\z

\ea
\label{Example_12.4}
\gll {baru} {Sarles} {ini} {\bluebold{de}} {di} {blakang} {bapa}\\ %
 and.then  Sarles  \textsc{d.prox}  \textsc{3sg}  at  backside  father\\
\glt 
‘but then Sarles here, \bluebold{he} was behind father’ \textstyleExampleSource{[081025-009b-Cv.0014]}
\z

\ea
\label{Example_12.5}
\gll {\bluebold{itu}} {satu} {saja} {blum} {brapa} {\ldots}\\ %
 \textsc{d.dist}  one  just  not.yet  several  \\
\glt 
[Conversation about cloths as a bride-price:] ‘\bluebold{that} is just one (cloth and) not yet several (cloths) {\ldots}’ \textstyleExampleSource{[081006-029-CvEx.0011]}
\z

\ea
\label{Example_12.6}
\gll {a} {\bluebold{itu}} {di} {Wakde} {sana}\\ %
 ah!  \textsc{d.dist}  at  \ili{Wakde}  \textsc{l.dist}\\
\glt 
‘ah, \bluebold{that}’s in \ili{Wakde} over there’ \textstyleExampleSource{[081006-016-Cv.0030]}
\z


If the subject can be inferred from the context it can also be elided. This is illustrated with the two nominal clauses in (\ref{Example_12.7}) and the prepositional clause in (\ref{Example_12.8}). In the two nominal clauses in (\ref{Example_12.7}), the predicates \textitbf{kitong pu ana} ‘our child’ and \textitbf{tong punya dara} ‘our blood’ are coreferential with \textitbf{de} ‘\textsc{3sg}’. As the subject was already introduced at the beginning of the utterance, it is omitted in the nominal clause. In the prepositional clause in (\ref{Example_12.8}), the elided subject is \textitbf{ko} ‘\textsc{2sg}’, that is, the addressee.


\begin{styleExampleTitle}
Elision of subjects in nonverbal clauses
\end{styleExampleTitle}

\ea
\label{Example_12.7}
\gll {{\bluebold{de}}} {{minta}} {{apa,}} {kitong} {kasi} {karna} {\bluebold{Ø}} {kitong} {punya} {ana}\\ %
 {\textsc{3sg}}  {request}  {what}  \textsc{1pl}  give  because {}  \textsc{1pl}  \textsc{poss}  child\\
\gll {\ldots}  {\bluebold{Ø}}  masi  {tong}  {punya}  {dara}\\
   {} {}  still  {\textsc{1pl}}  {\textsc{poss}}  {blood}\\
\glt 
‘\bluebold{she} requests something, we give (it to her), because (\bluebold{she}’s) our child, {\ldots} (\bluebold{she}’s) still our blood’ \textstyleExampleSource{[081006-025-CvEx.0020/0022]}
\z

\ea
\label{Example_12.8}
\gll {wa,} {sa} {pikir} {\bluebold{Ø}} {masi} {di} {Arbais?}\\ %
 wow!  \textsc{1sg}  think  {}  still  at  Arbais\\
\glt
[Addressing a guest:] ‘wow!, I thought (\bluebold{you}) were still in Arbais’ \textstyleExampleSource{[081011-011-Cv.0044]}
\z

\section{Nominal predicate clauses}
\label{Para_12.2}
In nonverbal clauses with nominal predicates, a \isi{noun} or a \isi{noun} phrase conveys the main semantic content.



In Papuan Malay, nominal clauses have three functions: (1) to describe the subject, (2) to identify the subject, and (3) to express possession of an \isi{indefinite} possessum. Nominal predicates always receive a static reading.



Nominal predicates conveying a description of the subject are also referred to as “ascriptive predications”, adopting \citegen[101]{Hengeveld.1992} terminology: they describe a particular entity that is denoted by the subject of the clause such that ``\textsc{s} is a member of \textsc{n}/\textsc{np}''. That is, an ascriptive clause asserts that this entity belongs to the class of entities specified in the nonreferential nominal predicate. By contrast, nominal predicates expressing identification are “equative predicates”. They are referential and equate the particular entity denoted by the subject of the clause to the entity specified in the predicate such that ``\textsc{s} is \textsc{n}/\textsc{np}''. (See \citealt[101]{Hengeveld.1992}; \citealt[105]{Payne.1997}.) In nominal clauses conveying the notion of possession the subject embodies the semantic role of possessor while the predicate functions as an \isi{indefinite} possessum such that ``\textsc{possessor} has a \textsc{possessum}''.



Papuan Malay ascriptive, equative, and possessive nominal predicates are different in terms of their semantics, but not in terms of their structure. That is, Papuan Malay does not distinguish the three nominal predicate types as far as their syntactic or intonational features are concerned; all three are formed by \isi{juxtaposition} of two \isi{noun} phrases with the subject preceding the predicate. This is illustrated with the ascriptive clauses in (\ref{Example_12.9}) and (\ref{Example_12.10}), the equative clauses in (\ref{Example_12.11}) and (\ref{Example_12.12}), and the possessive clauses in (\ref{Example_12.13}) to (\ref{Example_12.16}).



In the ascriptive clause in (\ref{Example_12.9}), the subject \textitbf{saya} ‘\textsc{1sg}’ is asserted to belong to the class of \textitbf{manusia} ‘human being’. In the ascriptive clause in (\ref{Example_12.10}), the subject \textitbf{ko} ‘\textsc{2sg}’ is part of the class of \textitbf{prempuang Demta} ‘Demta girls’. The equative clause in (\ref{Example_12.11}) identifies the predicate \textitbf{ade} ‘younger sibling’ with the subject \textitbf{dia} ‘\textsc{3sg}’. Along similar lines, the equative clause in (\ref{Example_12.12}) identifies the predicate \textitbf{klawar} ‘cave bat’ with the subject \textitbf{itu} ‘\textsc{d.dist}’. The example in (\ref{Example_12.11}) also shows that nonverbal predicates can be modified with adverbs, such as \textitbf{masi} ‘still’.


\begin{styleExampleTitle}
Ascriptive clauses
\end{styleExampleTitle}

\ea
\label{Example_12.9}
\gll {misalnya} {saya} {\bluebold{manusia}} {\bluebold{biasa}}\\ %
 for.example  \textsc{1sg}  human.being  be.usual\\
\glt 
[About humans and evil spirits:] ‘for example, I am \bluebold{a normal human being}’ \textstyleExampleSource{[081006-022-CvEx.0025]}
\z

\ea
\label{Example_12.10}
\gll {ko} {\bluebold{prempuang}} {\bluebold{Demta},} {ko} {pulang} {ke} {Demta}\textup{!}\\ %
 \textsc{2sg}  woman  Demta  \textsc{2sg}  go.home  to  Demta\\
\glt 
‘you are \bluebold{a Demta girl}, go home to Demta!’ \textstyleExampleSource{[081006-025-CvEx.0014]}
\z

\begin{styleExampleTitle}
Equative clauses
\end{styleExampleTitle}

\ea
\label{Example_12.11}
\gll {dia} {masi} {\bluebold{ade}}\\ %
 \textsc{3sg}  still  ySb\\
\glt 
‘she’s still (my) \bluebold{younger sister}’ \textstyleExampleSource{[080927-009-CvNP.0038]}
\z

\ea
\label{Example_12.12}
\gll {o,} {itu} {\bluebold{klawar}}\\ %
 oh!  \textsc{d.dist}  cave.bat\\
\glt 
‘oh, that was \bluebold{a bat}’ \textstyleExampleSource{[081023-001-Cv.0041]}
\z


The nominal clauses in (\ref{Example_12.13}) to (\ref{Example_12.16}) express possession of an \isi{indefinite} possessum. In (\ref{Example_12.13}), the subject \textitbf{saya} ‘1\textsc{sg}’ has the semantic role of possessor, while the predicate \textitbf{empat ana} ‘four children’ functions as the possessum. In (\ref{Example_12.14}), the possessor \textitbf{de} ‘3\textsc{sg}’ is juxtaposed to the possessum \textitbf{ana kecil} ‘small child’. The possessive clauses in (\ref{Example_12.13}) and (\ref{Example_12.14}) encode inalienable possession relations. The clauses in (\ref{Example_12.15}) and (\ref{Example_12.16}), by contrast, denote alienable possession relations, namely between a human referent and animate nonhuman \textitbf{ikang} ‘fish’ in (\ref{Example_12.15}) and inanimate \textitbf{glang puti} ‘silver/tin bracelets’ in (\ref{Example_12.16}). (Alternatively, possession of an \isi{indefinite} possessum can be encoded by an \isi{existential clause}; for details see §\ref{Para_11.4.2}. Possession of a \isi{definite} possessum is encoded by an \isi{adnominal possessive construction}; for details see \chapref{Para_9} and also §\ref{Para_11.4.1}.)


\begin{styleExampleTitle}
Possessive clauses: Possession of an \isi{indefinite} possessum
\end{styleExampleTitle}

\ea
\label{Example_12.13}
\gll {saya} {\bluebold{empat}} {\bluebold{ana}}\\ %
 \textsc{1sg}  four  child\\
\glt 
‘I have \bluebold{four children}’ \textstyleExampleSource{[081006-024-CvEx.0001]}
\z

\ea
\label{Example_12.14}
\gll {baru} {de} {\bluebold{ana}} {\bluebold{kecil}} {lagi}\\ %
 and.then  \textsc{3sg}  child  be.small  again\\
\glt 
‘moreover, she has \bluebold{a small child} again’ \textstyleExampleSource{[081010-001-Cv.0070]}\footnote{In a different context, \textitbf{de ana kecil} can also receive the equative reading ‘she (is) a small child’.}
\z

\ea
\label{Example_12.15}
\gll {de} {\bluebold{satu},} {sa} {\bluebold{satu}}\\ %
 \textsc{3sg}  one  \textsc{1sg}  one\\
\glt 
[Joke about two fishermen:] ‘he has \bluebold{one (fish)}, I have \bluebold{one (fish)}’ \textstyleExampleSource{[081109-011-JR.0008]}
\z

\ea
\label{Example_12.16}
\gll {orang} {Biak} {kang} {\bluebold{glang}} {\bluebold{puti}}\\ %
 person  \ili{Biak}  you.know  bracelet  be.white\\
\glt 
[About bride-price customs:] ‘you know, the \ili{Biak} people have \bluebold{silver/tin bracelets}’ \textstyleExampleSource{[081006-029-CvEx.0007]}
\z


These examples also show that the predicate of a nominal clause can be a \isi{noun} such as \textitbf{ade} ‘younger sibling’ in (\ref{Example_12.11}), or \textitbf{klawar} ‘cave bat’ in (\ref{Example_12.12}), or a \isi{noun} phrase, such as \textitbf{manusia biasa} ‘normal human being’ in (\ref{Example_12.9}) or \textitbf{empat ana} ‘four children’ in (\ref{Example_12.13}).



If speakers want to emphasize the predicate, they can front it as for instance \textitbf{orang pintar} ‘smart person’ in (\ref{Example_12.17}). The predicate is set-off by a boundary intonation in that the stressed penultimate syllable of the verbal modifier \textitbf{pintar} ‘be clever’ is marked with a slight increase in pitch (“~\'{~}~”). In the second clause in (\ref{Example_12.17}) the speaker repeats his statement, this time however returning to the canonical subject-predicate word order.


\begin{styleExampleTitle}
Fronted nominal predicates
\end{styleExampleTitle}
\ea
\label{Example_12.17}
\gll {trus} {\bluebold{orang}} {\bluebold{píntar}} {dia,} {dia} {orang} {pintar}\\ %
 next  person  be.clever  \textsc{3sg}  \textsc{3sg}  person  be.clever\\
\glt
‘and then \bluebold{a smart person} he is, he’s a smart person’ \textstyleExampleSource{[081029-005-Cv.0169]}
\z

\section{Numeral and {quantifier} predicate clauses}
\label{Para_12.3}
In \isi{numeral} and \isi{quantifier} clauses, a \isi{numeral} or \isi{quantifier} conveys the main semantic content; again, these predicates receive a static reading. As in nominal clauses, the subject precedes the predicate. Structurally, \isi{numeral} and \isi{quantifier} predicates are identical to \isi{noun} phrases with a postposed \isi{numeral} or \isi{quantifier} (see §\ref{Para_8.3}). Semantically, \isi{numeral} and \isi{quantifier} clauses have determining function in that they express specific properties of the subject, namely those of number and quantity, such that ``\textsc{s} is \textsc{num}/\textsc{qt}'' as illustrated in (\ref{Example_12.18}) to (\ref{Example_12.21}).



In (\ref{Example_12.18}), a husband relates that in a neighboring village a woman gave birth to a snake. His wife contradicts this statement, asserting that it was not one snake but that the \textitbf{ular} ‘snake’ were \textitbf{dua} ‘two’. The analysis of the \textitbf{dua} ‘two’ as a \isi{numeral} predicate and not as an adnominal modifier is confirmed by the following fact. In the context of these utterances it is possible to insert existential \textitbf{ada} ‘exist’ between the subject \textitbf{ular} ‘snake’ and the predicate \textitbf{dua} ‘two’ which gives the emphatic progressive reading \textitbf{ular ada dua} ‘the snakes were being two’ or ‘the snakes were indeed two’ (see also §\ref{Para_5.4.1}). If \textitbf{ular dua} was a \isi{noun} phrase with the reading of ‘two snakes’, existential \textitbf{ada} ‘exist’ would have to precede or follow the \isi{noun} phrase such that \textitbf{ada ular dua} ‘there were two snakes’ or \textitbf{ular dua ada} ‘the two snakes exist’. In (\ref{Example_12.19}), predicatively used \textitbf{satu} ‘one’ and \textitbf{dua blas} ‘twelve’ convey information about the numeric quantities of their respective subjects \textitbf{bulang} ‘moon’ and \textitbf{de} ‘3\textsc{sg}’. The first clause \textitbf{di langit ini bulang satu} cannot be interpreted as a \isi{prepositional predicate clause} (see §\ref{Para_12.4}) in which \textitbf{bulang satu} functions as a \isi{noun} phrase which takes the subject slot. Such a reading would imply that there are several moons with the speaker talking about one of them: \textitbf{bulang satu} ‘a certain moon’ (see §\ref{Para_5.9.4} for a more detailed discussion of \textitbf{satu} ‘one’).


\begin{styleExampleTitle}
Numeral predicates
\end{styleExampleTitle}
\ea
\label{Example_12.18}
\ea
\label{Example_12.18a}
\gll Husband:  dia  melahirkang  ular\\ %
  {}   \textsc{3sg}  give.birth  snake\\
\glt Husband: ‘she gave birth to a snake’
\vspace{5pt}
\ex
\label{Example_12.18b}
\gll    Wife:  ular  \bluebold{dua}\\
 {}    snake  two\\
\glt Wife: ‘\bluebold{two} snakes’ (Lit. ‘the snakes (were) \bluebold{two}’) \textstyleExampleSource{[081006-022-CvEx.0002-0003]}
\z
\z

\ea
\label{Example_12.19}
\gll {di} {langit} {ini} {bulang} {\bluebold{satu}} {tapi} {di} {kalender} {de} {\bluebold{dua}} {\bluebold{blas}}\\ %
 at  sky  \textsc{d.prox}  month  one  but  at  calendar  \textsc{3sg}  two  teens\\
\glt 
‘in this sky there is\bluebold{ one} moon, but in the calendar there are\bluebold{ twelve}’ (Lit. ‘the moon (is) \bluebold{one} {\ldots} it (is) \bluebold{twelve}’) \textstyleExampleSource{[081109-007-JR.0002]}
\z


In (\ref{Example_12.20}) and (\ref{Example_12.21}), predicatively-used \isi{universal quantifier} \textitbf{smua} ‘all’ and mid-range \isi{quantifier} \textitbf{banyak} ‘many’ express the nonnumeric quantities of their respective subjects \textitbf{orang Sulawesi} ‘Sulawesi people’ and \textitbf{pisang masak itu} ‘that ripe banana’.


\begin{styleExampleTitle}
Quantifier predicates
\end{styleExampleTitle}

\ea
\label{Example_12.20}
\gll {katanya} {orang} {Sulawesi} {\bluebold{smua}}\\ %
 it.is.being.said  person  Sulawesi  all\\
\glt 
‘it’s being said (that) they are \bluebold{all} Sulawesi people’ (Lit. ‘Sulawesi people (are) \bluebold{all}’) \textstyleExampleSource{[081029-005-Cv.0106]}
\z

\ea
\label{Example_12.21}
\gll {baru} {dong} {bawa} {pisang} {masak,} {pisang} {masak} {itu} {\bluebold{banyak}}\\ %
 and.then  \textsc{3pl}  bring  banana  cook  banana  cook  \textsc{d.dist}  many\\
\glt
‘and then they brought ripe bananas, those ripe bananas were \bluebold{many}’ \textstyleExampleSource{[081006-023-CvEx.0071]}
\z

\section{Prepositional predicate clauses}
\label{Para_12.4}
Nonverbal clauses with prepositional predicates convey information about the relation between a figure and a ground, such that ``\textsc{figure} is in relation to \textsc{ground}''. The figure is encoded by the clausal subject and the ground by the complement of the \isi{prepositional phrase}. This phrase is juxtaposed to the subject and functions as the clausal predicate. Semantically, two types of prepositional predicate clauses can be distinguished: \isi{locational} clauses (§\ref{Para_12.4.1}), and non\isi{locational} clauses (§\ref{Para_12.4.2}). The precise semantic relation between figure and ground is defined by the \isi{preposition} that heads the \isi{prepositional phrase}. (For a detailed discussion of prepositions and prepositional phrases see \chapref{Para_10}.)


\subsection{Locational prepositional clauses}
\label{Para_12.4.1}
Locational predicate clauses typically express information about the \isi{locational} relation, spatial or figurative, between a figure and the ground, as shown in (\ref{Example_12.22}) to (\ref{Example_12.26}). In addition, \isi{locational} predicates can have presentative function, as shown in (\ref{Example_12.28}). In Papuan Malay, the specific kind of relation is conveyed by prepositions encoding location, namely \isi{locative} \textitbf{di} ‘at, in’, allative \textitbf{ke} ‘to’, or elative \textitbf{dari} ‘from’, (see also §\ref{Para_10.1}). The ground can be encoded by a common (proper) \isi{noun} or a \isi{noun} phrase. Unlike prepositional phrases in verbal clauses, \isi{locative} \textitbf{di} ‘at, in’ and allative \textitbf{ke} ‘to’ cannot be omitted from prepositional clauses with nominal complements as this would result in nominal clauses with unacceptable semantics (for more details on the omission of prepositions encoding location, see §\ref{Para_10.1.5}); the exceptions are preposed prepositional clauses with \isi{locative} complements, as in (\ref{Example_12.27}) to (\ref{Example_12.29}).



Spatial \isi{locational} predicates denote static or dynamic relations between a figure and the ground, depending on the semantics of the \isi{preposition}. In (\ref{Example_12.22}), \isi{locative} \textitbf{di} ‘at, in’ expresses the spatial location of the figure \textitbf{dia} ‘\textsc{3sg}’ at the ground \textitbf{kampung} ‘village’. In (\ref{Example_12.23}) allative \textitbf{ke} ‘to’ signals the motion of the figure \textitbf{dep mama} ‘her mother’ toward the goal \textitbf{Pante-Barat}.\footnote{While this kind of prepositional predicate is not possible in English, it does occur in other languages such as colloquial \ili{German}. Hence, (\ref{Example_12.23}) easily translates into \textitbf{ihre Mutter ist nach}\textitbf{ {\ldots} Pante-Barat}.} In (\ref{Example_12.24}), elative \textitbf{dari} ‘from’ conveys the motion of the figure \textitbf{sa} ‘1\textsc{sg}’ away from the source \textitbf{Sawar}.\footnote{More frequently, however, motion away from a source is encoded by a verbal phrase such as \textitbf{kluar dari ruma} ‘left home’ in (\ref{Footnote_Example_12.1}) below:
\vspace{-5pt}
\ea
\label{Footnote_Example_12.1}
\gll sa \textitbf{kluar} \textitbf{dari} \textitbf{ruma} sa punya orang-tua\\
\textsc{1sg} go.out from house \textsc{1sg} \textsc{poss} parent\\
\glt ‘I \bluebold{left home}, my parents’ (Lit. ‘\bluebold{went out from} the house’) [081115-001b-Cv.0045]
\z
}

\begin{styleExampleTitle}
Static and dynamic spatial \isi{locational} relations between figure and ground
\end{styleExampleTitle}

\ea
\label{Example_12.22}
\gll {memang} {dia} {\bluebold{di}} {\bluebold{kampung}}\\ %
 indeed  \textsc{3sg}  at  village\\
\glt 
‘indeed, he was \bluebold{in the village}’ \textstyleExampleSource{[080918-001-CvNP.0014]}
\z

\ea
\label{Example_12.23}
\gll {dep} {mama} {\bluebold{ke}} {ini} {\bluebold{Pante-Barat}}\\ %
 \textsc{3sg}:\textsc{poss}  mother  to  \textsc{d.prox}  Pante-Barat\\
\glt 
‘her mother (went) \bluebold{to}, what’s-its-name, \bluebold{Pante-Barat}’ \textstyleExampleSource{[080919-006-CvNP.025]}
\z

\ea
\label{Example_12.24}
\gll {sa} {\bluebold{dari}} {\bluebold{Sawar}}\\ %
 \textsc{1sg}  from  Sawar\\
\glt 
‘I (just returned) \bluebold{from Sawar}’ \textstyleExampleSource{[080927-004-CvNP.0003]}
\z


Locational predicates also express figurative \isi{locational} relations between a figure and the ground. In (\ref{Example_12.25}), \isi{locative} \textitbf{di} ‘at, in’ conveys a figurative \isi{locational} relation between the figure \textitbf{saya} ‘1\textsc{sg}’ and the ground \textitbf{IPS satu} ‘Social Sciences I’. Along similar lines, elative \textitbf{dari} ‘from’ conveys a figurative relation in (\ref{Example_12.26}). This example is part of a conversation about a building project that was put on hold due to the lack of funding. The figure \textitbf{smua itu} ‘all that’ refers to the delayed project while the ground \textitbf{uang} ‘money’ denotes the nonspatial source from which this delay originates. Figurative predicates with allative \textitbf{ke} ‘to’ are unattested.


\begin{styleExampleTitle}
Figurative \isi{locational} relation between figure and ground
\end{styleExampleTitle}

\ea
\label{Example_12.25}
\gll {sa} {\bluebold{di}} {\bluebold{IPS}} {\bluebold{satu}}\\ %
 \textsc{1sg}  at  social.sciences  one\\
\glt 
[About course tracks in high school:] ‘I am \bluebold{in Social Sciences I}’ \textstyleExampleSource{[081023-004-Cv.0020]}
\z

\ea
\label{Example_12.26}
\gll {smua} {itu} {\bluebold{dari}} {\bluebold{uang}}\\ %
 all  \textsc{d.dist}  from  money\\
\glt 
‘all that (depends) \bluebold{on the money}’ (Lit. ‘\bluebold{from the money}’) \textstyleExampleSource{[080927-006-CvNP.0034]}
\z


If speakers want to emphasize the predicate, they can front it. The corpus, however, includes only three utterances with fronted prepositional predicates, which are presented in (\ref{Example_12.27}) to (\ref{Example_12.29}). In each case the \isi{locative} \isi{preposition} \textitbf{di} ‘at, in’ is omitted and the complement is a \isi{locative}, such as proximal \textitbf{sini} ‘\textsc{l.prox}’ in (\ref{Example_12.27}), or medial \textitbf{situ} ‘\textsc{l.med}’ in (\ref{Example_12.28}) and (\ref{Example_12.29}).\footnote{One anonymous reviewers suggests an alternative reading of the \textitbf{situ}{}-clause. The clause \textitbf{situ alang-alang} ‘there (\textsc{emph}) (is only) cogongrass’ parallels the preceding clause \textitbf{sebla tida ada ruma} ‘on that side aren’t (any houses)’. Hence, the predicate is not \textitbf{situ} ‘\textsc{l.med}’ but \textitbf{ada} ‘exist’, with the latter having been elided: \textitbf{situ Ø alang-alang} ‘there (\textsc{emph}) (is only) cogongrass’.} Fronted prepositional predicates with distal \textitbf{sana} ‘\textsc{l.dist}’ are also possible, but unattested in the corpus. (For more details on the omission of prepositions encoding location, see §\ref{Para_10.1.5}.)


\begin{styleExampleTitle}
Fronting of prepositional predicates
\end{styleExampleTitle}

\ea
\label{Example_12.27}
\gll {\bluebold{Ø}} {\bluebold{sini}} {bua{\Tilde}bua} {banyak}\\ %
  {}  \textsc{l.prox}  \textsc{rdp}{\Tilde}fruit  many\\
\glt 
‘\bluebold{here (}\blueboldSmallCaps{emph}\bluebold{)} are many different kinds of fruit (trees)’ \textstyleExampleSource{[080922-001a-CvPh.0418]}\footnote{Alternatively, the utterance in (\ref{Example_12.27}) could be interpreted as a \isi{numeral} predicate clause with a \isi{locational} adjunct, with \textitbf{bua{\Tilde}bua} ‘\textsc{rdp}{}-fruit’ as the subject, \textitbf{banyak} ‘many’ as the predicate, and \textitbf{sini} ‘\textsc{l.prox}’ as a preposed \isi{locational} adjunct, giving the literal reading ‘here the various fruit (trees) are many’.}
\z

\ea
\label{Example_12.28}
\gll {sebla} {tida} {ada} {ruma} {\bluebold{Ø}} {\bluebold{situ}} {alang-alang}\\ %
 side  \textsc{neg}  exist  house  {}  \textsc{l.med}  cogongrass\\
\glt 
‘on that side aren’t (any) houses, \bluebold{there (}\blueboldSmallCaps{emph}\bluebold{)} (is only) cogongrass’ \textstyleExampleSource{[081025-008-Cv.0149]}
\z

\ea
\label{Example_12.29}
\gll {\ldots} {\bluebold{Ø}} {\bluebold{situ}} {Natanael} {\bluebold{Ø}} {\bluebold{situ}} {Martin} {\bluebold{Ø}} {\bluebold{situ}} {Aleks}\\ %
  {} {}  \textsc{l.med}  Natanael {}   \textsc{l.med}  Martin  {}  \textsc{l.med}  Aleks\\
\glt
[Choosing among potential candidates for the upcoming local elections:] ‘[Burwas (village can have) two candidates,] \bluebold{there (}\blueboldSmallCaps{emph}\bluebold{)} is Natanael, \bluebold{there (}\blueboldSmallCaps{emph}\bluebold{)} is Martin, \bluebold{there (}\blueboldSmallCaps{emph}\bluebold{)} is Aleks’ \textstyleExampleSource{[080919-001-Cv.0117]}
\z

\subsection{Non{locational} prepositional clauses}
\label{Para_12.4.2}
Prepositional clauses with non\isi{locational} predicates convey information about the non\-lo\-ca\-tional, static relation between a figure and the ground. Semantically, Papuan Malay distinguishes three types of non\isi{locational} predicates, namely “\isi{associative}” or “\isi{comitative} predicates”, “simulative predicates”, and “benefactive predicates”, adopting \citegen[248–249]{Dryer.2007} terminology. Overall, however, non\isi{locational} prepositional clauses do not appear to be very common; the corpus contains only few examples.



Papuan Malay \isi{comitative} predicates are formed with prepositions encoding \isi{accompaniment}/instruments or goals, namely \isi{comitative} \textitbf{dengang} ‘with’, with its short form \textitbf{deng}, and goal \isi{preposition} \textitbf{sama} ‘to’ (see also §\ref{Para_10.2.1} and §\ref{Para_10.2.2}). In (\ref{Example_12.30}), \textitbf{deng(ang)} ‘with’ denotes the \isi{accompaniment} of the figure \textitbf{Roni} by the ground \textitbf{de pu temang{\Tilde}temang} ‘his friends’. In (\ref{Example_12.31}), \textitbf{sama} ‘to’ signals the association of the implied figure \textitbf{ana} ‘child’ with the ground \textitbf{saya} ‘1\textsc{sg}’.


\begin{styleExampleTitle}
Comitative predicates
\end{styleExampleTitle}

\ea
\label{Example_12.30}
\gll {Roni} {masi} {\bluebold{deng}} {\bluebold{de}} {\bluebold{pu}} {\bluebold{temang{\Tilde}temang}}\\ %
 Roni  still  with  \textsc{3sg}  \textsc{poss}  \textsc{rdp}{\Tilde}friend\\
\glt 
‘Roni is still \bluebold{with his friends}’ \textstyleExampleSource{[081006-031-Cv.0011]}
\z

\ea
\label{Example_12.31}
\gll {hanya} {tiga} {saja} {\bluebold{sama}} {\bluebold{saya}}\\ %
 only  three  just  to  \textsc{1sg}\\
\glt 
‘just only three (of my children) are \bluebold{with me}’ \textstyleExampleSource{[081006-024-CvEx.0001]}
\z


Simulative predicates are formed with prepositions encoding comparisons, that is, similative \textitbf{sperti} ‘similar to’ and \textitbf{kaya} ‘like’ and equative \textitbf{sebagey} ‘as’ (see also §\ref{Para_10.3}). In (\ref{Example_12.32}) \textitbf{sperti} ‘similar to’ establishes a simulative relation between the figure \textitbf{de} ‘3\textsc{sg}’ and the ground \textitbf{Sofia}. Along similar lines, \textitbf{kaya} ‘like’ denotes a simulative relation between the figure \textitbf{de} ‘\textsc{sg}’ and the ground \textitbf{de pu bapa} ‘his father’ / \textitbf{Siduas} in (\ref{Example_12.33}). In (\ref{Example_12.34}), \textitbf{sebagey} ‘as’ expresses equatability between the figure \textitbf{sa} ‘\textsc{1sg}’ and the ground \textitbf{kepala acara} ‘the head of the festivity’. (See §\ref{Para_10.3.1} and §\ref{Para_10.3.2} for a detailed discussion of the prepositions \textitbf{sperti} ‘similar to’ and \textitbf{kaya} ‘like’ and their semantics.)


\begin{styleExampleTitle}
Simulative predicates
\end{styleExampleTitle}

\ea
\label{Example_12.32}
\gll {de} {\bluebold{sperti}} {\bluebold{Sofia}}\\ %
 \textsc{3sg}  similar.to  Sofia\\
\glt 
‘she’s \bluebold{similar to Sofia}’ \textstyleExampleSource{[081115-001a-Cv.0283]}
\z

\ea
\label{Example_12.33}
\gll {de} {pu} {muka} {\bluebold{kaya}} {de} {pu} {bapa} {e} {\bluebold{kaya}} {Siduas}\\ %
 \textsc{3sg}  \textsc{poss}  face  like  \textsc{3sg}  \textsc{poss}  father  eh  like  Siduas\\
\glt 
‘his face is \bluebold{like} his father, eh, \bluebold{like} Siduas’ (face)’ \textstyleExampleSource{[080922-001a-CvPh.1446]}
\z

\ea
\label{Example_12.34}
\gll {paling} {sa} {tra} {kerja,} {sa} {\bluebold{sebagey}} {\bluebold{kepala}} {\bluebold{acara}}\\ %
 most  \textsc{1sg}  \textsc{neg}  work  \textsc{1sg}  as  head  festivity\\
\glt 
[About organizing a festivity:] ‘most likely I won’t (have to) work, I’ll be \bluebold{the head of the festivity}’ (Lit. ‘\bluebold{as the head {\ldots}}’) \textstyleExampleSource{[080919-004-NP.0068]}
\z


Benefactive predicates are formed with the benefactive \isi{preposition} \textitbf{untuk} ‘for’ (see also §\ref{Para_10.2.3}). In (\ref{Example_12.35}), for instance, \textitbf{untuk} ‘for’ conveys a benefactive relation between the figure \textitbf{itu} ‘\textsc{d.dist}’ and the ground \textitbf{masarakat} ‘community’. In the corpus, however, benefactive predicates are rare.


\begin{styleExampleTitle}
Benefactive predicates
\end{styleExampleTitle}

\ea
\label{Example_12.35}
\gll {uang} {{besarnya}} {itu} {\bluebold{untuk}} {\bluebold{masarakat}} {tapi} {pejabat}\\ %
 money  {be.big:\textsc{ 3possr}}  \textsc{d.dist}  for  community  but  official\\
\gll {yang}  makang  {banyak}\\
 {\textsc{rel}}  eat  {many}\\
\glt
‘most of that money, that’s \bluebold{for the community} but (it’s) the officials who take lots (of it)’ \textstyleExampleSource{[081029-004-Cv.0002]}
\z

\section{Summary}
\label{Para_12.5}
Papuan Malay employs three syntactically distinct types of nonverbal predicate clauses, namely nominal, \isi{numeral}, \isi{quantifier}, and prepositional predicate clauses. These clauses are formed by \isi{juxtaposition} of the two main constituents; no copula intervenes. The three clause types also have distinct semantic functions. Nominal predicates have ascriptive or equative function and also encode possession. Numeral and \isi{quantifier} predicates have determining function. Prepositional predicates encode \isi{locational} or non\isi{locational} relations between a figure and the ground.

