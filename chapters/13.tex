\chapter[Negative, interrogative, and directive clauses]{Negative, \isi{interrogative}, and directive clauses}
\label{Para_13}
This chapter describes negative, \isi{interrogative}, and directive clauses in Papuan Malay. Negative clauses formed with the negators \textitbf{tida}/\textitbf{tra} ‘\textsc{neg}’ and \textitbf{bukang} ‘\textsc{neg}’ are discussed in §\ref{Para_13.1}. Interrogative clauses, including polar and alternative questions, are described in §\ref{Para_13.2}. Directive clauses, including imperatives, adhortatives, permissions, obligations, and prohibitives, are the topic of §\ref{Para_13.3}.


\section{Negative clauses}
\label{Para_13.1}
In Papuan Malay, negative clauses are formed with the \isi{negation} adverbs \textitbf{tida}/\textitbf{tra} ‘\textsc{neg}’ or \textitbf{bukang} ‘\textsc{neg}’. Negator \textitbf{tida}/\textitbf{tra} ‘\textsc{neg}’ is used for the \isi{negation} of verbal, existential, and nonverbal prepositional clauses (§\ref{Para_13.1.1}). Negator \textitbf{bukang} ‘\textsc{neg}’ is used to negate nonverbal clauses, other than prepositional ones, and to mark contrastive \isi{negation} (§\ref{Para_13.1.2}). (Negative directives or prohibitives are discussed in §\ref{Para_13.3.3}.)


\subsection{Negation with \textitbf{tida}/\textitbf{tra} ‘\textsc{neg}’}
\label{Para_13.1.1}
The negators \textitbf{tida} ‘\textsc{neg}’ and \textitbf{tra} ‘\textsc{neg}’ negate different types of clauses; they always precede the predicate which they negate. Negation of verbal clauses is discussed in §\ref{Para_13.1.1.1}, of existential clauses in §\ref{Para_13.1.1.2}, and of nonverbal prepositional clauses in §\ref{Para_13.1.1.3}. Negator \textitbf{tida} ‘\textsc{neg}’ also provides negative responses to polar questions, as discussed in §\ref{Para_13.1.1.4}. With the exception of negative responses to polar questions, both negators are used interchangeably.


\subsubsection[Negation of verbal clauses]{Negation of verbal clauses}
\label{Para_13.1.1.1}
As a negator of verbal clauses, \textitbf{tida}/\textitbf{tra} ‘\textsc{neg}’ negates stative verbs such as \textitbf{baik} ‘be good’ in (\ref{Example_13.1}), dynamic verbs such as \textitbf{datang} ‘come’ in (\ref{Example_13.2}), \isi{bivalent} verbs such as \textitbf{pukul} ‘hit’ in (\ref{Example_13.3}), or \isi{trivalent} verbs such as \textitbf{bli} ‘buy’ in (\ref{Example_13.4}). The example in (\ref{Example_13.5}) illustrates \isi{negation} of a \isi{causative} construction.



The contrastive examples in (\ref{Example_13.1}) and (\ref{Example_13.2}) also show that \textitbf{tida} ‘\textsc{neg}’ and \textitbf{tra} ‘\textsc{neg}’ are used interchangeably with no differences in function or meaning. In the corpus, however, speakers more often use \textitbf{tida} ‘\textsc{neg}’ than \textitbf{tra} ‘\textsc{neg}’ (1,491 vs. 794 tokens) (\textitbf{bukang} ‘\textsc{neg}’ is attested with 208 tokens; see §\ref{Para_13.1.2}).


\begin{styleExampleTitle}
Negation of verbal clauses with \textitbf{tida}/\textitbf{tra} ‘\textsc{neg}’
\end{styleExampleTitle}

\ea
\label{Example_13.1}
\gll {{nanti}} {{dia}} {{pikir}} {saya} {\bluebold{tida}} {\bluebold{baik}} {\ldots} {nanti} {de}\\ %
 {very.soon}  {\textsc{3sg}}  {think}  \textsc{1sg}  \textsc{neg}  be.good {}   very.soon  \textsc{3sg}\\
\gll pikir  {kitong}  {\bluebold{tra}}  {\bluebold{baik}}\\
 think  {\textsc{1pl}}  {\textsc{neg}}  {be.good}\\
\glt 
‘very soon he’ll think that I’m\bluebold{ not good} {\ldots} very soon he’ll think that we are \bluebold{not good}’ \textstyleExampleSource{[080919-004-NP.0052-0053]}
\z

\ea
\label{Example_13.2}
\gll {de} {\bluebold{tra}} {datang} {\ldots} {de} {\bluebold{tida}} {datang}\\ %
 \textsc{3sg}  \textsc{neg}  come  {}  \textsc{3sg}  \textsc{neg}  come\\
\glt 
‘she did \bluebold{not} come {\ldots} she did \bluebold{not} come’ \textstyleExampleSource{[081010-001-Cv.0204-0205]}
\z

\ea
\label{Example_13.3}
\gll {sa} {\bluebold{tida}} {\bluebold{pukul}} {dorang}\\ %
 \textsc{1sg}  \textsc{neg}  hit  \textsc{3pl}\\
\glt 
‘I \bluebold{don’t hit} them’ \textstyleExampleSource{[080917-010-CvEx.0048]}
\z

\ea
\label{Example_13.4}
\gll {kalo} {bapa} {\bluebold{tra}} {\bluebold{bli}} {sa} {HP,} {biar} {suda} {tida} {apa{\Tilde}apa}\\ %
 if  father  \textsc{neg}  buy  \textsc{1sg}  cell.phone  let  already  \textsc{neg}  \textsc{rdp}{\Tilde}what\\
\glt 
‘if you (‘father’) \bluebold{won’t buy} me a cell phone, just let it be, no problem’ \textstyleExampleSource{[080922-001a-CvPh.0461]}
\z

\ea
\label{Example_13.5}
\gll {baru} {kamu} {\bluebold{tra}} {\bluebold{kas}} {\bluebold{kluar}} {uang} {bayar}\\ %
 and.then  \textsc{2pl}  \textsc{neg}  give  go.out  money  pay\\
\glt
[Encouraging teenagers to take free English classes:] ‘and then you \bluebold{won’t have to pay} fees’ (Lit. ‘\bluebold{not give (to) come out}’) \textstyleExampleSource{[081115-001a-Cv.0160]}
\z

\subsubsection[Negation of existential clauses]{Negation of existential clauses}
\label{Para_13.1.1.2}
Existential clauses are also negated with \textitbf{tida}/\textitbf{tra} ‘\textsc{neg}’, as illustrated in (\ref{Example_13.6}) to (\ref{Example_13.10}). Examples for negated one-argument clauses are given in (\ref{Example_13.6}) to (\ref{Example_13.8}), and for two-argument clauses in (\ref{Example_13.9}) to (\ref{Example_13.10}). (Existential clauses are discussed in detail in §\ref{Para_11.4}.)



The respective one-argument clauses in (\ref{Example_13.6}) and (\ref{Example_13.7}) illustrate negated existence and negated availability of \isi{indefinite}/nonidentifiable theme expressions. The example in (\ref{Example_13.8}) demonstrates \isi{negation} of a \isi{definite}/identifiable theme expression. One-argument clauses denoting negative possession of a \isi{definite}/identifiable possessum are unattested in the corpus; instead, the preferred type of \isi{existential clause} to express negative possession is a two-argument clause, as shown in (\ref{Example_13.9}) and (\ref{Example_13.10}).


\begin{styleExampleTitle}
Negation of one-argument existential clauses with \textitbf{tida}/\textitbf{tra} ‘\textsc{neg}’
\end{styleExampleTitle}

\ea
\label{Example_13.6}
\gll {\bluebold{tra}} {\bluebold{ada}} {kamar} {mandi}\\ %
 \textsc{neg}  exist  room  bathe\\
\glt 
‘(there) \bluebold{were no} bathrooms’ \textstyleExampleSource{[081025-009a-Cv.0059]}
\z

\ea
\label{Example_13.7}
\gll {\bluebold{tida}} {\bluebold{ada}} {air} {minum}\\ %
 \textsc{neg}  exist  water  drink\\
\glt 
‘(there) \bluebold{was no} drinking water’ \textstyleExampleSource{[081025-009a-Cv.0060]}
\z

\ea
\label{Example_13.8}
\gll {ketrampilang} {juga} {\bluebold{tra}} {\bluebold{ada}}\\ %
 skill  also  \textsc{neg}  exist\\
\glt 
[About training activities for women:] ‘\bluebold{neither} \bluebold{do} (they) \bluebold{have} (any) skills’ (Lit. ‘(their) skills also \bluebold{don’t exist}’) \textstyleExampleSource{[081010-001-Cv.0145]}
\z


Negation of two-argument existential clauses is shown in (\ref{Example_13.9}) and (\ref{Example_13.10}). The negated clauses attested in the corpus always express absence of possession, such as the negative possession of \textitbf{ana} ‘child(ren)’ in (\ref{Example_13.9}), or \textitbf{air} ‘water’ in (\ref{Example_13.10}).


\begin{styleExampleTitle}
Negation of two-argument existential clauses denoting possession
\end{styleExampleTitle}

\ea
\label{Example_13.9}
\gll {sodara} {prempuang} {itu} {\bluebold{tida}} {\bluebold{ada}} {ana}\\ %
 sibling  woman  \textsc{d.dist}  \textsc{neg}  exist  child\\
\glt 
‘that sister \bluebold{doesn’t have} children’ \textstyleExampleSource{[081006-024-CvEx.0005]}
\z

\ea
\label{Example_13.10}
\gll {dong} {\bluebold{tra}} {\bluebold{ada}} {air}\\ %
 \textsc{3pl}  \textsc{neg}  exist  water\\
\glt
‘they \bluebold{didn’t have} water’ \textstyleExampleSource{[080919-008-CvNP.0013]}
\z

\subsubsection[Negation of prepositional predicate clauses]{Negation of prepositional predicate clauses}
\label{Para_13.1.1.3}
Prepositional predicates are also negated with \textitbf{tida}/\textitbf{tra} ‘\textsc{neg}’, as shown in (\ref{Example_13.11}) and (\ref{Example_13.12}). (Negation of other types of nonverbal clauses is discussed in §\ref{Para_13.1.2}; for more details on nonverbal clauses see \chapref{Para_12}.)


\begin{styleExampleTitle}
Negation of nonverbal prepositional clauses with \textitbf{tida}/\textitbf{tra} ‘\textsc{neg}’
\end{styleExampleTitle}

\ea
\label{Example_13.11}
\gll {saya} {\bluebold{tida}} {sperti} {prempuang} {laing} {to?}\\ %
 \textsc{1sg}  \textsc{neg}  similar.to  woman  be.different  right?\\
\glt 
‘I’m \bluebold{not} similar to other women, right?’ \textstyleExampleSource{[081011-023-Cv.0173]}
\z

\ea
\label{Example_13.12}
\gll {tong} {\bluebold{tra}} {ke} {kampung}\\ %
 \textsc{1pl}  \textsc{neg}  to  village\\
\glt
‘we do \bluebold{not} (go) to the village’ \textstyleExampleSource{[080917-003a-CvEx.0048]}
\z

\subsubsection[Negation of polar questions]{Negation of polar questions}
\label{Para_13.1.1.4}
In addition, \textitbf{tida}/\textitbf{tra} ‘\textsc{neg}’ provides negative responses to polar questions, when negating verbal constructions, as shown in (\ref{Example_13.13}) and (\ref{Example_13.14}). Negator \textitbf{tida} ‘\textsc{neg}’ can stand alone as in (\ref{Example_13.13}), or it can occur in the negative existential phrase \textitbf{tida ada} ‘no’ (literally ‘(it) doesn’t exist’). Negator \textitbf{tra} ‘\textsc{neg}’, by contrast, cannot stand alone; it always occurs in the negative existential phrase \textitbf{tra ada} ‘no’, as in (\ref{Example_13.14}). (See also §\ref{Para_13.2.2.1}.)


\begin{styleExampleTitle}
Negator \textitbf{tida}/\textitbf{tra} ‘\textsc{neg}’ in responses to polar questions
\end{styleExampleTitle}

\ea
\label{Example_13.13}
\gll {Speaker-2:} {\bluebold{tida},} {dia} {balap}\\ %
 {}  \textsc{neg}  \textsc{3sg}  race\\
\glt 
[About an accident:] [Speaker-1: ‘what did he do? (was he) drunk?’]\\
Speaker-2: ‘\bluebold{no}, he was racing (his motorbike)’ \textstyleExampleSource{[081014-013-NP.0003-0004]}
\z

\ea
\label{Example_13.14}
\gll {Speaker-2:} {\bluebold{tra}} {\bluebold{ada},} {muara} {baru} {\ldots}\\ %
 {}  \textsc{neg}  exist  river.mouth  be.new  \\
\glt
[Discussing the depth of a river mouth:] [Speaker-1: ‘isn’t (it) deep?’]\\
Speaker-2: ‘\bluebold{no}, (this is) the new river mouth [(it’s) the old river mouth that is (deep)]’ \textstyleExampleSource{[080927-003-Cv.0010-0011]}
\z

\subsection{Negation with \textitbf{bukang} ‘\textsc{neg}’}
\label{Para_13.1.2}
Negator \textitbf{bukang} ‘\textsc{neg}’ has three functions. One function is to negate nonverbal clauses, a second one is to mark contrastive \isi{negation}, and a third function is to provide negative responses to polar questions.



Nonverbal clauses are typically negated with \textitbf{bukang} ‘\textsc{neg}’, which always precedes the nonverbal predicate. Prepositional predicates are the exception; they are negated with \textitbf{tida}/\textitbf{tra} ‘\textsc{neg}’ (see §\ref{Para_13.1.1.3}). (In the corpus, \textitbf{bukang} ‘\textsc{neg}’ is attested with 208 tokens, as compared to 1,491 \textitbf{tida} ‘\textsc{neg}’ and 794 \textitbf{tra} ‘\textsc{neg}’ tokens.)



In (\ref{Example_13.15}) and (\ref{Example_13.16}), \textitbf{bukang} ‘\textsc{neg}’ negates nominal predicates, and in (\ref{Example_13.17}) a \isi{quantifier} predicate. (Nonverbal clauses are discussed in detail in \chapref{Para_12}.)


\begin{styleExampleTitle}
Negation of nonverbal clauses with \textitbf{bukang} ‘\textsc{neg}’
\end{styleExampleTitle}

\ea
\label{Example_13.15}
\gll {de} {\bluebold{bukang}} {gembala} {sidang} {di} {situ}\\ %
 \textsc{3sg}  \textsc{neg}  pastor  (church.)gathering  at  \textsc{l.med}\\
\glt 
‘he’s \bluebold{not} a congregational pastor there’ \textstyleExampleSource{[080925-003-Cv.0032]}
\z

\ea
\label{Example_13.16}
\gll {sa} {\bluebold{bukang}} {orang} {yang} {seraka}\\ %
 \textsc{1sg}  \textsc{neg}  person  \textsc{rel}  be.greedy\\
\glt 
‘I’m \bluebold{not} a person who is greedy’ \textstyleExampleSource{[080917-010-CvEx.0214]}
\z

\ea
\label{Example_13.17}
\gll {pisang} {\bluebold{bukang}} {sedikit}\\ %
 banana  \textsc{neg}  few\\
\glt 
‘there (were) \bluebold{quite a few} bananas’ (Lit. ‘the bananas (were) \bluebold{not} few’) \textstyleExampleSource{[080925-003-Cv.0158]}
\z


A second function of \textitbf{bukang} ‘\textsc{neg}’ is to express contrastive \isi{negation} of an entire proposition. Contrastive \isi{negation} implies an alternative in the sense of ``the situation is not that X (but Y)''. Very often the alternative is expressed overtly, but this is not obligatory. Depending on its scope, \textitbf{bukang} ‘\textsc{neg}’ occurs between the subject and the predicate or clause-initially. Its contrastive uses in prepredicate position are shown with the examples in (\ref{Example_13.18}) and (\ref{Example_13.19}). Unlike \textitbf{tida}/\textitbf{tra} ‘\textsc{neg}’, contrastive \textitbf{bukang} ‘\textsc{neg}’ also occurs clause-initially, as shown in (\ref{Example_13.20}) and (\ref{Example_13.21}).


\begin{styleExampleTitle}
Contrastive \isi{negation} with \textitbf{bukang} ‘\textsc{neg}’
\end{styleExampleTitle}

\ea
\label{Example_13.18}
\gll {mama} {ni} {{\bluebold{bukang}}} {{hidup}} {{deng}} {{orang-tua}} {di} {kampung,}\\ %
 mother  \textsc{d.prox}  {\textsc{neg}}  {live}  {with}  {parent}  at  village\\
\gll mama  ni  hidup  {deng}  {orang}  {di}  {luar}\\
 mother  \textsc{d.prox}  live  {with}  {person}  {at}  {outside}\\
\glt 
‘(the situation was) \bluebold{not} (that) I (‘mother’) here lived with (my) parents in the village, (but) I (‘mother’) here lived with strangers away from home’ \textstyleExampleSource{[081115-001b-Cv.0043]}
\z

\ea
\label{Example_13.19}
\gll {pernikaang} {ini} {\bluebold{bukang}} {dari} {manusia,} {dari} {Tuhang} {to?}\\ %
 marriage  \textsc{d.prox}  \textsc{neg}  from  human.being  from  God  right?\\
\glt 
‘(the situation is) \bluebold{not} (that) marriage is from man, (but it is) from God, right?’ \textstyleExampleSource{[081110-006-CvEx.0239]}
\z

\ea
\label{Example_13.20}
\gll {\bluebold{bukang}} {dong} {maing,} {dong} {taguling} {di} {pecek}\\ %
 \textsc{neg}  \textsc{3pl}  play  \textsc{3pl}  be.rolled.over  at  mud\\
\glt 
‘(the situation was) \bluebold{not} (that) they played (football, but) they got rolled over in the mud’ \textstyleExampleSource{[081109-001-Cv.0025]}
\z

\ea
\label{Example_13.21}
\gll {\bluebold{bukang}} {dong} {taru} {ijing} {tapi} {dong} {taru} {hadir}\\ %
 \textsc{neg}  \textsc{3pl}  put  permission  but  \textsc{3pl}  put  attend\\
\glt 
[About students who falsified the attendance book:] ‘(the situation is) \bluebold{not} (that) they wrote down (their absences as) permitted (absences), but they wrote (them) down as (having) attended’ \textstyleExampleSource{[081023-004-Cv.0018]}
\z


This function of \textitbf{bukang} ‘\textsc{neg}’ to signal contrastive \isi{negation} has also been noted for \ili{Ambon Malay} \citep[278–279]{vanMinde.1997}, \ili{Manado Malay} \citep[59]{Stoel.2005}, \ili{Ternate Malay} \citep[224–225]{Litamahuputty.1994}, and Standard Malay and \ili{Standard Indonesian} (\citealt[127]{Himmelmann.2005}; \citealt{Kroeger.2012}).



Speakers also use \textitbf{bukang} ‘\textsc{neg}’ in single word clauses to contradict an interlocutor’s statements. They may submit an alternative to the negated proposition as in (\ref{Example_13.22}), or they may reply with bare \textitbf{bukang} ‘\textsc{neg}’.


\begin{styleExampleTitle}
Contradiction of an interlocutor’s statements with \textitbf{bukang} ‘\textsc{neg}’
\end{styleExampleTitle}

\ea
\label{Example_13.22}
\gll {Speaker-2:} {\bluebold{bukang},} {de} {punya} {pacar}\\ %
 {}  \textsc{neg}  \textsc{3sg}  \textsc{poss}  date\\
\glt 
[Speaker-1: ‘(it was) her husband!’]\\
Speaker-2: ‘\bluebold{no}, (it was) her lover’ \textstyleExampleSource{[081006-022-CvEx.0043-0045]}
\z


Finally, speakers employ \textitbf{bukang} ‘\textsc{neg}’ to give contrastive negative responses to polar questions, as in the elicited example in (\ref{Example_13.23}). This example contrasts with the one in (\ref{Example_13.13}) in which the speaker uses \textitbf{tida} ‘\textsc{neg}’ to respond to the same question as in (\ref{Example_13.23}). While \textitbf{tida} ‘\textsc{neg}’ in (\ref{Example_13.13}) merely negates a verbal construction, \textitbf{bukang} ‘\textsc{neg}’ in (\ref{Example_13.23}) marks contrastive \isi{negation}, similar to its uses in (\ref{Example_13.18}) to (\ref{Example_13.21}). Again, speakers can add the correct response as in (\ref{Example_13.23}) or reply with bare \textitbf{bukang} ‘\textsc{neg}’. (For more details on polar questions see §\ref{Para_13.2.2}.)

\begin{styleExampleTitle}
Contrastive uses of \textitbf{bukang} ‘\textsc{neg}’ in responses to polar questions
\end{styleExampleTitle}

\ea
\label{Example_13.23}
\gll {Speaker-2:} {\bluebold{bukang},} {dia} {balap}\\ %
{}   \textsc{neg}  \textsc{3sg}  race\\
\glt
[About an accident:] [Speaker-1: ‘what did he do? (was he) drunk?’]\\
Speaker-2: ‘\bluebold{no}, (it happened because) he was racing (his motorbike)’ \textstyleExampleSource{[Elicited MY131126.001]}
\z

\section{Interrogative clauses}
\label{Para_13.2}
In Papuan Malay, three types of \isi{interrogative} clauses can be distinguished: (1) content, or information questions which elicit new information (§\ref{Para_13.2.1}), (2) polar questions which elicit yes-no answers (§\ref{Para_13.2.2}), and (3) alternative questions which require the interlocutor to choose the supposedly right answer from a list of possible answers (§\ref{Para_13.2.3}).


\subsection{Content questions}
\label{Para_13.2.1}
In Papuan Malay, content questions eliciting new information are formed with the interrogatives, as discussed in §\ref{Para_5.8}. The discussion of their positions and functions within the clause entails a description of content questions. Therefore, content questions are not further discussed here.


\subsection{Polar questions}
\label{Para_13.2.2}
Papuan Malay polar questions, that is questions that elicit yes-no answers, can be unmarked and neutral, or marked and biased, as shown in §\ref{Para_13.2.2.1} and §\ref{Para_13.2.2.2}, respectively. Both sections also describe how polar questions are answered.


\subsubsection[Unmarked neutral polar questions]{Unmarked neutral polar questions}
\label{Para_13.2.2.1}
Generally speaking, unmarked polar questions are “neutral with respect to the answer the speaker expects” \citep[179]{Sadock.1985}. That is, neutral questions do not indicate whether speakers would like their interlocutors to answer with ``yes'' or with ``no''. More specifically, polar questions can express positive polarity or negative polarity. A negative \isi{polar question} differs “from the positive question in communicating [\ldots] that the speaker already has his own opinion, but that he is interested in getting the hearer’s reaction” \citep[67]{Grimes.1975}.



These observations also apply to Papuan Malay, as demonstrated in the examples in (\ref{Example_13.24}) to (\ref{Example_13.30}).



Syntactically, the examples show that neutral polar questions have the same structure as the corresponding declarative clauses. The only distinction between the two clause types is that polar questions are marked with a rising intonation, as shown in (\ref{Example_13.24a}).



The examples also show that polar questions can express positive polarity as in (\ref{Example_13.24}), (\ref{Example_13.25}), (\ref{Example_13.27}), and (\ref{Example_13.29}), or negative polarity as in (\ref{Example_13.26}), (\ref{Example_13.28}), and (\ref{Example_13.30}).



Furthermore, the examples in (\ref{Example_13.24}) to (\ref{Example_13.30}) show how neutral polar questions are answered. Polar questions with positive answers are presented in (\ref{Example_13.24}) to (\ref{Example_13.26}), and those with negative answers in (\ref{Example_13.27}) to (\ref{Example_13.29}). An alternative strategy to answer polar questions is illustrated in (\ref{Example_13.30}).



Positive answers to polar questions are typically formed with affirmative \textitbf{yo} ‘yes’ or the \isi{interjection} \textitbf{mm-mm} ‘mhm’. This applies to positive questions as in (\ref{Example_13.24}) and (\ref{Example_13.25}), as well as to negative ones as in (\ref{Example_13.26}). In answering, speakers may also echo part of the question and/or provide additional information, as in (\ref{Example_13.24b}) and (\ref{Example_13.26b}).


\begin{styleExampleTitle}
Polar questions: Positive answers
\end{styleExampleTitle}

\ea
\label{Example_13.24}
\ea
\label{Example_13.24a}
\glll \hspace{1.2cm} {\textstyleChBold{-- }}\hspace{2mm} {\textstyleChBold{-- }} {\textstyleChBold{{} -- {} -- }}\hspace{3.5mm}  {\textstyleChBold{\textsuperscript{{} — }}\hspace{1.2mm}\textstyleChBold{\textsubscript{{} — }}}\\ %
  {Speaker-1:}  {trek}  {de}  {isi}  {minyak}?\\
 {}    truck  \textsc{3sg}  fill  oil\\
\glt   Speaker-1: ‘does the truck load gasoline?’
\vspace{10pt}
\ex
\label{Example_13.24b}
\gll  Speaker-2:  \bluebold{yo,}  minyak  tana\\
 {}    yes  oil  ground\\
\glt Speaker-2: ‘\bluebold{yes}, kerosene’ \textstyleExampleSource{[080923-009-Cv.0037-0038]}
\z
\z

\ea
\label{Example_13.25}
\ea
\label{Example_13.25a}
\gll Speaker-1:  o,  Ise  sakit?\\ %
 {}   oh!  Ise  be.sick\\
\glt Speaker-1: ‘oh, is Ise sick?’
\vspace{10pt}
\ex
\label{Example_25b}
\gll    Speaker-2:  \bluebold{mm-mm}\\
 {}    mhm\\
\glt Speaker-1: ‘\bluebold{mhm}’ \textstyleExampleSource{[080919-006-CvNP.0030-0031]}
\z
\z

\ea
\label{Example_13.26}
\ea
\label{Example_13.26a}
\gll Speaker-1:  ade  hari  ini  ko  tra  skola?\\ %
 {}    ySb  day  \textsc{d.prox}  \textsc{2sg}  \textsc{neg}  go.to.school\\
\glt Speaker-1: ‘younger sister, don’t you go to school today?’
\vspace{10pt}
\ex
\label{Example_13.26b}
\gll  Speaker-2:  \bluebold{yo},  sa  minta  ijing\\
  {}   yes  \textsc{1sg}  request  permission\\
\glt Speaker-2: ‘\bluebold{yes}, I asked for a leave of absence’ (Lit. ‘request permission (to be absent from school)’) \textstyleExampleSource{[080922-001a-CvPh.0093-0094]}
\z
\z

Negative answers to neutral positive or negative polar questions are formed in three ways, as discussed in §\ref{Para_13.1.1.4} and §\ref{Para_13.1.2}. Negative replies to polar questions are formed with \textitbf{tida} ‘\textsc{neg}’ as shown in (\ref{Example_13.13}), repeated as (\ref{Example_13.27}), or with the negative existential phrase \textitbf{tida}/\textitbf{tra ada} ‘(it) doesn’t exist’, as in (\ref{Example_13.14}), repeated as (\ref{Example_13.28}), when negating verbal constructions. Negative answers to polar questions are formed with \textitbf{bukang} ‘\textsc{neg}’, as in (\ref{Example_13.23}), repeated as (\ref{Example_13.29}), when negating nonverbal constructions.


\begin{styleExampleTitle}
Polar questions: Negative answers
\end{styleExampleTitle}

\ea
\label{Example_13.27}
\ea
\label{Example_13.27a}
\gll Speaker-1:  dia  biking  apa?  mabuk?\\ %
 {}   \textsc{3sg}  make  what  be.drunk\\
\glt [About an accident:] Speaker-1: ‘what did he do?, (was he) drunk?’
\vspace{10pt}
\ex
\label{Example_13.27b}
\gll    Speaker-2:  \bluebold{tida},  dia  balap\\
  {}   \textsc{neg}  \textsc{3sg}  race\\
\glt Speaker-2: ‘\bluebold{no}, he raced (his motorbike)’ \textstyleExampleSource{[081014-013-NP.0003-0004]}
\z
\z

\ea
\label{Example_13.28}
\ea
\label{Example_13.28a}
\gll Speaker-1:  tra  dalam?\\ %
 {}    \textsc{neg}  inside\\
\glt [Discussing the depth of a river mouth:] Speaker-1: ‘isn’t (it) deep?’
\vspace{10pt}
\ex
\label{Example_13.28b}
\gll   Speaker-2:  \bluebold{tra}  \bluebold{ada},  muara  baru  \ldots\\
  {}   \textsc{neg}  exist  river.mouth  be.new  \\
\glt Speaker-2: ‘\bluebold{no}, (this is) the new river mouth [(it’s) the old river mouth that is (deep)]’ \textstyleExampleSource{[080927-003-Cv.0010-0011]}
\z
\z

\ea
\label{Example_13.29}
\ea
\label{Example_13.29a}
\gll Speaker-1:  de  punya  paytua?\\ %
  {}   \textsc{3sg}  \textsc{poss}  husband\\
\glt Speaker-1: ‘(was it) her husband?’
\vspace{10pt}
\ex
\label{Example_13.29b}
\gll    Speaker-2:  \bluebold{bukang},  de  punya  pacar\\
 {}    \textsc{neg}  \textsc{3sg}  \textsc{poss}  date\\
\glt Speaker-2: ‘\bluebold{no}, (it was) her lover’ \textstyleExampleSource{[081006-022-CvEx.0044-0045]}
\z
\z

At times, speakers employ an alternative strategy to respond to polar questions as shown in (\ref{Example_13.30}). Speakers may reply to a \isi{polar question} without giving an explicit answer in the affirmative or negative. Instead they provide additional information and leave it to their interlocutor to interpret this answer as a positive or a negative reply. This is shown with the implied negative answer in (\ref{Example_13.30b}). When interlocutors do not know the answer, they typically reply with \textitbf{tida}/\textitbf{tra taw} ‘(I) don’t know’.


\begin{styleExampleTitle}
Alternative answers to polar questions
\end{styleExampleTitle}
\ea
\label{Example_13.30}
\ea
\label{Examle_13.30a}
\gll Speaker-1:  di  sini  tra  pahit?\\ %
  {}   at  \textsc{l.prox}  \textsc{neg}  be.bitter\\
\glt [Discussing various melinjo varieties] Speaker-1: ‘(the melinjo varieties) here are not bitter?’
\vspace{10pt}
\ex
\label{Example_13.30b}
\gll  Speaker-2:  {\bluebold{Ø}},  Jayapura  pu  pahit\\
 {}  {}    Jayapura  \textsc{poss}  be.bitter\\
\glt Speaker-2: ‘(\bluebold{no}, the ones from) Jayapura are bitter’ \textstyleExampleSource{[080923-004-Cv.0011-0012]}
\z
\z

\subsubsection[Marked biased polar questions]{Marked biased polar questions}
\label{Para_13.2.2.2}
Marked polar questions are defined as questions which convey a bias toward the expected answer, hence “biased” questions (\citealt{Moravcsik.1971} in \citealt[180]{Sadock.1985}). Biased questions allow speakers “to express [{\ldots} their] belief that a particular answer is likely to be correct and to request assurance that this belief is true”  \citep[180]{Sadock.1985}. More specifically, positively biased questions signal that the speaker is in favor of a positive answer, while negatively biased questions indicate that the speaker expects a negative answer. 



Papuan Malay biased questions are presented in (\ref{Example_13.31}) to (\ref{Example_13.35}). While the corpus contains both positively and negatively biased questions, positively biased ones, as in (\ref{Example_13.31}) to (\ref{Example_13.33}), occur much more often than negatively biased ones, as in (\ref{Example_13.34}) or (\ref{Example_13.35}).



Biased questions are usually formed with the tags \textitbf{to} ‘right?’ or \textitbf{e} ‘eh?’. Prosodically, these questions are marked with a rising pitch on the \isi{tag} (see §\ref{Para_5.13.1} for more details concerning the semantics of both tags). The examples in (\ref{Example_13.31a}) and (\ref{Example_13.32a}) show positive bias, while (\ref{Example_13.34a}) and (\ref{Example_13.35a}) show negative bias, using the negator \textitbf{tida}/\textitbf{tra} ‘\textsc{neg}’. Less often, a positive bias is marked with affirmative \textitbf{yo} ‘yes’ as in (\ref{Example_13.33a}). Answers to biased polar questions follow the same patterns as answers to unbiased ones, as discussed in §\ref{Para_13.2.2.1}.

\largerpage[2]
\begin{styleExampleTitle}
Positively biased polar questions
\end{styleExampleTitle}

\ea
\label{Example_13.31}
\ea
\label{Example_13.31a}
\gll  {Speaker-1:} {yang} {dekat} {ada} {ruma} {\bluebold{to}?}\\ %
 {}    \textsc{rel}  near  exist  house  right?\\
\glt [Asking about a certain tree:] Speaker-1: ‘(the one that’s) close by (where) the houses are, \bluebold{right?}’
\vspace{10pt}
\ex
\label{Example_13.31b}
\gll    Speaker-2:  \bluebold{mm-mm},  ruma  di  pante\\
 {}    mhm  house  at  coast\\
\glt Speaker-2: ‘\bluebold{mhm}, the houses along the beach’ \textstyleExampleSource{[080917-009-CvEx.0012-0013]}
\vspace{10pt}
\z
\z
\ea
\label{Example_13.32}
\ea
\label{Example_13.32a}
\gll {Speaker-1:} {o,} {skarang} {orang} {su} {daftar} {\bluebold{e}?}\\ %
 {}   oh!  now  person  already  enroll  eh\\
\glt [About local elections:] Speaker-1: ‘oh, now people already (started) enrolling, \bluebold{eh?}’
\vspace{10pt}
\ex
\label{Example_13.32b}
\gll    Speaker-2:  \bluebold{yo},  tu  sa  pu  urusang\\
 {}    yes  \textsc{d.dist}  \textsc{1sg}  \textsc{poss}  affairs\\
\glt Speaker-2: ‘\bluebold{yes}, that’s my responsibility’ \textstyleExampleSource{[081005-001-Cv.0031-0032]}
\z
\z
\ea
\label{Example_13.33}
\ea
\label{Example_13.33a}
\gll {Speaker-1:} {jadi} {{itu}} {{nomor}} {{rekening}} {itu}  {pace}  {Natanael} {punya}  {\bluebold{yo}?}\\ %
 {}      so  {\textsc{d.dist}}  {number}  {bank.account}  \textsc{d.dist}  {man}  {Natanael} {\textsc{poss}}  {yes}\\
\glt Speaker-1: ‘so, what’s-its-name, that bank account number is Mr. Natanael’s, \bluebold{yes}?’
\vspace{10pt}
\ex
\label{Example_13.33b}
\gll    Speaker-2:  \bluebold{yo},  bukang  sa  punya\\
 {}   yes  \textsc{neg}  \textsc{1sg}  \textsc{poss}\\
\glt Speaker-2: ‘\bluebold{yes}, (it’s) not mine’ \textstyleExampleSource{[080922-001a-CvPh.0078-0079]}
\z
\z

\begin{styleExampleTitle}
Negatively biased polar questions
\end{styleExampleTitle}

\ea
\label{Example_13.34}
\ea
\label{Example_13.34a}
\gll {Speaker-1:} {ko} {\bluebold{tra}} {taw} {sa} {skola} {dari} {mana} {\bluebold{to}?}\\ %
  {}   \textsc{2sg}  \textsc{neg}  know  \textsc{1sg}  school  from  where  right?\\
\glt Speaker-1: ‘you don’t know from which school I am, \bluebold{right?}’
\vspace{10pt}
\ex
\label{Example_13.34b}
\gll    Speaker-2:  {\bluebold{Ø}}  sa  \bluebold{tida}  \bluebold{taw}  ((laughter))\\
  {} {}  \textsc{1sg}  \textsc{neg}  know  \\
\glt Speaker-2: ‘(\bluebold{yes}), I \bluebold{don’t know} ((laughter))’ \textstyleExampleSource{[080922-003-Cv.0031-0032]}
\z
\z
\ea
\label{Example_13.35}
\ea
\label{Example_13.35a}
\gll {Speaker-1:} {\bluebold{tida}} {di} {Beneraf} {\bluebold{e}?}\\ %
 {}    \textsc{neg}  at  \ili{Beneraf}  eh\\
\glt Speaker-1: ‘(they) are \bluebold{not} in \ili{Beneraf}, \bluebold{eh?}’
\vspace{10pt}
\ex
\label{Example_13.35b}
\gll   Speaker-2:  \bluebold{mm}{}-\bluebold{mm}\\
 {}    mhm\\
\glt Speaker-2: ‘\bluebold{mhm}’ \textstyleExampleSource{[080925-003-Cv.0173-0174]}
\z
\z

\subsection{Alternative questions}
\label{Para_13.2.3}
In Papuan Malay, alternative questions are formed with the alternative-marking \isi{conjunction} \textitbf{ka} ‘or’ (see also §\ref{Para_14.2.2.2}). They require the interlocutor to choose the supposedly right answer from a list of possible answers, as shown in (\ref{Example_13.36}) to (\ref{Example_13.41}).



The alternatives can be overtly listed as in (\ref{Example_13.36}) or (\ref{Example_13.37}), in which case they are linked with postposed \textitbf{ka} ‘or’. The question can also contain just one proposition and its \isi{negation}, as in (\ref{Example_13.38}) or (\ref{Example_13.39}), in which case the proposition is marked with \textitbf{ka} ‘or’ followed by negator \textitbf{tida} ‘\textsc{neg}’. Rather often, though, the negator is omitted, as in (\ref{Example_13.40}) and (\ref{Example_13.41}).


\ea
\label{Example_13.36}
\gll {bapa} {pake} {kartu} {apa} {ka?} {AS} {\bluebold{ka}?} {Simpati} {\bluebold{ka}?}\\ %
 father  use  card  what  or  AS  or  Simpati  or\\
\glt 
‘you (‘father’) use what (kind of SIM) card? AS \bluebold{or} Simpati?’ \textstyleExampleSource{[081014-016-Cv.0012]}
\z
\ea
\label{Example_13.37}
\gll {sa} {{tu}} {{biasa}} {{bilang}} {sama} {ana{\Tilde}ana} {di} {skola,}\\ %
 \textsc{1sg}  {\textsc{d.dist}}  {be.usual}  {say}  with  \textsc{rdp}{\Tilde}child  at  school\\
\gll {sala}  {\bluebold{ka}?}  {benar}  {\bluebold{ka}?}\\
 {be.wrong}  {or}  {be.true}  {or}\\
\glt 
‘I (\textsc{emph}) usually ask the kids in school, ``(is this) right \bluebold{or} wrong?''' \textstyleExampleSource{[081014-015-Cv.0029]}
\z

\ea
\label{Example_13.38}
\gll {kira{\Tilde}kira} {bisa} {kenal} {bapa} {\bluebold{ka}} {\bluebold{tida}?}\\ %
 \textsc{rdp}{\Tilde}think  be.able  know  father  or  \textsc{neg}\\
\glt 
‘do you think you can recognize me (‘father’) \bluebold{or not}?’ \textstyleExampleSource{[080922-001a-CvPh.1301]}
\z

\ea
\label{Example_13.39}
\gll {mama} {Rahab} {ada} {datang} {ke} {ruma} {\bluebold{ka}} {\bluebold{tida}?}\\ %
 mother  Rahab  exist  come  to  house  or  \textsc{neg}\\
\glt 
‘did mother Rahab come (\textsc{emph}) to the house \bluebold{or not}?’ \textstyleExampleSource{[081110-003-Cv.0001]}
\z

\ea
\label{Example_13.40}
\gll {de} {su} {datang} {\bluebold{ka}?}\\ %
 \textsc{3sg}  already  come  or\\
\glt 
‘did he already come \bluebold{or (not)}?’ \textstyleExampleSource{[080925-003-Cv.0138]}
\z

\ea
\label{Example_13.41}
\gll {ko} {ada} {karet} {\bluebold{ka}?}\\ %
 \textsc{2sg}  exist  rubber  or\\
\glt
‘do you have rubber bands \bluebold{or (not)}?’ \textstyleExampleSource{[081110-004-Cv.0008]}
\z

\section{Directive clauses}
\label{Para_13.3}
In Papuan Malay, three different types of directive clauses can be distinguished: imperatives and hortatives (§\ref{Para_13.3.1}), permissions and obligations (§\ref{Para_13.3.2}), and prohibitives (§\ref{Para_13.3.3}). They are used with any kind of predicate. Syntactically, directive clauses have the same structure as declarative clauses.


\subsection{Imperatives and hortatives}
\label{Para_13.3.1}
Papuan Malay employs imperatives and hortatives to issues commands. Imperatives always involve the second person, given that the addressee is the one who is expected to carry out the requested action, as shown in (\ref{Example_13.42}) to (\ref{Example_13.46}). In hortatives, by contrast, any person other than the addressee is expected to carry out the requested action. Hence, hortatives involve first and third persons, as shown in (\ref{Example_13.47}) to (\ref{Example_13.51}). In addition, Papuan Malay also employs a number of strategies to strengthen or soften commands, as demonstrated in (\ref{Example_13.52}) to (\ref{Example_13.61}).



Imperative constructions have a second person subject, as shown in (\ref{Example_13.42}) and (\ref{Example_13.43}). The clauses in (\ref{Example_13.42a}) and (\ref{Example_13.43a}) are formed with second singular \textitbf{ko} ‘\textsc{2sg}’ subjects. Depending on the context they can receive a declarative or an \isi{imperative} reading. It is also possible to omit the addressee, as demonstrated in (\ref{Example_13.42b}) and (\ref{Example_13.43b}). Single word imperatives, as in (\ref{Example_13.42b}), are rare, however. (The uses of \textitbf{suda} ‘already’ in directive clauses, as in (\ref{Example_13.43b}), are discussed together with the examples in (\ref{Example_13.54}) and (\ref{Example_13.55}).)


\begin{styleExampleTitle}
Imperatives: Syntactic structure\footnote{Documentation: \textitbf{bangung} ‘wake up’ 081006-022-CvEx.0081, 080918-001-CvNP.0038; \textitbf{pulang} ‘go home’ 081006-025-CvEx.0013, 081006-007-Cv.0001.}
\end{styleExampleTitle}
\vspace{-13pt}
\ea
\label{Example_13.42}
\begin{multicols}{2}
\ea
\label{Example_13.42a}
\gll  {\bluebold{ko}} {bangung}!  \\ %
   \textsc{2sg}  wake.up \\
\glt    ‘\bluebold{you} wake up!\\
\columnbreak
\ex
\label{Example_13.42b}
\gll {e} {bangung!}\\
 hey!  wake.up\\
 \glt  {‘hey, wake up!’}
 \z
 \end{multicols}
 \z
\ea
\label{Example_13.43}
\begin{multicols}{2}
\setlength{\columnsep}{5pt}
\ea
\label{Example_13.43a}
\gll  {\bluebold{ko}} {pulang}(!)\\ %
   \textsc{2sg}  go.home\\
\glt  \mbox{‘\bluebold{you} went home’/\bluebold{you} go home!’}
\ex
\label{Example_13.43b}
\gll {pulang} {suda!}\\
 go.home  already\\
 \glt   {‘go home already!’}
\z
\end{multicols}
\z

More examples of imperatives clauses are presented in (\ref{Example_13.44}) to (\ref{Example_13.46}), with second person singular addressees in (\ref{Example_13.44}) and (\ref{Example_13.46}), and second person plural addressees in (\ref{Example_13.45}). These examples also illustrate that imperatives are formed with \isi{trivalent} verbs as in (\ref{Example_13.44}), \isi{bivalent} verbs as in (\ref{Example_13.45}), or \isi{monovalent} verbs such as stative \textitbf{diam} ‘be quiet’ in (\ref{Example_13.46}); see also \isi{monovalent} dynamic \textitbf{pulang} ‘go home’ in (\ref{Example_13.43}).


\begin{styleExampleTitle}
Imperatives formed with tri-, bi-, and \isi{monovalent} verbs
\end{styleExampleTitle}

\ea
\label{Example_13.44}
\gll {\bluebold{ko}} {ambil} {sa} {air!}\\ %
 \textsc{2sg}  fetch  \textsc{1sg}  water\\
\glt 
‘\bluebold{you} fetch me water!’ \textstyleExampleSource{[081006-024-CvEx.0092]}
\z

\ea
\label{Example_13.45}
\gll {\ldots} {trus} {\bluebold{kam}} {\bluebold{dua}} {cuci} {celana} {di} {situ!}\\ %
 {}  next  \textsc{2pl}  two  wash  trousers  at  \textsc{l.med}\\
\glt 
[A mother addressing her young sons:] ‘[hey, you two go bathe in the sea already!,] then \bluebold{you two} wash (your) trousers there!’ \textstyleExampleSource{[080917-006-CvHt.0007]}
\z

\ea
\label{Example_13.46}
\gll {\bluebold{ko}} {jangang} {bicara} {lagi,} {\bluebold{ko}} {diam!}\\ %
 \textsc{2sg}  \textsc{neg.imp}  speak  again  \textsc{2sg}  be.quiet\\
\glt 
‘\bluebold{you} don’t talk again!, \bluebold{you} be quiet!’ \textstyleExampleSource{[081029-004-Cv.0072]}
\z


Hortatives are typically expressed with clause-initial \textitbf{biar} ‘let’. It exhorts the addressee to let or allow the desired future state of affairs come true, as illustrated in (\ref{Example_13.47}) to (\ref{Example_13.50}).


\begin{styleExampleTitle}
Hortatives with clause-initial \textitbf{biar} ‘let’
\end{styleExampleTitle}

\ea
\label{Example_13.47}
\gll {kalo} {nanti} {tong} {maing} {\bluebold{biar}} {\bluebold{sa}} {cadangang!}\\ %
 if  very.soon  \textsc{1pl}  play  let  \textsc{1sg}  reserve\\
\glt 
‘later when we play (volleyball), \bluebold{let me} be a reserve!’ \textstyleExampleSource{[081109-001-Cv.0154]}
\z

\ea
\label{Example_13.48}
\gll  \bluebold{biar}  \bluebold{tong}  tinggal  di  situ!\\
 let  \textsc{1pl}  stay  at  \textsc{l.med}\\
\glt 
‘\bluebold{let us} live there!’ \textstyleExampleSource{[081110-008-CvNP.0091]}
\z

\ea
\label{Example_13.49}
\gll  yo,  \bluebold{biar}  \bluebold{de}  juga  liat  sa!\\
 yes  let  \textsc{3sg}  also  see  \textsc{1sg}\\
\glt 
‘yes, \bluebold{let her} also see me!’ \textstyleExampleSource{[081015-005-NP.0013]}
\z

\ea
\label{Example_13.50}
\gll {\bluebold{biar}} {\bluebold{dong}} {ejek{\Tilde}ejek} {bapa!,} {tida} {apa{\Tilde}apa} {to?}\\ %
 let  \textsc{3pl}  \textsc{rdp}{\Tilde}mock  father  \textsc{neg}  \textsc{rdp}{\Tilde}what  right?\\
\glt 
‘\bluebold{let them} mock me (‘father’)!, it doesn’t matter, right?’ \textstyleExampleSource{[080922-001a-CvPh.0180]}
\z


First person plural hortatives can also be formed without \textitbf{biar} ‘let’, as shown in (\ref{Example_13.51}). In this case, the context shows whether the utterance is a \isi{hortative} such as the first \textitbf{kitong dua pulang} ‘(let) the two of us go home!’ token, or a declarative such as the second occurrence of \textitbf{kitong dua pulang} ‘the two of us went home’.


\begin{styleExampleTitle}
First person plural hortatives without clause-initial \textitbf{biar} ‘let’
\end{styleExampleTitle}

\ea
\label{Example_13.51}
\gll {dia} {bilang,} {\bluebold{Ø}} {\bluebold{kitong}} {\bluebold{dua}} {pulang!} {\ldots} {trus} {\bluebold{kitong}} {\bluebold{dua}} {pulang}\\ %
 \textsc{3sg}  say  {}  \textsc{1pl}  two  go.home {}   next  \textsc{1pl}  two  go.home\\
\glt 
‘he said, ‘\bluebold{(let) the two of us} go home!’ {\ldots} then \bluebold{the two of us} went home’ \textstyleExampleSource{[081015-005-NP.0035]}
\z


Papuan Malay also uses a number of strategies to strengthen or soften commands. Strengthening is illustrated in (\ref{Example_13.52}) to (\ref{Example_13.55}) and softening in (\ref{Example_13.56}) to (\ref{Example_13.61}).



Speakers can add \textitbf{ayo} ‘come on’ or \textitbf{suda} ‘already’ to commands or requests to make them more urgent and to strengthen them. Urgency-marking \textitbf{ayo} ‘come on’ can occur clause-initially, as in the \isi{imperative} in (\ref{Example_13.52}) and in the \isi{hortative} in (\ref{Example_13.53}), or clause-finally, also in (\ref{Example_13.52}); \textitbf{ayo} ‘come on!’ is unattested in hortatives with third persons. Urgency-marking \textitbf{suda} ‘already’, by contrast, always takes a postpredicate position as in the \isi{imperative} in (\ref{Example_13.54}) and the \isi{hortative} in (\ref{Example_13.55}).


\begin{styleExampleTitle}
Strengthening commands with \textitbf{ayo} ‘come on’ or \textitbf{suda} ‘already’
\end{styleExampleTitle}

\ea
\label{Example_13.52}
\gll {\bluebold{ayo}!,} {jalang} {ke} {Ise,} {\bluebold{ayo}!}\\ %
 come.on!  walk  to  Ise  come.on!\\
\glt 
‘\bluebold{come on}!, go to Ise, \bluebold{come on}!’ \textstyleExampleSource{[080917-008-NP.0065]}
\z

\ea
\label{Example_13.53}
\gll {\bluebold{ayo},} {kitong} {dua} {jalang} {cepat!,} {kitong} {dua} {jalang} {cepat!}\\ %
 come.on!  \textsc{1pl}  two  walk  be.fast  \textsc{1pl}  two  walk  be.fast\\
\glt 
‘\bluebold{come on!}, (let) the two of us walk fast!, (let) the two of us walk fast!’ \textstyleExampleSource{[081015-005-NP.0037]}
\z

\ea
\label{Example_13.54}
\gll {ey,} {kam} {dua} {pi} {mandi} {di} {laut} {\bluebold{suda}!}\\ %
 hey!  \textsc{2pl}  two  go  bathe  at  sea  already\\
\glt 
‘hey, you two go bathe in the sea \bluebold{already}!’ \textstyleExampleSource{[080917-006-CvHt.0007]}
\z

\ea
\label{Example_13.55}
\gll {ana} {kecil} {biar} {dong} {makang} {\bluebold{suda}!}\\ %
 child  be.small  let  \textsc{3pl}  eat  just\\
\glt 
‘(as for) the small children, let them eat \bluebold{already}!’ \textstyleExampleSource{[081002-001-CvNP.0051]}
\z


Requests or commands can be softened by adding clause-initial \textitbf{coba} ‘try’ as in (\ref{Example_13.56}), \textitbf{mari} ‘hither, (come) here’ as in (\ref{Example_13.57}), or \textitbf{tolong} ‘please’ (literally ‘help’) as in (\ref{Example_13.58}). This applies most often to imperatives, as in (\ref{Example_13.56}) and (\ref{Example_13.58}), and less often to hortatives, as in (\ref{Example_13.57}).


\begin{styleExampleTitle}
Softening commands with clause-initial \textitbf{coba} ‘try’, \textitbf{mari} ‘hither, (come) here’, or \textitbf{tolong} ‘help’
\end{styleExampleTitle}

\ea
\label{Example_13.56}
\gll {sa} {bilang,} {\bluebold{coba}} {ko} {tanya} {dorang!}\\ %
 \textsc{1sg}  say  try  \textsc{2sg}  ask  \textsc{3pl}\\
\glt 
‘I said, ``\bluebold{try} asking them!''' \textstyleExampleSource{[081025-008-Cv.0076]}
\z

\ea
\label{Example_13.57}
\gll {a,} {\bluebold{mari}} {kitong} {turung} {olaraga!}\\ %
 ah!  hither  \textsc{1pl}  descend  do.sports\\
\glt 
‘ah, \bluebold{come}, (let) us go down (to the beach) to do sports!’ \textstyleExampleSource{[080917-001-CvNP.0003]}
\z

\ea
\label{Example_13.58}
\gll {\bluebold{tolong}} {ceritra} {tu} {plang{\Tilde}plang!}\\ %
 help  tell  \textsc{d.dist}  \textsc{rdp}{\Tilde}be.slow\\
\glt 
[Addressing another adult:] ‘\bluebold{please}, talk (\textsc{emph}) slowly!’ \textstyleExampleSource{[081015-005-NP.0015]}
\z


Requests or commands can also be mitigated by adding in postpredicate position the temporal ad\isi{verb} \textitbf{dulu} ‘first, in the past’ as in (\ref{Example_13.59}), the focus ad\isi{verb} \textitbf{saja} ‘just’ as in (\ref{Example_13.60}), or the clause-final \isi{tag} \textitbf{e} ‘eh?’ as in (\ref{Example_13.61}). (For more details on adverbs see §\ref{Para_5.4} and on tags see §\ref{Para_5.13.1}.)


\begin{styleExampleTitle}
Softening commands with clause-final \textitbf{dulu} ‘first, in the past’, \textitbf{saja} ‘just’, or \textitbf{e} ‘eh’
\end{styleExampleTitle}

\ea
\label{Example_13.59}
\gll {sabar} {\bluebold{dulu}!,} {sabar} {\bluebold{dulu}!}\\ %
 be.patient  first  be.patient  first\\
\glt 
‘be patient \bluebold{for now}!, be patient \bluebold{for now}!’ \textstyleExampleSource{[080921-004b-CvNP.0051]}
\z

\ea
\label{Example_13.60}
\gll {sa} {blang,} {jalang} {\bluebold{saja}!}\\ %
 \textsc{1pl}  say  walk  just\\
\glt 
‘I said, ``just \bluebold{walk}!''' \textstyleExampleSource{[080917-008-NP.0117]}
\z

\ea
\label{Example_13.61}
\gll {ko} {kasi} {sama} {kaka} {mantri} {\bluebold{e}?!}\\ %
 \textsc{2sg}  give  to  oSb  male.nurse  eh\\
\glt
‘give (the keys) to the male nurse, \bluebold{eh}?!’ \textstyleExampleSource{[080922-010a-CvNF.0167]}
\z

\subsection{Permissions and obligations}
\label{Para_13.3.2}
 
Papuan Malay permissions are expressed with the auxiliary \isi{verb} \textitbf{bole} ‘may’, as illustrated in (\ref{Example_13.62}) to (\ref{Example_13.65}), while obligations are formed with the auxiliary \isi{verb} \textitbf{harus} ‘have to’, as shown in (\ref{Example_13.66}) and (\ref{Example_13.67}).



Permission-marking \textitbf{bole} ‘may’ most often occurs in single-word clauses, following a clause which depicts the permitted event or state, as in (\ref{Example_13.62}) or (\ref{Example_13.63}). Less often, \textitbf{bole} ‘may’ occurs between the subject and the predicate, as in (\ref{Example_13.64}). Only rarely, \textitbf{bole} ‘may’ occurs clause-initially, where it has scope over the entire clause, as (\ref{Example_13.65}).


\begin{styleExampleTitle}
Permissions with \textitbf{bole} ‘may’
\end{styleExampleTitle}

\ea
\label{Example_13.62}
\gll {kamu} {mo} {pacar,} {\bluebold{bole}}\\ %
 \textsc{2pl}  want  date  may\\
\glt 
[Addressing teenagers:] ‘(if) you want to date (someone) you \bluebold{may / are allowed to} (do so)’ \textstyleExampleSource{[081011-023-Cv.0269]}
\z

\ea
\label{Example_13.63}
\gll {ko} {mancing} {dari} {jembatang,} {\bluebold{bole}}\\ %
 \textsc{2sg}  fish  from  bridge  may\\
\glt 
[Addressing her son:] ‘(if) you’re fishing from the bridge, (you) \bluebold{may} (do so) / \bluebold{are allowed to} (fish)’ \textstyleExampleSource{[081025-003-Cv.0058]}
\z

\ea
\label{Example_13.64}
\gll {setiap} {kegiatang} {apa} {saja} {dorang} {\bluebold{bole}} {kerja}\\ %
 every  activity  what  just  \textsc{3pl}  may  work\\
\glt 
‘whatever activity, they \bluebold{may} / \bluebold{are allowed to} carry (it) out’ \textstyleExampleSource{[080923-007-Cv.0013]}
\z

\ea
\label{Example_13.65}
\gll {\ldots} {kalo} {tinggal} {di} {Arbais,} {\bluebold{bole}} {ko} {tokok} {sama{\Tilde}sama}\\ %
  {}   if  stay  at  Arbais  may  \textsc{2sg}  tap  \textsc{rdp}{\Tilde}be.same\\
\gll {dengang}  {kaka}\\
 {with}  {oSb}\\
\glt 
‘[my husband said, ‘(here in my village) don’t extract and crush the sago, you just knead and filter it,] when you’re staying in Arbais, (it is) \bluebold{allowed} (that) you (extract and) crush (the sago) together with (your) older sibling’ \textstyleExampleSource{[081014-007-CvEx.0058]}
\z


Obligation-marking \textitbf{harus} ‘have to’ typically takes a prepredicate position, as in (\ref{Example_13.66}). Alternatively, \textitbf{harus} ‘have to’ can occur clause-initially, where it has scope over the entire clause and reinforces the \isi{obligation}, as in (\ref{Example_13.67}).


\begin{styleExampleTitle}
Obligations with \textitbf{harus} ‘have to’
\end{styleExampleTitle}

\ea
\label{Example_13.66}
\gll {{besok}} {{pagi}} {{saya}} {{\bluebold{harus}}} {{cari}} {batrey,} {sa} {\bluebold{harus}}\\ %
 {tomorrow}  {morning}  {\textsc{1sg}}  {have.to}  {search}  battery  \textsc{1sg}  have.to\\
\gll bli  {pecis,}  {sa}  {\bluebold{harus}}  {ambil}  {senter}\\
 buy  {light.bulb}  {\textsc{1sg}}  {have.to}  {fetch}  {flashlight}\\
\glt 
[Getting ready for hunting:] ‘tomorrow morning I \bluebold{have to} get batteries, I \bluebold{have to} buy small light bulbs, I \bluebold{have to} take a flashlight’ \textstyleExampleSource{[080919-004-NP.0003]}
\z

\ea
\label{Example_13.67}
\gll {\bluebold{harus}} {kitong} {baik} {deng} {orang}\\ %
 have.to  \textsc{1pl}  be.good  with  person\\
\glt
‘we \bluebold{have to (}\blueboldSmallCaps{emph}\bluebold{)} be / (it’s) \bluebold{obligatory} (that) we are good to (other) people’ \textstyleExampleSource{[081110-008-CvNP.0166]}
\z

\newpage 
\subsection{Prohibitives}
\label{Para_13.3.3}
Papuan Malay prohibitives are typically formed with the negative \isi{imperative} \textitbf{jangang} ‘\textsc{neg.imp}, don’t’. Its main function is to signal the addressee that the action of the \isi{verb} is forbidden, as illustrated in (\ref{Example_13.68}) to (\ref{Example_13.71}). Quite often, however, a \isi{prohibitive} is softened, as shown in (\ref{Example_13.73}) to (\ref{Example_13.77}).



The main function of negative \isi{imperative} \textitbf{jangang} ‘\textsc{neg.imp}’, with its short form \textitbf{jang}, is to signal a straight-out \isi{prohibitive}. It occurs between the subject and the predicate, as in (\ref{Example_13.68}) and (\ref{Example_13.69}), or clause-initially where it has scope over the entire clause and reinforces the \isi{prohibitive}, as in (\ref{Example_13.70}) and (\ref{Example_13.71}). Besides, speakers also employ \textitbf{jangang} ‘\textsc{neg.imp}’ as stand-alone clauses which provide a response to a preceding \isi{prohibitive}, in the sense of ``(I would) never (do such a thing)'', as in (\ref{Example_13.72}).


\begin{styleExampleTitle}
Prohibitives with \textitbf{jangang} ‘\textsc{neg.imp}’
\end{styleExampleTitle}

\ea
\label{Example_13.68}
\gll {Wili} {ko} {\bluebold{jangang}} {gara{\Tilde}gara} {tanta} {dia} {itu!}\\ %
 Wili  \textsc{2sg}  \textsc{neg.imp}  \textsc{rdp}{\Tilde}irritate  aunt  \textsc{3sg}  \textsc{d.dist}\\
\glt 
[Addressing a young boy:] ‘you Wili \bluebold{don’t} irritate that aunt!’ \textstyleExampleSource{[081023-001-Cv.0038]}
\z

\ea
\label{Example_13.69}
\gll {kamorang} {\bluebold{jangang}} {pukul} {dia!}\\ %
 \textsc{2pl}  \textsc{neg.imp}  hit  \textsc{3sg}\\
\glt 
‘\bluebold{don’t} beat him!’ \textstyleExampleSource{[081015-005-NP.0024]}
\z

\ea
\label{Example_13.70}
\gll {\bluebold{jangang}} {ko} {pergi!}\\ %
 \textsc{neg.imp}  \textsc{2sg}  go\\
\glt 
‘\bluebold{don’t} you go!’ \textstyleExampleSource{[081025-006-Cv.0192]}
\z

\ea
\label{Example_13.71}
\gll {Klara,} {\bluebold{jangang}} {ko} {gara{\Tilde}gara} {dia!}\\ %
 Klara  \textsc{neg.imp}  \textsc{2sg}  \textsc{rdp}{\Tilde}irritate  \textsc{3sg}\\
\glt 
‘Klara, \bluebold{don’t} you irritate him!’ \textstyleExampleSource{[080917-003b-CvEx.0027]}
\z

\ea
\label{Example_13.72}
\gll {\ldots} {a,} {\bluebold{jangang}!,} {sa} {tida} {bisa} {buang} {takaroang}\\ %
  {} ah  \textsc{neg.imp}  \textsc{1sg}  \textsc{neg}  be.able  discard  be.chaotic\\
\glt 
‘[he said (to me), ‘don’t throw away (your betel nut waste)’, (I said),] ``ah, \bluebold{never}!, I can’t throw (it) away randomly''' \textstyleExampleSource{[081025-008-Cv.0012]}
\z


Prohibitives can be softened by employing \textitbf{tida}/\textitbf{tra bole} ‘shouldn’t’ (literally ‘may not’). Most often, \textitbf{tida}/\textitbf{tra bole} ‘may not’ occurs between the subject and the predicate, as in (\ref{Example_13.73}) and (\ref{Example_13.74}). Alternatively, although rarely, it occurs clause-initially, where it has scope over the entire clause, as in (\ref{Example_13.75}). In addition, speakers use \textitbf{tida}/\textitbf{tra bole} ‘may not’ as stand-alone clauses, which refer back to the speakers’ own or their interlocutors’ preceding statements about a state of affairs, as in (\ref{Example_13.76}) and (\ref{Example_13.77}), respectively.


\begin{styleExampleTitle}
Prohibitives with \textitbf{tida}/\textitbf{tra bole} ‘may not’
\end{styleExampleTitle}

\ea
\label{Example_13.73}
\gll {sa} {\bluebold{tida}} {\bluebold{bole}} {di} {depang!,} {saya} {harus} {di} {blakang} {skali}\\ %
 \textsc{1sg}  \textsc{neg}  may  at  front  \textsc{1sg}  have.to  at  backside  very\\
\glt 
‘I \bluebold{shouldn’t} be in front!, I had to stay in the very back’ \textstyleExampleSource{[081029-005-Cv.0133]}
\z

\ea
\label{Example_13.74}
\gll {mama} {\bluebold{tra}} {\bluebold{bole}} {lipat!,} {mama} {harus} {kas} {panjang} {kaki}\\ %
 mother  \textsc{neg}  may  fold  mother  have.to  give  be.long  foot\\
\glt 
[Addressing someone with a bad knee:] ‘you (‘mother’) \bluebold{shouldn’t} fold (your legs) under!, you (‘mother’) have to stretch out (your) legs’ \textstyleExampleSource{[080921-004a-CvNP.0069]}
\z

\ea
\label{Example_13.75}
\gll {\bluebold{tida}} {\bluebold{bole}} {ko} {ceritra} {orang!}\\ %
 \textsc{neg}  may  \textsc{2sg}  tell  person\\
\glt 
‘you \bluebold{shouldn’t (}\blueboldSmallCaps{emph}\bluebold{)} tell other people!’ \textstyleExampleSource{[081110-008-CvNP.0072]}
\z

\ea
\label{Example_13.76}
\gll {\ldots} {bunga{\Tilde}bunga} {suda} {habis,} {\bluebold{tida}} {\bluebold{bole}!}\\ %
  {} \textsc{rdp}{\Tilde}flower  already  be.used.up  \textsc{neg}  may\\
\glt 
[Addressing a child who had picked the speaker’s flowers:] ‘[(the flowers) over there (you) already picked (them) until (they were) all gone,] the flowers are already gone, (you) \bluebold{shouldn’t} (have done that)!’ \textstyleExampleSource{[081006-021-CvHt.0001]}
\z

\ea
\label{Example_13.77}
\gll {Speaker-2:} {a,} {\bluebold{tida}} {\bluebold{bole}!}\\ %
 {}  ah!  \textsc{neg}  may\\
\glt 
[About membership in a committee:] [Speaker-1: ‘the two of them are the committee’]\\
Speaker-2: ‘ah, (that) \bluebold{shouldn’t} be!’ \textstyleExampleSource{[080917-002-Cv.0015-0016]}
\z
