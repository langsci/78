\chapter[Conventions for examples]{Conventions for examples}\label{Para_0}

The examples in this book are taken from the recorded corpus. For each example the record number of the original WAV sound file (see §1.11.4.1) is given. This record number also includes a reference number, as each interlinear text is broken into referenced units. Hence, the example number 080919-007-CvNP.0015 refers to line or unit 15 in the record 080919-007-CvNP. Elicited examples, including constructed sentences for grammaticality judgments, are labeled as ``elicited''. For each of these examples the respective Toolbox record/reference number is given. All examples are numbered consecutively throughout each chapter (the same applies to tables and figures).

The conventions for presenting the Papuan Malay examples, interlinear gloss\-es, and the translations of the examples into English are presented in \tabref{Table_0.1a} and \tabref{Table_0.1b}.

\begin{table}[b]
\caption{Papuan Malay example and translation conventions\label{Table_0.1a}}

\begin{tabular}{lp{9.5cm}}
\lsptoprule
Convention & Meaning\\
\midrule

\multicolumn{2}{l}{Papuan Malay example}\\
\midrule
\bluebold{bold} & highlights parts of the example pertinent for the discussion\\
{\Tilde} & separates reduplicant and base\\
– & morpheme boundary\\
= & clitic boundary\\
Ø & omitted constituent\\
\textellipsis & ellipsis\\
{\textbar} & intonation breaks\\
\squarebrackets{} & surrounds utterances in a language other than Papuan Malay, or instances of unclear pronunciation\\
(( )) & surrounds nonverbal vocalizations, such as laughter or pauses\\
* & precedes ungrammatical examples\\
?? & precedes only marginally grammatical examples\\
á & acute accent signals a slight increase in pitch of the stressed syllable\\
VVV & vowel \isi{lengthening}\\
Is & utterance in the \ili{Isirawa} language\\
\textsc{up} & unclear pronunciation\\
\textsubscript{i}, \textsubscript{j} & subscript letters keep track of what different terms refer to\\
\lspbottomrule
\end{tabular}
\end{table}
\begin{table}
\caption{Papuan Malay example and translation conventions continued.\label{Table_0.1b}}

\begin{tabular}{lp{9.5cm}}
\lsptoprule
Convention & Meaning\\
\midrule
\multicolumn{2}{l}{Interlinear gloss}\\
\midrule
. & separates words glossing single Papuan Malay words for which English is lacking single-word equivalents, as with \textitbf{papeda} ‘sagu.porridge’\\
: & separates formally segmentable morphemes without marking the morpheme boundaries in the corresponding Papuan Malay words, either to keep the text intact and/or because it is not relevant, as in \textitbf{tujuangnya} `purpose:\textsc{3possr}'\\
\squarebrackets{} & surrounds truncated utterances, or speech mistakes\\

\textsc{tru} & truncated utterance which results from a false start, or an interruption, as in \textitbf{ora} ‘\textsc{tru}{}-person’; the untruncated lexeme is \textitbf{orang} ‘person’\\
\textsc{spm} & speech mistake, as in \textitbf{ar} ‘\textsc{spm}{}-fetch’; the correct form is \textitbf{ambil} `fetch'\\
\midrule
\multicolumn{2}{l}{Translation}\\
\midrule
\bluebold{bold} & highlights the part of the translation relevant for the discussion\\
( ) & surrounds parts of the translation which do not have a parallel in the example, such as explanations or omitted arguments\\
\squarebrackets{} & surrounds the record/reference number\\
\squarebrackets{} & surrounds utterances in the \ili{Isirawa} language, instances of unclear pronunciation, or speech mistakes\\
(( )) & surrounds nonverbal vocalizations, such as laughter or pauses\\
Is & utterance in the \ili{Isirawa} language\\
\textsc{spm} & speech mistake\\
\textsc{tru} & truncated utterance\\
\textsc{up} & unclear pronunciation\\
\textsubscript{i}, \textsubscript{j} & subscript letters keep track of what different terms refer to\\
\lspbottomrule
\end{tabular}
\end{table}
In the examples, commas mark intonation breaks, question marks signal question intonation, and exclamation marks indicate directive speech acts and exclamations. Where considered relevant for the discussion, intonation breaks are indicated with ``{\textbar}” rather than with a comma. Morpheme breaks are shown in \chapref{Para_3}, which discusses ``Word-formation”. In subsequent chapters, though, they are usually not shown, given the low functional load of \isi{affixation} in Papuan Malay; the exception is that hyphens are still used in compounds. Names are substituted with aliases to guard anonymity.

In the translations, gender, tense, and aspect are often not deducible; they are given as in the original context.

When parts of an example are quoted in the body text, they are marked in \textitbf{italic}.
