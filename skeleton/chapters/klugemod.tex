

%\setcounter{page}{1}\begin{styleBookTitlexvipt}
A grammar of Papuan Malay
\end{styleBookTitlexvipt}

\chapter[]{}
\chapter{}

Angela Kluge

13 January 2016

%\setcounter{page}{1}\begin{styleDedication}
buat Kori\textstyleunicode{\textsuperscript{†}}, Sarlota\textstyleunicode{\textsuperscript{†}}, Nela\textstyleunicode{\textsuperscript{†}} dorang
\end{styleDedication}

%\setcounter{page}{1}\chapter[Acknowledgements]{Acknowledgements}

This book is the revised version of my PhD dissertation, published by LOT by the same title, which I successfully defended in June 2014 at Leiden University. The original dissertation has undergone a number of revisions; overall, however, its contents have remained the same.



The dissertation and this book could not have been written without the help of many people who have assisted, supported, guided, and encouraged me over the last years.
\end{styleBodyvafter}


First of all I want to thank Kori\textsuperscript{†} and Sarlota\textsuperscript{†} Merne and their large family in Sarmi and Webro who welcomed me into their house and family. They were always willing to help me, patiently answered all my questions, and let me record their conversations. Their love and friendship made me feel at home during my stays in Sarmi. I can only mention of few other members of their family and guests here: Absalom, Alex, Anto, Aris, Beni, Desi, Dina, Domi, Edo, Elda, Febe, Fiki, Fritz\textsuperscript{†}, Hosea, Ice, Ici, Lina, Linda, Lodo, Martinus, Milka, Nela\textsuperscript{†}, Nesti, Ponti, Rolant, Rut, Salome, Sara, Sela, Siska, Soko, Thomas, Yako, Yos, Yulius, and Yusup.
\end{styleBodyvafter}


Next I want to thank my friends and my two home churches in Hoffnungsthal and Oberhausen who supported me financially, prayed for me, and encouraged me throughout this research project. I would especially like to thank my home Bible study group in Hoffnungsthal. I am also very grateful for the scholarships which the Ken Pike Endowment Fund and SIL Asia Area provided over the past few years. Another word of thanks goes to my office in Germany with its wonderful staff. I am so grateful how my supervisors, colleagues, and friends encouraged and assisted me in this entire undertaking. I thank you all for standing beside me so faithfully
\end{styleBodyvafter}


A word of thanks also goes to Chip Sanders of SIL International and to Mark Donohue and the participants of ISMIL 11 in Manokwari who encouraged me to take on this research project.
\end{styleBodyvafter}


I want to express my gratitude toward my promotor Maarten Mous and my co-promotor Marian Klamer. I want to thank Maarten for his constant support and encouragement during the process of this research project. I am very grateful to Marian for her supervision and for her commitment and encouragement throughout the years. I also want to thank her for her precise and detailed comments on different drafts of my original dissertation, which greatly improved its quality.
\end{styleBodyvafter}


I owe deep gratitude to my Papuan consultant and friend Ben Rumaropen. Working with him on Papuan Malay not only deepened my understanding of the language but was also a great joy and pleasure. Kawang, trima-kasi e?!
\end{styleBodyvafter}


I am very grateful to Antoinette Schapper for her generous support throughout the years and for working with me on different chapters of my dissertation. I also thank René van den Berg who read the first draft of my very first chapter so many years ago. His encouragement back then and throughout the past few years made a big difference for me in the entire process of this project.
\end{styleBodyvafter}


A number of scholars took the time to read individual chapters of my original dissertation. Thank you Geert Booij, Lourens de Vries, Harald Hammarström, Tom Headland, Erin Hesse, Paul Kroeger, Leonid Kulikov, Paul Lewis, Francesca Moro, Rick Nivens, Tom Payne, Christian Rapold, Uri Tadmor, Johnny Tjia, and Wilco van den Heuvel. Another word of thanks goes to those scholars and colleagues who discussed various aspects related to Papuan Malay with me, and assisted me in many different ways. I can only mention a few here: Jermy Balukh, Ted Bergman, Robert Blust, John Bowden, Joyce Briley, Adrian Clynes, Chuck Grimes, Arne Kirchner, Maarten Kossmann, Oliver Kroeger, Waruno Mahdi, Veronika Mattes, David Mead, Roger Mills, Edith Moravcsik, Scott Paauw, Angela Prinz, Werner Riderer, Jim Roberts, John Roberts, Silke Sauer, Ellie Scott, Graham Scott, Oliver Stegen, Joyce Sterner, Irene Tucker, Aone van Engelenhoven, and Catharina Williams-van Klinken.
\end{styleBodyvafter}


A word of thank also goes to the participants of the AN-Lang (Austronesian Languages) Mailing list and the SIL Linguistics Mailing list for responding to my inquiries and helping me in getting a better grasp of Papuan Malay.
\end{styleBodyvafter}


I also want to thank the two anonymous reviewers, LangSci’s consulting editor, and Martin Haspelmath, one of LangSci’s general editors. Their comments, suggestions and corrections greatly helped to improve the quality of this book.
\end{styleBodyvafter}


I am very grateful to the staff of SIL Asia Area, SIL Indonesia, and YABN, particularly Jackie Menanti, who provided me generously with their help in so many different ways. Another word of thanks goes to Emma Onim for transcribing some of the recorded texts, and to Lodowik Aweta for participating in the recording of the Papuan Malay word list. Thanks also go to Allan and Karen Buseman and Marcus Jaeger for their help with Toolbox and the Citavi.com staff for their wonderful bibliography and citation program.
\end{styleBodyvafter}


I owe deep gratitude to my colleagues Lenice and Jeff Harms. Staying with them in Sentani, and being able to share my fascination with Papuan Malay with Lenice was a joy and pleasure. I am also grateful to Rita Eltgroth and Joyce Briley. I enjoyed sharing offices in the library complex in Sentani with them and staying with Rita.
\end{styleBodyvafter}


Another word of thank goes to Anne-Christie Hellenthal, Francesca Moro, Victoria Nyst, and Christian Rapold. Over the years I spent many hours with them discussing “Gott und die Welt”, linguistics in general, and Papuan Malay in particular. It was always a source of great joy to spend time with you.
\end{styleBodyvafter}


I would like to thank the LUCL office staff, namely Gea Hakker-Prins, Merel van Wijk, Anne Rose Haverkamp, and Alice Middag-Cromwijk. I would like to thank them for their assistance with the administrative questions I had. I also want to thank the Accommodate staff, namely Linda Altman, Tessa van Witzenburg, and Eric van Wijk. Their help with housing issues made living in Leiden a pleasure. Thank you all very much!
\end{styleBodyvafter}


I am also very grateful to my colleagues who spent days proof-reading my original dissertation. I want to thank Sarah Hawton and Shirley McHale for their generous help. Another big thank-you goes to my colleagues Dennis Pepler and Stephen Palmstrom for their generous help with converting the Word version of this book to LaTeX. I am also grateful to Judy Lakeman and Jackie Gray for facilitating this help!
\end{styleBodyvafter}


Finally, I want to thank Gudrun and Peter. Eure Freundschaft und Unterstützung waren ein zuverlässiger Anker mitten in den Stürmen der letzten Jahre!
\end{styleBodyvafter}


My last and greatest thanks, however, goes to the triune God, who gives us life, love, and hope through his son Jesus Christ.
\end{styleBodyvafter}

%\setcounter{page}{1}\chapter[Summary contents]{Summary contents}

%\setcounter{page}{1}\chapter[List of abbreviations]{List of abbreviations}
\begin{multicols}{2}

\begin{tabular}{ll}
\lsptoprule
1, 2, 3 & 1st, 2nd, 3rd person\\
\textsc{acl} & accidental\\
\textsc{adjct} & adjunct\\
\textsc{adnom} & adnominal\\
AdPoss. & adnominal possessive\\
\textsc{adv} & adverb\\
\textsc{adv.t} & adverb, temporal\\
\textsc{advs} & adversative\\
\textsc{affr} & affricate\\
\textsc{affx} & affixation\\
\textsc{ag} & agent\\
\textsc{agt} & agentive\\
\textsc{al} & alienable\\
Alt. & alternative\\
\textsc{alv} & alveolar\\
\textsc{an} & animate\\
\textsc{apr} & approximant\\
\textsc{a-p-ult} & antepenultimate\\
\textsc{argt} & argument\\
\textsc{assct} & association\\
\textsc{attr} & attributive\\
\textsc{aug} & augmentation\\
\textsc{ben} & beneficiary\\
\textsc{c} & consonant\\
\textsc{caus} & causative\\
\textsc{circ} & circumstance\\
\textsc{cl} & clause\\
\textsc{cmpr} & comparative\\
\textsc{cnj} & conjunction\\
\textsc{cst} & contrastive\\
\textsc{d.dist} & demonstrative, distal\\
\textsc{d.prox} & demonstrative, proximal\\
\textsc{dem} & demonstrative\\
\textsc{dim} & diminution\\
\textsc{diph} & diphthong\\
\textsc{do} & direct object\\
\textsc{dy} & dynamic\\
\textsc{edc} & education\\
\textsc{+edc} & better educated\\
\textsc{\-edc} & less educated\\
\textsc{els} & elision\\
\textsc{emph} & emphasis, emphatic\\
\textsc{fric} & fricative\\
\textsc{glot} & glottal\\
\textsc{hum} & human\\
\textsc{ilct} & interlocutor\\
\textsc{inal} & inalienable\\
\textsc{inan} & inanimate\\
\textsc{ins} & instrument\\
\textsc{int} & interrogative\\
\textsc{intens} & intensity\\
\textsc{intr} & intransitive\\
Is & Isirawa\\
k.o. & kind of\\
\textsc{l.dist} & locative, distal\\
\textsc{l.med} & locative, medial\\
\textsc{l.prox} & locative, proximal\\
\textsc{l.t.} & low topics\\
\textsc{lab} & labial\\
\textsc{lat-aprx} & lateral-approximant\\
\textsc{lig} & ligature\\
lim. & limited\\
\textsc{liq} & liquid\\
\textsc{loc} & locative\\
\textsc{loct} & location\\
marg. & marginal\\
med. & medium\\
\textsc{mod} & modifier\\
\textsc{n} & noun\\
\textsc{n.com} & noun, common\\
\textsc{n.loc} & noun, location\\
\textsc{n.time} & noun, time-denoting\\
\textsc{nas} & nasal\\
\textsc{neg} & negation, negative\\
\textsc{neg.imp} & negative imperative\\
\textsc{nmlz} & nominalizer\\
\textsc{nom} & nominal\\
\textsc{np} & noun phrase\\
\textsc{num} & numeral\\
\textsc{o} & object\\
Obl. & oblique\\
\textsc{obstr} & obstruent\\
\textsc{omv} & object of mental verb\\
\textsc{orth} & orthography\\
Orthogr. & orthography\\
oSb & older sibling\\
\textsc{outsd} & outsider\\
\textsc{pal} & palatal\\
\textsc{pal-alv} & palato-alveolar\\
\textsc{pat} & patient\\
\textsc{pfx} & prefix\\
\textsc{phon} & phoneme\\
\textsc{pl} & plural\\
\textsc{pl-hold} & placeholder\\
\textsc{plos} & plosive\\
\textsc{pn} & noun, proper\\
\textsc{pol} & politics\\
\textsc{poss} & possessive\\
\textsc{possm} & possessum\\
\textsc{possr} & possessor\\
\textsc{pp} & prepositional phrase\\
prec. & precedes\\
\textsc{pred} & predicate, predicative\\
\textsc{pro} & personal pronoun\\
\textsc{prod} & productivity\\
\textsc{pronom} & pronominal\\
\textsc{p-ult} & penultimate\\
\textsc{qt} & quantifier\\
\textsc{quant} & quantity\\
\textsc{rc} & relative clause\\
\textsc{rdp} & reduplicant\\
\textsc{rec} & recipient\\
\textsc{recp} & reciprocal\\
\textsc{rel} & religion\\
\textsc{rel} & relativizer\\
\textsc{res} & result\\
\textsc{ret} & retention\\
\textsc{rhot} & rhotic\\
\textsc{s} & subject\\
s.o. & someone\\
s.th. & something\\
\textsc{sg} & singular\\
SI & Standard Indonesian\\
Sim. & similarity\\
\textsc{spk} & speaker\\
\textsc{spm} & speech mistake\\
\textsc{st} & stative\\
\textsc{stat} & status\\
+\textsc{stat} & higher social status\\
\textsc{\-stat} & lower social status\\
\textsc{std} & standard of comparison\\
\textsc{supl} & superlative\\
\textsc{sylb} & syllable\\
\textsc{top} & topic\\
\textsc{tru} & truncated\\
\textsc{ult} & ultimate\\
\textsc{up} & unclear pronunciation\\
\textsc{uv} & undergoer voice\\
\textsc{v} & vowel\\
\textsc{v} & verb\\
\textsc{v.bi} & verb, bivalent\\
\textsc{v.mo} & verb, monovalent\\
\textsc{v.tri} & verb, trivalent\\
\textsc{vblz} & verbalizer\\
\textsc{vdi} & verb, ditransitive\\
\textsc{vel} & velar\\
\textsc{voc} & vocative\\
\textsc{vp} & verb phrase\\
\textsc{vsi} & verb, intransitive stative\\
\textsc{vtr} & verb, transitive\\
ySb & younger sibling\\
\lspbottomrule
\end{tabular}
\end{multicols}

%\setcounter{page}{1}\chapter[Conventions for examples]{Conventions for examples}
\label{bkm:Ref434428827}
The examples in this book are taken from the recorded corpus. For each example the record number of the original WAV sound file (see §1.11.4.1) is given. This record number also includes a reference number, as each interlinear text is broken into referenced units. Hence, the example number 080919-007-CvNP.0015 refers to line or unit 15 in the record 080919-007-CvNP. Elicited examples, including constructed sentences for grammaticality judgments, are labeled as “elicited”. For each of these examples the respective Toolbox record/reference number is given. All examples are numbered consecutively throughout each chapter (the same applies to tables and figures).



The conventions for presenting the Papuan Malay examples, interlinear glosses, and the translations of the examples into English are presented in Table  ‎0 .1.
\end{styleBodyvvafter}

\begin{stylecaption}
\label{bkm:Ref434429302}Table ‎0.\stepcounter{Table}{\theTable}:  Papuan Malay example and translation conventions
\end{stylecaption}

\tablehead{
\multicolumn{2}{l}{ Convention} & \arraybslash Meaning\\
}
\begin{tabular}{lll}
\lsptoprule
\multicolumn{3}{l}{Papuan Malay example}\\
& \bluebold{bold} & highlights parts of the example pertinent for the discussion\\
& {\Tilde} & separates reduplicant and base\\
& – & morpheme boundary\\
& = & clitic boundary\\
& Ø & omitted constituent\\
& … & ellipsis\\
& {\textbar} & intonation breaks\\
& [ ] & surrounds utterances in a language other than Papuan Malay, or instances of unclear pronunciation\\
& (( )) & surrounds nonverbal vocalizations, such as laughter or pauses\\
& * & precedes ungrammatical examples\\
& ?? & precedes only marginally grammatical examples\\
& á & acute accent signals a slight increase in pitch of the stressed syllable\\
& VVV & vowel lengthening\\
& Is & utterance in the Isirawa language\\
& \textsc{up} & unclear pronunciation\\
& \textsuperscript{i}, \textsuperscript{j} & subscript letters keep track of what different terms refer to\\
\multicolumn{3}{l}{Interlinear gloss}\\
& . & separates words glossing single Papuan Malay words for which English is lacking single-word equivalents, as with \textitbf{papeda} ‘sagu.porridge’\\
& : & separates formally segmentable morphemes without marking the morpheme boundaries in the corresponding Papuan Malay words, either to keep the text intact and/or because it is not relevant, as in \textitbf{tujuangnya} ‘purpose:\textsc{3possr}’\\
& [ ] & surrounds truncated utterances, or speech mistakes\\
& \textsc{tru} & truncated utterance which results from a false start, or an interruption, as in \textitbf{ora} ‘\textsc{tru}{}-person’; the untruncated lexeme is \textitbf{orang} ‘person’\\
& \textsc{spm} & speech mistake, as in \textitbf{ar} ‘\textsc{spm}{}-fetch’; the correct form is \textitbf{ambil} ‘fetch’\\
\multicolumn{3}{l}{Translation}\\
& \bluebold{bold} & highlights the part of the translation relevant for the discussion\\
& ( ) & surrounds parts of the translation which do not have a parallel in the example, such as explanations or omitted arguments\\
& [ ] & surrounds the record/reference number\\
& [ ] & surrounds utterances in the Isirawa language, instances of unclear pronunciation, or speech mistakes\\
& (( )) & surrounds nonverbal vocalizations, such as laughter or pauses\\
& Is & utterance in the Isirawa language\\
& \textsc{spm} & speech mistake\\
& \textsc{tru} & truncated utterance\\
& \textsc{up} & unclear pronunciation\\
& \textsuperscript{i}, \textsuperscript{j} & subscript letters keep track of what different terms refer to\\
\lspbottomrule
\end{tabular}

In the examples, commas mark intonation breaks, question marks signal question intonation, and exclamation marks indicate directive speech acts and exclamations. Where considered relevant for the discussion, intonation breaks are indicated with “{\textbar}” rather than with a comma. Morpheme breaks are shown in Chapter 3, which discusses ‘Word-formation’. In subsequent chapters, though, they are usually not shown, given the low functional load of affixation in Papuan Malay; the exception is that hyphens are still used in compounds. Names are substituted with aliases to guard anonymity.
\end{styleBodyaftervbefore}


In the translations, gender, tense, and aspect are often not deducible; they are given as in the original context.
\end{styleBodyvafter}


When parts of an example are quoted in the body text, they are marked in \textitbf{bold italic}.
\end{styleBodyvafter}

%\setcounter{page}{1}\chapter[Maps]{Maps}
\section{Geographical maps}
\begin{styleCapFigure}
\label{bkm:Ref417137560}\label{bkm:Ref417122110}Figure ‎0.\stepcounter{Figure}{\theFigure}:  West Papua
\end{styleCapFigure}

\begin{styleCapFigure}
\label{bkm:Ref373252478}\label{bkm:Ref417122114}Figure ‎0.\stepcounter{Figure}{\theFigure}:  West Papua with its provinces Papua and Papua Barat
\end{styleCapFigure}

\begin{styleCapFigure}
\label{bkm:Ref375721640}\label{bkm:Ref417122217}Figure ‎0.\stepcounter{Figure}{\theFigure}:  Sarmi regency with some of its towns and villages
\end{styleCapFigure}

\section{Language map}
\begin{styleOvNvwnext}
In Figure  ‎0 .4, the Fedan language is listed as Podena, Dineor as Maremgi, Keijar as Keder, Mo as Wakde, and Sunum as Yamna (see §1.4).
\end{styleOvNvwnext}

\begin{styleCapFigure}
\label{bkm:Ref375722237}\label{bkm:Ref417121988}Figure ‎0.\stepcounter{Figure}{\theFigure}:  Austronesian and Papuan languages in the larger Sarmi region
\end{styleCapFigure}

%\setcounter{page}{1}\chapter[Introduction]{Introduction}

Papuan Malay is spoken in West Papua,\footnote{Formerly, West Papua was known as “Irian Jaya” or “West Irian”.\\
\\
\\
\\
\\
\\
\\
\\
\\
\\
\\
\\
\\
\\
\\
} which covers the western part of the island of New Guinea. The language is a non-standard variety of Malay, belonging to the Malayic branch within the Austronesian language family.\footnote{The Malayic branch also includes other eastern Malay varieties as well as Standard Malay and Indonesian {(Blust 2013: xxiv–xl)}. (See §1.2 for more details on the genetic affiliations of Papuan Malay.)\\
\\
\\
\\
\\
\\
\\
\\
\\
\\
\\
\\
\\
\\
\\
} Within the larger Malay continuum, Papuan Malay forms a distinct, structurally coherent unit.



In West Papua, Papuan Malay is the language of wider communication and the first or second language for an ever-increasing number of people of the area (ca. 1,100,000 or 1,200,000 speakers). While Papuan Malay is not officially recognized, and therefore not used in formal government or educational settings or for religious preaching, it is used in all other domains, including unofficial use in formal settings, and, to some extent, in the public media.
\end{styleBodyvafter}


This grammar describes Papuan Malay as spoken in the Sarmi area, which is located about 300 km west of Jayapura. Both towns are situated on the northeast coast of West Papua (see Figure  ‎0 .1 on p. \pageref{bkm:Ref417137560} and Figure  ‎0 .2 on p. \pageref{bkm:Ref373252478}). After a general introduction to the language, presented in this chapter, the grammar discusses the following topics, building up from smaller grammatical constituents to larger ones: phonology, word formation, word classes, noun phrases, adnominal possessive relations, prepositional phrases, verbal and nonverbal clauses, non-declarative clauses, and conjunctions and constituent combining.
\end{styleBodyvafter}


This chapter provides an introduction to Papuan Malay. The first section gives a brief introduction to the larger geographical setting of Papuan Malay (§1.1). The genetic affiliations and the dialect situation of the language are discussed in §1.2 and §1.3, respectively. The linguistic setting of Papuan Malay is examined in §1.4, followed in §1.5 by a description of its sociolinguistic profile and in §1.6 of its typological profile. Pertinent demographic information is given in §1.7, and an overview of the history of Papuan Malay is presented in §1.8. Previous research on the language is summarized in §1.9, followed in §1.10, by a brief overview of available materials in Papuan Malay. Finally, in §1.11, methodological aspects of the present study are described.
\end{styleBodyvxvafter}

\section{Geographical setting}
\label{bkm:Ref438036540}
Papuan Malay is mostly spoken in the coastal areas of West Papua. As there is a profusion of terms related to this geographical area, some terms need to be defined before providing more information on the geographical setting of Papuan Malay.


“West Papua”, the term adopted in this book, denotes the western part of the island of New Guinea. More precisely, the term describes the entire area west of the Papua New Guinea border up to the western coast of the Bird’s Head, as shown in Figure  ‎0 .1 (p. \pageref{bkm:Ref417137560}; see also §1.11.2 regarding the larger setting of the research location).\footnote{The term ‘West Papua’ is also used in the literature, as for instance in {\citet{King2004}, Kingsbury and \citet{Aveling2002}, }and {\citet{Tebay2005}}. More recently, {\citet{Gil2014}} has proposed the Malay term \textit{Tanah Papua} ‘Land of Papua’ for the western part of the island of New Guinea.\\
\\
\\
\\
\\
\\
\\
\\
\\
\\
\\
\\
\\
\\
\\
} In addition to the name “West Papua”, two related terms are used in subsequent sections, namely “Papua province” and “Papua Barat province”. Both refer to administrative entities within West Papua. As illustrated in Figure  ‎0 .2 (p. \pageref{bkm:Ref373252478}), Papua province covers the area west of the Papua New Guinea border up to the Bird’s Neck; the provincial capital is Jayapura. Papua Barat province, with its capital Manokwari, covers the Bird’s Head.
\end{styleBodyvafter}


West Papua occupies the western part of New Guinea which belongs to the eastern Malay Archipelago. With its 317,062 square km, it covers about 40\% of New Guinea’s landmass. Its length from the border with Papua New Guinea in the east to the western tip of the Bird’s Head is about 1,200 km. Its north-south extension along the border with Papua New Guinea is about 700 km. The central part of West Papua is dominated by the Maoke Mountains. They are an extension of the mountain ranges of Papua New Guinea and, for the most part, covered with tropical rainforest. The northern and southern lowlands are covered with lowland rain and freshwater swamp forests which are drained by major river systems, such as the Mamberamo in the north and the Digul in the south. (See {Encyclopædia Britannica Inc. 2001a-; 2001b-.})
\end{styleBodyvafter}


{Major areas with }substantial concentrations of Papuan Malay speakers are the coastal urban areas of Jayapura and Sarmi on the north coast, Merauke and Timika on the south coast, Fakfak and Sorong in the western part and Manokwari in the northeastern part of the Bird’s Head, and Serui on Yapen Island in Cenderawasih Bay. Other areas with substantial speaker numbers most likely include Nabire in the Bird’s Neck, Biak Island in Cenderawasih Bay, and possibly Wamena in the highlands in central West Papua. (See {Scott et al. 2008: 10}; see also Figure  ‎0 .2 on p. \pageref{bkm:Ref373252478}.{)}
\end{styleBodyvxvafter}

\section{Genetic affiliations}
\label{bkm:Ref370323838}
As a Malay language, Papuan Malay belongs to the Malayic sub-branch within the Malayo-Polynesian branch of the Austronesian language family. A review of the literature suggests, however, that the exact classification of Papuan Malay is difficult for three reasons. First, as discussed in §1.2.1, the internal classification of the Malayo-Polynesian subgroup is problematic. Moreover, there is a debate in the literature over the classification of the Malayic languages within Western-Malayo-Polynesian. Secondly, as discussed in §1.2.2, there is disagreement among scholars regarding the status of the eastern Malay varieties, including Papuan Malay, as to whether they are non-creole descendants of Low Malay or Malay-based creoles. Thirdly, there is an ongoing debate over the legitimacy of Papuan Malay as a distinct language, as discussed in §1.2.3.
\end{styleBodyxvafter}

\subsection{Papuan Malay, a Malayic language within Malayo-Polynesian}
\label{bkm:Ref437973391}
As a Malayic language, Papuan Malay belongs to the Malayo-Polynesian branch. Its classification within this branch is problematic, however.


In the literature the Malay languages are frequently classified as “Western Malayo-Polynesian” or “West-Malayo-Polynesian (see for instance {Adelaar 2001: 227}; {Nothofer 2009: 677}; {Tadmor 2009b: 791.})
\end{styleBodyvafter}


The existence of the Western Malayo-Polynesian subgroup, however, is not well established. {\citet[68]{Blust1999}, for instance, points }out that “Western Malayo-Polynesian does not meet the minimal criteria for an established subgroup”. Hence, {Blust} concludes that Western Malayo-Polynesian instead constitutes a “residue” of languages which do not belong to the Central- and Eastern-Malayo-Polynesian sub-branch {(1999: 68)}. Along similar lines, {\citet[14]{Adelaar2005c}} notes that Western Malayo-Polynesian “does not have a clear linguistic foundation […] and the genetic affiliations of its putative members remain to be investigated”. {Donohue and \citet{Grimes2008}} also discuss the problematic status of the Western Malayo-Polynesian subgroup. Based on phonological, morphological, and semantic innovations, the authors conclude that there is no basis for the Western Malayo-Polynesian and Central/Eastern-Malayo-Polynesian subgroups. In 2013, the status of the Western-Malayo-Polynesian (WMP) subgroup remains problematic, with {\citet[31]{Blust2013}} maintaining that it “is possible that WMP is not a valid subgroup, but rather consists of those MP [Malayo-Polynesian] languages that do not belong to CEMP [Central-Eastern Malayo-Polynesian]” (see also {Blust 2013: 741–742}).
\end{styleBodyvafter}


Moreover, there is disagreement among scholars with respect to the classification of the Malayic languages within Western Malayo-Polynesian. Based on phonological and morphological innovations, {Blust (1994: 31ff)} groups them within Malayo-Chamic which is one of five subgroups within Western-Malayo-Polynesian. The two branches of this grouping refer to the Malayic languages of insular Southeast Asia, and the Chamic languages of mainland Southeast Asia (see also {Blust 2013: 32}). {\citet{Adelaar2005a}}, by contrast, suggests that Malayic is part of a larger collection of languages, namely Malayo-Sumbawan. This group has three branches. One includes the sub-branches Malayic, Chamic, and Balinese-Sasak-Sumbawa, while the other two include Sundanese and Madurese. {\citet{Blust2010}}, however, rejects this larger Malayo-Sumbawan grouping. Based on lexical innovations, he argues that Malayic and Chamic form “an exclusive genetic unit” and should not be grouped together with Balinese, Sasak, and Sumbawanese ({2010: 80–81}; see also {Blust 2013: 736}). Hence, {Blust (2013: xxxii)} classifies Papuan Malay as a Malayic language within Malayo-Chamic. This classification for the Malay languages within Malayo-Chamic is also adopted by the \textstyleChItalic{Ethnologue} {\citep{LewisEtAl2015a}}.
\end{styleBodyvxvafter}

\subsection{Papuan Malay, a non-creole descendant of low Malay}
\label{bkm:Ref437441364}
Papuan Malay is a non-creole descendant of low Malay.\footnote{The term “low Malay” refers to “the colloquial form of Malay”, a trade language “existing in a diglossic situation […] with “High Malay” […] (which is usually defined as the classical literary language based upon the court language of Riau-Johor […])” ({Paauw 2003: 18–19; see also Paauw 2009: 18–25}).\\
\\
\\
\\
\\
\\
\\
\\
\\
\\
\\
\\
\\
\\
\\
} There is an ongoing discussion in the literature, however, regarding the status of the eastern Malay varieties, including Papuan Malay; that is, whether they are indeed non-creole descendants of low Malay or rather Malay-based creoles.



Three factors contribute to this discussion: (1) the “simple structure” of Papuan Malay and the other eastern Malay varieties, with their lack of inflectional morphology and limited derivational processes (see §1.6.1.2), (2) the influence from non-Austronesian languages which these languages, including Papuan Malay, show (see §1.6.2), and (3) the history of Malay as a trade language (see §1.8). These pertinent characteristics of the eastern Malay varieties receive different interpretations.
\end{styleBodyvafter}


Scholars such as {Adelaar and \citet[675]{Prentice1996}, }Donohue (2007b; 2011), and {Mc\citet{Whorter2001}} conclude that these languages best be characterized as Malay-based pidgins or creoles.
\end{styleBodyvafter}


By contrast, other scholars, such as {\citet{Collins1980}}, {\citet{Gil2001a}}, {\citet{Bisang2009}}, and {\citet{Paauw2013}}, and also {earlier contributions by \citet{Donohue2003} and Donohue and \citet{Smith1998},} challenge the alleged creole origins of the eastern Malay varieties, given that structural simplicity is also found in inherited Malay varieties and also given that linguistic borrowing is not limited to pidgins or creoles.
\end{styleBodyvafter}


This latter view is also the one adopted in the present description of Papuan Malay. The fact that Papuan Malay has a comparatively simple surface structure and some features typically found in Papuan but not in Austronesian languages is not sufficient evidence to classify Papuan Malay as a creole.
\end{styleBodyvafter}


Throughout the remainder of this section, the different positions regarding the creole versus non-creole status of the eastern Malay varieties are presented in more detail. The view that the eastern Malay varieties are creolized languages is discussed first.
\end{styleBodyvafter}


{Adelaar and \citet[675]{Prentice1996}} propose a list of eight structural features which illustrate the reduced morphology of the eastern Malay varieties and some of the linguistic features they borrowed from local languages. According to the authors, these features, which distinguish the eastern Malay varieties from the western Malay varieties and literary Malay, point to the pidgin origins of the eastern Malay varieties, including those of West Papua. Hence, {Adelaar and Prentice} propose the term \textstyleChItalic{Pidgin Malay Derived} dialects for these varieties. In a later study, {\citet[202]{Adelaar2005b}} refers to the same varieties as \textstyleChItalic{Pidgin-Derived} Malay varieties. Another researcher who supports the view that the (eastern Malay) varieties are creolized languages is {McWhorter (2001; 2005; 2007: 197–251}). Considering the structural simplicity of Malay and its history as a trade language, he comes to the conclusion that Malay is an “anomalously decomplexified” language which shows “the hallmark of a grammar whose transmission has been interrupted to a considerable degree {(2007: 197, 216)}. The \textstyleChItalic{Ethnologue} {\citep{LewisEtAl2015a}} also adopts the view that the eastern Malay varieties are creolized languages and classifies them as \textstyleChItalic{Malay-based creoles}; these varieties include Ambon, Banda, Kupang, Larantuka, Manado, North Moluccan, and Papuan Malay. (See also {Roosman 1982; Burung and Sawaki 2007}.)
\end{styleBodyvafter}


This view that the regional Malay varieties are creolized languages is further found in descriptions of individual eastern Malay varieties such as Ambon Malay, Kupang Malay, and Manado Malay.
\end{styleBodyvafter}


For Ambon Malay, {\citet[115]{Grimes1991}} argues that the language is a creole or nativized pidgin. This conclusion is based on linguistic, sociolinguistic, and historical data, which the author interprets in light of {Thomason and Kaufman’s (1988: 35)} framework of “contact-induced language change”. Following this framework, nativized pidgins are the long-term “result of mutual linguistic accommodation” and “simplification” in multi-lingual contact situations {(1988: 174, 205, 227)}. Along similar lines, {Jacob and \citet[337]{Grimes2011}} consider Kupang Malay to be a Malay-based creole that displays a substantial amount of influence from local substrate languages (see also {Jacob and Grimes 2006}). Manado Malay is also taken to be a creole that developed from a local variety of Bazaar Malay which is a \textstylest{Malay-}lexified pidgin ({Prentice 1994: 411; Stoel 2005: 8}).
\end{styleBodyvafter}


{Van \citet{Minde1997}}, in his description of Ambon Malay, and {\citet{Litamahuputty1994}}, in her grammar of Ternate Malay, by contrast, make no clear statements as to whether they consider the respective eastern Malay varieties to be creolized languages or not.
\end{styleBodyvafter}


In fact, the alleged creole status and pidgin origins of the regional (eastern) Malay varieties have been contested by a number of scholars. {\citet{Collins1980}}, {\citet{Wolff1988}}, {\citet{Gil2001a}}, {\citet{Bisang2009}}, and {\citet{Paauw2013}}, for instance, argue that structural simplicity per se is not evidence for the pidgin origins of a language. Nor is the borrowing of linguistic features. {\citet{Blust2013}} seems to have a similar viewpoint, although he does not overtly state this. Less clear is {Donohue’s (2003; 2007a; 2007b; 2011}) and {Donohue and Smith’s (1998) }position concerning the creole/non-creole status of the eastern Malay varieties.
\end{styleBodyvafter}


{\citet[35]{Bisang2009}} challenges the view that low degrees of complexity should be taken as an indication to the pidgin/creole origins of a given language. In doing so, he specifically addresses the viewpoints put forward by {McWhorter ( 2001; 2005}). Paying particular attention to the languages of East and Southeast Mainland Asia, {\citet{Bisang2009} }makes a distinction between overt and hidden complexity. The author shows that languages with a long-standing history may also have “simple surface structures […] which allow a number of different inferences and thus stand for hidden complexity” {(2009: 35)}. That is, such languages do not oblige their speakers to employ particular structures if those are understood from the linguistic or extralinguistic context.
\end{styleBodyvafter}


As far as particular regional Malay varieties are concerned, {\citet{Collins1980}}, for example, comes to the conclusion that Ambon Malay is not a creole. Examining sociocultural and linguistic evidence, the author compares Ambon Malay to Standard Malay and to the non-standard Malay variety Trengganu. Ambon Malay is spoken in a language-contact zone and held to be a creole. Trengganu Malay, by contrast, is spoken on the Malay Peninsula and considered an inherited Malay variety. This Malay variety, however, is also characterized by structural simplifications typically held to be characteristics of creole languages. In consequence, Trengganu Malay could well be classified as a creole Malay just like Ambon Malay {(1980: 42-53, 57-58)}. As a result of his study, {Collins} questions the basis on which Malay varieties such as Ambon Malay are classified as creole languages, while other varieties such as Trengganu are not. Arguing that the overly simplified categorization offered by creole theory does not do justice to the Austronesian languages, he comes to the following conclusion {(1980: 58–59)}:
\end{styleBodyvvafter}

\begin{styleIvI}
The term creole has no predictive strength. It is a convenient label for linguistic phenomena of a certain time and place but it does not encompass the linguistic processes which are taking place in eastern Indonesia.
\end{styleIvI}


In the context of his study on Banjarese Malay, a variety spoken in southwestern Borneo, {\citet{Wolff1988}} expresses a similar viewpoint. The author examines the question of whether Banjarese Malay represents a direct continuation of old Malay or is the result of rapid language change, such as creolization. {Wolff} concludes that there is “absolutely no proof that any of the living dialects of Indonesian/Malay are indeed creoles” {(1988: 86)}.



Another critique concerning the use of the term \textstyleChItalic{creoles} with respect to regional Malay varieties is put forward by {\citet{Steinhauer1991}} in his study on Larantuka Malay. Given that too little is known about the origins and historical developments of the eastern Malay varieties, the author argues that the label \textstyleChItalic{creole} is not very useful. Moreover, it becomes “meaningless” if it is too “broadly defined” in terms of the type of borrowing it takes for a language to be labeled a creole {(1991: 178)}.
\end{styleBodyvafter}


{\citet{Gil2001a}} also refutes the classification of the regional Malay varieties as creolized languages and {Adelaar and Prentice’s (1996)} notion of \textstyleChItalic{Pidgin Malay Derived} dialects. More specifically, he argues that {Adelaar and Prentice} do not give sufficient evidence that the original trade language was indeed a pidgin. Based on his research on Riau Indonesian, {Gil} maintains that structural simplicity in itself is not sufficient evidence to conclude that a language is a creole.
\end{styleBodyvafter}


{Paauw (2005; 2007; 2009; 2013}) also takes issue with the classification of the eastern Malay varieties as creolized languages. In his 2005 paper, {Paauw }points out that the features found in \textstyleChItalic{Pidgin Malay Derived} varieties {(Adelaar and Prentice 1996)} are also found in most of the inherited Malay varieties. Therefore, these features are better considered “markers of ‘low’ Malay, rather than contact Malay” {\citep[17]{Paauw2005}}. In another paper addressing the influence of local languages on the regional Malay varieties, {\citet{Paauw2007}} discusses some of the features which have been taken as evidence that these Malay varieties are creolized languages. He comes to the conclusion that borrowing in itself does not prove creolization. Otherwise, “it would be hard to find any language which couldn’t be considered a creole” {(2007: 3)}. In discussing the alleged pidgin origins and creolization of the eastern Malay varieties, {\citet[26]{Paauw2009}} maintains that there is not enough linguistic evidence for the claim that these are creoles. Likewise, {\citet[11]{Paauw2013}} points out that there is no linguistic evidence for the pidgin origins of the eastern Malay varieties, even though they developed under sociocultural and historical conditions which are typical for creolization. Instead, these varieties show many similarities with the inherited Malay varieties with respect to their lexicon, isolating morphology, and syntax.
\end{styleBodyvafter}


It seems that {\citet{Blust2013}} also questions the classification of the eastern Malay varieties as creoles. First, he lists the eastern Malay varieties as Malayo-Chamic languages rather than as creoles {(2013: xxvii)}. Second, in discussing pidginization and creolization among Austronesian languages, {Blust (2013: 65–66)} refers in detail to {Collins’s (1980)} study on Ambon Malay. {Blust} does not overtly state that he agrees with {Collins}. He does, however, quote {Collins’ (1980: 58-5)} above-mentioned conclusion that the label “creole has no predictive strength”, without critiquing it. This, in turn, suggests that {Blust} has a similar viewpoint on this issue.
\end{styleBodyvafter}


{Donohue’s} position about the creole/non-creole status of regional Malay varieties, including Papuan Malay, is less clear. {Donohue and \citet[68]{Smith1998}} argue that the different Malay varieties cannot be explained in terms of a single parameter such as “pure” versus “mixed or creolized”. With regard to Papuan Malay, {\citet[1]{Donohue2003}} remarks that the fact that Papuan Malay displays six of the eight features found in {Adelaar and Prentice’s (1996)} \textstyleChItalic{Pidgin Malay Derived} varieties does not prove the pidgin origins of this Malay variety. Due to areal influence these features may also have developed independently in non-pidgin or non-creole Malay varieties. In a later study on voice in Malay, {\citet{Donohue2007a}} takes a slightly different position in evaluating the contact which the Malay languages of eastern Indonesia had with non-Austronesian languages. He concludes this contact caused “some level of language assimilation” and “language adaptation”, but he does not assert that this contact had to result in creolization {(2007a: 1496)}. In another 2007 publication on voice variation in Malay, {\citet[72]{Donohue2007b}} notes that those Malay varieties spoken in areas far away from their traditional homeland show characteristics not found in the inherited Malay varieties. Moreover, in some areas these “transplanted” Malay varieties have undergone “extensive creolization”. Finally, in his 2011 study on the Melanesian influence on Papuan Malay verb and clause structure {Donohue} refers to Papuan Malay as one of the “ill-defined ‘eastern’ creoles” spoken between New Guinea and Kupang. As such, it does not represent “an Austronesian speech tradition”, with the exception of its lexicon {(2011: 433)}.
\end{styleBodyvafter}


In concluding this discussion about the creole versus non-creole status of Papuan Malay, the author agrees with those scholars who challenge the view that the eastern Malay varieties are creolized languages. Moreover, the author agrees with {Bisang (2009: 35, 43)}, who argues that complexity is not limited to the morphology or syntax of a language. Instead, complexity may instead be found in the pragmatic inferential system as applied to utterances in their discourse setting. Such “hidden complexity” is certainly a pertinent trait of Papuan Malay, as shown throughout this book. Two examples of hidden complexity are presented in (0) and (0). Due to the lack of morphosyntactic marking in Papuan Malay, a given construction can receive different readings, as illustrated in (0). Depending on the context, the \textitbf{kalo … suda} ‘when/if … already’ construction can receive a temporal or a counterfactual reading.\footnote{One anonymous reviewers suggests an alternative analysis for the example in (0). Rather than being ambiguous and exemplifying a case of “hidden complexity”, the \textitbf{kalo … suda} ‘when/if … already’ construction expresses an unspecified reason-consequence relation, with the context supplying the information on whether the reason has place.\\
\\
\\
\\
\\
\\
\\
\\
\\
\\
\\
\\
\\
\\
\\
} Example (0) illustrates the pervasive use of elision in Papuan Malay. Verbs allow but do not require core arguments. Therefore, core arguments are readily elided when they are understood from the context (“Ø” represents the omitted arguments).
\end{styleBodyvvafter}

\begin{styleExampleTitle}
Examples of hidden complexity
\end{styleExampleTitle}

\begin{tabular}{llllllllllll}
\lsptoprule
\label{bkm:Ref373925886}
\gll {\bluebold{kalo}} {\multicolumn{2}{l}{de}} {\bluebold{suda}} {\multicolumn{2}{l}{kasi}} {ana} {prempuang,} {suda} {tida} {ada}\\ %
& if & \multicolumn{2}{l}{\textsc{3sg}} & already & \multicolumn{2}{l}{give} & child & woman & already & \textsc{neg} & exist\\
& \multicolumn{2}{l}{prang} & \multicolumn{3}{l}{suku} & \multicolumn{6}{l}{lagi}\\
& \multicolumn{2}{l}{war} & \multicolumn{3}{l}{ethnic.group} & \multicolumn{6}{l}{again}\\
\lspbottomrule
\end{tabular}
\ea
\glt 
[About giving children to one’s enemy:]\\
Temporal reading: ‘\bluebold{once} she has given (her) daughter (to the other group), there will be no more ethnic war’\\
Counterfactual reading: ‘\bluebold{if} she \bluebold{had} given (her) daughter (to the other group), there would have been no more ethnic war’ \textstyleExampleSource{[081006-027-CvEx.0012]}
\z

\begin{tabular}{llllllllllllllllllllllllll}
\lsptoprule
\label{bkm:Ref373925885}
\gll {…} {\multicolumn{4}{l}{karna}} {de} {\multicolumn{2}{l}{tida}} {\multicolumn{4}{l}{bisa}} {bicara} {\multicolumn{5}{l}{bahasa,}} {\multicolumn{4}{l}{maka}} {\bluebold{Ø}} {\multicolumn{2}{l}{pake}}\\ %
&  & \multicolumn{4}{l}{because} & \textsc{3sg} & \multicolumn{2}{l}{\textsc{neg}} & \multicolumn{4}{l}{be.able} & speak & \multicolumn{5}{l}{language} & \multicolumn{4}{l}{therefore} &  & \multicolumn{2}{l}{use}\\
& \multicolumn{4}{l}{bahasa} & \multicolumn{3}{l}{orang} & \multicolumn{2}{l}{bisu,} & \multicolumn{2}{l}{…} & \multicolumn{3}{l}{baru} & \bluebold{Ø} & \bluebold{Ø} & \multicolumn{5}{l}{foto,} & \multicolumn{4}{l}{foto,}\\
& \multicolumn{4}{l}{language} & \multicolumn{3}{l}{person} & \multicolumn{2}{l}{be.mute} & \multicolumn{2}{l}{} & \multicolumn{3}{l}{and.then} &  &  & \multicolumn{5}{l}{photograph} & \multicolumn{4}{l}{photograph}\\
& \multicolumn{2}{l}{a,} & \bluebold{Ø} & \multicolumn{7}{l}{snang,} & \multicolumn{4}{l}{prempuang} & \multicolumn{3}{l}{bawa} & \multicolumn{2}{l}{babi,} & \bluebold{Ø} & \multicolumn{3}{l}{kasi} & \bluebold{Ø} & \bluebold{Ø}\\
& \multicolumn{2}{l}{ah!} &  & \multicolumn{7}{l}{feel.happy(.about)} & \multicolumn{4}{l}{woman} & \multicolumn{3}{l}{bring} & \multicolumn{2}{l}{pig} &  & \multicolumn{3}{l}{give} &  & \\
\lspbottomrule
\end{tabular}
\ea
\glt
[First outside contact between a Papuan group living in the jungle and a group of pastors:] ‘[but they can’t speak Indonesian,] because she can’t speak Indonesian, therefore (\bluebold{she}) uses sign language … (\bluebold{the pastor are taking}) pictures, pictures, ah, (\bluebold{the women} are) happy, the women bring a pig, (\bluebold{they}) give (\bluebold{it to the pastors})’ \textstyleExampleSource{[081006-023-CvEx.0073]}
\end{styleFreeTranslEngxvpt}

\subsection{Papuan Malay, a distinct language within the Malay continuum}
\label{bkm:Ref437973179}
Papuan Malay is part of the larger Malay language continuum. The Malay varieties are situated geographically in a contiguous arrangement from the Malay Peninsula (Malaysia and Singapore) in the west across Malaysia, the Sultanate of Brunei Darussalam, and the Indonesian archipelago all the way to West Papua in the east (see Figure  ‎0 .1 on p. \pageref{bkm:Ref417137560}).


This arrangement suggests a chaining pattern for the Malay continuum in which the individual Malay speech groups have contact relationships with the other Malay groups surrounding them which results in the linguistic similarity of adjoining groups. In consequence, adjacent varieties are likely to have higher degrees of inherent intelligibility than varieties that are situated at some distance to each other. That is, intelligibility decreases as the distance between the varieties along the chain increases, due to the increasing dissimilarities between the respective language systems (see {Karam 2000: 126}).
\end{styleBodyvafter}


The chaining pattern of the Malay cluster raises the question whether Papuan Malay is a distinct language or a dialect of a larger Malay language, such as Standard Indonesian which is expected to serve as a transvarietal standard {for other regional Malay varieties.} To answer this question, three factors need to be taken into account: structural similarity, inherent intelligibility, and shared ethnolinguistic identity with other Malay varieties. These are also the three criteria applied by the ISO 639-3 standard “for defining a language in relation to varieties which may be considered dialects” ({Lewis et al. 2015b; see also Hymes 1974: 123)}.
\end{styleBodyvafter}


First, structural similarity with other Malay varieties: As a Malay variety, Papuan Malay shares many structural and lexical features with other Malay varieties. At the same time, however, Papuan Malay also exhibits a considerable amount of unique phonological, morphological, syntactic, lexical, and discourse features. These structural characteristics distinguish the language from other eastern Malay varieties, such as Ambon, Manado, or North Moluccan Malay, as well as from the standard varieties of Malay, such as Standard Indonesian. (See {Anderbeck 2007: 3; Donohue 2003: 1; 2007b: 73; Paauw 2009: 20; Scott et al. 2008: 110–111}.)
\end{styleBodyvafter}


Second, inherent intelligibility with other Malay varieties: For Papuan Malay speakers with no prior contact, the mentioned structural uniqueness has direct implication for their comprehension of other Malay varieties, in that they have difficulties understanding these varieties. That is, there is only limited or no inherent intelligibility between Papuan Malay and other Malay varieties. This applies especially to Standard Indonesian and the western Malay varieties in general. (See {Anderbeck 2007: 3; Donohue 2007b: 72–73; Paauw 2009: 20; Suharno 1979: 27–28; Yembise 2011: 213–214}.)
\end{styleBodyvafter}


Third, shared ethnolinguistic identity with other Malay varieties: ethnolinguistically, Papuans typically identify with their respective indigenous vernacular languages, regardless as to whether or not they are still active speakers of that language. Beyond this local identity, they have a well-established, distinct identity as Papuans, especially vis-à-vis Indonesians from the western parts of Indonesia. This has largely to do with the ongoing Indonesian occupation (since 1963) and the negative attitudes that the Indonesian government and Indonesian institutions express toward Papuans and “Papuaness” and also toward Papuan Malay (for more details on language attitudes see §1.5.2). Vice versa, Papuan attitudes towards Indonesia and “Indonesianess” are also rather negative. (See for instance {Chauvel 2002; King 2004}.). Papuans summarize their distinct identity as follows: \textitbf{suku beda, bahasa beda, agama beda, adat beda} ‘(our) ethnicity is different, (our) language(s) is/are different, (our) religion is different, (and our) customs are different’. This statement was made to the author on numerous occasions during her stays in West Papua. This distinct ethnolinguistic identity vis-à-vis Papuan Malay is also evidenced by the names which Papuans use to refer to their language, names such as \textitbf{logat Papua} ‘Papuan speech variety’ or \textitbf{bahasa tanah} ‘home language’. Indonesian, by contrast, is always \textitbf{bahasa Indonesia} ‘Indonesian language’. These names for Papuan Malay not only indicate a strong, indigenous identification with their language. They also imply that Papuans are able to distinguish between their language and Indonesian (Scott et al. 2008: 19). (See also the discussion on language awareness in §1.5.2 ‘Language attitudes’.)
\end{styleBodyvafter}


Given its structural uniqueness, limited or non-existent inherent intelligibility, and the lack of shared ethnolinguistic identity with other Malay varieties, it is concluded here that Papuan Malay is a distinct language within the larger Malay continuum. The ISO 639-3 code for Papuan Malay is [pmy] {\citep{LewisEtAl2015a}}.
\end{styleBodyvxvafter}

\section{Dialect situation}
\label{bkm:Ref369774161}
Papuan Malay is a structurally coherent unit with slight dialectal variations across the various regions where the language is spoken.



The identification of regional varieties of Papuan Malay is complicated, however, due to its linguistic and sociolinguistic setting, as {\citet{Paauw2009}} points out. In West Papua both Papuan and Austronesian languages are spoken. “Each of these languages has its own grammatical and phonological system which can influence the Malay spoken by individuals and communities” {(2009: 75)}. Besides, “a large number of speakers of Papuan Malay are second-language speakers, and this too influences the linguistic systems of individuals and communities” {(2009: 76)}.
\end{styleBodyvafter}


To explore how many distinct varieties of Papuan Malay exist, a linguistic and sociolinguistic survey of the language was conducted in 2007 across West Papua {\citep{ScottEtAl2008}}. The survey was carried out in and around seven coastal urban areas, namely Fakfak, Jayapura, Manokwari, Merauke, Timika, Serui, and Sorong (for details see §1.9.2 and §1.9.3; see also Figure  ‎0 .2 on p. \pageref{bkm:Ref373252478}). In these locations different Papuan and Austronesian languages are spoken and second-language Papuan Malay speakers come from different linguistic backgrounds (see also §1.4).
\end{styleBodyvafter}


The survey results suggest that regional differences of Papuan Malay are minor and limited to “differences in accent, pronunciation, and perhaps some differences in vocabulary” {(Scott et al. 2008: 18)}.
\end{styleBodyvafter}


With respect to the phonology, {Scott et al. (2008: 24–44)} mention the following regional features: (1) word-final voiceless plosives seem to be present in the eastern but not in the western parts West Papua; (2) the word-final lateral seems to fluctuate freely with the flap in the eastern part of West Papua, while the word-final lateral seems to be missing in the western part; (3) nasal assimilation seems to occur in western but not in the eastern parts of West Papua; (4) vowel harmony of [ə] to a vowel in another segment possibly occurs in the western but not in the eastern parts; and (5) the glottal fricative may be missing in the urban areas of Merauke. Overall, however, these differences are minor. At most, they possibly support an Eastern and Western Papuan Malay divide with Timika “sometimes following the Western regions of Fakfak and Sorong and sometimes following the Eastern regions of Jayapura and Merauke” {(2008: 43)}. This “possible East-West divide”, however, requires further research {(2008: 44)}. Some of {Scott et al.’s (2008)} findings are modified by the current study (see Chapter 2): As for the word-final lateral, the corpus data do not show any fluctuation with the flap;\footnote{The word-final rhotic trill, however, may be devoiced if it occurs before a pause or in utterance-final position (§2.3.1.3).\\
\\
\\
\\
\\
\\
\\
\\
\\
\\
\\
\\
\\
\\
\\
} nasal assimilation does occur (§2.2.1). As far as the lexicon is concerned, regional differences also appear to be minor {(2008: 46, 96, 99)}. Regional differences with respect to the grammar were not observed.
\end{styleBodyvafter}


Overall, the data indicate that Papuan Malay as spoken across West Papua forms a structurally coherent unit despite its larger linguistic and sociolinguistic setting.
\end{styleBodyvafter}


Moreover, while speakers are able “to identify others from different regions” according to their usage of Papuan Malay, these regional variations do not impede comprehension: “Papuan Malay spoken in different regions of Papua is readily intelligible by Papuans from different regions of the province”; even children would understand Papuan Malay speakers from different regions “upon first exposure” {(Scott et al. 2008: 18)}.
\end{styleBodyvafter}


Taken together, these findings suggest that regional varieties of Papuan Malay are dialects of the same language rather than distinct, albeit closely related, languages. (See also Anderbeck 2007: 3, and the ISO 639-3 criteria for language identification in {Lewis et al. 2015a}.\footnote{The ISO 639-3 standard applies three basic criteria for defining a language in relation to varieties which may be considered dialects. The first criterion considers intelligibility between speech varieties: “Two related varieties are normally considered varieties of the same language if speakers of each variety have inherent understanding of the other variety at a functional level (that is, can understand based on knowledge of their own variety without needing to learn the other variety)” {\citep{LewisEtAl2015b}.}\\
\\
\\
\\
\\
\\
\\
\\
\\
\\
\\
\\
\\
\\
\\
})
\end{styleBodyvafter}


The proposition that Papuan Malay is a structurally coherent unit modifies {Donohue’s (2003: 1)} conclusion that “[it] is in a very real sense misleading to write about ‘Papuan Malay’ […] as if there was one unified variety of Malay spoken in the west of New Guinea”. {Donohue }suggests that there are at least four distinct Papuan Malay varieties, without, however, addressing the question “whether these different varieties of Malay constitute an entity that can be called Papuan Malay in any linguistic sense” {(2003: 1)}. Instead, {Donohue }leaves this question “for a later date” {(2003: 2)}. The most salient Papuan Malay varieties are listed below {(2003: 1–2}; see also Figure  ‎0 .2 on p. \pageref{bkm:Ref373252478}):
\end{styleBodyvvafter}

%\setcounter{itemize}{0}
\begin{itemize}
\item \begin{styleIIndented}
North Papua Malay, spoken along West Papua’s north coast between Sarmi and the Papua New Guinea border; it shows a clear influence from Manado Malay / North Moluccan Malay.
\end{styleIIndented}\item \begin{styleIIndented}
Serui Malay, spoken in Cenderawasih Bay (except for the Numfor and Biak islands); it is rather similar to North Papua Malay.
\end{styleIIndented}\item \begin{styleIIndented}
Bird’s Head Malay, spoken on the west of the Bird’s Head (in and around Sorong, Fakfak, Koiwai), is closely related to Ambon Malay; the varieties spoken on the east of the Birds’ Head (in and around Manokwari and other towns) are similar to Serui Malay.
\end{styleIIndented}\item \begin{styleIvI}
South Coast Malay, spoken in and around Merauke.
\end{styleIvI}\end{itemize}

{Donohue maintains, }as mentioned,{ that the northern Papuan Malay varieties show “}a clear influence” from Manado Malay and/or North Moluccan Malay {(2003: 2). As one example of this influence, he presents the lexical item }\textitbf{kelemarin} ‘yesterday’, which is found in North-Moluccan Malay {\citep[3]{Voorhoeve1983}}, but not in South Coast Malay. The present corpus, by contrast, does not include any \textitbf{kelemarin} tokens. Instead, all attested 153 Papuan Malay tokens for ‘yesterday’ are realized with an alveolar rhotic. Neither do {\citet{ScottEtAl2008}} make reference to the alternative realization of the word-internal rhotic as a lateral.



In summary, the findings of a linguistic and sociolinguistic language survey of different coastal regions of West Papua suggest that Papuan Malay forms a structurally coherent unit. Regional variations do occur, but they minor and the observed differences support at most dialectal divisions, such as a possible East-West divide.
\end{styleBodyvxvafter}

\section{Linguistic setting}
\label{bkm:Ref370323841}\label{bkm:Ref370323228}
West Papua is the home of 274 languages, according to {Lewis et al. }{(2015a)}. Of these, 216 are non-Austronesian, or Papuan, languages (79\%).\footnote{For a discussion of the term ‘Papuan languages’ see Footnote 24 in §1.6.2 (p. \pageref{bkm:Ref389055489}).\\
\\
\\
\\
\\
\\
\\
\\
\\
\\
\\
\\
\\
\\
\\
} The remaining 58 languages are Austronesian (21\%).\footnote{The \textstyleChItalic{Ethnologue} {\citep{LewisEtAl2015a}} lists Papuan Malay as a Malay-based creole, while here it is counted among the Austronesian languages (see also §1.2.2). A listing of West Papua’s languages is available at \url{http://www.ethnologue.com/country/id/languages} and \url{http://www.ethnologue.com/map/ID_pe_} (accessed 8 January 2016).\\
\\
\\
\\
\\
\\
\\
\\
\\
\\
\\
\\
\\
\\
\\
}



In the Sarmi regency, where most of the research for this description of Papuan Malay was conducted, both Papuan and Austronesian languages are found, as shown in Figure  ‎0 .4 (p. \pageref{bkm:Ref375722237}). Between Bonggo in the east and the Mamberamo River in the west, 23 Papuan languages are spoken. Most of these languages belong to the Tor-Kwerba language family (21 languages). One of them is Isirawa, the language of the author’s host family. The other twenty Papuan languages are Airoran, Bagusa, Beneraf, Berik, Betaf, Dabe, Dineor, Itik, Jofotek-Bromnya, Kauwera, Keijar, Kwerba, Kwerba Mamberamo, Kwesten, Kwinsu, Mander, Mawes, Samarokena, Trimuris, and Wares. The remaining two languages are Yoke which is a Lower Mamberamo language, and the isolate Massep. In addition, eleven Austronesian languages are spoken in the Sarmi regency. All eleven languages belong to the Sarmi branch of the Sarmi-Jayapura Bay subgroup, namely Anus, Bonggo, Fedan, Kaptiau, Liki, Masimasi, Mo, Sobei, Sunum, Tarpia, and Yarsun. While all of these languages are listed in the \textstyleChItalic{Ethnologue} {\citep{LewisEtAl2015a}}, three of them are not included in Figure  ‎0 .4 (p. \pageref{bkm:Ref375722237}), namely Jofotek-Bromnya and Kaptiau, both of which are spoken in the area around Bonggo, and Kwinsu which is spoken in the area east of Sarmi.
\end{styleBodyvafter}


Of the 23 Papuan languages, one is “developing” (Kwerba) and five are “vigorous” (see Table  ‎1 .1). The remaining languages are “threatened” (7 languages), “shifting” to Papuan Malay (7 languages), “moribund” (1 language), or “nearly extinct” (2 languages). One of the threatened languages is Isirawa, the language of the author’s host family.\footnote{The \textstyleChItalic{Ethnologue} {\citep{LewisEtAl2015a}} gives the following definitions for the status of these languages: 5 (Developing) – The language is in vigorous use, with literature in a standardized form being used by some though this is not yet widespread or sustainable; 6a (Vigorous) – The language is used for face-to-face communication by all generations and the situation is sustainable; 6b (Threatened) – The language is used for face-to-face communication within all generations, but it is losing users; 7 (Shifting) – The child-bearing generation can use the language among themselves, but it is not being transmitted to children; 8a (Moribund) – The only remaining active users of the language are members of the grandparent generation and older; 8b (Nearly Extinct) – The only remaining users of the language are members of the grandparent generation or older who have little opportunity to use the language. For details see \url{http://www.ethnologue.com/about/language-status} (accessed 8 January 2016).\\
\\
\\
\\
\\
\\
\\
\\
\\
\\
\\
\\
\\
\\
\\
}
\end{styleBodyvafter}


Most of the 23 Papuan languages are spoken by populations of 500 or less (16 languages), and another three have between 600 and 1,000 speakers. Only three have larger populations of between 1,800 and 2,500 speakers. One of them is the “developing” language Kwerba.
\end{styleBodyvvafter}

\begin{stylecaption}
\label{bkm:Ref374442095}Table ‎1.\stepcounter{Table}{\theTable}:  Papuan languages in the Sarmi regency: Status and populations
\end{stylecaption}

\tablehead{
\multicolumn{2}{l}{ Name \& ISO 639-3 code} & Status & \arraybslash Population\\
}
\begin{tabular}{llll}
\lsptoprule
Aironan & [air] & 6a (Vigorous) & \raggedleft\arraybslash 1,000\\
Bagusa & [bqb] & 6a Vigorous & \raggedleft\arraybslash 600\\
Beneraf & [bnv] & 7 (Shifting) & \raggedleft\arraybslash 200\\
Berik & [bkl] & 7 (Shifting) & \raggedleft\arraybslash 200\\
Betaf & [bfe] & 6b (Threatened) & \raggedleft\arraybslash 600\\
Dabe & [dbe] & 7 (Shifting) & \raggedleft\arraybslash 440\\
Dineor & [mrx] & 8a (Moribund) & \raggedleft\arraybslash 55\\
Isirawa & [srl] & 6b (Threatened) & \raggedleft\arraybslash 1,800\\
Itik & [itx] & 6b (Threatened) & \raggedleft\arraybslash 80\\
Jofotek-Bromnya & [jbr] & 6b (Threatened) & \raggedleft\arraybslash 200\\
Kauwera & [xau] & 6a (Vigorous) & \raggedleft\arraybslash 400\\
Keijar & [kdy] & 7 (Shifting) & \raggedleft\arraybslash 370\\
Kwerba & [kwe] & 5 (Developing) & \raggedleft\arraybslash 2,500\\
Kwerba Mamberamo & [xwr] & 6a (Vigorous) & \raggedleft\arraybslash 300\\
Kwesten & [kwt] & 7 (Shifting) & \raggedleft\arraybslash 2,000\\
Kwinsu & [kuc] & 7 (Shifting) & \raggedleft\arraybslash 500\\
Mander & [mqr] & 8b (Nearly extinct) & \raggedleft\arraybslash 20\\
Massep & [mvs] & 8b (Nearly extinct) & \raggedleft\arraybslash 25\\
Mawes & [mgk] & 6b (Threatened) & \raggedleft\arraybslash 850\\
Samarokena & [tmj] & 6b (Threatened) & \raggedleft\arraybslash 400\\
Trimuris & [tip] & 6a (Vigorous) & \raggedleft\arraybslash 300\\
Wares & [wai] & 7 (Shifting) & \raggedleft\arraybslash 200\\
Yoke & [yki] & 6b (Threatened) & \raggedleft\arraybslash 200\\
\lspbottomrule
\end{tabular}

Three of the 23 Papuan languages have been researched to some extent, namely “shifting” Berik, “threatened” Isirawa, and “developing” Kwerba. The resources on these languages include word lists, descriptions of selected grammatical topics, issues related to literacy in these languages, anthropological studies, and materials written in these languages. Isirawa especially has a quite substantial corpus of resources, including the New Testament of the Bible. Moreover, the language has seen a five-year literacy program. In spite of these language development efforts, the language is losing its users. In four languages, a sociolinguistic study was carried out in 1998 {\citep{ClouseEtAl2002}}, namely in Aironan, Massep, Samarokena, and Yoke. Limited lexical resources are also available in Samarokena and Yoke, as well as in another eight languages (Beneraf, Dabe, Dineor, Itik, Kauwera, Kwesten, Mander, and Mawes). For the remaining eight languages no resources are available except for their listing in the \textstyleChItalic{Ethnologue} {\citep{LewisEtAl2015a}} and \textstyleChItalic{Glottolog} {\citep{NordhoffEtAl2013}}: Bagusa, Betaf, Jofotek-Bromnya, Keijar, Kwerba Mamberamo, Kwinsu, Trimuris, and Wares. (For more details see Appendix C.)\footnote{The \textstyleChItalic{Ethnologue} {\citep{LewisEtAl2015a}} provides basic information about these languages including their linguistic classification, alternate names, dialects, their status in terms of their overall development, population totals, and location. The \textstyleChItalic{Ethnologue} is available at \url{http://www.ethnologue.com} (accessed 8 January 2016). \textstyleChItalic{Glottolog} {\citep{NordhoffEtAl2013}} is an online resource that provides a comprehensive catalogue of the world’s languages, language families and dialects. \textstyleChItalic{Glottolog} is available at \url{http://glottolog.org/} (accessed 8 January 2016).\\
\\
\\
\\
\\
\\
\\
\\
\\
\\
\\
\\
\\
\\
\\
}
\end{styleBodyaftervbefore}


Of the eleven Austronesian languages, one is threatened, four are “shifting” to Papuan Malay, five are “moribund”, and one is “nearly extinct” (see Table  ‎1 .2). Most of these languages have less than 650 speakers. The exception is Sobei with a population of 1,850 speakers. Sobei is also the only Austronesian language that has been researched to some extent. The resources on Sobei include word lists, descriptions of some of its grammatical features, anthropological studies, and one lexical resource. In another four languages limited lexical resources are available. For the remaining six languages no resources are available, except for their listing in the \textstyleChItalic{Ethnologue} {\citep{LewisEtAl2015a}} and \textstyleChItalic{Glottolog} {\citep{NordhoffEtAl2013}}: Fedan, Kaptiau, Liki, Masimasi, Sunum, and Yarsun. (For more details see Appendix C.)
\end{styleBodyvvafter}

\begin{stylecaption}
\label{bkm:Ref374442097}Table ‎1.\stepcounter{Table}{\theTable}:  Austronesian languages in the Sarmi regency: Status and populations
\end{stylecaption}

\tablehead{
\multicolumn{2}{l}{ Name \& ISO 639-3 code} & Status & \arraybslash Population\\
}
\begin{tabular}{llll}
\lsptoprule
Anus & [auq] & 7 (Shifting) & \raggedleft\arraybslash 320\\
Bonggo & [bpg] & 8a (Moribund) & \raggedleft\arraybslash 320\\
Fedan & [pdn] & 8a (Moribund) & \raggedleft\arraybslash 280\\
Kaptiau & [kbi] & 7 (Shifting) & \raggedleft\arraybslash 230\\
Liki & [lio] & 8a (Moribund) & \raggedleft\arraybslash 11\\
Masimasi & [ism] & 8b (nearly extinct) & \raggedleft\arraybslash 10\\
Mo & [wkd] & 7 (Shifting) & \raggedleft\arraybslash 550\\
Sobei & [sob] & 7 (Shifting) & \raggedleft\arraybslash 1,850\\
Sunum & [ynm] & 6b (Threatened) & \raggedleft\arraybslash 560\\
Tarpia & [tpf] & 8a (Moribund) & \raggedleft\arraybslash 630\\
Yarsun & [yrs] & 8a (Moribund) & \raggedleft\arraybslash 200\\
\lspbottomrule
\end{tabular}
\section{Sociolinguistic profile}
\label{bkm:Ref370323844}
This section discusses the sociolinguistic profile of Papuan Malay. In summary, this profile presents itself as follows:


\begin{itemize}
\item \begin{styleIIndented}
Strong and increasing language vitality of Papuan Malay;
\end{styleIIndented}\item \begin{styleIIndented}
Substantial language contact between Papuan Malay and Indonesian;
\end{styleIIndented}\item \begin{styleIIndented}
Functional distribution of Papuan Malay as the \textsc{low} variety, and Indonesian as the \textsc{high} variety, in terms of {Ferguson’s (1972)} notion of diglossia;
\end{styleIIndented}\item \begin{styleIIndented}
Positive to somewhat ambivalent language attitudes toward Papuan Malay; and
\end{styleIIndented}\end{itemize}
\begin{itemize}
\item \begin{styleIvI}
Lack of language awareness of many Papuan Malay speakers about the status of Papuan Malay as a language distinct from Indonesian.
\end{styleIvI}\end{itemize}

Papuan Malay is spoken in a rich linguistic and sociolinguistic environment, which includes indigenous Papuan and Austronesian languages, as well as Indonesian and other languages spoken by migrants who have come to live and work in West Papua (see §1.4 and §1.7.1). As in other areas of New Guinea, many Papuans living in the coastal areas of West Papua speak two or more languages ({Foley 1986: 15–47}; see also {Mühlhäusler 1996}). The linguistic repertoire of individual speakers may include one or more local Papuan and/or Austronesian vernaculars, Papuan Malay, and – depending on the speaker’s education levels – Indonesian, and also English, all of which are being used as deemed necessary and appropriate.



Many of the indigenous Papuan and Austronesian languages are threatened by extinction. By contrast, the vitality of Papuan Malay is strong and increasing. This applies especially to urban coastal communities where Papuan Malay serves as a language of wider communication between members of different ethnic groups {(Scott et al. 2008: 10–18)}. In the Sarmi regency, for instance, many vernacular languages are shifting, or have shifted, to Papuan Malay (see §1.4).
\end{styleBodyvafter}


There is also substantial language contact between Papuan Malay and Indonesian.
\end{styleBodyvafter}


The co-existence and interaction of indigenous vernacular languages, Papuan Malay, and Indonesian with their varying and overlapping roles creates a triglossic situation. More investigation is needed, however, to determine whether the interplay between all three best be explained in terms of {Fasold’s (1984: 44–50)} notion of \textstyleChItalic{double overlapping diglossia} or whether their functional distribution represents an instance of \textstyleChItalic{linear polyglossia}. For the present discussion, however, the status of the indigenous vernaculars vis-à-vis Papuan Malay and Indonesian is not further taken into consideration. Instead, the remainder of this section focuses on the interplay of Papuan Malay and Indonesian.
\end{styleBodyvafter}


Both languages are in a diglossic distribution. In this situation, according to {Ferguson’s (1972)} notion of diglossia, Indonesian serves as \textsc{h}, the \textsc{high} variety which is acquired through formal education, and Papuan Malay as \textsc{l}, the \textsc{low} variety, which is acquired in informal domains, including the home domain.
\end{styleBodyvafter}


Papuan Malay speakers display the typical language behavior of \textsc{low} speakers in their language use patterns as well as with respect to their language attitudes. Language use and the diglossic distribution of Papuan Malay and Indonesian are discussed in §1.5.1, and language attitudes, together with language awareness, in §1.5.2.
\end{styleBodyvxvafter}

\subsection{Language use}
\label{bkm:Ref373219004}
The diglossic, or functional, distribution of Indonesian as the \textsc{high} variety and Papuan Malay as the \textsc{low} variety implies that in certain situations Indonesian is more appropriate while in other situations Papuan Malay is more appropriate.



In terms of {Fishman’s (1965: 86)} “domains of language choice”, three factors influence such language choices: the topics discussed, the relationships between the interlocutors, and the locations where the communication takes place. Another factor to be taken into account is speaker education levels, given that Indonesian is acquired through formal education. Below the four factors are discussed in more detail.\footnote{Not further taken into account here is the growing influence of the mass media, namely TV, even in more remote areas which exposes Papuans more and more to colloquial varieties of Indonesian, especially Jakartan Indonesian (see also {Sneddon 2006}).\\
\\
\\
\\
\\
\\
\\
\\
\\
\\
\\
\\
\\
\\
\\
}
\end{styleBodyvxafter}

%\setcounter{itemize}{0}
\begin{itemize}
\item \begin{styleOvNvwnext}
\label{bkm:Ref376432567}Speaker education levels
\end{styleOvNvwnext}\end{itemize}

In diglossic situations, the \textsc{low} variety is known by everyone while the \textsc{high} variety is acquired through formal education {\citep{Ferguson1972}}. This also applies to the diglossic distribution of Papuan Malay and Indonesian. While Papuan Malay is known by almost everyone in West Papua’s coastal areas, knowledge of Indonesian depends on speakers’ education levels.



The results of the mentioned 2007 survey {(Scott et al. 2008: 14–17)} show that bilingualism/multilingualism is “a common feature of the Papuan linguistic landscape”. The report does not, however, give details about the degree to which Papuans are bilingual in Indonesian, but notes that bilingualism levels remain uncertain.
\end{styleBodyvafter}


During her 3-month fieldwork in Sarmi (see §1.11.3), the author did not investigate bilingualism in Indonesian. She did, however, note changes in speakers’ language behavior depending on their education levels. Papuan Malay speakers with higher education levels displayed a general and marked tendency to “dress up” their Papuan Malay with Indonesian features. This tendency was even more pronounced when discussing high topics (see Factor 2 ‘Topical regulation‘), or when interacting with group outsiders (see Factor 3 ‘Relationships between interlocutors‘). The observed features include lexical choices. Such choices are made between lexical items that are distinct in both languages, for example Indonesian \textitbf{desa} ‘village’ or \textitbf{mereka} ‘\textsc{3pl}’ instead of Papuan Malay \textitbf{kampung} ‘village’ and \textitbf{dorang}/\textitbf{dong} ‘\textsc{3pl}’, respectively. Lexical choices are also made between lexical items that are rather similar in both languages, such as Indonesian \textitbf{adik} [\textstyleChCharisSIL{ˈa.dɪk̚}] ‘younger sibling’ or \textitbf{tidak} [\textstyleChCharisSIL{ˈti.dɐk̚}] ‘\textsc{neg}’, instead of Papuan Malay \textitbf{ade} [\textstyleChCharisSIL{ˈa.dɛ}] ‘younger sibling’ and \textitbf{tida} [\textstyleChCharisSIL{ˈti.da}] ‘\textsc{neg}’, respectively. Other features are syntactic ones, such as Indonesian causatives formed with suffix \textitbf{\-kan} ‘\textsc{caus}’, passives formed with prefix \textitbf{di\-} ‘\textsc{uv}’, or possessives formed with suffix \textscItalBold{\-}\textitbf{nya} ‘\textsc{3possr}’.\footnote{For detailed grammatical descriptions of Indonesian see for instance {\citet{Mintz1994}} and {\citet{Sneddon2010}}.\\
\\
\\
\\
\\
\\
\\
\\
\\
\\
\\
\\
\\
\\
\\
} Less-educated speakers, by contrast, did not display this general tendency of mixing and switching to Indonesian given their more limited exposure to the \textsc{high} variety Indonesian. They only showed this tendency to “dress-up” their Papuan Malay with Indonesian features or lexical items when discussing \textsc{high} topics (see Factor 2 ‘Topical regulation‘), or when interacting with fellow-Papuans of higher social standing or with group outsiders (see Factor 3 ‘Relationships between interlocutors‘).


\begin{itemize}
\item \begin{styleOvNvwnext}
\label{bkm:Ref376429016}Topical regulation
\end{styleOvNvwnext}\end{itemize}

As {\citet[71]{Fishman1965}} points out, “certain topics are somehow handled better in one language than in another”. The results of the 2007 survey provide only limited information about this issue, however. The findings only state that Papuan Malay is the preferred language for humor and that politics are typically discussed in the indigenous vernaculars {(Scott et al. 2008: 17)}. The author’s own observations during her 3-month fieldwork in late 2008 modify these findings (see §1.11.3). The observed Papuan Malay speakers displayed a notable tendency to change their language behavior when discussing \textsc{high} topics. That is, when talking about topics associated with the formal domains of government, politics, education, or religion they tended to “dress up” their Papuan Malay and make it more Indonesian-like.
\end{styleBodyxafter}

\begin{itemize}
\item \begin{styleOvNvwnext}
\label{bkm:Ref376429590}Relationships between interlocutors
\end{styleOvNvwnext}\end{itemize}

Language behavior is not only influenced by the topics of communication and speaker education levels, but also by role relations. That is, individual speakers display certain language behaviors depending on the role relations between them {\citep[76]{Fishman1965}}.



As for Papuan Malay, the 2007 survey results {(Scott et al. 2008: 13, 14)} indicate that family members and friends typically communicate in Papuan Malay or in the vernacular, but not in Indonesian. The same applies to informal interactions between customers and vendors, or between patients and local health workers. Teachers may also address their students in Papuan Malay in informal interactions (in informal interactions in primary school, students may even address their teachers in Papuan Malay). The report does not discuss which language(s) Papuans use when they interact with fellow-Papuans of higher social standing or with outsiders.
\end{styleBodyvafter}


During her 3-month fieldwork in Sarmi (see §1.11.3), however, the author did note changes in speakers’ language behavior depending on the role relations between interlocutors in terms of their status and community membership.
\end{styleBodyvafter}


In interactions with fellow-Papuans of equally low status, less-educated Papuans typically used the \textsc{low} variety Papuan Malay. (At times, they also switched to Isirawa, the vernacular language for most of them.) By contrast, when interacting with fellow-Papuans of higher social standing, such as teachers, mayors and other government officials, and pastors, or when conversing with group outsiders, that is non-Papuans, the observed speakers showed a marked tendency to change their language behavior. That is, in such interactions, their speech showed influences from the \textsc{high} variety Indonesian, similar to the general language behavior of better-educated speakers, described under Factor 1 ‘Speaker education levels’. As for the language behavior of better-educated speakers, their general tendency to “dress-up” their Papuan Malay with Indonesian features was even more marked when they interacted with group outsiders, such as the author. This tendency to “dress-up” one’s Papuan Malay with Indonesian features reflects role relations, in that the use of Papuan Malay indicates intimacy, informality, and equality, while the use of Indonesian features signals social inequality and distance, as well as formality (see also {Fishman 1965: 70}).\footnote{All observed Papuans of higher social standing were also better educated, whereas none of the observed less-educated Papuans was of high social standing.\\
\\
\\
\\
\\
\\
\\
\\
\\
\\
\\
\\
\\
\\
\\
}
\end{styleBodyvxafter}

\begin{itemize}
\item \begin{styleOvNvwnext}
Locations
\end{styleOvNvwnext}\end{itemize}

Language behaviors are also influenced by the locations where communication takes place, in that speakers consider certain languages to be more appropriate in certain settings {(Fishman 1965: 71, 75)}. This also applies to Papuan Malay. In certain domains, Papuan Malay speakers consider Indonesian to be more appropriate than Papuan Malay due to the diglossic distribution of both languages {(Scott et al. 2008: 11–18)}. That is, Indonesian is the preferred language for formal interactions in the education and religious domains (such as formal instruction, leadership, or preaching) or other public domains such as government offices. Papuan Malay strongly dominates all other domains. In addition, it is also the preferred language for informal interactions in public domains such as schools, churches, and government offices.
\end{styleBodyxvafter}

\subsection{Language attitudes}
\label{bkm:Ref373219005}
{Fishman’s (1965: 70)} considerations of intimacy and distance, informality and formality also apply to Papuan Malay.



The findings of the 2007 survey indicate that Papuans associate Papuan Malay with intimacy and informality, while they associate Indonesian with social distance and formality {\citep{ScottEtAl2008}}. The names which the interviewees used to refer to Papuan Malay reflect these positive feelings toward their language: \textitbf{bahasa tanah} ‘home language’, \textitbf{bahasa santay} ‘language to relax’, \textitbf{bahasa sehari-hari} ‘every-day language’, or \textitbf{bahasa pasar} ‘market/trade language’. Especially the name \textitbf{bahasa tanah} ‘home language’ suggests “a strong, indigenous identification with this speech form” {(2008: 18)}. Most interviewees also stated that they are interested in the development of Papuan Malay. Moreover, the majority of interviewees stated that Papuan Malay and Indonesian are of equal value and that Indonesian speakers do not deserve more respect than Papuan Malay speakers. Given these findings, the researchers come to the conclusion that among the interviewed Papuans attitudes toward Papuan Malay are “remarkably positive” {(2008: 18–22)}.
\end{styleBodyvafter}


The expressed attitude that Papuan Malay and Indonesian are of equal value is remarkable, given that in diglossic communities speakers usually consider the \textsc{high} variety to be superior. The \textsc{low} variety, by contrast, is usually held “to be inferior, even to the point that its existence is denied” {\citep[36]{Fasold1984}}.
\end{styleBodyvafter}


The author’s own observations agree with the survey findings that Papuans find Papuan Malay suitable for intimate communication, while they feel at a distance with Indonesian. Many Papuan Malay speakers she met referred to their speech variety as \textitbf{logat Papua} ‘Papuan speech variety’, a name that like \textitbf{bahasa tanah} ‘home language’ indicates a strong, indigenous identification with their language.
\end{styleBodyvafter}


At the same time, though, it is questioned here to what extent Papuans feel at ease with Papuan Malay and how positive their attitudes really are. While most of the 2007 interviewees said that Papuan Malay and Indonesian are of equal value, the same interviewees also stated that Indonesian was more appropriate in certain domains. Besides, the author’s own observations suggest that Papuans also consider Indonesian to be more appropriate for certain topics and with certain interlocutors. These language behaviors suggest that language attitudes toward Papuan Malay are somewhat ambivalent as far as formal domains are concerned.
\end{styleBodyvafter}


A “low level of correlation between attitudes and actual behavior” is not unusual, though, as scholars such {Agheyisi and \citet[140]{Fishman1970}} point out (see also {Cooper and Fishman 1974: 10;} {Baker 1992: 16}). As for Papuan Malay, the observed mismatch can perhaps be accounted for in terms of {Kelman’s (1971)} distinction of sentimental and instrumental attachments. Applying this distinction, one can say that Papuans are “sentimentally attached” to Papuan Malay but “instrumentally attached” to Indonesian. Papuan Malay is associated with sentimental attachments, in that it makes Papuans feel good about being Papuan. Indonesian, by contrast, is associated with instrumental attachments in that it allows them to achieve social status and their education and to get things done. {(1971: 25)}
\end{styleBodyvafter}


In this context, the attitudes which Indonesians and Indonesian institutions express toward Papuan Malay are also important. Overall, it seems that Indonesians who live in West Papua but do not speak Papuan Malay consider the language to be poor or bad Indonesian {(Scott et al. 2008: 19)}. In West Papua, this view is implicitly communicated by Indonesian government institutions, for instance by hanging banners across major roads which demand \textstyleChItalic{mari kita berbicara bahasa Indonesia yang baik dan benar} ‘let us speak good and correct Indonesian’. Such negative language attitudes are widespread and at times rather demeaning. Moreover, they are not only directed towards the language but also towards its speakers. {\citet[94]{King2002}}, for instance, reports that Indonesians in Papua consider Papuans to be stupid and backwards: “‘Papua bodoh’ – stupid Papuans; backward Papuans”. Moreover, negative attitudes towards Papuan Malay, and the eastern Malay varieties in general, are also found among Indonesian academics. {Masinambow and \citet[106]{Haenen2002}, for example,} report that scholars in Indonesia continue to regard the eastern Malay varieties as second-class, mixed languages which are opposed by the pure High Malay language.\footnote{{Masinambow and \citet{Haenen2002}} uses “High Malay” as a cover term which also includes Standard Indonesian.\\
\\
\\
\\
\\
\\
\\
\\
\\
\\
\\
\\
\\
\\
\\
} (For a discussion of Indonesian language planning see {Sneddon 2003: 114–143;} for a discussion of the role of Papuan Malay in the context of Indonesian language politics see {Besier 2012: 13–17}.)
\end{styleBodyvafter}


Hence, the need for Papuans to distinguish between sentimental and instrumental attitudes is confounded by the negative attitudes which Indonesian institutions and individuals have toward Papuan Malay.
\end{styleBodyvafter}


Notably, Papuan Malay is not recognized by the Papuan independence movement OPM (\textstyleChItalic{Organisasi Papua Merdeka} – ‘Free Papua Movement’) either.
\end{styleBodyvafter}


The First Papuan People’s Congress, held on 16-19 October 1961, issued a manifesto which declared that \textstyleChItalic{Papua Barat} ‘West Papua’ would be the name of their nation, \textstyleChItalic{Papua} the name of the people, \textstyleChItalic{Hai Tanahku Papua} ‘My land Papua’ the national anthem, the \textstyleChItalic{Bintang Kejora} ‘Morning Star’ the national flag, the \textstyleChItalic{burung Mambruk} ‘Mambruk bird’ the national symbol, and \textstyleChItalic{Satu Rakyat dan Satu Jiwa} ‘One People One Soul’ the national motto. Moreover, the Congress decided that the national language should not be Malay, as it was the colonizer’s language {(Alua 2006: 40–43)}. The Second Papuan People’s Congress, held from 29 May until 4 June 2000 at Cenderawasih University in Jayapura, reconfirmed the national anthem, flag, and symbol, and again rejected Papuan Malay as the national language. Instead the Congress decided that English should be the official language. In addition, Papuan Malay and Tok Pisin should serve as “common” languages {(2004: 50)}.\footnote{{King’s (2004)} report is based on an \textstyleChItalic{Agence France Presse} summary, dated 6 January 2000, which is titled “The constitution of the ‘State of Papua’ as envisaged in Jayapura”.\\
\\
\\
\\
\\
\\
\\
\\
\\
\\
\\
\\
\\
\\
\\
} Likewise, the Third Papuan People’s Congress, held from 17-19 October 2011 in Abepura, rejected Papuan Malay as the national language {\citep[19]{Besier2012}}.
\end{styleBodyvafter}


This desire of Papuan nationals “of a clean linguistic break” is an utopian dream, however, as {\citet[407]{Rutherford2005}} points out. Moreover, it presents a dilemma since only few people in West Papua speak these other languages, whereas Papuan Malay is the de facto language of wider communication. (See also {Besier 2012: 17–22}.)
\end{styleBodyvafter}


The fact that Papuan Malay has not been officially recognized in spite of its large numbers of speakers reflects the lack of esteem held by the main stakeholders vis-à-vis this language, by the Indonesian or OPM stakeholders. (See also {Besier 2012: 32}.)
\end{styleBodyvafter}


Another factor to be considered in the context of language attitudes is the issue of language awareness.
\end{styleBodyvafter}


The findings of the 2007 sociolinguistic survey indicate a potential lack of language awareness. Papuan interviewees stated that “lesser educated [… Papuan Malay] speakers would likely be unaware of the differences” between their language and Indonesian; that is, “they would consider the speech form they use to be coincident with standard Indonesian” {(Scott et al. 2008: 11)}. Along similar lines, {\citet[76]{Paauw2009}} reports that many Papuan Malay speakers are not aware of the fact that their speech variety is distinct from Indonesian. (See also {Burung 2008a: 5–7}.)
\end{styleBodyvafter}


The author made similar observations during her 2008 fieldwork in Sarmi. Many Papuan Malay speakers she met thought that they were speaking Indonesian with a local Papuan flavor when conversing with other Papuans.
\end{styleBodyvafter}


This lack of language awareness is not surprising, however, given the negative language attitudes that Papuans experience from the Indonesian government and Indonesian institutions which sanction Indonesian as the only acceptable variety of Malay. Through this “ideological erasure” of Papuan Malay from official quarters, the language has become “invisible”, using {Gal and Irvine’s (1995: 974)} terminology. This erasure has led to the perception among many Papuans that Papuan Malay did not exist as a distinct language. (See also {Errington 2001: 30.)}
\end{styleBodyvafter}


In summarizing this discussion on language attitudes, it is concluded that overall Papuans’ attitudes toward Papuan Malay are positive to somewhat ambivalent, rather than wholly positive.
\end{styleBodyvxvafter}

\section{Typological profile of Papuan Malay}
\label{bkm:Ref438029386}
This section presents an overview of the typological profile of Papuan Malay as described in this book. General typological features of the language are discussed in §1.6.1, followed in §1.6.2 by a comparison of some of its features with those found in Austronesian and in Papuan languages. In §1.6.3, some features of Papuan Malay are compared to those found in other eastern Malay varieties.
\end{styleBodyxvafter}

\subsection{General typological profile}
\label{bkm:Ref369187828}
In presenting the pertinent typological features of Papuan Malay, an overview of its phonology is given in §1.6.1.1, its morphology in §1.6.1.2, its word classes in §1.6.1.3, and its basic word order in §1.6.1.4.
\end{styleBodyxvafter}

\paragraph[Phonology]{Phonology}
\label{bkm:Ref369507792}
Papuan Malay has 18 consonant and five vowel phonemes. The consonant system consists of the following phonemes: /\textstyleChCharisSIL{p}, \textstyleChCharisSIL{b}, \textstyleChCharisSIL{t}, \textstyleChCharisSIL{d}, \textstyleChCharisSIL{g}, \textstyleChCharisSIL{k}, \textstyleChCharisSIL{tʃ}, \textstyleChCharisSIL{dʒ}, \textstyleChCharisSIL{s}, \textstyleChCharisSIL{h}, \textstyleChCharisSIL{m}, \textstyleChCharisSIL{n}, \textstyleChCharisSIL{ɲ}, \textstyleChCharisSIL{ŋ}, \textstyleChCharisSIL{r}, \textstyleChCharisSIL{l}, \textstyleChCharisSIL{j}, \textstyleChCharisSIL{w}/. All consonants occur as onsets,\footnote{Velar /\textstyleChCharisSILviiivpt{ŋ}/ however, only occurs in the root-internal and not in the word-initial onset position.\\
\\
\\
\\
\\
\\
\\
\\
\\
\\
\\
\\
\\
\\
\\
} while the range of consonants occurring in the coda position is much smaller. The five vowels are /\textstyleChCharisSIL{i}, \textstyleChCharisSIL{ɛ}, \textstyleChCharisSIL{u}, \textstyleChCharisSIL{ɔ}, \textstyleChCharisSIL{a}/. All five occur in stressed and unstressed, open and closed syllables. A restricted sample of like segments can occur in sequences. Papuan Malay shows a clear preference for disyllabic roots and for CV and CVC syllables; the maximal syllable is CCVC. Stress typically falls on the penultimate syllable. Adding to its 18 native consonant system, Papuan Malay has adopted one loan segment, the voiceless labio-dental fricative /\textstyleChCharisSIL{f}/. (Chapter 2)


\paragraph[Morphology]{Morphology}
\label{bkm:Ref369507793}
Papuan Malay is a language near the isolating end of the analytic-synthetic continuum. That is, the language has very little productive morphology and words are typically single root morphemes. Inflectional morphology is lacking, as nouns and verbs are not marked for any grammatical category such as gender, number, or case. Word formation is limited to the two derivational processes of reduplication and affixation.



Reduplication is a very productive process. Three types of lexeme formation are attested, namely full reduplication, which is the most common one, partial and imitative reduplication. Usually, content words undergo reduplication; reduplication of function words is rare. The overall meaning of reduplication is “a \textsc{higher}/\textsc{lower} \textsc{degree} \textsc{of} …”, employing {Kiyomi’s (2009: 1151)} terminology. (Chapter 4)
\end{styleBodyvafter}


Affixation has very limited productivity. Papuan Malay has two affixes which are somewhat productive. Verbal prefix \textscItalBold{ter\-} ‘\textsc{acl}’ derives monovalent verbs from mono- or bivalent bases. The derived verbs denote accidental or unintentional actions or events. Nominal suffix \textitbf{\-ang} ‘\textsc{pat}’ typically derives nominals from verbal bases. The derived nouns denote the patient or result of the event or state specified by the verbal base. In addition, Papuan Malay has one nominal prefix, \textscItalBold{pe(n)\-} ‘\textsc{ag}’, which is, at best, marginally productive. The derived nouns denote the agent or instrument of the event or state specified by the verbal base.\footnote{The small caps designate the abstract representation of affixes that have more than one form of realization; prefixes \textscItalBold{ter\-} and \textscItalBold{pe(n)\-} have two allomorphs each, namely \textitbf{ter}\textitbf{\-} and \textitbf{ta}\textitbf{\-} (§3.1.2.1), and \textitbf{pe(}\textscItalBold{n}\textitbf{)}\- and \textitbf{pa(}\textscItalBold{n}\textitbf{)}\- (small-caps \textscItalBold{n} represents the different realizations of the nasal) (§3.1.4.1), respectively.\\
\\
\\
\\
\\
\\
\\
\\
\\
\\
\\
\\
\\
\\
\\
} (§3.1, in Chapter 3)
\end{styleBodyvafter}


Compounding is a third word-formation process. Its degree of productivity remains uncertain, though, as the demarcation between compounds and phrasal expressions is unclear. (§3.2, in Chapter 3)
\end{styleBodyvafter}


Papuan Malay has no morphologically marked passive voice. Instead, speakers prefer to encode actions and events in active constructions. An initial survey of the corpus shows that speakers can use an analytical construction to signal that the undergoer is adversely affected. This construction is formed with bivalent \textitbf{dapat} ‘get’ or \textitbf{kena} ‘hit’, as in \textitbf{dapat pukul} ‘get hit’ or \textitbf{kena hujang} ‘hit (by) rain’.\footnote{In this book, Papuan Malay strategies to express passive voice are not further discussed; instead, this topic is left for future research.\\
\\
\\
\\
\\
\\
\\
\\
\\
\\
\\
\\
\\
\\
\\
}
\end{styleBodyvxvafter}

\paragraph[Word classes]{Word classes}
\label{bkm:Ref438297867}
The open word classes in Papuan Malay are nouns, verbs, and adverbs. The major closed word classes are personal pronouns, interrogatives, demonstratives, locatives, numerals, quantifiers, prepositions, and conjunctions. The distinguishing criteria for these classes are their syntactic properties, given the lack of inflectional morphology and the limited productivity of derivational patterns. A number of categories display membership overlap, most of which involves verbs. This includes overlap between verbs and nouns as is typical of Malay languages and other Austronesian languages of the larger region.



One major distinction between nouns and verbs is that nouns cannot be negated with \textitbf{tida}/\textitbf{tra} ‘\textsc{neg}’ (§5.2 and §5.3, in Chapter 5). In his discussion of pertinent typological characteristics of “western Austronesian” languages,\footnote{{\citet[111]{Himmelmann2005}} employs the term “western Austronesian” as a “rather loose geographical expression”; it is “strictly equivalent to \textit{non-Oceanic Austronesian languages}”.\\
\\
\\
\\
\\
\\
\\
\\
\\
\\
\\
\\
\\
\\
\\
} {\citet[128]{Himmelmann2005}} points out that “in languages where negators provide a diagnostic context for distinguishing nouns and verbs, putative adjectives always behave like verbs”. This also applies to Papuan Malay, in that the semantic types usually associated with adjectives are encoded by monovalent stative verbs. Verbs are divided into monovalent stative, monovalent dynamic, bivalent, and trivalent verbs. A number of adverbs are derived from monovalent stative verbs (§5.14, in Chapter 5). Personal pronouns, demonstratives, and locatives are distinct from nouns in that all four of them can modify nouns, while nouns do not modify the former (Chapter 5).
\end{styleBodyvxvafter}

\paragraph[Basic word order]{Basic word order}
\label{bkm:Ref369361475}
Papuan Malay has a basic SVO word order, as is typical of western Austronesian languages ({Himmelmann 2005: 141–144}; see also {Donohue 2007c: 355–359}). This VO order is shown in (0). Very commonly, however, arguments are omitted if the identity of their referent was established earlier. This is the case with the omitted subject \textitbf{tong} ‘\textsc{1pl}’ in the second clause and the direct object \textitbf{bua} ‘fruit’ in the third clause. An initial survey of the corpus also shows that topicalized constituents are always fronted to the clause initial position, such as the direct object \textitbf{bapa desa pu motor itu} ‘that motorbike of the mayor’ in (0).\footnote{{\citet[433]{Donohue2011}} suggests that the frequent topicalization of non-subject arguments “is an adaptive strategy that allows the OV order of the substrate languages in New Guinea […] to surface in what is nominally a VO language, Papuan Malay”.\\
In this book the issue of topicalization is not further discussed; instead, this topic is left for future research.\\
\\
\\
\\
\\
\\
\\
\\
\\
\\
\\
\\
\\
\\
}


\begin{styleExampleTitle}
Word order: Basic SVO order, elision of core arguments, and fronting of topicalized arguments
\end{styleExampleTitle}

\begin{tabular}{llllllllllll}
\lsptoprule
\label{bkm:Ref368998983}
\gll {tong} {\bluebold{liat}} {bua,} {Ø} {\bluebold{liat}} {bua} {dang} {tong} {\bluebold{mulay}} {\bluebold{tendang{\Tilde}tendang}} {Ø}\\ %
& \textsc{1pl} & see & fruit &  & see & fruit & and & \textsc{1pl} & start & \textsc{rdp}{\Tilde}kick & \\
\lspbottomrule
\end{tabular}
\ea
\glt 
‘we \bluebold{saw} a fruit, (we) \bluebold{saw} a fruit and we \bluebold{started kicking} (it)’ \textstyleExampleSource{[081006-014-Cv.0001]}
\z

\begin{tabular}{llllllllllll}
\lsptoprule
\label{bkm:Ref436750646}\label{bkm:Ref368998984}
\gll {\multicolumn{2}{l}{\bluebold{bapa}}} {\multicolumn{3}{l}{\bluebold{desa}}} {\bluebold{pu}} {\bluebold{motor}} {\bluebold{itu}} {Hurki} {de} {ada}\\ %
& \multicolumn{2}{l}{father} & \multicolumn{3}{l}{village} & \textsc{poss} & motorbike & \textsc{d.dist} & Hurki & \textsc{3sg} & exist\\
& taru & \multicolumn{2}{l}{\bluebold{Ø}} & di & \multicolumn{7}{l}{Niwerawar}\\
& put & \multicolumn{2}{l}{} & at & \multicolumn{7}{l}{Niwerawar}\\
\lspbottomrule
\end{tabular}
\ea
\glt 
‘(as for) \bluebold{that motorbike of the mayor}, Hurki is storing (\bluebold{it}) at Niwerawar’ \textstyleExampleSource{[081014-003-Cv.0024]}
\z


A Papuan Malay verb takes maximally three arguments, that is, the subject and two objects, namely a recipient-like R argument and a theme-like T argument. In double object constructions with trivalent verbs, the typical word order is ‘\textsc{subject} – \textsc{verb} – R – T’. However, trivalent verbs do not require, but allow three syntactic arguments. Most often, speakers use alternative strategies to reduce the number of arguments. (§11.1.3, in Chapter 11)



As is typical cross-linguistically, the SVO word order correlates with a number of other word order characteristics, as discussed in {\citet{Dryer2007c}}.
\end{styleBodyvafter}


Papuan Malay word order agrees with the predicted word order with respect to the order of verb and adposition, verb and adpositional phrase, main verb and auxiliary verb, mark and standard, parameter and standard, clause and complementizer, and head nominal and relative clause. In two aspects, the word order differs from the predicted order. In adnominal possessive constructions, the possessor precedes rather than follows the possessum, and in interrogative clauses, the question marker is clause-final rather than clause-initial. Six word order correlations do not apply to Papuan Malay. The word order of verb and manner adverb, of copula and predicate, and of article or plural word and noun are nonapplicable, as Papuan Malay does not have manner adverbs, a copula, an article, and a plural word. Nor does the order of main and subordinate clause and the position of adverbial subordinators apply, as in combining clauses Papuan Malay does not make a morphosyntactic distinction between main and subordinate clause.
\end{styleBodyvvafter}

\begin{stylecaption}
Table ‎1.\stepcounter{Table}{\theTable}:  Predicted word order for VO languages {\citep[130]{Dryer2007c}} versus Papuan Malay word order
\end{stylecaption}

\tablehead{
 Predicted word order & Papuan Malay word order & \arraybslash Examples\\
}
\begin{tabular}{lll}
\lsptoprule
prepositions & as predicted & (0), (0)\\
verb – prepositional phrase & as predicted & (0), (0)\\
auxiliary verb – main verb & as predicted & (0),\\
mark – standard\footnotemark{} & as predicted & (0), (0)\\
parameter – standard & as predicted & (0), (0)\\
initial complementizer & as predicted & (0)\\
noun – relative clause & as predicted & (0)\\
noun – genitive & \textsc{possessor} \textsc{lig} \textsc{possessum} & (0)\\
initial question particle & clause final question & (0)\\
verb – manner adverb & nonapplicable & \\
copula – predicate & nonapplicable & \\
article – noun & nonapplicable & \\
plural word – noun & nonapplicable & \\
main clause – subordinate clause & nonapplicable & \\
initial adverbial subordinator & nonapplicable & \\
\lspbottomrule
\end{tabular}
\footnotetext{\\
{\citet[130]{Dryer2007c}} uses the term “marker” rather than “mark”. The terminology for comparative constructions employed in this book, however, follows {Dixon’s (2008)} terminology; hence, “mark” rather than “marker” (see §11.5).\\
\\
\\
\\
\\
\\
\\
\\
\\
\\
\\
\\
\\
\\
}

Papuan Malay has prepositions, with the prepositional phrase following the verb, as illustrated in (0) and (0); auxiliary verbs precede the main verb as shown in (0) (§13.3, in Chapter 13\footnote{\\
Auxiliary verbs are briefly mentioned in §13.3, in Chapter 13; a detailed description of these verbs is left for future research.\\
\\
\\
\\
\\
\\
\\
\\
\\
\\
\\
\\
\\
\\
}) (see also {Donohue 2007c: 373–379}). The example in (0) shows that aspect-marking adverbs also precede the verb (§5.4.1, in Chapter 5); cross-linguistically, however, the order of aspect marker and verb does not correlate with the order of verb and object {\citep[130]{Dryer2007c}}.


\begin{styleExampleTitle}
Word order: Auxiliary verb – main verb – prepositional phrase
\end{styleExampleTitle}

\begin{tabular}{llllll}
\lsptoprule
\label{bkm:Ref368990622}
\gll {ko} {\bluebold{harus}} {pulang} {\bluebold{ke}} {\bluebold{tempat}}\\ %
& \textsc{2sg} & have.to & go.home & to & place\\
\lspbottomrule
\end{tabular}
\ea
\glt 
‘you \bluebold{have to} go home \bluebold{to (your own) place}’ \textstyleExampleSource{[080922-010a-CvNF.0143]}
\z

\begin{tabular}{llllll}
\lsptoprule
\label{bkm:Ref369012004}
\gll {de} {\bluebold{suda}} {naik} {\bluebold{di}} {\bluebold{kapal}}\\ %
& \textsc{3sg} & already & ascend & at & ship\\
\lspbottomrule
\end{tabular}
\ea
\glt 
‘he \bluebold{already} went \bluebold{on board}’ \textstyleExampleSource{[080923-015-CvEx.0025]}
\z


In Papuan Malay comparison clauses, the parameter precedes the mark, both of which precede the standard, as in (0) and (0). The position of the index differs depending on the type of comparison clause. In degree-marking clauses the parameter follows the index, as in the superlative clause in (0). In identity-marking clauses, by contrast, the parameter precedes the index as in the similarity clause in (0), or it is omitted. The word-order of index and parameter, however, does not correlate with that of verb and object {\citep[130]{Dryer2007c}}. (§11.5, in Chapter 11)


\begin{styleExampleTitle}
Word order: \textsc{parameter} – \textsc{mark} – \textsc{standard}
\end{styleExampleTitle}

\begin{tabular}{llllll}
\lsptoprule
\label{bkm:Ref375920248}
\gll {\textsc{comparee}} {\textsc{index}} {\textsc{parameter}} {\textsc{mark}} {\textsc{standard}}\\ %
& dia & \bluebold{lebi} & \bluebold{tinggi} & dari & saya\\
& \textsc{3sg} & more & be.high & from & \textsc{1sg}\\
\lspbottomrule
\end{tabular}
\ea
\glt 
‘he/she is \bluebold{taller} than me’ (Lit. ‘be \bluebold{more tall} from me’) \textstyleExampleSource{[Elicited BR111011.002]}
\z

\begin{tabular}{llllll}
\lsptoprule
\label{bkm:Ref376158727}
\gll {\textsc{comparee}} {\textsc{parameter}} {\textsc{index}} {\textsc{mark}} {\textsc{standard}}\\ %
& de & \bluebold{sombong} & \bluebold{sama} & deng & ko\\
& \textsc{3sg} & be.arrogant & be.same & with & \textsc{2sg}\\
\lspbottomrule
\end{tabular}
\ea
\glt 
‘she’ll be \bluebold{as arrogant as} you (are)’ (Lit. ‘be \bluebold{arrogant same} with you’) \textstyleExampleSource{[081006-005-Cv.0002]}
\z


The complementizer \textitbf{bahwa} ‘that’ occurs in clause-initial position, with the complement clause following the verb, as in (0). (§14.3.1, in Chapter 14)


\begin{styleExampleTitle}
Word order: Initial complementizer
\end{styleExampleTitle}

\begin{tabular}{lllllllllll}
\lsptoprule
\label{bkm:Ref368990624}
\gll {sa} {tida} {taw} {\bluebold{bahwa}} {jam} {tiga} {itu} {de} {su} {meninggal}\\ %
& \textsc{1sg} & \textsc{neg} & know & that & hour & three & \textsc{d.dist} & \textsc{3sg} & already & die\\
\lspbottomrule
\end{tabular}
\ea
\glt 
‘I didn’t know \bluebold{that} by three o’clock (in the afternoon) she had already died’ \textstyleExampleSource{[080917-001-CvNP.0005]}
\z


Within the noun phrase, the relative clause follows its head nominal, as shown in (0) (§8.2.8, in Chapter 8). Other modifiers, such as demonstratives, or monovalent stative verbs, also occur to the right of the head nominal. This order of head nominal and modifier is typical for western Austronesian languages ({Himmelmann 2005: 142}; see also {Donohue 2007c: 359–373}). Cross-linguistically, however, the order of head nominal and demonstrative, numeral, or stative verb does not correlate with the order of verb and object {\citep[130]{Dryer2007c}}. Numerals and quantifiers precede or follow the head nominal, depending on the semantics of the phrasal structure (§8.3, in Chapter 8).


\begin{styleExampleTitle}
Word order: Head nominal – relative clause
\end{styleExampleTitle}

\begin{tabular}{lllllllllll}
\lsptoprule
\label{bkm:Ref368990625}
\gll {…} {karna} {liat} {ada} {makangang} {dalam} {\bluebold{kantong}} {\bluebold{yang}} {saya} {bawa}\\ %
&  & because & see & exist & food & inside & bag & \textsc{rel} & \textsc{1sg} & bring\\
\lspbottomrule
\end{tabular}
\ea
\glt 
‘[she was already glad] because (she) saw there was food in \bluebold{the bag that} I brought’ \textstyleExampleSource{[080919-004-NP.0032]}
\z


Likewise in noun phrases with adnominally used nouns, the modifier noun follows the head nominal, as in \textitbf{tulang bahu} ‘shoulder bone’ (§8.2.2, in Chapter 8). By contrast, adnominal possession in Papuan Malay is typically expressed with a construction in which the \textsc{possessor} precedes the \textsc{possessum}; both are linked with the possessive marker \textitbf{pu(nya)} ‘\textsc{poss}’, as illustrated in (0) (Chapter 9). This word order does not correlate with the general VO order, but it is typical for the eastern Malay varieties in general and other Austronesian languages of the larger region, as discussed in more detail in §1.6.2.


\begin{styleExampleTitle}
Word order: \textsc{possessor} – \textsc{possessum}
\end{styleExampleTitle}

\begin{tabular}{lllllllllll}
\lsptoprule
\label{bkm:Ref369004065}
\gll {…} {sa} {pegang} {\bluebold{sa}} {pu} {parang} {\bluebold{sa}} {punya} {jubi} {…}\\ %
&  & \textsc{1sg} & hold & \textsc{1sg} & \textsc{poss} & short.machete & \textsc{1sg} & \textsc{poss} & bow.and.arrow & \\
\lspbottomrule
\end{tabular}
\ea
\glt 
‘[so, in the morning I got up, I fed the dogs,] I took \bluebold{my} short machete, \bluebold{my} bow and arrows …’ \textstyleExampleSource{[080919-003-NP.0003]}
\z


In alternative interrogative clauses, the question marker occurs in clause-final position. Such questions are formed with the alternative-marking conjunction \textitbf{ka} ‘or’ which is also used to mark interrogative clauses, as demonstrated in (0) (§13.2.3, in Chapter 13; see also §14.2.2.2, Chapter 14). Again, this word order does not correlate with the general VO order.


\begin{styleExampleTitle}
Word order: Clause-final question marker \textitbf{ka} ‘or’
\end{styleExampleTitle}

\begin{tabular}{llll}
\lsptoprule
\label{bkm:Ref369004758}
\gll {ko} {sendiri} {\bluebold{ka}?}\\ %
& \textsc{2sg} & be.alone & or\\
\lspbottomrule
\end{tabular}
\ea
\glt 
‘are you alone \bluebold{or (not)}?’ \textstyleExampleSource{[080921-010-Cv.0003]}
\z


As mentioned, in a number of aspects the predicted word order does not apply to Papuan Malay. Papuan Malay has no manner adverbs. Instead monovalent stative verbs express manner; they take a post-predicate position (§5.4.8, in Chapter 5). The language has no copula either. Hence, in nonverbal predicate clauses, the nonverbal predicate is juxtaposed to the subject (Chapter 12). Neither does Papuan Malay have an article or plural word. Instead, free personal pronouns signal the person, number, and definiteness of their referents (Chapter 6). In combining clauses, Papuan Malay makes no morphosyntactic distinction between main and subordinate clauses; dependency relations are purely semantic (§14.2, in Chapter 14).



In negative clauses, the negators occur in pre-predicate position: \textitbf{tida}/\textitbf{tra} ‘\textsc{neg}’ negates verbal, existential, and nonverbal prepositional clauses, while \textitbf{bukang} ‘\textsc{neg}’ negates nonverbal clauses, other than prepositional ones; besides, \textitbf{bukang} ‘\textsc{neg}’ also marks contrastive negation (§13.1, in Chapter 13). This negator-predicate order is typical for western Austronesian languages {\citep[141]{Himmelmann2005}}. Cross-linguistically, however, it does not correlate with the order of verb and object {\citep[130]{Dryer2007c}}.
\end{styleBodyvxvafter}

\subsection{Papuan Malay as a language of the Papuan contact zone}
\label{bkm:Ref369507788}
In this section, some of the typological features of Papuan Malay are compared to pertinent features found in Austronesian languages in general, as well as to features typical for Austronesian languages spoken in the larger region, and to some features of Papuan languages.\footnote{\\
\label{bkm:Ref389055489}The term “Papuan” is a collective label used for “the non-Austronesian languages spoken in New Guinea and archipelagos to the West and East”; that is, the term “does not refer to a superordinate category to which all the languages belong” {(Klamer et al. 2008: 107)}.\\
\\
\\
\\
\\
\\
\\
\\
\\
\\
\\
\\
\\
\\
}



The reason for this investigation is the observation that Papuan Malay is lacking some of the features typical for Austronesian languages, while it has a number of features which are found in Papuan languages. This investigation is not based on a comparative study, which would explore whether and to what extent Papuan Malay, as spoken in Sarmi on West Papua’s northeast coast, has adopted features found in the languages of the larger region, such as Isirawa, a Tor-Kwerba language and the language of the author’s hosts, or the Tor-Kwerba languages Kwesten and Mo, or the Austronesian languages Mo and Sobei. Such a study is left for future research. (See also Table  ‎1 .1 in §1.4.)
\end{styleBodyvafter}


Instead this investigation is based on studies on areal diffusion. For a long time, scholars have noted that in the area east of Sulawesi, Sumba, and Flores, all the way to the Bird’s Head of New Guinea, a number of linguistic features have diffused from Papuan into Austronesian languages and vice versa.
\end{styleBodyvafter}


{\citet{KlamerEtAl2008}} and{ Klamer and \citet{Ewing2010}} propose the term “East Nusantara” for this area. More specifically, {Klamer and \citet[1]{Ewing2010}} define\footnote{\\
As {Klamer and \citet[1]{Ewing2010}} point out, though, there is an ongoing discussion about “the exact geographic delimitations of the East Nusantara region” and “whether (parts of) New Guinea are also considered to be part of it” (see also Footnote 3 in {Klamer and Ewing 2010: 1}).\\
\\
\\
\\
\\
\\
\\
\\
\\
\\
\\
\\
\\
\\
}
\end{styleBodyvvafter}

\begin{styleIvI}
East Nusantara as a geographical area that extends from Sumbawa to the west, across the islands of East Nusa Tenggara, Maluku […] including Halmahera, and to the Bird’s Head of New Guinea in the east […]. In the northwest, the area is bounded by Sulawesi.
\end{styleIvI}


According to the above definition, only parts of West Papua belong to East Nusantara, namely the Bird’s Head but not West Papua’s north coast. Yet, it seems useful to examine the typological profile of Papuan Malay in light of the observed diffusion of linguistic features, discussed in {\citet{KlamerEtAl2008}} and {Klamer and \citet{Ewing2010}}.



This comparison shows that Papuan Malay is lacking some of the features which are typical for Austronesian languages. At the same time, it has a number of features which are untypical for Austronesian languages, but which are found in Austronesian languages of East Nusantara. Moreover, Papuan Malay has some features not typically found in Austronesian languages of East Nusantara but found in Papuan languages. These features are summarized in Table  ‎1 .4 to Table  ‎1 .6; the listed features are taken from {\citet{KlamerEtAl2008}} and {Klamer and \citet{Ewing2010}}, unless mentioned otherwise.
\end{styleBodyvafter}


Table  ‎1 .4 presents seven features found in Austronesian languages in general, six of which are listed in {\citet[113]{KlamerEtAl2008}}.\footnote{\\
The noun-genitive order is not explicitly mentioned in {\citet{KlamerEtAl2008}}.\\
\\
\\
\\
\\
\\
\\
\\
\\
\\
\\
\\
\\
\\
} Papuan Malay shares five of these features. It does not, however, share the typical noun-genitive order which is used to express adnominal possession. Papuan Malay noun phrases with post-head nominal modifiers are used to denote important features for subclassification of the head nominal rather than for adnominal possession (§8.2.2, in Chapter 8). Also, Papuan Malay does not distinguish between inclusive and exclusive first person plural in its pronominal paradigm.
\end{styleBodyvvafter}

\begin{stylecaption}
\label{bkm:Ref369169323}Table ‎1.\stepcounter{Table}{\theTable}:  Pertinent features of Austronesian languages in general vis-à-vis Papuan Malay features
\end{stylecaption}

\tablehead{
 Austronesian languages & \multicolumn{2}{l}{ Papuan Malay}\\
}
\begin{tabular}{lll}
\lsptoprule
Phonemic l/r distinction & yes & (Chap. 2)\\
Preference for CVCV roots & yes & (Chap. 2)\\
Reduplication & yes & (Chap. 4)\\
Head-initial & yes & (Chap. 8)\\
Negator precedes the predicate & yes & (Chap. 13)\\
Noun-genitive order & no & (Chap. 8 \& 9)\\
Inclusive/exclusive distinction in personal pronouns & no & (Chap. 5 \& 6)\\
\lspbottomrule
\end{tabular}

Table  ‎1 .5 lists 17 linguistic features “found in many of the Austronesian languages of East Nusantara” {(Klamer and Ewing 2010: 10)};\footnote{\\
This list of features in {Klamer and \citet{Ewing2010}} builds on {\citet{Klamer2002}}, {\citet{Himmelmann2005}}, {\citet{Donohue2007c}}, and {\citet{KlamerEtAl2008}}.\\
\\
\\
\\
\\
\\
\\
\\
\\
\\
\\
\\
\\
\\
} some of these features are also listed in Table  ‎1 .4. Papuan Malay shares eight of them, such as the preference for CVCV roots or the lack of a productive voice system on verbs. Another eight features, however, are unattested in the corpus, such as metathesis or clause-final negators.


\begin{stylecaption}
\label{bkm:Ref369192976}Table ‎1.\stepcounter{Table}{\theTable}:  Pertinent features of Austronesian languages of East Nusantara vis-à-vis Papuan Malay features
\end{stylecaption}

\tablehead{
\multicolumn{2}{l}{ Austronesian languages of East Nusantara} & \multicolumn{2}{l}{ Papuan Malay}\\
}
\begin{tabular}{llll}
\lsptoprule
\multicolumn{4}{l}{Phonology}\\
& Preference for CVCV roots & yes & (Chap. 2)\\
& Prenasalized consonants & no & (Chap. 2)\\
& Metathesis & no & (Chap. 2)\\
\multicolumn{4}{l}{Morphology}\\
& No productive voice system on verbs & yes & (Chap. 3 \& 5)\\
& Left-headed compounds\footnotemark{} & yes & (Chap. 3)\\
& Agent/subject indexed on verb as prefix/proclitic & no & (Chap. 3 \& 5)\\
& Inclusive/exclusive distinction in personal pronouns & no & (Chap. 5 \& 6)\\
& Morphological distinction between alienable and inalienable nouns & no & (Chap. 3 \& 5)\\
\multicolumn{4}{l}{Syntax}\\
& Verb-object order & yes & (Chap. 11)\\
& Prepositions & yes & (Chap. 10)\\
& Genitive-noun order (“preposed possessor”) & yes & (Chap. 8 \& 9)\\
& Noun-Numeral order & yes & (Chap. 8)\\
& Absence of a passive construction & yes & (Chap. 11)\\
& Clause-final negators & no & (Chap. 13)\\
& Clause-initial indigenous complementizers\footnotemark{} & no & (Chap. 14)\\
& Formally marked adverbial/complement clauses & no & (Chap. 14)\\
\multicolumn{4}{l}{Other}\\
& Parallelisms without stylistic optionality & {}-{}-{}- & not yet researched\\
\lspbottomrule
\end{tabular}
\addtocounter{footnote}{-2}
\stepcounter{footnote}\footnotetext{\\
In Papuan Malay the demarcation between compounds and phrasal expressions is unclear, however. Hence, it remains uncertain to what degree compounding is a productive process. (For more details see §3.2.)\\
\\
\\
\\
\\
\\
\\
\\
\\
\\
\\
\\
\\
\\
}
\stepcounter{footnote}\footnotetext{\\
The Papuan Malay complementizer is \textitbf{bahwa} ‘that’. According to {\citet{Jones2007}} it originates from Sanskrit.\\
\\
\\
\\
\\
\\
\\
\\
\\
\\
\\
\\
\\
\\
}

Two of the nonshared morphological and two of the shared syntactic features require additional commenting, that is, indexing on the verb, the distinction between alienable and inalienable nouns, the noun-numeral order, and the absence of a passive construction.
\end{styleBodyaftervbefore}


Papuan Malay does not have indexing on the verb. Instead, Papuan Malay uses free personal pronouns (Chapter 6).
\end{styleBodyvafter}


Overall, Papuan Malay does not distinguish between alienable and inalienable possessed items, with one exception: adnominal possessive constructions with omitted possessive marker signal inalienable possession of body parts or kinship relations. Just as commonly, however, inalienable possession of these entities is encoded in the same way as possession of alienable items, that is, in a \textsc{possessor} \textsc{ligature} \textsc{possessum} construction. Examples are \textitbf{sa maytua} ‘my wife’, \textitbf{dia pu maytua} ‘his wife’, or \textitbf{sa pu motor} ‘my motorbike’ (literally ‘\textsc{1sg} wife’, ‘\textsc{3sg} \textsc{poss} wife’, ‘\textsc{1sg} \textsc{poss} motorbike’). (Chapter 9)
\end{styleBodyvafter}


In Papuan Malay noun phrases, numerals and quantifiers follow the head nominal. As mentioned in §1.6.1, however, they can also precede the head nominal, depending on the semantics of the phrasal structure (§8.3, in Chapter 8).
\end{styleBodyvafter}


Like other East Nusantara Austronesian languages, Papuan Malay does not have a dedicated passive construction. Instead, speakers encode actions and events in active constructions (see also §1.6.1.2).\footnote{\\
As mentioned in §1.6.1.2, passive constructions are not further discussed in this book; instead, this topic is left for future research.\\
\\
\\
\\
\\
\\
\\
\\
\\
\\
\\
\\
\\
\\
}
\end{styleBodyvafter}


East Nusantara Austronesian languages also often make use of parallelisms without stylistic optionality.\footnote{\\
{Klamer (2002: 370, 371)} defines ‘Parallelisms without stylistic optionality’ as follows: “Many languages in Eastern Indonesia employ the verbal art form of parallelism [… It] is a structurally defined verbal art form that functions as a stylistic device in the ritual language […] In parallelism, semantically synonymic words or phrases are combined in (minimally two) parallel utterances. […] Though parallelism is a property of oral literature, it is not purely stylistic: the pairings are obligatory; there is generally no stylistic optionality involved in the choice of a proper pair.”\\
\\
\\
\\
\\
\\
\\
\\
\\
\\
\\
\\
\\
\\
} Whether, and to what extent, Papuan Malay employs this feature has not been researched for the present study; instead this topic is left for future research.
\end{styleBodyvafter}


Papuan Malay also has a number of features which are not usually found in the East Nusantara Austronesian languages. Instead, these features are typical characteristics of Papuan languages.
\end{styleBodyvafter}


Table  ‎1 .6 presents 15 linguistic features typically found in Papuan languages {(Klamer and Ewing 2010: 10)}.\footnote{\\
This list of features in {Klamer and \citet{Ewing2010}} builds on {\citet{Foley1986}}, {\citet{Foley2000}}, {\citet{Pawley2005}}, and {Aikhenvald and \citet{Stebbins2007}}.\\
Tail-head linkage is not mentioned in {\citet{KlamerEtAl2008}}. It is, however, a typical Papuan feature (see Foley 1986: 200–201; {Foley 2000: 390}).\\
\\
\\
\\
\\
\\
\\
\\
\\
\\
\\
\\
\\
} Papuan Malay shares six of them, such as the subject-verb order, or the genitive-noun order. There is also limited overlap between Papuan Malay and Papuan languages with respect to the position of conjunctions. All Papuan Malay conjunctions are clause-initial, but two of them can also take a clause-final position (Chapter 14). Eight of the 15 features are not found in Papuan Malay, such as gender marking or postpositions.
\end{styleBodyvvafter}

\begin{stylecaption}
\label{bkm:Ref369186426}Table ‎1.\stepcounter{Table}{\theTable}:  Pertinent features of Papuan languages vis-à-vis Papuan Malay features
\end{stylecaption}

\tablehead{
\multicolumn{2}{l}{ Papuan languages} & \multicolumn{2}{l}{ Papuan Malay}\\
}
\begin{tabular}{llll}
\lsptoprule
\multicolumn{4}{l}{Phonology}\\
& No phonemic l/r distinction & no & (Chap. 2)\\
\multicolumn{4}{l}{Morphology}\\
& No inclusive/exclusive distinction in personal pronouns & yes & (Chap. 5\& 6)\\
& Marking of gender & no & (Chap. 3 \& 5)\\
& Subject marked as suffix on verb & no & (Chap. 3 \& 5)\\
& Morphological distinction between alienable and inalienable nouns & no & (Chap. 3 \& 5)\\
\multicolumn{4}{l}{Syntax}\\
& Subject-verb order & yes & (Chap. 11)\\
& Genitive-noun order (“preposed possessor”) & yes & (Chap. 8 \& 9)\\
& Serial verb constructions\footnotemark{} & yes & (Chap. 11)\\
& Clause-chaining & yes & (example (0))\\
& Tail-head linkage & yes & (example (0))\\
& Clause-final conjunctions & few & (Chap. 14)\\
& Object-verb order & no & (Chap. 11)\\
& Postpositions & no & (Chap. 10)\\
& Clause-final negator & no & (Chap. 13)\\
& Switch reference & no & (Chap. 14)\\
\lspbottomrule
\end{tabular}
\footnotetext{\\
\\
Serial verb constructions are briefly mentioned in §11.2, in Chapter 11; a detailed description of this topic is left for future research.\\
\\
\\
\\
\\
\\
\\
\\
\\
\\
\\
\\
\\
}

Among the syntactic features, three need to be commented on, namely clause-chaining, switch reference, and tail-head linkage.
\end{styleBodyaftervbefore}


Clause chaining is not discussed in the present study. An initial survey of the corpus indicates, however, that it is very common in Papuan Malay. One example is given in (0).
\end{styleBodyvvafter}

\begin{styleExampleTitle}
Clause-chaining in Papuan Malay
\end{styleExampleTitle}

\begin{tabular}{llllllll}
\lsptoprule
\label{bkm:Ref369280880}
\gll {langsung} {\bluebold{sa}} {\bluebold{pegang}} {\bluebold{sa}} {\bluebold{putar}} {\bluebold{sa}} {\bluebold{cari}}\\ %
& immediately & \textsc{1sg} & hold & \textsc{1sg} & turn.around & \textsc{1sg} & search\\
\lspbottomrule
\end{tabular}
\ea
\glt 
‘immediately \bluebold{I held} (the plate), \bluebold{I turned around}, \bluebold{I looked around}’ \textstyleExampleSource{[081011-005-Cv.0034]}
\z


Following {Klamer and \citet[11]{Ewing2010}}, clause-chaining in Papuan languages is often characterized by “some concomitant switch reference system”. This, however, does not seem to apply to Papuan Malay. That is, so far dedicated switch-references devices have not been identified, a finding which contrasts with {Donohue’s (2011)} observations. {Donohue (2011: 431–432)} suggests that the sequential-marking conjunction \textitbf{trus} ‘next’ “is a commonly used connective when there is a same-subject coreference condition between clauses”, while the sequential-marking conjunction \textitbf{baru} ‘and then’ tends “to indicate switch reference”. An initial investigation of the attested \textitbf{trus} ‘next, and then’ and \textitbf{baru} ‘and then’ tokens in the corpus shows, however, that both conjunction more often link clauses with a switch in reference, than those with same-subject coreference (§14.2.3.1 and §14.2.3.2, in Chapter 14). Neither do any of the other conjunctions function as dedicated switch-references devices.



Tail-head linkage is not treated in the present study. This feature denotes a “structure in which the final clause of the previous sentence initiates the next sentence, often in a reduced form” ({Foley 2000: 390}; see also {de Vries 2005}). An initial survey of the corpus shows, however, that tail-head linkage is very common in Papuan Malay. In the example in (0), for instance, the speaker repeats part of the first clause at the beginning of the second clause: \textitbf{kasi senter} ‘give a flashlight’.
\end{styleBodyvvafter}

\begin{styleExampleTitle}
Tail-head linkage in Papuan Malay
\end{styleExampleTitle}

\begin{tabular}{lllllllllll}
\lsptoprule
\label{bkm:Ref369280882}
\gll {\multicolumn{2}{l}{skarang}} {\multicolumn{2}{l}{dong}} {\bluebold{kasi}} {dia} {\bluebold{senter},} {\bluebold{kasi}} {\bluebold{senter}} {dong}\\ %
& \multicolumn{2}{l}{now} & \multicolumn{2}{l}{\textsc{3pl}} & give & \textsc{3sg} & flashlight & give & flashlight & \textsc{3pl}\\
& mo & \multicolumn{2}{l}{kasi} & \multicolumn{7}{l}{pisow}\\
& want & \multicolumn{2}{l}{give} & \multicolumn{7}{l}{knife}\\
\lspbottomrule
\end{tabular}
\ea
\glt
‘now they \bluebold{give} him \bluebold{a flashlight}, (having) \bluebold{given} (him) \bluebold{a flashlight}, they want to give (him) a knife’ \textstyleExampleSource{[081108-003-JR.0002]}
\end{styleFreeTranslEngxvpt}

\subsection{Papuan Malay as an eastern Malay variety}
\label{bkm:Ref369507789}
This section compares some of the features found in Papuan Malay to those found in other eastern Malay varieties, namely in Ambon Malay (AM) {(van Minde 1997)}, Banda Malay (BM) {\citep{Paauw2009}}, Kupang Malay (KM) {\citep{Steinhauer1983}}, Larantuka Malay (LM) {\citep{Paauw2009}}, Manado Malay (MM) {\citep{Stoel2005}}, North Moluccan or Ternate Malay (NMM/TM) ({Taylor 1983; Voorhoeve 1983;} {Litamahuputty 2012}).\footnote{\\
\\
In their contributions, {\citet{Taylor1983}} and {\citet{Voorhoeve1983}} label the Malay variety spoken in the northern Moluccas as North Moluccan Malay, while {\citet{Litamahuputty2012}} uses the term Ternate Malay for the same variety in her in-depth grammar. Given that the three studies differ in depth, all three of them are included here, with Taylor’s (1983) and Voorhoeve’s (1983) summarily listed under North Moluccan Malay.\\
\\
\\
\\
\\
\\
\\
\\
\\
\\
\\
\\
\\
}



These comparisons are far from systematic and exhaustive. Instead, they pertain to a limited number of topics as they came up during the analysis and description of the phonology, morphology, and syntax of Papuan Malay. (A detailed typological study of the eastern Malay varieties is {Paauw 2009}.) The comparisons discussed here touch upon the following phenomena:
\end{styleBodyvvafter}

\begin{itemize}
\item \begin{styleIIndented}
Affixation (§3.1, in Chapter 3)
\end{styleIIndented}\item \begin{styleIIndented}
Reduplication (Chapter 4)
\end{styleIIndented}\item \begin{styleIIndented}
Adnominal uses of the personal pronouns (§6.2, in Chapter 6)
\end{styleIIndented}\item \begin{styleIIndented}
Existence of diphthongs (§2.1.2, in Chapter 2)
\end{styleIIndented}\item \begin{styleIIndented}
Non-canonical functions of the possessive ligature in adnominal possessive constructions (§9.3, in Chapter 9)
\end{styleIIndented}\item \begin{styleIIndented}
Argument elision in verbal clauses (§11.1, in Chapter 11)
\end{styleIIndented}\item \begin{styleIIndented}
Morphosyntactic status of the reciprocity marker \textitbf{baku} ‘\textsc{recp}’ (§11.3, in Chapter 11)
\end{styleIIndented}\end{itemize}
\begin{itemize}
\item \begin{styleIvI}
Contrastive uses of negator \textitbf{bukang} ‘\textsc{neg}’ (§13.1.2, in Chapter 13)
\end{styleIvI}\end{itemize}

The remainder of this section gives an overview how Papuan Malay compares to the other eastern Malay varieties with respect to these phenomena. (In Table  ‎1 .7 to Table  ‎1 .10 empty cells signal that a given feature is not mentioned in the available literature. One reason could be that the respective feature is nonexistent. It is, however, just as likely that such empty cells result from gaps in the available literature.)



Affixation is one area in which Papuan Malay has a number of features which are distinct from those found in other eastern Malay varieties. Table  ‎1 .7 presents three prefixes and one suffix and shows that the Papuan Malay affixes are different both in terms of their form and their degree of productivity. In most of the eastern Malay varieties, the three prefixes are realized as \textitbf{ta\-}, \textitbf{pa(}\textscItalBold{n}\textitbf{)}\-, and \textitbf{ba\-}. By contrast, the Papuan Malay affixes \textscItalBold{ter\-} (\textsc{acl}), \textscItalBold{pe(n)\-} (\textsc{ag}), and \textscItalBold{ber\-} (\textsc{vblz}) are most commonly realized as \textitbf{ter}\-, \textitbf{pe(}\textscItalBold{n}\textitbf{)}\textitbf{\-}, and \textitbf{ber}\-, respectively; hence, they have more resemblance with the corresponding Standard Indonesian affixes.
\end{styleBodyvafter}


Papuan Malay prefix \textscItalBold{ter\-} has only limited productivity, while prefix \textscItalBold{ber\-} is unproductive. In the other eastern Malay varieties, by contrast, the corresponding prefixes \textitbf{ta\-} and \textitbf{ba\-} are very productive. Papuan Malay prefix \textscItalBold{pe(n)\-} is, at best, marginally productive. In Manado Malay \textitbf{paŋ-} is productive (in addition an unproductive form \textitbf{pa}\textsc{\-} exists). Likewise, in North Moluccan / Ternate Malay prefixation with \textitbf{pang-} is productive {\citep[30]{Litamahuputty2012}}.\footnote{\\
\\
{\citet[4]{Voorhoeve1983}}, by contrast, suggests that \textitbf{pa}\- “is no longer morphologically distinct”.\\
\\
\\
\\
\\
\\
\\
\\
\\
\\
\\
\\
\\
} In Ambon Malay the prefix occurs but it is unproductive. The Papuan Malay prefix \-\textitbf{ang} has only limited productivity. In Ambon Malay, the suffix also occurs but according to {van Minde}{ }{(1997: 106)} it is difficult to determine whether and to what degree it is productive.
\end{styleBodyvvafter}

\begin{stylecaption}
\label{bkm:Ref369348527}Table ‎1.\stepcounter{Table}{\theTable}:  Affixation: Form and productivity
\end{stylecaption}

\tablehead{ & PM & AM & BM & KM & LM & MM & \multicolumn{2}{l}{ NMM   /   TM}\\
}
\begin{tabular}{lllllllll}
\lsptoprule
\multicolumn{9}{l}{Prefix \textscItalBold{ter\-}}\\
Form & \textscItalBold{ter\-} & \textitbf{ta\-} & \textitbf{ta\-} & \textitbf{ta\-} & \textitbf{tə(r)\-} & \textitbf{ta\-} & \textitbf{ta\-} & \arraybslash \textitbf{ta\-}\\
\textsc{prod} & lim. & yes & yes & yes & yes & yes & yes & \arraybslash yes\\
\multicolumn{9}{l}{Prefix \textscItalBold{pe(n)\-}}\\
Form & \textscItalBold{pe(n)\-} & \textitbf{pa(}\textscItalBold{n}\textitbf{)}\- &  &  &  & \textitbf{paŋ\-} & \textitbf{pa}\textitbf{\-} & \arraybslash \textitbf{pang}\-\\
\textsc{prod} & marg. & no &  &  &  & yes & no & \arraybslash yes\\
\multicolumn{9}{l}{Prefix \textscItalBold{ber\-}}\\
Form & \textscItalBold{ber\-} & \textitbf{ba\-} & \textitbf{ba\-} & \textitbf{ba\-} & \textitbf{bə(r)\-} & \textitbf{ba\-} & \textitbf{ba\-} & \arraybslash \textitbf{ba\-}\\
\textsc{prod} & no & yes & yes & yes & yes & yes & yes & \arraybslash yes\\
\multicolumn{9}{l}{Prefix \-\textitbf{ang}}\\
\textsc{prod} & lim. & ? &  &  &  &  &  & \\
\lspbottomrule
\end{tabular}

Reduplication is another phenomenon in which Papuan Malay displays a number of features which differ from those described for other eastern Malay varieties (Chapter 4). As shown in Table  ‎1 .8, Papuan Malay and the other eastern Malay varieties employ full reduplication. Partial and imitative reduplication however, is only reported for Papuan Malay, Ambon Malay, and Larantuka Malay. Besides, Papuan Malay shares especially many features with Ambon Malay regarding the morpheme types which can undergo full reduplication (§4.3.1, in Chapter 4).
\end{styleBodyaftervbefore}


In general, reduplication conveys a wide range of different meaning aspects. These meaning aspects differ with respect to the range of word classes they attract for reduplication. Among the eastern Malay varieties, the attested meaning aspects in Papuan Malay attract the largest range of different word classes, followed by a medium range of attracted word classes in Ambon Malay. In the other eastern Malay varieties, by contrast, this range of attracted word classes seems to be much smaller. (§4.3.2, in Chapter 4)
\end{styleBodyvafter}


In Papuan Malay, the reduplicated items can also undergo “interpretational shift” or “type coercion”. This feature is also attested in Ambon, Larantuka, Manado, and Ternate Malay. Again, Papuan Malay and Ambon Malay share pertinent features, in that in both varieties nouns and verbs can undergo interpretational shift, while in Manado Malay only nouns and in Larantuka and Ternate Malay only verbs are affected. (§4.3.3, in Chapter 4)
\end{styleBodyvafter}


These findings suggest that reduplication in Papuan Malay has more in common with Ambon Malay than with the other eastern Malay varieties.
\end{styleBodyvvafter}

\begin{stylecaption}
\label{bkm:Ref369357547}Table ‎1.\stepcounter{Table}{\theTable}:  Reduplication
\end{stylecaption}

\tablehead{ & PM & AM & BM & KM & LM & \multicolumn{2}{l}{ NMM   /   TM}\\
}
\begin{tabular}{llllllll}
\lsptoprule
\multicolumn{8}{l}{Type of reduplication}\\
Full & yes & yes & yes & yes & yes & yes & \arraybslash yes\\
Partial & yes & yes &  &  & yes &  & \\
Imitative & yes & yes &  &  & yes &  & \\
\multicolumn{8}{l}{Meaning aspects and range of attracted word classes}\\
Range & large & med. & small & small & small & small & \arraybslash small\\
\multicolumn{8}{l}{Interpretational shift of reduplicated lexemes}\\
Shift & yes & yes &  &  & yes &  & \arraybslash yes\\
\lspbottomrule
\end{tabular}

Papuan Malay is also distinct from other eastern Malay varieties with respect to the adnominal uses of its personal pronouns (§6.2, in Chapter 6). In Papuan Malay, the second and third singular person pronouns have adnominal uses. They signal definiteness and person-number values, whereby they allow the unambiguous identification of their referents. In other eastern Malay varieties, by contrast, ‘\textsc{n} \textsc{pro-sg}’ expressions are analyzed as topic-comment constructions. Besides, the first, second, and third person plural pronouns in Papuan Malay also have adnominal uses; they express associative plurality. In the other eastern Malay varieties, by contrast, associative plural expressions are only formed with the third person plural pronoun.


\begin{stylecaption}
Table ‎1.\stepcounter{Table}{\theTable}:  Personal pronouns: Adnominal uses of singular and plural pronouns
\end{stylecaption}

\begin{tabular}{lllllllll} & PM & AM & BM & KM & LM & MM & \multicolumn{2}{l}{ NMM   /   TM}\\
\lsptoprule
\textsc{2/3sg} & yes & no & no &  &  &  & no & \\
\textsc{1/2pl} & yes & no &  & no &  & no &  & \arraybslash no\\
\textsc{3pl}\footnotemark{} & yes & yes &  & yes &  & yes &  & \arraybslash no\\
\lspbottomrule
\end{tabular}
\footnotetext{\\
\\
Adnominal uses of the third person plural pronoun are also reported for Balai Berkuak Malay {\citep[7]{Tadmor2002}}, Dobo Malay {(R.J. Nivens, p.c. 2013)}, and Sri Lanka Malay {\citep{Slomanson2013}}; in Balai Berkuak Malay and Manado Malay the personal pronoun occurs in pre-head position.\\
\\
\\
\\
\\
\\
\\
\\
\\
\\
\\
\\
\\
}

In addition, Papuan Malay is compared to the other eastern Malay varieties in terms of one phonological and four syntactic features, summarized in Table  ‎1 .10.
\end{styleBodyaftervbefore}


Papuan Malay has no diphthongs; instead the vowel combinations /\textstyleChCharisSIL{ai}/ and /\textstyleChCharisSIL{au}/ are analyzed as V.V or VC sequences (§2.1.2, in Chapter 2). The same analysis applies to Larantuka and Manado Malay. For Ambon and North Moluccan Malay, by contrast, the same vowel sequences are analyzed as diphthongs. Most likely, though, the different analyses result from differences between the analysts rather than from distinctions between the respective Malay varieties.



In adnominal possessive constructions, the ligature \textitbf{pu(nya)} ‘\textsc{poss}’ not only marks possessive relations, but also has a number of non-canonical functions, such as that of an emphatic marker. Such non-canonical functions of the ligature are also reported for two other eastern Malay varieties, namely Ambon and Ternate Malay.
\end{styleBodyvafter}


In Papuan Malay verbal clauses, core arguments are very often elided (see §1.6.1.4 and §11.1, in Chapter 11). The same observation applies to Ambon and Manado Malay.
\end{styleBodyvafter}


In Papuan Malay verbal clauses, the reciprocity marker \textitbf{baku} ‘\textsc{recp}’ is analyzed as a separate word (§11.3, in Chapter 11). For Ambon, Banda, Kupang, Manado, and North Moluccan / Ternate Malay, by contrast, the same marker is analyzed as a prefix. Most likely, this different analysis is again due to differences between the analysts rather than due to linguistic differences between the respective Malay varieties.
\end{styleBodyvafter}


In Papuan Malay negative clauses, the negator \textitbf{bukang} ‘\textsc{neg}’ not only negates nouns and nominal predicate clauses, but also signals contrast (§13.1.2, in Chapter 13). The same observation applies to Ambon, Manado, and Ternate Malay.
\end{styleBodyvvafter}

\begin{stylecaption}
\label{bkm:Ref369360797}Table ‎1.\stepcounter{Table}{\theTable}:  Some phonological and syntactic features in Papuan Malay and other eastern Malay varieties
\end{stylecaption}

\begin{tabular}{lllllllll}
\lsptoprule

\multicolumn{9}{l}{Phonology: Diphthongs}\\
& PM & AM & BM & KM & LM & MM & \multicolumn{2}{l}{ NMM   /   TM}\\
\textsc{diph} & no & yes &  &  & no & no & yes & \\
\multicolumn{9}{l}{Adnominal possessive constructions: Non-canonical uses of the ligature (\textsc{lig})}\\
& PM & AM & BM & KM & LM & MM & \multicolumn{2}{l}{ NMM   /   TM}\\
\textsc{lig} use & yes & yes &  &  &  &  &  & \arraybslash yes\\
\multicolumn{9}{l}{Verbal clauses: Argument elision}\\
& PM & AM & BM & KM & LM & MM & \multicolumn{2}{l}{ NMM   /   TM}\\
Elision & yes & yes &  &  &  & yes &  & \\
\multicolumn{9}{l}{Verbal clauses: Morphosyntactic status of reciprocity marker \textitbf{baku} ‘\textsc{recp}’}\\
& PM & AM & BM & KM & LM & MM & \multicolumn{2}{l}{ NMM   /   TM}\\
\textsc{recp} & word & prefix & prefix & prefix &  & prefix & prefix & \arraybslash prefix\\
\multicolumn{9}{l}{Negative clauses: Contrastive function of \textitbf{bukang} ‘\textsc{neg}’}\\
& PM & AM & BM & KM & LM & MM & \multicolumn{2}{l}{ NMM   /   TM}\\
\textsc{cst} & yes & yes &  &  &  & yes &  & \arraybslash yes\\
\lspbottomrule
\end{tabular}

The overview presented in this section shows several differences and commonalities between Papuan Malay and the other eastern Malay varieties.
\end{styleBodyaftervbefore}


The differences pertain to affixation (form and degree of productivity of the affixes), and the adnominal uses of the personal pronouns. The discussed commonalities involve reduplication, the non-canonical uses of the possessive ligature, elision of core arguments in verbal clauses, and the contrastive uses of negator \textitbf{bukang} ‘\textsc{neg}’. The observed commonalities suggest that Papuan Malay has more in common with Ambon Malay than with the other eastern Malay varieties. It is important to note, however, that these differences and commonalities could also result from gaps in the descriptions of the other eastern Malay varieties. The noted differences concerning the morphosyntactic status of the reciprocity marker and the phonological status of VV sequences most likely result from differences between the analysts rather than from linguistic differences between the compared Malay varieties.
\end{styleBodyvafter}


Overall, the noted distinctions and similarities support the conclusion put forward in §1.8 that the history of Papuan Malay is different from that of the other eastern Malay varieties, and that Ambon Malay was influential in its genesis. (See §1.8 ‘History of Papuan Malay’ for more details.)
\end{styleBodyvxvafter}

\section{Demographic information}
\label{bkm:Ref438030150}
This section presents demographic information about the Papuan Malay speakers. Speaker numbers are discussed in §1.7.1, occupation details in §1.7.2, education and literacy rates in §1.7.3, and religious affiliations in §1.7.4.
\end{styleBodyxvafter}

\subsection{Speaker numbers}
\label{bkm:Ref370485345}\label{bkm:Ref370305152}
The conservative assessment presented in this section estimates the number of Papuan Malay speakers in West Papua to be about 1,100,000 or 1,200,000.



Previous work provides different estimates for the number of people who speak Papuan Malay. With respect to first language speakers, {\citet[1]{Clouse2000}} estimates their number at 500,000. As for its uses as a language of wider communication. {Burung and \citet{Sawaki2007}}, for instance, give an estimate of one million speakers, while {\citet[71]{Paauw2009}} approximates their number at 2.2 million speakers. None of the authors provides information, however, on how they arrived at these numbers.
\end{styleBodyvafter}


The attempt here to approximate the number of Papuan Malay speakers is based on the 2010 census, conducted by the Non-Departmental Government Institution Badan Pusat Statistik (BPS-Statistics Indonesia). More specifically, the speaker estimate is based on the statistics published by the BPS-Statistics branches for Papua province and Papua Barat province.\footnote{\\
\\
Statistics from BPS-Statistics Indonesia are available at \url{http://www.bps.go.id/} (accessed 8 January 2016). Statistics for Papua province are available at \url{http://papua.bps.go.id} (accessed 8 January 2016), and statistics for Papua Barat province are available at \url{http://irjabar.bps.go.id/} (accessed 8 January 2016).\\
\\
\\
\\
\\
\\
\\
\\
\\
\\
\\
\\
\\
}
\end{styleBodyvafter}


According to the BPS-Statistics for Papua province and Papua Barat province, the total population of West Papua is 3,593,803; this includes 2,833,381 inhabitants of Papua province and 760,422 inhabitants of Papua Barat province\footnote{\\
\\
Population totals for Papua province are also available at \url{http://papua.bps.go.id/yii/9400/index.php/post/552/Jumlah Penduduk Papua} (accessed 21 Oct 2013), and for Papua Barat province at \url{http://irjabar.bps.go.id/publikasi/2011/Statistik Daerah Provinsi Papua Barat 2011/baca_publikasi.php} (accessed 21 Oct 2013).\\
\\
\\
\\
\\
\\
\\
\\
\\
\\
\\
\\
\\
} ({Bidang Neraca Wilayah dan Analisis Statistik 2011b: 11–14; 2012b: 92}). The census data does not discuss the number of Papuan Malay speakers. The (online) data does, however, give information about ethnicity (Papuan versus non-Papuan)\footnote{\\
\\
A “Papuan” is defined as someone who has at least one Papuan parent, is married to a Papuan, has been adopted into a Papuan family, or has been living in Papua for 35 years {(Bidang Neraca Wilayah dan Analisis Statistik 2011b: 11)}.\\
\\
\\
\\
\\
\\
\\
\\
\\
\\
\\
\\
\\
} by regency (for detailed population totals see Appendix D).
\end{styleBodyvafter}


The present attempt at approximating the number of Papuan Malay speakers is based on the following assumptions: (1) Papuans who live in the coastal regencies of West Papua are most likely to speak Papuan Malay, (2) Papuans living in the interior regencies are less likely to speak Papuan Malay, and (3) non-Papuans living in West Papua are less likely to speak Papuan Malay. It is acknowledged, of course, that there might be older Papuans living in remote coastal areas who do not speak Papuan Malay, that there might be Papuans living in the interior who speak Papuan Malay, and that there might be non-Papuans who speak Papuan Malay.
\end{styleBodyvafter}


For Papua province, the census data by regency and ethnicity give a total of 2,810,008 inhabitants, including 2,150,376 Papuans (76.53\%) and 659,632 non-Papuans (23.47\%), who live in its 29 regencies.\footnote{\\
\\
\label{bkm:Ref438387027}The statistics for Papua province do not give population details by regency and ethnicity per se. They do, however, include this information in providing population details by religious affiliation under the category \textstyleChItalic{Sosial Budaya} ‘Social (affairs) and Culture’; see \url{http://papua.bps.go.id/yii/9400/index.php/site/page?view=sp2010} (accessed 21 Oct 2013). By adding up the population details according to religious affiliation it is possible to arrive at overall totals by regency and ethnicity.\\
\\
\\
\\
\\
\\
\\
\\
\\
\\
\\
\\
\\
} (This total of 2,810,008 more or less matches the total given for the entire province which lists the entire population of Papua province with 2,833,381). Of the 29 regencies, 14 are essentially coastal; the remaining 15 are located in the interior.\footnote{\\
\\
Coastal regencies: Asmat, Biak Numfor, Jayapura, Kota Jayapura, Keerom, Yapen, Mamberamo Raya, Mappi, Merauke, Mimika, Nabire, Sarmi, Supiori, Waropen.\\
Interior regencies: Boven Digoel, Deiyai, Dogiyai, Intan Jaya, Jayawijaya, Lanny Jaya, Mamberamo Tengah, Nduga, Paniai, Pegunungan Bintang, Puncak, Puncak Jaya, Tolikara, Yahukimo, Yalimo.\\
\\
\\
\\
\\
\\
\\
\\
\\
\\
\\
\\
} The total population for the 14 coastal regencies is 1,364,505, which includes 756,335 Papuans and 608,170 non-Papuans. Based on the above assumptions that Papuans living in coastal areas can speak Papuan Malay, and that non-Papuans are less likely to speak it, the number of Papuan Malay speakers living in Papua province is estimated at 760,000 speakers.
\end{styleBodyvafter}


For Papua Barat province, the census data by regency and ethnicity gives a total of 760,422 inhabitants, including 405,074 Papuans (53.27\%) and 355,348 non-Papuans (46.73\%) living in its 11 regencies.\footnote{\\
\\
\\
Papua Barat regencies: Fakfak, Kaimana, Kota Sorong, Manokwari, Maybrat, Raja Ampat, Sorong, Sorong Selatan, Tambrauw, Teluk Bintuni, and Teluk Wondama.\\
\\
\\
\\
\\
\\
\\
\\
\\
\\
\\
\\
} Ten of its regencies are essentially coastal; the exception is Maybrat, which is located in the interior. The total population for the ten regencies is 727,341, including 373,302 Papuans and 354,039 non-Papuans. Based on the above assumptions, the number of Papuan Malay speakers living in Papua Barat province is estimated with 380,000 speakers. {(Bidang Neraca Wilayah dan Analisis Statistik 2011b: 11–14)}
\end{styleBodyvafter}


These findings give a total of between 1,100,000 to 1,200,000 potential speakers of Papuan Malay. This estimate is conservative, as people living in the interior are excluded. Moreover, non-Papuans are excluded from this total. However, the results of a sociolinguistic survey carried out in 2007 by the Papuan branch of SIL Indonesia in several costal regencies indicate “substantive use” of Papuan Malay by “non-Papuan residents of the region” {(Scott et al. 2008: 11)}.
\end{styleBodyvafter}


The population estimate presented here does not make any statements about the potential number of first language Papuan Malay speakers. The results of the 2007 survey indicate, however, that large numbers of children learn Papuan Malay at home: all of the 14 interviewed focus groups stated that Papuan Malay is spoken in their region; moreover, 70\% of the focus groups indicated that Papuan Malay is “the first language children learn in the home as well as the language most commonly used in their region” {(Scott et al. 2008: 11)}.
\end{styleBodyvxvafter}

\subsection{Occupation details}
\label{bkm:Ref373169806}\label{bkm:Ref370286929}
Most of West Papua’s population works in the agricultural sector: 70\% in Papua province, and 54\% in Papua Barat province. As subsistence farmers, they typically grow bananas, sago, taro, and yams in the lowlands, and sweet potatoes in the highlands; pig husbandry, fishing, and forestry are also widespread. The second most important domain is the public service sector. In Papua province, 10\% of the population works in this sector, and 17\% in Papua Barat province. Furthermore, 9\% in Papua province and 12\% in Papua Barat province work in the commerce sector. Other minor sectors are transport, construction, industry, and communications. {(Encyclopædia Britannica Inc. 2001a-; 2001b-; Bidang Neraca Wilayah dan Analisis Statistik 2011b: 21; 2012a: 12; see also Bidang Neraca Wilayah dan Analisis Statistik 2012b: 83)}.



The census data does not provide information about occupation by ethnicity. However, the author made the following observations for the areas of Sarmi and Jayapura (see Figure  ‎0 .2 on p. \pageref{bkm:Ref373252478} and Figure  ‎0 .3 on p. \pageref{bkm:Ref375721640}). Papuans typically work in the agricultural sector; those living in coastal areas are also involved in small-scale fishing. Those with a secondary education degree usually (try to find) work in the public sector. The income generating commerce and transportation sectors, by contrast, are in the hands of non-Papuans. This assessment is also shared by {\citet[124]{Chauvel2002}} who maintains that “Indonesian settlers dominate the economy of [West] Papua”. The author does not provide details about the origins of these settlers. Given Indonesia’s \textstyleChItalic{transmigration} program, however, it can be assumed that most, or at least substantial numbers, of these settlers originate from the overcrowded islands of Java, Madura, Bali, and/or Lombok. Moreover, substantial numbers of active and retired military personnel have settled in West Papua.\footnote{\\
\\
\\
\label{bkm:Ref438386876}\textstyleChItalic{Transmigration} is a program by the Indonesian government to resettle millions of inhabitants. Coming from the overcrowded islands of Java, Madura, Bali, and Lombok, they settle in the less populated areas of the archipelago, such as West Papua. The first transmigration project was launched in 1905 {\citep[553]{Fearnside1997}}. During the second World War, the project was put on hold, “until the current transmigration program was launched in 1950” {(1997: 554)}. Between 1905 and 1989, “a cumulative total of approximately […] five million people […] had been shipped to the outer islands as part of the official program, plus anywhere from two to three times this many had moved independent of the program” ({1997: 554; }see also {Embassy of the Republic of Indonesia in London 2009}).\\
\\
\\
\\
\\
\\
\\
\\
\\
\\
\\
\\
} (See {Fearnside 1997; Embassy of the Republic of Indonesia in London 2009}.)
\end{styleBodyvxvafter}

\subsection{Education and literacy rates}
\label{bkm:Ref370323231}
The 2010 census data provides information about school enrollment and literacy rates in Standard Indonesian. In West Papua, most children attend school. For older teenagers and young adults, however, the rates of those who are still enrolled in a formal education program are much lower. Literacy rates for the adult population aged 45 years or older are lower than the rates for the younger population. Overall, education and literacy rates are (much) lower for Papua province than for Papua Barat province. Details are given in Table  ‎1 .11 to Table  ‎1 .13.



Most children under the age of 15 go to school, as shown in Table  ‎1 .11. However, the data also indicates that this rate is much lower for Papua province than for Papua Barat province. The number of teenagers aged between 16-18 who are still enrolled in school, is much lower for both provinces, again with Papua province having the lower rate. As for young adults who are still enrolled in a formal education program, the rate is even lower, at less than 15\%. The data in Table  ‎1 .11 gives no information about the school types involved. That is, these figures also include children and teenagers who are enrolled in a school type that is not typical for their age group. (For enrollment figures by school types see Table  ‎1 .12.)\footnote{\\
\\
\\
The school participation rates by school types in Table  ‎1 .11 are available at \textstyleChUnderl{http://www.bps.go.id/eng/tab\_sub/view.php?kat=1\& tabel=1\& daftar=1\& id\_subyek=28\& notab=3} (accessed 21 Oct 2013).\\
\\
\\
\\
\\
\\
\\
\\
\\
\\
\\
\\
}
\end{styleBodyvvafter}

\begin{stylecaption}
\label{bkm:Ref370284850}Table ‎1.\stepcounter{Table}{\theTable}:  Formal education participation rates by age groups
\end{stylecaption}

\begin{tabular}{lllll}
\lsptoprule

 Province & 7-12 & 13-15 & 16-18 & \arraybslash 19-24\\
Papua & \raggedleft 76.22\% & \raggedleft 74.35\% & \raggedleft 48.28\% & \raggedleft\arraybslash 13.18\%\\
Papua Barat & \raggedleft 94.43\% & \raggedleft 90.25\% & \raggedleft 60.12\% & \raggedleft\arraybslash 14.66\%\\
\lspbottomrule
\end{tabular}

The 2010 census data also show that most children get a primary school education (76.22\% in Papua province, and 92.29\% in Papua Barat province). Enrollment figures for junior high school are considerably lower with only about half of the children and teenagers being enrolled. Figures for senior high school enrollment are even lower, at less than 50\%. The data in Table  ‎1 .12 also shows that overall Papua Barat province has higher enrollment rates than Papua province, especially for primary schools.\footnote{\\
\\
\\
The enrollment rates by school types in Table  ‎1 .12 are available at \url{http://www.bps.go.id/eng/tab_sub/view.php?kat=1 & tabel=1 & daftar=1 & id_subyek=28 & notab=4} (accessed 21 Oct 2013).\\
\\
\\
\\
\\
\\
\\
\\
\\
\\
\\
\\
}


\begin{stylecaption}
\label{bkm:Ref370284179}Table ‎1.\stepcounter{Table}{\theTable}:  School enrollment rates by school type
\end{stylecaption}

\begin{tabular}{llll}
\lsptoprule

 Province & Primary & Junior high & \arraybslash Senior high\\
Papua & \raggedleft 76.22\% & \raggedleft 49.62\% & \raggedleft\arraybslash 36.06\%\\
Papua Barat & \raggedleft 92.29\% & \raggedleft 50.10\% & \raggedleft\arraybslash 44.75\%\\
\lspbottomrule
\end{tabular}

Literacy rates in 2010 differ considerably between the populations of both provinces. In Papua province only about three quarters of the population is literate, while this rate is above 90\% for Papua Barat province, as shown in Table  ‎1 .13. In Papua province, the literacy rates are especially low in the Mamberamo area, in the highlands, and along the south coast {(Bidang Neraca Wilayah dan Analisis Statistik 2011a: 27–30)}.\footnote{\\
\\
\\
The literacy rates in Table  ‎1 .13 are available at \url{http://www.bps.go.id/eng/tab_sub/view.php?kat=1 & tabel=1 & daftar=1 & id_subyek=28 & notab=2} (accessed 21 Oct 2013).\\
\\
\\
\\
\\
\\
\\
\\
\\
\\
\\
\\
}


\begin{stylecaption}
\label{bkm:Ref370283437}Table ‎1.\stepcounter{Table}{\theTable}:  Illiteracy rates by age groups
\end{stylecaption}

\begin{tabular}{llll}
\lsptoprule

 Province & {\textless}15 & 15-44 & \arraybslash 45+\\
Papua & \raggedleft 31.73\% & \raggedleft 30.73\% & \raggedleft\arraybslash 36.14\%\\
Papua Barat & \raggedleft 4.88\% & \raggedleft 3.34\% & \raggedleft\arraybslash 9.91\%\\
\lspbottomrule
\end{tabular}

The census data provides no information about education and literacy rates according to rural versus urban regions. The author assumes, however, that education and literacy rates are lower in rural than in urban areas. The census data also does not include information about education and literacy rates by ethnicity. As mentioned in §1.7.2, the author has the impression that Papuans typically work in the agriculture sector while non-Papuans are more often found in the income generating commerce and transportation sectors. This, in turn, gives non-Papuans better access to formal education, as they are in a better position to pay tuition fees.


\subsection{Religious affiliations}
\label{bkm:Ref370323232}
West Papua is predominantly Christian. For most Papuans their Christian faith is a significant part of their Papuan identity. It distinguishes them from the Muslim Indonesians who have come from Java, Madura, and Lombok and settled in West Papua, as a result of Indonesia’s transmigration program (see Footnote 43 in §1.7.2, p. \pageref{bkm:Ref438386876}).



Papua province has 2,810,008 inhabitants, including 2,150,376 Papuans and 659,632 non-Papuans. Almost all Papuans are Christians (2,139,208 = 99.48\%), while only 10,759 are Muslims (0.05\%); the remaining 0.02\% has other religious affiliations. Of the 659,632 non-Papuans, two thirds are Muslims (439,337 = 66.60\%), while one third are Christians (216,582 = 32.83\%); the remaining 0.57\% has other religious affiliations.\footnote{\\
\\
\\
Detailed data by regency are available under the category \textstyleChItalic{Sosial Budaya} ‘Social (affairs) and Culture’ at \url{http://papua.bps.go.id/yii/9400/index.php/site/page?view=sp2010} (accessed 21 Oct 2013).\\
\\
\\
\\
\\
\\
\\
\\
\\
\\
\\
\\
}
\end{styleBodyvafter}


Papua Barat province has 760,422 inhabitants, including 405,074 Papuans and 355,348 non Papuans. For Papua Barat province, no census data is published by ethnicity and religion. Based on the data given in {Bidang Neraca Wilayah dan Analisis Statistik (2011b: 11–14)}, however, the following picture emerges: most Papuans are Christians (352,171 = 86.94\%), while 52,903 are Muslims (13.06\%), most of whom live in the Fakfak regency. Of the 355,348 non-Papuans, about two thirds are Muslims (239,099 = 67.29\%) and one third are Christians (110,166 = 31.00\%); the remaining 1.71\% have other religious affiliations.
\end{styleBodyvxvafter}

\section{History of Papuan Malay}
\label{bkm:Ref438030300}
Papuan Malay is a rather young language. It only developed over approximately the last 130 years, unlike other Malay languages in the larger region. As discussed in this section, though, the precise origins of Papuan Malay remain unclear. That is, it is not known exactly which Malay varieties had which amount of influence in which regions of West Papua in the formation of Papuan Malay.



Malay has a long history as a trade language across the Malay Peninsula and the Indonesian archipelago. The language spread to the Moluccas through extensive trading networks. It was already firmly established there before the arrival of the first Europeans in the sixteenth century. (See Adelaar and Prentice 1996{;} Collins 1998{;} {Paauw 2009: 42–79}.) From the Moluccas, Malay spread to West Papua where it developed into today’s Papuan Malay.
\end{styleBodyvafter}


The southwestern part of West Papua was under the influence of the island of Seram in the central Moluccas, with trade relationships firmly established from about the fourteenth century, long before the first Europeans arrived. A special lingua franca, called Onin, was used in the context of these trade relations. Onin was “a mixture of Malay and local languages spoken along the coasts of the Bomberai Peninsula” {\citep[1]{Goodman2002}}. Unfortunately, {Goodman} does not discuss the relationship between Onin and Malay in more detail. It is noted, though, that today Malay is spoken in Fakfak, the main urban center on the Bomberai Peninsula, as well as in the areas around Sorong and Kaimana. According to {Donohue}{ }{(2003: 2)}, the Malay spoken in these areas “is essentially a variety of Ambon Malay” (see also {Walker 1982}).
\end{styleBodyvafter}


The Bird’s Head and Geelvink Bay, now Cenderawasih Bay, were under the authority of the Sultanate of Tidore. The first mention of Tidore’s authority over this part of West Papua dates back to 15 January 1710 and can be found in the \textstyleChItalic{Memorie van Overgave} ‘Memorandum of Transfer’ by the outgoing Governor of Ternate Jacob Claaszoon. In summarizing this memorandum,\footnote{\\
\\
\\
While {Haga (1884: 192–195)} gives no further bibliographical details for this memorandum, the following details are found in {\citet[262]{Andaya1993}}: VOC 1794. Memorie van overgave, Jacob Claaszoon, 14 July 1710, fols 55-56.\\
\\
\\
\\
\\
\\
\\
\\
\\
\\
\\
\\
} {Haga (1884: 192–195)} lists the locations on New Guinea’s coast which belonged to Tidore’s territory. Included in this list is the west coast of Geelvink Bay, with {Haga} pointing out that Tidore also claimed authority over Geelvink Bay’s south coast. In the second half of the nineteenth century, however, Tidore’s authority over Geelvink Bay declined after the Dutch banned Tidore’s raiding expeditions to New Guinea on 22 February 1861 {(Bosch 1995: 28–29)}. Roughly 35 years later, in 1895, the outgoing Resident of Ternate, J. van Oldenborgh noted that, due to this ban, Tidore’s authority on New Guinea had been reduced to zero as the sultans no longer had the means to enforce their authority in this area {(van Oldenborgh 1995: 81)}. In 1905, the last sultan of Tidore, Johar Mulki (1894-1905), relinquished all rights to western New Guinea to the Dutch (van der Eng 2004; see also {Overweel 1995: 138}).
\end{styleBodyvafter}


Due to Tidorese influence in the eighteenth and nineteenth centuries, the Bird’s Head and Geelvink Bay were firmly connected with the wider Moluccan trade network (Seiler 1982: 72{;}{ }{van Velzen 1995: 314–315;}{ }{Timmer 2002: 2–3}; see also Huizinga 1998 on the relations between Tidore and New Guinea’s north coast in the nineteenth century). However, scholars disagree on how firmly Malay was established in this area, especially in Geelvink Bay, during these early trading relations.
\end{styleBodyvafter}


{Rowley}{ }{(1972: 53)}, for instance, suggests that the Malay presence along West Papua’s western coast may date back to the fourteenth century. Malay influence began with Javanese trading settlements and then continued with trading settlements which were under the control of Seram and Tidore. At that time, the Dutch did not yet show any direct interest in this region. It was the British who, in 1793, established the first European post at Dorey, now Manokwari, which they maintained for two years. During this period Dorey was already under the influence of Tidore and its inhabitants had to pay an annual tribute to the Tidore sultan. {Van Velzen}{ }{(1995: 314–315)} also claims that Malay was a regional language of wider communication long before the arrival of the first Europeans is. He refers to {Haga’s (1885)} account of one of the first European visits to the Yapen Waropen area, which took place in 1705. On Yapen Island the crew was able to communicate in Malay with some of the local inhabitants. Given that these inhabitants were ethnically Biak, {van \citet{Velzen1995}} concludes that it may have been the Biak who first introduced Malay to Geelvink Bay.\footnote{\\
\\
\\
Along similar lines {\citet[3]{Samaun1979} }states that Malay, namely Ambon or Ternate Malay, “was long ago introduced” in West Papua. The author does not, however, provide a more precise date, instead maintaining that Malay has been used in West Papua “for more than a century” {(1979: 3)}.\\
\\
\\
\\
\\
\\
\\
\\
\\
\\
\\
\\
}
\end{styleBodyvafter}


This claim of the long-standing presence of Malay in the Geelvink Bay is not, however, supported by the reports of explorers who visited the Geelvink area in the nineteenth century. These early visits occurred after the Dutch had first shown interest in this region. This was only in 1820, after the British had established their post at Dorey in 1793; this first Dutch interest “was due in part to the fear that other attempts would be made” {\citep[53]{Rowley1972}}.
\end{styleBodyvafter}


For instance, when the French explorer and rear admiral {Dumont d‘Urville}{ }{(1833: 606)} stayed in Dorey (Manokwari) in September 1827, he noted that the Papuans, who formed the majority of inhabitants in Dorey, hardly knew any Malay; only the upper-class of Dorey spoke Malay more or less fluently. A similar statement about the Papuans’ abilities to speak Malay comes from {van Hasselt}{ }{(1936)}. He reports how the first missionaries to West Papua, the Germans \textstylest{Ottow and }Geissler, together with his father van Hasselt and the Dutch researcher Croockewit attempted to learn and study the local language after they had arrived in Geelvink Bay in 1858. The author notes that it was very difficult for them to learn the local language, as the Papuans knew little or no Malay {(1936: 116)}. Along similar lines, the British naturalist {Wallace}{ (1890: 380)} relates that, when he came to Dorey (Manokwari) in 1858, the local Papuans could not speak any Malay.
\end{styleBodyvafter}


Based on these reports, it can be concluded that in the early eighteen hundreds Malay was not yet well established in Geelvink, including the area in and around today’s Manokwari. Hence, the author agrees with {Seiler}{ }{(1982: 73)}, who comes to the conclusion that, in light of accounts such as the one by {Dumont d‘\citet{Urville1833}},
\end{styleBodyvvafter}

\begin{styleIvI}
[t]here is no reason to assume that Malay was better known at other places along New Guinea’s north coast; Manokwari was one of the most visited places in the area and if anything, Malay should have been known to a larger extent there than anywhere else.
\end{styleIvI}


The history of Malay along West Papua’s north and northeast coast is also disputed among scholars.



{Rowley}{ }{(1972: 56–57)} states that “Malay adventurers” went eastwards to the Sepik area “in expeditions for birds of paradise”. Even long before the nineteenth century, Malay traders made sporadic visits to the northeastern coasts of New Guinea and the Bismarck Archipelago. Hence, {Rowley} concludes that Malay influence along West Papua’s north and northeast coast began long before the Dutch started taking an interest this area.
\end{styleBodyvafter}


The Danish anthropologist {Parkinson}{ }{(1900)} came to a similar conclusion after having visited the north coast of today’s Papua New Guinea. Based on his acquaintanceship with some Malay-speaking inhabitants, Malay artifacts, and some inherited Malay words, the explorer concludes that Malay seafarers from the East India islands have undertaken trips along the coast of New Guinea “for a long time” {(1900: 20–21)}.
\end{styleBodyvafter}


This conclusion is not supported, however, by the observations of other European explorers who visited West Papua’s northeast coast in the nineteenth century after the Dutch had annexed the western part of New Guinea in 1828.\footnote{\\
\\
\\
In 1828, the Dutch annexed today’s West Papua as far as 141 degrees of east longitude (today’s border with Papua New Guinea) {\citep[509]{Burke1831}}.\\
\\
\\
\\
\\
\\
\\
\\
\\
\\
\\
\\
}
\end{styleBodyvafter}


Twenty years after this annexation, in 1848, the Dutch laid formal claim on West Papua’s north coast, including Humboldt Bay in the east, now Yos Sudarso Bay with the provincial capital Jayapura {\citep[56]{Rowley1972}}. In 1850, the Dutch sent a first expedition fleet eastwards to mark their claim; this expedition included Sultanese boats and a number of pirate boats. The fleet did not, however, reach Humboldt Bay, although the Cyclops Mountains were in sight. Two years later, though, the Dutch were able to establish a garrison in Humboldt Bay; the troops were from Ternate. However, it seems that this garrison did not include any Europeans, because, according to {Seiler}{ }{(1982: 74)}, it was only in the course of the “Etna expedition” in 1858 that the Dutch first reached Humboldt Bay. The report of this expedition states that the Papuans living in Humboldt Bay did not know any Malay and had had no contact with the outside world {(Commissie voor Nieuw Guinea et al. 1862: 182–183)}.
\end{styleBodyvafter}


Twenty years later it was still not possible to communicate in Malay with the Papuans of Humboldt Bay. {Robidé van der Aa}{ }{(1879: 127–129)}, for instance, reported that when the \textstylest{Government commissioner }van der Crab visited Humboldt Bay in 1871, his interpreter could not communicate with the local population because of their very poor Malay. The commissioner also noted that outside trading in this area was very limited due to tense relations between the Papuan population and outside traders and due to the wild sea.
\end{styleBodyvafter}


Around this time, however, outside trading between the Moluccas and West Papua’s northeast coast, including Humboldt Bay and the areas to its east, started to take off. As a result of this increase in outside contacts, knowledge of Malay, especially of the North Moluccan varieties, also started to spread rapidly in this region. {Seiler (1982; 1985}) gives an overview of these developments, citing government officials, merchants, and missionaries who visited West Papua’s northeast coast in the late nineteenth century.
\end{styleBodyvafter}


One of them is the Protestant missionary {\citet{Bink1894}}. In 1893, about twenty years after van der Crab’s 1871 visit to this area, {Bink} travelled to Humboldt Bay. In his report he noted the presence of Malay traders from Ternate who were shooting birds of paradise in the area {(1894: 325)}. Another observer is the German geologist {Wichmann}{ }{(1917)}. In 1903, he travelled to Humboldt Bay and Jautefa Bay, where today’s Abepura is located. {Wichmann} reported the presence of Malay traders who were living on Metu Debi Island in Jautefa Bay {(1917: 150)}. A third observer is {van Hasselt}{ }{(1926)}. When he visited Jamna Island (located off the northeast coast between Sarmi and Jayapura) in 1911, he noted that several Papuans could already speak Malay, because they had been in regular contact with traders {(1926: 134)}.
\end{styleBodyvafter}


Based on the reports of these observers, {Seiler}{ }{(1985: 147)} comes to the following conclusion:
\end{styleBodyvvafter}

\begin{styleIvI}
It would appear that Malays started regular trading visits to areas east of Geelvink Bay sometime after the middle of the 19th century, at the same time as the Dutch began to explore their long-forgotten colony. This was just prior to the beginning of the German activities in the area. Twenty years or so of contact between the local people and Malays could easily account for the knowledge of Malay on the part of the coastal people.
\end{styleIvI}


In the early twentieth century, the use of Malay throughout West Papua increased when the Dutch decided to increase their influence in this area and to enforce the use of Malay in the domains of education, administration, and proselytization. A major resource for these efforts was the Malay-language school system already established in the Moluccas. It provided the Dutch with the personnel necessary for bringing the population and the resources of West Papua under their control {\citep[64]{Collins1998}}. Therefore West Papua saw a constant influx of Ambon Malay speaking teachers, clerks, police, and preachers during this period {(Donohue and Sawaki 2007: 254–255)}. This link between West Papua and Ambon was especially close, as until 1947 West Papua was part of the Moluccan administration, which had its capital in Ambon. So Ambon Malay played an important role in the genesis of Papuan Malay, as well as North Moluccan Malay.



After World War II, the Dutch government recruited additional personnel for West Papua from other areas, such as North Sulawesi, Flores, Timor, and the Kei Islands. In addition, fishermen and traders from Sulawesi and, to some extent, from \textstylest{East Nusa Tenggara came to }West Papua\textstylest{. }({Roosman 1982: 96; Adelaar and Prentice 1996: 682; Donohue and Sawaki 2007: 254–255}) At the same time, increasing numbers of Papuans received a primary school education. Furthermore, the Dutch established schools to train Papuans for public services. As a result, more and more Papuans become government officials, teachers, and police officers. During this period, Standard Malay was the official language in public domains, including trade and the religious domain. ({Chauvel 2002: 120; Donohue and Sawaki 2007: 255; see also Adelaar 2001: 234.}) Outside the coastal urban centers, however, Malay played only a very limited role. This is evidenced by that fact that along West Papua’s north coast Papuan Malay is still “restricted to a coastal fringe, and does not extend inland to any great extent except where agricultural projects were in force” {(Donohue and Sawaki 2007: 255)}.
\end{styleBodyvafter}


After Indonesia annexed West Papua in 1963, Standard Indonesian became the official language of West Papua. It is used in all public domains, including primary school education, the mass media, and the religious domain.
\end{styleBodyvafter}


West Papua’s Malay, by contrast, is not recognized as a language in its own right vis-à-vis Indonesian (for details on the sociolinguistic profile of Papuan Malay, see §1.5). Only recently has Papuan Malay received attention from linguistics as an independent language (for details see §1.9). Materials in Papuan Malay are equally recent (for details see §1.10).
\end{styleBodyvafter}


In speaking about Papuan Malay and its history and genesis one aspect needs to be highlighted, however. As {Paauw}{ }{(2009: 73)} points out,
\end{styleBodyvvafter}

\begin{styleIvI}
there is linguistic evidence that both North Moluccan Malay (on the north and east coasts of the Bird’s Head and in parts of Cendrawasih Bay, including the islands of Biak and Numfoor) and Ambon Malay (in the western and southern Bird’s Head, the Bomberai peninsula, and in other parts of Cendrawasih Bay, including the island of Yapen) have been influential.
\end{styleIvI}


It is still unknown, though, exactly how much influence each variety had in the various regions of West Papua. Overall, however, regional differences in the usage of Papuan Malay across the language area seem to be minor, as discussed in §1.3.



The developments described in this section show that the history of Papuan Malay is quite distinct from that of other eastern Malay varieties. Other eastern Malay varieties were already well established before the first Europeans arrived in these areas in the sixteenth century. This applies to Ambon and North Moluccan Malay, both of which contributed to Papuan Malay. It also applies to Manado Malay, which apparently developed out of North Moluccan Malay. Likewise, it applies to Kupang Malay. ({Paauw 2009: 42–79}; see also {Adelaar and Prentice 1996}{;}{ }{Collins 1998}.) Papuan Malay, by contrast, only developed over the last 130 years or so.
\end{styleBodyvxvafter}

\section{Previous research on Papuan Malay}
\label{bkm:Ref438028967}
Until the second half of the twentieth century, the Malay varieties spoken in New Guinea had received almost no attention. Linguists only started taking more notice of the language in the second half of the twentieth century. An overview of these early studies is given in §1.9.1. More recent studies, starting from the early years of the twenty first century, are discussed in §1.9.2. In addition, Papuan Malay has received attention in the context of sociolinguistic and socio-historical studies (§1.9.3).
\end{styleBodyxvafter}

\subsection{Early linguistic studies on the Malay varieties of West Papua}
\label{bkm:Ref368904450}
{Zöller (1891)} mentions Malay in his description of the \textstyleChItalic{Papua Sprachen} ‘languages of Papua’ {(1891: 351–426)}, as well as in his 300-item word list of 48 languages of Papua {(1891: 443–529)}; the 48 languages include 29 languages of German New Guinea, and 17 languages of British New Guinea, as well as Malay and Numfor of Netherlands New Guinea (for comparative reasons, the word list also includes Maori and Samoan, besides the 48 languages of Papua).



Likewise,{ \citet{Teutscher1954}} mentions Malay in his article on the languages spoken in New Guinea. As a lingua franca it is used in formal and informal domains. Moreover, for Papuans this Malay has become a \textstyleChItalic{tweede moedertaal} ‘second mother tongue’ {(1954: 123)}.
\end{styleBodyvafter}


Also available is a \textstyleChItalic{Beknopte leergang Maleis voor Nieuw-Guinea} ‘A concise language course in the Malay variety spoken in New Guinea’ {(Bureau Cursussen en Vertalingen 1950)}.
\end{styleBodyvafter}


The Malay of New Guinea is also mentioned by {Anceaux and Veldkamp} in their Malay-Dutch-Dani word list {(1960)} as well as in their penciled New Guinea Malay-Dutch word list {(no date)}.
\end{styleBodyvafter}


In addition, {\citet[49]{Teeuw1961}} states that after 1950 a variety of publications were produced specifically for western New Guinea; they were written in Malay with a “distinctly local colour”. At the same time, however, the author notes that there were no publications which discussed the Malay of Netherlands New Guinea or the language policies regarding this Malay variety.
\end{styleBodyvafter}


Around the same time, {\citet{Moeliono1963}} mentions Indonesian in his study of the languages spoken in West Papua. The author refers to the language as a \textstyleChItalic{logat bahasa Indonesia} ‘speech variety of the Indonesian language’ without, however, discussing its features. The author does state, though, that this “dialect” is spoken in the coastal and urban areas of West Papua and used by the Dutch colonial government for letters and announcements. Moreover, it is used as a lingua franca, both in formal and informal domains.
\end{styleBodyvafter}


Early linguistic studies on the Malay varieties spoken in West Papua date back to the second half of the twentieth century.
\end{styleBodyvafter}


{\citet{Samaun1979}} highlights some morphological, syntactical, and lexical features in which the \textstyleChItalic{dialek Indonesia Irian} ‘Irian Indonesian dialect’ of Jayapura differs from Standard Indonesian. While explaining these differences as mere simplifications, the author also notes that due to some of these modifications, this \textstyleChItalic{dialek} of Indonesian sounds non-Indonesian.
\end{styleBodyvafter}


Along similar lines, {Suharno (1979; 1981}) describes some aspects of Papuan Malay phonology, morphology, lexicon, and grammar in comparison to Standard Indonesian. While referring to Papuan Malay as an Indonesian dialect, the author suggests that this variety of Indonesian is autonomous and deserves more research. The author also maintains that this dialect is a suitable language for development programs. In formal situations, however, the language variety is unacceptable.
\end{styleBodyvafter}


Unlike {\citet{Samaun1979}} and {Suharno} ({1979; 1981}), {\citet{Roosman1982}} does not refer to Papuan Malay as a dialect of Indonesian. Instead, he considers Papuan Malay as a form of Ambon Malay which has “pidgin Malay as its basic stratum” {(1982: 1)}. In his paper, the author presents phonetic inventories of Ambon Malay (Irian Malay), Pidgin Malay, and Indonesian and comments on some of the differences he found.
\end{styleBodyvafter}


Another scholar who mentions various features of the Malay spoken in West Papua is {\citet{Walker1982}}. In the context of his study on language use at Namatota, a village located on West Papua’s southwest coast, the author discusses some of the similarities which Malay shares with Indonesian and some of the distinctions between both languages.
\end{styleBodyvafter}


{\citet{Ajamiseba1984}} mentions the Malay variety spoken in West Papua in the context of his study on the linguistic diversity found in this part of New Guinea. Referring to this speech variety as \textstyleChItalic{Irian Indonesian}, the author compares some of its features to those of other languages spoken in West Papua. This comparison, however, seems to be based on Standard Indonesian rather than on Papuan Malay.
\end{styleBodyvafter}


In 1995, {van Velzen} published his “Notes on the variety of Malay used in Serui and vicinity”. Similar to previous studies, the author highlights some aspects of Serui Malay in comparison to Standard Indonesian. Based on phonological, morphological, and lexical features, {van \citet[315]{Velzen1995}} concludes that Serui Malay and the other Malay varieties of West Papua’s north coast “are probably more closely related to Tidorese or Ternatan Malay” than to Ambon Malay, as suggested by {\citet{Roosman1982}}.\footnote{\\
\\
\\
With respect to this quote, {R.J. Nivens (p.c. 2013)} suggests that {van \citet[315]{Velzen1995}} made this comment “because the sultan of Tidore once claimed sovereignty over parts of Papua”, but it is doubtful “that he had any actual linguistic data to back up this claim”.\\
\\
\\
\\
\\
\\
\\
\\
\\
\\
\\
\\
}
\end{styleBodyvxvafter}

\subsection{Recent linguistic descriptions of Papuan Malay}
\label{bkm:Ref368904451}
More recently, Papuan Malay has received attention from linguistics as a language in its own right vis-à-vis the other eastern Malay varieties as well as vis-à-vis Indonesian. Three studies give an overview of the most pertinent features of Papuan Malay: {\citet{Donohue2003}}, {\citet{Paauw2009}}, and {\citet{ScottEtAl2008}}.



{\citet{Donohue2003}} discusses various linguistic features of Papuan Malay as spoken in the area around Geelvink Bay. The described features include, among others, phonology, noun phrases, verbal morphosyntax, and clause linkages.
\end{styleBodyvafter}


In the context of his typological study of seven eastern Malay varieties, {\citet{Paauw2009}} compares Papuan Malay with Ambon, Banda, Kupang, Larantuka, Manado, and North Moluccan Malay.\footnote{\\
\\
\\
The basis for the description of Papuan Malay is textual data collected in Manokwari {\citep[35]{Paauw2009}}, as well as data available in previous studies: {\citet{Suharno1981}}; {Van \citet{Velzen1995};} {\citet{Donohue2003}};{ }{Burung and \citet{Sawaki2007}}; {\citet{KimEtAl2007}} (this study is an earlier version of {Scott et al. 2008}); {\citet{Sawaki2007}}.\\
\\
\\
\\
\\
\\
\\
\\
\\
\\
\\
\\
} The described features include phonology, lexical categories, word order, clause structure, noun phrases, prepositional phrases, and verb phrases.
\end{styleBodyvafter}


{Scott et al.’s (2008)} study is part of a larger sociolinguistics language survey of the Papuan Malay varieties of West Papua (see §1.9.3). The authors describe different aspects of the lexicon, phonology, morphology, syntax, and discourse of Papuan Malay as spoken in (and around) the urban areas of Fakfak, Jayapura, Manokwari, Merauke, Timika, Serui, and Sorong (see also Figure  ‎0 .2 on p. \pageref{bkm:Ref373252478}).
\end{styleBodyvafter}


In addition, there are a number of studies which explore specific aspects of Papuan Malay.
\end{styleBodyvafter}


One of the investigated features is the personal pronoun system. {Donohue and \citet{Sawaki2007}} examine the innovative forms and functions of the pronoun system in Papuan Malay as spoken along West Papua’s north coast. In their study on the development of Austronesian first-person pronouns, {Donohue and \citet{Smith1998}} explore the loss of the inclusive-exclusive distinction in non-singular personal pronouns in Papuan Malay as spoken in Serui and Merauke, as well as in other nonstandard Malay varieties. {\citet{Saragih2012}} investigates the use of person reference in everyday language on the social networking service Facebook.
\end{styleBodyvafter}


Besides the personal pronoun system, the voice system – that is to say, the lack thereof – has also received attention. {\citet{Donohue2007a}} investigates the variation in the voice systems of six different Indonesian/Malay varieties, including Papuan Malay as spoken in the areas around Jayapura and Serui (see also {Donohue 2005b;} {2007b}).\footnote{\\
\\
\\
{\citet{Donohue2007a} refers to Papuan Malay as spoken in the area of Serui as ‘Serui Malay’.}\\
\\
\\
\\
\\
\\
\\
\\
\\
\\
\\
\\
}
\end{styleBodyvafter}


In a more recent study on the Melanesian influence on Papuan Malay, {\citet{Donohue2011}} investigates pronominal agreement, aspect marking, serial verb constructions, and various aspects of clause linkage in Papuan Malay.
\end{styleBodyvafter}


In addition to these more in-depth studies on Papuan Malay, initial research has been conducted on a variety of different topics. {\citet{Burung2004}} examines comparative constructions in Papuan Malay. {\citet{Burung2005}} discusses three types of textual continuity, namely topic, action, and thematic continuity. {Burung and \citet{Sawaki2007}} describe different types of causative constructions. {\citet{Burung2008a}} presents a brief typological profile of Papuan Malay. {\citet{Burung2008b}} investigates how Papuan Malay expresses the semantic prime FEEL, applying the Natural Semantic Metalanguage (NSM) framework. {\citet{Lumi2007}} investigates similarities and differences of the plural personal pronouns in Ambon, Manado, and Papuan Malay. {\citet{Sawaki2004}} discusses serial-verb constructions and word order in different clause types, and gives an overview of the pronominal system. {\citet{Sawaki2007}} investigates how Papuan Malay expresses passive voice. {\citet{Warami2005}} examines the uses of a number of different lexical items, including selected interjections and conjunctions.
\end{styleBodyvafter}


Other materials on Papuan Malay mentioned in the literature but not consulted by the author are the following (listed in alphabetical order): {Donohue’s (1997)} study on contact and change in Papuan Malay as spoken in Merauke,\footnote{\\
\\
\\
{\citet{Donohue1997} refers to Papuan Malay as spoken in the Merauke area as ‘Merauke Malay’.}\\
\\
\\
\\
\\
\\
\\
\\
\\
\\
\\
\\
} {Hartanti’s }{(2008)} analysis of SMS texts in Papuan Malay, {Mundhenk’s (2002)} description of final particles in Papuan Malay, {Podungge’s (2000)} description of slang in Papuan Malay, {Sawaki’s (2005a)} paper on nominal agreement in Papuan Malay, {Sawaki’s (2005b)} paper \textstyleChItalic{Melayu Papua: Tong Pu Bahasa}, and {Silzer’s (1978; 1979}) \textstyleChItalic{Notes on Irianese Indonesian}.
\end{styleBodyvxvafter}

\subsection{Sociolinguistic and socio-historical studies}
\label{bkm:Ref368904453}
To date, sociolinguistic studies on Papuan Malay are scarce.



The earliest one is {Walker’s (1982)} study on language use at Namatota, mentioned in §1.9.2. Examining the different functions Malay and other languages have in this multilingual community, the author highlights the pervasive role of Malay in the community.
\end{styleBodyvafter}


A more recent study is the sociolinguistic survey mentioned in §1.3, §1.5, and §1.9.2, which the Papuan branch of SIL International carried out in (and around) the coastal urban areas of Fakfak, Jayapura, Manokwari, Merauke, Timika, Serui, and Sorong {\citep{ScottEtAl2008}}. In the context of this study, sociolinguistic and linguistic data were collected to explore how many distinct varieties of Papuan Malay exist and which one(s) of those varieties might be best suited for language development and standardization efforts. (See also Figure  ‎0 .2 on p. \pageref{bkm:Ref373252478}.)
\end{styleBodyvafter}


Another study on Papuan Malay, mentioned in §1.5, is {Besier’s (2012)} thesis. The author explores the role of Papuan Malay in society in terms of the language policies of the Indonesian government, as well as its role in the independence movement, in formal education, and in the church and mission organizations.
\end{styleBodyvafter}


{\citet{Burung2008a}} discusses the issue of Papuan Malay language awareness and vitality. Unlike {Scott et al. (2008: 10–17)} (see §1.5), {\citet{Burung2008a}} suggests that Papuan Malay is increasingly losing domains of use to Standard Indonesian due to the increasing influence of Indonesian throughout West Papua and the lack of language awareness among Papuans. (See also {Burung 2009}).
\end{styleBodyvafter}


In addition to these sociolinguistic studies, there are also three socio-historical studies, which need to be mentioned: Adelaar and \citet{Prentice1996}, Gil and \citet{Tadmor1997}, and Paauw (2005; 2007). These studies propose classifications of Malay in general and of the eastern Malay varieties in particular, including Papuan Malay, from a socio-historical perspective.
\end{styleBodyvafter}


Focusing on the period of European colonialism, {Adelaar and \citet[674]{Prentice1996}} identify three distinct sociolects of Malay: (1) “literary Malay”, (2) “lingua franca Malay”, and (3) “inherited Malay”. Within this framework, Papuan Malay is classified as a (“Pidgin Malay Derived”) lingua franca or trade language {(1996: 675)}, as already discussed in §1.2.2.
\end{styleBodyvafter}


Another, “tentative typology of Malay/Indonesian dialects” is proposed by {Gil and \citet{Tadmor1997}}. As their primary parameter, the authors propose the “lectal cline”, and thus distinguish between acrolectal (that is, Standard Malay/Indonesian) and basilectal (that is, nonstandard) Malay varieties {(1997: 1)}. The basilectal varieties are further divided into varieties with and without native speakers. For the former, a classification according two parameters is proposed: (1) ethnically homogeneous vs. ethnically heterogeneous and (2) ethnically Malay vs. ethnically non-Malay. According to this typology, Papuan Malay is classified as an “ethnically heterogeneous / non-Malay” variety (1997: 1).
\end{styleBodyvafter}


A different approach is taken by {Paauw (2005}; {2007}). Taking into account the diglossic nature of Malay, {Paauw} distinguishes between “national languages”, “inherited varieties”, and “contact varieties”. Among the latter, {\citet[2]{Paauw2007}} further differentiates four subtypes, one of them being the eastern Malay “nativized” varieties. Within this framework, Papuan Malay is classified as a “nativized” eastern Malay “contact variety” {(2007: 2; see also 2005: 14)}.
\end{styleBodyvxvafter}

\section{Available materials in Papuan Malay}
\label{bkm:Ref369971555}
At this point, materials in Papuan Malay are still scarce. Most of them seem to come in the form of jokes, or \textitbf{mop} ‘humor’. These jokes are published in newspapers or posted on dedicated websites, such as \textstyleChItalic{MopPapua}. Some of them are also published in book form, such as {Warami’s (2003; 2004}) jokes collections. Humor in Papuan Malay also comes in the form of comedy, such as the sketch series \textstyleChItalic{Epen ka, cupen toh} ‘Is it important? It’s important enough, indeed!’ from Merauke, which is accessible via YouTube.\footnote{\\
\\
\\
\textstyleChItalic{MopPapua} is available at \url{https://instagram.com/moppapua/} (accessed 8 January 2016).\\
\textstyleChItalic{Epen ka, cupen toh} is available at \url{http://www.youtube.com/watch?v=IWiQK0qKIj8} (accessed 8 May 2015).\\
\\
\\
\\
\\
\\
\\
\\
\\
\\
\\
}



In 2006, the movie \textstyleChItalic{Denias} came out, a film in Papuan Malay about a boy from the highlands who wants to go to school.\footnote{\\
\\
\\
\\
\textstyleChItalic{Denias} is available at \url{http://www.youtube.com/watch?v=kc683zv6H_E} (accessed 8 January 2016).\\
\\
\\
\\
\\
\\
\\
\\
\\
\\
\\
}
\end{styleBodyvafter}


Other materials in Papuan Malay are only available on the internet, such as:
\end{styleBodyvvafter}

%\setcounter{itemize}{0}
\begin{itemize}
\item \begin{styleOvNvwnext}
\textstyleChItalic{Kamus Bahasa Papua} ‘Dictionary of the Papuan Language’
\end{styleOvNvwnext}\end{itemize}
\begin{itemize}
\item \begin{styleIIndented}
A Papuan Malay – Indonesian dictionary with currently 164 items (last updated on 24 March 2011)
\end{styleIIndented}\item \begin{styleIiI}
Online URL: \url{http://kamusiana.com/index.php/index/20.xhtml} (accessed 8 January 2016)
\end{styleIiI}\end{itemize}
%\setcounter{itemize}{0}
\begin{itemize}
\item \begin{styleOvNvwnext}
\textstyleChItalic{Kitong pu bahasa} ‘Our Language’
\end{styleOvNvwnext}\end{itemize}
\begin{itemize}
\item \begin{styleIIndented}
A Christian website in Papuan Malay, Indonesian, and English which includes information about the Papuan Malay language and its history, the books of Jonah and Ruth from the Older Testament and the Easter story from the New Testament of the Bible in PDF format, and Christian texts and songs in audio format.
\end{styleIIndented}\end{itemize}
\begin{itemize}
\item \begin{styleIiI}
Online URL: \url{http://kitongpubahasa.com/en/_5699} (8 January 2016)
\end{styleIiI}\end{itemize}

Also, mention needs to be made of a language development program launched by Yayasan Betania Indonesia, a Papuan non-governmental organization located in Abepura, West Papua. The program’s goal is to develop written and audio resources with a focus on Bible translation, seeking to promote and develop the use of the language in the religious domain. {(L. Harms, p.c. 2015)}



An online resource providing materials on issues relevant to West Papua is ‘West Papua Web’.\footnote{\\
\\
\\
\\
‘West Papua Web’ is available at \url{http://www.papuaweb.org/} (last updated in January 2012) (accessed 8 January 2016).\\
\\
\\
\\
\\
\\
\\
\\
\\
\\
\\
} This resource is hosted by The University of Papua, Cenderawasih University, and the Australian National University. To date, however, the website does not provide materials in Papuan Malay.
\end{styleBodyvxvafter}

\section{Present study}
\label{bkm:Ref370323854}
This study primarily deals with the Papuan Malay language as it is spoken in the Sarmi area, which is located about 300 km west of Jayapura. Both towns are located on West Papua’s northeast coast. The description of the language is based on 16 hours of recordings of spontaneous conversations between Papuan Malay speakers.



The following sections provide pertinent background information for the study. After discussing some theoretical considerations in §1.11.1, the general setting of the research location Sarmi is presented in §1.11.2. The methodological approach and the field work are described in §1.11.3. Details on the recorded corpus and the sample of speakers contributing to this corpus are presented in §1.11.4. The procedures for the data transcription and analysis are discussed in §1.11.5. Finally, §1.11.6 describes the procedures involved in eliciting the word list.
\end{styleBodyvxvafter}

\subsection{Theoretical considerations}
\label{bkm:Ref376531198}
Papuan Malay is spoken in a rich linguistic and sociolinguistic environment in the coastal areas of West Papua (see §1.4 and §1.5). Many Papuans speak two or more languages which they use as deemed appropriate and necessary. That is, depending on the setting of the communicative event, speakers may use one or the other code or switch between them.



The conversations, recorded in Sarmi in late 2008, reveal some of this linguistic richness. They include conversations in which the interlocutors freely switch between different codes, such as Papuan Malay, Isirawa, and Indonesian. These recordings illustrate how intertwined and close to the speakers’ minds the languages that are part of their linguistic repertoire are.
\end{styleBodyvafter}


With a few exceptions, however, this description of Papuan Malay does not take into account language contact issues and therefore does not reflect the rich linguistic environment which Papuan Malay is part of. Instead, the description creates an abstraction of Papuan Malay as if it were a linguistic entity spoken in isolation, rather than spoken in the context of a larger, complex linguistic and sociolinguistic reality.
\end{styleBodyvafter}


That is, in terms of {de Saussure’s (1959)} distinction between \textstyleChItalic{langue} and \textstyleChItalic{parole}, this description of Papuan Malay focuses on the language system as “a collection of necessary conventions” {(1959: 9)}. The rationale for this abstraction is twofold. First, it is needed in order to identify, analyze, illustrate, and discuss pertinent linguistic features which are characteristics of Papuan Malay and which distinguish this speech variety from others, such as other eastern Malay varieties. Second, the abstraction is necessary in order to appreciate the complexity of Papuan Malay as \textstyleChItalic{parole}; as discussed below, however, the investigation of this complexity is beyond the scope of the present research.
\end{styleBodyvafter}


It is pointed out, however, that this abstraction of Papuan Malay as \textstyleChItalic{langue} is based on natural speech or \textstyleChItalic{parole}, which represents “the executive side of speaking” {(de Saussure 1959: 13, 14)}. Moreover, Papuan Malay as \textstyleChItalic{langue} is accessible and recognized by its speakers, although not without some difficulty. Furthermore, in being extracted from a “heterogeneous mass of speech facts”, the examples and texts presented in this book reflect at least part of the larger linguistic reality of the recorded speakers {(1959: 14)}.
\end{styleBodyvafter}


Given this focus on \textstyleChItalic{langue}, the present isolated analysis of Papuan Malay remains incomplete. After having extracted Papuan Malay from its complex (socio)linguistic reality, the next step in presenting an adequate linguistic description of the language needs to focus on Papuan Malay as \textstyleChItalic{parole}, with its “heterogeneous mass of speech facts” {(de Saussure 1959: 14)}. More specifically, this next step needs to consider the larger linguistic environment and the interactions between the different codes which are at the disposal of the coastal Papuan communities. This step, however, is beyond the scope of this book and is left for future research.
\end{styleBodyvxvafter}

\subsection{Setting of the research location}
\label{bkm:Ref370323233}
The research for the present description of Papuan Malay was conducted in Sarmi, the capital of the Sarmi regency (see Figure  ‎0 .3 on p. \pageref{bkm:Ref375721640}). In the planning stages of this research, it was suggested to the author that Sarmi would be a good site for collecting Papuan Malay language data, due to its location, which was still remote in late 2008 when the first period of this research was conducted (see also §1.11.3). It was anticipated that Papuan Malay as spoken in Sarmi would show less Indonesian influence than in other coastal urban areas such as Jayapura, Manokwari, or Sorong.



The coastal stretch of West Papua’s north coast, where Sarmi is located, is dominated by sandy beaches. The flat hinterland is covered with thick forest and gardens grown by local subsidiary farmers. The town of Sarmi is situated on a peninsula, about 300 km west of Jayapura on West Papua’s northeast coast; in 2010, the town had a population of 4,001 inhabitants; the regency’s population was 32,971.\footnote{\\
\\
\\
\\
Detailed 2010 census data are available at \url{http://bps.go.id/eng/download_file/Population_of_Indonesia_by_Village_2010.pdf} (accessed 21 Oct 2013) (see also §1.7.1).\\
\\
\\
\\
\\
\\
\\
\\
\\
\\
\\
}
\end{styleBodyvafter}


During the first period of this research, in late 2008, it was still difficult to get to Sarmi, as there were no bridges yet across the Biri and Tor rivers, located between Bonggo and Sarmi. Both rivers had to be crossed with small ferries with the result that public transport between Jayapura and Sarmi was limited, time-consuming, and expensive. A cheaper alternative was travel by ship, since the Sarmi harbor allows larger ships to anchor. This was also time-consuming, as the traffic between both cities was limited to about one to two ships per week. There is also a small airport but in 2008 there were no regular flight connections and tickets were too expensive for the local population. Today, there are bridges across the Biri and Tor rivers and public transport between Sarmi and Jayapura is both regular and less time-consuming and expensive than in 2008.
\end{styleBodyvafter}


In late 2008, the most western part of Sarmi regency was not yet accessible by road; the sand/gravel road ended in Martewar, 20 km west of Sarmi town. The villages between Martewar and Webro, that is, Wari, Aruswar, Niwerawar, and Arbais, were accessible by motorbike via the beach during low tide; the villages further west, that is, Waim, Karfasia, Masep, and Subu, were only accessible by boat. Today, the coastal road extends to Webro. The villages further west are still not accessible via road. Travel to the inland villages (Apawer Hulu, Burgena, Kamenawari, Kapeso, Nisro, Siantoa, and Samorkena) is also difficult as there are no proper roads to these remote areas. Some villages located along rivers are accessible by boat. Other villages are at times accessible via dirt road, constructed by logging enterprises. After heavy rains, however, these roads are impassable for most cars and trucks.
\end{styleBodyvafter}


Most of the Sarmi regency’s Papuan population work as subsistence farmers. Employment in the public sector is highly valued, and those who have adequate education levels try to find work as civil servants in the local government offices, in the health sector, or in the educational domain. However, secondary school education is not widely available. While the larger villages west of Sarmi have primary and junior high schools, there are no senior high schools in these villages. Hence, teenagers from families who have the financial means to pay tuition fees have to come to Sarmi. Here, they usually live with their extended families. This also applies to the author’s host family, most of whom are from Webro (see §1.11.3).
\end{styleBodyvafter}


Public health services are basic in the regency. There is a small hospital in Sarmi, but its medical services are rather limited. For surgery and the treatment of serious illnesses, the local population has to travel to Jayapura. Financial and postal services are available in Sarmi but not elsewhere in the regency. Communication via cell-phone is also possible in Sarmi and the surrounding villages, but it is limited in the more rural areas. Many villages are still not connected to telecommunication networks, as there are not enough cell sites to cover the entire regency.
\end{styleBodyvxvafter}

\subsection{Methodological approach and fieldwork}
\label{bkm:Ref437420654}\label{bkm:Ref368494722}
The description of Papuan Malay is based on 16 hours of recordings of spontaneous conversations between Papuan Malay speakers. The corpus includes only a few texts obtained via focused elicitation. The rationale for this methodological approach is discussed below.



The fieldwork was conducted in West Papua in four periods between September 2008 and December 2011. The first period took place in Sarmi from the beginning of September until mid-December 2008. During this time the texts which form the basis for the present study were recorded. The remaining three fieldwork periods took place in Sentani, located about 40 km west of Jayapura, from early October until mid-December 2009, from mid-October until mid-December 2010, and from early September until the end of November 2011. During these periods, the recordings were transcribed, about one third of the texts was translated into English, additional examples were elicited, and grammaticality judgment tests were conducted (see §1.11.5). During the fourth fieldwork in late 2011, the word list was recorded (see §1.11.6), and a 150-minute extract of the corpus was transcribed more thoroughly.
\end{styleBodyvafter}


During the first fieldwork I lived with a pastor, Kornelius\textsuperscript{†} Merne, his wife Sarlota\textsuperscript{†}, and three of their five children. Also living in the house were one of Sarlota’s sisters and eight teenagers (three males and five females). The teenagers were part of the extended family and came from the Mernes’ home village Webro, located about 30 km west of Sarmi, or nearby villages, which, like Webro, belong to the Pante-Barat district. At that time, the eight teenagers were junior or senior high school students. Furthermore, there was a constant coming and going of guests from villages of the Sarmi regency: relatives, pastoral workers, and/or local officials passing through or staying for several days up to several weeks. Hence, the household included between 14 and about 30 persons. The Mernes, their household members and many guests belonged to the Isirawa language group (Tor-Kwerba language family), to which Webro and the neighboring villages belong. Some guests originated from other language areas, such as the Papuan languages Samarokena, Sentani, and Tor, or the Austronesian languages Biak and Ambon Malay.
\end{styleBodyvafter}


At the beginning of my stay with his family, pastor Merne had given me permission to do recordings in his house. Besides recording spontaneous conversations, I had planned to elicit different text genre such as narratives, procedurals, and expositories. This, however, soon proved to be impossible for two reasons, namely the diglossic distribution of Papuan Malay and Indonesian, and the lack of language awareness, discussed in §1.5. As a result of these two factors, it proved de facto impossible for the household members and guests to talk with me in Papuan Malay. They always switched to Indonesian. This made both focused elicitation and language learning difficult. Therefore, after a few unsuccessful attempts to elicit texts, I decided to refrain from further elicitation and to record spontaneous conversations instead. From then on, I always carried a small recording device with internal microphone which I turned on when two or more people were conversing. After a few days the household members were used to my constant recording. I never had the impression that they were trying to avoid being recorded (there were only two situations in which speakers distanced themselves from me in order not to be recorded). Most of the sixteen hours of text were recorded in this manner, as discussed in more detail in §1.11.4.1. There are a few exceptions, though, which are also discussed in §1.11.4.1.
\end{styleBodyvafter}


Given that my hosts and their guests typically switched to Indonesian when talking with me, most of my language learning was by listening to Papuans talking to each other in Papuan Malay, by applying what I observed during these conservations and in the recorded data, and by discussing these observations with those speakers who were interested in talking about language related issues. The procedures involved in transcribing and analyzing the recorded texts are described in §1.11.5
\end{styleBodyvafter}


During the fourth period of fieldwork, from the beginning of September until the end of November 2011, I recorded a 2,458-item word list {\citep{KlugeEtAl2014}}. The items were extracted from the transcribed corpus and recorded in isolation to investigate the Papuan Malay phonology at the word level. The consultants from whom the list was recorded were two Papuan Malay speakers, Ben Rumaropen and Lodowik Aweta. The procedures involved in recording this list are described in §1.11.6. 
\end{styleBodyvxvafter}

\subsection{Papuan Malay corpus and speaker sample}
\label{bkm:Ref368494726}
During the first fieldwork period in late 2008, 220 texts totaling almost 16 hours were recorded. Almost all of them were recorded in Sarmi (217/220 texts); the remaining three were recorded in Webro. The texts were recorded from a sample of about 60 different Papuan Malay speakers. The corpus is described in §1.11.4.1, and the sample of recorded speakers in §1.11.4.2.
\end{styleBodyxvafter}

\paragraph[Recorded texts]{Recorded texts}
\label{bkm:Ref368418518}\label{bkm:Ref368402283}
The basis for the current study is a 16-hour corpus. In all, 220 texts were recorded (see Appendix B). The texts were recorded in the form of WAV files with a Marantz PMD620 using the recorder’s internal microphone. Each WAV file was labeled with a record number which includes the date of its recording, a running number for all texts recorded during one day, and a code for the type of text recorded. This is illustrated with the record number 080919-007-CvNP: 080919 stands for “2008, September 19”; 007 stands for “recorded text \#7 of that day”; and CvNP stands for “Personal Narrative (NP) which occurred during a Conversation (Cv)”. The same record numbers are used in Toolbox for the transcribed texts (see §1.11.5.1) and the examples given in this book (see ‘Conventions for examples‘, p. \pageref{bkm:Ref434428827}).



Most texts are spontaneous conversations which occurred between two or more Papuan speakers (157/220 texts – 71.4\%). Details concerning the contents of these conversations are given in Table  ‎1 .15. The remaining 63 texts (28.6\%) fall into two groups: conversations with the author (see Table  ‎1 .16) and elicited texts (see Table  ‎1 .17). (See also Appendix B for a detailed listing of the 220 recorded texts.)
\end{styleBodyvvafter}

\begin{stylecaption}
Table ‎1.\stepcounter{Table}{\theTable}:  Overview of 16-hour corpus
\end{stylecaption}

\tablehead{
 Text types & \multicolumn{2}{l}{ Count of texts} & \multicolumn{2}{l}{ Count of hours}\\
}
\begin{tabular}{lllll}
\lsptoprule
Spontaneous conversations & \raggedleft 157 & \raggedleft 71.4\% & \raggedleft 10:08:02 & \raggedleft\arraybslash 63.4\%\\
Conversations with the author & \raggedleft 40 & \raggedleft 18.2\% & \raggedleft 04:27:15 & \raggedleft\arraybslash 27.9\%\\
Elicited texts & \raggedleft 23 & \raggedleft 10.4\% & \raggedleft 01:23:17 & \raggedleft\arraybslash 8.7\%\\
Total & \raggedleft 220 & \raggedleft 100\% & \raggedleft 15:58:34 & \raggedleft\arraybslash 100\%\\
\lspbottomrule
\end{tabular}

Most of the texts in the corpus are spontaneous conversations between two or more Papuans. While being present during these conversations, I usually did not participate in the talks unless being addressed by one of the interlocutors. The recorded conversations cover a wide range of text genre and topics. The majority of conversations are casual and about everyday topics related to family life, relations with others, work, education, local politics, and religion. Five conversations were conducted over the phone. A substantial number of the recorded conversations are narratives about personal experiences such as journeys or childhood experiences. Included are also 14 expositories, five hortatories, two folk stories, and one brief procedural.\footnote{\\
\\
\\
\\
In expository discourse the speaker describes or explains a topic. In hortatory discourse the speaker attempts to persuade the addressee to fulfill the commands given in the discourse. In procedural discourse the speaker describes how to do something. {\citep{LoosEtAl2003}}.\\
\\
\\
\\
\\
\\
\\
\\
\\
\\
\\
} In all, the corpus contains 157 such conversations (157/220 – 71.4\%), accounting for about ten hours of the 16-hour corpus (63.4\%).


\begin{stylecaption}
\label{bkm:Ref368316407}Table ‎1.\stepcounter{Table}{\theTable}:  Spontaneous conversations\footnote{\\
\\
\\
\\
As percentages are rounded to one decimal place, they do not always add up to 100\%.\\
\\
\\
\\
\\
\\
\\
\\
\\
\\
\\
}
\end{stylecaption}

\tablehead{
 Contents & \multicolumn{2}{l}{ Count of texts} & \multicolumn{2}{l}{ Count of hours}\\
}
\begin{tabular}{lllll}
\lsptoprule
Casual conversations & \raggedleft 105 & \raggedleft 66.9\% & \raggedleft 05:59:55 & \raggedleft\arraybslash 59.2\%\\
Phone conversations & \raggedleft 5 & \raggedleft 3.2\% & \raggedleft 01:13:19 & \raggedleft\arraybslash 12.1\%\\
Expositories & \raggedleft 14 & \raggedleft 8.9\% & \raggedleft 00:59:48 & \raggedleft\arraybslash 9.8\%\\
Hortatories & \raggedleft 5 & \raggedleft 3.2\% & \raggedleft 00:03:48 & \raggedleft\arraybslash 0.6\%\\
Narratives (folk stories) & \raggedleft 2 & \raggedleft 1.3\% & \raggedleft 00:39:45 & \raggedleft\arraybslash 6.5\%\\
Narratives (personal experiences) & \raggedleft 25 & \raggedleft 15.9\% & \raggedleft 01:05:17 & \raggedleft\arraybslash 10.7\%\\
Procedurals & \raggedleft 1 & \raggedleft 0.6\% & \raggedleft 00:06:10 & \raggedleft\arraybslash 1.0\%\\
Total & \raggedleft 157 & \raggedleft 100\% & \raggedleft 10:08:02 & \raggedleft\arraybslash 100\%\\
\lspbottomrule
\end{tabular}

The corpus also includes 40 texts which I recorded when visiting two relatives of the Merne family. Unlike the other family members and guests of the Merne household, two of Sarlota Merne’s relatives, a young female pastor and her husband who also lived in Sarmi, had no difficulties talking to me in Papuan Malay. I visited them regularly to chat, elicit personal narratives, and discuss local customs and beliefs. In all, the corpus contains 40 such texts (40/220 – 18.2\%) (see Table  ‎1 .16). These texts account for about four and a half hours of the 16-hour corpus (27.9\%).


\begin{stylecaption}
\label{bkm:Ref368316408}Table ‎1.\stepcounter{Table}{\theTable}:  Conversations with the author
\end{stylecaption}

\tablehead{
 Contents & \multicolumn{2}{l}{ Count of texts} & \multicolumn{2}{l}{ Count of hours}\\
}
\begin{tabular}{lllll}
\lsptoprule
Casual conversations & \raggedleft 13 & \raggedleft 32.5\% & \raggedleft 01:17:05 & \raggedleft\arraybslash 28.8\%\\
Expositories & \raggedleft 17 & \raggedleft 42.5\% & \raggedleft 02:10:15 & \raggedleft\arraybslash 48.7\%\\
Narratives (personal experiences) & \raggedleft 8 & \raggedleft 20.0\% & \raggedleft 00:50:36 & \raggedleft\arraybslash 18.9\%\\
Procedurals & \raggedleft 2 & \raggedleft 5.0\% & \raggedleft 00:09:19 & \raggedleft\arraybslash 3.5\%\\
Total & \raggedleft 40 & \raggedleft 100\% & \raggedleft 04:27:15 & \raggedleft\arraybslash 100\%\\
\lspbottomrule
\end{tabular}

The corpus also contains 23 elicited texts (23/220 – 10\%) (see Table  ‎1 .17). These texts account for about one and a half hours of the 16-hour corpus (8.7\%). During the first two weeks of my first fieldwork, I elicited a few texts, as mentioned in §1.11.3. Two were short procedurals which I recorded on a one-to-one basis. Besides, I elicited three personal narratives with the help of Sarlota Merne, who was one of the few who were aware of the language variety I wanted to study and record. She was present during these elicitations and explained that I wanted to record texts in \textitbf{logat Papua} ‘Papuan speech variety’. She also monitored the speech of the narrators; that is, when they switched to Indonesian, she made them aware of the switch and asked them to continue in \textitbf{logat Papua}. Toward the end of my stay in Sarmi, when I was already well-integrated into the family and somewhat proficient in Papuan Malay, I recorded one narrative in a group situation from one of Sarlota Merne’s sisters and another three personal narratives on a one-to-one basis from one of the teenagers living with the Mernes. Also toward the end of this first fieldwork, I recorded 14 jokes which two of the teenagers also living in the house told each other. A sample of texts is presented in Appendix A.


\begin{stylecaption}
\label{bkm:Ref368316409}Table ‎1.\stepcounter{Table}{\theTable}:  Elicited texts
\end{stylecaption}

\tablehead{
 Contents & \multicolumn{2}{l}{ Count of texts} & \multicolumn{2}{l}{ Count of hours}\\
}
\begin{tabular}{lllll}
\lsptoprule
Jokes & \raggedleft 14 & \raggedleft 60.9\% & \raggedleft 00:13:12 & \raggedleft\arraybslash 15.8\%\\
Narratives (personal experiences) & \raggedleft 7 & \raggedleft 30.4\% & \raggedleft 01:06:47 & \raggedleft\arraybslash 80.2\%\\
Procedurals & \raggedleft 2 & \raggedleft 8.7\% & \raggedleft 00:03:18 & \raggedleft\arraybslash 4.0\%\\
Total & \raggedleft 23 & \raggedleft 100\% & \raggedleft 01:23:17 & \raggedleft\arraybslash 100\%\\
\lspbottomrule
\end{tabular}
\paragraph[Sample of recorded Papuan Malay speakers]{Sample of recorded Papuan Malay speakers}
\label{bkm:Ref368402281}
The corpus was recorded from about 60 different speakers. This sample includes 44 speakers personally known to the author. Table  ‎1 .18 to Table  ‎1 .20 provide more information with respect to their language backgrounds, gender, age groups, and occupations.



The sample also includes a fair number of speakers who visited the Merne household briefly and who took part in the recorded conversations. In transcribing their contributions to the ongoing conversations, their gender and approximate age were noted; additional information on their language backgrounds or occupations is unknown, however.
\end{styleBodyvafter}


Table  ‎1 .18 presents details with respect to the vernacular languages spoken by the 44 recorded Papuan Malay speakers. Most of them are speakers of Isirawa, a Tor-Kwerba language (38/44 – 86). The vernacular languages of the remaining six speakers are the Austronesian languages Biak and Ambon Malay, and the Papuan languages Samarokena, Sentani, and Tor.
\end{styleBodyvvafter}

\begin{stylecaption}
\label{bkm:Ref368400189}Table ‎1.\stepcounter{Table}{\theTable}:  The recorded Papuan Malay speakers by vernacular languages
\end{stylecaption}

\tablehead{
 Vernacular language & \arraybslash Total\\
}
\begin{tabular}{ll}
\lsptoprule
Isirawa & \raggedleft\arraybslash 38\\
Ambon Malay & \raggedleft\arraybslash 1\\
Biak & \raggedleft\arraybslash 1\\
Samarokena & \raggedleft\arraybslash 2\\
Sentani & \raggedleft\arraybslash 1\\
Tor & \raggedleft\arraybslash 1\\
Total & \raggedleft\arraybslash 44\\
\lspbottomrule
\end{tabular}

Table  ‎1 .19 gives an overview of the recorded 44 speakers in terms of their gender and age groups. The sample includes 20 males (45\%) and 24 females (55\%). Age wise, the sample is divided into three groups: 19 adults in their thirties or older (19/44 – 43\%), 20 young adults in their teens or twenties (20/44 – 45\%), and five children of between about five to 13 years of age.


\begin{stylecaption}
\label{bkm:Ref368400190}Table ‎1.\stepcounter{Table}{\theTable}:  The recorded Papuan Malay speakers by gender and age groups
\end{stylecaption}

\tablehead{
 Age groups & Males & Females & \arraybslash Total\\
}
\begin{tabular}{llll}
\lsptoprule
Adult (thirties and older) & \raggedleft 10 & \raggedleft 9 & \raggedleft\arraybslash 19\\
Young adult (teens and twenties) & \raggedleft 6 & \raggedleft 14 & \raggedleft\arraybslash 20\\
Child (5-13 years) & \raggedleft 4 & \raggedleft 1 & \raggedleft\arraybslash 5\\
Total & \raggedleft 20 & \raggedleft 24 & \raggedleft\arraybslash 44\\
\lspbottomrule
\end{tabular}

Table  ‎1 .20 provides an overview of the speakers and their occupations. The largest subgroups are pupils (13/44 – 30\%), farmers (10/44 – 23\%), and government or business employees (5/44 – 11\%). Eight of the 13 students were the teenagers living in the Merne household. The two BA students were the Merne’s oldest children who were studying in Jayapura and only once in a while came home to Sarmi. In addition to the ten full-time farmers, three of the government employees worked as part-time farmers. Of the total of five children, three were not yet in school; the remaining two were in primary school.


\begin{stylecaption}
\label{bkm:Ref368400191}Table ‎1.\stepcounter{Table}{\theTable}:  The recorded Papuan Malay speakers by occupation
\end{stylecaption}

\tablehead{
 Occupation & Males & Females & \arraybslash Total\\
}
\begin{tabular}{llll}
\lsptoprule
Student (BA studies) & \raggedleft 1 & \raggedleft 1 & \raggedleft\arraybslash 2\\
Pupil (high school) & \raggedleft 1 & \raggedleft 4 & \raggedleft\arraybslash 5\\
Pupil (middle school) & \raggedleft 1 & \raggedleft 5 & \raggedleft\arraybslash 6\\
Pupil (primary school) & \raggedleft 2 & \raggedleft 0 & \raggedleft\arraybslash 2\\
Farmer & \raggedleft 2 (+3) & \raggedleft 8 & \raggedleft\arraybslash 10 (+3)\\
Employee (government/business) & \raggedleft 5 & \raggedleft 0 & \raggedleft\arraybslash 5\\
Pastor & \raggedleft 2 & \raggedleft 1 & \raggedleft\arraybslash 3\\
(ex-)Mayor & \raggedleft 2 & \raggedleft 0 & \raggedleft\arraybslash 2\\
Housewife & \raggedleft 0 & \raggedleft 2 & \raggedleft\arraybslash 2\\
BA graduate & \raggedleft 0 & \raggedleft 1 & \raggedleft\arraybslash 1\\
Church verger & \raggedleft 1 & \raggedleft 0 & \raggedleft\arraybslash 1\\
Nurse & \raggedleft 1 & \raggedleft 0 & \raggedleft\arraybslash 1\\
Teacher & \raggedleft 0 & \raggedleft 1 & \raggedleft\arraybslash 1\\
Child & \raggedleft 2 & \raggedleft 1 & \raggedleft\arraybslash 3\\
Total & \raggedleft 24 & \raggedleft 20 & \raggedleft\arraybslash 44\\
\lspbottomrule
\end{tabular}
\subsection{Data transcription, analysis, and examples}
\label{bkm:Ref368650541}\label{bkm:Ref368494728}
This section discusses the transcription and analysis of the recorded Papuan Malay texts. In §1.11.5.1, the procedures for transcribing and translating the recorded data are discussed. In §1.11.5.2, the procedures related to the data analysis are described, including grammaticality judgments and focused elicitation.
\end{styleBodyxvafter}

\paragraph[Data transcription and translation into English]{Data transcription and translation into English}
\label{bkm:Ref368492925}\label{bkm:Ref368418052}
Two Papuan Malay consultants transcribed the recorded texts during the second fieldwork in late 2009 and the third fieldwork in late 2010. The two consultants were Ben Rumaropen, who was one of my main consultants throughout the entire research project, and Emma Onim.



B. Rumaropen grew up in Abepura, located about 20 km west of Jayapura; his parents are from Biak. In 2004, B. Rumaropen graduated with a BA in English from Cenderawasih University in Jayapura. From 2002 until 2008, he worked with the SIL Papua survey team. During this time he was one of the researchers involved in the mentioned 2007 sociolinguistics survey of Papuan Malay {\citep{ScottEtAl2008}}. E. Onim grew up in Jayapura; her parents are from Wamena. In 2010, E. Onim graduated with a BA in finance from Cenderawasih University in Jayapura. Since then, she has been the finance manager of a local NGO.
\end{styleBodyvafter}


The two consultants transcribed the texts in Microsoft Word, listening to the recordings with Speech Analyzer, a computer program for acoustic analysis of speech sound, developed by SIL International.\footnote{\\
\\
\\
\\
Speech Analyzer is available at \url{http://www-01.sil.org/computing/sa/} (accessed 8 January 2016).\\
\\
\\
\\
\\
\\
\\
\\
\\
\\
\\
} B. Rumaropen transcribed 121 texts, and E. Onim 99 texts; each text was transcribed in a separate Word file. Using Indonesian orthography, both consultants transcribed the data as literally as possible, including hesitation markers, false starts, truncation, speech mistakes, and nonverbal vocalizations, such as laughter or coughing. Once a recording had been transcribed, I checked the transcription by listening to the recording. Transcribed passages which did not match with the recordings were double-checked with the consultants. After having checked the transcribed texts is this manner, I imported the Word files into Toolbox, a data management and analysis tool developed by SIL International.\footnote{\\
\\
\\
\\
Toolbox is available at \url{http://www-01.sil.org/computing/toolbox/} (accessed 8 January 2016).\\
\\
\\
\\
\\
\\
\\
\\
\\
\\
\\
} In Toolbox, I interlinearized the 220 texts into English and Indonesian and compiled a basic dictionary. Each text was imported into a separate Toolbox record, receiving the same record number as its respective WAV file (for details see §1.11.4.1).
\end{styleBodyvafter}


During the second fieldwork in late 2009, B. Rumaropen and I translated 83 of the 220 texts into English, which accounts for a good five hours of the 16-hour corpus. The translated texts also contain explanations and additional comments which B. Rumaropen provided during the translation process. Appendix A presents 12 of these texts.
\end{styleBodyvafter}


During the fourth fieldwork period in late 2011, B. Rumaropen transcribed a 150-minute extract of the corpus more thoroughly, that is, close to phonetically. In addition to the wordlist (§1.11.6), this extract also aided in the analysis of the Papuan Malay phonology.
\end{styleBodyvafter}


The entire text material, including the recordings and the Toolbox files are archived with SIL International. Due to privacy considerations, however, they are not publically available. The examples in this book are taken from the entire corpus; that is, examples taken from the 137 texts which have not yet been translated were translated as needed. In the examples, proper names are substituted with aliases to guard anonymity.
\end{styleBodyvxvafter}

\paragraph[Data analysis, grammaticality judgments, and focused elicitation]{Data analysis, grammaticality judgments, and focused elicitation}
\label{bkm:Ref368492926}
In early 2010, after B. Rumaropen had transcribed a substantial number of texts and we had translated the mentioned 83 texts, I started with the analysis of the Papuan Malay corpus. This analysis was greatly facilitated by the Toolbox concordance tool, in which all occurrences of a word, phrase, or construction can be retrieved. The retrieved data was imported into Word for further sorting and analysis. Another helpful feature was the Toolbox export command, which allows different fields to be chosen for export into Word, such as the text, morpheme, or speech part fields.



During the analysis, I compiled a list of questions about analytical issues and comprehension problems encountered in the corpus. During the third and fourth fieldwork periods in late 2010 and late 2011, I worked through these questions with Papuan Malay consultants. Most of this work was done with B. Rumaropen. I also consulted informally with other Papuan Malay speakers on various occasions.
\end{styleBodyvafter}


During both fieldwork periods, I also worked with B. Rumaropen on grammaticality judgments. That is, based on the analysis of the corpus data, I constructed sentences which I submitted to B. Rumaropen to comment upon. When I found gaps in the data, I discussed them with B. Rumaropen to establish whether a given expression or construction exists in Papuan Malay, and I asked him to provide some example sentences. Beyond these fieldwork periods, B. Rumaropen and I stayed in contact via email and Skype and continued working on grammaticality judgments and the elicitation of example sentences, as needed.
\end{styleBodyvafter}


The elicited examples and the constructed sentences for grammaticality judgments were entered into a separate Toolbox database file. Where used in this grammar, these examples are explicitly labeled as “elicited”. All other examples are taken from the Papuan Malay corpus. Throughout this book all generic statements, both positive and negative, are based on the occurrences in the corpus, unless stated otherwise.
\end{styleBodyvxvafter}

\subsection{Word list}
\label{bkm:Ref368494266}
During the fourth fieldwork period in late 2011, I recorded a 2,458-item word list with two Papuan Malay consultants, namely B. Rumaropen and Lodowik Aweta. Originally from Webro, L. Aweta was one of the young people living in the Mernes’ household during my first fieldwork in 2008. In 2011, L. Aweta was a student at Cenderawasih University.



The word list was extracted from the compiled Toolbox dictionary. During the elicitation, B. Rumaropen provided the stimulus, while L. Aweta repeated the stimulus within one of two different frame sentences.
\end{styleBodyvafter}


The frame sentences, which are given in (0) and (0), were used alternatively and served two purposes. First, I anticipated that by repeating the target word within a larger sentence, L. Aweta would potentially be less influenced by B. Rumaropen’s pronunciation. This precaution was taken in case that the pronunciations of the two consultants differed, with one being from Sentani and the other one from Sarmi. Second, eliciting the target word as part of a larger sentence allowed me to analyze how some of the word-final segments were pronounced when they occurred in sentence final position and when they were followed by another word. This proved especially helpful in analyzing the realizations of the plosives and the rhotic when occurring in the word-final coda position (see §2.1.1.1, §2.3.1.2, and §2.3.1.3 in Chapter 2).
\end{styleBodyvvafter}

\begin{styleExampleTitle}
Frame sentences for word list elicitation
\end{styleExampleTitle}

\begin{tabular}{llllllllll}
\lsptoprule
\label{bkm:Ref373486882}
\gll {sa} {blum} {taw} {ko} {pu} {kata} {itu,} {kata} {\_\_\_}\\ %
& \textsc{1sg} & not.yet & know & \textsc{2sg} & \textsc{poss} & word & \textsc{d.dist} & word & \_\_\_\\
\lspbottomrule
\end{tabular}
\ea
\glt 
‘I don’t yet know that word of yours, the word \_\_\_’
\z

\begin{tabular}{lllllllll}
\lsptoprule
\label{bkm:Ref368417489}
\gll {ko} {pu} {kata} {\_\_\_} {itu,} {sa} {blum} {taw}\\ %
& \textsc{2sg} & \textsc{poss} & word & \_\_\_ & \textsc{d.dist} & \textsc{1sg} & not.yet & know\\
\lspbottomrule
\end{tabular}
\ea
\glt 
‘that word \_\_\_ of yours, I don’t yet know (it)’
\z


B. Rumaropen recorded each elicited word in a separate WAV file, using Speech Analyzer. Subsequently, I transcribed the recorded target words as separate records in Toolbox. Each record includes the orthographic representation of the target word, its phonetic transcription, English gloss, and the word class it belongs to. The word list is found in Appendix A. The sound files and the Toolbox database file are found in {\citet{KlugeEtAl2014}}.



After having entered the target words in Toolbox, I analyzed the lexical data with Phonology Assistant. This analysis tool, developed by SIL International, creates consonant and vowel inventory charts and assists in the phonological analysis.\footnote{\\
\\
\\
\\
Phonology Assistant is available at \url{http://phonologyassistant.sil.org} (accessed 8 January 2016).\\
\\
\\
\\
\\
\\
\\
\\
\\
\\
\\
}
\end{styleBodyvafter}


The description of the Papuan Malay phonology in Chapter 2 is based on a word list of 1,117 lexical roots, extracted from the 2,458-item list. In addition, 380 items, historically derived by (unproductive) affixation, are investigated. The corpus also includes a large number of loan words, originating from different donor languages, such as Arabic, Chinese, Dutch, English, Persian, Portuguese, or Sanskrit. Hence, a sizeable percentage of the attested lexical items are loan words. So far, 719 items of the 2,458-item word list (29\%) have been identified as loan words, using the following sources: {\citet{Jones2007}} and {\citet{Tadmor2009a}} (on borrowing in Malay in general see also {Blust 2013: 151–156}). Upon further investigation, some of the 1,117 lexical roots listed as inherited Papuan Malay words may also turn out to be loan words. In addition, the corpus includes a number of lexical items which are typically used in Standard Indonesian but not in Papuan Malay; examples are Indonesian \textitbf{desa} ‘village’ and \textitbf{mereka} ‘\textsc{3pl}’ (the corresponding Papuan Malay words are \textitbf{kampung} ‘village’ and \textitbf{dorang}/\textitbf{dong} ‘\textsc{3pl}’, respectively). Given that these words are inherited Malay lexical items, they are not treated as loan words in this book. However, neither are these items included in the word list in Appendix A.
\end{styleBodyvafter}

%\setcounter{page}{1}\chapter[Phonology]{Phonology}
\label{bkm:Ref357271598}
Papuan Malay has 18 consonant phonemes and a basic five-vowel system. The consonant system consists of six stops, two affricates, two fricatives, four nasals, two liquids, and two approximants. The vowel system includes two front and two back vowels, and one open central vowel. Papuan Malay shows a clear preference for disyllabic roots and for CV and CVC syllables; the maximal syllable is CCVC. Stress typically falls on the penultimate syllable, although lexical roots with ultimate stress are also attested.



The description of Papuan Malay phonology is based on a word list of 1,117 lexical roots plus 380 items, historically derived by (unproductive) affixation, extracted from the 2,458-item word list, mentioned in §1.11.6. The native consonant and vowel phoneme inventories are presented in §2.1. The phonological changes that the consonant and vowel segments can undergo are discussed in §2.2. A number of surface phenomena are described in §2.3. The phonotactics of Papuan Malay are investigated in §2.4, including a discussion of the segment distribution and possible sequences, syllable structures, and stress patterns. As already mentioned in §1.11.6, the corpus also includes a large number of loan words; so far 719 items of the 2,458-item word list (29\%) have been identified. Papuan Malay has also adopted one loan segment, the voiceless labio-dental fricative /\textstyleChCharisSIL{f}/, and developed three substitution strategies to realize another non-native segment, the voiceless postalveolar fricative /\textstyleChCharisSIL{ʃ}/. The non-native segments and loan words are discussed in §2.5. Given the rather large percentage of loan words, this discussion is rather detailed, including a description of the phonological and phonetic processes and the phonotactics attested in loan words.



This chapter closes with an account of the orthographic conventions used in this grammar in §2.6 and a summary in §2.7.\footnote{\\
\\
\\
\\
Two important sources for the description of the Papuan Malay phonology are {\citet{Donohue2003}} and {Sutri Narfafan and Donohue (under review)}.\\
\\
\\
\\
\\
\\
\\
\\
\\
\\
\\
}
\end{styleBodyvxvafter}

\section{Segment inventory}
\label{bkm:Ref324759936}
The Papuan Malay consonant system is presented in §2.1.1, and the vowel system in §2.1.2.
\end{styleBodyxvafter}

\subsection{Consonant system}
\label{bkm:Ref324760127}\paragraph[Consonant inventory]{Consonant inventory}
\label{bkm:Ref334076527}
Papuan Malay has 18 consonant phonemes, shown in Table  ‎2 .1. The system consists of three pairs of stops, one pair of affricates, four nasals, two fricatives, two liquids, and two approximants.


\begin{stylecaption}
\label{bkm:Ref323742688}Table ‎2.\stepcounter{Table}{\theTable}:  Papuan Malay consonant inventory
\end{stylecaption}

\tablehead{ & \multicolumn{2}{l}{ \textsc{lab}} & \multicolumn{2}{l}{ \textsc{alv}} & \multicolumn{2}{l}{ \textsc{pal-alv}} & \multicolumn{2}{l}{ \textsc{pal}} & \multicolumn{2}{l}{ \textsc{vel}} & \multicolumn{2}{l}{ \textsc{glot}}\\
}
\begin{tabular}{lllllllllllll}
\lsptoprule
\textsc{stop} & \textstyleChCharisSIL{p} & \textstyleChCharisSIL{b} & \textstyleChCharisSIL{t} & \textstyleChCharisSIL{d} &  &  &  &  & \textstyleChCharisSIL{k} & \textstyleChCharisSIL{g} &  & \\
\textsc{affr} &  &  &  &  & \textstyleChCharisSIL{t}\textstyleChCharisSIL{ʃ} & \textstyleChCharisSIL{d}\textstyleChCharisSIL{ʒ} &  &  &  &  &  & \\
\textsc{nas} &  & \textstyleChCharisSIL{m} &  & \textstyleChCharisSIL{n} &  &  &  & \textstyleChCharisSIL{ɲ} &  & \textstyleChCharisSIL{ŋ} &  & \\
\textsc{fric} &  &  & \textstyleChCharisSIL{s} &  &  &  &  &  &  &  & \textstyleChCharisSIL{h} & \\
\textsc{rhot} &  &  &  & \textstyleChCharisSIL{r} &  &  &  &  &  &  &  & \\
\textsc{lat-aprx} &  &  &  & \textstyleChCharisSIL{l} &  &  &  &  &  &  &  & \\
\textsc{aprx} &  &  &  &  &  &  &  & \textstyleChCharisSIL{j} &  & \textstyleChCharisSIL{w} &  & \\
\lspbottomrule
\end{tabular}

The 18 phonemes and their realizations are presented in Table  ‎2 .2. The rhotic has three allophones; the phonological and phonetic processes involved in their variation are discussed in §2.2.2 and §2.3.1.3, respectively. The voiceless stops are typically unreleased in the coda position. However, when occurring in the word-final coda position before a pause, they can be slightly released.


\begin{stylecaption}
\label{bkm:Ref323731609}Table ‎2.\stepcounter{Table}{\theTable}:  Papuan Malay stops
\end{stylecaption}

\tablehead{
\multicolumn{2}{l}{ Phoneme} & \multicolumn{2}{l}{ Realization}\\
}
\begin{tabular}{llll}
\lsptoprule
Stop & /\textstyleChCharisSIL{p}/ & [\textstyleChCharisSIL{p}], & a voiceless bilabial stop\\
&  & [\textstyleChCharisSIL{p̚}], & an unreleased voiceless bilabial stop\\
& /\textstyleChCharisSIL{b}/ & [\textstyleChCharisSIL{b}], & a voiced bilabial stop\\
& /\textstyleChCharisSIL{t}/ & [\textstyleChCharisSIL{t}], & a voiceless alveolar stop\\
&  & [\textstyleChCharisSIL{t̚}], & an unreleased voiceless alveolar stop\\
& /\textstyleChCharisSIL{d}/ & [\textstyleChCharisSIL{d}], & a voiced alveolar stop\\
& /\textstyleChCharisSIL{k}/ & [\textstyleChCharisSIL{k}], & a voiceless velar stop\\
&  & [\textstyleChCharisSIL{k̚}], & an unreleased voiceless velar stop\\
& /\textstyleChCharisSIL{g}/ & [\textstyleChCharisSIL{g}], & a voiced velar stop\\
Affricate & /\textstyleChCharisSIL{t}\textstyleChCharisSIL{ʃ}/ & [\textstyleChCharisSIL{t}\textstyleChCharisSIL{ʃ}], & a voiceless palato-alveolar affricate\\
& /\textstyleChCharisSIL{d}\textstyleChCharisSIL{ʒ}/ & [\textstyleChCharisSIL{d}\textstyleChCharisSIL{ʒ}], & a voiced palato-alveolar affricate\\
Nasal & /\textstyleChCharisSIL{m}/ & [\textstyleChCharisSIL{m}], & a voiced bilabial nasal\\
& /\textstyleChCharisSIL{n}/ & [\textstyleChCharisSIL{n}], & a voiced alveolar nasal\\
& /\textstyleChCharisSIL{ɲ}/ & [\textstyleChCharisSIL{ɲ}], & a voiced palatal nasal\\
& /\textstyleChCharisSIL{ŋ}/ & [\textstyleChCharisSIL{ŋ}], & a voiced velar nasal\\
Fricative & /\textstyleChCharisSIL{s}/ & [\textstyleChCharisSIL{s}], & a voiceless alveolar fricative\\
& /\textstyleChCharisSIL{h}/ & [\textstyleChCharisSIL{h}], & a voiceless glottal fricative\\
Liquid & /\textstyleChCharisSIL{r}/ & [\textstyleChCharisSIL{r}], & a voiced alveolar trill\\
&  & [\textstyleChCharisSIL{r̥}], & a voiceless alveolar trill\\
&  & [\textstyleChCharisSIL{ɾ}], & a voiced alveolar tap\\
& /\textstyleChCharisSIL{l}/ & [\textstyleChCharisSIL{l}], & a voiced alveolar lateral\\
Approximant & /\textstyleChCharisSIL{j}/ & [\textstyleChCharisSIL{j}], & a voiced palatal approximant\\
& /\textstyleChCharisSIL{w}/ & [\textstyleChCharisSIL{w}], & a voiced labio-velar approximant\\
\lspbottomrule
\end{tabular}
\paragraph[Contrast between similar consonants]{Contrast between similar consonants}

Contrast between similar consonants is presented in minimal or near-minimal pairs in the following tables: in word-initial position in Table  ‎2 .3, in root-internal position in Table  ‎2 .4, and in word-final position in Table  ‎2 .5. When (near-)minimal pairs could not be found, another word containing a contrasting consonant is given. Some segments have a restricted distribution; the palatal nasal, for instance, does not occur in the coda position (§2.4.1).


\begin{stylecaption}
\label{bkm:Ref323303088}Table ‎2.\stepcounter{Table}{\theTable}:  Consonant contrast in word-initial position
\end{stylecaption}

\tablehead{
\multicolumn{2}{l}{ Contrast} & Item & Orthogr. & Gloss\\
}
\begin{tabular}{lllll}
\lsptoprule
\textstyleChCharisSIL{p{\Tilde}b{\Tilde}m} &  & [\textstyleChCharisSIL{ˈpu.lu}] & \textitbf{pulu} & ‘tens’\\
&  & [\textstyleChCharisSIL{ˈbu.lu}] & \textitbf{bulu} & ‘body hair’\\
&  & [\textstyleChCharisSIL{ˈmu.lʊt̚}] & \textitbf{mulut} & ‘mouth’\\
\textstyleChCharisSIL{t{\Tilde}d{\Tilde}n} & \textstyleChCharisSIL{t{\Tilde}d} & [\textstyleChCharisSIL{ˈtɔ̞ŋ}] & \textitbf{tong} & ‘1\textsc{pl}’\\
&  & [\textstyleChCharisSIL{ˈdɔ̞ŋ}] & \textitbf{dong} & ‘3\textsc{pl}’\\
& \textstyleChCharisSIL{t{\Tilde}n} & [\textstyleChCharisSIL{ˈti.kɐr}] & \textitbf{tikar} & ‘plaited mat’\\
&  & [\textstyleChCharisSIL{ˈni.ka}] & \textitbf{nika} & ‘marry officially’\\
& \textstyleChCharisSIL{d{\Tilde}n} & [\textstyleChCharisSIL{ˈdɛ.kɐt̚}] & \textitbf{dekat} & ‘near’\\
&  & [\textstyleChCharisSIL{ˈnɛ.kɐt̚}] & \textitbf{nekat} & ‘be determined’\\
\textstyleChCharisSIL{k{\Tilde}g} &  & [\textstyleChCharisSIL{ˈka.ja}] & \textitbf{kaya} & ‘like’\\
&  & [\textstyleChCharisSIL{ˈga.ja}] & \textitbf{gaya} & ‘manner’\\
\textstyleChCharisSIL{tʃ{\Tilde}dʒ{\Tilde}t}/\textstyleChCharisSIL{d} & \textstyleChCharisSIL{tʃ{\Tilde}dʒ} & [\textstyleChCharisSIL{ˈtʃu.ɾɐŋ}] & \textitbf{curang} & ‘be dishonest’\\
&  & [\textstyleChCharisSIL{ˈdʒu.ɾɐŋ}] & \textitbf{jurang} & ‘steep decline’\\
& \textstyleChCharisSIL{tʃ{\Tilde}t} & [\textstyleChCharisSIL{ˈtsɐm.pʊr}] & \textitbf{campur} & ‘mix’\\
&  & [\textstyleChCharisSIL{ˈtɐm.pɐr}] & \textitbf{tampar} & ‘beat’\\
& \textstyleChCharisSIL{dʒ{\Tilde}d} & [\textstyleChCharisSIL{ˈdʒa.ɾi}] & \textitbf{jari} & ‘digit’\\
&  & [\textstyleChCharisSIL{ˈda.ɾi}] & \textitbf{dari} & ‘from’\\
\textstyleChCharisSIL{s{\Tilde}h} &  & [\textstyleChCharisSIL{ˈsɐn.tɐŋ}] & \textitbf{santang} & ‘coconut milk’\\
&  & [\textstyleChCharisSIL{ˈhɐn.tɐm}] & \textitbf{hantam} & ‘strike’\\
\textstyleChCharisSIL{m{\Tilde}n{\Tilde}ɲ} & \textstyleChCharisSIL{m{\Tilde}n} & [\textstyleChCharisSIL{ˈma.si}] & \textitbf{masi} & ‘still’\\
&  & [\textstyleChCharisSIL{ˈna.si}] & \textitbf{nasi} & ‘cooked rice’\\
& \textstyleChCharisSIL{m{\Tilde}ɲ} & [\textstyleChCharisSIL{ˈmɛ.mɐŋ}] & \textitbf{memang} & ‘indeed’\\
&  & [\textstyleChCharisSIL{ˈɲa.mɐŋ}] & \textitbf{nyamang} & ‘be comfortable’\\
& \textstyleChCharisSIL{n{\Tilde}ɲ} & [\textstyleChCharisSIL{ˈna.kɐl}] & \textitbf{nakal} & ‘be mischievous’\\
&  & [\textstyleChCharisSIL{ˈɲa.wa}] & \textitbf{nyawa} & ‘soul’\\
\textstyleChCharisSIL{l{\Tilde}r} &  & [\textstyleChCharisSIL{ˈra.wɐŋ}] & \textitbf{rawang} & ‘be haunted’\\
&  & [\textstyleChCharisSIL{ˈla.wɐŋ}] & \textitbf{lawang} & ‘oppose’\\
\textstyleChCharisSIL{j{\Tilde}ɲ} &  & [\textstyleChCharisSIL{ˈjɐŋ}] & \textitbf{yang} & ‘\textsc{rel}’\\
&  & [\textstyleChCharisSIL{ˈɲa.wa}] & \textitbf{nyawa} & ‘soul’\\
\textstyleChCharisSIL{j{\Tilde}w} &  & [\textstyleChCharisSIL{ˈjɐŋ}] & \textitbf{yang} & ‘\textsc{rel}’\\
&  & [\textstyleChCharisSIL{ˈwa.ɾʊŋ}] & \textitbf{warung} & ‘food stall’\\
\lspbottomrule
\end{tabular}
\begin{styleCaptionxbefore}
\label{bkm:Ref323303089}Table ‎2.\stepcounter{Table}{\theTable}:  Consonant contrast in root-internal position
\end{styleCaptionxbefore}

\tablehead{
\multicolumn{2}{l}{ Contrast} & Item & Orthogr. & Gloss\\
}
\begin{tabular}{lllll}
\lsptoprule
\textstyleChCharisSIL{p{\Tilde}b{\Tilde}m} & \textstyleChCharisSIL{p{\Tilde}b} & [\textstyleChCharisSIL{ˈkɛ.pʊŋ}] & \textitbf{kepung} & ‘surround’\\
&  & [\textstyleChCharisSIL{ˈkɛ.bʊŋ}] & \textitbf{kebung} & ‘garden’\\
& \textstyleChCharisSIL{p{\Tilde}m} & [\textstyleChCharisSIL{ˈra.pi}] & \textitbf{rapi} & ‘be neat’\\
&  & [\textstyleChCharisSIL{ˈra.mɛ}] & \textitbf{rame} & ‘be bustling’\\
& \textstyleChCharisSIL{b{\Tilde}m} & [\textstyleChCharisSIL{ˈsu.bʊr}] & \textitbf{subur} & ‘be fertile’\\
&  & [\textstyleChCharisSIL{ˈsu.mʊr}] & \textitbf{sumur} & ‘(a) well’\\
\textstyleChCharisSIL{t{\Tilde}d{\Tilde}n} & \textstyleChCharisSIL{t{\Tilde}d} & [\textstyleChCharisSIL{ˈhi.tʊŋ}] & \textitbf{hitung} & ‘count’\\
&  & [\textstyleChCharisSIL{ˈhi.dʊŋ}] & \textitbf{hidung} & ‘nose’\\
& \textstyleChCharisSIL{t{\Tilde}n} & [\textstyleChCharisSIL{ˈbu.tu}] & \textitbf{butu} & ‘need’\\
&  & [\textstyleChCharisSIL{ˈbu.nu}] & \textitbf{bunu} & ‘kill’\\
& \textstyleChCharisSIL{d{\Tilde}n} & [\textstyleChCharisSIL{ˈa.dɛ}] & \textitbf{ade} & ‘younger sibling’\\
&  & [\textstyleChCharisSIL{ˈa.nɛ}] & \textitbf{ane} & ‘be strange’\\
\textstyleChCharisSIL{k{\Tilde}g{\Tilde}ŋ} &  & [\textstyleChCharisSIL{ˈla.ki}] & \textitbf{laki} & ‘husband’\\
&  & [\textstyleChCharisSIL{ˈla.gi}] & \textitbf{lagi} & ‘again’\\
&  & [\textstyleChCharisSIL{ˈla.ŋɪt̚}] & \textitbf{langit} & ‘sky’\\
\textstyleChCharisSIL{tʃ{\Tilde}dʒ{\Tilde}t}/\textstyleChCharisSIL{d} & \textstyleChCharisSIL{tʃ{\Tilde}dʒ} & [\textstyleChCharisSIL{ˈbɐn.tʃi}] & \textitbf{banci} & ‘homosexual male’\\
&  & [\textstyleChCharisSIL{ˈbɐn.dʒɪr}] & \textitbf{banjir} & ‘flood’\\
& \textstyleChCharisSIL{tʃ{\Tilde}t} & [\textstyleChCharisSIL{ˈtʃa.tʃɐt̚}] & \textitbf{cacat} & ‘be disabled’\\
&  & [\textstyleChCharisSIL{ˈtʃa.tɐt̚}] & \textitbf{catat} & ‘make a note’\\
& \textstyleChCharisSIL{dʒ{\Tilde}d} & [\textstyleChCharisSIL{ˈtʊn.dʒʊk̚}] & \textitbf{tunjuk} & ‘show’\\
&  & [\textstyleChCharisSIL{ˈtʊn.dʊk̚}] & \textitbf{tunduk} & ‘bow’\\
\textstyleChCharisSIL{s{\Tilde}h} &  & [\textstyleChCharisSIL{ˈpa.sɪr}] & \textitbf{pasir} & ‘sand’\\
&  & [\textstyleChCharisSIL{ˈpa.hɪt̚}] & \textitbf{pahit} & ‘be bitter’\\
\textstyleChCharisSIL{m{\Tilde}n{\Tilde}ɲ{\Tilde}ŋ} & \textstyleChCharisSIL{m{\Tilde}n} & [\textstyleChCharisSIL{ˈmɛ.mɐŋ}] & \textitbf{memang} & ‘indeed’\\
&  & [\textstyleChCharisSIL{mɛ.ˈnɐŋ}] & \textitbf{menang} & ‘win’\\
& \textstyleChCharisSIL{m{\Tilde}ɲ} & [\textstyleChCharisSIL{ˈta.mu}] & \textitbf{tamu} & ‘guest’\\
&  & [\textstyleChCharisSIL{ˈta.ɲa}] & \textitbf{tanya} & ‘ask’\\
& \textstyleChCharisSIL{m{\Tilde}ŋ} & [\textstyleChCharisSIL{ˈla.mɐr}] & \textitbf{lamar} & ‘apply for’\\
&  & [\textstyleChCharisSIL{ˈla.ŋɐr}] & \textitbf{langar} & ‘collide with’\\
& \textstyleChCharisSIL{n{\Tilde}ɲ{\Tilde}ŋ} & [\textstyleChCharisSIL{ˈta.nɐm}] & \textitbf{tanam} & ‘plant’\\
&  & [\textstyleChCharisSIL{ˈta.ɲa}] & \textitbf{tanya} & ‘ask’\\
&  & [\textstyleChCharisSIL{ˈta.ŋɐŋ}] & \textitbf{tangang} & ‘hand’\\
\textstyleChCharisSIL{l{\Tilde}r} &  & [\textstyleChCharisSIL{ˈbu.lu}] & \textitbf{bulu} & ‘body hair’\\
&  & [\textstyleChCharisSIL{ˈbu.ɾu}] & \textitbf{buru} & ‘hunt’\\
\textstyleChCharisSIL{j{\Tilde}ɲ} &  & [\textstyleChCharisSIL{ˈa.jɐm}] & \textitbf{ayam} & ‘chicken’\\
&  & [\textstyleChCharisSIL{ˈa.ɲɐm}] & \textitbf{anyam} & ‘plait’\\
\textstyleChCharisSIL{j{\Tilde}w} &  & [\textstyleChCharisSIL{ˈla.jɐŋ}] & \textitbf{layang} & ‘serve’\\
&  & [\textstyleChCharisSIL{ˈla.wɐŋ}] & \textitbf{lawang} & ‘oppose’\\
\textstyleChCharisSIL{w{\Tilde}ŋ} &  & [\textstyleChCharisSIL{ˈba.wɐŋ}] & \textitbf{bawang} & ‘onion’\\
&  & [\textstyleChCharisSIL{ˈba.ŋʊŋ}] & \textitbf{bangung} & ‘wake up’\\
\lspbottomrule
\end{tabular}
\begin{styleCaptionxbefore}
\label{bkm:Ref323303090}Table ‎2.\stepcounter{Table}{\theTable}:  Consonant contrast in word-final position
\end{styleCaptionxbefore}

\tablehead{
\multicolumn{2}{l}{ Contrast} & Item & Orthogr. & Gloss\\
}
\begin{tabular}{lllll}
\lsptoprule
\textsc{plos}{\Tilde}\textsc{nasal} & \textstyleChCharisSIL{p{\Tilde}m} & [\textstyleChCharisSIL{ˈa.sɐp[2FA?]}] & \textitbf{asap} & ‘smoke’\\
&  & [\textstyleChCharisSIL{ˈa.sɐm}] & \textitbf{asam} & ‘sour’\\
& \textstyleChCharisSIL{t{\Tilde}ŋ} & [\textstyleChCharisSIL{ˈbu.ɐt}] & \textitbf{buat} & ‘make’\\
&  & [\textstyleChCharisSIL{ˈbu.ɐŋ}] & \textitbf{buang} & ‘discard’\\
& \textstyleChCharisSIL{k{\Tilde}ŋ} & [\textstyleChCharisSIL{ˈdʒa.ɾɐk}] & \textitbf{jarak} & ‘distance between’\\
&  & [\textstyleChCharisSIL{ˈdʒa.ɾɐŋ}] & \textitbf{jarang} & ‘rarely’\\
\textstyleChCharisSIL{l{\Tilde}r} &  & [\textstyleChCharisSIL{ˈmɐn.dʊl}] & \textitbf{mandul} & ‘be sterile’\\
&  & [\textstyleChCharisSIL{ˈmʊn.dʊr}] & \textitbf{mundur} & ‘smoke’\\
\textstyleChCharisSIL{j{\Tilde}w} &  & [\textstyleChCharisSIL{ˈtɐj}] & \textitbf{tay} & ‘excrement’\\
&  & [\textstyleChCharisSIL{ˈtɐw}] & \textitbf{taw} & ‘know’\\
\lspbottomrule
\end{tabular}
\subsection{Vowel system}
\label{bkm:Ref324760130}\paragraph[Vowel inventory]{Vowel inventory}

The Papuan Malay vowel inventory, presented in Table  ‎2 .6, consists of two front and two back vowels, and one open central vowel.


\begin{stylecaption}
\label{bkm:Ref334111306}Table ‎2.\stepcounter{Table}{\theTable}:  Papuan Malay vowel inventory
\end{stylecaption}

\begin{tabular}{llll} & \textsc{front} & \textsc{central} & \arraybslash \textsc{back}\\
\lsptoprule
\textsc{close} & \textstyleChCharisSIL{i} &  & \arraybslash \textstyleChCharisSIL{u}\\
\textsc{open-mid} & \textstyleChCharisSIL{ɛ} &  & \arraybslash \textstyleChCharisSIL{ɔ}\\
\textsc{open} &  & \textstyleChCharisSIL{a} & \\
\lspbottomrule
\end{tabular}

Three of the five vowels have three allophones each: /\textstyleChCharisSIL{i}/ can be realized as [\textstyleChCharisSIL{i}], [\textstyleChCharisSIL{ɪ}], or [\textstyleChCharisSIL{e}], /u/ as [\textstyleChCharisSIL{u}], [\textstyleChCharisSIL{ʊ}], or [\textstyleChCharisSIL{o}], and /\textstyleChCharisSIL{ɛ}/ as [\textstyleChCharisSIL{ɛ}], [\textstyleChCharisSIL{ɛ̞}], or [\textstyleChCharisSIL{ə}]. The remaining two vowels have two allophones each: /\textstyleChCharisSIL{ɔ}/ can be realized as [\textstyleChCharisSIL{ɔ}] or [\textstyleChCharisSIL{ɔ̞}], and /\textstyleChCharisSIL{a}/ as [\textstyleChCharisSIL{a}] or [\textstyleChCharisSIL{ɐ}].\footnote{\\
\\
\\
\\
The diacritic “\textstyleChCharisSILviiivpt{~~̞}” signals that the vowel is lowered.\\
\\
\\
\\
\\
\\
\\
\\
\\
\\
\\
} While the centralized allophones for the two close vowels /\textstyleChCharisSIL{i}/ and /\textstyleChCharisSIL{u}/ and for the open vowel /\textstyleChCharisSIL{a}/ are represented with distinct entries in the IPA chart, this is not the case for the open-mid vowels /\textstyleChCharisSIL{ɛ}/ and /\textstyleChCharisSIL{ɔ}/. In terms of their degree of openness, their centralized allophones [\textstyleChCharisSIL{ɛ̞}] and [\textstyleChCharisSIL{ɔ̞}] are distinctly lower than their non-centralized allophones [\textstyleChCharisSIL{ɛ}] and [\textstyleChCharisSIL{ɔ}]. They are higher, however, than the respective open-near vowels /\textstyleChCharisSIL{æ}/ and /\textstyleChCharisSIL{ɒ}/ found in other languages, as described in the “IPA chart” ({The International Phonetic Association 2005}; see also {SIL International 1996–2008}). Hence, as they lie in-between the open-mid and open-near vowels, these two allophones are represented as [\textstyleChCharisSIL{ɛ̞}] and [\textstyleChCharisSIL{ɔ̞}]. Figure  ‎2 .5 presents the vowel space for the five vowels and their allophones.\footnote{\\
\\
\\
\\
The vowel space in Figure  ‎2 .5 is based on the author’s impressions rather than on measured spectrographic data.\\
\\
\\
\\
\\
\\
\\
\\
\\
\\
\\
}


\begin{styleFigure}
  
%%please move the includegraphics inside the {figure} environment
%%\includegraphics[width=\textwidth]{kluge-img1}
 
\end{styleFigure}

\begin{styleCapFigure}
\label{bkm:Ref417128501}Figure ‎2.\stepcounter{Figure}{\theFigure}:  Vowel space for the Papuan Malay vowels
\end{styleCapFigure}


The phonological processes involved in the allophonic variation of the Papuan Malay vowels are discussed in §2.2.
\end{styleBodyxvafter}

\paragraph[Contrast between the vowel segments]{Contrast between the vowel segments}

Contrast between the five vowel segments in disyllabic lexical items is presented in minimal or near-minimal pairs in the following tables: in open stressed penultimate syllables in Table  ‎2 .7, in closed stressed penultimate syllables in Table  ‎2 .8, and in open unstressed ultimate syllables in Table  ‎2 .9. When minimal or near-minimal pairs could not be found, another word containing a contrasting vowel segment is given.


\begin{stylecaption}
\label{bkm:Ref323820394}Table ‎2.\stepcounter{Table}{\theTable}:  Vowel contrast in open stressed penultimate syllables
\end{stylecaption}

\tablehead{
 Contrast & Item & Orthogr. & Gloss\\
}
\begin{tabular}{llll}
\lsptoprule
\textstyleChCharisSIL{i{\Tilde}ɛ} & [\textstyleChCharisSIL{ˈi.kʊt̚}] & \textitbf{ikut} & ‘follow’\\
& [\textstyleChCharisSIL{ˈɛ.kɔ̞r}] & \textitbf{ekor} & ‘tail’\\
\textstyleChCharisSIL{i{\Tilde}a} & [\textstyleChCharisSIL{ˈi.ŋɪŋ}] & \textitbf{inging} & ‘wish’\\
& [\textstyleChCharisSIL{ˈa.ŋɪŋ}] & \textitbf{anging} & ‘wind’\\
\textstyleChCharisSIL{i{\Tilde}u} & [\textstyleChCharisSIL{ˈɪ.ɾɪs}] & \textitbf{iris} & ‘cut’\\
& [\textstyleChCharisSIL{ˈʊ.ɾʊs}] & \textitbf{urus} & ‘arrange’\\
\textstyleChCharisSIL{i{\Tilde}ɔ} & [\textstyleChCharisSIL{ˈi.tu}] & \textitbf{itu} & ‘\textsc{d.dist}’\\
& [\textstyleChCharisSIL{ˈɔ.tɔ̞t[2FA?]}] & \textitbf{otot} & ‘muscle’\\
\textstyleChCharisSIL{ɛ{\Tilde}a} & [\textstyleChCharisSIL{ˈɛ.dʒɛ̞k̚}] & \textitbf{ejek} & ‘mock’\\
& [\textstyleChCharisSIL{ˈa.dʒɐk}] & \textitbf{ajak} & ‘invite’\\
\textstyleChCharisSIL{ɛ{\Tilde}u} & [\textstyleChCharisSIL{ˈɛ.kɔ̞r}] & \textitbf{ekor} & ‘tail’\\
& [\textstyleChCharisSIL{ˈu.kʊr}] & \textitbf{ukur} & ‘measure’\\
\textstyleChCharisSIL{ɛ{\Tilde}ɔ} & [\textstyleChCharisSIL{ˈɛ.dʒɛ̞k[2FA?]}] & \textitbf{ejek} & ‘mock’\\
& [\textstyleChCharisSIL{ˈɔ.dʒɛ̞k[2FA?]}] & \textitbf{ojek} & ‘motorbike taxi’\\
\textstyleChCharisSIL{a{\Tilde}u} & [\textstyleChCharisSIL{ˈa.ɾa}] & \textitbf{ara} & ‘direction’\\
& [\textstyleChCharisSIL{ˈu.ɾɐt̚}] & \textitbf{urat} & ‘vein’\\
\textstyleChCharisSIL{u{\Tilde}ɔ} & [\textstyleChCharisSIL{ˈu.dʒʊŋ}] & \textitbf{ujung} & ‘end’\\
& [\textstyleChCharisSIL{ˈɔ.dʒɛ̞k[2FA?]}] & \textitbf{ojek} & ‘motorbike taxi’\\
\lspbottomrule
\end{tabular}
\begin{styleCaptionxbefore}
\label{bkm:Ref323820397}Table ‎2.\stepcounter{Table}{\theTable}:  Vowel contrast in closed stressed penultimate syllables
\end{styleCaptionxbefore}

\tablehead{
 Contrast & Item & Orthogr. & Gloss\\
}
\begin{tabular}{llll}
\lsptoprule
\textstyleChCharisSIL{i{\Tilde}u} & [\textstyleChCharisSIL{ˈmɪn.ta}] & \textitbf{minta} & ‘request’\\
& [\textstyleChCharisSIL{ˈmʊn.ta}] & \textitbf{munta} & ‘vomit’\\
\textstyleChCharisSIL{i{\Tilde}ɛ} & [\textstyleChCharisSIL{ˈtɪm.bɐŋ}] & \textitbf{timbang} & ‘weigh’\\
& [\textstyleChCharisSIL{ˈtɛ̞m.bɐk̚}] & \textitbf{tembak} & ‘shoot’\\
\textstyleChCharisSIL{i{\Tilde}a} & [\textstyleChCharisSIL{ˈtɪm.ba}] & \textitbf{timba} & ‘fetch’\\
& [\textstyleChCharisSIL{ˈtɐm.ba}] & \textitbf{tamba} & ‘add’\\
\textstyleChCharisSIL{i{\Tilde}ɔ} & [\textstyleChCharisSIL{ˈtɪŋ.kɐt[2FA?]}] & \textitbf{tingkat} & ‘level’\\
& [\textstyleChCharisSIL{ˈtɔ̞ŋ.kɐt[2FA?]}] & \textitbf{tongkat} & ‘cane’\\
\textstyleChCharisSIL{ɛ{\Tilde}a} & [\textstyleChCharisSIL{ˈsɛ̞n.tu}] & \textitbf{sentu} & ‘touch’\\
& [\textstyleChCharisSIL{ˈsɐn.tɛ}] & \textitbf{sante} & ‘relax’\\
\textstyleChCharisSIL{ɛ{\Tilde}u} & [\textstyleChCharisSIL{ˈtɛ̞m.bɐk̚}] & \textitbf{tembak} & ‘shoot’\\
& [\textstyleChCharisSIL{ˈtʊm.bʊk̚}] & \textitbf{tumbuk} & ‘pound’\\
\textstyleChCharisSIL{ɛ{\Tilde}ɔ} & [\textstyleChCharisSIL{ˈbɛ̞ŋ.kɔ̞k̚}] & \textitbf{bengkok} & ‘be crooked’\\
& [\textstyleChCharisSIL{ˈbɔ̞ŋ.kɔ̞k̚}] & \textitbf{bongkok} & ‘be bent over’\\
\textstyleChCharisSIL{a{\Tilde}u} & [\textstyleChCharisSIL{ˈbɐn.tu}] & \textitbf{bantu} & ‘help’\\
& [\textstyleChCharisSIL{ˈbʊn.tu}] & \textitbf{buntu} & ‘be blocked’\\
\textstyleChCharisSIL{a{\Tilde}ɔ} & [\textstyleChCharisSIL{ˈsɐm.bʊŋ}] & \textitbf{sambung} & ‘continue’\\
& [\textstyleChCharisSIL{ˈsɔ̞m.bɔ̞ŋ}] & \textitbf{sombong} & ‘be arrogant’\\
\textstyleChCharisSIL{u{\Tilde}ɔ} & [\textstyleChCharisSIL{ˈsʊm.bɐŋ}] & \textitbf{sumbang} & ‘donate’\\
& [\textstyleChCharisSIL{ˈsɔ̞m.bɔ̞ŋ}] & \textitbf{sombong} & ‘be arrogant’\\
\lspbottomrule
\end{tabular}
\begin{styleCaptionxbefore}
\label{bkm:Ref323820398}Table ‎2.\stepcounter{Table}{\theTable}:  Vowel contrast in open unstressed syllables
\end{styleCaptionxbefore}

\tablehead{
 Contrast & Item & Orthogr. & Gloss\\
}
\begin{tabular}{llll}
\lsptoprule
\textstyleChCharisSIL{i{\Tilde}ɛ} & [\textstyleChCharisSIL{ˈpɛ.lɛ}] & \textitbf{pele} & ‘cover\\
& [\textstyleChCharisSIL{ˈpi.li}] & \textitbf{pili} & ‘choose\\
\textstyleChCharisSIL{i{\Tilde}a} & [\textstyleChCharisSIL{ˈka.li}] & \textitbf{kali} & ‘river’\\
& [\textstyleChCharisSIL{ˈka.la}] & \textitbf{kala} & ‘be defeated’\\
\textstyleChCharisSIL{i{\Tilde}u} & [\textstyleChCharisSIL{ˈla.gi}] & \textitbf{lagi} & ‘again’\\
& [\textstyleChCharisSIL{ˈla.gu}] & \textitbf{lagu} & ‘song’\\
\textstyleChCharisSIL{i{\Tilde}ɔ} & [\textstyleChCharisSIL{ˈba.bi}] & \textitbf{kali} & ‘river’\\
& [\textstyleChCharisSIL{ˈbɔ.bɔ}] & \textitbf{bobo} & ‘palm liquor’\\
\textstyleChCharisSIL{ɛ{\Tilde}u} & [\textstyleChCharisSIL{ˈpa.kɛ}] & \textitbf{pake} & ‘use’\\
& [\textstyleChCharisSIL{ˈpa.ku}] & \textitbf{paku} & ‘nail’\\
\textstyleChCharisSIL{ɛ{\Tilde}ɔ} & [\textstyleChCharisSIL{ˈga.lɛ̞}] & \textitbf{gale} & ‘dig up’\\
& [\textstyleChCharisSIL{ˈga.ɾɔ}] & \textitbf{garo} & ‘scratch’\\
\textstyleChCharisSIL{a{\Tilde}u} & [\textstyleChCharisSIL{ˈbi.sa}] & \textitbf{bisa} & ‘be able’\\
& [\textstyleChCharisSIL{ˈbi.su}] & \textitbf{bisu} & ‘mute’\\
\textstyleChCharisSIL{u{\Tilde}ɔ} & [\textstyleChCharisSIL{ˈtu.bu}] & \textitbf{tubu} & ‘body’\\
& [\textstyleChCharisSIL{ˈtɔ.bɔ}] & \textitbf{tobo} & ‘dive’\\
\lspbottomrule
\end{tabular}
\section{Phonological processes}
\label{bkm:Ref332787528}
In Papuan Malay, two phonological processes are attested for the consonants and one for the vowels: nasal place assimilation (§2.2.1), tap/trill alternation of the alveolar rhotic (§2.2.2), and centralization of vowels (§2.2.3).
\end{styleBodyxvafter}

\subsection{Nasal place assimilation}
\label{bkm:Ref358302308}\label{bkm:Ref358302307}
Nasal place assimilation applies to nasals as coda in the domain of the prosodic word. While all four nasals occur in the onset position (although velar /\textstyleChCharisSIL{ŋ}/ only occurs in the word-internal onset position), only two nasals occur as coda, namely bilabial /\textstyleChCharisSIL{m}/ and velar /\textstyleChCharisSIL{ŋ}/, as shown in Table  ‎2 .10. The velar nasal as a coda assimilates in place of articulation to a following stop or affricate. When preceding the alveolar fricative, the nasal is always realized as velar [\textstyleChCharisSIL{ŋ}], as in \textitbf{bongso} ‘youngest offspring’ or \textitbf{langsung} ‘immediately’. These patterns agree with {Padgett’s (1994: 489)} cross-linguistic findings that nasals either do “not assimilate in place to fricatives” or that such assimilation is, at least, “highly disfavored, while assimilation to stops and affricates is pervasive”. (See also {de Lacy 2006: 146–147;} {Zsiga 2006: 554; Blust 2012}.) An exception to these patterns of nasal assimilation is the prefix \textscItalBold{pe(n)\-} ‘\textsc{ag}’ (§3.1.4). When preceding the alveolar fricative /s/, the nasal is not realized as alveolar [\textstyleChCharisSIL{n}] but as palatal [\textstyleChCharisSIL{ɲ}], as in \textitbf{penyakit} [\textstyleChCharisSIL{pɛ̞ɲ}–\textstyleChCharisSIL{sakɪt}] ‘disease’, with /\textstyleChCharisSIL{s}/ being deleted (see also {Blust 2012}; for the allomorphy of \textscItalBold{pe(n)\-} see §3.1.4.1).
\end{styleBodyafterxivpt}


Cross-linguistically, the preservation of the bilabial nasal is not unusual, as {de Lacy (2006: 78–207)} points out. It is due to the fact, that on the “Place of Articulation” hierarchy, the labial nasal is more marked that the dental or velar ones {(2006: 129)}. Such marked elements “can be specifically targeted for preservation. Consequently, highly marked elements can survive a process that less-marked elements undergo” {(2006: 146)}.\footnote{\\
\\
\\
\\
One anonymous reviewer suggests, however, a different analysis. Given that the nasal in this position obtains its place features from the following segment, not two, but only one nasal phoneme (or ‘archiphoneme’) occurs in the word-internal coda position.\\
\\
\\
\\
\\
\\
\\
\\
\\
\\
\\
}


\begin{stylecaption}
\label{bkm:Ref330540106}Table ‎2.\stepcounter{Table}{\theTable}:  Nasal place assimilation in the word-internal coda position
\end{stylecaption}

\tablehead{
 Phoneme & Realization & Item & Orthogr. & \arraybslash Gloss\\
}
\begin{tabular}{lllll}
\lsptoprule
/\textstyleChCharisSIL{m}/ & [\textstyleChCharisSIL{m}] & [\textstyleChCharisSIL{ˈsɪ}\textstyleChCharisSILBlueBold{m}\textstyleChCharisSIL{.}\textstyleChCharisSILBlueBold{p}\textstyleChCharisSIL{ɐŋ}] & \textitbf{simpang} & ‘store’\\
&  & [\textstyleChCharisSIL{kɛ̞}\textstyleChCharisSILBlueBold{m}\textstyleChCharisSIL{.ˈ}\textstyleChCharisSILBlueBold{b}\textstyleChCharisSIL{a.li}] & \textitbf{kembali} & ‘return’\\
/\textstyleChCharisSIL{ŋ}/ & [n] & [\textstyleChCharisSIL{ˈmɪ}\textstyleChCharisSILBlueBold{n}\textstyleChCharisSIL{.}\textstyleChCharisSILBlueBold{t}\textstyleChCharisSIL{a}] & \textitbf{minta} & ‘ask’\\
&  & [\textstyleChCharisSIL{ˈmɐ}\textstyleChCharisSILBlueBold{n}\textstyleChCharisSIL{.}\textstyleChCharisSILBlueBold{d}\textstyleChCharisSIL{i}] & \textitbf{mandi} & ‘bathe’\\
&  & [\textstyleChCharisSIL{ˈhɐ}\textstyleChCharisSILBlueBold{n}\textstyleChCharisSIL{.}\textstyleChCharisSILBlueBold{tʃ}\textstyleChCharisSIL{ʊr}] & \textitbf{hancur} & ‘be shattered’\\
&  & [\textstyleChCharisSIL{ˈɪ}\textstyleChCharisSILBlueBold{n}\textstyleChCharisSIL{.}\textstyleChCharisSILBlueBold{dʒ}\textstyleChCharisSIL{ɐk̚}] & \textitbf{injak} & ‘step on’\\
& [\textstyleChCharisSIL{ŋ}] & [\textstyleChCharisSIL{ˈɐ}\textstyleChCharisSILBlueBold{ŋ}\textstyleChCharisSIL{.}\textstyleChCharisSILBlueBold{k}\textstyleChCharisSIL{ɐt̚}] & \textitbf{angkat} & ‘pick-up’\\
&  & [\textstyleChCharisSIL{ˈtɪ}\textstyleChCharisSILBlueBold{ŋ}\textstyleChCharisSIL{.}\textstyleChCharisSILBlueBold{g}\textstyleChCharisSIL{i}] & \textitbf{tinggi} & ‘be tall’\\
&  & [\textstyleChCharisSIL{ˈbɔ̞}\textstyleChCharisSILBlueBold{ŋ}\textstyleChCharisSIL{.}\textstyleChCharisSILBlueBold{s}\textstyleChCharisSIL{ɔ̞}] & \textitbf{bongso} & ‘youngest offspring’\\
&  & [\textstyleChCharisSIL{ˈlɐ}\textstyleChCharisSILBlueBold{ŋ}\textstyleChCharisSIL{.}\textstyleChCharisSILBlueBold{s}\textstyleChCharisSIL{ʊŋ}] & \textitbf{langsung} & ‘immediately’\\
\lspbottomrule
\end{tabular}

Nasal place assimilation also occurs across word boundaries, when the nasal is in the word-final coda position, as shown in Table  ‎2 .11. While bilabial /\textstyleChCharisSIL{m}/ is preserved, velar /\textstyleChCharisSIL{ŋ}/ assimilates in place of articulation to a following stop or affricate, similar to the processes illustrated in Table  ‎2 .10. When preceding a fricative-initial or vowel-initial word, or when occurring before a pause or at the end of an utterance, by contrast, the velar nasal is most commonly realized as velar [\textstyleChCharisSIL{ŋ}]. In Table  ‎2 .11, this is illustrated with \textitbf{minum} ‘drink’, \textitbf{biking} ‘make’ and \textitbf{bilang} ‘say’. Overall, however, assimilation across word boundaries is applied less often than within the prosodic word.


\begin{stylecaption}
\label{bkm:Ref330230611}Table ‎2.\stepcounter{Table}{\theTable}:  Nasal place assimilation in the word-final coda position
\end{stylecaption}

\tablehead{
 Phoneme & Item & Orthogr. & \arraybslash Gloss\\
}
\begin{tabular}{llll}
\lsptoprule
/\textstyleChCharisSIL{m}/ & [\textstyleChCharisSIL{ˈmi.nʊ}\textstyleChCharisSILBlueBold{m}\textstyleChCharisSIL{ ˈ}\textstyleChCharisSILBlueBold{b}\textstyleChCharisSIL{ɔ.bɔ}] & \textitbf{minum bobo} & ‘drink schnapps’\\
& [\textstyleChCharisSIL{ˈmi.nʊ}\textstyleChCharisSILBlueBold{m}\textstyleChCharisSIL{ ˈ}\textstyleChCharisSILBlueBold{d}\textstyleChCharisSIL{u.lu}] & \textitbf{minum dulu} & ‘drink first’\\
& [… \textstyleChCharisSIL{ˈmi.nʊ}\textstyleChCharisSILBlueBold{m}\textstyleChCharisSIL{ ˈ}\textstyleChCharisSILBlueBold{k}\textstyleChCharisSIL{i.ˈtɔ̞ŋ}] & \textitbf{… minum kitong} & ‘(give) us to drink’\\
& [\textstyleChCharisSIL{ˈmi.nʊ}\textstyleChCharisSILBlueBold{m}\textstyleChCharisSIL{ ˈ}\textstyleChCharisSILBlueBold{i}\textstyleChCharisSIL{.tu}] & \textitbf{minum itu} & ‘drink that’\\
& [\textstyleChCharisSIL{ˈmi.nʊ}\textstyleChCharisSILBlueBold{m}\textstyleChCharisSIL{, ˈ}\textstyleChCharisSILBlueBold{t}\textstyleChCharisSIL{a.pi}] & \textitbf{minum, tapi} & ‘drink, but’\\
/\textstyleChCharisSIL{ŋ}/ & [\textstyleChCharisSIL{ˈbi.kɪ}\textstyleChCharisSILBlueBold{m}\textstyleChCharisSIL{ ˈ}\textstyleChCharisSILBlueBold{b}\textstyleChCharisSIL{a.gʊs}] & \textitbf{biking bagus} & ‘make good’\\
& [\textstyleChCharisSIL{ˈbi.kɪ}\textstyleChCharisSILBlueBold{n}\textstyleChCharisSIL{ ˈ}\textstyleChCharisSILBlueBold{d}\textstyleChCharisSIL{ɪ.a}] & \textitbf{biking dia} & ‘make him/her’\\
& [\textstyleChCharisSIL{ˈbi.kɪ}\textstyleChCharisSILBlueBold{ŋ}\textstyleChCharisSIL{ ˈ}\textstyleChCharisSILBlueBold{k}\textstyleChCharisSIL{ɔ.tɔ̞r}] & \textitbf{biking kotor} & ‘make dirty’\\
& [\textstyleChCharisSIL{ˈbi.kɪ}\textstyleChCharisSILBlueBold{ŋ}\textstyleChCharisSIL{ ˈ}\textstyleChCharisSILBlueBold{s}a] & \textitbf{biking sa} & ‘make me’\\
& [\textstyleChCharisSIL{ˈbi.kɪ}\textstyleChCharisSILBlueBold{ŋ}\textstyleChCharisSIL{ ˈ}\textstyleChCharisSILBlueBold{a}\textstyleChCharisSIL{.pa}] & \textitbf{biking apa} & ‘make what’\\
& [\textstyleChCharisSIL{ˈbi.kɪ}\textstyleChCharisSILBlueBold{ŋ}\textstyleChCharisSIL{, ˈ}\textstyleChCharisSILBlueBold{m}\textstyleChCharisSIL{ɛ.mɐŋ}] & \textitbf{biking, memang} & ‘make, indeed’\\
/\textstyleChCharisSIL{ŋ}/ & [\textstyleChCharisSIL{ˈbi.lɐ}\textstyleChCharisSILBlueBold{m}\textstyleChCharisSIL{ ˈ}\textstyleChCharisSILBlueBold{b}\textstyleChCharisSIL{a.pa}] & \textitbf{bilang bapa} & ‘tell father’\\
& [\textstyleChCharisSIL{ˈbi.lɐ}\textstyleChCharisSILBlueBold{n}\textstyleChCharisSIL{ ˈ}\textstyleChCharisSILBlueBold{d}\textstyleChCharisSIL{ɪ.a}] & \textitbf{bilang dia} & ‘tell him/her’\\
& [\textstyleChCharisSIL{ˈbi.lɐ}\textstyleChCharisSILBlueBold{ŋ}\textstyleChCharisSIL{ ˈ}\textstyleChCharisSILBlueBold{k}\textstyleChCharisSIL{a.ka}] & \textitbf{bilang kaka} & ‘tell older sibling’\\
& [\textstyleChCharisSIL{ˈbi.lɐ}\textstyleChCharisSILBlueBold{ŋ}\textstyleChCharisSIL{ ˈ}\textstyleChCharisSILBlueBold{s}a\textstyleChCharisSIL{.ma}] & \textitbf{bilang sama} & ‘say to’\\
& [\textstyleChCharisSIL{ˈbi.lɐ}\textstyleChCharisSILBlueBold{ŋ}\textstyleChCharisSIL{ ˈ}\textstyleChCharisSILBlueBold{i}\textstyleChCharisSIL{.ni}] & \textitbf{bilang ini} & ‘say this’\\
& [\textstyleChCharisSIL{ˈbi.lɐ}\textstyleChCharisSILBlueBold{ŋ}\textstyleChCharisSIL{, ˈ}\textstyleChCharisSILBlueBold{b}\textstyleChCharisSIL{lʊm}] & \textitbf{bilang, blum} & ‘say, not yet’\\
\lspbottomrule
\end{tabular}

In summary, the data presented in Table  ‎2 .10 and Table  ‎2 .11 show that Papuan Malay has only two underlying nasals in the coda position, namely bilabial /\textstyleChCharisSIL{m}/ and velar /\textstyleChCharisSIL{ŋ}/, with the latter assimilating to a following stop or affricate.


\subsection{Tap/trill alternation of the alveolar rhotic}
\label{bkm:Ref337886849}
The rhotic /r/ is most commonly realized as the voiced alveolar trill [\textstyleChCharisSIL{r}]. In inter-vocalic position, however, the rhotic is realized as the voiced tap [\textstyleChCharisSIL{ɾ}] as illustrated in (0) and Table  ‎2 .12.\footnote{\\
\\
\\
\\
In the examples in this chapter, the first line gives the orthographic representation, while the second lines give the IPA transcription.\\
\\
\\
\\
\\
\\
\\
\\
\\
\\
\\
} In the C\textsubscript{2} position in CC clusters, the rhotic is also most commonly realized as the voiced trill [r]. The voiced tap, however, is also quiet common in this position.


\begin{tabular}{lllllllllllllll}
\lsptoprule
\label{bkm:Ref331238799}
\gll {ta} {\multicolumn{2}{l}{pake}} {…} {\multicolumn{2}{l}{garam}} {\multicolumn{3}{l}{srey}} {\multicolumn{2}{l}{rica}} {…} {daging} {ini}\\ %
& ta & \multicolumn{2}{l}{ pake} & … & \multicolumn{2}{l}{ ga\bluebold{ɾ}ɐɐɐm} & \multicolumn{3}{l}{ s\bluebold{r}ej} & \multicolumn{2}{l}{ \bluebold{r}itʃaaa} & … & dagɪŋ & ini\\
& \textsc{1pl} & \multicolumn{2}{l}{take} &  & \multicolumn{2}{l}{salt} & \multicolumn{3}{l}{lemongrass} & \multicolumn{2}{l}{red pepper} &  & meat & \textsc{d.prox}\\
& \multicolumn{2}{l}{saja} & \multicolumn{2}{l}{asar} & dia & \multicolumn{2}{l}{kasi} & kring & \multicolumn{2}{l}{di } & \multicolumn{4}{l}{para-para}\\
& \multicolumn{2}{l}{ sa\textstyleChCharisSIL{dʒ}a} & \multicolumn{2}{l}{ as\textstyleChCharisSIL{ɐ}\bluebold{r}} & dia & \multicolumn{2}{l}{ kase} & k\bluebold{ɾ}ɪŋ & \multicolumn{2}{l}{ di} & \multicolumn{4}{l}{ pa\bluebold{ɾ}apa\bluebold{ɾ}a}\\
& \multicolumn{2}{l}{\textsc{1sg}} & \multicolumn{2}{l}{smoke} & \textsc{3sg} & \multicolumn{2}{l}{give} & be.dry & \multicolumn{2}{l}{at} & \multicolumn{4}{l}{platform}\\
\lspbottomrule
\end{tabular}
\ea
\glt 
‘we used … \bluebold{salt}, \bluebold{lemongrass}, \bluebold{red pepper}, … this (pig) meat, I \bluebold{smoked} it (and) \bluebold{dried} (it) on a \bluebold{platform}’ \textstyleExampleSource{[080919-004-NP.0037-0038]}
\z

\begin{stylecaption}
\label{bkm:Ref332110989}Table ‎2.\stepcounter{Table}{\theTable}:  Tap/trill alternation of rhotic /\textstyleChCharisSIL{r}/
\end{stylecaption}

\tablehead{
 Realization & Item & Orthogr. & \arraybslash Gloss\\
}
\begin{tabular}{llll}
\lsptoprule
[\textstyleChCharisSIL{r}] & [\textstyleChCharisSIL{ˈ}\textstyleChCharisSILBlueBold{r}\textstyleChCharisSIL{a.kʊs}] & \textitbf{rakus} & ‘be greedy’\\
& [\textstyleChCharisSIL{ˈk}\textstyleChCharisSILBlueBold{r}\textstyleChCharisSIL{i.ŋɐt̚}] & \textitbf{kringat} & ‘sweat’\\
& [\textstyleChCharisSIL{ˈmʊ}\textstyleChCharisSILBlueBold{r}\textstyleChCharisSIL{.ni}] & \textitbf{murni} & ‘be pure’\\
& [\textstyleChCharisSIL{ˈdʒɐŋ.k}\textstyleChCharisSILBlueBold{r}\textstyleChCharisSIL{ɪk̚}] & \textitbf{jangkrik} & ‘cricket’\\
[\textstyleChCharisSIL{ɾ}] & [\textstyleChCharisSIL{ˈba.}\textstyleChCharisSILBlueBold{ɾ}\textstyleChCharisSIL{ɐŋ}] & \textitbf{barang} & ‘stuff’\\
& [\textstyleChCharisSIL{ˈgɔ.}\textstyleChCharisSILBlueBold{ɾ}\textstyleChCharisSIL{ɛ̞ŋ}] & \textitbf{goreng} & ‘fry’\\
& [\textstyleChCharisSIL{ˈʊ.}\textstyleChCharisSILBlueBold{ɾ}\textstyleChCharisSIL{ʊs}] & \textitbf{urus} & ‘arrange’\\
\lspbottomrule
\end{tabular}
\subsection{Centralization of vowels}
\label{bkm:Ref338085454}\label{bkm:Ref338057372}
In closed syllables the five vowels are centralized. Close /\textstyleChCharisSIL{i}/ is centralized to [\textstyleChCharisSIL{ɪ}] and /\textstyleChCharisSIL{u}/ to [\textstyleChCharisSIL{ʊ}], open-mid /\textstyleChCharisSIL{ɛ}/ is centralized to [\textstyleChCharisSIL{ɛ̞}] and /\textstyleChCharisSIL{ɔ}/ to [\textstyleChCharisSIL{ɔ̞}], and open /\textstyleChCharisSIL{a}/ is centralized to [\textstyleChCharisSIL{ɐ}], as illustrated in Table  ‎2 .13. In unstressed closed syllables with a coda nasal, open-mid /\textstyleChCharisSIL{ɛ}/ can alternatively be centralized to [\textstyleChCharisSIL{ə}] rather than to [\textstyleChCharisSIL{ɛ̞}].


\begin{stylecaption}
\label{bkm:Ref324250297}Table ‎2.\stepcounter{Table}{\theTable}:  Vowel centralization in closed syllables
\end{stylecaption}

\tablehead{
 Phoneme & Realization & Item & Orthogr. & \arraybslash Gloss\\
}
\begin{tabular}{lllll}
\lsptoprule
/\textstyleChCharisSIL{i}/ & [\textstyleChCharisSIL{ɪ}] & [\textstyleChCharisSIL{ˈt}\textstyleChCharisSILBlueBold{ɪ}\textstyleChCharisSIL{ŋ.gi}] & \textitbf{tinggi} & ‘be high’\\
&  & [\textstyleChCharisSIL{ˈa.d}\textstyleChCharisSILBlueBold{ɪ}\textstyleChCharisSIL{l}] & \textitbf{adil} & ‘be fair’\\
/\textstyleChCharisSIL{u}/ & [\textstyleChCharisSIL{ʊ}] & [\textstyleChCharisSIL{ˈb}\textstyleChCharisSILBlueBold{ʊ}\textstyleChCharisSIL{ŋ.kʊs}] & \textitbf{bungkus} & ‘pack’\\
&  & [\textstyleChCharisSIL{ˈi.k}\textstyleChCharisSILBlueBold{ʊ}\textstyleChCharisSIL{t̚}] & \textitbf{ikut} & ‘follow’\\
/\textstyleChCharisSIL{ɛ}/ & [\textstyleChCharisSIL{ɛ̞}] & [\textstyleChCharisSIL{ˈg}\textstyleChCharisSILBlueBold{ɛ̞}\textstyleChCharisSIL{n.dɔ̞ŋ}] & \textitbf{gendong} & ‘hold’\\
&  & [\textstyleChCharisSIL{ˈdɔ.ŋ}\textstyleChCharisSILBlueBold{ɛ̞}\textstyleChCharisSIL{ŋ}] & \textitbf{dongeng} & ‘legend’\\
& [\textstyleChCharisSIL{ə}] & [\textstyleChCharisSILBlueBold{ə}\textstyleChCharisSIL{m.ˈpɐt̚}] & \textitbf{empat} & ‘four’\\
&  & [\textstyleChCharisSIL{s}\textstyleChCharisSILBlueBold{ə}\textstyleChCharisSIL{m.ˈbi.lɐŋ}] & \textitbf{sembilang} & ‘nine’\\
/\textstyleChCharisSIL{ɔ}/ & [\textstyleChCharisSIL{ɔ̞}] & [\textstyleChCharisSIL{ˈl}\textstyleChCharisSILBlueBold{ɔ̞}\textstyleChCharisSIL{m.ba}] & \textitbf{lomba} & ‘contest’\\
&  & [\textstyleChCharisSIL{ˈbɛ.l}\textstyleChCharisSILBlueBold{ɔ̞}\textstyleChCharisSIL{k̚}] & \textitbf{belok} & ‘turn’\\
/\textstyleChCharisSIL{a}/ & [\textstyleChCharisSIL{ɐ}] & [\textstyleChCharisSIL{ˈ}\textstyleChCharisSILBlueBold{ɐ}\textstyleChCharisSIL{n.dʒɪŋ}] & \textitbf{anjing} & ‘dog’\\
&  & [\textstyleChCharisSIL{ˈbɪn.t}\textstyleChCharisSILBlueBold{ɐ}\textstyleChCharisSIL{ŋ}] & \textitbf{bintang} & ‘star’\\
\lspbottomrule
\end{tabular}
\section{Phonetic processes}
\label{bkm:Ref338492444}
In Papuan Malay, a number of phonetic processes occur in addition to the predictable phonological processes described in §2.2. These surface phenomena involve unpredictable variation. For the consonants, the following phenomena are attested: lenition of the stops and the voiced affricates as well as fortition of the voiceless affricate and the palatal approximant (§2.3.1.1), elision of the voiceless stops, the alveolar fricative, the velar nasal, and the liquids (§2.3.1.2), and devoicing of the alveolar rhotic (§2.3.1.3). The vowels can undergo the following phonetic processes: centralization and lowering (§2.3.2.1), nasalization (§2.3.2.2), and lengthening (§2.3.2.3). In addition, this section includes a discussion on alternative realizations of the VC sequences /\textstyleChCharisSIL{aj}/ and /\textstyleChCharisSIL{aw}/ (§2.3.3)


\subsection{Phonetic processes for consonants}
\paragraph[Lenition and fortition]{Lenition and fortition}
\label{bkm:Ref338065059}
Lenition, or weakening, is attested for the stops and affricates and can occur in word-internal inter-vocalic position, and word-initial position. Fortition, or strengthening, occurs very rarely and is only attested for the voiceless affricate and the palatal approximant as word-initial onset.



Inter-vocalically, the stops and the voiced affricate can be lenited by means of spirantization to fricatives, as illustrated in Table  ‎2 .14: /\textstyleChCharisSIL{p}/ is lenited to [\textstyleChCharisSIL{ɸ}], /\textstyleChCharisSIL{b}/ to [$\beta $], /\textstyleChCharisSIL{d}/ to [\textstyleChCharisSIL{ð}], /\textstyleChCharisSIL{k}/ to [\textstyleChCharisSIL{x}], /\textstyleChCharisSIL{g}/ to [\textstyleChCharisSIL{ɣ}], and /\textstyleChCharisSIL{d}\textstyleChCharisSIL{ʒ}/ to [\textstyleChCharisSIL{ʝ}]. This process is unattested, however, for the voiceless alveolar and palato-alveolar segments. The voiceless affricate /\textstyleChCharisSIL{t}\textstyleChCharisSIL{ʃ}/ can be lenited to the palatal approximant [\textstyleChCharisSIL{j}], while lenition of alveolar /\textstyleChCharisSIL{t}/ is unattested.


\begin{stylecaption}
\label{bkm:Ref323731610}Table ‎2.\stepcounter{Table}{\theTable}:  Lenition of stops and affricates in word-internal inter-vocalic position
\end{stylecaption}

\tablehead{
 Phoneme & Item & Orthogr. & \arraybslash Gloss\\
}
\begin{tabular}{llll}
\lsptoprule
/\textstyleChCharisSIL{p}/ & [\textstyleChCharisSIL{ˈba.}\textstyleChCharisSILBlueBold{ɸ}\textstyleChCharisSIL{a}] & \textitbf{bapa} & ‘father’\\
/\textstyleChCharisSIL{b}/ & [\textstyleChCharisSIL{ˈsa.}\textstyleChCharisSILBlueBold{$\beta $}\textstyleChCharisSIL{ɐr}] & \textitbf{sabar} & ‘be patient’\\
/\textstyleChCharisSIL{d}/ & [\textstyleChCharisSIL{ˈsʊ.}\textstyleChCharisSILBlueBold{ð}\textstyleChCharisSIL{a}] & \textitbf{suda} & ‘already’\\
/\textstyleChCharisSIL{k}/ & [\textstyleChCharisSIL{ˈma.}\textstyleChCharisSILBlueBold{x}\textstyleChCharisSIL{ɐŋ}] & \textitbf{makang} & ‘eat’\\
/\textstyleChCharisSIL{g}/ & [\textstyleChCharisSIL{ˈba.}\textstyleChCharisSILBlueBold{ɣ}\textstyleChCharisSIL{i}] & \textitbf{bagi} & ‘divide’\\
/\textstyleChCharisSIL{d}\textstyleChCharisSIL{ʒ}/ & [\textstyleChCharisSIL{ˈsa.}\textstyleChCharisSILBlueBold{ʝ}\textstyleChCharisSIL{a}] & \textitbf{saja} & ‘just’\\
/\textstyleChCharisSIL{t}\textstyleChCharisSIL{ʃ}/ & [\textstyleChCharisSIL{ˈpa.}\textstyleChCharisSILBlueBold{j}\textstyleChCharisSIL{ɛ}] & \textitbf{pace} & ‘man’\\
\lspbottomrule
\end{tabular}

Most of the stops and the voiced affricate can also be lenited in word-initial position when following a word with final vowel. In this environment, however, lenition of the voiced affricate occurs less often than lenition of the stops. Inter-vocalically across word-boundaries, the word-initial obstruents are lenited to the same fricatives as word-internally, as shown in Table  ‎2 .15. Also, /\textstyleChCharisSIL{p}/ can be lenited to [\textstyleChCharisSIL{f}], and /\textstyleChCharisSIL{d}/ and /\textstyleChCharisSIL{d}\textstyleChCharisSIL{ʒ}/ can be lenited to [\textstyleChCharisSIL{j}]. Word-initial lenition to a fricative is also attested for /\textstyleChCharisSIL{b}/, /\textstyleChCharisSIL{d}/, and /\textstyleChCharisSIL{k}/ when following a nasal. In this environment, /\textstyleChCharisSIL{d}/ can also be lenited to [\textstyleChCharisSIL{n}]. Again, lenition to a fricative is unattested for the voiceless alveolar and palato-alveolar segments. Likewise, lenition in word-initial position is unattested for /\textstyleChCharisSIL{g}/.\footnote{\\
\\
\\
\\
One lexical item in particular undergoes lenition of its word-initial stop: the long and the short forms of the third person singular pronoun, \textitbf{dia}/\textitbf{de} ‘\textsc{3sg}’. Onset /\textstyleChCharisSILviiivpt{d}/ can be lenited to [\textstyleChCharisSILviiivpt{j}] when following a lexical item with a voiceless stop, the alveolar fricative /\textstyleChCharisSILviiivpt{s}/, or the rhotic /\textstyleChCharisSILviiivpt{r}/ in word-final coda position.\\
\\
\\
\\
\\
\\
\\
\\
\\
\\
\\
}


\begin{stylecaption}
\label{bkm:Ref330281655}Table ‎2.\stepcounter{Table}{\theTable}:  Lenition of stops and affricates in word-initial position
\end{stylecaption}

\tablehead{
 Phoneme & Item & Orthogr. & \arraybslash Gloss\\
}
\begin{tabular}{llll}
\lsptoprule
/\textstyleChCharisSIL{p}/ & [\textstyleChCharisSIL{ˈdɛ }\textstyleChCharisSILBlueBold{ɸ}\textstyleChCharisSIL{u}] & \textitbf{de pu} & ‘his (grandson)’\\
&  & \textsc{3sg} \textsc{poss} & \\
& [\textstyleChCharisSIL{ˈdi.a ˈ}\textstyleChCharisSILBlueBold{f}\textstyleChCharisSIL{luŋ.ku}] & \textitbf{dia palungku} & ‘he punched’\\
&  & \textsc{3sg} punch & \\
/\textstyleChCharisSIL{b}/ & [\textstyleChCharisSIL{ˈjɛ ˈ}\textstyleChCharisSILBlueBold{$\beta $}\textstyleChCharisSIL{i.lɐŋ}] & \textitbf{de bilang} & ‘he/she said’\\
&  & \textsc{3sg} say & \\
& [\textstyleChCharisSIL{ˈdʒa.rɪm ˈ}\textstyleChCharisSILBlueBold{$\beta $}\textstyleChCharisSIL{ɔ.lɛ}] & \textitbf{jaring bole} & (the) net (is) permitted\\
&  & net may & \\
/\textstyleChCharisSIL{d}/ & [\textstyleChCharisSIL{mˈla, ɛ ˈ}\textstyleChCharisSILBlueBold{ð}\textstyleChCharisSIL{ɛ̞p̚}] & \textitbf{mulay, eh dep} & ‘(he) started, uh his’\\
&  & start uh \textsc{3sg:poss} & \\
& [\textstyleChCharisSIL{ˈsa.dʒa }\textstyleChCharisSILBlueBold{j}\textstyleChCharisSIL{ɛ̞.ˈŋɐŋ}] & \textitbf{saja dengang} & ‘just with’\\
&  & just with & \\
& [\textstyleChCharisSIL{ˈspʊl ˈba.ðɐn ˈ}\textstyleChCharisSILBlueBold{ð}\textstyleChCharisSIL{i}] & \textitbf{spul badang di} & ‘wash (your) body in’\\
&  & wash body at & \\
& [\textstyleChCharisSIL{ˈki.tɔ̞n ˈ}\textstyleChCharisSILBlueBold{n}\textstyleChCharisSIL{u.a}] & \textitbf{kitong dua} & ‘we two’\\
&  & \textsc{1pl} two & \\
/\textstyleChCharisSIL{k}/ & [\textstyleChCharisSIL{ˈa.dɛ̞.ˈ}\textstyleChCharisSILBlueBold{x}\textstyleChCharisSIL{a.xa}] & \textitbf{ade-kaka} & ‘siblings’\\
&  & ySb oSb & \\
& [\textstyleChCharisSIL{dɛ̞.ˈŋɐŋ ˈ}\textstyleChCharisSILBlueBold{x}\textstyleChCharisSIL{a.xa}] & \textitbf{dengang kaka} & ‘with (the) older sibling’\\
&  & with oSb & \\
/\textstyleChCharisSIL{d}\textstyleChCharisSIL{ʒ}/ & [\textstyleChCharisSIL{ˈsa pu ˈ}\textstyleChCharisSILBlueBold{ʝ}\textstyleChCharisSIL{ɛ̞.kɛ̞t[2FA?]}] & \textitbf{sa pu jeket} & ‘my jacket’\\
&  & 1\textsc{sg} \textsc{poss} jacket & \\
& [\textstyleChCharisSIL{… ˈi.tu, ˈ}\textstyleChCharisSILBlueBold{j}\textstyleChCharisSIL{a.ŋɐŋ}] & \textitbf{… itu, jangang} & ‘those (big ones), don’t’\\
&  & \textsc{d.dist} \textsc{neg.imp} & \\
\lspbottomrule
\end{tabular}

Fortition occurs very rarely and is attested only for the voiceless affricate and the palatal approximant in word-initial position. In the mentioned, more thoroughly transcribed 150-minute extract of the corpus, fortition of /\textstyleChCharisSIL{t}\textstyleChCharisSIL{ʃ}/ is attested once and strengthening of /\textstyleChCharisSIL{j}/ twice, as shown in Table  ‎2 .16.


\begin{stylecaption}
\label{bkm:Ref337898659}Table ‎2.\stepcounter{Table}{\theTable}:  Fortition of the voiceless affricate and the voiced palatal approximant
\end{stylecaption}

\tablehead{
 Phoneme & Item & Orthogr. & \arraybslash Gloss\\
}
\begin{tabular}{llll}
\lsptoprule
/\textstyleChCharisSIL{tʃ}/ & [\textstyleChCharisSIL{ˈdɛ̞p[2FA?] ˈ}\textstyleChCharisSILBlueBold{t}\textstyleChCharisSIL{u.tʃu}] & \textitbf{de pu cucu} & ‘his grandchild’\\
&  & 3\textsc{sg} \textsc{poss} grandchild & \\
/\textstyleChCharisSIL{j}/ & [\textstyleChCharisSIL{ˈej ˈ}\textstyleChCharisSILBlueBold{dʒ}\textstyleChCharisSIL{ɐŋ bɛ.ˌsɐr{\Tilde}bɛ.ˈsɐr}] & \textitbf{ey yang besar{\Tilde}besar} & ‘hey those big (ones)’\\
&  & hey \textsc{rel} \textsc{rdp}{\Tilde}be.big & \\
& [\textstyleChCharisSIL{ˈ}\textstyleChCharisSILBlueBold{ʝ}\textstyleChCharisSIL{a}] & \textitbf{yo} & ‘yes’\footnotemark{}\\
&  & yes & \\
\lspbottomrule
\end{tabular}
\footnotetext{\\
\\
\\
\\
Affirmative \textitbf{yo} ‘yes’ is frequently realized as \textitbf{ya} (see §5.4.3).\\
\\
\\
\\
\\
\\
\\
\\
\\
\\
\\
}
\paragraph[Elision]{Elision}
\label{bkm:Ref338065061}\label{bkm:Ref337904185}
Elision of a word-final segment is attested for the voiceless stops, the alveolar fricative, the velar nasal, and both liquids, as shown in Table  ‎2 .17. Concerning the voiceless stops, elision applies most frequently to /\textstyleChCharisSIL{k}/. Elision of /\textstyleChCharisSIL{t}/ occurs less frequently and is unattested for /\textstyleChCharisSIL{p}/. Word-final /\textstyleChCharisSIL{s}/ is much less prone to elision than word-final stops, with the corpus containing only two lexical items with deleted /\textstyleChCharisSIL{s}/. When the word-final velar nasal is omitted, it is always realized as nasalization on the preceding vowel.\footnote{\\
\\
\\
\\
More in-depth acoustic phonetic analysis is needed to determine whether the nasalized vowels remain centralized. Since these vowels occur in open syllables they are represented as their non-centralized allophones (for more details see §2.2.3) pending further results.\\
\\
\\
\\
\\
\\
\\
\\
\\
\\
\\
} Elision of the liquids occurs only very rarely. The exception is \textitbf{ambil} ‘fetch’. Of its 221 tokens, 49 tokens are realized without word-final /\textstyleChCharisSIL{l}/: [\textstyleChCharisSIL{ˈɐm.bi}] (48 tokens) and [\textstyleChCharisSIL{ˈɐm.be.a}] (1 token).


\begin{stylecaption}
\label{bkm:Ref330230236}Table ‎2.\stepcounter{Table}{\theTable}:  Elision of the voiceless stops, the alveolar fricative, the velar nasal, and the liquids in word-final position
\end{stylecaption}

\tablehead{
 Phoneme & Item & Orthogr. & \arraybslash Gloss\\
}
\begin{tabular}{llll}
\lsptoprule
/\textstyleChCharisSIL{t}/ & [\textstyleChCharisSIL{ˈsa.ki}] & \textitbf{sakit} & ‘be sick’\\
/\textstyleChCharisSIL{k}/ & [\textstyleChCharisSIL{ˈma.sa}] & \textitbf{masak} & ‘cook’\\
/\textstyleChCharisSIL{s}/ & [\textstyleChCharisSIL{ˈtru}] & \textitbf{trus} & ‘be continuous’\\
/\textstyleChCharisSIL{ŋ}/ & [\textstyleChCharisSIL{ˈɐn.dʒ\~{i}}] & \textitbf{anjing} & ‘dog’\\
/\textstyleChCharisSIL{r}/ & [\textstyleChCharisSIL{ˈla.pa}] & \textitbf{lapar} & ‘be hungry’\\
/\textstyleChCharisSIL{l}/ & [\textstyleChCharisSIL{ˈɐm.bi} / \textstyleChCharisSIL{ˈɐm.be.a}] & \textitbf{ambil} & ‘fetch’\\
\lspbottomrule
\end{tabular}
\paragraph[Devoicing]{Devoicing}
\label{bkm:Ref338065064}
Devoicing applies only to the rhotic trill as word-final coda. In this position, it is most commonly realized as [\textstyleChCharisSIL{r}]. Before a pause or in utterance-final position, however, the trill can also be devoiced to [\textstyleChCharisSIL{\textsuperscript{r̥}}], as illustrated in (0).


\begin{tabular}{lllllllllll}
\lsptoprule
\label{bkm:Ref337903009}
\gll {skarang} {dong} {kasi} {dia} {senter} {kasi} {senter} {dong} {kasi} {..}\\ %
& \bluebold{skaɾɐn} & dɔ̞ŋ & kasi & dɪa & \bluebold{sɛ̞ntɛ̞r̥}, & kasi & \bluebold{sɛ̞ntɛ̞r} & dɔ̞ŋ & kasi & \\
& now & \textsc{3pl} & give & \textsc{3sg} & flashlight & give & flashlight & \textsc{3pl} & give & \\
\lspbottomrule
\end{tabular}
\ea
\glt
‘\bluebold{now} they give him a \bluebold{flashlight}, (having) given (him) a \bluebold{flashlight} they give (him) …’ \textstyleExampleSource{[081108-003-JR.0002]}
\end{styleFreeTranslEngxvpt}

\paragraph[Palatalization]{Palatalization}
\label{bkm:Ref338493249}
Palatalization of /\textstyleChCharisSIL{s}/ is rare. It occurs only in lexical roots with a /\textstyleChCharisSIL{si.}V/ sequence, if this root has three or more syllables and if the syllable containing /s/ is unstressed. The palatalization of /\textstyleChCharisSIL{s}/ co-occurs with the elision of the close front vowel /\textstyleChCharisSIL{i}/, which reduces the number of syllables by one, as illustrated in Table  ‎2 .18. Hence, /\textstyleChCharisSIL{si.}V/ is realized as [\textstyleChCharisSIL{sʲ}V]. Attested is one polysyllabic lexical root with a /\textstyleChCharisSIL{si.}V/ sequence, the high frequency item \textitbf{siapa} ‘who’. In lexical roots with a /\textstyleChCharisSIL{si.}V/ sequence in which the syllable containing /s/ is stressed, palatalization of the fricative is unattested. Attested are the three lexical roots listed in Table  ‎2 .18, all of which are disyllabic: \textitbf{sial} ‘be unfortunate’, \textitbf{siang} ‘midday’, and \textitbf{siap} ‘be ready’.
\end{styleBodyafterxivpt}


This lack of assimilation in stressed syllables does, however, also apply to lexical items with more than two syllables, as evidenced by three polysyllabic loan words, presented in §2.5.2.3. The occurrence of /\textstyleChCharisSIL{s}/ in a /\textstyleChCharisSIL{si.}V/ sequence together with the stress pattern of the respective lexical item does not, however, condition the palatalization of the fricative. This is evidenced by the fact that \textitbf{siapa} ‘who’ is realized quite commonly without palatalization: [\textstyleChCharisSIL{ˈsa.pa}].



The frequency counts in Table  ‎2 .18 are based on the broad transcription of the entire 16-hour corpus (16-\textsc{h-c}) and the more thoroughly transcribed 150-minute extract (150-\textsc{m-c}) of the corpus.\footnote{\\
\\
\\
\\
The broad transcription of the 16-hour corpus makes no distinction between the unpalatalized and the palatalized realizations of \textitbf{siapa} ‘who’, [\textstyleChCharisSILviiivpt{si.ˈa.pa}] and [\textstyleChCharisSILviiivpt{ˈsʲa.pa}], respectively. Hence, a more thorough transcription of all 196 /\textstyleChCharisSILviiivpt{siapa}/ tokens is required to establish whether speakers sometimes realize the interrogative as the trisyllabic item [\textstyleChCharisSILviiivpt{si.ˈa.pa}] or whether they always palatalize the fricative and thereby realize the item as disyllabic [\textstyleChCharisSILviiivpt{ˈsʲa.pa}]. In the more thoroughly transcribed 150-minute extract of the corpus the trisyllabic \textitbf{siapa} [\textstyleChCharisSILviiivpt{si.ˈa.pa}] ‘who’ is unattested.\\
\\
\\
\\
\\
\\
\\
\\
\\
\\
\\
}


\begin{stylecaption}
\label{bkm:Ref338496606}Table ‎2.\stepcounter{Table}{\theTable}:  Palatalization of the alveolar fricative in loan words
\end{stylecaption}

\tablehead{
 Stress & Orthogr. & Gloss & Realization & Freq. & \arraybslash Freq.\\
&  &  &  & 16-\textsc{h-c} & \arraybslash 150-\textsc{m-c}\\
}
\begin{tabular}{llllll}
\lsptoprule
/\textstyleChCharisSIL{si}/ unstressed & \textitbf{siapa} & ‘who’ & [\textstyleChCharisSILBlueBold{si.ˈa}\textstyleChCharisSIL{.pa}] & \raggedleft 196 & \raggedleft\arraybslash {}-{}-{}-\\
&  &  & [\textstyleChCharisSIL{ˈ}\textstyleChCharisSILBlueBold{sʲa}\textstyleChCharisSIL{.pa}] & \raggedleft {}-{}-{}- & \raggedleft\arraybslash 40\\
&  &  & [\textstyleChCharisSIL{ˈ}\textstyleChCharisSILBlueBold{sa}\textstyleChCharisSIL{.pa}] & \raggedleft 115 & \raggedleft\arraybslash 10\\
/\textstyleChCharisSIL{ˈsi}/ stressed & \textitbf{sial} & ‘be unfortunate’ & [\textstyleChCharisSIL{ˈ}\textstyleChCharisSILBlueBold{si.ɐ}\textstyleChCharisSIL{l}] & \raggedleft 1 & \raggedleft\arraybslash 1\\
& \textitbf{siang} & ‘midday’ & [\textstyleChCharisSIL{ˈ}\textstyleChCharisSILBlueBold{si.ɐ}\textstyleChCharisSIL{ŋ}] & \raggedleft 55 & \raggedleft\arraybslash 6\\
& \textitbf{siap} & ‘be ready’ & [\textstyleChCharisSIL{ˈ}\textstyleChCharisSILBlueBold{si.ɐ}\textstyleChCharisSIL{p̚}] & \raggedleft 54 & \raggedleft\arraybslash 2\\
\lspbottomrule
\end{tabular}
\subsection{Phonetic processes for vowels}
\paragraph[Centralization and lowering]{Centralization and lowering}
\label{bkm:Ref338065066}
In addition to the regular decentralization of the vowels in closed syllables, the data indicates two environments where centralization of vowels occurs on an irregular basis in open syllables: (1) under the influence of central vowel /\textstyleChCharisSIL{a}/, and (2) under the influence of the corresponding centralized allophone occurring in closed syllables. In addition, the close vowels are very commonly lowered in fast speech.
\end{styleBodyafterxivpt}


In open syllables, the close and open-mid vowels are frequently centralized under the influence of the central vowel /\textstyleChCharisSIL{a}/, similar to the process of centralization in closed syllables (§2.2.3) In unstressed open syllables, open-mid /\textstyleChCharisSIL{ɛ}/ can alternatively be centralized to [\textstyleChCharisSIL{ə}] rather than to [\textstyleChCharisSIL{ɛ̞}].


\begin{styleCaptionxivptSpace}
Table ‎2.\stepcounter{Table}{\theTable}:  Vowel centralization under the influence of central vowel /\textstyleChCharisSIL{a}/
\end{styleCaptionxivptSpace}

\tablehead{
 Phoneme & Realization & Item & Orthogr. & \arraybslash Gloss\\
}
\begin{tabular}{lllll}
\lsptoprule
/\textstyleChCharisSIL{i}/ & [\textstyleChCharisSIL{ɪ}] & [\textstyleChCharisSIL{ˈd}\textstyleChCharisSILBlueBold{ɪ}\textstyleChCharisSIL{.a}] & \textitbf{dia} & ‘\textsc{3sg}’\\
&  & [\textstyleChCharisSIL{ˈh}\textstyleChCharisSILBlueBold{ɪ}\textstyleChCharisSIL{.lɐŋ}] & \textitbf{hilang} & ‘be lost’\\
/\textstyleChCharisSIL{u}/ & [\textstyleChCharisSIL{ʊ}] & [\textstyleChCharisSIL{ˈl}\textstyleChCharisSILBlueBold{ʊ}\textstyleChCharisSIL{.ɐs}] & \textitbf{luas} & ‘be vast’\\
&  & [\textstyleChCharisSIL{ˈb}\textstyleChCharisSILBlueBold{ʊ}\textstyleChCharisSIL{.kɐŋ}] & \textitbf{bukang} & ‘\textsc{neg}’\\
/\textstyleChCharisSIL{ɛ}/ & [\textstyleChCharisSIL{ɛ̞}] & [\textstyleChCharisSIL{ˈb}\textstyleChCharisSILBlueBold{ɛ̞}\textstyleChCharisSIL{.ɾa}] & \textitbf{bera} & ‘defecate’\\
&  & [\textstyleChCharisSIL{ˈh}\textstyleChCharisSILBlueBold{ɛ̞}\textstyleChCharisSIL{.la}] & \textitbf{hela} & ‘haul’\\
& [\textstyleChCharisSIL{ə}] & [\textstyleChCharisSIL{b}\textstyleChCharisSILBlueBold{ə}\textstyleChCharisSIL{.ˈkɐs}] & \textitbf{bekas} & ‘trace’\\
&  & [\textstyleChCharisSIL{l}\textstyleChCharisSILBlueBold{ə}\textstyleChCharisSIL{.ˈpɐs}] & \textitbf{lepas} & ‘free’\\
/\textstyleChCharisSIL{ɔ}/ & [\textstyleChCharisSIL{ɔ̞}] & [\textstyleChCharisSIL{ˈh}\textstyleChCharisSILBlueBold{ɔ̞}\textstyleChCharisSIL{.sa}] & \textitbf{hosa} & ‘pant’\\
&  & [\textstyleChCharisSIL{ˈk}\textstyleChCharisSILBlueBold{ɔ̞}\textstyleChCharisSIL{.lɐm}] & \textitbf{kolam} & ‘big hole’\\
\lspbottomrule
\end{tabular}

In open syllables, the close and open-mid vowels can also be centralized under the influence of the corresponding centralized allophone occurring in a closed syllable, as illustrated in Table  ‎2 .20 (see also §2.2.3).


\begin{stylecaption}
\label{bkm:Ref332647712}Table ‎2.\stepcounter{Table}{\theTable}:  Vowel centralization harmony\footnote{\\
\\
\\
\\
The following lexemes are loan words: \textitbf{propinsi} ‘province’ and \textitbf{skripsi} ‘minithesis’. The following lexemes are historically derived by (unproductive) affixation: \textitbf{bertemu} ‘be friends’, \textitbf{berkebung} ‘farm’, and \textitbf{meleset} ‘miss a target’.\\
\\
\\
\\
\\
\\
\\
\\
\\
\\
\\
}
\end{stylecaption}

\tablehead{
 Phoneme & Environment & Item & Orthogr. & \arraybslash Gloss\\
}
\begin{tabular}{lllll}
\lsptoprule
/\textstyleChCharisSIL{i}/ & [\textstyleChCharisSIL{ɪ}] in open \textsc{sylb} preceded by [\textstyleChCharisSIL{ɪ}C] & [\textstyleChCharisSIL{prɔ.ˈpɪn.s}\textstyleChCharisSILBlueBold{ɪ}] & \textitbf{propinsi} & ‘province’\\
&  & [\textstyleChCharisSIL{ˈskrɪp̚.s}\textstyleChCharisSILBlueBold{ɪ}] & \textitbf{skripsi} & ‘minithesis’\\
& [\textstyleChCharisSIL{ɪ}] in open \textsc{sylb} followed by [\textstyleChCharisSIL{ɪ}C] & [\textstyleChCharisSIL{ˈm}\textstyleChCharisSILBlueBold{ɪ}\textstyleChCharisSIL{.ɾɪŋ}] & \textitbf{miring} & ‘be sideways’\\
&  & [\textstyleChCharisSIL{ˈg}\textstyleChCharisSILBlueBold{ɪ}\textstyleChCharisSIL{.lɪŋ}] & \textitbf{giling} & ‘grind’\\
/\textstyleChCharisSIL{u}/ & [\textstyleChCharisSIL{ʊ}] in open \textsc{sylb} preceded by [\textstyleChCharisSIL{ʊ}C] & [\textstyleChCharisSIL{ˈbʊm.b}\textstyleChCharisSILBlueBold{ʊ}] & \textitbf{bumbu} & ‘bamboo’\\
&  & [\textstyleChCharisSIL{ˈbʊn.t}\textstyleChCharisSILBlueBold{ʊ}] & \textitbf{buntu} & ‘be blocked’\\
& [\textstyleChCharisSIL{ʊ}] in open \textsc{sylb} followed by [\textstyleChCharisSIL{ʊ}C] & [\textstyleChCharisSIL{ˈl}\textstyleChCharisSILBlueBold{ʊ}\textstyleChCharisSIL{.ɾʊs}] & \textitbf{lurus} & ‘be straight’\\
&  & [\textstyleChCharisSIL{ˈt}\textstyleChCharisSILBlueBold{ʊ}\textstyleChCharisSIL{.ɾʊŋ}] & \textitbf{turung} & ‘descend’\\
/\textstyleChCharisSIL{ɛ}/ & [\textstyleChCharisSIL{ɛ̞}] in open \textsc{sylb} preceded by [\textstyleChCharisSIL{ɛ̞}C] & [\textstyleChCharisSIL{bɛ̞r.ˈt}\textstyleChCharisSILBlueBold{ɛ̞}\textstyleChCharisSIL{.mu}] & \textitbf{bertemu} & ‘be friends’\\
&  & [\textstyleChCharisSIL{ˌbɛ̞r.k}\textstyleChCharisSILBlueBold{ɛ̞}\textstyleChCharisSIL{.ˈbʊŋ}] & \textitbf{berkebung} & ‘do farming’\\
& [\textstyleChCharisSIL{ɛ̞}] in open \textsc{sylb} followed by [\textstyleChCharisSIL{ɛ̞}C] & [\textstyleChCharisSIL{ˈ}\textstyleChCharisSILBlueBold{ɛ̞}\textstyleChCharisSIL{.pɛ̞ŋ}] & \textitbf{epeng} & ‘important’\\
&  & [\textstyleChCharisSIL{mɛ.ˈl}\textstyleChCharisSILBlueBold{ɛ̞}\textstyleChCharisSIL{.sɛ̞t̚}] & \textitbf{meleset} & ‘miss a target’\\
/\textstyleChCharisSIL{ɔ}/ & [\textstyleChCharisSIL{ɔ̞}] in open \textsc{sylb} preceded by [\textstyleChCharisSIL{ɔ̞}C] & [\textstyleChCharisSIL{ˈbɔ̞ŋ.s}\textstyleChCharisSILBlueBold{ɔ̞}] & \textitbf{bongso} & ‘youngest child’\\
&  & [\textstyleChCharisSIL{ˈtʃɔ̞n.t}\textstyleChCharisSILBlueBold{ɔ̞}] & \textitbf{conto} & ‘example’\\
& [\textstyleChCharisSIL{ɔ̞}] in open \textsc{sylb} followed by [\textstyleChCharisSIL{ɔ̞}C] & [\textstyleChCharisSIL{ˈr}\textstyleChCharisSILBlueBold{ɔ̞}\textstyleChCharisSIL{.kɔ̞k̚}] & \textitbf{rokok} & ‘cigarette’\\
&  & [\textstyleChCharisSIL{ˈk}\textstyleChCharisSILBlueBold{ɔ̞}\textstyleChCharisSIL{.dɔ̞k̚}] & \textitbf{kodok} & ‘frog’\\
\hhline{-~---}
\lspbottomrule
\end{tabular}

In fast speech, the close vowels /\textstyleChCharisSIL{i}/ and /\textstyleChCharisSIL{u}/ are very commonly lowered and realized as the close-mid vowels [\textstyleChCharisSIL{e}] and [\textstyleChCharisSIL{o}] respectively, as demonstrated in (0) to (0). In (0) the verb \textitbf{kasi} ‘give’ is realized as [\textstyleChCharisSIL{ˈka.se}], and in (0) the verb \textitbf{balik} ‘turn around’ is realized as [\textstyleChCharisSIL{ˈba.le}].\footnote{\\
\\
\\
\\
Concerning the elision of the word-final stop see §2.3.1.2.\\
\\
\\
\\
\\
\\
\\
\\
\\
\\
\\
} In (0) the numeral \textitbf{dua} ‘two’ is realized as [\textstyleChCharisSIL{ˈdo.a}] and in (0) the common noun \textitbf{lubang} ‘hole’ is realized as [\textstyleChCharisSIL{ˈlo.bɐŋ}]


\begin{tabular}{lllllllll}
\lsptoprule
\label{bkm:Ref332636414}
\gll {…} {mo} {biking} {papeda} {mo} {kasi} {ana{\Tilde}ana} {makang}\\ %
& … & mɔ & bikɪn & papɛda & mɔ & \bluebold{kase} & anana & makɐn\\
&  & want & \textsc{3pl} & sagu.porridge & want & give & \textsc{rdp}{\Tilde}child & eat\\
\lspbottomrule
\end{tabular}
\ea
\glt 
‘[they said (they) wanted to catch chickens and then] (they) wanted to make sagu porridge to \bluebold{give} the children to eat’ \textstyleExampleSource{[081010-001-Cv.0191]}
\z

\begin{tabular}{llllllll}
\lsptoprule
\label{bkm:Ref332636418}
\gll {itu} {Bop} {Bop} {itu,} {de} {biasa} {balik}\\ %
& itu & Bɔp & Bɔp & itu, & de & biasa & \bluebold{bale}\\
& \textsc{d.dist} & Bop & Bop & \textsc{d.dist} & \textsc{3sg} & be.usual & turn.around\\
\lspbottomrule
\end{tabular}
\ea
\glt 
‘that was Bob, that Bob, he usually \bluebold{(flies) a circle}’ (Lit. ‘turns around’) \textstyleExampleSource{[081011-010-Cv.0019]}
\z

\begin{tabular}{lllll}
\lsptoprule
\label{bkm:Ref332636419}
\gll {skarang} {dong} {dua} {mancing}\\ %
& skaɾɐn & dɔ̞ŋ & \bluebold{doa} & mɐn.tʃɪŋ\\
& now & \textsc{3pl} & two & fish\\
\lspbottomrule
\end{tabular}
\ea
\glt 
‘now the \bluebold{two} of them are fishing’ \textstyleExampleSource{[081109-010-JR.0002]}
\z

\begin{tabular}{lllll}
\lsptoprule
\label{bkm:Ref332636420}
\gll {de} {masuk} {lubang} {tu}\\ %
& dɛ & masʊk̚ & \bluebold{lobɐŋ} & tu\\
& \textsc{3sg} & enter & hole & \textsc{d.dist}\\
\lspbottomrule
\end{tabular}
\ea
\glt
‘it (the chicken) went into that \bluebold{hole} (in the floor)’ \textstyleExampleSource{[080921-004a-CvNP.0096]}
\end{styleFreeTranslEngxvpt}

\paragraph[Nasalization]{Nasalization}
\label{bkm:Ref338065068}
The five vowels /\textstyleChCharisSIL{i}, \textstyleChCharisSIL{u}, \textstyleChCharisSIL{ɛ}, \textstyleChCharisSIL{ɔ}, \textstyleChCharisSIL{a}/ can be nasalized and realized as [\textstyleChCharisSIL{\~\i}, \textstyleChCharisSIL{\~{u}}, \textstyleChCharisSIL{ɛ}, \textstyleChCharisSIL{ɔ}, \textstyleChCharisSIL{\~{a}}] as a result of the elision of the word-final velar nasal /\textstyleChCharisSIL{ŋ}/, discussed in §2.3.1.2.


\begin{stylecaption}
Table ‎2.\stepcounter{Table}{\theTable}:  Nasalization of the vowels
\end{stylecaption}

\tablehead{
 Phoneme & Item & Orthogr. & \arraybslash Gloss\\
}
\begin{tabular}{llll}
\lsptoprule
/\textstyleChCharisSIL{i}/ & [\textstyleChCharisSIL{ˈɐn.dʒ\~{i}}] & \textitbf{anjing} & ‘dog’\\
/\textstyleChCharisSIL{u}/ & [\textstyleChCharisSIL{ˈlaŋ.s\~{u}}] & \textitbf{langsung} & ‘immediately’\\
/\textstyleChCharisSIL{ɛ}/ & [\textstyleChCharisSIL{ˈd\~{e}}] & \textitbf{dengang} & ‘with’\footnotemark{}\\
/\textstyleChCharisSIL{ɔ}/ & [\textstyleChCharisSIL{ˈdɔ}] & \textitbf{dong} & ‘\textsc{3pl}’\\
/\textstyleChCharisSIL{a}/ & [\textstyleChCharisSIL{ˈbil\~{a}}] & \textitbf{bilang} & ‘say’\\
\lspbottomrule
\end{tabular}
\footnotetext{\\
\\
\\
\\
Comitative \textitbf{dengang} ‘with’ is frequently realized as mono-syllabic \textitbf{deng} ‘with’ (see §14.2.1.1)\\
\\
\\
\\
\\
\\
\\
\\
\\
\\
\\
}
\paragraph[Lengthening]{Lengthening}
\label{bkm:Ref338065070}\label{bkm:Ref334163730}
Vowel length is not phonemic in Papuan Malay. Very commonly, however, vowel lengthening occurs as a manifestation of emphasis, as in (0) and (0). In (0) the speaker relates how, after a long journey, they finally got to their destination \textitbf{sampeee di pohong} ‘all the way up to the tree’. In (0), an irritated mother explains to her son for the nth time that their date of departure has \textitbf{beluuum} ‘not yet’ come.


\begin{tabular}{lllllll}
\lsptoprule
\label{bkm:Ref332809170}
\gll {kitong} {dua} {turung} {sampeee} {di} {pohong}\\ %
& kitɔ̞ŋ & dʊa & tʊɾʊn & \bluebold{sɐmpɛːː} & di & pɔhɔ̞n\\
& \textsc{1pl} & two & descend & reach & at & tree\\
\lspbottomrule
\end{tabular}
\ea
\glt 
‘we two came down \textsc{all} \textsc{the} \textsc{way} to the tree’ \textstyleExampleSource{[080917-008-NP.0024]}
\z

\begin{tabular}{llllll}
\lsptoprule
\label{bkm:Ref332809171}
\gll {itu} {bluuum} {tong} {blum} {jalang}\\ %
& itu & \bluebold{bɛlʊːːm}, & tɔ̞ŋ & blʊm & dʒalɐn\\
& \textsc{d.dist} & not.yet & \textsc{1pl} & not.yet & walk\\
\lspbottomrule
\end{tabular}
\ea
\glt
‘that’s \textsc{not} \textsc{yet}, we’re not going yet’ \textstyleExampleSource{[080921-001-CvNP.0007]}
\end{styleFreeTranslEngxvpt}

\subsection{Alternative realizations of the VC sequences /aj/ and /aw/}
\label{bkm:Ref357180858}
The VC sequences /aj/ and /aw/ have alternative realizations on an irregular basis. They tend to be centralized to [\textstyleChCharisSIL{ɛ̞j}] and [\textstyleChCharisSIL{ɔ̞w}], respectively, as shown in Table  ‎2 .22, or they can be reduced to the open-mid vowels [\textstyleChCharisSIL{ɛ}] and [\textstyleChCharisSIL{ɔ}], respectively, as illustrated in Table  ‎2 .23 and Table  ‎2 .24.
\end{styleBodyafterxivpt}


When /\textstyleChCharisSIL{aj}/ and /\textstyleChCharisSIL{aw}/ occur in disyllabic roots, they tend to be centralized to [\textstyleChCharisSIL{ɛ̞j}] and [\textstyleChCharisSIL{ɔ̞w}], respectively, in the following environments (see Table  ‎2 .22). The VC sequence /\textstyleChCharisSIL{aj}/ is centralized to [\textstyleChCharisSIL{ɛ̞j}] when following a liquid, as in \textitbf{serey} [\textstyleChCharisSIL{sɛ.ˈrɛ̞j}] ‘lemongrass’ or \textitbf{laley} [\textstyleChCharisSIL{ˈla.lɛ̞j}] ‘be careless’.\footnote{\\
\\
\\
\\
All ten participants in a rapid unrepresentative orthography test, by contrast, realized \textitbf{laley} ‘be careless’ as [\textstyleChCharisSILviiivpt{ˈla.lɐj}] and not as [\textstyleChCharisSILviiivpt{ˈla.lɛ̞j}]. At this point in the research on Papuan Malay the reasons for the realization as [\textstyleChCharisSILviiivpt{ˈla.lɐj}] remain uncertain, however.\\
\\
\\
\\
\\
\\
\\
\\
\\
\\
\\
} With other onset consonants /\textstyleChCharisSIL{aj}/ remains unaffected. As for the centralization of /aw/ to [\textstyleChCharisSIL{ɔ̞w}], the data is less clear. Attested are only three lexical items: /aw/ is centralized to [\textstyleChCharisSIL{ɔ̞w}] following the lateral /\textstyleChCharisSIL{l}/ in \textitbf{pulow} [\textstyleChCharisSIL{ˈpu.lɔ̞w}] ‘island’, the affricate /\textstyleChCharisSIL{dʒ}/ in \textitbf{hijow} [\textstyleChCharisSIL{ˈhi.dʒɔ̞w}] ‘green’, and the fricative /\textstyleChCharisSIL{s}/ in \textitbf{pisow} [\textstyleChCharisSIL{ˈpi.sɔ̞w}] ‘knife’. With other onset consonants /\textstyleChCharisSIL{aw}/ is not centralized. More data is needed to explore whether centralization in these contexts is indeed unpredictable or whether it constitutes a predictable phonological process.\footnote{\\
\\
\\
\\
The corpus includes only eight lexical roots containing /\textstyleChCharisSILviiivpt{aj}/ and ten roots with /\textstyleChCharisSILviiivpt{aw}/.\\
\\
\\
\\
\\
\\
\\
\\
\\
\\
\\
}


\begin{styleCaptionxivptSpace}
\label{bkm:Ref332643601}Table ‎2.\stepcounter{Table}{\theTable}:  Realization of /\textstyleChCharisSIL{aj}/ as [\textstyleChCharisSIL{ɛ̞j}] and of /\textstyleChCharisSIL{aw}/ as [\textstyleChCharisSIL{ɔ̞w}]
\end{styleCaptionxivptSpace}

\tablehead{
 Phoneme & Realization & \multicolumn{2}{l}{ Item} & \arraybslash Gloss\\
}
\begin{tabular}{lllll}
\lsptoprule
/\textstyleChCharisSIL{aj}/ & [\textstyleChCharisSIL{ɐj}] vs. [\textstyleChCharisSIL{ɛ̞j}] & [\textstyleChCharisSIL{tʃɛ.ˈɾ}\textstyleChCharisSILBlueBold{ɛ̞j}] & \textitbf{cerey} & ‘divorce’\\
&  & [\textstyleChCharisSIL{ˈla.l}\textstyleChCharisSILBlueBold{ɛ̞j}] & \textitbf{laley} & ‘be careless’\\
&  & [\textstyleChCharisSIL{sɛ.ˈɾ}\textstyleChCharisSILBlueBold{ɛ̞j}] & \textitbf{serey} & ‘lemongrass’\\
&  & [\textstyleChCharisSIL{ˈda.m}\textstyleChCharisSILBlueBold{ɐj}] & \textitbf{damay} & ‘peace’\\
&  & [\textstyleChCharisSIL{ˈtu.p}\textstyleChCharisSILBlueBold{ɐj}] & \textitbf{tupay} & ‘squirrel’\\
/\textstyleChCharisSIL{aw}/ & [\textstyleChCharisSIL{ɐw}] vs. [\textstyleChCharisSIL{ɔ̞w}] & [\textstyleChCharisSIL{ˈhi.dʒ}\textstyleChCharisSILBlueBold{ɔ̞w}] & \textitbf{hijow} & ‘green’\\
&  & [\textstyleChCharisSIL{ˈpi.s}\textstyleChCharisSILBlueBold{ɔ̞w}] & \textitbf{pisow} & ‘knife’\\
&  & [\textstyleChCharisSIL{ˈpu.l}\textstyleChCharisSILBlueBold{ɔ̞w}] & \textitbf{pulow} & ‘island’\\
&  & [\textstyleChCharisSIL{ˈhi.r}\textstyleChCharisSILBlueBold{ɐw}] & \textitbf{hiraw} & ‘heed’\\
&  & [\textstyleChCharisSIL{ˈki.tʃ}\textstyleChCharisSILBlueBold{ɐw}] & \textitbf{kicaw} & ‘be naughty’\\
\lspbottomrule
\end{tabular}

When /aj/ and /aw/ occur in unstressed CVC syllables of non-monosyllabic roots, they tend to be reduced to open-mid vowels under the influence of the central vowel /\textstyleChCharisSIL{a}/; that is, /aj/ is realized as front /\textstyleChCharisSIL{ɛ}/, and /aw/ as back /\textstyleChCharisSIL{ɔ}/.
\end{styleBodyaftervbeforexivpt}


The tendency to realize /aj/ as [\textstyleChCharisSIL{ɛ}] applies especially to unstressed CVC syllables with an onset stop, as shown in Table  ‎2 .23. In this environment, the realization of /aj/ as [\textstyleChCharisSIL{ɐ}j] occurs much less often or not at all. Examples are \textitbf{cape} ‘be tired’ or \textitbf{pake} ‘use’. The VC sequence typically remains unaffected in the following environments: in unstressed CVC syllables with an initial consonant other than a stop, as in \textitbf{damay} ‘peace’, when preceded by a syllable containing a vowel other than central /a/, as in \textitbf{sungay} ‘river’, or in stressed syllables as in \textitbf{selesay} ‘finish’.


\begin{styleCaptionxivptSpace}
\label{bkm:Ref324929953}Table ‎2.\stepcounter{Table}{\theTable}:  Realization of /\textstyleChCharisSIL{aj}/ as [\textstyleChCharisSIL{ɐj}] or [\textstyleChCharisSIL{ɛ}]
\end{styleCaptionxivptSpace}

\tablehead{
\multicolumn{2}{l}{ [\textstyleChCharisSIL{ɐj}]} & \multicolumn{2}{l}{ [\textstyleChCharisSIL{ɛ}]} & Orthogr. & \arraybslash Gloss\\
 Item & Freq. & Item & Freq. &  & \\
}
\begin{tabular}{llllll}
\lsptoprule
[\textstyleChCharisSIL{ˈtʃa.p}\textstyleChCharisSILBlueBold{ɐj}] & \raggedleft 1 & [\textstyleChCharisSIL{ˈtʃa.p}\textstyleChCharisSILBlueBold{ɛ}] & \raggedleft 23 & \textitbf{cape} & ‘be tired’\\
 {}-{}-{}-\footnotemark{} & \raggedleft {}-{}-{}- & [\textstyleChCharisSIL{ˈpa.k}\textstyleChCharisSILBlueBold{ɛ}] & \raggedleft 213 & \textitbf{pake} & ‘use’\\
[\textstyleChCharisSIL{ˈsɐn.t}\textstyleChCharisSILBlueBold{ɐj}] & \raggedleft 1 & [\textstyleChCharisSIL{ˈsɐn.t}\textstyleChCharisSILBlueBold{ɛ}] & \raggedleft 7 & \textitbf{sante} & ‘relax’\\
[\textstyleChCharisSIL{ˈda.m}\textstyleChCharisSILBlueBold{ɐj}] & \raggedleft 9 & {}-{}-{}- & \raggedleft {}-{}-{}- & \textitbf{damay} & ‘peace’\\
[\textstyleChCharisSIL{pɛ.ˈga.w}\textstyleChCharisSILBlueBold{ɐj}] & \raggedleft 110 & [\textstyleChCharisSIL{pɛ.ˈga.w}\textstyleChCharisSILBlueBold{ɛ}] & \raggedleft 3 & \textitbf{pegaway} & ‘employee’\\
[\textstyleChCharisSIL{ˌsɛ.lɛ.ˈs}\textstyleChCharisSILBlueBold{ɐj}] & \raggedleft 154 & {}-{}-{}- & \raggedleft {}-{}-{}- & \textitbf{selesay} & ‘finish’\\
[\textstyleChCharisSIL{ˈsu.ŋ}\textstyleChCharisSILBlueBold{ɐj}] & \raggedleft 6 & {}-{}-{}- & \raggedleft {}-{}-{}- & \textitbf{sungay} & ‘river’\\
[\textstyleChCharisSIL{ˈtu.p}\textstyleChCharisSILBlueBold{ɐj}] & \raggedleft 1 & {}-{}-{}- & \raggedleft {}-{}-{}- & \textitbf{tupay} & ‘squirrel’\\
\lspbottomrule
\end{tabular}
\footnotetext{\\
\\
\\
\\
Standard Malay realizes this lexical item as {\textless}\textitbf{pakai}{\textgreater} ‘use, wear’ {\citep{Mintz2002}}.\\
\\
\\
\\
\\
\\
\\
\\
\\
\\
\\
}

The tendency to realize /\textstyleChCharisSIL{aw}/ as [\textstyleChCharisSIL{ɔ}] also applies to unstressed syllables with an onset consonant. This consonant, however, does not need to be a stop, as shown in Table  ‎2 .24. Examples are \textitbf{dano} ‘lake’ and \textitbf{kaco} ‘be confused’.\footnote{\\
\\
\\
\\
In addition, the corpus also contains three loan words in which /\textstyleChCharisSILviiivpt{aw}/ is realized as /\textstyleChCharisSILviiivpt{ɔ}/:\\
\ea%1
    \label{ex:1}
    \langinfo{lg}{fam}{src}\\
    \gll\\
	\\
    \ea
\glt
    \z

	 \textitbf{ato} ‘or’: /\textstyleChCharisSILviiivpt{ˈa.tɔ}/ (113 tokens) vs. /\textstyleChCharisSILviiivpt{ˈa.taw}/ (85 tokens)\\
\ea%2
    \label{ex:2}
    \langinfo{lg}{fam}{src}\\
    \gll\\
	\\
    \ea
\glt
    \z

	 \textitbf{kalo} ‘if’: /\textstyleChCharisSILviiivpt{ˈka.lɔ}/ (1,028 tokens) vs. /\textstyleChCharisSILviiivpt{ˈka.law}/ (230 tokens)\\
\ea%3
    \label{ex:3}
    \langinfo{lg}{fam}{src}\\
    \gll\\
	\\
    \ea
\glt
    \z

	 \textitbf{sodara} ‘sibling’: /\textstyleChCharisSILviiivpt{sɔ.ˈda.ra}/ (138 tokens) vs. /\textstyleChCharisSILviiivpt{saw.ˈda.ra}/ (14 tokens).\\
\\
\\
\\
\\
\\
\\
\\
} When preceded by a syllable containing a vowel other than central /a/, the VC sequence typically remains unaffected, and its realization as [\textstyleChCharisSIL{ɔ}] is rare. The corpus includes only one lexeme with an alternative [\textstyleChCharisSIL{ɔ}] realization, namely \textitbf{pulow} ‘island’.


\begin{styleCaptionxivptSpace}
\label{bkm:Ref324929954}Table ‎2.\stepcounter{Table}{\theTable}:  Realization of /\textstyleChCharisSIL{aw}/ as [\textstyleChCharisSIL{ɐw}] or [\textstyleChCharisSIL{ɔ}]
\end{styleCaptionxivptSpace}

\tablehead{
\multicolumn{2}{l}{ [\textstyleChCharisSIL{aw}]} & \multicolumn{2}{l}{ [\textstyleChCharisSIL{ɔ}]} & Orthogr. & \arraybslash Gloss\\
 Item & Freq. & Item & Freq. &  & \\
}
\begin{tabular}{llllll}
\lsptoprule
[\textstyleChCharisSIL{ˈda.n}\textstyleChCharisSILBlueBold{ɐw}] & \raggedleft 1 & [\textstyleChCharisSIL{ˈda.n}\textstyleChCharisSILBlueBold{ɔ}] & \raggedleft 3 & \textitbf{dano} & ‘lake’\\
[\textstyleChCharisSIL{ˈka.tʃ}\textstyleChCharisSILBlueBold{ɐw}] & \raggedleft 2 & [\textstyleChCharisSIL{ˈka.tʃ}\textstyleChCharisSILBlueBold{ɔ}] & \raggedleft 12 & \textitbf{kaco} & ‘be confused’\\
[\textstyleChCharisSIL{ˈhi.dʒ}\textstyleChCharisSILBlueBold{ɐw}] & \raggedleft 1 & {}-{}-{}- & \raggedleft {}-{}-{}- & \textitbf{hijow} & ‘be green’\\
[\textstyleChCharisSIL{ˈhi.r}\textstyleChCharisSILBlueBold{ɐw}] & \raggedleft 2 & {}-{}-{}- & \raggedleft {}-{}-{}- & \textitbf{hiraw} & ‘heed’\\
[\textstyleChCharisSIL{ˈki.tʃ}\textstyleChCharisSILBlueBold{ɐw}] & \raggedleft 1 & {}-{}-{}- & \raggedleft {}-{}-{}- & \textitbf{kicaw} & ‘be naughty’\\
[\textstyleChCharisSIL{ˈpu.l}\textstyleChCharisSILBlueBold{ɐw}] & \raggedleft 7 & [\textstyleChCharisSIL{ˈpu.l}\textstyleChCharisSILBlueBold{ɔ}] & \raggedleft 5 & \textitbf{pulow} & ‘island’\\
[\textstyleChCharisSIL{ˈpɪs.}\textstyleChCharisSILBlueBold{ɐw}] & \raggedleft 5 & {}-{}-{}- & \raggedleft {}-{}-{}- & \textitbf{pisow} & ‘knife’\\
\lspbottomrule
\end{tabular}

In monosyllabic words, /aj/ and /aw/ are never realized as /\textstyleChCharisSIL{ɛ}/ and /\textstyleChCharisSIL{ɔ}/, respectively. Examples are \textitbf{tay} /\textstyleChCharisSIL{taj}/ ‘excrement’ and \textitbf{taw} /\textstyleChCharisSIL{taw}/ ‘know’. There is one exception, though, monosyllabic \textitbf{mo} ‘want’. In the corpus this item is typically realized as /\textstyleChCharisSIL{mɔ}/ (750 tokens), rather than as /\textstyleChCharisSIL{maw}/ (212). In the historically affixed lexical items \textitbf{kemawang} ‘will’ and \textitbf{mawnya} ‘the wanting’, however, the root is realized as /\textstyleChCharisSIL{maw}/, as the syllable containing the root is stressed.


\section{Phonotactics}
\label{bkm:Ref324759941}
This section describes how in Papuan Malay segments combine to form syllables, how syllables combine into words, and what the stress patterns of these words are. The distribution and sequences of the consonant phonemes are presented in §2.4.1 and those of the vowel phonemes in section §2.4.2. The syllable structures are described in §2.4.3 and the stress patterns in §2.4.4.



For all of the identified segment sequences, as well as for most of the syllable types and stress patterns, the attested lexical items were investigated as to whether they are inherited Malay roots or loan words, by using the following sources: {\citet{Jones2007}} and {\citet{Tadmor2009a}}.\footnote{\\
\\
\\
\\
\\
\\
\\
Additional input was provided by {A. Clynes (p.c. 2012)}, {R. van den Berg (p.c. 2012)}, {C. Williams-van Klinken (p.c. 2012)}, {W. Mahdi (p.c. 2012)}, {R. Mills (p.c. 2012)}, {R.A. Blust (p.c. 2012)}, and {C.E. Grimes (p.c. 2012)}.\\
\\
\\
\\
\\
\\
\\
\\
} For high frequency syllable types and stress patterns, however, not all of the attested entries were checked. Hence, upon further investigation some of these lexical items may turn out to be loan words.
\end{styleBodyvxvafter}

\subsection{Consonant phoneme distribution and sequences}
\label{bkm:Ref324758216}
Table  ‎2 .25 provides an overview of the distribution of the consonant phonemes. All consonants occur in the onset position, both word-initially and word-internally, except for the velar nasal /\textstyleChCharisSIL{ŋ}/. While it occurs rather commonly in the word-internal onset position, it is unattested as word-initial onset.\footnote{\\
\\
\\
\\
\\
\\
\\
This restricted phonotactic distribution of the velar nasal is cross-linguistically rather common. Following {\citet[7]{Anderson2011}} it has to do with “word-edge” and “word-medial” phonotactics in general: “word-edge coda and onset positions seem to be more restricted than corresponding coda and onset positions in non-edge positions”.\\
\\
\\
\\
\\
\\
\\
\\
}
\end{styleBodyafterxivpt}


The range of consonants occurring as a coda is considerably smaller. The voiceless stops, fricative /\textstyleChCharisSIL{s}/, and the four sonorants (liquids and approximants) occur as coda, both word-internally and word-finally. By contrast, the following segments are unattested as coda, both word-internally and word-finally: the voiced stops, the affricates, and the glottal fricative.\footnote{\\
\\
\\
\\
\\
\\
\\
In the word-final coda position, the glottal fricative /\textstyleChCharisSILviiivpt{h}/ is only attested in interjections.\\
\\
\\
\\
\\
\\
\\
\\
} As for the nasals, only bilabial /m/ and velar /\textstyleChCharisSIL{ŋ}/ occur as word-internal or word-final codas, with the velar nasal assimilating to a following stop or affricate (§2.2.1).


\begin{stylecaption}
\label{bkm:Ref324336580}Table ‎2.\stepcounter{Table}{\theTable}:  Distribution of the consonant phonemes
\end{stylecaption}

\begin{tabular}{lllllllllllllllllll} & \multicolumn{6}{l}{ \textsc{stop}} & \multicolumn{2}{l}{ \textsc{affr}} & \multicolumn{2}{l}{ \textsc{fric}} & \multicolumn{4}{l}{ \textsc{nas}} & \multicolumn{2}{l}{ \textsc{liq}} & \multicolumn{2}{l}{ \textsc{apr}}\\
\lsptoprule
& \textstyleChCharisSIL{p} & \textstyleChCharisSIL{b} & \textstyleChCharisSIL{t} & \textstyleChCharisSIL{d} & \textstyleChCharisSIL{k} & \textstyleChCharisSIL{g} & \textstyleChCharisSIL{tʃ} & \textstyleChCharisSIL{dʒ} & \textstyleChCharisSIL{s} & \textstyleChCharisSIL{h} & \textstyleChCharisSIL{m} & \textstyleChCharisSIL{n} & \textstyleChCharisSIL{ɲ} & \textstyleChCharisSIL{ŋ} & \textstyleChCharisSIL{r} & \textstyleChCharisSIL{l} & \textstyleChCharisSIL{j} & \arraybslash \textstyleChCharisSIL{w}\\
\textsc{onset} & + & + & + & + & + & + & + & + & + & + & + & + & + & (+) & + & + & + & \arraybslash +\\
\textsc{coda} & + & – & + & – & + & – & – & – & + & – & \textstyleChCharisSIL{m} & \textstyleChCharisSIL{ŋ} & \textstyleChCharisSIL{ŋ} & \textstyleChCharisSIL{ŋ} & + & + & + & \arraybslash +\\
\lspbottomrule
\end{tabular}

A restricted sample of consonants can occur in onset CC clusters, as illustrated in Table  ‎2 .26. The range of consonants occurring in word-initial clusters is considerably larger than the range of consonants occurring in word-internal clusters.


\begin{stylecaption}
\label{bkm:Ref324338904}Table ‎2.\stepcounter{Table}{\theTable}:  CC clusters – Examples
\end{stylecaption}

\tablehead{
\multicolumn{3}{l}{ Word-initial position} & \multicolumn{3}{l}{ Word-internal position}\\
}
\begin{tabular}{llllll}
\lsptoprule
\multicolumn{6}{l}{ Stops in C1 position}\\
/\textstyleChCharisSIL{p}C\textsubscript{2}/ & /\textstyleChCharisSIL{ˈ}\textstyleChCharisSILBlueBold{pr}\textstyleChCharisSIL{aŋ}/ & ‘war’ &  &  & \\
& /\textstyleChCharisSIL{ˈ}\textstyleChCharisSILBlueBold{pl}\textstyleChCharisSIL{aŋ}/ & ‘be slow’ &  &  & \\
/\textstyleChCharisSIL{b}C\textsubscript{2}/ & /\textstyleChCharisSIL{ˈ}\textstyleChCharisSILBlueBold{br}\textstyleChCharisSIL{at}/ & ‘be heavy’ & /\textstyleChCharisSIL{b}C\textsubscript{2}/ & /\textstyleChCharisSIL{ˈta.}\textstyleChCharisSILBlueBold{br}\textstyleChCharisSIL{ak}/ & ‘hit against’\\
& /\textstyleChCharisSIL{ˈ}\textstyleChCharisSILBlueBold{bl}\textstyleChCharisSIL{a.kaŋ}/ & ‘back’ &  & /\textstyleChCharisSIL{ˈtʃɔ.}\textstyleChCharisSILBlueBold{bl}\textstyleChCharisSIL{ɔs}/ & ‘punch’\\
/\textstyleChCharisSIL{t}C\textsubscript{2}/ & /\textstyleChCharisSIL{ˈ}\textstyleChCharisSILBlueBold{tr}\textstyleChCharisSIL{aŋ}/ & ‘be clear’ &  &  & \\
& /\textstyleChCharisSIL{ˈ}\textstyleChCharisSILBlueBold{tl}\textstyleChCharisSIL{an.dʒaŋ}/ & ‘be naked’ &  &  & \\
/\textstyleChCharisSIL{d}C\textsubscript{2}/ & /\textstyleChCharisSIL{ˈ}\textstyleChCharisSILBlueBold{dl}\textstyleChCharisSIL{a.paŋ}/ & ‘eight’ & /\textstyleChCharisSIL{d}C\textsubscript{2}/ & /\textstyleChCharisSIL{ˈgɔn.}\textstyleChCharisSILBlueBold{dr}\textstyleChCharisSIL{ɔŋ}/ & ‘be long haired’\\
/\textstyleChCharisSIL{tʃ}C\textsubscript{2}/ & /\textstyleChCharisSIL{ˈ}\textstyleChCharisSILBlueBold{tʃr}\textstyleChCharisSIL{ɛ.wɛt}/ & ‘chatty’ &  &  & \\
/\textstyleChCharisSIL{k}C\textsubscript{2}/ & /\textstyleChCharisSIL{ˈ}\textstyleChCharisSILBlueBold{kn}\textstyleChCharisSIL{a.pa}/ & ‘why’ & /\textstyleChCharisSIL{k}C\textsubscript{2}/ & /\textstyleChCharisSIL{ˈdʒaŋ.}\textstyleChCharisSILBlueBold{kr}\textstyleChCharisSIL{ik}/ & ‘cricket’\\
& /\textstyleChCharisSIL{ˈ}\textstyleChCharisSILBlueBold{kr}\textstyleChCharisSIL{iŋ}/ & ‘be dry’ &  &  & \\
& /\textstyleChCharisSIL{ˈ}\textstyleChCharisSILBlueBold{kl}\textstyleChCharisSIL{ɔm.pɔk}/ & ‘group’ &  &  & \\
& /\textstyleChCharisSIL{ˈ}\textstyleChCharisSILBlueBold{kw}\textstyleChCharisSIL{ali}/ & ‘frying pan’ &  &  & \\
/\textstyleChCharisSIL{g}C\textsubscript{2}/ & /\textstyleChCharisSIL{ˈ}\textstyleChCharisSILBlueBold{gn}\textstyleChCharisSIL{ɛ.mɔ}/ & ‘melinjo tree’ &  &  & \\
& /\textstyleChCharisSIL{ˈ}\textstyleChCharisSILBlueBold{gl}\textstyleChCharisSIL{ap}/ & ‘be dark’ &  &  & \\
\multicolumn{6}{l}{ Fricatives in C\textsubscript{1} position}\\
/\textstyleChCharisSIL{s}C\textsubscript{2}/ & /\textstyleChCharisSIL{ˈ}\textstyleChCharisSILBlueBold{sp}\textstyleChCharisSIL{ɛr.ti}/ & ‘like’ & /\textstyleChCharisSIL{s}C\textsubscript{2}/ & /\textstyleChCharisSIL{ka.ˈ}\textstyleChCharisSILBlueBold{sw}\textstyleChCharisSIL{a.ri}/ & ‘cassowary’\\
& /\textstyleChCharisSIL{ˈ}\textstyleChCharisSILBlueBold{sk}\textstyleChCharisSIL{a.raŋ}/ & ‘now’ &  &  & \\
& /\textstyleChCharisSIL{ˈ}\textstyleChCharisSILBlueBold{sm}\textstyleChCharisSIL{ut}/ & ‘ant’ &  &  & \\
& /\textstyleChCharisSIL{ˈ}\textstyleChCharisSILBlueBold{sn}\textstyleChCharisSIL{aŋ}/ & ‘be happy’ &  &  & \\
& /\textstyleChCharisSIL{ˈ}\textstyleChCharisSILBlueBold{sr}\textstyleChCharisSIL{iŋ}/ & ‘often’ &  &  & \\
& /\textstyleChCharisSIL{ˈ}\textstyleChCharisSILBlueBold{sl}\textstyleChCharisSIL{a.taŋ}/ & ‘south’ &  &  & \\
& /\textstyleChCharisSIL{ˈ}\textstyleChCharisSILBlueBold{sw}\textstyleChCharisSIL{ak}/ & ‘be exhausted’ &  &  & \\
\lspbottomrule
\end{tabular}

Cross-linguistically, the creation of consonant clusters tends to be constrained and guided by the “Sonority Sequencing Principle that requires onsets to rise in sonority toward the nucleus” {\citep[254]{Kenstowicz1994}}: vowels are the most sonorous, followed by glides, liquids, nasals, and obstruents. Following the Sonority Sequencing Principle, C\textsubscript{1} “may be added to the onset only if it is less sonorous” than C\textsubscript{2} {(1994: 255)}. Hence, CC clusters are most commonly formed with an obstruent in C\textsubscript{1} position and a glide in C\textsubscript{2} position. The second most common are liquids or nasals occurring in C\textsubscript{2} position, while CC clusters with an obstruent in C\textsubscript{2} position are the least common. For the most part, the attested Papuan Malay CC clusters agree with the Sonority Sequencing Principle, as illustrated in Table  ‎2 .27: all CC clusters to the right of the bold line obey the Sonority Sequencing Principle. Only two clusters are attested that do not agree with this principle. They are found to the left of the bold line. Both clusters have alveolar /\textstyleChCharisSIL{s}/ in C\textsubscript{1} position and /\textstyleChCharisSIL{p}/ or /\textstyleChCharisSIL{k}/ in C\textsubscript{2} position.
\end{styleBodyaftervbeforexivpt}


All CC clusters listed in Table  ‎2 .27 occur as word-initial onset, while some of them are also found as word-internal onset. In Table  ‎2 .27 the latter clusters are underlined. Consonant sequences in the coda position are unattested. The data show a clear preference for CC clusters with the lateral /\textstyleChCharisSIL{l}/ in C\textsubscript{2} position (29 entries), followed by clusters with rhotic /\textstyleChCharisSIL{r}/ in C\textsubscript{2} position (18 entries). CC clusters with the velar approximant /\textstyleChCharisSIL{w}/ (4 entries) or a nasal (3 entries) in C\textsubscript{2} position are much less common. Clusters with a stop in C\textsubscript{2} position are even less common (2 entries).


\begin{stylecaption}
\label{bkm:Ref324336583}Table ‎2.\stepcounter{Table}{\theTable}:  CC clusters – Overview
\end{stylecaption}

\tablehead{
\multicolumn{2}{l}{ C\textsubscript{1}C\textsubscript{2}} & \multicolumn{10}{l}{ \textsc{obstr}} & \multicolumn{4}{l}{ \textsc{nas}} & \multicolumn{2}{l}{ \textsc{liq}} & \multicolumn{2}{l}{ \textsc{apr}}\\
\multicolumn{2}{l}{} & \textstyleChCharisSIL{p} & \textstyleChCharisSIL{b} & \textstyleChCharisSIL{t} & \textstyleChCharisSIL{d} & \textstyleChCharisSIL{tʃ} & \textstyleChCharisSIL{dʒ} & \textstyleChCharisSIL{k} & \textstyleChCharisSIL{g} & \textstyleChCharisSIL{s} & \textstyleChCharisSIL{h} & \textstyleChCharisSIL{m} & \textstyleChCharisSIL{n} & \textstyleChCharisSIL{ɲ} & \textstyleChCharisSIL{ŋ} & \textstyleChCharisSIL{r} & \textstyleChCharisSIL{l} & \textstyleChCharisSIL{j} & \arraybslash \textstyleChCharisSIL{w}\\
}
\begin{tabular}{llllllllllllllllllll}
\lsptoprule
 \textsc{obstr} & \textstyleChCharisSIL{p} &  &  &  &  &  &  &  &  &  &  &  &  &  &  & \textstyleChCharisSIL{pr} & \textstyleChCharisSIL{pl} &  & \\
& \textstyleChCharisSIL{b} &  &  &  &  &  &  &  &  &  &  &  &  &  &  & \textstyleChCharisSILUnderl{br} & \textstyleChCharisSILUnderl{bl} &  & \\
& \textstyleChCharisSIL{t} &  &  &  &  &  &  &  &  &  &  &  &  &  &  & \textstyleChCharisSIL{tr} & \textstyleChCharisSIL{tl} &  & \\
& \textstyleChCharisSIL{d} &  &  &  &  &  &  &  &  &  &  &  &  &  &  & \textstyleChCharisSILUnderl{dr} & \textstyleChCharisSILUnderl{dl} &  & \\
& \textstyleChCharisSIL{tʃ} &  &  &  &  &  &  &  &  &  &  &  &  &  &  & \textstyleChCharisSIL{tʃr} &  &  & \\
& \textstyleChCharisSIL{dʒ} &  &  &  &  &  &  &  &  &  &  &  &  &  &  &  &  &  & \\
& \textstyleChCharisSIL{k} &  &  &  &  &  &  &  &  &  &  &  & \textstyleChCharisSIL{kn} &  &  & \textstyleChCharisSILUnderl{kr} & \textstyleChCharisSIL{kl} &  & \arraybslash \textstyleChCharisSIL{kw}\\
& \textstyleChCharisSIL{g} &  &  &  &  &  &  &  &  &  &  &  & \textstyleChCharisSIL{gn} &  &  &  & \textstyleChCharisSIL{gl} &  & \\
& \textstyleChCharisSIL{s} & \textstyleChCharisSIL{sp} &  &  &  &  &  & \textstyleChCharisSIL{sk} &  &  &  & \textstyleChCharisSIL{sm} & \textstyleChCharisSIL{sn} &  &  & \textstyleChCharisSIL{sr} & \textstyleChCharisSIL{sl} &  & \arraybslash \textstyleChCharisSILUnderl{sw}\\
& h &  &  &  &  &  &  &  &  &  &  &  &  &  &  &  &  &  & \\
 \textsc{nas} & \textstyleChCharisSIL{p} &  &  &  &  &  &  &  &  &  &  &  &  &  &  &  &  &  & \\
& \textstyleChCharisSIL{n} &  &  &  &  &  &  &  &  &  &  &  &  &  &  &  &  &  & \\
& \textstyleChCharisSIL{ɲ} &  &  &  &  &  &  &  &  &  &  &  &  &  &  &  &  &  & \\
& \textstyleChCharisSIL{ŋ} &  &  &  &  &  &  &  &  &  &  &  &  &  &  &  &  &  & \\
 \textsc{liq} & \textstyleChCharisSIL{r} &  &  &  &  &  &  &  &  &  &  &  &  &  &  &  &  &  & \\
& \textstyleChCharisSIL{l} &  &  &  &  &  &  &  &  &  &  &  &  &  &  &  &  &  & \\
 \textsc{apr} & \textstyleChCharisSIL{j} &  &  &  &  &  &  &  &  &  &  &  &  &  &  &  &  &  & \\
& \textstyleChCharisSIL{w} &  &  &  &  &  &  &  &  &  &  &  &  &  &  &  &  &  & \\
\hhline{~-------------------}
\lspbottomrule
\end{tabular}
\subsection{Vowel phoneme distribution and sequences}
\label{bkm:Ref324758217}
All five vowels occur in stressed and unstressed, open and closed syllables, as illustrated in Table  ‎2 .28.


\begin{stylecaption}
\label{bkm:Ref324430910}Table ‎2.\stepcounter{Table}{\theTable}:  Distribution of vowels in stressed and unstressed syllables
\end{stylecaption}

\tablehead{
 Phoneme & \multicolumn{2}{l}{ Stressed open \textsc{sylb}} & \multicolumn{2}{l}{ Stressed closed \textsc{sylb}}\\
}
\begin{tabular}{lllll}
\lsptoprule
/\textstyleChCharisSIL{i}/ & /\textstyleChCharisSIL{ˈbi.su}/ & ‘be mute’ & /\textstyleChCharisSIL{ˈtim.ba}/ & ‘fetch’\\
/\textstyleChCharisSIL{u}/ & /\textstyleChCharisSIL{ˈpu.ti}/ & ‘be white’ & /\textstyleChCharisSIL{ˈmun.ta}/ & ‘vomit’\\
/\textstyleChCharisSIL{ɛ}/ & /\textstyleChCharisSIL{ˈmɛ.ra}/ & ‘be red’ & /\textstyleChCharisSIL{ˈsɛn.tu}/ & ‘touch’\\
/\textstyleChCharisSIL{ɔ}/ & /\textstyleChCharisSIL{ˈgɔ.dɛ}/ & ‘be fat’ & /\textstyleChCharisSIL{ˈlɔm.ba}/ & ‘contest’\\
/\textstyleChCharisSIL{a}/ & /\textstyleChCharisSIL{ˈra.dʒu}/ & ‘pout’ & /\textstyleChCharisSIL{ˈgaŋ.gu}/ & ‘disturb’\\
 Phoneme & \multicolumn{2}{l}{ Unstressed open \textsc{sylb}} & \multicolumn{2}{l}{ Unstressed closed \textsc{sylb}}\\
/\textstyleChCharisSIL{i}/ & /\textstyleChCharisSIL{ˈba.bi}/ & ‘pig’ & /\textstyleChCharisSIL{ˈma.nis}/ & ‘be sweet’\\
/\textstyleChCharisSIL{u}/ & /\textstyleChCharisSIL{ˈka.ju}/ & ‘wood’ & /\textstyleChCharisSIL{ˈta.kut}/ & ‘fear’\\
/\textstyleChCharisSIL{ɛ}/ & /\textstyleChCharisSIL{ˈtʃa.pɛ}/ & ‘be tired’ & /\textstyleChCharisSIL{ˈsɔ.bɛk}/ & ‘tear’\\
/\textstyleChCharisSIL{ɔ}/ & /\textstyleChCharisSIL{ˈga.rɔ}/ & ‘scratch’ & /\textstyleChCharisSIL{ˈbɛ.sɔk}/ & ‘tomorrow’\\
/\textstyleChCharisSIL{a}/ & /\textstyleChCharisSIL{ˈbu.ta}/ & ‘be blind’ & /\textstyleChCharisSIL{ˈli.pat}/ & ‘fold’\\
\lspbottomrule
\end{tabular}

A restricted set of vowel segments can occur in V.V vowel sequences, as shown in Table  ‎2 .29. As far as attested, two examples are given for each V.V sequence. The first has a /\textstyleChCharisSIL{ˈ}(C)V.V/ stress pattern in which the syllable containing V\textsubscript{1} is stressed. The second example has a /CV.\textstyleChCharisSIL{ˈ}V/ stress pattern in which V\textsubscript{2} is stressed. Of the 51 lexical roots containing V.V sequences, 43 items (84\%) have a /\textstyleChCharisSIL{ˈ}(C)V.V/ stress pattern, while only eight items (16\%) show a /CV.\textstyleChCharisSIL{ˈ}V/ stress pattern. The V.V sequences are realized without an inserted glottal stop.\footnote{\\
\\
\\
\\
\\
\\
\\
Very commonly, speakers realize a /\textstyleChCharisSILviiivpt{i.}V/ sequence with a brief transitional glide. Since this is an almost universal phenomenon, the transitional glide is not transcribed.\\
\\
\\
\\
\\
\\
\\
\\
}


\begin{stylecaption}
\label{bkm:Ref333398465}Table ‎2.\stepcounter{Table}{\theTable}:  V.V sequences – Examples
\end{stylecaption}

\tablehead{
 \textsubscript{V1}.V\textsubscript{2} & Stress & Item & Gloss & \arraybslash Freq.\\
}
\begin{tabular}{lllll}
\lsptoprule
/\textstyleChCharisSIL{i.u}/ & /\textstyleChCharisSIL{ˈ}C\textstyleChCharisSIL{i.u}/ & /\textstyleChCharisSIL{ˈtʃ}\textstyleChCharisSILUnderl{i.u}\textstyleChCharisSIL{m}/ & ‘kiss’ & \raggedleft\arraybslash 2\\
/\textstyleChCharisSIL{i.a}/ & /\textstyleChCharisSIL{ˈ}C\textstyleChCharisSIL{i.a}/ & /\textstyleChCharisSIL{ˈd}\textstyleChCharisSILUnderl{i.a}\textstyleChCharisSIL{m}/ & ‘be quiet’ & \raggedleft\arraybslash \textstyleChBold{\textmd{12}}\\
& /C\textstyleChCharisSIL{i.ˈa}/ & /\textstyleChCharisSIL{g}\textstyleChCharisSILUnderl{i.ˈa}\textstyleChCharisSIL{.was}/ & ‘guava’ & \raggedleft\arraybslash \textstyleChBold{\textmd{4}}\\
/\textstyleChCharisSIL{u.a}/ & /\textstyleChCharisSIL{ˈu.a}/ & /\textstyleChCharisSIL{ˈ}\textstyleChCharisSILUnderl{u.a}\textstyleChCharisSIL{ŋ}/ & ‘money’{85} & \raggedleft\arraybslash \textstyleChBold{\textmd{1}}\\
& /\textstyleChCharisSIL{ˈ}C\textstyleChCharisSIL{u.a}/ & /\textstyleChCharisSIL{ˈb}\textstyleChCharisSILUnderl{u.a}\textstyleChCharisSIL{t}/ & ‘make’ & \raggedleft\arraybslash \textstyleChBold{\textmd{15}}\\
& /C\textstyleChCharisSIL{u.ˈa}/ & /\textstyleChCharisSIL{b}\textstyleChCharisSILUnderl{u.ˈa}\textstyleChCharisSIL{.ja}/ & ‘crocodile’ & \raggedleft\arraybslash \textstyleChBold{\textmd{4}}\\
/\textstyleChCharisSIL{a.i}/ & /\textstyleChCharisSIL{ˈa.i}/ & /\textstyleChCharisSIL{ˈ}\textstyleChCharisSILUnderl{a.i}\textstyleChCharisSIL{r}/ & ‘water’ & \raggedleft\arraybslash \textstyleChBold{\textmd{1}}\\
& /\textstyleChCharisSIL{ˈ}C\textstyleChCharisSIL{a.i}/ & /\textstyleChCharisSIL{ˈb}\textstyleChCharisSILUnderl{a.i}\textstyleChCharisSIL{k}/ & ‘be good’ & \raggedleft\arraybslash \textstyleChBold{\textmd{7}}\\
/\textstyleChCharisSIL{a.u}/ & /\textstyleChCharisSIL{ˈ}C\textstyleChCharisSIL{a.u}/ & /\textstyleChCharisSIL{ˈd}\textstyleChCharisSILUnderl{a.u}\textstyleChCharisSIL{ŋ}/ & ‘leaf’ & \raggedleft\arraybslash \textstyleChBold{\textmd{5}}\\
\lspbottomrule
\end{tabular}

The attested V.V sequences with their frequencies are summarized in Table  ‎2 .30. This overview, together with the data presented in Table  ‎2 .29, shows that the V\textsubscript{1} position is typically taken by a close vowel (38/51 lexical roots – 74\%), while the open central vowel (36/51 lexical roots – 71\%) typically takes the V\textsubscript{2} position.


\begin{stylecaption}
\label{bkm:Ref324430913}Table ‎2.\stepcounter{Table}{\theTable}:  V.V sequences and frequencies – Overview
\end{stylecaption}

\begin{tabular}{llllllll}
\lsptoprule

 \textsubscript{V1}.V\textsubscript{2} & \multicolumn{2}{l}{ \textstyleChCharisSIL{i}} & \multicolumn{2}{l}{ \textstyleChCharisSIL{u}} & \multicolumn{2}{l}{ \textstyleChCharisSIL{a}} & \arraybslash Total\\
 i & \textstyleChCharisSIL{{}-{}-{}-} & \raggedleft 0 & \textstyleChCharisSIL{i.u} & \raggedleft 2 & \textstyleChCharisSIL{i.a} & \raggedleft 16 & \raggedleft\arraybslash 18\\
 u & \textstyleChCharisSIL{{}-{}-{}-} & \raggedleft 0 & \textstyleChCharisSIL{{}-{}-{}-} & \raggedleft 0 & \textstyleChCharisSIL{u.a} & \raggedleft 20 & \raggedleft\arraybslash 20\\
 a & \textstyleChCharisSIL{a.i} & \raggedleft 8 & \textstyleChCharisSIL{a.u} & \raggedleft 5 & \textstyleChCharisSIL{{}-{}-{}-} & \raggedleft 0 & \raggedleft\arraybslash 13\\
Total &  & \raggedleft 8 &  & \raggedleft 7 &  & \raggedleft 36 & \raggedleft\arraybslash 51\\
\lspbottomrule
\end{tabular}

Following {Parker’s (2008: 60)} “hierarchy of relative sonority”, most of the Papuan Malay V.V sequences are sequences of rising sonority with the open vowel /\textstyleChCharisSIL{a}/ in V\textsubscript{2} position having higher sonority than the close vowels /\textstyleChCharisSIL{i}/ and /\textstyleChCharisSIL{u}/ in V\textsubscript{1} position (36/51 – 71\%). There are two exceptions: first, the two lexical entries with an /\textstyleChCharisSIL{i.u}/ vowel sequence, with both vowels having the same relative sonority, and second, the 13 lexical roots with an /\textstyleChCharisSIL{a.i}/ or /\textstyleChCharisSIL{a.u}/ vowel sequence.
\end{styleBodyaftervbeforexivpt}


The remainder of this section discusses the analysis of the vowel combinations /\textstyleChCharisSIL{ai}/ and /\textstyleChCharisSIL{au}/ as the V.V sequences /\textstyleChCharisSIL{a.i}/ and /\textstyleChCharisSIL{a.u}/, or rather as the VC sequences /\textstyleChCharisSIL{aj}/ and /\textstyleChCharisSIL{aw}/, respectively. When /\textstyleChCharisSIL{ai}/ and /\textstyleChCharisSIL{au}/ occur in closed syllables, they are analyzed as the V.V sequences /\textstyleChCharisSIL{a.i}/ and /\textstyleChCharisSIL{a.u}/. The actual pronunciations of /\textstyleChCharisSIL{ai}/ and /\textstyleChCharisSIL{au}/ do not indicate, however, that they are V.V sequences. Examples are \textitbf{baik} /\textstyleChCharisSIL{ˈba.ik}/ ‘be good’ or \textitbf{laut} /\textstyleChCharisSIL{ˈla.ut}/ ‘sea’. When /\textstyleChCharisSIL{ai}/ and /\textstyleChCharisSIL{au}/ occur at syllable boundaries, they are analyzed as the VC sequences /\textstyleChCharisSIL{aj}/ and /\textstyleChCharisSIL{aw}/, respectively. Examples are \textitbf{damay} /\textstyleChCharisSIL{ˈ}da.\textstyleChCharisSIL{m}aj/ ‘peace’ and \textitbf{baw} /\textstyleChCharisSIL{ˈbaw}/ ‘smell’. This analysis is based on phonological and prosodic evidence, that is, the distribution of the vowel and consonant phonemes, as well as the syllable structures and stress patterns.



The first piece of evidence to be discussed is the vowel phoneme distribution. The five vowels occur in stressed and unstressed, open and closed syllables, as shown in Table  ‎2 .28. If the vowel combinations /\textstyleChCharisSIL{ai}/ and /\textstyleChCharisSIL{au}/ were diphthongs, they should occur in the same contexts where the five vowels occur. This, however, is not the case, as demonstrated in Table  ‎2 .31. The putative diphthong /ai/ (or centralized [\textstyleChCharisSIL{ɛɪ}]) occurs in stressed and unstressed open syllables. As for closed syllables, however, /\textstyleChCharisSIL{ai}/ occurs only once in a stressed syllable while it is unattested in unstressed syllables. The distribution of the putative diphthong /\textstyleChCharisSIL{au}/ is even more restricted. In disyllabic roots, /\textstyleChCharisSIL{au}/ only occurs in unstressed open syllables. In addition, the corpus contains eight monosyllabic items with /\textstyleChCharisSIL{au}/: three open monosyllabic items such as [\textstyleChCharisSIL{ˈtaʊ}] ‘know’ and five closed items such as [\textstyleChCharisSIL{ˈdaʊŋ}] ‘leaf’. The same distributional patterns apply to loan words.


\begin{stylecaption}
\label{bkm:Ref357271820}Table ‎2.\stepcounter{Table}{\theTable}:  Distribution of the putative diphthongs /\textstyleChCharisSIL{ai}/ and /\textstyleChCharisSIL{au}/ in stressed and unstressed syllables
\end{stylecaption}

\begin{tabular}{lllll} & \multicolumn{2}{l}{ Stressed open \textsc{sylb}} & \multicolumn{2}{l}{ Stressed closed \textsc{sylb}}\\
\lsptoprule
/a\textstyleChCharisSIL{i}/ & [\textstyleChCharisSIL{tʃɛ.ˈɾɛɪ}] & ‘divorce’ & [\textstyleChCharisSIL{mu.ˈdʒaɪr}] & ‘tilapiine fish’\\
/\textstyleChCharisSIL{au}/ & ([\textstyleChCharisSIL{ˈtaʊ}] & ‘know’) & ([\textstyleChCharisSIL{ˈdaʊŋ}] & ‘leaf’)\\
& \multicolumn{2}{l}{ Unstressed open \textsc{sylb}} & \multicolumn{2}{l}{ Unstressed closed \textsc{sylb}}\\
/\textstyleChCharisSIL{ai}/ & [\textstyleChCharisSIL{ˈtu.paɪ}] & ‘squirrel’ & {}-{}-{}- & \arraybslash {}-{}-{}-\\
/\textstyleChCharisSIL{au}/ & [\textstyleChCharisSIL{ˈki.tʃaʊ}] & ‘be naughty’ & {}-{}-{}- & \arraybslash {}-{}-{}-\\
\lspbottomrule
\end{tabular}

This constraint against diphthongs in unstressed (and stressed) closed syllables supports the analysis of /\textstyleChCharisSIL{ai}/ and /\textstyleChCharisSIL{au}/ as VC combinations or vowel sequences, rather than as diphthongs. Hence, when /\textstyleChCharisSIL{ai}/ and /\textstyleChCharisSIL{au}/ occur at syllable boundaries, they are analyzed as VC combinations. Examples are \textitbf{cerey} /\textstyleChCharisSIL{tʃɛ.ˈrɛj}/ ‘divorce’, \textitbf{taw} /\textstyleChCharisSIL{ˈtaw}/ ‘know’, \textitbf{tupay} /\textstyleChCharisSIL{ˈtu.paj}/ ‘squirrel’, and \textitbf{kicaw} /\textstyleChCharisSIL{ˈki.tʃaw} / ‘be naughty’. By contrast, when the second vowel, that is /i/ or /u/, occurs in a closed syllable, /\textstyleChCharisSIL{ai}/ and /\textstyleChCharisSIL{au}/ are analyzed as vowel sequences. Examples are \textitbf{mujair} /\textstyleChCharisSIL{mu.ˈdʒa.ir}/ ‘tilapiine fish’, and \textitbf{daung} /\textstyleChCharisSIL{ˈda.uŋ}/ ‘leaf’.
\end{styleBodyaftervbeforexivpt}


The second piece of evidence is the consonant phoneme distribution (see also §2.4.1). As already mentioned, /\textstyleChCharisSIL{ai}/ and /\textstyleChCharisSIL{au}/ are analyzed as the VC sequences /\textstyleChCharisSIL{aj}/ and /\textstyleChCharisSIL{aw}/ respectively when they occur at syllable boundaries. If instead /\textstyleChCharisSIL{ai}/ and /\textstyleChCharisSIL{au}/ were analyzed as diphthongs, this would affect the consonant phoneme distribution, since in that case the two approximants /\textstyleChCharisSIL{j}/ and /\textstyleChCharisSIL{w}/ would only occur in the onset position of a syllable but not in the coda position. This distribution, however, does not agree with that of the other sonorants, given that the liquids and also the nasals, although not all of them, occur in both positions. The analysis of /\textstyleChCharisSIL{ai}/ and /\textstyleChCharisSIL{au}/ as /\textstyleChCharisSIL{aj}/ and /\textstyleChCharisSIL{aw}/ at syllable boundaries fills this gap. Given, however, that coda /\textstyleChCharisSIL{j}/ and /\textstyleChCharisSIL{w}/ do not freely follow all vowels but only /\textstyleChCharisSIL{a}/, this could also be taken as evidence that /\textstyleChCharisSIL{ai}/ and /\textstyleChCharisSIL{au}/ are better analyzed as diphthongs.



The third piece of evidence has to do with syllable structures and stress patterns. Papuan Malay has a clear preference for disyllabic roots and CV(C) syllables (see §2.4.3), and stress typically falls on the penultimate syllable (see §2.4.4). The corpus contains 26 lexical roots with an /\textstyleChCharisSIL{ai}/ or /\textstyleChCharisSIL{au}/ vowel combination. Of these, 13 are analyzed as VC combinations (eight /aj/ and five /aw/ combinations). The remaining 13 vowel combination are analyzed as vowel sequences (eight /a.i/ and five /a.u/ sequences). These 13 vowel sequences occur in lexical roots with penultimate stress; that is, /a/ belongs to the stressed penultimate syllable, while the close vowel belongs to the unstressed ultimate syllable. If these 13 sequences are analyzed as diphthongs instead, the syllable structure of the respective roots changes and 12 of them become monosyllabic. This increases the number of monosyllabic roots from 44 to 56, an increase of 27\%. Such an increase, however, seems to be disproportionally high given the strong preference for disyllabic roots. With respect to the stress patterns, evidence comes from one lexical root and four (historically) affixed items. In the lexical root \textitbf{mujair} /\textstyleChCharisSIL{mu.ˈdʒa.ir}/ ‘tilapiine fish’ stress falls on the preferred penultimate syllable. If /\textstyleChCharisSIL{ai}/ is analyzed as a diphthong, stress instead falls on the dispreferred ultimate syllable, [\textstyleChCharisSIL{mu.ˈdʒair}]. Further, as mentioned above, the actual pronunciation of the /\textstyleChCharisSIL{ai}/ or /\textstyleChCharisSIL{au}/ vowel combinations does not suggest that they are V.V sequences. This, however, does not apply to four (historically) affixed items with penultimate stress, presented in Table  ‎2 .32. In these items, the penultimate stress audibly breaks up the /\textstyleChCharisSIL{ai}/ and /\textstyleChCharisSIL{ai}/ vowel combinations with the close vowel receiving stress. This is taken as evidence that in the four respective roots /\textstyleChCharisSIL{ai}/ and /\textstyleChCharisSIL{au}/ are V.V sequences rather than diphthongs.


\begin{styleCaptionxivptSpace}
\label{bkm:Ref357261347}Table ‎2.\stepcounter{Table}{\theTable}:  Vowel combinations /\textstyleChCharisSIL{ai}/ and /\textstyleChCharisSIL{au}/ in (historically) affixed items
\end{styleCaptionxivptSpace}

\begin{tabular}{llllll}
\lsptoprule

\multicolumn{2}{l}{ (Historically) affixed items} & Gloss & \multicolumn{2}{l}{ Roots} & \arraybslash Gloss\\
\textitbf{kebaikang} & [\textstyleChCharisSIL{ˌkɛ.ba.ˈɪ.kɐŋ}] & ‘goodness’ & \textitbf{baik} & /\textstyleChCharisSIL{ˈba.ik}/ & ‘be good’\\
\textitbf{maingang} & [\textstyleChCharisSIL{ma.ˈɪ.ŋɐŋ}] & ‘toy’ & \textitbf{maing} & /\textstyleChCharisSIL{ˈma.iŋ}/ & ‘play’\\
\textitbf{lautang} & [\textstyleChCharisSIL{la.ˈʊ.tɐŋ}] & ‘ocean’ & \textitbf{laut} & /\textstyleChCharisSIL{ˈla.ut}/ & ‘sea’\\
\textitbf{permaingang} & [\textstyleChCharisSIL{ˌpɛ̞r.ma.ˈɪ.ŋɐŋ}] & ‘game’ & \textitbf{maing} & /\textstyleChCharisSIL{ˈma.iŋ}/ & ‘play’\\
\lspbottomrule
\end{tabular}

Based on the evidence presented here, it is concluded that the analysis of the /\textstyleChCharisSIL{ai}/ and /\textstyleChCharisSIL{au}/ vowel combinations as VC combinations at syllable boundaries and as V.V sequences in closed syllables is the most efficient one. At the same time it is acknowledged, however, that there is evidence supporting the analysis of /\textstyleChCharisSIL{ai}/ and /\textstyleChCharisSIL{au}/ as diphthongs.
\end{styleBodyaftervbeforexivpt}


In the literature on eastern Malay varieties there is also some discussion concerning the question of whether these varieties have diphthongs at all, or whether vowel combinations such as /ai/ and /au/ better be analyzed as sequences of distinct vowels. For a number of eastern Malay varieties, diphthongs have been posited. For North Moluccan / Ternate Malay, {\citet[15]{Litamahuputty2012}} posits five diphthongs, /\textstyleChCharisSIL{ai}/, /\textstyleChCharisSIL{ae}/, /\textstyleChCharisSIL{ao}/, /\textstyleChCharisSIL{oi}/, and /\textstyleChCharisSIL{ei}/. In earlier studies on North Moluccan Malay, {\citet[2]{Voorhoeve1983}} suggests five diphthongs, /\textstyleChCharisSIL{ai}/, /\textstyleChCharisSIL{ae}/, /\textstyleChCharisSIL{au}/, /\textstyleChCharisSIL{ao}/, and /\textstyleChCharisSIL{oi}/, while {\citet[17]{Taylor1983}} adds a sixth diphthong, /\textstyleChCharisSIL{ei}/). For three other eastern Malay varieties, such vowel combinations have been analyzed as sequences of distinct vowels rather than as diphthongs, that is Ambon Malay {(van Minde 1997: 24)}, Larantuka Malay {\citep[105]{Paauw2009}}, and Manado Malay {\citep[12]{Stoel2005}}.


\subsection{Syllable structures}
\label{bkm:Ref333656882}\label{bkm:Ref324758218}
In Papuan Malay the minimal syllable and prosodic word consists of a single consonant and a single vowel. The maximal syllable is CCVC. Papuan Malay shows a clear preference for disyllabic roots and for CV(C) syllables. In Table  ‎2 .33 to Table  ‎2 .36 the possible arrangements of C and V for mono-and polysyllabic roots are presented in more detail. For each type the number of occurrences is given plus one example. The investigation of the syllable structure is based on the 1,117-root word list, extracted from the above-mentioned 2,458-item list.



Monosyllabic roots, with their different arrangements of C and V, are presented in Table  ‎2 .33. All roots have an onset C(C), while monosyllabic roots with (onset) V are unattested. In addition, the data shows a clear preference for closed syllables: (C)CVC (33/44 entries – 75\%).
\end{styleBodyvvafter}

\begin{stylecaption}
\label{bkm:Ref324850236}Table ‎2.\stepcounter{Table}{\theTable}:  Monosyllabic roots (44 entries)
\end{stylecaption}

\tablehead{
 Syllable types & Count & Item & \arraybslash Gloss\\
}
\begin{tabular}{llll}
\lsptoprule
CV\footnotemark{} & \raggedleft 8 & /\textstyleChCharisSIL{ˈkɔ}/ & ‘2\textsc{sg}’\\
CVC & \raggedleft 13 & /\textstyleChCharisSIL{ˈlur}/ & ‘spy on’\\
CCV & \raggedleft 3 & /\textstyleChCharisSIL{ˈbli}/ & ‘buy’\\
CCVC & \raggedleft 20 & /\textstyleChCharisSIL{ˈglap}/ & ‘dark’\\
\lspbottomrule
\end{tabular}
\footnotetext{\\
\\
\\
\\
\\
\\
\\
The corpus includes eight CV roots all of which are function words, that is, personal pronouns, prepositions, or conjunctions.\\
\\
\\
\\
\\
\\
\\
\\
}

Roots with two syllables are the most common ones. The data shows a clear preference for syllables with onset C, as shown in Table  ‎2 .34. The most common roots are CV.CV(C) (615/1,004 entries – 61\%) and CVC.CV(C) (222/1,004 entries – 22\%), while roots with onset V are rare (86/1,004 entries – 9\%). Roots with onset CC clusters are also rare (42/1,004 – 4\%).


\begin{stylecaption}
\label{bkm:Ref324850237}Table ‎2.\stepcounter{Table}{\theTable}:  Disyllabic roots (1,004 items)
\end{stylecaption}

\tablehead{
 Syllable types & Count & Item & \arraybslash Gloss\\
}
\begin{tabular}{llll}
\lsptoprule
V.VC & \raggedleft 2 & /\textstyleChCharisSIL{ˈa.ir}/ & ‘water’\footnotemark{}\\
V.CV & \raggedleft 15 & /\textstyleChCharisSIL{ˈa.pi}/ & ‘fire’\\
V.CVC & \raggedleft 52 & /\textstyleChCharisSIL{ˈi.kaŋ}/ & ‘fish’\\
VC.CVC & \raggedleft \textstyleChBold{\textmd{17}} & /\textstyleChCharisSIL{ˈam.pas}/ & ‘waste’\\
CV.V & \raggedleft \textstyleChBold{\textmd{4}} & /\textstyleChCharisSIL{ˈdua}/ & ‘two’\\
CV.VC & \raggedleft \textstyleChBold{\textmd{35}} & /\textstyleChCharisSIL{ˈbu.at}/ & ‘make’\\
CV.CV & \raggedleft 223 & /\textstyleChCharisSIL{ˈba.bi}/ & ‘pig’\\
CV.CVC & \raggedleft 392 & /\textstyleChCharisSIL{ˈgɔ.rɛŋ}/ & ‘fry’\\
CV.CCVC & \raggedleft \textstyleChBold{\textmd{3}} & /\textstyleChCharisSIL{ˈta.brak}/ & ‘hit against’\\
CVC.CV & \raggedleft \textstyleChBold{\textmd{60}} & /\textstyleChCharisSIL{ˈpan.tɛ}/ & ‘coast’\\
CVC.CVC & \raggedleft \textstyleChBold{\textmd{162}} & /\textstyleChCharisSIL{ˈtum.buk}/ & ‘pound’\\
CVC.CCVC & \raggedleft \textstyleChBold{\textmd{2}} & /\textstyleChCharisSIL{ˈdʒaŋ.krik}/ & ‘cricket’\footnotemark{}\\
CCV.VC & \raggedleft \textstyleChBold{\textmd{1}} & /\textstyleChCharisSIL{ˈklʊ.ɐr}/ & ‘go out’\footnotemark{}\\
CCV.CV & \raggedleft \textstyleChBold{\textmd{11}} & /\textstyleChCharisSIL{ˈbra.ni}/ & ‘be courageous’\\
CCV.CVC & \raggedleft \textstyleChBold{\textmd{14}} & /\textstyleChCharisSIL{ˈbla.kaŋ}/ & ‘backside’\\
CCVC.CV & \raggedleft \textstyleChBold{\textmd{5}} & /\textstyleChCharisSIL{ˈklam.bu}/ & ‘mosquito net’\\
CCVC.CVC & \raggedleft \textstyleChBold{\textmd{6}} & /\textstyleChCharisSIL{ˈglɔm.baŋ}/ & ‘wave’\\
\lspbottomrule
\end{tabular}
\addtocounter{footnote}{-3}
\stepcounter{footnote}\footnotetext{\\
\\
\\
\\
\\
\\
\\
The second item displaying a V.VC syllable structure is \textitbf{uang} ‘money’. In {\citet{Jones2007}}, \textitbf{uang} ‘money’ is not listed as a loan word, whereas {\citet{Tadmor2009a}} classifies it as a “probably borrowed”.\\
\\
\\
\\
\\
\\
\\
\\
}
\stepcounter{footnote}\footnotetext{\\
\\
\\
\\
\\
\\
\\
The second item with a CVC.CCVC syllable structure is \textitbf{gondrong} ‘be long haired’.\\
\\
\\
\\
\\
\\
\\
\\
}
\stepcounter{footnote}\footnotetext{\\
\\
\\
\\
\\
\\
\\
Neither {\citet{Jones2007} nor \citet{Tadmor2009a}} list kluar ‘go out’ as a loan word.\\
\\
\\
\\
\\
\\
\\
\\
}

Trisyllabic roots with their possible arrangements of C and V are presented in Table  ‎2 .35. Again, the data shows a clear preference for syllables with onset C. The most common roots are CV.CV.CV(C) (40/67 entries – 60\%) and CVC.CV.CV(C) (15/67 entries – 22\%). Roots with an onset CC clusters are, with one entry, very rare.


\begin{stylecaption}
\label{bkm:Ref324850238}Table ‎2.\stepcounter{Table}{\theTable}:  Trisyllabic roots (67 items)\footnote{\\
\\
\\
\\
\\
\\
\\
Three of the syllable types presented in Table  ‎2 .35 are attested only once. However, none of these items are listed as loan words in {\citet{Jones2007}}. Nor could other literature sources be found that would identify them as loans.\\
\\
\\
\\
\\
\\
\\
\\
}
\end{stylecaption}

\tablehead{
 Syllable types & Count & Item & \arraybslash Gloss\\
}
\begin{tabular}{llll}
\lsptoprule
CV.V.CV & \raggedleft 5 & /\textstyleChCharisSIL{bu.ˈa.ja}/ & ‘crocodile’\\
CV.V.CVC & \raggedleft 2 & /\textstyleChCharisSIL{ti.ˈa.rap}/ & ‘lie face downward’\footnotemark{}\\
CV.CV.VC & \raggedleft 1 & /\textstyleChCharisSIL{mu.ˈdʒa.ir}/ & ‘tilapiine fish’\\
CV.CV.CV & \raggedleft 14 & /\textstyleChCharisSIL{tɛ.ˈli.ŋa}/ & ‘ear’\\
CV.CV.CVC & \raggedleft 26 & /\textstyleChCharisSIL{bɛ.ˈla.laŋ}/ & ‘grasshopper’\\
CV.CVC.CV & \raggedleft 2 & /\textstyleChCharisSIL{pa.ˈluŋ.ku}/ & ‘punch’\footnotemark{}\\
CV.CVC.CVC & \raggedleft 1 & /\textstyleChCharisSIL{ˌgɛ.mɛn.ˈtar}/ & ‘tremble’\\
CVC.CV.CV & \raggedleft 9 & /\textstyleChCharisSIL{sɛn.ˈdi.ri}/ & ‘be alone’\\
CVC.CV.CVC & \raggedleft 6 & /\textstyleChCharisSIL{tam.ˈpɛ.lɛŋ}/ & ‘slap on face or ears’\\
CCVC.CV.VC & \raggedleft 1 & /\textstyleChCharisSIL{prɛm.ˈpu.aŋ}/ & ‘woman’\\
\lspbottomrule
\end{tabular}
\addtocounter{footnote}{-2}
\stepcounter{footnote}\footnotetext{\\
\\
\\
\\
\\
\\
\\
The second item displaying a CV.V.CVC syllable structure is \textitbf{kecuali} ‘except’.\\
\\
\\
\\
\\
\\
\\
\\
}
\stepcounter{footnote}\footnotetext{\\
\\
\\
\\
\\
\\
\\
The second item with a CV.CVC.CV syllable structure is \textitbf{kaswari} ‘cassowary’.\\
\\
\\
\\
\\
\\
\\
\\
}

Quadrisyllabic roots are presented in Table  ‎2 .36. With only two entries, they are extremely rare.\footnote{\\
\\
\\
\\
\\
\\
\\
Neither item is listed as a loan in {\citet{Jones2007}}. In addition, {A. Clynes (p.c. 2012)} and {W. Mahdi (p.c. 2012)} maintain that both items are morphologically indivisible Malay roots.\\
\\
\\
\\
\\
\\
\\
\\
} Again, the attested data show a preference for CV.


\begin{stylecaption}
\label{bkm:Ref324866747}Table ‎2.\stepcounter{Table}{\theTable}:  Quadrisyllabic roots (2 items)
\end{stylecaption}

\tablehead{
 Syllable types & Count & Item & \arraybslash Gloss\\
}
\begin{tabular}{llll}
\lsptoprule
V.CV.CV.CV & \raggedleft 1 & /\textstyleChCharisSIL{ˌɔ.la.ˈra.ga}/ & ‘do sports’\\
CV.CV.V.CV & \raggedleft 1 & /\textstyleChCharisSIL{ˌkɛ.tʃu.ˈa.li}/ & ‘except’\\
\lspbottomrule
\end{tabular}

The data presented in Table  ‎2 .33 to Table  ‎2 .36 shows that Papuan Malay has a clear preference for disyllabic roots. Roots with one or three syllables are considerably less common, while quadrisyllabic roots are rare. Table  ‎2 .37 presents a frequency count for the mono- and polysyllabic roots.


\begin{stylecaption}
\label{bkm:Ref324770127}Table ‎2.\stepcounter{Table}{\theTable}:  Frequencies of mono- and polysyllabic roots
\end{stylecaption}

\tablehead{
 Syllable types & Count & \arraybslash \%\\
}
\begin{tabular}{lll}
\lsptoprule
Monosyllabic & \raggedleft 44 & \raggedleft\arraybslash 3.9\%\\
Disyllabic & \raggedleft 1,004 & \raggedleft\arraybslash 89.9\%\\
Trisyllabic & \raggedleft 67 & \raggedleft\arraybslash 6.0\%\\
Quadrisyllabic & \raggedleft 2 & \raggedleft\arraybslash 0.2\%\\
Total & \raggedleft 1,117 & \raggedleft\arraybslash 100\%\\
\lspbottomrule
\end{tabular}

The data presented in Table  ‎2 .33 to Table  ‎2 .36 also indicates that Papuan Malay has a preference for CV(C) syllables, with the maximal syllable being (C)CVC. With these “modest expansions of the simple CV syllable type”, Papuan Malay displays a “moderately complex syllable structure” which is “by far the most common type” cross-linguistically, following {Maddieson’s (2011b: 4)} typology of syllable structure.
\end{styleBodyaftervbefore}


In his analysis, {\citet[5]{Maddieson2011b}} also observes an areal overlap and a significant, albeit not strong, correlation between consonant inventories and syllable structure:
\end{styleBodyvvafter}

\begin{styleIvI}
… languages with simple canonical syllable structure have an average of 19.1 consonants in their inventory, languages with moderately complex syllable structure have an average of 22.0 consonants, and those with complex syllable structures have an average of 25.8 consonants.
\end{styleIvI}


Hence, given its consonant inventory with 18 segments, one would expect Papuan Malay to have a simple rather than a moderately complex canonical structure.
\end{styleBodyxvafter}

\subsection{Stress patterns}
\label{bkm:Ref376624491}\label{bkm:Ref357254593}\label{bkm:Ref324758219}
In Papuan Malay, primary stress typically falls on the penultimate syllable of the lexical root, while secondary stress is assigned to the alternating syllable preceding the one carrying the primary stress. These stress patterns apply to lexical roots (§2.4.4.1) as well as to lexical items that are historically derived by (unproductive) affixation (§2.4.4.2).
\end{styleBodyxvafter}

\paragraph[Stress patterns for lexical roots]{Stress patterns for lexical roots}
\label{bkm:Ref338250048}
The basic stress patterns for di-, tri-, and quadrisyllabic lexical roots are illustrated in Table  ‎2 .38 to Table  ‎2 .40. The basis for this investigation forms the above-mentioned word list with 1,117 lexical roots.



Most disyllabic roots have penultimate stress (900/1,004 items – 90\%), as illustrated in Table  ‎2 .38. The remaining 104 items (10\%) have ultimate stress and display the following pattern. In 101 of the 104 roots (97\%), the unstressed penultimate syllable contains the front open-near vowel /\textstyleChCharisSIL{ɛ}/. In the remaining three lexical roots, the unstressed penultimate syllable contains a close vowel (one item with front /i/ and two items with back /u/).\footnote{\\
\\
\\
\\
\\
\\
\\
The three items are: \textitbf{kitong} /\textstyleChCharisSILviiivpt{ki.ˈtɔŋ}/ ‘\textsc{1pl}’, \textitbf{kumur} /\textstyleChCharisSILviiivpt{ku.ˈmur}/ ‘rinse mouth’, and \textitbf{kuskus} /\textstyleChCharisSILviiivpt{kus.ˈkus}/ ‘cuscus’.\\
\\
\\
\\
\\
\\
\\
\\
} Front open-near /\textstyleChCharisSIL{ɛ}/, however, does not condition ultimate stress, as in 61 of the 899 lexical roots with penultimate stress (7\%) the stressed syllable also contains front /\textstyleChCharisSIL{ɛ}/.\footnote{\\
\\
\\
\\
\\
\\
\\
Examples are \textitbf{bebas} /\textstyleChCharisSILviiivpt{ˈbɛ.bas}/ ‘be free’ (see Table  ‎2 .38), \textitbf{leher} /\textstyleChCharisSILviiivpt{ˈlɛ.hɛr}/ ‘neck’, or \textitbf{sentu} /\textstyleChCharisSILviiivpt{ˈsɛn.tu}/ ‘touch’.\\
\\
\\
\\
\\
\\
\\
\\
}


\begin{stylecaption}
\label{bkm:Ref331266383}Table ‎2.\stepcounter{Table}{\theTable}:  Stress patterns for disyllabic lexical roots (1,004 items)
\end{stylecaption}

\tablehead{
 Stress & Item & Orthogr. & \arraybslash Gloss\\
}
\begin{tabular}{llll}
\lsptoprule
\textsc{p-ult} & /\textstyleChCharisSIL{ˈu.aŋ}/ & \textitbf{uang} & ‘money’\\
& /\textstyleChCharisSIL{ˈa.pi}/ & \textitbf{api} & ‘fire’\\
& /\textstyleChCharisSIL{ˈi.kaŋ}/ & \textitbf{ikang} & ‘fish’\\
& /\textstyleChCharisSIL{ˈbu.at}/ & \textitbf{buat} & ‘make’\\
& /\textstyleChCharisSIL{ˈbɛ.bas}/ & \textitbf{bebas} & ‘be free’\\
& /\textstyleChCharisSIL{ˈgɔ.rɛŋ}/ & \textitbf{goreng} & ‘fry’\\
& /\textstyleChCharisSIL{ˈtum.buk}/ & \textitbf{tumbuk} & ‘pound’\\
& /\textstyleChCharisSIL{ˈbla.kaŋ}/ & \textitbf{blakang} & ‘backside’\\
\textsc{ult} & /\textstyleChCharisSIL{ɛ.ˈnam}/ & \textitbf{enam} & ‘six’\\
& /\textstyleChCharisSIL{ɛm.ˈpat}/ & \textitbf{empat} & ‘four’\\
& /\textstyleChCharisSIL{pɛ.ˈnu}/ & \textitbf{penu} & ‘be full’\\
& /\textstyleChCharisSIL{ku.ˈmur}/ & \textitbf{kumur} & ‘rinse mouth’\\
& /\textstyleChCharisSIL{rɛn.ˈda}/ & \textitbf{renda} & ‘be low’\\
& /\textstyleChCharisSIL{dʒɛm.ˈpɔl}/ & \textitbf{jempol} & ‘thumb’\\
\lspbottomrule
\end{tabular}

Examples of trisyllabic words with penultimate and ultimate stress are presented in Table  ‎2 .39. Most trisyllabic roots have penultimate stress (63/67 items – 94\%) while only four lexical roots (6\%) have ultimate stress. Again, a pattern similar to that for disyllabic lexical roots emerges. In all four roots, the unstressed penultimate syllable contains the front open-near vowel /\textstyleChCharisSIL{ɛ}/. As in disyllabic roots, however, front open-near /\textstyleChCharisSIL{ɛ}/ does not condition ultimate stress, as in four of the 63 lexical roots with penultimate stress (6\%) the stressed syllable contains front /\textstyleChCharisSIL{ɛ}/.\footnote{\\
\\
\\
\\
\\
\\
\\
The four items are: \textitbf{papeda} /\textstyleChCharisSILviiivpt{pa.ˈpɛ.da}/ ‘sagu porridge’, \textitbf{padede} /\textstyleChCharisSILviiivpt{pa.ˈdɛ.dɛ}/ ‘whine’, \textitbf{tampeleng} /\textstyleChCharisSILviiivpt{tam.ˈpɛ.lɛŋ}/ ‘slap in face’, and \textitbf{wewenang} /\textstyleChCharisSILviiivpt{wɛ.ˈwɛ.naŋ}/ ‘authority’.\\
\\
\\
\\
\\
\\
\\
\\
}


\begin{stylecaption}
\label{bkm:Ref331266502}Table ‎2.\stepcounter{Table}{\theTable}:  Stress patterns for trisyllabic lexical roots (66 items)
\end{stylecaption}

\tablehead{
 Stress & Item & Orthogr. & \arraybslash Gloss\\
}
\begin{tabular}{llll}
\lsptoprule
\textsc{p-ult} & /\textstyleChCharisSIL{bu.ˈa.ja}/ & \textitbf{buaya} & ‘crocodile’\\
& /\textstyleChCharisSIL{ti.ˈa.rap}/ & \textitbf{tiarap} & ‘lie face downward’\\
& /\textstyleChCharisSIL{mu.ˈdʒa.ir}/ & \textitbf{mujair} & ‘tilapiine fish’\\
& /\textstyleChCharisSIL{tɛ.ˈli.ŋa}/ & \textitbf{telinga} & ‘ear’\\
& /\textstyleChCharisSIL{bɛ.ˈla.laŋ}/ & \textitbf{belalang} & ‘grasshopper’\\
& /\textstyleChCharisSIL{tam.ˈpɛ.lɛŋ}/ & \textitbf{tampeleng} & ‘slap in face’\\
& /\textstyleChCharisSIL{prɛm.ˈpu.aŋ}/ & \textitbf{prempuang} & ‘woman’\\
\textsc{ult} & /\textstyleChCharisSIL{ˌpɛ.lɛ.ˈpa}/ & \textitbf{pelepa} & ‘palm stem/midrib’\\
& /\textstyleChCharisSIL{ˌsɛ.lɛ.ˈsaj}/ & \textitbf{selesay} & ‘finish’\\
& /\textstyleChCharisSIL{ˌgɛ.mɛn.ˈtar}/ & \textitbf{gementar} & ‘tremble’\\
& /\textstyleChCharisSIL{ˌtɛŋ.gɛ.ˈlam}/ & \textitbf{tenggelam} & ‘sink’\\
\lspbottomrule
\end{tabular}

In the two attested lexical roots of four syllables, primary stress also falls on the penultimate syllable, as shown in Table  ‎2 .40.


\begin{stylecaption}
\label{bkm:Ref331266380}Table ‎2.\stepcounter{Table}{\theTable}:  Stress patterns for quadrisyllabic lexical roots (2 items)
\end{stylecaption}

\tablehead{
 Stress & Item & Orthogr. & \arraybslash Gloss\\
}
\begin{tabular}{llll}
\lsptoprule
\textsc{p-ult} & /\textstyleChCharisSIL{ˌɔ.la.ˈra.ga}/ & \textitbf{olaraga} & ‘do sports’\\
& /\textstyleChCharisSIL{ˌkɛ.tʃu.ˈa.li}/ & \textitbf{kecuali} & ‘except’\\
\lspbottomrule
\end{tabular}

The data presented in Table  ‎2 .38 to Table  ‎2 .40 demonstrates that Papuan Malay has a clear preference for penultimate stress. Of the 1,073 lexical roots with more than one syllable, 965 roots (90\%) have penultimate stress, as shown in Table  ‎2 .41. There are, however, also many lexical roots that deviate from this basic pattern and have ultimate stress (108/1,073 – 10\%). As already mentioned, in 105 of the 108 lexical roots with ultimate stress (97\%), the penultimate syllable contains the front open-near vowel /\textstyleChCharisSIL{ɛ}/. Ultimate stress, however, is not conditioned by the front open-near vowel. These findings suggest that while stress in Papuan Malay is not phonemic, it has lexicalized for these items. Minimal pairs are unattested, however.


\begin{stylecaption}
\label{bkm:Ref333585867}Table ‎2.\stepcounter{Table}{\theTable}:  Stress patterns for lexical roots – Frequencies
\end{stylecaption}

\tablehead{
 Syllable types & \textsc{p-ult} stress & \textsc{ult} stress & \arraybslash Total\\
}
\begin{tabular}{llll}
\lsptoprule
Disyllabic & \raggedleft 900 & \raggedleft 104 & \raggedleft\arraybslash 1,004\\
Trisyllabic & \raggedleft 63 & \raggedleft 4 & \raggedleft\arraybslash 67\\
Quadrisyllabic & \raggedleft 2 & \raggedleft 0 & \raggedleft\arraybslash 2\\
Total & \raggedleft 965 & \raggedleft 108 & \raggedleft\arraybslash 1,073\\
\lspbottomrule
\end{tabular}
\paragraph[Stress patterns for historically derived lexical items]{Stress patterns for historically derived lexical items}
\label{bkm:Ref374465654}\label{bkm:Ref338250050}
Lexical items that are historically derived by (unproductive) affixation show the same stress patterns as lexical roots (for details on derivation processes in Papuan Malay see §3.1). These findings are based on a word list with 380 items, extracted from the above-mentioned 2,458-item word list. The basic stress patterns of these items are exemplified in Table  ‎2 .42 to Table  ‎2 .44; the ‘Affix’ column presents the historical affix.



Stress patterns for disyllabic items are presented in Table  ‎2 .42. Most disyllabic items have penultimate stress (16/21 items – 76\%). The remaining five items (24\%) have ultimate stress. In prefixed items in which the prefix is reduced to a consonant and forms a CC cluster with the onset consonant of the lexical root, stress is assigned to the penultimate syllable of the derived lexical item, as in \textitbf{brangkat} /\textstyleChCharisSIL{ˈbraŋ.kat}/ ‘leave’ or \textitbf{spulu} /\textstyleChCharisSIL{ˈspu.lu}/ ‘ten’. In items with an unreduced prefix, stress remains on the lexical root and thereby on the ultimate syllable, as in \textitbf{bergrak} /\textstyleChCharisSIL{bɛr.ˈgrak}/ ‘move’.


\begin{stylecaption}
\label{bkm:Ref333851866}Table ‎2.\stepcounter{Table}{\theTable}:  Stress patterns for disyllabic affixed lexical items\footnote{\\
\\
\\
\\
\\
\\
\\
Note that the (historical) affixes have phonological allomorphs: /\textstyleChCharisSILviiivpt{ta-}/ and /\textstyleChCharisSILviiivpt{tɛr-}/, for example, are allomorphs of prefix \textscItalBold{ter-}, /\textstyleChCharisSILviiivpt{pl-}/ is an allomorph of prefix \textscItalBold{pe(n)-}, and /\textstyleChCharisSILviiivpt{br-}/ and /\textstyleChCharisSILviiivpt{ba-}/ are allomorphs of prefix \textscItalBold{ber-} (the small caps designate abstract representations of the affixes as they have more than one form of realization). (For details see §3.1.2.1, §3.1.4.1, and §3.1.5.1, respectively.)\\
\\
\\
\\
\\
\\
\\
\\
}
\end{stylecaption}

\tablehead{
 Stress & Item & Affix & Orthography & \arraybslash Gloss\\
}
\begin{tabular}{lllll}
\lsptoprule
\textsc{p-ult} & /\textstyleChCharisSIL{ˈbraŋ.kat}/ & /\textstyleChCharisSIL{br}–\textstyleChCharisSIL{\_\_}/ & \textitbf{brangkat} & ‘leave’\\
& /\textstyleChCharisSIL{ˈpla.dʒar}/ & /\textstyleChCharisSIL{pɛl}–\textstyleChCharisSIL{\_\_}/ & \textitbf{plajar} & ‘teacher’\\
& /\textstyleChCharisSIL{ˈspu.lu}/ & /\textstyleChCharisSIL{sɛ}–\textstyleChCharisSIL{\_\_}/ & \textitbf{spulu} & ‘ten’\\
& /\textstyleChCharisSIL{ˈgra.kaŋ}/ & /\textstyleChCharisSIL{\_\_}–\textstyleChCharisSIL{aŋ}/ & \textitbf{grakang} & ‘movement’\\
\textsc{ult} & /\textstyleChCharisSIL{bɛr.ˈgrak}/ & /\textstyleChCharisSIL{bɛr}–\textstyleChCharisSIL{\_\_}/ & \textitbf{bergrak} & ‘move’\\
& /\textstyleChCharisSIL{sɛ.ˈblas}/ & /\textstyleChCharisSIL{sɛ}–\textstyleChCharisSIL{\_\_}/ & \textitbf{seblas} & ‘eleven’\\
& /\textstyleChCharisSIL{ta.ˈbla}/ & /\textstyleChCharisSIL{ta}–\textstyleChCharisSIL{\_\_}/ & \textitbf{tabla} & ‘be cracked open’\\
\lspbottomrule
\end{tabular}

Stress patterns for trisyllabic lexical items are presented in Table  ‎2 .43. Almost all of them have penultimate stress (259/272 items – 95\%). That is, when a disyllabic lexical root is suffixed, the stress moves from the penultimate syllable of the root to its ultimate syllable, as in \textitbf{ikat} /\textstyleChCharisSIL{ˈi.kat}/ ‘tie up’ versus \textitbf{ikatang} /\textstyleChCharisSIL{i.ˈka.taŋ}/ ‘tie’. The remaining 13 items (5\%) have ultimate stress, with the antepenultimate syllable carrying secondary stress. The respective roots of the 13 items also carry ultimate stress, as in \textitbf{kebung} /\textstyleChCharisSIL{kɛ.ˈbuŋ}/ ‘garden’ versus \textitbf{berkebung} /\textstyleChCharisSIL{ˌbɛr.kɛ.ˈbuŋ}/ ‘do farming’.


\begin{stylecaption}
\label{bkm:Ref333851868}Table ‎2.\stepcounter{Table}{\theTable}:  Stress patterns for trisyllabic affixed lexical items
\end{stylecaption}

\tablehead{
 Stress & Item & Affix & Orthogr. & \arraybslash Gloss\\
}
\begin{tabular}{lllll}
\lsptoprule
\textsc{p-ult} & /\textstyleChCharisSIL{ba.ˈi.si}/ & /\textstyleChCharisSIL{ba}–\textstyleChCharisSIL{\_\_} & \textitbf{baisi} & ‘be muscular’\\
& /\textstyleChCharisSIL{pɛ.ˈmu.da}/ & /\textstyleChCharisSIL{pɛ}–\textstyleChCharisSIL{\_\_}/ & \textitbf{pemuda} & ‘young person’\\
& /\textstyleChCharisSIL{kɛ.ˈdu.a}/ & /\textstyleChCharisSIL{kɛ}–\textstyleChCharisSIL{\_\_}/ & \textitbf{kedua} & ‘second’\\
& /\textstyleChCharisSIL{ta.ˈgɔ.jaŋ}/ & /\textstyleChCharisSIL{ta}–\textstyleChCharisSIL{\_\_}/ & \textitbf{tagoyang} & ‘be shaken’\\
& /\textstyleChCharisSIL{sɛ.ˈti.ap}/ & /\textstyleChCharisSIL{sɛ}–\textstyleChCharisSIL{\_\_}/ & \textitbf{setiap} & ‘every’\\
& /\textstyleChCharisSIL{i.ˈka.taŋ}/ & /\textstyleChCharisSIL{\_\_}–\textstyleChCharisSIL{aŋ}/ & \textitbf{ikatang} & ‘tie’\\
& /\textstyleChCharisSIL{mi.ˈsal.ɲa}/ & /\textstyleChCharisSIL{\_\_}–\textstyleChCharisSIL{ɲa}/ & \textitbf{misalnya} & ‘for example’\\
\textsc{ult} & /\textstyleChCharisSIL{ˌbɛr.kɛ.ˈbuŋ}/ & /\textstyleChCharisSIL{bɛr}–\textstyleChCharisSIL{\_\_}/ & \textitbf{berkebung} & ‘do farming’\\
& /\textstyleChCharisSIL{ˌkɛ.ɛm.ˈpat}/ & /\textstyleChCharisSIL{kɛ}–\textstyleChCharisSIL{\_\_}/ & \textitbf{keempat} & ‘fourth’\\
& /\textstyleChCharisSIL{ˌmɛ.ɲɛ.ˈbraŋ}/ & /\textstyleChCharisSIL{mɛ}–\textstyleChCharisSIL{\_\_}/ & \textitbf{menyebrang} & ‘cross’\\
& /\textstyleChCharisSIL{ˌtɛr.lɛ.ˈpas}/ & /\textstyleChCharisSIL{tɛr}–\textstyleChCharisSIL{\_\_}/ & \textitbf{terlepas} & ‘be loose’\\
\lspbottomrule
\end{tabular}

Examples of derived lexical items with four syllables are presented in Table  ‎2 .44. All 88 items have penultimate stress, while secondary stress falls on the alternating syllable preceding the one carrying the primary stress. Again, when suffixed the stress moves to the ultimate syllable of the root, as in \textitbf{dalam} /\textstyleChCharisSIL{ˈda.lam}/ ‘inside’ versus \textitbf{pedalamang} /\textstyleChCharisSIL{ˌpɛ.da.ˈla.maŋ}/ ‘interior’.


\begin{stylecaption}
\label{bkm:Ref333851869}Table ‎2.\stepcounter{Table}{\theTable}:  Stress patterns for quadrisyllabic affixed lexical items
\end{stylecaption}

\tablehead{
 Stress & Item & Affix & Orthogr. & \arraybslash Gloss\\
}
\begin{tabular}{lllll}
\lsptoprule
\textsc{p-ult} & /\textstyleChCharisSIL{ˌpɛ.da.ˈla.maŋ}/ & /\textstyleChCharisSIL{pɛ}–\textstyleChCharisSIL{\_\_}–\textstyleChCharisSIL{aŋ}/ & \textitbf{pedalamang} & ‘interior’\\
& /\textstyleChCharisSIL{ˌkɛ.gi.ˈa.taŋ}/ & /\textstyleChCharisSIL{kɛ}–\textstyleChCharisSIL{\_\_}–\textstyleChCharisSIL{aŋ}/ & \textitbf{kegiatang} & ‘activity’\\
& /\textstyleChCharisSIL{ˌkɛn.da.ˈra.aŋ}/ & /\textstyleChCharisSIL{\_\_}–\textstyleChCharisSIL{aŋ}/ & \textitbf{kendaraang} & ‘vehicle’\\
& /\textstyleChCharisSIL{ˌsɛ.bɛ.ˈnar.ɲa}/ & /\textstyleChCharisSIL{sɛ}–\textstyleChCharisSIL{\_\_}–\textstyleChCharisSIL{ɲa}/ & \textitbf{sebenarnya} & ‘actually’\\
\lspbottomrule
\end{tabular}

The data presented in Table  ‎2 .42 to Table  ‎2 .44 show that the Papuan Malay preference for penultimate stress also applies to lexical items that are historically derived by (unproductive) affixation. The vast majority of the 380 items (362 – 95\%) have penultimate stress, as shown in Table  ‎2 .45. For suffixed items, this stress pattern implies a stress-shift from the penultimate syllable of the root to its ultimate syllable. Only a small number of items deviates from this basic stress pattern and displays ultimate stress (18/380 – 5\%). For 13 of the 18 items, their respective lexical roots also have ultimate stress, while another four have monosyllabic roots; the remaining item has non-compositional semantics (\textitbf{tagait} ‘be hooked’).\footnote{\\
\\
\\
\\
\\
\\
\\
The historical root \textitbf{gait} does not exist in Papuan Malay.\\
\\
\\
\\
\\
\\
\\
\\
}


\begin{stylecaption}
\label{bkm:Ref333829087}Table ‎2.\stepcounter{Table}{\theTable}:  Stress patterns for historically derived lexical items – Frequencies
\end{stylecaption}

\tablehead{
 Syllable types & \textsc{p-ult} stress & \textsc{ult} stress & \arraybslash Total\\
}
\begin{tabular}{llll}
\lsptoprule
Disyllabic & \raggedleft 16 & \raggedleft 5 & \raggedleft\arraybslash 21\\
Trisyllabic & \raggedleft 259 & \raggedleft 13 & \raggedleft\arraybslash 272\\
Quadrisyllabic & \raggedleft 87 & \raggedleft {}-{}-{}- & \raggedleft\arraybslash 87\\
Total & \raggedleft 362 & \raggedleft 18 & \raggedleft\arraybslash 380\\
\lspbottomrule
\end{tabular}
\section{Non-native segments and loan words}
\label{bkm:Ref338668244}
This section describes non-native segments and loan words attested in the Papuan Malay corpus. So far, 719 items of the 2,458-item word list (29\%) have been identified as loan words, originating from different donor languages, such as Arabic, Chinese, Dutch, English, Persian, Portuguese, or Sanskrit. Not included here are inherited Malay words which are typically used in Standard Indonesian but not in Papuan Malay, such as Indonesian \textitbf{desa} ‘village’ or \textitbf{mereka} ‘\textsc{3pl}’ (the corresponding Papuan Malay words are \textitbf{kampung} ‘village’ and \textitbf{dorang}/\textitbf{dong} ‘\textsc{3pl}’, respectively). (See also §1.11.6.)



The non-native segments are presented in §2.5.1, followed in §2.5.2 by a description of the phonological and phonetic processes that native and non-native segments can undergo in loan words. The phonotactics found in loan words are investigated in §2.5.3.
\end{styleBodyvxvafter}

\subsection{Non-native segments}
\label{bkm:Ref338525868}
In the investigated loan words, two consonantal segments occur that are not part of the Papuan Malay consonant inventory: the voiceless labio-dental fricative /\textstyleChCharisSIL{f}/ and the voiceless postalveolar fricative /\textstyleChCharisSIL{ʃ}/.
\end{styleBodyafterxivpt}


The voiceless labio-dental fricative /\textstyleChCharisSIL{f}/ is attested in 49 loan words. It occurs as word-initial and word-internal onset and as word-final coda, as illustrated in Table  ‎2 .46.


\begin{styleCaptionxivptSpace}
\label{bkm:Ref338322245}Table ‎2.\stepcounter{Table}{\theTable}:  Labio-dental fricative /\textstyleChCharisSIL{f}/
\end{styleCaptionxivptSpace}

\tablehead{
 Position & Item & Orthogr. & Gloss & \arraybslash Donor language\\
}
\begin{tabular}{lllll}
\lsptoprule
Word-initial onset & [\textstyleChCharisSIL{ˈ}\textstyleChCharisSILBlueBold{f}\textstyleChCharisSIL{a.dʒɐr̥}] & \textitbf{fajar} & ‘dawn’ & Arabic\\
& [\textstyleChCharisSIL{ˈ}\textstyleChCharisSILBlueBold{f}\textstyleChCharisSIL{ɔ.tɔ}] & \textitbf{foto} & ‘photo’ & Dutch\\
Word-initial onset & [\textstyleChCharisSIL{ˈsi.}\textstyleChCharisSILBlueBold{f}\textstyleChCharisSIL{ɐt̚}] & \textitbf{sifat} & ‘characteristic’ & Arabic\\
& [\textstyleChCharisSIL{ˈtɾɐns.}\textstyleChCharisSILBlueBold{f}\textstyleChCharisSIL{ɛ̞r}] & \textitbf{transfer} & ‘transfer’ & English\\
Word-final coda & [\textstyleChCharisSIL{ma.ˈɐ}\textstyleChCharisSILBlueBold{f}] & \textitbf{maaf} & ‘pardon’ & Arabic\\
& [\textstyleChCharisSIL{ɪn.ˈsɛ̞n.tɪ}\textstyleChCharisSILBlueBold{f}] & \textitbf{insentif} & ‘incentive’ & English\\
\lspbottomrule
\end{tabular}

The second non-native segment occurs in loan words of Arabic origins containing the voiceless postalveolar fricative /\textstyleChCharisSIL{ʃ}/. Standard Malay and Standard Indonesian have adopted the fricative into their consonant inventory, realizing it as /\textstyleChCharisSIL{ʃ}/ {\textless}\textitbf{sy}{\textgreater} as in \textitbf{syurga} ‘heaven’ {\citep[13]{Mintz2002}}.\footnote{\\
\\
\\
\\
\\
\\
\\
{\citet[13]{Mintz2002}} represents /\textstyleChCharisSILviiivpt{ʃ}/ as /\textstyleChCharisSILviiivpt{š}/ and defines it as “a palatal fricative”: \textitbf{syurga} /\textstyleChCharisSILviiivpt{šur.ga}/ ‘heaven’.\\
\\
\\
\\
\\
\\
\\
\\
} Papuan Malay, by contrast, has not adopted the postalveolar fricative. Instead, Papuan Malay speakers employ three different substitution strategies to realize the fricative in loan words of Arabic origins, some of which may have been borrowed into Papuan Malay via Standard Indonesian. The most common strategy is to replace /\textstyleChCharisSIL{ʃ}/ with the alveolar fricative [s]. Alternative strategies are to substitute /\textstyleChCharisSIL{ʃ}/ with the palatalized alveolar fricative [s\textsuperscript{j}], or with the consonant sequence [s.j]. In the same utterance or conversation, speakers may employ more than one strategy.
\end{styleBodyaftervbeforexivpt}


The three substitution strategies are illustrated in Table  ‎2 .47. The item \textitbf{masarakat} ‘community’, for example, is most commonly realized with the alveolar fricative [s]. The items \textitbf{syarat} ‘condition and \textitbf{syukur} ‘thanks to God’ are, instead, realized with the palatalized alveolar fricative [s\textsuperscript{j}]. Alternatively, speakers sometimes replace /\textstyleChCharisSIL{ʃ}/ with the consonant sequence [s.j], thereby changing the syllable pattern of the target item as in \textitbf{dasyat} [\textstyleChCharisSIL{ˈdɐs.jɐt̚}] ‘terrifying’.


\begin{stylecaption}
\label{bkm:Ref333064468}Table ‎2.\stepcounter{Table}{\theTable}:  Strategies to realize the Standard Indonesian postalveolar fricative
\end{stylecaption}

\tablehead{
 Orthogr. & Gloss & Realization & Freq. & \arraybslash Item in SI\\
}
\begin{tabular}{lllll}
\lsptoprule
\textitbf{masarakat} & ‘community’ & [\textstyleChCharisSIL{ˌma.}\textstyleChCharisSILBlueBold{s}\textstyleChCharisSIL{a.ˈɾa.kɐt̚}] & 27 & \textitbf{masyarakat}\\
&  & [\textstyleChCharisSIL{ˌma.}\textstyleChCharisSILBlueBold{sʲ}\textstyleChCharisSIL{a.ˈɾa.kɐt̚}] & 11 & \\
\textitbf{asik} & ‘be passionate’ & [\textstyleChCharisSIL{ˈa.}\textstyleChCharisSILBlueBold{s}\textstyleChCharisSIL{ɪk̚}] & 1 & \textitbf{asyik}\\
\textitbf{dasyat} & ‘terrifying’ & [\textstyleChCharisSIL{ˈda.}\textstyleChCharisSILBlueBold{sʲ}\textstyleChCharisSIL{ɐt̚}] & 2 & \textitbf{dasyat}\\
&  & [\textstyleChCharisSIL{ˈdɐ}\textstyleChCharisSILBlueBold{s.j}\textstyleChCharisSIL{ɐt̚}] & 4 & \\
\textitbf{syarat} & ‘condition’ & [\textstyleChCharisSIL{ˈ}\textstyleChCharisSILBlueBold{sʲ}\textstyleChCharisSIL{a.ɾɐt̚}] & 2 & \textitbf{syarat}\\
\textitbf{syukur} & ‘thanks to God’ & [\textstyleChCharisSILBlueBold{sʲ}\textstyleChCharisSIL{u.kʊr}] & 3 & \textitbf{syukur}\\
\lspbottomrule
\end{tabular}
\subsection{Phonological and phonetic processes in loan words}
\label{bkm:Ref338525870}
Overall, the same phonological and phonetic processes apply for loan words as for inherited Malay roots (see §2.2 and §2.3). Three processes, however, need to be discussed in more detail: the lack of nasal place assimilation (§2.5.2.1), lenition (§2.5.2.2), and palatalization of the alveolar fricative (§2.5.2.3).
\end{styleBodyxvafter}

\paragraph[Lack of nasal place assimilation]{Lack of nasal place assimilation}
\label{bkm:Ref338399473}
In loan words, a nasal in the word-internal coda position typically obtains its place features from the following segment in the same way as it does in inherited Malay roots (§2.2.1). When preceding the alveolar fricative, the nasal is typically realized as /\textstyleChCharisSIL{ŋ}/. Examples are \textitbf{jambu} ‘rose apple’, \textitbf{cinta} ‘love’, or \textitbf{bengkel} ‘repair shop’, and \textitbf{bangsa} ‘people group’ or \textitbf{fungsi} ‘function’.
\end{styleBodyafterxivpt}


In some loan words, however, the nasal does not undergo assimilation, as illustrated in Table  ‎2 .48. Instead, the bilabial or the alveolar nasal is followed by a consonant with different place features as in \textitbf{jumla} ‘sum’ or \textitbf{tanpa} ‘without’.
\end{styleBodyvvafter}

\begin{stylecaption}
\label{bkm:Ref338328987}Table ‎2.\stepcounter{Table}{\theTable}:  Lack of nasal place assimilation in the word-internal coda in loan words
\end{stylecaption}

\tablehead{
 Realization & Item & Orthogr. & Gloss & \arraybslash Donor language\\
}
\begin{tabular}{lllll}
\lsptoprule
[m] & [\textstyleChCharisSIL{aˈlʊ}\textstyleChCharisSILBlueBold{m.n}\textstyleChCharisSIL{i}] & \textitbf{alumni} & ‘alumnus’ & Latin\\
& [\textstyleChCharisSIL{ˈdzʊ}\textstyleChCharisSILBlueBold{m.l}\textstyleChCharisSIL{a}] & \textitbf{jumla} & ‘sum’ & Arabic\\
& [\textstyleChCharisSIL{kɔ̞nˈsʊ}\textstyleChCharisSILBlueBold{m.s}\textstyleChCharisSIL{i}] & \textitbf{konsumsi} & ‘consumption’ & Dutch\\
[n] & [\textstyleChCharisSIL{ˈta}\textstyleChCharisSILBlueBold{n.p}\textstyleChCharisSIL{a}] & \textitbf{tanpa} & ‘without’ & (uncertain\footnotemark{})\\
& [\textstyleChCharisSIL{mɐ}\textstyleChCharisSILBlueBold{n.ˈf}\textstyleChCharisSIL{a.ɐt̚}] & \textitbf{manfaat} & ‘benefit’ & Arabic\\
& [\textstyleChCharisSIL{ˌɪ}\textstyleChCharisSILBlueBold{n.f}\textstyleChCharisSIL{ɔ̞r.ˈma.si}] & \textitbf{informasi} & ‘information’ & Dutch\\
\lspbottomrule
\end{tabular}
\footnotetext{\\
\\
\\
\\
\\
\\
\\
In {\citet{Jones2007}}, \textitbf{tanpa} ‘without’ is not listed as a loan words. {\citet{Tadmor2009a}}, however, classifies the item as “clearly borrowed”, listing Sudanese, Balinese, and Javanese as “uncertain” donor languages.\\
\\
\\
\\
\\
\\
\\
\\
}
\paragraph[Lenition]{Lenition}
\label{bkm:Ref438304390}
Lenition is attested only for the bilabial voiceless stop in two lexical items, namely \textitbf{kopi} ‘coffee’ and \textitbf{pikir} ‘think’. Inter-vocalically, the bilabial stop in \textitbf{kopi} [\textstyleChCharisSIL{ˈkɔ.pi}] ‘coffee’ can be lenited by means of spirantization to fricative [\textstyleChCharisSIL{f}] giving [\textstyleChCharisSIL{ˈkɔ.fi}] ‘coffee’. When following a lexeme with word-final vowel, the word-initial stop in \textitbf{pikir} [\textstyleChCharisSIL{ˈpi.kɪr}] ‘think’ can be lenited to [\textstyleChCharisSIL{f}], as in [\textstyleChCharisSIL{ˈsa ˈfi.kɪr}] \textitbf{sa pikir} ‘I think’ or [\textstyleChCharisSIL{ˈsu.da ˈfi.kɪr}] \textitbf{suda pikir} ‘already thought’.\footnote{\\
\\
\\
\\
\\
\\
\\
Notably, for both loan words, the source forms contain fricative /\textstyleChCharisSILviiivpt{f}/ rather than stop /\textstyleChCharisSILviiivpt{p}/: the source form for \textitbf{kopi} ‘coffee’ is Dutch \textitbf{koffie} and the source form for \textitbf{pikir} ‘think’ is Arabic \textitbf{fikr}.\\
\\
\\
\\
\\
\\
\\
\\
}


\paragraph[Palatalization of the alveolar fricative]{Palatalization of the alveolar fricative}
\label{bkm:Ref338492499}
Palatalization of the alveolar fricative /s/ occurs in loan words in an environment identical to that found in inherited Malay roots (§2.3.1.4). That is, palatalization of alveolar /s/ occurs in loan words with a /\textstyleChCharisSIL{si.}V/ sequence, if the lexical item consists of three of more syllables and if the syllable containing /s/ is unstressed. Attested are three loan words with /\textstyleChCharisSIL{si.ɔ}/ or /\textstyleChCharisSIL{si.a}/ sequences, presented in Table  ‎2 .49. Again, the palatalization of /\textstyleChCharisSIL{s}/ co-occurs with the elision of close front /\textstyleChCharisSIL{i}/, which reduces the number of syllables by one. Hence, /\textstyleChCharisSIL{si.ɔ}/ is realized as [\textstyleChCharisSIL{sʲɔ}] and /\textstyleChCharisSIL{si.a}/ as [\textstyleChCharisSIL{sʲa}]. In loan words with a /\textstyleChCharisSIL{si.a}/ sequence in which the syllable containing /s/ is stressed, /\textstyleChCharisSIL{s}/ is not palatalized, as in \textitbf{manusia} ‘human being’.\footnote{\\
\\
\\
\\
\\
\\
\\
Loan words with a /\textstyleChCharisSILviiivpt{si.ɔ}/ sequence in which the syllable containing /s/ is stressed are unattested.\\
\\
\\
\\
\\
\\
\\
\\
}


\begin{stylecaption}
\label{bkm:Ref332976267}Table ‎2.\stepcounter{Table}{\theTable}:  Palatalization of the alveolar fricative in loan words
\end{stylecaption}

\tablehead{
 Stress & Orthogr. & Gloss & Realization & \arraybslash Freq.\\
}
\begin{tabular}{lllll}
\lsptoprule
/\textstyleChCharisSIL{si}/ unstressed & \textitbf{misionaris} & ‘missionary’ & [\textstyleChCharisSIL{ˌmi.}\textstyleChCharisSILBlueBold{si.ɔ}\textstyleChCharisSIL{.ˈna.rɹs}] & \raggedleft\arraybslash 1\\
&  &  & [\textstyleChCharisSIL{ˌmi.}\textstyleChCharisSILBlueBold{sʲɔ}\textstyleChCharisSIL{.ˈna.rɹs}] & \raggedleft\arraybslash 10\\
& \textitbf{nasional} & ‘national’ & [\textstyleChCharisSIL{ˌna.}\textstyleChCharisSILBlueBold{si.ɔ}\textstyleChCharisSIL{.ˈnɐl}] & \raggedleft\arraybslash 1\\
&  &  & [\textstyleChCharisSIL{ˌna.}\textstyleChCharisSILBlueBold{sʲɔ.}\textstyleChCharisSIL{ˈnɐl}] & \raggedleft\arraybslash 2\\
& \textitbf{sosial} & ‘social’ & [\textstyleChCharisSIL{ˌsɔ.}\textstyleChCharisSILBlueBold{sɪ.ˈɐ}\textstyleChCharisSIL{l}] & \raggedleft\arraybslash 2\\
&  &  & [\textstyleChCharisSIL{sɔ.ˈ}\textstyleChCharisSILBlueBold{sʲɐ}\textstyleChCharisSIL{l}] & \raggedleft\arraybslash 3\\
/\textstyleChCharisSIL{ˈsi}/ stressed & \textitbf{manusia} & ‘human being’ & [\textstyleChCharisSIL{ˌma.nu.ˈ}\textstyleChCharisSILBlueBold{si.a}] & \raggedleft\arraybslash 49\\
& \textitbf{rahasia} & ‘secret’ & [\textstyleChCharisSIL{ˌra.ha.ˈ}\textstyleChCharisSILBlueBold{sɪ.a}] & \raggedleft\arraybslash 4\\
& \textitbf{usia} & ‘age’ & [\textstyleChCharisSIL{u.ˈ}\textstyleChCharisSILBlueBold{si.a}] & \raggedleft\arraybslash 5\\
\lspbottomrule
\end{tabular}
\subsection{Phonotactics in loan words}
\label{bkm:Ref338525872}
This section describes the phonotactics found in loan words: the consonant distribution and sequences are described in §2.5.3.1, the vowel distribution and sequences in §2.5.3.2, and the syllable structures and stress patterns in §2.5.3.3.
\end{styleBodyxvafter}

\paragraph[Consonant distribution and sequences]{Consonant distribution and sequences}
\label{bkm:Ref338523904}
The distribution of consonants in loan words corresponds to their distribution in inherited Malay roots (see §2.4.1). This also applies to the loan fricative /\textstyleChCharisSIL{f}/, which has the same distribution as the alveolar fricative /s/ and occurs in all positions.
\end{styleBodyafterxivpt}


In loan words a restricted sample of consonants can occur in consonant clusters, as illustrated in Table  ‎2 .50 to Table  ‎2 .52. The range of consonants occurring in word-initial consonant clusters is considerably larger than the range of consonants occurring in word-internal clusters, similar to their distribution in inherited Malay roots.
\end{styleBodyvvafter}

\begin{stylecaption}
\label{bkm:Ref338424457}Table ‎2.\stepcounter{Table}{\theTable}:  Onset CC clusters: Stops in C\textsubscript{1} position
\end{stylecaption}

\tablehead{
\multicolumn{3}{l}{ Word-initial position} & \multicolumn{3}{l}{ Word-internal position}\\
}
\begin{tabular}{llllll}
\lsptoprule
/\textstyleChCharisSIL{p}C\textsubscript{2}/ & /\textstyleChCharisSIL{ˈ}\textstyleChCharisSILBlueBold{pr}\textstyleChCharisSIL{ak.tɛk}/ & ‘practicum’ & /\textstyleChCharisSIL{p}C\textsubscript{2}/ & /\textstyleChCharisSIL{ɔ.ˈ}\textstyleChCharisSILBlueBold{pr}\textstyleChCharisSIL{a.si}/ & ‘operation’\\
& /\textstyleChCharisSIL{ˈ}\textstyleChCharisSILBlueBold{pl}\textstyleChCharisSIL{as.tik}/ & ‘plastic’ &  & /\textstyleChCharisSIL{ˈam.}\textstyleChCharisSILBlueBold{pl}\textstyleChCharisSIL{ɔp}/ & ‘envelop’\\
/\textstyleChCharisSIL{b}C\textsubscript{2}/ & /\textstyleChCharisSIL{ˈ}\textstyleChCharisSILBlueBold{br}\textstyleChCharisSIL{i.ta}/ & ‘news’ & /\textstyleChCharisSIL{b}C\textsubscript{2}/ & /\textstyleChCharisSIL{ˈdɔ.}\textstyleChCharisSILBlueBold{br}\textstyleChCharisSIL{ak}/ & ‘smash’\\
&  &  &  & /\textstyleChCharisSIL{ˈi.}\textstyleChCharisSILBlueBold{bl}\textstyleChCharisSIL{is}/ & ‘devil’\\
/\textstyleChCharisSIL{t}C\textsubscript{2}/ & /\textstyleChCharisSILBlueBold{tr}\textstyleChCharisSIL{a.ˈdi.si}/ & ‘tradition’ & /\textstyleChCharisSIL{t}C\textsubscript{2}/ & /\textstyleChCharisSIL{ba.ˈ}\textstyleChCharisSILBlueBold{tr}\textstyleChCharisSIL{ɛj}/ & ‘battery\\
/\textstyleChCharisSIL{d}C\textsubscript{2}/ & /\textstyleChCharisSIL{ˈ}\textstyleChCharisSILBlueBold{dr}\textstyleChCharisSIL{am.bɛn}/ & ‘marching band’ &  &  & \\
/\textstyleChCharisSIL{k}C\textsubscript{2}/ & /\textstyleChCharisSIL{ˈ}\textstyleChCharisSILBlueBold{kn}\textstyleChCharisSIL{al.pɔt}/ & ‘muffler’ & /\textstyleChCharisSIL{k}C\textsubscript{2}/ & /\textstyleChCharisSIL{ˌrɛ.}\textstyleChCharisSILBlueBold{kr}\textstyleChCharisSIL{ɛ.ˈa.si}/ & ‘recreation’\\
& /\textstyleChCharisSILBlueBold{kr}\textstyleChCharisSIL{ɛ.ˈma.si}/ & ‘cremation’ &  & /\textstyleChCharisSIL{bis.ˈ}\textstyleChCharisSILBlueBold{kw}\textstyleChCharisSIL{it}/ & ‘cracker’\\
& /\textstyleChCharisSIL{ˈ}\textstyleChCharisSILBlueBold{kl}\textstyleChCharisSIL{as}/ & ‘class’ &  &  & \\
& /\textstyleChCharisSIL{ˈ}\textstyleChCharisSILBlueBold{kw}\textstyleChCharisSIL{a}/ & ‘broth’ &  &  & \\
/\textstyleChCharisSIL{g}C\textsubscript{2}/ & /\textstyleChCharisSIL{ˈ}\textstyleChCharisSILBlueBold{gr}\textstyleChCharisSIL{ɔ.bak}/ & ‘wheelbarrow’ & /\textstyleChCharisSIL{g}C\textsubscript{2}/ & /\textstyleChCharisSIL{nɛ.ˈ}\textstyleChCharisSILBlueBold{gr}\textstyleChCharisSIL{i}/ & ‘state’\\
& /\textstyleChCharisSIL{ˈ}\textstyleChCharisSILBlueBold{gl}\textstyleChCharisSIL{ɔ.dʒɔ}/ & ‘be greedy’ &  &  & \\
\lspbottomrule
\end{tabular}
\begin{styleCaptionxbefore}
Table ‎2.\stepcounter{Table}{\theTable}:  Onset CC and CCC clusters: Fricatives in C\textsubscript{1} position
\end{styleCaptionxbefore}

\begin{tabular}{llllll}
\lsptoprule

\multicolumn{3}{l}{ Word-initial position} & \multicolumn{3}{l}{ Word-internal position}\\
/\textstyleChCharisSIL{f}C\textsubscript{2}/ & /\textstyleChCharisSIL{ˈ}\textstyleChCharisSILBlueBold{fr}\textstyleChCharisSIL{ɛj}/ & ‘be blank’ &  &  & \\
/\textstyleChCharisSIL{s}C\textsubscript{2}/ & /\textstyleChCharisSIL{ˈ}\textstyleChCharisSILBlueBold{sp}\textstyleChCharisSIL{a.tu}/ & ‘shoe’ & /\textstyleChCharisSIL{s}C\textsubscript{2}/ & /\textstyleChCharisSIL{in.ˈ}\textstyleChCharisSILBlueBold{st}\textstyleChCharisSIL{an.si}/ & ‘level’\\
& /\textstyleChCharisSIL{ˈ}\textstyleChCharisSILBlueBold{st}\textstyleChCharisSIL{a.tus}/ & ‘status’ &  &  & \\
& /\textstyleChCharisSIL{ˈ}\textstyleChCharisSILBlueBold{sk}\textstyleChCharisSIL{ɔ.la}/ & ‘school’ &  &  & \\
& /\textstyleChCharisSIL{ˈ}\textstyleChCharisSILBlueBold{sm}\textstyleChCharisSIL{ɛn}/ & ‘cement’ &  &  & \\
& /\textstyleChCharisSIL{ˈ}\textstyleChCharisSILBlueBold{sn}\textstyleChCharisSIL{ɛk}/ & ‘snack’ &  &  & \\
& /\textstyleChCharisSIL{ˈ}\textstyleChCharisSILBlueBold{sl}\textstyleChCharisSIL{a.mat}/ & ‘be safe’ &  &  & \\
& /\textstyleChCharisSIL{ˈ}\textstyleChCharisSILBlueBold{sw}\textstyleChCharisSIL{a.mi}/ & ‘husband’ &  &  & \\
& /\textstyleChCharisSIL{ˈ}\textstyleChCharisSILBlueBold{spr}\textstyleChCharisSIL{ɛj}/ & ‘bedsheet’ &  &  & \\
& /\textstyleChCharisSIL{ˈ}\textstyleChCharisSILBlueBold{str}\textstyleChCharisSIL{ap}/ & ‘punish’ &  &  & \\
& /\textstyleChCharisSIL{ˈ}\textstyleChCharisSILBlueBold{skr}\textstyleChCharisSIL{ip.si}/ & ‘minithesis’ &  &  & \\
\lspbottomrule
\end{tabular}
\begin{styleCaptionxbefore}
\label{bkm:Ref386121408}Table ‎2.\stepcounter{Table}{\theTable}:  Coda CC clusters
\end{styleCaptionxbefore}

\begin{tabular}{lll}
\lsptoprule

\multicolumn{3}{l}{ Word-final position}\\
/\textstyleChCharisSIL{rt}/ & /\textstyleChCharisSIL{ˈɛr.pɔ}\textstyleChCharisSILBlueBold{rt}/ & ‘airport’\\
/\textstyleChCharisSIL{ks}/ & /\textstyleChCharisSIL{ˈkɔm.plɛ}\textstyleChCharisSILBlueBold{ks}/ & ‘complex’\\
\lspbottomrule
\end{tabular}

The data presented in Table  ‎2 .50 to Table  ‎2 .52 shows considerable similarities between loan words and inherited Malay roots in terms of the distribution of consonants in CC clusters (see Table  ‎2 .26). There are, however, also some differences. A number of CC clusters that are found in inherited Malay roots are unattested in loan words: /\textstyleChCharisSIL{tl}/, /\textstyleChCharisSIL{dl}/, /\textstyleChCharisSIL{tʃr}/, /\textstyleChCharisSIL{gn}/, and /\textstyleChCharisSIL{sr}/. By contrast, the following onset CC attested in loan words are unattested in inherited Malay roots: /\textstyleChCharisSIL{gr}/, /\textstyleChCharisSIL{fr}/, /\textstyleChCharisSIL{st}/. In addition, two word-final CC clusters are found in loan words, /\textstyleChCharisSIL{rt}/ and /\textstyleChCharisSIL{ks}/.\footnote{\\
\\
\\
\\
\\
\\
\\
Four loan words are attested with word-final CC cluster: \textitbf{erport} ‘airport’, \textitbf{kompleks} ‘complex’, \textitbf{petromaks} ‘kerosene lantern’ and \textitbf{raport} ‘school report book’. Rather commonly, however, these items are realized without the word-final CC cluster, as in [\textstyleChCharisSILviiivpt{ˈɛ̞r.pɔ̞r}] ‘airport’, [\textstyleChCharisSILviiivpt{ˈkɔ̞m.plɛ̞k}] ‘complex’, or [\textstyleChCharisSILviiivpt{ˌpɛ.tɾɔ.ˈmɐs}] ‘kerosene lantern’.\\
\\
\\
\\
\\
\\
\\
\\
} Finally, three onset CCC clusters are attested: /\textstyleChCharisSIL{spr}/, /\textstyleChCharisSIL{str}/, and /\textstyleChCharisSIL{skr}/.
\end{styleBodyaftervbeforexivpt}


Table  ‎2 .53 presents an overview of the attested consonant clusters. For the most part, the consonant clusters attested in loan words agree with {Kenstowicz’s (1994: 254)} Sonority Sequencing Principle (see §2.4.1).
\end{styleBodyvafter}


Almost all clusters listed in Table  ‎2 .53 occur in word-initial position. The exception is /\textstyleChCharisSIL{bl}/ which occurs only as word-internal onset. Those clusters that are attested as word-initial and word-internal onset are underlined; /\textstyleChCharisSIL{bl}/ is also underlined. The two CC clusters in word-final coda position are double-underlined.


\begin{stylecaption}
\label{bkm:Ref338425181}Table ‎2.\stepcounter{Table}{\theTable}:  CC and CCC clusters – Overview\footnote{\\
\\
\\
\\
\\
\\
\\
As nasals and approximants do not occur in C\textsubscript{1} position, they are excluded from Table  ‎2 .53.\\
\\
\\
\\
\\
\\
\\
\\
}
\end{stylecaption}

\begin{tabular}{lllllllllllllllllllll}
\lsptoprule

\multicolumn{2}{l}{ C\textsubscript{1}C\textsubscript{2}} & \multicolumn{11}{l}{ \textsc{obstr}} & \multicolumn{4}{l}{ \textsc{nas}} & \multicolumn{2}{l}{ \textsc{liq}} & \multicolumn{2}{l}{ \textsc{apr}}\\
\multicolumn{2}{l}{} & \textstyleChCharisSIL{p} & \textstyleChCharisSIL{b} & \textstyleChCharisSIL{t} & \textstyleChCharisSIL{d} & \textstyleChCharisSIL{tʃ} & \textstyleChCharisSIL{dʒ} & \textstyleChCharisSIL{k} & \textstyleChCharisSIL{g} & \textstyleChCharisSIL{f} & \textstyleChCharisSIL{s} & \textstyleChCharisSIL{h} & \textstyleChCharisSIL{m} & \textstyleChCharisSIL{n} & \textstyleChCharisSIL{ɲ} & \textstyleChCharisSIL{ŋ} & \textstyleChCharisSIL{r} & \textstyleChCharisSIL{l} & \textstyleChCharisSIL{j} & \arraybslash \textstyleChCharisSIL{w}\\
 \textsc{obstr} & \textstyleChCharisSIL{p} &  &  &  &  &  &  &  &  &  &  &  &  &  &  &  & \textstyleChCharisSILUnderl{pr} & \textstyleChCharisSILUnderl{pl} &  & \\
& \textstyleChCharisSIL{b} &  &  &  &  &  &  &  &  &  &  &  &  &  &  &  & \textstyleChCharisSILUnderl{br} & \textstyleChCharisSILUnderl{bl} &  & \\
& \textstyleChCharisSIL{t} &  &  &  &  &  &  &  &  &  &  &  &  &  &  &  & \textstyleChCharisSILUnderl{tr} &  &  & \\
& \textstyleChCharisSIL{d} &  &  &  &  &  &  &  &  &  &  &  &  &  &  &  & \textstyleChCharisSIL{dr} &  &  & \\
& \textstyleChCharisSIL{tʃ} &  &  &  &  &  &  &  &  &  &  &  &  &  &  &  &  &  &  & \\
& \textstyleChCharisSIL{dʒ} &  &  &  &  &  &  &  &  &  &  &  &  &  &  &  &  &  &  & \\
& \textstyleChCharisSIL{k} &  &  &  &  &  &  &  &  &  & \textstyleChCharisSIL{ks} &  &  & \textstyleChCharisSIL{kn} &  &  & \textstyleChCharisSILUnderl{kr} & \textstyleChCharisSIL{kl} &  & \arraybslash \textstyleChCharisSILUnderl{kw}\\
& \textstyleChCharisSIL{g} &  &  &  &  &  &  &  &  &  &  &  &  &  &  &  & \textstyleChCharisSILUnderl{gr} & \textstyleChCharisSIL{gl} &  & \\
& \textstyleChCharisSIL{f} &  &  &  &  &  &  &  &  &  &  &  &  &  &  &  & \textstyleChCharisSIL{fr} &  &  & \\
& \textstyleChCharisSIL{s} & \textstyleChCharisSIL{sp} &  & \textstyleChCharisSIL{st} &  &  &  & \textstyleChCharisSIL{sk} &  &  &  &  & \textstyleChCharisSIL{sm} & \textstyleChCharisSIL{sn} &  &  & \textstyleChCharisSIL{spr} & \textstyleChCharisSIL{sl} &  & \arraybslash \textstyleChCharisSIL{sw}\\
&  &  &  &  &  &  &  &  &  &  &  &  &  &  &  &  & \textstyleChCharisSIL{str} &  &  & \\
&  &  &  &  &  &  &  &  &  &  &  &  &  &  &  &  & \textstyleChCharisSIL{skr} &  &  & \\
& \textstyleChCharisSIL{h} &  &  &  &  &  &  &  &  &  &  &  &  &  &  &  &  &  &  & \\
 \textsc{liq} & \textstyleChCharisSIL{r} &  &  & \textstyleChCharisSIL{rt} &  &  &  &  &  &  &  &  &  &  &  &  &  &  &  & \\
\hhline{-~~~~~~~~~~~~~~~~~~~~} & \textstyleChCharisSIL{l} &  &  &  &  &  &  &  &  &  &  &  &  &  &  &  &  &  &  & \\
\hhline{~--------------------}
\lspbottomrule
\end{tabular}
\paragraph[Vowel distribution and sequences]{Vowel distribution and sequences}
\label{bkm:Ref338523906}
The distribution of vowels in loan words corresponds to that in inherited Malay roots (see §2.4.2). A restricted sample of vowels occurs in V.V vowel sequences, as shown in Table  ‎2 .54. Again, for each V.V sequence two examples are given, as far as attested. The first example displays a /\textstyleChCharisSIL{ˈ}(C)V.V/ stress pattern with the syllable containing V\textsubscript{1} being stressed. The second example has a /CV.\textstyleChCharisSIL{ˈ}V/ stress pattern in which V\textsubscript{2} is stressed. Of the 56 loan words with V.V sequences, 36 items (56\%) have a /\textstyleChCharisSIL{ˈ}CV.V/ stress pattern, while 20 items (44\%) show a /CV.\textstyleChCharisSIL{ˈ}V/ stress pattern. Again, the V.V sequences are realized without an inserted glottal stop.


\begin{stylecaption}
\label{bkm:Ref376253202}Table ‎2.\stepcounter{Table}{\theTable}:  V.V sequences – Examples
\end{stylecaption}

\tablehead{
 \textsubscript{V1}.V\textsubscript{2} & Stress & Item & Gloss & \arraybslash Freq.\\
}
\begin{tabular}{lllll}
\lsptoprule
/\textstyleChCharisSIL{i.u}/ & /C\textstyleChCharisSIL{i.ˈu}/ & /\textstyleChCharisSIL{ˌsɛ.r}\textstyleChCharisSILUnderl{i.ˈu}\textstyleChCharisSIL{s}/ & ‘be serious’ & \raggedleft\arraybslash 1\\
/\textstyleChCharisSIL{i.ɔ}/ & /\textstyleChCharisSIL{ˈ}C\textstyleChCharisSIL{i.ɔ}/ & /\textstyleChCharisSIL{ˈk}\textstyleChCharisSIL{i.ɔ}\textstyleChCharisSIL{s}/ & ‘kiosk’ & \raggedleft\arraybslash 6\\
& /C\textstyleChCharisSIL{i.ˈɔ}/ & /\textstyleChCharisSIL{pr}\textstyleChCharisSIL{i.ˈɔ}\textstyleChCharisSIL{.dɛ}/ & ‘period’ & \raggedleft\arraybslash 2\\
/\textstyleChCharisSIL{i.a}/ & /\textstyleChCharisSIL{ˈ}C\textstyleChCharisSIL{i.a}/ & /\textstyleChCharisSIL{tʃɛ.ˈr}\textstyleChCharisSILUnderl{i.a}/ & ‘be cheerful’ & \raggedleft\arraybslash \textstyleChBold{\textmd{15}}\\
& /C\textstyleChCharisSIL{i.ˈa}/ & /\textstyleChCharisSIL{p}\textstyleChCharisSILUnderl{i.ˈa}\textstyleChCharisSIL{.ra}/ & ‘raise’ & \raggedleft\arraybslash \textstyleChBold{\textmd{8}}\\
/\textstyleChCharisSIL{u.a}/ & /\textstyleChCharisSIL{ˈ}C\textstyleChCharisSIL{u.a}/ & /\textstyleChCharisSIL{ˈsm}\textstyleChCharisSILUnderl{u.a}/ & ‘all’ & \raggedleft\arraybslash \textstyleChBold{\textmd{1}}\\
& /C\textstyleChCharisSIL{u.ˈa}/ & /\textstyleChCharisSIL{p}\textstyleChCharisSILUnderl{u.ˈa}\textstyleChCharisSIL{.sa}/ & ‘fast’ & \raggedleft\arraybslash \textstyleChBold{\textmd{4}}\\
/\textstyleChCharisSIL{ɛ.ɔ}/ & /\textstyleChCharisSIL{ˈ}C\textstyleChCharisSIL{ɛ.ɔ}/ & /\textstyleChCharisSIL{fi.ˈd}\textstyleChCharisSIL{ɛ.ɔ}/ & ‘video’ & \raggedleft\arraybslash \textstyleChBold{\textmd{2}}\\
/\textstyleChCharisSIL{ɛ.a}/ & /C\textstyleChCharisSIL{ɛ.ˈa}/ & /\textstyleChCharisSIL{r}\textstyleChCharisSIL{ɛ.ˈa}\textstyleChCharisSIL{k.si}/ & ‘reaction’ & \raggedleft\arraybslash \textstyleChBold{\textmd{2}}\\
/\textstyleChCharisSIL{ɔ.i}/ & /\textstyleChCharisSIL{ˈ}C\textstyleChCharisSIL{ɔ.i}/ & /\textstyleChCharisSIL{ɛ.ˈg}\textstyleChCharisSIL{ɔ.i}\textstyleChCharisSIL{s}/ & ‘be egoistic’ & \raggedleft\arraybslash \textstyleChBold{\textmd{1}}\\
/\textstyleChCharisSIL{ɔ.a}/ & /\textstyleChCharisSIL{ˈ}C\textstyleChCharisSIL{ɔ.a}/ & /\textstyleChCharisSIL{ˈs}\textstyleChCharisSIL{ɔ.a}\textstyleChCharisSIL{k}/ & ‘be weak’ & \raggedleft\arraybslash \textstyleChBold{\textmd{5}}\\
& /C\textstyleChCharisSIL{ɔ.ˈa}/ & /\textstyleChCharisSIL{ˌɔn.d}\textstyleChCharisSIL{ɔ.ˈa}\textstyleChCharisSIL{.fi}/ & ‘traditional chief’ & \raggedleft\arraybslash \textstyleChBold{\textmd{1}}\\
/\textstyleChCharisSIL{a.i}/ & /\textstyleChCharisSIL{ˈ}C\textstyleChCharisSIL{a.i}/ & /\textstyleChCharisSIL{a.ˈdʒ}\textstyleChCharisSILUnderl{a.i}\textstyleChCharisSIL{p}/ & ‘be miraculous’ & \raggedleft\arraybslash \textstyleChBold{\textmd{2}}\\
/\textstyleChCharisSIL{a.u}/ & /\textstyleChCharisSIL{ˈ}C\textstyleChCharisSIL{a.u}/ & /\textstyleChCharisSIL{ˈm}\textstyleChCharisSILUnderl{a.u}\textstyleChCharisSIL{t}/ & ‘death’ & \raggedleft\arraybslash 1\\
/\textstyleChCharisSIL{a.ɛ}/ & /C\textstyleChCharisSIL{a.ˈɛ}/ & /\textstyleChCharisSIL{d}\textstyleChCharisSIL{a.ˈɛ}\textstyleChCharisSIL{.ɾa}/ & ‘area’ & \raggedleft\arraybslash 1\\
/\textstyleChCharisSIL{a.a}/ & /\textstyleChCharisSIL{ˈ}C\textstyleChCharisSIL{a.a}/ & /\textstyleChCharisSIL{dʒɛ.ˈm}\textstyleChCharisSIL{a.a}\textstyleChCharisSIL{t}/ & ‘congregation’ & \raggedleft\arraybslash 3\\
& /C\textstyleChCharisSIL{a.ˈa}/ & /\textstyleChCharisSIL{m}\textstyleChCharisSILUnderl{a.ˈa}\textstyleChCharisSIL{f}/ & ‘pardon’ & \raggedleft\arraybslash \textstyleChBold{\textmd{1}}\\
\lspbottomrule
\end{tabular}

The attested V.V sequences and their frequencies are summarized in Table  ‎2 .55. V.V sequences that are attested only once are underlined. Similar to inherited Malay roots, the V\textsubscript{1} position is most often occupied by a close vowel (37/56 items – 66\%). Open-mid and open vowels, however, are also quite common in this position (19/56 items – 34\%). The V\textsubscript{2} position is again most often taken by the open central vowel (40/56 lexical roots – 71\%), although close and open-mid vowels are also permitted in this position (16/56 lexical roots – 29\%).


\begin{stylecaption}
\label{bkm:Ref333398193}Table ‎2.\stepcounter{Table}{\theTable}:  V.V sequences and frequencies – Overview
\end{stylecaption}

\begin{tabular}{llllllllllll}
\lsptoprule

 \textsubscript{V1}.V\textsubscript{2} & \multicolumn{2}{l}{ \textstyleChCharisSIL{i}} & \multicolumn{2}{l}{ \textstyleChCharisSIL{u}} & \multicolumn{2}{l}{ \textstyleChCharisSIL{ɛ}} & \multicolumn{2}{l}{ \textstyleChCharisSIL{ɔ}} & \multicolumn{2}{l}{ \textstyleChCharisSIL{a}} & \arraybslash Total\\
 i & \textstyleChCharisSIL{{}-{}-{}-} & \raggedleft 0 & \textstyleChCharisSILUnderl{i.u} & \raggedleft 1 & \textstyleChCharisSIL{{}-{}-{}-} & \raggedleft 0 & \textstyleChCharisSIL{i.ɔ} & \raggedleft 8 & \textstyleChCharisSIL{i.a} & \raggedleft 23 & \raggedleft\arraybslash 32\\
 u & \textstyleChCharisSIL{{}-{}-{}-} & \raggedleft 0 & \textstyleChCharisSIL{{}-{}-{}-} & \raggedleft 0 & \textstyleChCharisSIL{{}-{}-{}-} & \raggedleft 0 & \textstyleChCharisSIL{{}-{}-{}-} & \raggedleft 0 & \textstyleChCharisSIL{u.a} & \raggedleft 5 & \raggedleft\arraybslash 5\\
 ɛ & \textstyleChCharisSIL{{}-{}-{}-} & \raggedleft 0 & \textstyleChCharisSIL{{}-{}-{}-} & \raggedleft 0 & \textstyleChCharisSIL{{}-{}-{}-} & \raggedleft 0 & \textstyleChCharisSIL{ɛ.ɔ} & \raggedleft 2 & \textstyleChCharisSIL{ɛ.a} & \raggedleft 2 & \raggedleft\arraybslash 4\\
 ɔ & \textstyleChCharisSILUnderl{ɔ.i} & \raggedleft 1 & \textstyleChCharisSIL{{}-{}-{}-} & \raggedleft 0 & \textstyleChCharisSIL{{}-{}-{}-} & \raggedleft 0 & \textstyleChCharisSIL{{}-{}-{}-} & \raggedleft 0 & \textstyleChCharisSIL{ɔ.a} & \raggedleft 6 & \raggedleft\arraybslash 7\\
 a & \textstyleChCharisSIL{a.i} & \raggedleft 2 & \textstyleChCharisSILUnderl{a.u} & \raggedleft 1 & \textstyleChCharisSILUnderl{a.ɛ} & \raggedleft 1 & \textstyleChCharisSIL{{}-{}-{}-} & \raggedleft 0 & \textstyleChCharisSIL{a.a} & \raggedleft 4 & \raggedleft\arraybslash 8\\
Total &  & \raggedleft 3 &  & \raggedleft 2 &  & \raggedleft 1 &  & \raggedleft 10 &  & \raggedleft 40 & \raggedleft\arraybslash 56\\
\lspbottomrule
\end{tabular}

Most of the V.V sequences found in loan words (44/56 – 79\%) are sequences of rising sonority, similar to the V.V sequences in inherited Malay roots (see §2.4.2). The remaining twelve vowel sequences include seven V.V sequences of equal sonority (/\textstyleChCharisSIL{i.u}/, /\textstyleChCharisSIL{ɛ.ɔ}/ and /\textstyleChCharisSIL{a.a}/), and five V.V sequences of falling sonority (/\textstyleChCharisSIL{ɔ.i}/, /\textstyleChCharisSIL{a.i}/, /\textstyleChCharisSIL{a.u}/, and /\textstyleChCharisSIL{a.ɛ}/).


\paragraph[Syllable structure and stress patterns]{Syllable structure and stress patterns}
\label{bkm:Ref338523907}
The syllable types and stress patterns for mono- and polysyllabic loan words are illustrated in Table  ‎2 .56 to Table  ‎2 .60. The basis for this investigation is the above-mentioned word list with 719 loan words.



Monosyllabic loan words with their different arrangements of C and V are presented in Table  ‎2 .56. The data indicates a clear preference for closed syllables with an onset consonant (85/86 – 99\%); only one item contains an onset vowel. The data also shows that monosyllabic loan words with onset consonant clusters are very common: 32 items (37\%) have a CC cluster and another four items (5\%) have a CCC cluster.
\end{styleBodyvvafter}

\begin{stylecaption}
\label{bkm:Ref338409408}Table ‎2.\stepcounter{Table}{\theTable}:  Monosyllabic loan words (86 items)
\end{stylecaption}

\begin{tabular}{llll}
\lsptoprule

 Syllable types & Count & Item & \arraybslash Gloss\\
VC & 1 & /\textstyleChCharisSIL{ˈɔm}/ & ‘uncle’\\
CV & 4 & /\textstyleChCharisSIL{ˈ}\textstyleChCharisSIL{tɛ}/ & ‘tea’\\
CVC & 45 & /\textstyleChCharisSIL{ˈdʒiŋ}/ & ‘genie’\\
CCV & 2 & /\textstyleChCharisSIL{ˈ}kwa/ & ‘broth’\\
CCVC & 30 & /\textstyleChCharisSIL{ˈtrɛk}/ & ‘truck’\\
CCCVC & 4 & /\textstyleChCharisSIL{ˈstrɔm}/ & ‘electricity’\\
\lspbottomrule
\end{tabular}

Disyllabic loan words with their attested syllable types and stress patterns are presented in Table  ‎2 .57. They are, with 422 items, the most common, a preference corresponding to that found for inherited Malay roots. While CV(C) syllables are preferred, the data also shows that consonant clusters are quite common: the corpus includes 59 items (14\%) with an onset CC cluster, three items (0.7\%) with an onset CCC cluster, and four items (1\%) with a coda CC cluster. By contrast, only 42 of the attested 1,004 inherited disyllabic Malay roots (4\%) have an onset CC cluster (§2.4.3).
\end{styleBodyaftervbefore}


Most of the disyllabic loanwords have penultimate stress (376/422 – 89\%), while 46 items have ultimate stress (11\%). This corresponds to the stress patterns observed for inherited disyllabic Malay roots: 104 of the 1,004 roots (10\%) have ultimate stress (§2.4.4.1).
\end{styleBodyvvafter}

\begin{stylecaption}
\label{bkm:Ref338409410}Table ‎2.\stepcounter{Table}{\theTable}:  Disyllabic loan words (422 items)
\end{stylecaption}

\tablehead{
 Syllable types & Count & Item & \arraybslash Gloss\\
}
\begin{tabular}{llll}
\lsptoprule
\multicolumn{4}{l}{Ultimate stress}\\
V.CV & \raggedleft 1 & /\textstyleChCharisSIL{a.ˈtɔ}/ & ‘or’\\
V.CVC & \raggedleft 1 & /\textstyleChCharisSIL{i.ˈman}/ & ‘faith’\\
CV.VC & \raggedleft 1 & /\textstyleChCharisSIL{ma.}\textstyleChCharisSIL{ˈ}\textstyleChCharisSIL{af}/ & ‘pardon’\\
CV.CV & \raggedleft 2 & /\textstyleChCharisSIL{pɛ.ˈta}/ & ‘map’\\
CV.CVC & \raggedleft 18 & /\textstyleChCharisSIL{mi.ˈnit}/ & ‘minute’\\
CV.CCV & \raggedleft 1 & /\textstyleChCharisSIL{nɛ.ˈgri}/ & ‘state’\\
CV.CCVC & \raggedleft 2 & /\textstyleChCharisSIL{rɛ.ˈtrit}/ & ‘retreat’\\
CVC.CV & \raggedleft 4 & /\textstyleChCharisSIL{pɛr.ˈlu}/ & ‘need’\\
CVC.CVC & \raggedleft 12 & /\textstyleChCharisSIL{kɔm.ˈbɔŋ}/ & ‘be inflated’\\
CVC.CCV & \raggedleft 1 & /\textstyleChCharisSIL{mɛn.ˈtri}/ & ‘cabinet minister’\\
CVC.CCVC & \raggedleft 1 & /\textstyleChCharisSIL{bis.ˈkwit}/ & ‘cracker’\\
CCV.CVC & \raggedleft 2 & /\textstyleChCharisSIL{plɛ.ˈtɔn}/ & ‘platoon’\\
\multicolumn{4}{l}{Penultimate stress}\\
V.CV & \raggedleft \textstyleChBold{\textmd{4}} & /\textstyleChCharisSIL{ˈi.dɛ}/ & ‘idea’\\
V.CVC & \raggedleft 18 & /\textstyleChCharisSIL{ˈi.dʒiŋ}/ & ‘permission’\\
V.CCVC & \raggedleft 1 & /\textstyleChCharisSIL{ˈi.blis}/ & ‘devil’\\
VC.CV & \raggedleft 6 & /\textstyleChCharisSIL{ˈil.mu}/ & ‘knowledge’\\
VC.CVC & \raggedleft 9 & /\textstyleChCharisSIL{ˈɛm.bɛr}/ & ‘bucket’\\
VC.CVCC & \raggedleft 1 & /\textstyleChCharisSIL{ˈɛr.pɔrt}/ & ‘airport’\\
VC.CCV & \raggedleft 2 & /\textstyleChCharisSIL{ˈin.trɔ}/ & ‘introduction’\\
VC.CCCV & \raggedleft 1 & /\textstyleChCharisSIL{ˈɛk.stra}/ & ‘extra’\\
VC.CCVC & \raggedleft 1 & /\textstyleChCharisSIL{ˈam.plɔp}/ & ‘envelope’\\
CV.V & \raggedleft 2 & /\textstyleChCharisSIL{ˈdɔ.a}/ & ‘prayer’\\
CV.VC & \raggedleft 5 & /\textstyleChCharisSIL{ˈta.at}/ & ‘be obedient’\\
CV.CV & \raggedleft 72 & /\textstyleChCharisSIL{ˈka.ja}/ & ‘be rich’\\
CV.CVC & \raggedleft 103 & /\textstyleChCharisSIL{ˈhɔ.nɔr}/ & ‘honorarium’\\
CV.CVCC & \raggedleft 1 & /\textstyleChCharisSIL{ˈra.pɔrt}/ & ‘school report book’\\
CV.CCVC & \raggedleft 2 & /\textstyleChCharisSIL{ˈdɔ.brak}/ & ‘smash’\\
CVC.CV & \raggedleft 48 & /\textstyleChCharisSIL{ˈwak.tu}/ & ‘time’\\
CVC.CVC & \raggedleft 51 & /\textstyleChCharisSIL{ˈkɔr.baŋ}/ & ‘sacrifice’\\
CVC.CCV & \raggedleft 2 & /\textstyleChCharisSIL{ˈman.tri}/ & ‘male nurse’\\
CVC.CCVC & \raggedleft 5 & /\textstyleChCharisSIL{ˈdis.trik}/ & ‘district’\\
CVC.CCVCC & \raggedleft 1 & /\textstyleChCharisSIL{ˈkɔm.plɛks}/ & ‘complex’\\
CCV.V & \raggedleft 1 & /\textstyleChCharisSIL{ˈsmu.a}/ & ‘all’\\
CCV.CV & \raggedleft 13 & /\textstyleChCharisSIL{ˈkwa.sa}/ & ‘power’\\
CCV.CVC & \raggedleft 11 & /\textstyleChCharisSIL{ˈsla.mat}/ & ‘be safe’\\
CCV.CCVC & \raggedleft 1 & /\textstyleChCharisSIL{ˈprɔ.gram}/ & ‘program’\\
CCVC.CV & \raggedleft 2 & /\textstyleChCharisSIL{ˈprik.sa}/ & ‘check’\\
CCVC.CVC & \raggedleft 9 & /\textstyleChCharisSIL{ˈknal.pɔt}/ & ‘muffler’\\
CCVCC.CVC & \raggedleft 1 & /\textstyleChCharisSIL{ˈtrans.fɛr}/ & ‘transfer’\\
CCCV.CV & \raggedleft 1 & /\textstyleChCharisSIL{ˈstri.ka}/ & ‘iron’\\
CCCVC.CV & \raggedleft 1 & /\textstyleChCharisSIL{ˈskrip.si}/ & ‘minithesis’\\
\lspbottomrule
\end{tabular}

Trisyllabic loan words with their attested syllable types and stress patterns are presented in Table  ‎2 .58. With 160 items they are considerably less common than disyllabic loan words. Again the preferred syllable structure is CV(C). In addition, however, the corpus includes a considerable number of loan words with consonant clusters, that is, 17 items (11\%) with an onset CC cluster, one item with an onset CCC cluster, and one item with a word-final CC cluster. By contrast, only one of the attested 66 inherited trisyllabic Malay roots has an onset CC cluster (§2.4.3).
\end{styleBodyaftervbefore}


Most of the trisyllabic loanwords have penultimate stress (136/160 – 85\%), while 23 items have ultimate stress (14\%) and one has antepenultimate stress. By comparison, only four of the 66 inherited trisyllabic Malay roots (6\%) have ultimate stress (§2.4.4.1).
\end{styleBodyvvafter}

\begin{stylecaption}
\label{bkm:Ref338409411}Table ‎2.\stepcounter{Table}{\theTable}:  Trisyllabic loan words (160 items)
\end{stylecaption}

\tablehead{
 Syllable types & Count & Item & \arraybslash Gloss\\
}
\begin{tabular}{llll}
\lsptoprule
\multicolumn{2}{l}{ Ultimate stress} &  & \\
V.CV.CVC & \raggedleft 1 & /\textstyleChCharisSIL{ˌɔ.tɔ.ˈnɔm}/ & ‘autonomous’\\
VC.CV.CVC & \raggedleft 1 & /\textstyleChCharisSIL{ˌin.si.ˈɲur}/ & ‘engineer’\\
CV.CV.VC & \raggedleft 2 & /\textstyleChCharisSIL{ˌsɛ.ri.ˈus}/ & ‘be serious’\\
CV.CV.CV & \raggedleft 1 & /\textstyleChCharisSIL{ˌrɛ.dʒɛ.ˈki}/ & ‘livelihood’\\
CV.CV.CVC & \raggedleft 6 & /\textstyleChCharisSIL{ˌdɔ.mi.ˈnan}/ & ‘dominate’\\
CV.CV.CCVC & \raggedleft 2 & /\textstyleChCharisSIL{ˌrɛ.pu.ˈblik}/ & ‘republic’\\
CV.CCV.CVCC & \raggedleft 1 & /\textstyleChCharisSIL{ˌpɛ.trɔ.ˈmaks}/ & ‘kerosene lantern’\\
CV.CVC.CV & \raggedleft 1 & /\textstyleChCharisSIL{ˌsu.pɛr.ˈmi}/ & ‘instant noodles’\\
CV.CVC.CVC & \raggedleft 4 & /\textstyleChCharisSIL{ˌkɔ.man.ˈdan}/ & ‘commandant’\\
CCV.CV.CVC & \raggedleft 3 & /\textstyleChCharisSIL{ˌprɛ.si.ˈdɛn}/ & ‘president’\\
CVC.CV.CVC & \raggedleft 1 & /\textstyleChCharisSIL{ˌkar.ta.ˈpɛl}/ & ‘slingshot’\\
\multicolumn{2}{l}{ Penultimate stress} &  & \\
V.CV.V & \raggedleft 2 & /\textstyleChCharisSIL{a.ˈrɔ.a}/ & ‘departed spirit’\\
V.CV.VC & \raggedleft 2 & /\textstyleChCharisSIL{ɛ.ˈgɔ.is}/ & ‘be egoistic’\\
V.CV.CV & \raggedleft 8 & /\textstyleChCharisSIL{a.ˈca.ra}/ & ‘ceremony’\\
V.CV.CVC & \raggedleft 3 & /\textstyleChCharisSIL{a.ˈla.mat}/ & ‘address’\\
V.CCV.CV & \raggedleft 1 & /\textstyleChCharisSIL{ɔ.ˈpra.si}/ & ‘surgery’\\
V.CVC.CV & \raggedleft 4 & /\textstyleChCharisSIL{a.ˈgɛn.da}/ & ‘agenda’\\
VC.CV.CV & \raggedleft 4 & /\textstyleChCharisSIL{as.ˈra.ma}/ & ‘dormitory’\\
VC.CV.CVC & \raggedleft 2 & /\textstyleChCharisSIL{ɔk.ˈtɔ.bɛr}/ & ‘October’\\
VC.CVC.CVC & \raggedleft 1 & /\textstyleChCharisSIL{in.ˈsɛn.tif}/ & ‘incentive’\\
VC.CCVC.CV & \raggedleft 1 & /\textstyleChCharisSIL{in.ˈstan.si}/ & ‘level’\\
CV.V.CV & \raggedleft 5 & /\textstyleChCharisSIL{pi.ˈa.ra}/ & ‘raise’\\
CV.V.CVC & \raggedleft 1 & /\textstyleChCharisSIL{di.ˈa.lɛk}/ & ‘dialect’\\
CV.VC.CV & \raggedleft 1 & /\textstyleChCharisSIL{rɛ.ˈak.si}/ & ‘reaction’\\
CV.CV.V & \raggedleft 9 & /\textstyleChCharisSIL{tʃɛ.ˈri.a}/ & ‘be cheerful’\\
CV.CV.VC & \raggedleft 2 & /\textstyleChCharisSIL{dʒɛ.ˈma.at}/ & ‘congregation’\\
CV.CV.CV & \raggedleft 35 & /\textstyleChCharisSIL{pɛ.ˈpa.ja}/ & ‘papaya’\\
CV.CV.CCV & \raggedleft 1 & /\textstyleChCharisSIL{tʃɛ.ˈri.tra}/ & ‘talk’\\
CV.CV.CVC & \raggedleft 4 & /\textstyleChCharisSIL{na.ˈsi.hat}/ & ‘advice’\\
CV.CVC.CV & \raggedleft 4 & /\textstyleChCharisSIL{ta.ˈlɛn.ta}/ & ‘gift’\\
CV.CVC.CVC & \raggedleft 8 & /\textstyleChCharisSIL{kɛ.ˈtum.bar}/ & ‘coriander’\\
CCV.V.CV & \raggedleft 1 & /\textstyleChCharisSIL{pri.ˈɔ.dɛ}/ & ‘period’\\
CCV.VC.CV & \raggedleft 1 & /\textstyleChCharisSIL{klu.ˈar.ga}/ & ‘family’\\
CCV.CV.CV & \raggedleft 3 & /\textstyleChCharisSIL{pri.ˈba.di}/ & ‘be private’\\
CCV.CVC.CV & \raggedleft 1 & /\textstyleChCharisSIL{prɔ.ˈpin.si}/ & ‘province’\\
CCCV.CV.CV & \raggedleft 1 & /\textstyleChCharisSIL{stra.ˈtɛ.gi}/ & ‘strategy’\\
CVC.CV.VC & \raggedleft 1 & /\textstyleChCharisSIL{man.ˈfa.at}/ & ‘benefit’\\
CVC.CV.CV & \raggedleft 20 & /\textstyleChCharisSIL{pɛr.ˈtʃa.ja}/ & ‘trust’\\
CVC.CV.CVC & \raggedleft 5 & /\textstyleChCharisSIL{kɔm.ˈpu.tɛr}/ & ‘computer’\\
CVC.CVC.CV & \raggedleft 3 & /\textstyleChCharisSIL{sɛm.ˈpur.na}/ & ‘be perfect’\\
CVC.CCV.CVC & \raggedleft 2 & /\textstyleChCharisSIL{kɔm.ˈplɔ.taŋ}/ & ‘(half)circle’\\
\multicolumn{2}{l}{ Antepenultimate stress} &  & \\
CV.CV.CVC & \raggedleft 1 & /\textstyleChCharisSIL{ˈdʒɛ.ri.ˌkɛn}/ & ‘jerry can’\\
\lspbottomrule
\end{tabular}

The corpus also contains 42 loan words of four syllables. Their syllable types and stress patterns are presented in Table  ‎2 .59. While they are quite rare among loan words (42/718 – 6\%), their portion is higher than that attested for inherited Malay roots (two out of 1,117 items) (§2.4.3). The preferred syllable structure is again CV(C). In addition, the corpus includes five loan words (12\%) with an onset CC cluster. By contrast, neither of the two attested inherited quadrisyllabic Malay roots has a consonant cluster. Most of the quadrisyllabic loanwords have penultimate stress (36/42 – 86\%), while five items have ultimate stress (12\%) and one has antepenultimate stress. By comparison, both inherited quadrisyllabic Malay roots have penultimate stress (§2.4.4.1).


\begin{stylecaption}
\label{bkm:Ref338409412}Table ‎2.\stepcounter{Table}{\theTable}:  Quadrisyllabic loan words (42 items)
\end{stylecaption}

\tablehead{
 Syllable types & Count & Item & \arraybslash Gloss\\
}
\begin{tabular}{llll}
\lsptoprule
\multicolumn{2}{l}{ Ultimate stress} &  & \\
VC.CV.CV.CVC & \raggedleft 1 & /\textstyleChCharisSIL{is.ˌti.ra.ˈhat}/ & ‘rest’\\
CV.CV.V.CVC & \raggedleft 1 & /\textstyleChCharisSIL{ˌna.si.ɔ.ˈnal}/ & ‘national’\\
CV.CV.CV.CVC & \raggedleft 2 & /\textstyleChCharisSIL{ma.ˌjɔ.ri.ˈtas}/ & ‘majority’\\
CCV.V.CV.CVC & \raggedleft 1 & /\textstyleChCharisSIL{pri.ˌɔ.ri.ˈtas}/ & ‘priority’\\
\multicolumn{2}{l}{ Penultimate stress} &  & \\
V.CV.CV.CV & \raggedleft \textstyleChBold{\textmd{1}} & /\textstyleChCharisSIL{ˌɔ.tɔ.ˈnɔ.mi}/ & ‘autonomy’\\
V.CV.CV.CVC & \raggedleft \textstyleChBold{\textmd{1}} & /\textstyleChCharisSIL{ˌɔ.tɔ.ˈma.tis}/ & ‘be automatic’\\
VC.CV.V.CV & \raggedleft 1 & /\textstyleChCharisSIL{ˌɔn.dɔ.ˈa.fi}/ & ‘traditional chief’\\
VC.CV.CV.CV & \raggedleft 1 & /\textstyleChCharisSIL{ˌis.ti.ˈmɛ.wa}/ & ‘be special’\\
VC.CV.CV.CVC & \raggedleft 1 & /\textstyleChCharisSIL{ˌan.ti.ˈfi.rus}/ & ‘antivirus’\\
VC.CCV.CV.CV & \raggedleft 1 & /\textstyleChCharisSIL{ˌas.trɔ.ˈnɔ.mi}/ & ‘astronomy’\\
VC.CVC.CV.CV & \raggedleft 1 & /\textstyleChCharisSIL{ˌin.fɔr.ˈma.si}/ & ‘information’\\
CV.V.CV.CV & \raggedleft 2 & /\textstyleChCharisSIL{ˌbi.ɔ.ˈlɔ.gi}/ & ‘biology’\\
CV.CV.CV.V & \raggedleft 5 & /\textstyleChCharisSIL{ˌma.nu.ˈsi.a}/ & ‘human being’\\
CV.CV.V.CV & \raggedleft 1 & /\textstyleChCharisSIL{ˌdʒa.nu.ˈa.ri}/ & ‘January’\\
CV.CV.V.CVC & \raggedleft 1 & /\textstyleChCharisSIL{ˌka.ri.ˈa.waŋ}/ & ‘employee’\\
CV.CV.CV.CV & \raggedleft 6 & /\textstyleChCharisSIL{ˌtɛ.lɛ.ˈfi.si}/ & ‘television’\\
CV.CV.CV.CVC & \raggedleft 3 & /\textstyleChCharisSIL{ˌma.sa.ˈra.kat}/ & ‘community’\\
CV.CV.CVC.CVC & \raggedleft 2 & /\textstyleChCharisSIL{ˌrɛ.fɛ.ˈrɛn.dum}/ & ‘referendum’\\
CV.CVC.CV.CV & \raggedleft 4 & /\textstyleChCharisSIL{ˌwa.wan.ˈtʃa.ra}/ & ‘interview’\\
CV.CVC.CV.CVC & \raggedleft 1 & /\textstyleChCharisSIL{ˌsɛ.kɛr.ˈta.ris}/ & ‘secretary’\\
CV.CCV.V.CV & \raggedleft 2 & /\textstyleChCharisSIL{ˌfɛ.bru.ˈa.ri}/ & ‘February’\\
CVC.CV.CV.CV & \raggedleft 1 & /\textstyleChCharisSIL{ˌkɔr.di.ˈna.si}/ & ‘coordinate’\\
CVC.CV.CV.CVC & \raggedleft 1 & /\textstyleChCharisSIL{ˌkɔr.di.ˈna.tɔr}/ & ‘coordinator’\\
\multicolumn{2}{l}{ Antepenultimate stress} &  & \\
V.CCV.CV.V & \raggedleft 1 & /\textstyleChCharisSIL{a.ˈgra.ri.ˌa}/ & ‘agrarian affairs’\\
\lspbottomrule
\end{tabular}

In addition, the corpus also contains ten pentasyllabic roots which are presented in Table  ‎2 .60. Most of them have penultimate stress (6/9 – 67\%), while two have ultimate stress and one has antepenultimate stress.


\begin{stylecaption}
\label{bkm:Ref338409413}Table ‎2.\stepcounter{Table}{\theTable}:  Pentasyllabic roots (9 items)
\end{stylecaption}

\tablehead{
 Syllable types & Count & Item & \arraybslash Gloss\\
}
\begin{tabular}{llll}
\lsptoprule
\multicolumn{2}{l}{ Ultimate stress} &  & \\
V.CV.CVC.CV.CVC & \raggedleft 1 & /\textstyleChCharisSIL{ˌu.ni.ˌfɛr.si.ˈtas}/ & ‘university’\\
CV.CV.CV.CV.CV & \raggedleft 1 & /\textstyleChCharisSIL{ˌpi.si.ˌkɔ.lɔ.ˈgi}/ & ‘psychology’\\
\multicolumn{2}{l}{ Penultimate stress} &  & \\
V.CVC.CV.V.CV & \raggedleft 1 & /\textstyleChCharisSIL{ɔ.ˌlɪm.pi.ˈa.dɛ}/ & ‘Olympiad’\\
V.CVC.CV.CV.CV & \raggedleft 1 & /\textstyleChCharisSIL{ɛ.ˌman.si.ˈpa.si}/ & ‘emancipation’\\
CV.CV.V.CV.CVC & \raggedleft 1 & /\textstyleChCharisSIL{mi.ˌsi.ɔ.ˈna.ris}/ & ‘missionary’\\
CV.CV.CV.CV.CV & \raggedleft 2 & /\textstyleChCharisSIL{ma.ˌtɛ.ma.ˈti.ka}/ & ‘mathematics’\\
CV.CVC.CV.CV.V & \raggedleft 1 & /\textstyleChCharisSIL{sɛ.ˌkɛr.ta.ˈri.a}/ & ‘secretariat’\\
\multicolumn{2}{l}{ Antepenultimate stress} &  & \\
CV.V.CV.CV.V & \raggedleft 1 & /\textstyleChCharisSIL{ˌtɛ.ɔ.ˈlɔ.gi.ˌa}/ & ‘theology’\\
\lspbottomrule
\end{tabular}

The data presented in Table  ‎2 .56 to Table  ‎2 .60 show that for loan words in Papuan Malay the preferred syllable types and stress patterns correspond to those attested in inherited Malay roots: most of the 719 loan words are disyllabic (422/719 – 59\%) and most of the items with two or more syllables have penultimate stress (554/633 – 88\%). Table  ‎2 .61 presents a frequency count for the attested syllable types and stress patterns. Also corresponding to inherited Malay roots, the preferred syllable structure is CV(C). Unlike native roots, however, a considerable number of loan words have consonant clusters, most of which are onset CC clusters.


\begin{stylecaption}
\label{bkm:Ref338419330}Table ‎2.\stepcounter{Table}{\theTable}:  Syllable types and stress patterns for loan words – Frequencies
\end{stylecaption}

\tablehead{
 Syllable types & \multicolumn{3}{l}{ Stress patterns} & \arraybslash Total\\
}
\begin{tabular}{lllll}
\lsptoprule
Monosyllabic & \multicolumn{3}{l}{ (n/a)} & \\
Polysyllabic & \textsc{ult}: & \textsc{p-ult}: & \textsc{a-p-ult}: & \\
Disyllabic & \raggedleft 46 & \raggedleft 376 & \raggedleft {}-{}-{}- & \raggedleft\arraybslash 422\\
Trisyllabic & \raggedleft 23 & \raggedleft 136 & \raggedleft 1 & \raggedleft\arraybslash 160\\
Quadrisyllabic & \raggedleft 5 & \raggedleft 36 & \raggedleft 1 & \raggedleft\arraybslash 42\\
Pentasyllabic & \raggedleft 2 & \raggedleft 6 & \raggedleft 1 & \raggedleft\arraybslash 9\\
& \raggedleft 76 & \raggedleft 554 & \raggedleft 3 & \raggedleft\arraybslash 633\\
\lspbottomrule
\end{tabular}

Quite often, but not always, the adaption of loan words into Papuan Malay involves stress shift from a syllable other than the penultimate one in the original item to the preferred penultimate syllable in the Papuan Malay word. This is illustrated in Table  ‎2 .62 with three loan words: \textitbf{astronomi} ‘astronomy’ and \textitbf{strategi} ‘strategy’ are loan words from Dutch which have ultimate stress, while \textitbf{transfer} ‘transfer’ is an English loan word which has ultimate stress. In Papuan Malay, by contrast, the three items are realized with stress on the penultimate syllable.


\begin{stylecaption}
\label{bkm:Ref379792791}Table ‎2.\stepcounter{Table}{\theTable}:  Stress shift in loan words\footnote{\\
\\
\\
\\
\\
\\
\\
The Dutch examples are taken from {Woorden.org MMXI (2010-)} and the English examples from {Oxford University Press (2000-)}.\\
\\
\\
\\
\\
\\
\\
\\
}
\end{stylecaption}

\begin{tabular}{lllll}
\lsptoprule

\multicolumn{3}{l}{ Papuan Malay} & Dutch & \arraybslash English\\
\textitbf{astronomi} & \textstyleChCharisSIL{ˌɐs.trɔ.ˈnɔ.mi} & ‘astronomy’ & \textstyleChCharisSIL{ɑs.tro.no.ˈmi} & \textstyleChCharisSIL{ə.ˈstrɑ.nə.mi}\\
\textitbf{strategi} & \textstyleChCharisSIL{stra.ˈtɛ.gi} & ‘strategy’ & \textstyleChCharisSIL{stra.tə.ˈxi} & \textstyleChCharisSIL{ˈstræ.tɪ.dʒɪ}\\
\textitbf{transfer} & \textstyleChCharisSIL{ˈtɾɐns.fɛ̞r} & ‘transfer’ & \textstyleChCharisSIL{trɑns.’f[28F?]}\textstyleChCharisSIL{:r} & \textstyleChCharisSIL{trɑːns.ˈfɜː(r)}\\
\lspbottomrule
\end{tabular}
\section{Orthographic conventions}
\label{bkm:Ref324759940}
The orthographic conventions for the Papuan Malay consonant and vowel phonemes used in this grammar are presented in Table  ‎2 .63.


\begin{stylecaption}
\label{bkm:Ref324345565}Table ‎2.\stepcounter{Table}{\theTable}:  Orthographic conventions
\end{stylecaption}

\begin{tabular}{llllllllm{4.5984238E-4in}llllllllllllllllllm{-9.4015896E-4in}lllllll}
\lsptoprule

\multicolumn{35}{l}{Consonants}\\
\textsc{phon} & p & \multicolumn{2}{l}{ \textstyleChCharisSIL{b}} & \multicolumn{2}{l}{ \textstyleChCharisSIL{t}} & \multicolumn{2}{l}{ \textstyleChCharisSIL{d}} & \multicolumn{2}{l}{ \textstyleChCharisSIL{tʃ}} & \multicolumn{2}{l}{ \textstyleChCharisSIL{dʒ}} & \multicolumn{2}{l}{ \textstyleChCharisSIL{k}} & \multicolumn{2}{l}{ \textstyleChCharisSIL{g}} & \multicolumn{2}{l}{ \textstyleChCharisSIL{s}} & \textstyleChCharisSIL{h} & \multicolumn{2}{l}{ \textstyleChCharisSIL{m}} & \multicolumn{2}{l}{ \textstyleChCharisSIL{n}} & \multicolumn{3}{l}{ \textstyleChCharisSIL{ɲ}} & \multicolumn{2}{l}{ \textstyleChCharisSIL{ŋ}} & \multicolumn{2}{l}{ \textstyleChCharisSIL{r}} & \multicolumn{2}{l}{ \textstyleChCharisSIL{l}} & \multicolumn{2}{l}{ \textstyleChCharisSIL{j}} & \arraybslash \textstyleChCharisSIL{w}\\
\textsc{orth} & p & \multicolumn{2}{l}{ b} & \multicolumn{2}{l}{ t} & \multicolumn{2}{l}{ d} & \multicolumn{2}{l}{ c} & \multicolumn{2}{l}{ j} & \multicolumn{2}{l}{ k} & \multicolumn{2}{l}{ g} & \multicolumn{2}{l}{ s} & h & \multicolumn{2}{l}{ m} & \multicolumn{2}{l}{ n} & \multicolumn{3}{l}{ ny} & \multicolumn{2}{l}{ ng} & \multicolumn{2}{l}{ r} & \multicolumn{2}{l}{ l} & \multicolumn{2}{l}{ y} & \arraybslash w\\
\multicolumn{35}{l}{Vowels}\\
\textsc{phon} & \multicolumn{2}{l}{ \textstyleChCharisSIL{i}} & \multicolumn{2}{l}{ \textstyleChCharisSIL{ɛ}} & \multicolumn{2}{l}{ \textstyleChCharisSIL{u}} & \multicolumn{2}{l}{ \textstyleChCharisSIL{ɔ}} & \multicolumn{2}{l}{ \textstyleChCharisSIL{a}} & \multicolumn{2}{l}{} & \multicolumn{2}{l}{} & \multicolumn{2}{l}{} &  & \multicolumn{2}{l}{} & \multicolumn{2}{l}{} & \multicolumn{2}{l}{} &  & \multicolumn{2}{l}{} & \multicolumn{2}{l}{} & \multicolumn{2}{l}{} & \multicolumn{2}{l}{} & \multicolumn{2}{l}{}\\
\textsc{orth} & \multicolumn{2}{l}{ i} & \multicolumn{2}{l}{ e} & \multicolumn{2}{l}{ u} & \multicolumn{2}{l}{ o} & \multicolumn{2}{l}{ a} & \multicolumn{2}{l}{} & \multicolumn{2}{l}{} & \multicolumn{2}{l}{} &  & \multicolumn{2}{l}{} & \multicolumn{2}{l}{} & \multicolumn{2}{l}{} &  & \multicolumn{2}{l}{} & \multicolumn{2}{l}{} & \multicolumn{2}{l}{} & \multicolumn{2}{l}{} & \multicolumn{2}{l}{}\\
\lspbottomrule
\end{tabular}

The orthographic representation of the affricates, the palatal and velar nasals, and the palatal approximant follows the conventions for Standard Indonesian, as these are also used by Papuan Malay speakers when writing Papuan Malay. Stress is not marked in the examples and texts in this book. In the word lists in Appendix A, however, stress is marked; and those lexemes which do not carry penultimate stress but ultimate or antepenultimate stress are marked with “x” for the interested reader.
\end{styleBodyaftervbefore}


For the representation of the velar nasal in the word-internal coda position, the surface realization is used rather than the underlying phonemic form, as in \textitbf{bantu} ‘help’ and \textitbf{janji} ‘promise’. In representing the palatalized alveolar fricative, the surface realization is used instead of the underlying phonemic form. That is, [\textstyleChCharisSIL{sʲ}] is represented as {\textless}\textitbf{sy}{\textgreater} as in \textitbf{syukur} ‘thanks to God’. For vocalic allophones, their surface realization instead of their underlying phonemic form is used if that allophone is also an independent phoneme. Examples are the alternative realizations of the vowel combinations /\textstyleChCharisSIL{ai}/ and /\textstyleChCharisSIL{au}/ (see §2.3.3), such as \textitbf{capay} or \textitbf{cape} ‘be tired’, or \textitbf{pulaw} or \textitbf{pulow} ‘island’. These conventions also apply to the orthographic representation of the (historical) affixes, if one element of the affix is also an independent segment; hence, \textitbf{bakalay} /\textstyleChCharisSIL{ba}\textsc{\-}\textstyleChCharisSIL{ˈkalaj}/ ‘to fight’ versus \textitbf{bertriak} /\textstyleChCharisSIL{bɛr-ˈtriak}/ ‘to scream’ or \textitbf{talipat} /\textstyleChCharisSIL{ta\-lipat}/ ‘be folded’ versus \textitbf{terpaksa} /\textstyleChCharisSIL{tɛr-ˈpaksa}/ ‘be forced’ (see §3.1 for a detailed discussion on derivation processes in Papuan Malay and the realizations of the (historical) affixes).



In fast speech, Papuan Malay speakers very often shorten disyllabic lexical items to monosyllabic ones. This affects most often the personal pronouns (see §5.5 and Chapter 6), the possessive marker (see §9.1), and the following lexical items: \textitbf{dengang} ‘with’ is shortened to \textitbf{deng}, \textitbf{bilang} ‘say’ to \textitbf{blang}, \textitbf{ini} ‘\textsc{d.prox}’ to \textitbf{ni}, \textitbf{itu} ‘\textsc{d.dist}’ to \textitbf{tu}, \textitbf{kasi} ‘give’ to \textitbf{kas}, \textitbf{pergi}/\textitbf{pigi} ‘go’ to \textitbf{pi}, and \textitbf{suda} ‘already’ to \textitbf{su}. Whenever speakers use these short forms, they are also given in the examples and texts in this grammar.



Vowel length is not phonemic in Papuan Malay. It does, however, have the pragmatic function of adding emphasis to a speaker’s utterance, as discussed in §2.3.2.3. To indicate this emphasis in the context of this grammar, vowel lengthening is represented orthographically and realized with triple vowels.


\section{Summary}
\label{bkm:Ref324759993}
The Papuan Malay phoneme inventory consists of 18 consonants (six stops, two affricates, two fricatives, four nasals, two liquids, and two approximants) and five vowels. In terms of {Lass’ (1984: 134–159)} system typology of consonants and vowels, the Papuan Malay consonant and vowel systems show, overall, no typologically unexpected constellations, with the exception of the fricatives.



Consonant system: The obstruent system with its “‘cardinal’ set /\textstyleChCharisSIL{p t k}/” and its palato-alveolar affricate set as “one ‘intermediate’ place” of articulation, using {Lass’ (1984: 147)} terminology{, }shows no typologically unexpected constellations. The fricative system with alveolar /\textstyleChCharisSIL{s}/ and glottal /\textstyleChCharisSIL{h}/ is cross-linguistically less typical. Following {Lass’ (1984: 154)} obstruent frequency hierarchy, systems with only two fricatives typically consist of alveolar /\textstyleChCharisSIL{s}/, to which labial /\textstyleChCharisSIL{f}/ rather than glottal /\textstyleChCharisSIL{h}/ is added. While the stop system is symmetric in terms of voice, the fricative system lacks a voiced series, while the nasal system lacks a voiceless series. The lack of these two series, however, is cross-linguistically quite common. They correspond to {Maddieson’s (2011a: 4)} findings that “fricatives are more commonly voiceless”. They also agree with {Lass’ (1984: 155–157)} findings that nasals show a clear “preference for voice”. All consonants occur as onsets, while the range of consonants occurring in the coda position is considerably smaller.



Vowel system: The cross-linguistically very common “5-vowel” system with its “two heights in front and back with a low central vowel”, applying {\citep[143]{Lass1984}} terminology, shows no typologically unexpected constellations. As is typical of such systems cross-linguistically, the front vowels are unrounded while the back vowels are rounded. All five vowels occur in stressed and unstressed, open and closed syllables.



A restricted sample of like segments can occur in sequences. The constraints on their linear sequencing correspond to the Sonority Sequencing Principle if this is taken as a functional principle by which to explain the linear ordering of like segments. In CC clusters, the less sonorous segment precedes the more sonorous segment. The first consonant is typically a stop while the second consonant is a liquid. For V.V sequences the rise in sonority is less marked. The first vowel is most often a close vowel, while the second one is usually the open central vowel.



Papuan Malay shows a clear preference for disyllabic roots and for CV(C) syllables, which is typologically the most common structure. Thereby, the language displays a “moderately complex syllable structure”, in terms of {Maddieson’s (2011b: 4)} typology of the syllable structure. Cross-linguistically, however, Papuan Malay would be more likely to have a simple rather than a moderately complex canonical structure, as it consists of only 18 consonants. Primary stress typically falls on the penultimate syllable, although this stress pattern is not rigid. Secondary stress usually falls on the alternating syllable preceding the one carrying the primary stress. This stress pattern applies to lexical roots as well as to lexical items that are historically derived by (unproductive) affixation.



Adding to its 18 native consonant system, Papuan Malay has adopted one loan segment, the voiceless labio-dental fricative /\textstyleChCharisSIL{f}/. Also, Papuan Malay has developed three substitution strategies to realize the voiceless postalveolar fricative /\textstyleChCharisSIL{ʃ}/ found in loan words of Arabic origins. For the most part, the phonological and phonetic processes found in loan words correspond to those found in inherited Malay roots. The exception is the process of nasal assimilation, which is applied less rigorously. Consonants and vowels in loan words show the same distribution as in inherited Malay roots. In sequences of like segments, the range of attested consonants and vowels is wider in comparison to that found in inherited Malay roots. Further, for V.V sequences the rise in sonority is less marked. The preferred syllable types and stress patterns attested in loan words correspond to those found in inherited Malay roots. Compared to Malay roots, however, a larger number of loan words employ consonant clusters.


%\setcounter{page}{1}\chapter[Word{}-formation]{Word-formation}
\label{bkm:Ref374451653}\label{bkm:Ref374441467}\label{bkm:Ref374441464}\label{bkm:Ref374436736}\label{bkm:Ref374434741}\label{bkm:Ref374433197}
Papuan Malay has very little productive morphology. Words are typically single root morphemes and word formation is limited to the two derivational processes of reduplication and affixation. Compounding is a third word-formation process; it remains uncertain, however, to what degree it is a productive process. Inflectional morphology is lacking, as nouns and verbs are not marked for any grammatical category such as gender, number, or case. There is also no voice system on verbs.



In discussing word-formation in Papuan Malay, a major issue is to what degree these processes are productive. Following {Plag}{ (2006a: 127)}, the “productivity of a word-formation process can be defined as its general potential to be used to create new words and as the degree to which this potential is exploited by the speakers.” Given this definition, the data in the corpus indicates that reduplication in Papuan Malay is a very productive process, whereas affixation has only very limited productivity. The productivity of compounding as a word-formation process remains debatable.
\end{styleBodyvafter}


This chapter discusses two word-formation processes in detail: affixation in §3.1 and compounding in §3.2. Reduplication is described in Chapter 4. The main points of this chapter are summarized in §3.3.
\end{styleBodyvxvafter}

\section{Affixation}
\label{bkm:Ref379889998}\label{bkm:Ref374455807}\label{bkm:Ref374452977}\label{bkm:Ref374451203}\label{bkm:Ref374450009}\label{bkm:Ref374436734}\label{bkm:Ref374433235}\label{bkm:Ref373929586}\label{bkm:Ref357329204}\label{bkm:Ref357328128}\label{bkm:Ref341973800}\subsection{Introduction}
\label{bkm:Ref341964824}
In Papuan Malay, affixation is a morphological process whereby an affix is attached to a lexical root to derive new lexemes. This process, however, applies to the open word classes only, namely to nouns and verbs.



The corpus contains a considerable number of morphologically complex lexical items with the 2,458-item word list mentioned in §1.11.6 including 523 affixed lexemes (21\%). The most commonly employed (historical) affixes are the prefixes \textscItalBold{ter\-} ‘\textsc{acl}’, \textscItalBold{pe(n)\-} ‘\textsc{ag}’, and \textscItalBold{ber\-} ‘\textsc{vblz}’, the suffixes \-\textitbf{ang} ‘\textsc{pat}’ and \textitbf{\-nya} ‘\textsc{3possr}’, and the circumfix \textitbf{ke}\-/\-\textitbf{ang} ‘\textsc{nmlz}’.\footnote{\\
\\
\\
\\
\\
\\
\\
The small caps designate the abstract representation of affixes that have more than one form of realization; prefixes \textscItalBold{ter\-}, \textscItalBold{pe(n)\-}, and \textscItalBold{ber\-}, have two allomorphs each, namely \textitbf{ter}\textitbf{\-} and \textitbf{ta}\textitbf{\-} (§3.1.2.1), \textitbf{pe(}\textscItalBold{n}\textitbf{)}\- and \textitbf{pa(}\textscItalBold{n}\textitbf{)}\- (small-caps \textscItalBold{n} represents the different realizations of the nasal) (§3.1.4.1), and \textitbf{ber}\- and \textitbf{ba}\- (§3.1.5.1), respectively.\\
\\
\\
\\
\\
\\
\\
\\
}
\end{styleBodyvafter}


Before examining these affixes in detail, the remainder of this introduction discusses methodological issues related to examining the productivity of affixation in Papuan Malay.
\end{styleBodyvafter}


Morphological patterns are considered to be productive if language users apply them “to create new well-formed complex words” by systematically extending the pattern “to new cases” {(Booij 2007: 67, 68)}. By contrast, a morphological pattern is said to be unproductive when the morphological rule involved “is not used for coining new words” but “has become obsolete” {(2007: 68)}. The productivity of a given pattern is a matter of degree, however, as pointed out by scholars such as {Aikhenvald (2007: 49–58)}, {Bauer (1983: 62–100)}, {Booij (2007: 67–71) or Pike (1967: 169–172)}. This degree depends on the amount “to which the structural possibilities of a word-formation pattern are actually used” {\citep[68]{Booij2007}}. That is, depending on their functional load, some patterns are “fully active” or productive, while others are “inactive” or unproductive, with “semi-active” or semi-productive patterns found in-between {(Pike 1967: 169–171)}.\footnote{\\
\\
\\
\\
\\
\\
\\
{Pike (1967: 169–171)} talks about the “Activeness of Morphemes” rather than of “morphological patterns”.\\
\\
\\
\\
\\
\\
\\
\\
} Therefore, productivity is best viewed as a “cline” {\citep[97]{Bauer1983}} or a “scalar phenomenon” {\citep[126]{Bauer2001}}.\footnote{\\
\\
\\
\\
\\
\\
\\
As {\citet[125]{Bauer2001}} elaborates, however, there is an ongoing discussion among scholars “whether productivity is a gradable/scalar phenomenon or not”.\\
\\
\\
\\
\\
\\
\\
\\
} On such a cline of productivity, fully productive patterns are viewed as one end-point, and completely unproductive patterns as the other end-point of the continuum, with semi-productive patterns found in-between.
\end{styleBodyvafter}


To investigate whether and to what degree Papuan Malay speakers employ a given affix to create new words, one technique would be to devise a test along the lines of {Aronoff and Schvaneveldt’s (1978)} “Productivity Experiment”. This psycholinguistic experiment involved a lexical-decision task which required testees to make judgments about possible but non-occurring affixed words. That is, the testees had to judge whether or not these words were instances of English.
\end{styleBodyvafter}


For the present study no productivity tests were conducted to determine whether and to what extent a given affix can be attached to Papuan Malay roots to derive new lexical items. Tests such as the mentioned lexical-decision tasks were considered unworkable due to the sociolinguistic profile of the Papuan Malay speech variety and speech communities, discussed in §1.5:
\end{styleBodyvvafter}

\begin{itemize}
\item \begin{styleIIndented}
Functional distribution of Papuan Malay as the \textsc{low} variety, and Indonesian as the \textsc{high} variety, in terms of {Ferguson’s (1972)} notion of diglossia;
\end{styleIIndented}\item \begin{styleIIndented}
Positive to somewhat ambivalent language attitudes toward Papuan Malay; and
\end{styleIIndented}\end{itemize}
\begin{itemize}
\item \begin{styleIvI}
Lack of language awareness of many Papuan Malay speakers about the status of Papuan Malay as a language distinct from Indonesian.
\end{styleIvI}\end{itemize}

Given this sociolinguistic profile and the formal setting of a test situation as well as the fairly high degree of linguistic relatedness between Papuan Malay and Indonesian, an undesirable amount of interference from Indonesian was expected. This assumption is based on {Weinreich’s (1953: 1)} definition of “interference” as “instances of deviation from the norms of either language which occur in the speech of bilinguals as a result of their familiarity with more than one language, i.e. as a result of language contact”. Even in a monolingual test situation, such interference would most likely have had a skewing impact on testees’ naïve judgments, given that, when in the “monolingual speech mode […] bilinguals rarely deactivate the other language totally”, as {\citet[59]{Grosjean1992}} points out.



Given these problems, the attested affixes and derived words were instead examined in terms of six language internal and three language external factors. These factors were deemed relevant in examining the productivity of these affixes.
\end{styleBodyvxafter}

%\setcounter{itemize}{0}
\begin{itemize}
\item \begin{styleOvNvwnext}
Language internal factors
\end{styleOvNvwnext}\end{itemize}

The affixes are examined with respect to the following six language internal factors: (a) syntactic properties, (b) type frequencies, token frequencies, and hapaxes, (c) form{}-function relationship between the derivation and its base word, (d) alternative strategies, (e) formally complex words with non-compositional semantics, and (f) status of the affixed lexemes as part of the Papuan Malay lexicon or as code-switches with Indonesian.
\end{styleBodyxafter}

%\setcounter{itemize}{0}
\begin{itemize}
\item \begin{styleOvNvwnext}
Syntactic properties
\end{styleOvNvwnext}\end{itemize}
\begin{styleIvI}
If an affix is “polyfunctional”, that is, if it can take bases from more than one lexical category, this is taken as evidence that the process is more productive ({Booij 2002: 90–91}; see also {Zwanenburg 2000}). Hence, the syntactic properties for each affix are examined as to whether it can be attached to verbal, nominal, adverbial, and/or other bases. Likewise, the syntactic properties of the affixed lexemes are described, as to which word class they belong to.
\end{styleIvI}

\begin{itemize}
\item \begin{styleOvNvwnext}
Type frequencies, token frequencies and hapaxes\footnote{\\
\\
\\
\\
\\
\\
\\
Type frequency is defined as “the number of types of a class of linguistic units in a corpus”, while token frequency refers to “the number of tokens of a linguistic unit or a class of linguistic units in a corpus” {\citep[323]{Booij2007}}. Hapaxes are “new word types that occur only once in the corpus, and clearly do not belong to the set of established words” {(2007: 69)}.\\
\\
\\
\\
\\
\\
\\
\\
}
\end{styleOvNvwnext}\end{itemize}
\begin{styleIIndented}
If an affix is represented by a large number of words (high type frequency) which, in turn, have low token frequencies, this is taken as an indication that the affixation process is more likely to be productive. (For the purposes of this study, type frequencies of ten or more are considered as “(relatively) high” while token frequencies of less than 20 are considered as “(relatively) low”.)
\end{styleIIndented}

\begin{styleIvptafter}
{As Hay (2001: 1044–1047)} points out, “the frequency of the base form is involved in facilitating decomposability. When the base is more frequent than the whole, the word is easily and readily decomposable. However, when the derived form is more frequent than the base it contains, it is more difficult to decompose and appears to be less complex”. In terms of processing, morphologically complex words with a low relative frequency are accessed via their parts, that is, via a “decomposed access” or “parsing route”. Morphologically complex words with a high relative frequency, by contrast, are accessed as whole words via a “whole-word access” or “direct route” {(2001: 1055)}.
\end{styleIvptafter}

\begin{styleIvptafter}
Building on {\citet{Hay2001}}, {Hay and Baayen (2002: 203–204)} argue that “for an affix to remain productive, words containing that affix must be parsed sufficiently often that the resting activation level of that affix remains high”. The findings of their study confirm this link between productivity and parsing. {Hay and \citet{Baayen2002}} show that affixes which derive words with low relative frequencies and high rates of decomposition are more likely to be productive. By contrast, affixes which derive words with high relative frequencies and low rates of decomposition are less likely to be productive.
\end{styleIvptafter}

\begin{styleIvptafter}
Along similar lines {\citet[542]{Plag2006b}} discusses the decomposability of derived words with low token frequencies which “tend to be words that are unlikely to be familiar to the hearer”. They can, however, be understood if “an available word-formation rule allows the decomposition of the newly encountered word into its constituent morphemes and thus the computation of the meaning on the basis of the meaning of the parts” {(2006b: 542)}. Hence, productive morphological patterns tend to be characterized by “large numbers of low frequency words and small numbers of high frequency words, with the former keeping the rule alive. In contrast, unproductive morphological categories will be characterized by a preponderance of words with rather high frequencies and by a small number of words with low frequencies” {(2006b: 542)}.
\end{styleIvptafter}

\begin{styleIvptafter}
Among the derived words with low token frequency, hapaxes are especially useful in determining the productivity of a morphological pattern, as “the highest proportion of neologisms” is found here {\citep[542]{Plag2006b}}; or in other words, “[the] higher the number of hapaxes, the greater the productivity” {(2006b: 544)}. Therefore, as {Booij (2007: 69–70)} points out, “one might define the degree of productivity \textstyleChBold{P} of a particular morphological process as the proportion between the number of hapaxes of that type (n\textsubscript{1}) to the total number of tokens N” for that particular affix; a definition which is based on {Baayen’s (1992: 115)} formula P~=~n\textsubscript{1}/N.
\end{styleIvptafter}

\begin{styleIvvptafter}
For the present study, however, it remains unclear to what extent the attested hapaxes are useful in determining productivity. That is, the limited size of the corpus makes it difficult to verify which hapaxes are neologisms in Papuan Malay and which ones merely reflect the limited size of the corpus. Moreover, the literature does not mention thresholds which would allow interpreting a calculated P value in terms of the degree of productivity of a given morphological pattern. For the interested reader, however, the number of hapaxes and their respective P values for each affix are given in footnotes throughout this chapter.
\end{styleIvvptafter}

\begin{itemize}
\item \begin{styleOvNvwnext}
\label{bkm:Ref364758123}Form-function relationship between the derivation and its base
\end{styleOvNvwnext}\end{itemize}
\begin{styleIIndented}
Typical derivational processes include nominalization, verbalization, or class-preserving valency-changing operations, among others. In each case, the derivational process “results in the creation of a new word with a new meaning”, as {\citet[35]{Aikhenvald2007}} points out.
\end{styleIIndented}

\begin{styleIvptafter}
Following {Booij (2007: 240, 323)}, one “necessary” albeit not “sufficient” condition for the productivity of such derivational processes is their transparency, which is defined as “the presence of a systematic form-meaning correspondence in a morphologically complex form”. Therefore, if the form-function relationship between the affixed lexemes and their base is transparent, this is taken as evidence that a given affixation process is more productive. If, by contrast, this relationship is opaque, this is considered evidence that the process is less productive.
\end{styleIvptafter}

\begin{styleIvvptafter}
For the present study, pairs of words in which the affixed words and their respective bases have the same semantics are not taken as parts of a larger derivational paradigm. Instead these sets are taken as pairs of words belonging to different speech varieties, namely Papuan Malay and Indonesian. This conclusion is based on the fact that, in general, non-standard varieties of Malay “have lost most or all of this system of affixation”, whereas “Standard Malay exhibits a rich system of affixation” {\citep[20]{Paauw2009}}. Hence, for pairs of words with the same semantics, the unaffixed base words are taken to be the native Papuan Malay lexemes, whereas the affixed words are taken to be code-switches with the corresponding Indonesian lexemes.
\end{styleIvvptafter}

\begin{itemize}
\item \begin{styleOvNvwnext}
Alternative strategies
\end{styleOvNvwnext}\end{itemize}
\begin{styleIvI}
If speakers employ alternative strategies that do not involve affixation and that express the same meanings as the affixed forms, these alternative strategies are taken as evidence that the affixation process is less productive.
\end{styleIvI}

\begin{itemize}
\item \begin{styleOvNvwnext}
Formally complex words with non-compositional semantics
\end{styleOvNvwnext}\end{itemize}
\begin{styleIvI}
Affixed lexemes for which there is no corresponding base have lost their status as complex words. They are so-called “formally complex words” {\citep[17]{Booij2007}}. Such a word “behaves as a complex word although there is no corresponding semantic complexity” {(2007: 313)}. A high number of formally complex words are taken as evidence that the affixation process is less productive. Their non-compositional semantics suggest that these lexemes are either lexicalized forms or code-switches with Indonesian. (For each affix, the number of formally complex words is given with a few examples. Given, however, that they have lost their status as complex words, these items are not further discussed.)
\end{styleIvI}

\begin{itemize}
\item \begin{styleOvNvwnext}
\label{bkm:Ref346541608}Status of the affixed lexemes as part of the Papuan Malay lexicon or as code-switches with Indonesian
\end{styleOvNvwnext}\end{itemize}
\begin{styleIIndented}
If a large number of affixed lexemes are not part of the Papuan Malay lexicon but code-switches with Indonesian, this is taken as evidence that the derivation process for a given affix is less productive.
\end{styleIIndented}

\begin{styleIvptafter}
Sources such as {\citet{Jones2007}}, or {\citet{Tadmor2009a}} allow the identification of foreign, non-Malay loan words in the corpus. They do not, however, allow identifying code-switches with Indonesian. Hence, an alternative approach was deemed necessary to explore whether the affixed lexemes are part of the Papuan Malay lexicon or constitute code-switches with Indonesian.
\end{styleIvptafter}

\begin{styleIvptafter}
All 533 attested affixed lexemes were discussed with a Papuan Malay consultant who has a high level of language awareness, both with respect to Papuan Malay and to Indonesian. Based on his knowledge of both languages, the consultant classified the affixed lexemes as “Papuan Malay” or “borrowings from Indonesian”. The statement that a lexeme is considered to be Papuan Malay does not imply, however, that the respective lexeme does not exist in other Malay varieties as well. Across Southeast Asia, all Malay varieties have large sets of shared lexical items; this also applies to Papuan Malay, the other eastern Malay varieties and also to Indonesian.
\end{styleIvptafter}

\begin{styleIviptafter}
While the consultant’s tentative classification is subjective and not necessarily representative, it provides one more piece of evidence as to the potential productivity of the attested affixes. In Table  ‎3 .1 to Table  ‎3 .24, these alleged borrowings or code-switches with (Standard) Indonesian, are underlined.
\end{styleIviptafter}

%\setcounter{itemize}{0}
\begin{itemize}
\item \begin{styleOvNvwnext}
\label{bkm:Ref346712254}Language external factors: Variables of the communicative event
\end{styleOvNvwnext}\end{itemize}

The affixes were examined as to whether they are employed without sociolinguistic restrictions or whether their use is conditioned by variables of the speech situation in terms of {Fishman’s (1965: 86)} “domains of language choice”. The main factors which influence language choices are (1) the topics discussed, (2) the relationships between the interlocutors, and (3) the locations where the communication takes place {(1965: 67, 75)}. Speaker education levels are a fourth pertinent factor.



If the use of the affixes seems to be conditioned by language external factors, this is taken as evidence that the affixation process is less productive. For the present study, the pertinent “domains of language choice” are (a) the topics, (b) speaker education levels, and (c) the relationships between the interlocutors, all of which are discussed in the following. The locations of communication were not considered pertinent domains since all recorded conversations took place in the same informal setting of the home. (For details on the sociolinguistic profile of Papuan Malay, see §1.5.)
\end{styleBodyvvafter}

%\setcounter{itemize}{0}
\begin{itemize}
\item \begin{styleOvNvwnext}
Speaker education levels
\end{styleOvNvwnext}\end{itemize}
\begin{styleIvI}
In West Papua, as is typical of diglossic situations, the \textsc{high} variety Indonesian is acquired in school. Given their amount of access to the \textsc{high} variety, better-educated speakers are more likely to display language behaviors influenced by the \textsc{high} variety Indonesian than less-educated speakers. Therefore, if better-educated speakers employ a particular affix considerably more often than less-educated ones, this is taken as evidence that the affixed lexemes are not the result of a productive process but that they constitute code-switches with Indonesian. (See also Factor 1 ‘Speaker education levels’ in §1.5.1.)
\end{styleIvI}

\begin{itemize}
\item \begin{styleOvNvwnext}
Topics
\end{styleOvNvwnext}\end{itemize}
\begin{styleIvI}
Following {\citet[71]{Fishman1965}}, the topics under discussion may also bring “another language to the fore” as “certain topics are somehow handled better in one language than in another”. This notion of topical regulation suggests that Papuan Malay speakers consider Indonesian, and not Papuan Malay, the appropriate language to use when discussing \textsc{high} topics associated with formal domains such as politics, education, or religion. Therefore, if Papuan Malay speakers use a particular affix much more often when discussing \textsc{high} topics than when discussing casual daily-life issues (\textsc{low} topics), this is taken as evidence that the affixed lexemes are code-switches with Indonesian. This applies especially to less-educated Papuans, as better-educated Papuans already display a general tendency to include Indonesian features when speaking Papuan Malay, although this tendency is more pronounced when the latter discuss high topics. (See also Factor 2 ‘Topical regulation’ in §1.5.1.)
\end{styleIvI}

\begin{itemize}
\item \begin{styleOvNvwnext}
Relationships between interlocutors
\end{styleOvNvwnext}\end{itemize}
\begin{styleIiI}
Given the diglossic distribution of Papuan Malay and Indonesian, it is expected that the language behavior of Papuans shows influences from the \textsc{high} variety Indonesian when they interact with fellow-Papuans of higher status or with group outsiders. As discussed under Factor 3 ‘Relationships between interlocutors’ in §1.5.1, the use of features from the \textsc{high} variety serves to signal social inequality, distance, and formality. Therefore, if speakers use a given affix much more often when conversing with interlocutors of higher status or with group outsiders than when interacting with peers, this is taken as evidence that the affixed lexemes are code-switches with Indonesian. Again, this applies especially to less-educated Papuans, given that better-educated Papuans already show a general tendency to “dress-up” their Papuan Malay with Indonesian features, although this tendency is more marked when the latter interact with group outsiders, such as the author. (See also Factor 3 ‘Relationships between interlocutors’ in §1.5.1.)
\end{styleIiI}


In examining the attested affixes and affixed lexemes as outlined above, none of the factors was taken in isolation. Instead, the findings pertaining to all nine factors were taken together as an indication of the degree of productivity for the affix in question. The results of this multifaceted investigation indicate that in Papuan Malay:


\begin{itemize}
\item \begin{styleIIndented}
Prefix \textscItalBold{ter\-} ‘\textsc{acl}’ and suffix \-\textitbf{ang} ‘\textsc{pat}’ are somewhat productive;
\end{styleIIndented}\item \begin{styleIIndented}
Prefix \textscItalBold{pe(n)\-} ‘\textsc{ag}’ is, at best, marginally productive; and
\end{styleIIndented}\end{itemize}
\begin{itemize}
\item \begin{styleIvI}
Prefix \textscItalBold{ber\-} ‘\textsc{vblz}’, suffix \textitbf{\-nya} ‘\textsc{3possr}’, and circumfix \textitbf{ke}\-/\-\textitbf{ang} ‘\textsc{nmlz}’ are unproductive.
\end{styleIvI}\end{itemize}

The unproductive derivations are considered to be lexicalized forms borrowed into the language or code-switches with Indonesian; in the examples, however, no attempt is made to distinguish the two.



In the following, the six affixes are discussed in detail in terms of the factors outlined above: \textscItalBold{ter\-} in §3.1.2, \-\textitbf{ang} in §3.1.3, \textscItalBold{pe(n)\-} in §3.1.4, \textscItalBold{ber\-} in §3.1.5, \textitbf{\-nya} in §3.1.6, and \textitbf{ke}\-/\-\textitbf{ang} in §3.1.7. For the three somewhat productive affixes (\textscItalBold{ter\-}, \-\textitbf{ang}, and \textscItalBold{pe(n)\-}) the mentioned variables of the communicative event are investigated in detail within the respective sections. For the remaining three affixes (\textscItalBold{ber\-}, \textitbf{\-nya}, and \textitbf{ke}\-/\-\textitbf{ang}) the variables of the communicative event are summarily discussed in §3.1.8. The main points on affixation are summarized in §3.3.
\end{styleBodyvxvafter}

\subsection{Prefix \textscItalBold{ter\-} ‘\textsc{acl}’}
\label{bkm:Ref439089897}\label{bkm:Ref374451713}\label{bkm:Ref341890821}
Affixation with \textscItalBold{ter\-} ‘\textsc{acl}’ derives monovalent verbs from verbal bases. The derived verbs denote accidental or unintentional actions or events, as shown in (0). This derivation process appears to be somewhat productive in Papuan Malay, as discussed below.
\end{styleBodyxafter}

\begin{tabular}{llllllllllllll}
\lsptoprule
\label{bkm:Ref346628910}
\gll {bos} {pagi} {\multicolumn{2}{l}{su}} {\multicolumn{2}{l}{br\bluebold{–}angkat}} {\multicolumn{2}{l}{ke}} {\multicolumn{2}{l}{Sarmi}} {\multicolumn{2}{l}{begini}} {adu}\\ %
& boss & morning & \multicolumn{2}{l}{already} & \multicolumn{2}{l}{\textsc{vblz}\bluebold{–}leave} & \multicolumn{2}{l}{to} & \multicolumn{2}{l}{Sarmi} & \multicolumn{2}{l}{like.this} & oh.no!\\
& \multicolumn{3}{l}{sial–ang} & \multicolumn{2}{l}{\bluebold{ter–paksa}} & \multicolumn{2}{l}{tong} & \multicolumn{2}{l}{dua} & \multicolumn{2}{l}{jalang} & \multicolumn{2}{l}{kaki}\\
& \multicolumn{3}{l}{be.unfortunate–\textsc{pat}} & \multicolumn{2}{l}{\textsc{acl}–force} & \multicolumn{2}{l}{\textsc{1pl}} & \multicolumn{2}{l}{two} & \multicolumn{2}{l}{walk} & \multicolumn{2}{l}{foot}\\
\lspbottomrule
\end{tabular}
\ea
\glt 
‘as the boss had already left for Sarmi in the morning, oh no, damn it!, the two of us were \bluebold{forced} to walk on foot’ \textstyleExampleSource{[080921-002-Cv.0001]}
\z


Prefix \textscItalBold{ter\-} is a reflex of Proto-Malayic *\textitbf{t}\textscItalBold{a}\textitbf{r}\-, which, following {\citet[155]{Adelaar1992}}, “contributed the notion of unintentionality or feasibility to the VTR or VDI to which it was affixed”. In Standard Malay, “\textitbf{tər}\- denotes an ‘accidental’ state, process or action” when affixed to bivalent bases and “a superlative degree” when affixed to monovalent bases {(1992: 150–151)}. In eastern Malay varieties, the prefix also denotes accidental or unintentional actions, or events that happened unexpectedly or unintentionally. These productive uses of the prefix are attested for Ambon Malay {(van Minde 1997: 98)}, Banda Malay {\citep[250]{Paauw2009}}, Kupang Malay {\citep[46]{Steinhauer1983}}, Larantuka Malay {\citep[256]{Paauw2009}}, Manado Malay {(Stoel }{2005: 22)}, and North Moluccan / Ternate Malay ({Taylor 1983: 18};\footnote{\\
\\
\\
\\
\\
\\
\\
While {\citet[18]{Taylor1983}} considers the prefix to be productive, {\citet[4]{Voorhoeve1983}} believes that it is unproductive.\\
\\
\\
\\
\\
\\
\\
\\
} {Litamahuputty 2012: 133}).



The corpus includes 43 monovalent verbs (167 tokens) prefixed with \textscItalBold{ter\-}:\footnote{\\
\\
\\
\\
\\
\\
\\
The 43 verbs include 21 hapaxes (P=0.1257); the 38 bivalent verbs include 17 hapaxes (P=0.1111); the five monovalent verbs include four hapaxes (P=0.2857).\\
\\
\\
\\
\\
\\
\\
\\
}
\end{styleBodyvvafter}

%\setcounter{itemize}{0}
\begin{itemize}
\item \begin{styleIIndented}
Verbs with bivalent bases (38 items with 153 tokens)
\end{styleIIndented}\item \begin{styleIvI}
Verbs with monovalent bases (five items with 14 tokens)
\end{styleIvI}\end{itemize}

The corpus also contains ten formally complex words with non-compositional semantics, such as \textitbf{tertawa} ‘laugh’, \textitbf{tergrak} ‘be moved’, or \textitbf{trapung} ‘be drifting’.\footnote{\\
\\
\\
\\
\\
\\
\\
The historical roots \textitbf{tawa}, \textitbf{grak}, or \textitbf{apung} do not exist in Papuan Malay.\\
\\
\\
\\
\\
\\
\\
\\
}



Before discussing \textscItalBold{ter\-}affixation of bivalent bases in §3.1.2.2 and of monovalent bases in §3.1.2.3, the allomorphy of \textscItalBold{ter\-} is examined in §3.1.2.1. Variables of the communicative event that may impact the use of \textscItalBold{ter\-} are explored in §3.1.2.4. The main points on prefix \textscItalBold{ter\-} are summarized and evaluated in §3.1.2.5.
\end{styleBodyvxvafter}

\paragraph[Allomorphy of ter\-]{Allomorphy of \textscItalBold{ter\-}}
\label{bkm:Ref438905998}\label{bkm:Ref354566119}\label{bkm:Ref337023244}
Prefix \textscItalBold{ter\-} has two allomorphs, \textitbf{ter}\textitbf{\-} and \textitbf{ta}\textitbf{\-}. The allomorphs are not governed by phonological processes.



The form \textitbf{ter}\textitbf{\-}, in turn, has three allomorphs that are the effect of, what {Booij}{ (2007: 75)} calls “morphologically conditioned phonological rules”. More specifically, the three allomorphs are conditioned by the word-initial segment of the base word, as shown in Table  ‎3 .1: /\textstyleChCharisSIL{tɛr\-}/, /\textstyleChCharisSIL{tɛ\-}/, and /\textstyleChCharisSIL{tr\-}/. Most commonly, \textitbf{ter}\- is realized as /\textstyleChCharisSIL{tɛr\-}/. With onset rhotic /\textstyleChCharisSIL{r}/, however, it is realized as /\textstyleChCharisSIL{tɛ\-}/. With onset vowels, the prefix is usually realized as /\textstyleChCharisSIL{tr\-}/.


\begin{stylecaption}
\label{bkm:Ref357331884}Table ‎3.\stepcounter{Table}{\theTable}:  Realizations of allomorph \textitbf{ter}\-
\end{stylecaption}

\tablehead{
 \textitbf{ter}\-base & Orthogr. & \arraybslash Gloss\\
}
\begin{tabular}{lll}
\lsptoprule
/\textstyleChCharisSIL{tɛr}–\textstyleChCharisSIL{pukul}/ & \textitbf{terpukul} & ‘be beaten’\\
/\textstyleChCharisSIL{tɛ}–\textstyleChCharisSIL{rɛndam}/ & \textitbf{terendam} & ‘be soaked’\\
/\textstyleChCharisSIL{tr}–a\textstyleChCharisSIL{ŋkat}/ & \textitbf{trangkat} & ‘be lifted’\\
\lspbottomrule
\end{tabular}

Allomorph \textitbf{ta}\- is used in about one third of the affixed items; that is, 17 items with a total of 41 \textitbf{ta}\- tokens, listed in Table  ‎3 .2. Some of the derived items are alternatively realized with allomorph \textitbf{ter}\-. Hence, for each item the frequencies for \textitbf{ta}\- and for \textitbf{ter}\- are given.\footnote{\\
\\
\\
\\
\\
\\
\\
In addition, the 2,459-item word list (see Chapter 2) contains five items realized with /\textstyleChCharisSILviiivpt{ta\-}/ rather than with /\textstyleChCharisSILviiivpt{tɛr\-}/: /\textstyleChCharisSILviiivpt{tabla}/ ‘be cracked open’, /\textstyleChCharisSILviiivpt{takumpul}/ ‘be gathered’, /\textstyleChCharisSILviiivpt{takupas}/ ‘be peeled’, /\textstyleChCharisSILviiivpt{tamasuk}/ ‘be included’, and /\textstyleChCharisSILviiivpt{tatutup}/ ‘be closed’. In the corpus these items are realized with /\textstyleChCharisSILviiivpt{tɛr\-}/. Further, the word list also includes three items realized with /\textstyleChCharisSILviiivpt{tɛr\-}/ whereas in the corpus these items are most commonly realized with /\textstyleChCharisSILviiivpt{ta\-}/: \textitbf{talempar} ‘be thrown’, \textitbf{talipat} ‘be folded’, and \textitbf{tarangkat} ‘be lifted up’.\\
\\
\\
\\
\\
\\
\\
\\
} If in a greater number of tokens the prefix is realized with /\textstyleChCharisSIL{ta\-}/ than with /\textstyleChCharisSIL{tɛr\-}/, then its orthographic representation is \textitbf{ta}\- as in \textitbf{tagoyang} ‘be shaken’. If both realizations occur with the same frequency, then the orthographic representation follows its realization in the recorded texts, as in \textitbf{terlepas} ‘be loose’.


\begin{stylecaption}
\label{bkm:Ref357331885}Table ‎3.\stepcounter{Table}{\theTable}:  Realizations of allomorph \textitbf{ta}\-
\end{stylecaption}

\tablehead{
 \textitbf{ta}\-base & Orthogr. & Gloss & \textitbf{ta}\- \# & \arraybslash \textitbf{ter}\- \#\\
}
\begin{tabular}{lllll}
\lsptoprule
/\textstyleChCharisSIL{ta}–\textstyleChCharisSIL{g}ɔ\textstyleChCharisSIL{jaŋ}/ & \textitbf{tagoyang} & ‘be shaken’ & \raggedleft 9 & \raggedleft\arraybslash 0\\
/\textstyleChCharisSIL{ta}–\textstyleChCharisSIL{putar}/ & \textitbf{taputar} & ‘be turned around’ & \raggedleft 7 & \raggedleft\arraybslash 2\\
/\textstyleChCharisSIL{ta}–\textstyleChCharisSIL{lipat}/ & \textitbf{talipat} & ‘be folded’ & \raggedleft 6 & \raggedleft\arraybslash 1\\
/\textstyleChCharisSIL{ta}–\textstyleChCharisSIL{lɛmpar}/ & \textitbf{talempar} & ‘be thrown’ & \raggedleft 4 & \raggedleft\arraybslash 1\\
/\textstyleChCharisSIL{ta}–\textstyleChCharisSIL{guliŋ}/ & \textitbf{taguling} & ‘be rolled over’ & \raggedleft 3 & \raggedleft\arraybslash 0\\
/\textstyleChCharisSIL{ta}–\textstyleChCharisSIL{gant}ɔ\textstyleChCharisSIL{ŋ}/ & \textitbf{tergantong} & ‘be dependent’ & \raggedleft 1 & \raggedleft\arraybslash 6\\
/\textstyleChCharisSIL{ta}–\textstyleChCharisSIL{lɛpas}/ & \textitbf{terlepas} & ‘be loose’ & \raggedleft 1 & \raggedleft\arraybslash 1\\
/\textstyleChCharisSIL{ta}–\textstyleChCharisSIL{balik}/ & \textitbf{tabalik} & ‘be turned upside down’ & \raggedleft 1 & \raggedleft\arraybslash 0\\
/\textstyleChCharisSIL{ta}–\textstyleChCharisSIL{bantiŋ}/ & \textitbf{tabanting} & ‘be tossed around’ & \raggedleft 1 & \raggedleft\arraybslash 0\\
/\textstyleChCharisSIL{ta}–\textstyleChCharisSIL{tʃukur}/ & \textitbf{tacukur} & ‘be scalped’ & \raggedleft 1 & \raggedleft\arraybslash 0\\
/\textstyleChCharisSIL{ta}–\textstyleChCharisSIL{gait}/ & \textitbf{tagait} & ‘be hooked & \raggedleft 1 & \raggedleft\arraybslash 0\\
/\textstyleChCharisSIL{ta}–\textstyleChCharisSIL{hambur}/ & \textitbf{tahambur} & ‘be scattered about’ & \raggedleft 1 & \raggedleft\arraybslash 0\\
/\textstyleChCharisSIL{ta}–\textstyleChCharisSIL{kantʃiŋ}/ & \textitbf{takancing} & ‘be locked’ & \raggedleft 1 & \raggedleft\arraybslash 0\\
/\textstyleChCharisSIL{ta}–\textstyleChCharisSIL{lɛm}/ & \textitbf{talem} & ‘be glued’ & \raggedleft 1 & \raggedleft\arraybslash 0\\
/\textstyleChCharisSIL{ta}–\textstyleChCharisSIL{sala}/ & \textitbf{tasala} & ‘be mistaken’ & \raggedleft 1 & \raggedleft\arraybslash 0\\
/\textstyleChCharisSIL{ta}–\textstyleChCharisSIL{tikam}/ & \textitbf{tatikam} & ‘be stabbed’ & \raggedleft 1 & \raggedleft\arraybslash 0\\
/\textstyleChCharisSIL{ta}–\textstyleChCharisSIL{t}ɔ\textstyleChCharisSIL{ŋkat}/ & \textitbf{tatongkat} & ‘be beaten’ & \raggedleft 1 & \raggedleft\arraybslash 0\\
\lspbottomrule
\end{tabular}
\begin{styleNiixxptafterviiibefore}
In realizing the prefix most commonly as \textitbf{ter}\- rather than as \textitbf{ta}\-, Papuan Malay differs from other eastern Malay varieties such as Ambon Malay {(van Minde 1997: 98)}, Banda Malay {\citep[250]{Paauw2009}}, Kupang Malay {\citep[46]{Steinhauer1983}}, Manado Malay {\citep[22]{Stoel2005}}, and North Moluccan / Ternate Malay ({Taylor 1983: 18}{;} {Voorhoeve 1983: 4}{;} {Litamahuputty 2012: 133}). In these varieties the prefix is always realized as \textitbf{ta}\-. Instead, the \textscItalBold{ter\-}prefixed items have more resemblance with the corresponding items in Indonesian, where the prefix is realized as \textitbf{ter}\-. In addition, in Larantuka Malay the prefix is also realized as \textitbf{tə(r)}\- {\citep[253]{Paauw2009}}. The different behavior of Papuan Malay \textscItalBold{ter\-} supports the conclusion put forward in §1.8 that the history of Papuan Malay is different from that of the other eastern Malay varieties.
\end{styleNiixxptafterviiibefore}

\paragraph[Prefixed items derived from bivalent verbal bases]{Prefixed items derived from bivalent verbal bases}
\label{bkm:Ref336339154}
The corpus contains 38 \textscItalBold{ter\-}prefixed lexemes (with 153 tokens) with bivalent verbal base words (BW), as listed in Table  ‎3 .3. The affixation derives monovalent verbs with non-agent arguments through a valency-changing operation, in which \textscItalBold{ter\-} removes agent arguments. All but one of the derived lexemes are low frequency words (37 lexemes, attested with less than 20 tokens). Besides, the token frequencies for the respective bases are (much) higher for most of the derived words (29 lexemes).


\begin{stylecaption}
\label{bkm:Ref357331886}Table ‎3.\stepcounter{Table}{\theTable}:  Affixation with \textscItalBold{ter\-} of bivalent verbal bases\footnote{\\
\\
\\
\\
\\
\\
\\
As mentioned in language internal factor (f in §3.1.1 (p. \pageref{bkm:Ref346541608}), alleged borrowings or code-switches with (Standard) Indonesian are underlined.\\
\\
\\
\\
\\
\\
\\
\\
}
\end{stylecaption}

\tablehead{
 BW & Gloss & Item & Gloss & \textscItalBold{ter\-} \# & \arraybslash BW \#\\
}
\begin{tabular}{llllll}
\lsptoprule
\textitbf{jadi} & ‘become’ & \textitbf{terjadi} & ‘happen’ & \raggedleft 39 & \raggedleft\arraybslash 120\\
\textitbf{paksa} & ‘force’ & \textitbf{terpaksa} & ‘be forced’ & \raggedleft 10 & \raggedleft\arraybslash 10\\
\textitbf{masuk} & ‘enter’ & \textitbf{termasuk} & ‘be included’ & \raggedleft 9 & \raggedleft\arraybslash 261\\
\textitbf{putar} & ‘turn around’ & \textitbf{taputar} & ‘be turned around’ & \raggedleft 9 & \raggedleft\arraybslash 33\\
\textitbf{goyang} & ‘shake’ & \textitbf{tagoyang} & ‘be shaken’ & \raggedleft 9 & \raggedleft\arraybslash 10\\
\textitbf{gantong} & ‘suspend’ & \textitbf{tergantong} & ‘be dependent’ & \raggedleft 7 & \raggedleft\arraybslash 14\\
\textitbf{lipat} & ‘fold’ & \textitbf{talipat} & ‘be folded’ & \raggedleft 7 & \raggedleft\arraybslash 1\\
\textitbf{buka} & ‘open’ & \textitbf{terbuka} & ‘be opened’ & \raggedleft 6 & \raggedleft\arraybslash 1\\
\textitbf{angkat} & ‘lift’ & \textitbf{trangkat} & ‘be lifted’ & \raggedleft 5 & \raggedleft\arraybslash 81\\
\textitbf{lempar} & ‘throw’ & \textitbf{talempar} & ‘be thrown’ & \raggedleft 5 & \raggedleft\arraybslash 12\\
\textitbf{rendam} & ‘soak’ & \textitbf{terendam} & ‘be soaked’ & \raggedleft 5 & \raggedleft\arraybslash 1\\
\textitbf{pukul} & ‘beat’ & \textitbf{terpukul} & ‘be beaten’ & \raggedleft 4 & \raggedleft\arraybslash 59\\
\textitbf{bakar} & ‘burn’ & \textitbf{terbakar} & ‘be burnt’ & \raggedleft 3 & \raggedleft\arraybslash 55\\
\textitbf{guling} & ‘roll over’ & \textitbf{taguling} & ‘be rolled over’ & \raggedleft 3 & \raggedleft\arraybslash 2\\
\textitbf{tutup} & ‘close’ & \textitbf{tertutup} & ‘be closed’ & \raggedleft 3 & \raggedleft\arraybslash 53\\
\textitbf{bagi} & ‘divide’ & \textitbf{terbagi} & ‘be split up’ & \raggedleft 2 & \raggedleft\arraybslash 66\\
\textitbf{tarik} & ‘pull’ & \textitbf{tertarik} & ‘be pulled’ & \raggedleft 2 & \raggedleft\arraybslash 32\\
\textitbf{lepas} & ‘free’ & \textitbf{talepas} & ‘be loose’ & \raggedleft 2 & \raggedleft\arraybslash 23\\
\textitbf{kumpul} & ‘gather’ & \textitbf{terkumpul} & ‘be collected’ & \raggedleft 2 & \raggedleft\arraybslash 16\\
\textitbf{tolak} & ‘reject’ & \textitbf{tertolak} & ‘be rejected’ & \raggedleft 2 & \raggedleft\arraybslash 11\\
\textitbf{kupas} & ‘peel’ & \textitbf{terkupas} & ‘be peeled’ & \raggedleft 2 & \raggedleft\arraybslash 1\\
\textitbf{buat} & ‘make’ & \textitbfUndl{terbuat} & ‘be made’ & \raggedleft 1 & \raggedleft\arraybslash 135\\
\textitbf{kenal} & ‘know’ & \textitbf{terkenal} & ‘be well-known’ & \raggedleft 1 & \raggedleft\arraybslash 57\\
\textitbf{balik} & ‘turn over’ & \textitbf{tabalik} & ‘be turned over’ & \raggedleft 1 & \raggedleft\arraybslash 37\\
\textitbf{ganggu} & ‘disturb’ & \textitbf{terganggu} & ‘be disturbed’ & \raggedleft 1 & \raggedleft\arraybslash 18\\
\textitbf{bla} & ‘split’ & \textitbf{terbla} & ‘be split’ & \raggedleft 1 & \raggedleft\arraybslash 13\\
\textitbf{pengaru} & ‘influence’ & \textitbfUndl{terpengaru} & ‘be affected’ & \raggedleft 1 & \raggedleft\arraybslash 7\\
\textitbf{banting} & ‘throw’ & \textitbf{tabanting} & ‘be tossed around’ & \raggedleft 1 & \raggedleft\arraybslash 6\\
\textitbf{tukar} & ‘exchange’ & \textitbf{tertukar} & ‘get changed’ & \raggedleft 1 & \raggedleft\arraybslash 6\\
\textitbf{tongkat} & ‘cane’ & \textitbf{tatongkat} & ‘be beaten up’ & \raggedleft 1 & \raggedleft\arraybslash 5\\
\textitbf{singgung} & ‘offend’ & \textitbf{tersinggung} & ‘be offended’ & \raggedleft 1 & \raggedleft\arraybslash 3\\
\textitbf{cinta} & ‘love’ & \textitbfUndl{tercinta} & ‘be beloved’ & \raggedleft 1 & \raggedleft\arraybslash 3\\
\textitbf{cukur} & ‘flatten’ & \textitbf{tacukur} & ‘be scalped’ & \raggedleft 1 & \raggedleft\arraybslash 2\\
\textitbf{hambur} & ‘scatter’ & \textitbf{tahambur} & ‘be scattered about’ & \raggedleft 1 & \raggedleft\arraybslash 1\\
\textitbf{wesel} & ‘transfer’ & \textitbfUndl{terwesel} & ‘be transferred’ & \raggedleft 1 & \raggedleft\arraybslash 2\\
\textitbf{tikam} & ‘stab’ & \textitbf{tatikam} & ‘be stabbed’ & \raggedleft 1 & \raggedleft\arraybslash 2\\
\textitbf{kancing} & ‘lock’ & \textitbf{takancing} & ‘be locked’ & \raggedleft 1 & \raggedleft\arraybslash 0\\
\textitbf{lem} & ‘glue’ & \textitbfUndl{talem} & ‘be glued’ & \raggedleft 1 & \raggedleft\arraybslash 0\\
\lspbottomrule
\end{tabular}

The derived verbs denote accidental or unintentional states, processes, or actions. The term “accidental” covers “such concepts as involuntary, unmotivated, agentless, sudden, and unexpected action (or state resulting therefrom)”, employing {Adelaar’s (1992: 150)} terminology. Hence, \textscItalBold{ter\-} is glossed as ‘\textsc{acl}’ (‘accidental’). Two \textscItalBold{ter\-}prefixed items are given in context: \textitbf{tagoyang} ‘be shaken’ in (0) and \textitbf{tertutup} ‘be closed’ in (0). Both examples, together with the one in (0), illustrate how \textscItalBold{ter\-} decreases valency by “removing agent-like participants”.


\begin{styleExampleTitle}
Prefix \textscItalBold{ter\-}: Semantics of bivalent verbal bases and derived lexemes
\end{styleExampleTitle}

\begin{tabular}{llllllllll}
\lsptoprule
\label{bkm:Ref338854301}
\gll {de} {bilang,} {mama} {sa} {liat} {pohong} {ini} {de} {\bluebold{ta–goyang}}\\ %
& \textsc{3sg} & say & mother & \textsc{1sg} & see & tree & \textsc{d.prox} & \textsc{3sg} & \textsc{acl}–shake\\
\lspbottomrule
\end{tabular}
\ea
\glt 
‘she said, ‘mama, I saw this tree, it was \bluebold{shaking}’’ \textstyleExampleSource{[080917-008-NP.0031]}
\z

\begin{tabular}{llllllllll}
\lsptoprule
\label{bkm:Ref338854302}
\gll {…} {bapa} {Markus} {S.} {doseng} {satu} {de} {\bluebold{goyang}} {kepala}\\ %
&  & father & Markus & S. & lecturer & one & \textsc{3sg} & shake & head\\
\lspbottomrule
\end{tabular}
\ea
\glt 
‘… Mr. Markus S., a certain lecturer, he \bluebold{shook} (his) head’ \textstyleExampleSource{[080917-010-CvEx.0194]}
\z

\begin{tabular}{lllllllll}
\lsptoprule
\label{bkm:Ref338854303}
\gll {kalo} {ko} {\bluebold{tutup}} {pintu} {berkat} {juga} {\bluebold{ter–tutup}} {…}\\ %
& if & \textsc{2sg} & close & door & blessing & also & \textsc{acl}–close & \\
\lspbottomrule
\end{tabular}
\ea
\glt 
‘if you \bluebold{close} the door (of your house), the blessing is also \bluebold{closed off} [(because) guests cannot come into (your) house]’ \textstyleExampleSource{[081110-008-CvNP.0096]}
\z


Of the 38 \textscItalBold{ter\-}prefixed bivalent verbs, one Papuan Malay consultant classified four as borrowings from Standard Indonesian (SI-borrowings) (see language internal factor (f in §3.1.1, p. \pageref{bkm:Ref346541608}), namely \textitbf{terbuat} ‘be made’, \textitbf{terpengaru} ‘be influenced’, \textitbf{tercinta} ‘be beloved’, and \textitbf{terwesel} ‘be transferred’ (in Table  ‎3 .3 these items are underlined). The same consultant also stated that Papuan Malay speakers usually employ the respective bases rather than the prefixed forms. One such contrastive set of examples is given in (0) and (0). Instead of using the prefixed form \textitbf{terpengaru} ‘be influenced’, as in (0), speakers more often employ the base \textitbf{pengaru} ‘influence’ in the sense of ‘be influenced’, as in (0).
\end{styleBodyxafter}

\begin{styleExampleTitle}
Prefix \textscItalBold{ter\-}: Use patterns of base words versus derived lexemes
\end{styleExampleTitle}

\begin{tabular}{lllllllll}
\lsptoprule
\label{bkm:Ref338854297}
\gll {…} {tapi} {de} {ana} {juga} {cepat} {ikut} {\bluebold{ter–pengaru}}\\ %
&  & but & \textsc{3sg} & child & also & be.fast & follow & \textsc{acl}–influence\\
\lspbottomrule
\end{tabular}
\ea
\glt 
‘… but he/she, a kid, also quickly follows (others) to \bluebold{be influenced}’ \textstyleExampleSource{[080917-010-CvEx.0001]}
\z

\begin{tabular}{llllllll}
\lsptoprule
\label{bkm:Ref338854298}
\gll {de} {su} {\bluebold{pengaru}} {dengang} {orang{\Tilde}orang} {yang} {minum}\\ %
& \textsc{3sg} & already & influence & with & \textsc{rdp}{\Tilde}person & \textsc{rel} & drink\\
\lspbottomrule
\end{tabular}
\ea
\glt
‘he has already \bluebold{been influenced} by people who drink’ \textstyleExampleSource{[080919-007-CvNP.0018]}
\end{styleFreeTranslEngxvpt}

\paragraph[Prefixed items derived from monovalent verbal bases]{Prefixed items derived from monovalent verbal bases}
\label{bkm:Ref336348236}
The corpus contains five \textscItalBold{ter\-}prefixed lexemes (with 14 tokens) with monovalent verbal bases, as listed in Table  ‎3 .4. Contrasting with the affixation of bivalent bases, \textscItalBold{ter\-}affixation of monovalent bases is not a valency-changing operation, nor does it derive verbs with non-agent arguments. Instead, \textscItalBold{ter\-} downplays the level of control of its arguments by deriving monovalent verbs which denote accidental or unintentional states or actions, such as \textitbf{terlambat} ‘be late’ or \textitbf{tersendiri} ‘be separate’. All five lexemes are low frequency words, attested with less than 20 tokens. Moreover, the token frequencies for the respective bases are (much) higher for four of the five derived words.


\begin{stylecaption}
\label{bkm:Ref336240072}Table ‎3.\stepcounter{Table}{\theTable}:  Affixation with \textscItalBold{ter\-} of monovalent verbal bases
\end{stylecaption}

\tablehead{
 BW & Gloss & Item & Gloss & \textscItalBold{ter\-} \# & \arraybslash BW \#\\
}
\begin{tabular}{llllll}
\lsptoprule
\textitbf{lambat} & ‘be slow’ & \textitbf{terlambat} & ‘be late’ & \raggedleft 10 & \raggedleft\arraybslash 3\\
\textitbf{sendiri} & ‘be alone’ & \textitbfUndl{tersendiri} & ‘be separate’ & \raggedleft 1 & \raggedleft\arraybslash 232\\
\textitbf{biasa} & ‘be used to’ & \textitbfUndl{terbiasa} & ‘be accustomed’ & \raggedleft 1 & \raggedleft\arraybslash 186\\
\textitbf{jatu} & ‘fall’ & \textitbf{terjatu} & ‘be dropped, fall’ & \raggedleft 1 & \raggedleft\arraybslash 64\\
\textitbf{sala} & ‘be wrong’ & \textitbf{tasala} & ‘be mistaken’ & \raggedleft 1 & \raggedleft\arraybslash 42\\
\lspbottomrule
\end{tabular}

Two items indicating uncontrolled and/or unexpected actions are given in context: \textitbf{terjatu} ‘be dropped, fall’ in (0) and \textitbf{terlambat} ‘be late’ in (0). Both examples, along with the example in (0), show that the verbal valency is not further decreased and that the derivation does not result in a loss of agentivity. That is, the referents of the derived verbs \textitbf{terjatu} ‘be dropped, fall’ and \textitbf{terlambat} ‘be late’ and the referents of the bases \textitbf{jatu} ‘fall’ and \textitbf{lambat} ‘be slow’, respectively, have the same semantic functions. With \textscItalBold{ter\-}prefixed verbs, however, the level of control the referents have is downplayed, as mentioned above.


\begin{tabular}{lllllllll}
\lsptoprule
\label{bkm:Ref338854304}
\gll {dia} {\bluebold{ter–jatu}} {de} {\bluebold{jatu}} {baru} {motor} {tindis} {dia}\\ %
& \textsc{3sg} & \textsc{acl}–fall & \textsc{3sg} & fall & and.then & motorbike & overlap & \textsc{3sg}\\
\lspbottomrule
\end{tabular}
\ea
\glt 
‘he \bluebold{fell (off unexpectedly)}, he \bluebold{fell} (off), and then the motorbike crushed him’ \textstyleExampleSource{[080923-010-CvNP.0012]}
\z

\begin{tabular}{llllllll}
\lsptoprule
\label{bkm:Ref338854305}
\gll {kaka} {tadi} {\bluebold{ter–lambat}} {karna} {lagi} {ada} {duka}\\ %
& oSb & earlier & \textsc{acl}–be.slow & because & again & exist & grief\\
\lspbottomrule
\end{tabular}
\ea
\glt 
‘a short while ago I (‘older brother’) was \bluebold{(unintentionally) late} because there was (still) mourning (going on)’ \textstyleExampleSource{[080918-001-CvNP.0003]}
\z

\begin{tabular}{lllllllll}
\lsptoprule
\label{bkm:Ref338854306}
\gll {kalo} {Niwerawar} {Aruswar} {nanti} {dia} {agak} {\bluebold{lambat}} {sedikit}\\ %
& if & Niwerawar & Aruswar & very.soon & \textsc{3sg} & rather & be.slow & few\\
\lspbottomrule
\end{tabular}
\ea
\glt
[About a street construction project:] ‘as for (the area of) Niwerawar (and) Aruswar, (there) it (the bulldozer) will be somewhat \bluebold{slow}’ \textstyleExampleSource{[081006-033-Cv.0051]}
\end{styleFreeTranslEngxvpt}

\paragraph[Variables of the communicative event]{Variables of the communicative event}
\label{bkm:Ref346647621}
To explore the issue of \textscItalBold{ter\-}productivity in Papuan Malay further, a domain analysis was conducted which focused on the variables of speaker education levels, topics, and role-relations (for details see ‘Language external factors’ in §3.1.1, p. \pageref{bkm:Ref346712254}). In all, 43 \textscItalBold{ter\-}prefixed items, totaling 167 tokens, were examined:


\begin{itemize}
\item \begin{styleIIndented}
38 prefixed items derived from bivalent verbal bases (153 tokens)
\end{styleIIndented}\item \begin{styleIvI}
Five prefixed items derived from monovalent verbal bases (14 tokens)
\end{styleIvI}\end{itemize}

For the 43 prefixed lexemes, most tokens (143/167 – 86\%) can be accounted for in terms of speaker education levels, topics, and/or role-relations. The remaining 24/167 tokens (14\%), however, cannot be explained in terms of these variables of the communicative event. These tokens occurred when less-educated speakers (\textsc{\-edc-spk}) conversed with fellow-Papuans of equally low social standing (\textsc{\-stat}) about \textsc{low} topics, that is, casual daily-life issues.\footnote{\\
\\
\\
\\
\\
\\
\\
As mentioned under Factor 3 ‘Relationships between interlocutors’ in §1.5.1 (p. \pageref{bkm:Ref376429590}), all of the recorded less-educated speakers belong to the group of Papuans with lower social status (\textsc{\-stat}), while the recorded Papuans with higher social status (\textsc{+stat}), such as teachers, government officials, or pastors, are all better educated.\\
\\
\\
\\
\\
\\
\\
\\
} (See Table  ‎3 .5 and Figure  ‎3 .6.)



If the prefixed items were the result of a productive affixation process, one would expect the percentage of tokens that cannot be explained in terms of speaker education levels, topics, and/or role-relations to be much higher than 14\%. Instead, most tokens (86\%) seem to be conditioned by these variables of the communicative event. These findings do not support the conclusion that the respective lexemes result from a productive affixation process. Instead, they appear to be code-switches with Indonesian.
\end{styleBodyvafter}


Table  ‎3 .5 and Figure  ‎3 .6 (p. \pageref{bkm:Ref439952307}, p. \pageref{bkm:Ref438910494}) present the token frequencies for \textscItalBold{ter\-}prefixed lexemes by speakers and topics/interlocutors. Before discussing the data in more detail, the layouts of Table  ‎3 .5 and Figure  ‎3 .6 are explained.
\end{styleBodyvvafter}

\begin{stylecaption}
\label{bkm:Ref439952307}\label{bkm:Ref346813643}Table ‎3.\stepcounter{Table}{\theTable}:  Token frequencies for \textscItalBold{ter\-}prefixed lexemes with bi- and monovalent verbal bases by speakers, topics, and interlocutors (43 items)
\end{stylecaption}

\tablehead{ & \multicolumn{4}{l}{ Topics (\textsc{top})} & \multicolumn{3}{l}{ Interlocutors (\textsc{ilct})} & \arraybslash Tokens\\
}
\begin{tabular}{lllllllll}
\lsptoprule
\multicolumn{9}{l}{Prefixed lexemes with bivalent bases (38 items)}\\
& \textsc{pol} & \textsc{edc} & \textsc{rel} & \textsc{l.t.} & \textsc{+stat} & \textsc{\-stat} & \textsc{outsd} & \arraybslash Total\\
\textsc{+edc-spk} & \raggedleft 6 & \raggedleft 10 & \raggedleft 10 & \raggedleft 15 & \raggedleft {}-{}-{}- & \raggedleft {}-{}-{}- & \raggedleft 9 & \raggedleft\arraybslash 50\\
\textsc{\-edc-spk} & \raggedleft 2 & \raggedleft 1 & \raggedleft 26 & \raggedleft {}-{}-{}- & \raggedleft 45 & \raggedleft \textstyleChBold{23} & \raggedleft 6 & \raggedleft\arraybslash 103\\
Subtotal & \raggedleft 8 & \raggedleft 11 & \raggedleft 36 & \raggedleft 15 & \raggedleft 45 & \raggedleft \textstyleChBold{23} & \raggedleft 15 & \raggedleft\arraybslash 153\\
\multicolumn{9}{l}{Prefixed lexemes with monovalent bases (5 items)}\\
& \textsc{pol} & \textsc{edc} & \textsc{rel} & \textsc{low} & \textsc{+stat} & \textsc{\-stat} & \textsc{outsd} & \arraybslash Total\\
\textsc{+edc-spk} & \raggedleft 0 & \raggedleft 0 & \raggedleft 1 & \raggedleft 4 & \raggedleft {}-{}-{}- & \raggedleft {}-{}-{}- & \raggedleft 0 & \raggedleft\arraybslash 5\\
\textsc{\-edc-spk} & \raggedleft 0 & \raggedleft 1 & \raggedleft 5 & \raggedleft {}-{}-{}- & \raggedleft 2 & \raggedleft \textstyleChBold{1} & \raggedleft 0 & \raggedleft\arraybslash 9\\
Subtotal & \raggedleft 0 & \raggedleft 1 & \raggedleft 6 & \raggedleft 4 & \raggedleft 2 & \raggedleft \textstyleChBold{1} & \raggedleft 0 & \raggedleft\arraybslash 14\\
\multicolumn{9}{l}{\textstyleChBold{TOTAL} (42 items)}\\
& \textsc{pol} & \textsc{edc} & \textsc{rel} & \textsc{low} & \textsc{+stat} & \textsc{\-stat} & \textsc{outsd} & \arraybslash Total\\
\textsc{+edc-spk} & \raggedleft 6 & \raggedleft 10 & \raggedleft 11 & \raggedleft 19 & \raggedleft {}-{}-{}- & \raggedleft {}-{}-{}- & \raggedleft 9 & \raggedleft\arraybslash 55\\
\textsc{\-edc-spk} & \raggedleft 2 & \raggedleft 2 & \raggedleft 31 & \raggedleft {}-{}-{}- & \raggedleft 47 & \raggedleft \textstyleChBold{24} & \raggedleft 6 & \raggedleft\arraybslash 112\\
\textstyleChBold{Total} & \raggedleft 8 & \raggedleft 12 & \raggedleft 42 & \raggedleft 19 & \raggedleft 47 & \raggedleft \textstyleChBold{24} & \raggedleft 15 & \raggedleft\arraybslash \textstyleChBold{167}\\
\lspbottomrule
\end{tabular}

Table  ‎3 .5 is divided into three major parts. The top part lists the token frequencies for prefixed lexemes with bivalent bases, while the middle part gives the frequencies for prefixed lexemes with monovalent bases. The bottom part gives the frequencies for all verbal bases. The layout of each of these parts represents the three variables of speaker education levels, topics, and role relations (this layout also applies to the tables and figures presented in §3.1.3.3, §3.1.4.4, and §3.1.8). The token frequencies according to the variable ‘Speaker education levels’ are given in the rows labeled “\textsc{+edc-spk”} and “\textsc{\-edc-spk”}, while the token frequencies according to the variables ‘Topics’ and ‘Role-relations’ are presented in the columns labeled “Topics (\textsc{top})” and “Interlocutors (\textsc{ilct})”, respectively. The token frequencies by speaker education levels are presented in two rows: the first row labeled “\textsc{+edc-spk}” gives the token frequencies for better-educated speakers while the second row labeled “\textsc{\-edc-spk”} lists the token frequencies for less-educated speakers. The token frequencies by topics are presented in the first four columns. The three columns headed \textsc{pol}, \textsc{edc}, and \textsc{rel} list the frequencies for tokens when speakers conversed about the \textsc{high} topics of politics, education, and religion, respectively. The column headed \textsc{low} lists the number of tokens produced during conversations about \textsc{low} topics, that is, casual daily-life issues. The token frequencies by role-relations are presented in the next three columns. The columns headed with \textsc{+stat},\textsc{ \-stat}, and \textsc{outsd} give the number of tokens produced during conversations with fellow-Papuans of higher social standing (\textsc{+stat}), fellow-Papuans of lower social standing (\textsc{\-stat}), and group outsiders (\textsc{outsd}), respectively.
\end{styleBodyaftervbefore}


The layout of Table  ‎3 .5 is based on four assumptions. First, when discussing \textsc{high} topics, the language behavior of Papuans is likely to show influences from Indonesian, regardless of their own education levels and also regardless of the social standing of their fellow-Papuan interlocutors. Therefore, these token frequencies are totaled in the respective ‘Topics’ cells and not broken down according to the social standing of their interlocutors. For \textscItalBold{ter\-}prefixed lexemes with bivalent bases, the respective token frequencies for better-educated speakers (\textsc{+edc-spk}) are as follows: 6 tokens for discussions about politics (\textsc{pol}), 10 about education (\textsc{edc}), and 10 tokens about religion (\textsc{rel}). For less-educated speakers (\textsc{\-edc-spk}) the respective frequencies are 2, 1, and 26 tokens. (See the left top part of Table  ‎3 .5).
\end{styleBodyvafter}


Second, when discussing \textsc{low} topics, the language behavior of better-educated speakers (\textsc{+edc-spk}) is presumably not affected by the social standing of their fellow-Papuan interlocutors, given that they already have the general tendency to “dress-up” their Papuan Malay with Indonesian features. Therefore, these token frequencies are totaled in the \textsc{low}{}-topic cell of the \textsc{+edc-spk} row. That is, in this total are included the token frequencies for interactions with interlocutors of equally high social standing (\textsc{+stat}) and with those of lower status (\textsc{\-stat}). The columns to the right of the \textsc{low}{}-topic column give the token frequencies according to the social status of the speakers’ interlocutors. However, given that for the better-educated speakers (\textsc{+edc-spk}), the total in \textsc{low}{}-topic cell includes both \textsc{+stat} and \textsc{\-stat} interlocutors, the respective cells for \textsc{+stat} and \textsc{\-stat} interlocutors are left empty. For \textscItalBold{ter\-}prefixed lexemes with bivalent bases, the respective token frequency is 15 (see the \textsc{low}{}-topic column in the top part of Table  ‎3 .5), while the \textsc{+stat} and \textsc{\-stat} cells to the right are left empty.
\end{styleBodyvafter}


Third, when discussing \textsc{low} topics, the language behavior of less-educated speakers (\textsc{\-edc-spk}) is likely to be affected by the status of their fellow-Papuan interlocutors. Therefore, these total token frequencies are not totaled in the \textsc{low}{}-topic cell of the \textsc{\-edc-spk} row. Instead the \textsc{low}{}-topic token frequencies are broken down according to the status of their fellow-Papuan interlocutors; hence, the respective \textsc{low}{}-topic cell is left empty. For \textscItalBold{ter\-}prefixed lexemes with bivalent bases, the token respective frequencies are 45 for \textsc{+stat} Papuan interlocutors and 23 for \textsc{\-stat} Papuan interlocutors (see the \textsc{+stat-} and \textsc{\-stat} -interlocutor columns in the top part of Table  ‎3 .5), while the \textsc{low}{}-topic cell to the left is left empty.
\end{styleBodyvafter}


Fourth, the language behavior of both better and less-educated speakers is likely to be affected when they converse with a non-Papuan outsider, regardless of the topic under discussion. Therefore, all tokens produced during conversations with an outsider, namely the author, are totaled in the \textsc{outsd} column of the \textsc{+edc-spk} and \textsc{\-edc-spk} rows. For \textscItalBold{ter\-}prefixed lexemes with bivalent bases, this token frequency is nine for better-educated speakers and six for less-educated speakers (see the \textsc{outsd}{}-interlocutor column in the top part of Table  ‎3 .5)
\end{styleBodyvafter}


Figure  ‎3 .6 gives a graphic representation of the data listed in Table  ‎3 .5. The horizontal category (X) axis presents the different categories according to which the token frequencies are listed, that is, the four topic categories and the three interlocutor categories. The vertical value (Y) axis gives the token totals for each of these categories, according to speaker education levels. The light (or blue) shaded columns denote the token frequencies for the better-educated speakers, while the dark (or red) shaded columns indicate the frequencies for the less-educated speakers.
\end{styleBodyvvafter}

\begin{styleFigure}
  
%%please move the includegraphics inside the {figure} environment
%%\includegraphics[width=\textwidth]{kluge-img2.jpg}
 
\end{styleFigure}

\begin{styleCapFigure}
\label{bkm:Ref438910494}\label{bkm:Ref417128575}Figure ‎3.\stepcounter{Figure}{\theFigure}:  Token frequencies for \textscItalBold{ter\-}prefixed lexemes with bi- and monovalent verbal bases by speakers, topics, and interlocutors
\end{styleCapFigure}


The data given in Table  ‎3 .5 and Figure  ‎3 .6 show that for the 43 \textscItalBold{ter\-}prefixed lexemes, most tokens (143/167 – 86\%) can be explained in terms of speaker education levels, topics, and/or role-relations between the speakers and their interlocutors; this total includes 130/153 tokens (85\%) with bivalent bases, and 13/14 tokens (93\%) with monovalent bases.
\end{styleBodyaftervbefore}


Only 55/167 tokens (33\%) were produced by better-educated speakers (\textsc{+edc-spk}) while most tokens (112/167 – 67\%) were produced by less-educated speakers (\textsc{\-edc-spk}). The \textsc{+edc-spk} produced half of their tokens (27/55 – 49\%) during discussions about \textsc{high} topics, that is, political, educational or religious affairs (\textsc{pol}, \textsc{edc} and \textsc{rel}, respectively). Another 19 tokens (35\%) occurred during conversations with fellow-Papuans (both \textsc{+stat} and \textsc{\-stat} speakers) about \textsc{low} topics. The remaining nine tokens (16\%) occurred while conversing with an outsider, namely the author (\textsc{outsd}).
\end{styleBodyvafter}


The \textsc{\-edc-spk} produced most of their tokens (47/112 – 42\%) while discussing \textsc{low} topics with \textsc{+stat} speakers (47 tokens). Another 35/112 tokens (31\%) were produced during discussions about \textsc{high} topics, while 6/112 tokens (5\%) occurred during conversations with the author. The remaining 24/112 tokens (21\%) occurred when \textsc{\-edc-spk} discussed \textsc{low} topics with \textsc{\-stat} Papuans, and therefore cannot be explained in terms of speaker education levels, topics, and/or role-relations. This total of 24 tokens refers to 14\% of all 167 \textscItalBold{ter\-}tokens, including 23/153 tokens (15\%) with bivalent bases and 1/14 tokens (7\%) with monovalent bases.\footnote{\\
\\
\\
\\
\\
\\
\\
As for the 21 hapaxes (17 with bivalent and four with monovalent bases), 18 appear to be conditioned by the variables of speaker education levels, topics, and/or role-relations, and therefore seem to be code-switches with Indonesian. This leaves only three hapaxes (with bivalent bases) that are unaccounted for in terms of language external factors and that might result from a productive derivation process. For three hapaxes P=0.0180 as opposed to P=0.1257 for 21 hapaxes (N=167).\\
\\
\\
\\
\\
\\
\\
\\
}
\end{styleBodyvxvafter}

\paragraph[Summary and conclusions]{Summary and conclusions}
\label{bkm:Ref346647622}
For most of the derived verbs with bivalent bases, the data suggest a productive form-function relationship between the derived lexemes and their bases. This conclusion is based on four observations: (1) the valency-decreasing or -reducing function of \textscItalBold{ter\-} of removing or downplaying agent-like participants, (2) the transparent form-function relationships between derived lexemes and bases, (3) the large number of low frequency words and small number of high frequency words, and (4) the relative token frequencies with most bases having higher frequencies than the affixed lexemes.



For the prefixed verbs with monovalent bases, the derivation process also seems to be productive, given (1) the transparent form-function relationships between derived lexemes and bases, (2) the comparatively large number of low frequency words and small number of high frequency words, and (3) the relative token frequencies with most bases having higher frequencies than the affixed lexemes. However, the low type frequency, with only five derived verbs, suggests that \textscItalBold{ter\-}prefixation of monovalent bases plays a minor role.
\end{styleBodyvafter}


As for the speech situations during which the derived lexemes occurred, a sizable number of verbs with bivalent bases cannot be explained in terms of pertinent variables of the communicative event. Most tokens, however, including those with bivalent bases, seem to be conditioned by the variables of speaker education levels, topics, and/or role-relations and therefore are best explained as code-switches with Indonesian.
\end{styleBodyvafter}


These findings suggest that in Papuan Malay \textscItalBold{ter\-}affixation is a productive process to derive monovalent verbs that denote accidental or unintentional actions. The degree of productivity appears to be limited, however, given that most of the attested tokens are best explained as code-switches with Indonesian.
\end{styleBodyvxvafter}

\subsection{Suffix \textitbf{\-an}\textitbf{g} ‘\textsc{pat}’}
\label{bkm:Ref374451716}\label{bkm:Ref374450294}\label{bkm:Ref347514351}
Affixation with \textitbf{\-ang} ‘\textsc{pat}’ typically derives nominals from verbal bases. The derived nouns denote the patient or result of the action, event, or state specified by the verbal base, as illustrated in (0). Some lexical items are also derived from nominal and quantifier bases. The derivation process seems to be productive in Papuan Malay to some degree, as discussed below.
\end{styleBodyxafter}

\begin{tabular}{lllll}
\lsptoprule
\label{bkm:Ref346630421}
\gll {\bluebold{pake–ang}} {itu} {basa} {smua}\\ %
& use–\textsc{pat} & \textsc{d.dist} & be.wet & all\\
\lspbottomrule
\end{tabular}
\ea
\glt 
‘all those \bluebold{clothes} were wet’ \textstyleExampleSource{[080917-008-NP.0139]}
\z


Suffix \textitbf{\-ang} is a reflex of Proto-Malayic *\textitbf{\-}\textscItalBold{a}\textitbf{n}, which “was a noun-forming suffix occurring on the basis of VTRs and denoting the goal or result of an act” {\citep[174]{Adelaar1992}}. In Standard Malay, when affixed to monovalent bases, the suffix designates “something that has the quality of” the monovalent base, while with transitive bases it denotes the “goal or result of an action, or place where the action takes place” or “the instrument” {(1992: 172–173)}. As for the eastern Malay varieties, the suffix is only mentioned for Ambon Malay. Also realized as \textitbf{\-ang}, it “refers to the object of the transitive verb or an instrument used in an act of V” {(van Minde 1997: 106)}. It is left unclear, however, whether and to what degree the Ambon Malay suffix is productive. These observations are again an indication of the distinct history of Papuan Malay vis-à-vis the other Malay varieties, discussed in §1.8. Moreover, the similarities between Papuan Malay and Ambon Malay reflect the link between both speech communities, also discussed in §1.8.
\end{styleBodyaftervbefore}


The corpus contains 84 nouns (441 tokens) suffixed with \textitbf{\-}\textitbf{ang}:\footnote{\\
\\
\\
\\
\\
\\
\\
The 84 nouns include 28 hapaxes (P=0.0635); the 69 nouns with verbal bases include 23 hapaxes (P=0.0571); the 15 nouns with nominal or quantifier bases include five hapaxes (P=0.1316).\\
\\
\\
\\
\\
\\
\\
\\
}
\end{styleBodyvvafter}

%\setcounter{itemize}{0}
\begin{itemize}
\item \begin{styleIIndented}
Nouns with verbal bases (69 items with 403 tokens)
\end{styleIIndented}\item \begin{styleIvI}
Nouns with nominal or quantifier bases (15 items with 38 tokens)
\end{styleIvI}\end{itemize}

The corpus also includes 28 formally complex words that have non-compositional semantics, such as \textitbf{kasiang} ‘pity’, \textitbf{lapangang} ‘field’, or \textitbf{grakang} ‘movement’.



Suffixed items with verbal bases are examined in §3.1.3.1, and those with nominal bases in §3.1.3.2. Variables of the communicative event that may impact the use of \textitbf{\-ang} are explored in §3.1.3.3. The main findings on suffix \textitbf{\-ang} are summarized and evaluated in §3.1.3.4.
\end{styleBodyvxvafter}

\paragraph[Suffixed items derived from verbal bases]{Suffixed items derived from verbal bases}
\label{bkm:Ref340228782}
The corpus contains 69 \textitbf{\-ang}{}-suffixed items (with 403 tokens) with verbal bases, including bases such as bivalent \textitbf{pake} ‘use’, monovalent dynamic \textitbf{jalang} ‘walk’, or monovalent stative \textitbf{dulu} ‘be prior’. Affixation with \textitbf{\-ang} typically derives nouns that denote the object of the action, event, or state indicated by the verbal base.



Derived words with token frequencies of five or more are listed in Table  ‎3 .6. Most of the affixed lexemes are low frequency words (63 lexemes, attested with less than 20 tokens). Moreover, the token frequencies for the respective bases are (much) higher for most of the derived words (64 lexemes). While all 69 derived lexemes are structurally nouns, three of them have other than nominal functions in their actual uses: \textitbf{jualang} ‘merchandise’, \textitbf{duluang} ‘be prior’, and \textitbf{latiang} ‘practice; illustrations are provided in (0) to (0).
\end{styleBodyvafter}


Seven of the 69 lexemes were tentatively classified as borrowings from Standard Indonesian (SI-borrowings) (for more details see language internal factor (f on p. \pageref{bkm:Ref346541608} in §3.1.1). As their token frequencies are four or less, they are not included in Table  ‎3 .6.
\end{styleBodyvvafter}

\begin{stylecaption}
\label{bkm:Ref340161865}Table ‎3.\stepcounter{Table}{\theTable}:  Affixation with \textitbf{\-ang} of verbal bases
\end{stylecaption}

\tablehead{
 BW & Gloss & Item & Gloss & \textitbf{\-ang} \# & \arraybslash BW \#\\
}
\begin{tabular}{llllll}
\lsptoprule
\textitbf{makang} & ‘eat’ & \textitbf{makangang} & ‘food’ & \raggedleft 57 & \raggedleft\arraybslash 414\\
\textitbf{pake} & ‘use’ & \textitbf{pakeang} & ‘clothes’ & \raggedleft 38 & \raggedleft\arraybslash 218\\
\textitbf{dulu} & ‘be prior’ & \textitbf{duluang} & ‘be prior to others’ & \raggedleft 29 & \raggedleft\arraybslash 351\\
\textitbf{bagi} & ‘divide’ & \textitbf{bagiang} & ‘part’ & \raggedleft 28 & \raggedleft\arraybslash 63\\
\textitbf{pikir} & ‘think’ & \textitbf{pikirang} & ‘thought’ & \raggedleft 23 & \raggedleft\arraybslash 102\\
\textitbf{uji} & ‘test’ & \textitbf{ujiang} & ‘examination’ & \raggedleft 21 & \raggedleft\arraybslash 1\\
\textitbf{lati} & ‘practice’ & \textitbf{latiang} & ‘practice’ & \raggedleft 17 & \raggedleft\arraybslash 3\\
\textitbf{kubur} & ‘burry’ & \textitbf{kuburang} & ‘grave’ & \raggedleft 14 & \raggedleft\arraybslash 8\\
\textitbf{atur} & ‘arrange’ & \textitbf{aturang} & ‘regulation’ & \raggedleft 8 & \raggedleft\arraybslash 24\\
\textitbf{ikat} & ‘tie up’ & \textitbf{ikatang} & ‘tie’ & \raggedleft 8 & \raggedleft\arraybslash 14\\
\textitbf{jual} & ‘sell’ & \textitbf{jualang} & ‘merchandise, sell’ & \raggedleft 8 & \raggedleft\arraybslash 14\\
\textitbf{turung} & ‘descend’ & \textitbf{turungang} & ‘descendant’ & \raggedleft 8 & \raggedleft\arraybslash 192\\
\textitbf{ulang} & ‘repeat’ & \textitbf{ulangang} & ‘repetition’ & \raggedleft 8 & \raggedleft\arraybslash 16\\
\textitbf{bantu} & ‘help’ & \textitbf{bantuang} & ‘help’ & \raggedleft 7 & \raggedleft\arraybslash 34\\
\textitbf{alas} & ‘put down as base’ & \textitbf{alasang} & ‘reason’ & \raggedleft 6 & \raggedleft\arraybslash 7\\
\textitbf{bangun}\textitbf{g} & ‘build’ & \textitbf{bangungan}\textitbf{g} & ‘building’ & \raggedleft 6 & \raggedleft\arraybslash 25\\
\textitbf{libur} & ‘have vacation’ & \textitbf{liburang} & ‘vacation’ & \raggedleft 6 & \raggedleft\arraybslash 10\\
\textitbf{campur} & ‘mix’ & \textitbf{campurang} & ‘mixture’ & \raggedleft 5 & \raggedleft\arraybslash 5\\
\textitbf{jalang} & ‘walk’ & \textitbf{jalangang} & ‘route’ & \raggedleft 5 & \raggedleft\arraybslash 485\\
\textitbf{lapor} & ‘report’ & \textitbf{laporang} & ‘report’ & \raggedleft 5 & \raggedleft\arraybslash 14\\
\textitbf{tulis} & ‘write’ & \textitbf{tulisang} & ‘writing’ & \raggedleft 5 & \raggedleft\arraybslash 12\\
\lspbottomrule
\end{tabular}

Affixing verbal bases with \textitbf{\-ang} typically derives nouns that denote the object of the action specified by the verbal base. The suffixed nouns include patients such as \textitbf{makangang} ‘that which is eaten’ or ‘food’, or results such as \textitbf{bagiang} ‘that which is divided’ or ‘part’. “Objective nominalization” that derives “nouns designating the result, or the typical or ‘cognate’ object of an action” has also been observed for other languages {(Comrie and Thompson 2007: 340)}. This polysemy can be explained in terms of a “domain shift” in that “one may go from one semantic domain to another, related one, and thus derive new interpretations” {\citep[221]{Booij2007}}. Hence, suffix \textitbf{\-ang} is glossed as ‘\textsc{pat}’ (‘patient’) in the sense of ‘patients or results which are \textsc{base}{}-ed’.
\end{styleBodyaftervbefore}


Two derived nouns together with their bases are given in context: \textitbf{makangang} ‘food’ with its bivalent base \textitbf{makang} ‘eat’ in (0), and \textitbf{jalangang} ‘route’ with its monovalent base \textitbf{jalang} ‘walk’ in (0).
\end{styleBodyvxafter}

\begin{styleExampleTitle}
Suffix \textitbf{\-ang}: Semantics of verbal bases and derived lexemes
\end{styleExampleTitle}

\begin{tabular}{llllllll}
\lsptoprule
\label{bkm:Ref340236752}
\gll {maytua} {bilang,} {\bluebold{makang}} {karna} {\bluebold{makang–ang}} {suda} {masak}\\ %
& wife & say & eat & because & eat–\textsc{pat} & already & cook\\
\lspbottomrule
\end{tabular}
\ea
\glt 
‘(my) wife said, ‘\bluebold{eat}, because the \bluebold{food} has already been cooked’ \textstyleExampleSource{[080919-004-NP.0039]}
\z

\begin{tabular}{llllllllllllll}
\lsptoprule
\label{bkm:Ref346552567}
\gll {\multicolumn{2}{l}{trus}} {kitong} {dua} {\multicolumn{3}{l}{pulang,}} {\multicolumn{2}{l}{sampe}} {di} {\bluebold{jalang–ang}} {sa} {istirahat,}\\ %
& \multicolumn{2}{l}{next} & \textsc{1pl} & two & \multicolumn{3}{l}{go.home} & \multicolumn{2}{l}{reach} & at & walk–\textsc{pat} & \textsc{1sg} & rest\\
& de & \multicolumn{2}{l}{bilang,} & \multicolumn{2}{l}{kitong} & dua & \multicolumn{2}{l}{\bluebold{jalang}} & \multicolumn{5}{l}{suda!}\\
& \textsc{3sg} & \multicolumn{2}{l}{say} & \multicolumn{2}{l}{\textsc{1pl}} & two & \multicolumn{2}{l}{walk} & \multicolumn{5}{l}{already}\\
\lspbottomrule
\end{tabular}
\ea
\glt 
‘and then we two went home, on the \bluebold{way} I rested, he said, ‘let the two of us \bluebold{walk} (on)!’’ \textstyleExampleSource{[081015-005-NP.0036]}
\z


Some of the suffixed items, listed in Table  ‎3 .6, differ from the other suffixed items, as for example \textitbf{jual\-ang} ‘sell-\textsc{pat}’ and \textitbf{dulu\-ang} ‘be.prior-\textsc{pat}’. Suffixed with \textitbf{\-ang}, these items are structurally nouns. In a sentence, however, \textitbf{jualang} also functions as the verb ‘sell’ in the same way as its base \textitbf{jual} ‘sell’, as shown in (0). Likewise \textitbf{duluang} ‘be.prior-\textsc{pat}’ in (0) is used in the same way as its base \textitbf{dulu} ‘be prior’ in (0).


\begin{styleExampleTitle}
Suffix \textitbf{\-ang}: Verbal reading of derived lexemes
\end{styleExampleTitle}

\begin{tabular}{llllllllll}
\lsptoprule
\label{bkm:Ref340236753}
\gll {mama} {saya} {pergi} {\bluebold{jual}} {pinang,} {sa} {pu} {mama} {\bluebold{jual–ang}}\\ %
& mother & \textsc{1sg} & go & sell & betel.nut & \textsc{1sg} & \textsc{poss} & mother & sell–\textsc{pat}\\
& \multicolumn{9}{l}{pinang}\\
& \multicolumn{9}{l}{betel.nut}\\
\lspbottomrule
\end{tabular}
\ea
\glt 
‘my mother went to \bluebold{sell} betel nuts, my mother \bluebold{sells} betel nuts’ \textstyleExampleSource{[081014-014-NP.0002]}
\z

\begin{tabular}{lllllll}
\lsptoprule
\label{bkm:Ref346550843}
\gll {nanti} {kam} {dari} {blakang,} {bapa} {\bluebold{dulu–ang}}\\ %
& very.soon & \textsc{2pl} & from & backside & father & be.prior–\textsc{pat}\\
\lspbottomrule
\end{tabular}
\ea
\glt
[About an upcoming official meeting:] ‘then you two (go in) second, (and) the gentleman \bluebold{(goes) ahead}’ \textstyleExampleSource{[081011-001-Cv.0199]}
\end{styleFreeTranslEngxvpt}

\begin{tabular}{llll}
\lsptoprule
\label{bkm:Ref439254549}
\gll {dua} {orang} {dulu}\\ %
& two & person & be.prior\\
\lspbottomrule
\end{tabular}
\ea
\glt
[About the number of potential nominees for the upcoming local election:] ‘two people \bluebold{(go) ahead}’ \textstyleExampleSource{[080919-001-Cv.0065]}
\end{styleFreeTranslEngxvpt}

\paragraph[Suffixed items derived from nominal and quantifier bases]{Suffixed items derived from nominal and quantifier bases}
\label{bkm:Ref340228783}
The corpus contains 13 \textitbf{\-ang}{}-suffixed lexemes with nominal bases (36 tokens) and two derived lexemes with quantifier bases (2 tokens), as listed in Table  ‎3 .7. In most cases, the bases and the derived nouns differ in their semantics. In some cases, the affixed nouns designate a magnification of the base, such as \textitbf{laut} ‘sea’ and \textitbf{lautang} ‘ocean’, or \textitbf{ruang} ‘room’ and \textitbf{ruangang} ‘large room’. In some cases, the meanings of the derived nouns are an extension of the meanings of their bases with suffix \textitbf{\-ang} having generalizing function, as for instance \textitbf{ana} ‘child’ and \textitbf{anaang} ‘offspring’, or \textitbf{musim} ‘season’ and \textitbf{musimang} ‘each season’. In yet other cases, the affixed nouns have unpredictable meanings compared to the semantics of their bases, such as \textitbf{rambut} ‘hair’ and \textitbf{rambutang} ‘rambutan’, and \textitbf{obat} ‘medicine’ and \textitbf{obatang} ‘magic spell’. And in a few cases, the base and the derived noun have the same semantics, as in \textitbf{pasang} ‘pair’ and \textitbf{pasangang} ‘pair’ or \textitbf{pangkal} ‘base’ and \textitbf{pangkalang} ‘base’.



All 13 derived lexemes are low frequency words, attested with less than 20 tokens. Moreover, the token frequencies for the respective bases are (much) higher for most of the derived words (10 lexemes); for one lexeme, the base is unattested in the corpus, although it does exist. Four of the 15 derived nouns were tentatively classified as SI-borrowings; in Table  ‎3 .7 these items are underlined (for more details see language internal factor (f in §3.1.1, p. \pageref{bkm:Ref346541608}).
\end{styleBodyvvafter}

\begin{stylecaption}
\label{bkm:Ref346610610}Table ‎3.\stepcounter{Table}{\theTable}:  Affixation with \textitbf{\-ang} of nominal and quantifier bases
\end{stylecaption}

\tablehead{
 BW & Gloss & Item & Gloss & \textitbf{\-ang} \# & \arraybslash BW \#\\
}
\begin{tabular}{llllll}
\lsptoprule
\textitbf{bayang} & ‘image’ & \textitbf{bayangang} & ‘shadow’ & \raggedleft 6 & \raggedleft\arraybslash 2\\
\textitbf{ana} & ‘child’ & \textitbf{anaang} & ‘offspring’ & \raggedleft 4 & \raggedleft\arraybslash 741\\
\textitbf{tingkat} & ‘floor’ & \textitbfUndl{tingkatang} & ‘level’ & \raggedleft 4 & \raggedleft\arraybslash 5\\
\textitbf{hukum} & ‘law’ & \textitbf{hukumang} & ‘punishment’ & \raggedleft 4 & \raggedleft\arraybslash 3\\
\textitbf{rambut} & ‘hair’ & \textitbf{rambutang} & ‘rambutan’ & \raggedleft 3 & \raggedleft\arraybslash 23\\
\textitbf{obat} & ‘medicine’ & \textitbf{obatang} & ‘magic spell’ & \raggedleft 3 & \raggedleft\arraybslash 9\\
\textitbf{pasang} & ‘pair’ & \textitbf{pasangang} & ‘pair’ & \raggedleft 3 & \raggedleft\arraybslash 2\\
\textitbf{laut} & ‘sea’ & \textitbf{lautang} & ‘ocean’ & \raggedleft 2 & \raggedleft\arraybslash 68\\
\textitbf{pinggir} & ‘border’ & \textitbfUndl{pinggirang} & ‘edges’ & \raggedleft 2 & \raggedleft\arraybslash 23\\
\textitbf{ruang} & ‘room’ & \textitbf{ruangang} & ‘large room’ & \raggedleft 2 & \raggedleft\arraybslash 3\\
\textitbf{kandung} & ‘womb’ & \textitbfUndl{kandungang} & ‘womb’ & \raggedleft 1 & \raggedleft\arraybslash 8\\
\textitbf{musim} & ‘season’ & \textitbfUndl{musimang} & ‘each season’ & \raggedleft 1 & \raggedleft\arraybslash 5\\
\textitbf{pangkal} & ‘base’ & \textitbf{pangkalang} & ‘base’ & \raggedleft 1 & \raggedleft\arraybslash 0\\
\textitbf{pulu} & ‘tens’ & \textitbf{puluang} & ‘tens’ & \raggedleft 1 & \raggedleft\arraybslash 78\\
\textitbf{ratus} & ‘hundreds’ & \textitbf{ratusang} & ‘hundreds’ & \raggedleft 1 & \raggedleft\arraybslash 34\\
\lspbottomrule
\end{tabular}

The data listed in Table  ‎3 .7 show that most of the nominal bases and affixed nouns differ in their semantics. The magnifying function of suffix \textitbf{\-ang} is illustrated in (0) and (0), the generalizing function in (0), and its unpredictable semantics in (0) and (0).
\end{styleBodyaftervbefore}


The magnifying function of \textitbf{\-ang} is demonstrated with \textitbf{laut} ‘sea’ in (0) and \textitbf{lautang} ‘ocean’ in (0). While \textitbf{laut} refers to the ‘sea’ close to the coast, \textitbf{lautang} denotes the open and deep ‘ocean’ off the coast.
\end{styleBodyvvafter}

\begin{styleExampleTitle}
Suffix \textitbf{\-ang}: Magnifying function
\end{styleExampleTitle}

\begin{tabular}{lllllll}
\lsptoprule
\label{bkm:Ref346618901}
\gll {dong} {dua} {pergi} {mancing} {di} {\bluebold{laut}}\\ %
& \textsc{3pl} & two & go & fish.with.rod & at & sea\\
\lspbottomrule
\end{tabular}
\ea
\glt 
‘the two of them went fishing on the \bluebold{sea}’ \textstyleExampleSource{[081109-005-JR.0005]}
\z

\begin{tabular}{llllllllll}
\lsptoprule
\label{bkm:Ref346618900}
\gll {banyak} {mati} {…} {di} {pulow{\Tilde}pulow} {banyak} {mati} {di} {\bluebold{laut–ang}}\\ %
&  &  &  &  &  &  &  &  & sea–\textsc{pat}\\
& many & die &  & at & \textsc{rdp}{\Tilde}island & many & die & at & ocean\\
\lspbottomrule
\end{tabular}
\ea
\glt 
‘many died … on the islands, many died on the (open) \bluebold{ocean}’ \textstyleExampleSource{[081029-002-Cv.0025]}
\z


The generalizing function of \textitbf{\-ang} is illustrated with \textitbf{ana} ‘child’ and \textitbf{anaang} ‘offspring’ in (0).


\begin{styleExampleTitle}
Suffix \textitbf{\-ang}: Generalizing function
\end{styleExampleTitle}

\begin{tabular}{lllllllllllllllll}
\lsptoprule
\label{bkm:Ref346619994}
\gll {kalo} {\multicolumn{2}{l}{mo}} {antar} {\multicolumn{3}{l}{\bluebold{ana}}} {\multicolumn{2}{l}{\bluebold{prempuang}}} {ke} {\bluebold{ana}} {\multicolumn{2}{l}{\bluebold{laki{\Tilde}laki}}} {\multicolumn{2}{l}{…}} {kitorang}\\ %
& if & \multicolumn{2}{l}{want} & bring & \multicolumn{3}{l}{child} & \multicolumn{2}{l}{woman} & to & child & \multicolumn{2}{l}{\textsc{rdp}{\Tilde}husband} & \multicolumn{2}{l}{} & \textsc{1pl}\\
& \multicolumn{2}{l}{itu} & \multicolumn{3}{l}{harus} & … & \multicolumn{2}{l}{bawa} & \multicolumn{2}{l}{\bluebold{ana–ang}} & \multicolumn{2}{l}{pinang} & \multicolumn{2}{l}{\bluebold{ana–ang}} & \multicolumn{2}{l}{sagu}\\
& \multicolumn{2}{l}{} & \multicolumn{3}{l}{} &  & \multicolumn{2}{l}{} & \multicolumn{2}{l}{child–\textsc{pat}} & \multicolumn{2}{l}{} & \multicolumn{2}{l}{child–\textsc{pat}} & \multicolumn{2}{l}{}\\
& \multicolumn{2}{l}{\textsc{d.dist}} & \multicolumn{3}{l}{have.to} &  & \multicolumn{2}{l}{bring} & \multicolumn{2}{l}{offspring} & \multicolumn{2}{l}{betel.nut} & \multicolumn{2}{l}{offspring} & \multicolumn{2}{l}{sago}\\
\lspbottomrule
\end{tabular}
\ea
\glt 
[About wedding preparations:] ‘if we want to bring our \bluebold{daughter} to (their) \bluebold{son} … we have to … bring betel nut \bluebold{seedlings} (and) sago \bluebold{seedlings}’ (Lit. ‘female/male \bluebold{child}; betel nut/sago \bluebold{offspring}’) \textstyleExampleSource{[081110-005-CvPr.0055-0057]}
\z


In some cases, the semantics of the affixed nouns are unpredictable, although a connection between the base word and the derived word can still be seen. This is demonstrated with \textitbf{rambut} ‘hair’ in (0) and \textitbf{rambutang} ‘rambutan’ in (0), which refers to the fruit of the rambutan tree (\textstyleChItalic{Nephelium lappaceum}). The leathery reddish skin of the fruit is covered with numerous hairy protuberances, which is depicted by the label \textitbf{rambut-ang} ‘hair-\textsc{pat}’.


\begin{styleExampleTitle}
Suffix \textitbf{\-ang}: Unpredictable semantics
\end{styleExampleTitle}

\begin{tabular}{lllllll}
\lsptoprule
\label{bkm:Ref346621956}
\gll {sa} {mo} {cuci} {de} {pu} {\bluebold{rambut}}\\ %
& \textsc{1sg} & want & wash & \textsc{3sg} & \textsc{poss} & hair\\
\lspbottomrule
\end{tabular}
\ea
\glt 
‘I want to wash her \bluebold{hair}’ \textstyleExampleSource{[081025-001-CvHt.0006]}
\z

\begin{tabular}{llllllllll}
\lsptoprule
\label{bkm:Ref346621954}
\gll {di} {sini} {ada} {jambu} {di} {sini} {ada} {ada} {\bluebold{rambut–ang}}\\ %
&  &  &  &  &  &  &  &  & hair–\textsc{pat}\\
& at & \textsc{l.prox} & exist & rose.apple & at & \textsc{l.prox} & exist & exist & rambutan\\
\lspbottomrule
\end{tabular}
\ea
\glt
‘here are rose apples, here are are \bluebold{rambutan}’ \textstyleExampleSource{[081029-001-Cv.0006]}
\end{styleFreeTranslEngxvpt}

\paragraph[Variables of the communicative event]{Variables of the communicative event}
\label{bkm:Ref346647730}
To further investigate the issue of productivity of \textitbf{\-ang} in Papuan Malay, a domain analysis was conducted which focused on the variables of speaker education levels, topics, and/or role-relations (for details see ‘Language external factors’ in §3.1.1, p. \pageref{bkm:Ref346712254}). In all, 84 items suffixed with \textitbf{\-ang}, totaling 441 tokens, were investigated:


\begin{itemize}
\item \begin{styleIIndented}
69 suffixed items derived from verbal bases (403 tokens)
\end{styleIIndented}\item \begin{styleIvI}
15 suffixed items derived from nominal and quantifier bases (38 tokens)
\end{styleIvI}\end{itemize}

For the 84 suffixed lexemes, 352/441 tokens (80\%) can be accounted for in terms of speaker education levels, topics, and/or role-relations. The remaining 89/441 tokens (20\%) occurred when less-educated speakers (\textsc{\-edc-spk}) talked with fellow-Papuans of equally low social standing (\textsc{\-stat}) about \textsc{low} topics.\footnote{\\
\\
\\
\\
\\
\\
\\
As mentioned under Factor 3 ‘Relationships between interlocutors’ in §1.5.1 (p. \pageref{bkm:Ref376429590}), all of the recorded less-educated speakers belong to the group of Papuans with lower social status (\textsc{\-stat}), while the recorded Papuans with higher social status (\textsc{+stat}), such as teachers, government officials, or pastors, are all better educated.\\
\\
\\
\\
\\
\\
\\
\\
} (See Table  ‎3 .8 and Figure  ‎3 .7.)



That is, a considerable number of tokens (20\%) cannot be explained in terms of these variables of the communicative event. Therefore, it cannot be concluded that the respective lexemes are code-switches with Indonesian. This total of 89/441 tokens (20\%) includes 80/403 tokens (20\%) with verbal bases and 9/38 tokens (24\%) with nominal or quantifier bases. The vast majority of \textitbf{\-ang}{}-suffixed tokens (352/441 – 80\%), however, seems to be conditioned by variables of the communicative event.
\end{styleBodyvafter}


As for the rather high number of unaccounted tokens with nominal or quantifier bases (9/38 – 24\%), one observation is made. Four of the nine tokens refer to the same lexeme produced by the same speaker during three conversations about the same topic, namely the death of a young mother. This speaker has a reputation of speaking incoherently due to his unsuccessful attempts to approximate Standard Indonesian. Excluding these four tokens brings down the number of unaccounted lexemes to 15\% (5/34). If affixation of nominal bases was a productive process, however, one would expect this percentage to be much higher. In turn, this finding does not support the conclusion that the suffixed lexemes with nominal or quantifier bases result from a productive derivation process. Instead, they seem to be code-switches with Indonesian.
\end{styleBodyvafter}


The data presented in Table  ‎3 .8 and Figure  ‎3 .7 are discussed in more detail below.
\end{styleBodyvvafter}

\begin{stylecaption}
\label{bkm:Ref354668804}Table ‎3.\stepcounter{Table}{\theTable}:  Token frequencies for \textitbf{\-ang}{}-suffixed lexemes with verbal, nominal, and quantifier bases by speakers, topics, and interlocutors (84 items)
\end{stylecaption}

\tablehead{ & \multicolumn{4}{l}{ Topics (\textsc{top})} & \multicolumn{3}{l}{ Interlocutors (\textsc{ilct})} & \arraybslash Tokens\\
}
\begin{tabular}{lllllllll}
\lsptoprule
\multicolumn{9}{l}{Suffixed lexemes with verbal bases (69 items)}\\
& \textsc{pol} & \textsc{edc} & \textsc{rel} & \textsc{low} & \textsc{+stat} & \textsc{\-stat} & \textsc{outsd} & \arraybslash Total\\
\textsc{+edc-spk} & \raggedleft 30 & \raggedleft 26 & \raggedleft 15 & \raggedleft 46 & \raggedleft {}-{}-{}- & \raggedleft {}-{}-{}- & \raggedleft 75 & \raggedleft\arraybslash 192\\
\textsc{\-edc-spk} & \raggedleft 15 & \raggedleft 40 & \raggedleft 47 & \raggedleft {}-{}-{}- & \raggedleft 26 & \raggedleft \textstyleChBold{80} & \raggedleft 3 & \raggedleft\arraybslash 211\\
Subtotal & \raggedleft 45 & \raggedleft 66 & \raggedleft 62 & \raggedleft 46 & \raggedleft 26 & \raggedleft \textstyleChBold{80} & \raggedleft 78 & \raggedleft\arraybslash 403\\
\multicolumn{9}{l}{Suffixed lexemes with nominal and quantifier bases (15 items)}\\
& \textsc{pol} & \textsc{edc} & \textsc{rel} & \textsc{low} & \textsc{+stat} & \textsc{\-stat} & \textsc{outsd} & \arraybslash Total\\
\textsc{+edc-spk} & \raggedleft 4 & \raggedleft 1 & \raggedleft 0 & \raggedleft 4 & \raggedleft {}-{}-{}- & \raggedleft {}-{}-{}- & \raggedleft 8 & \raggedleft\arraybslash 17\\
\textsc{\-edc-spk} & \raggedleft 3 & \raggedleft 0 & \raggedleft 6 & \raggedleft {}-{}-{}- & \raggedleft 1 & \raggedleft \textstyleChBold{9} & \raggedleft 2 & \raggedleft\arraybslash 20\\
Subtotal & \raggedleft 7 & \raggedleft 1 & \raggedleft 6 & \raggedleft 4 & \raggedleft 1 & \raggedleft \textstyleChBold{9} & \raggedleft 10 & \raggedleft\arraybslash 38\\
\multicolumn{9}{l}{\textstyleChBold{TOTAL} (84 items)}\\
& \textsc{pol} & \textsc{edc} & \textsc{rel} & \textsc{low} & \textsc{+stat} & \textsc{\-stat} & \textsc{outsd} & \arraybslash Total\\
\textsc{+edc-spk} & \raggedleft 34 & \raggedleft 27 & \raggedleft 15 & \raggedleft 50 & \raggedleft {}-{}-{}- & \raggedleft {}-{}-{}- & \raggedleft 83 & \raggedleft\arraybslash 209\\
\textsc{\-edc-spk} & \raggedleft 18 & \raggedleft 40 & \raggedleft 53 & \raggedleft {}-{}-{}- & \raggedleft 27 & \raggedleft \textstyleChBold{89} & \raggedleft 5 & \raggedleft\arraybslash 232\\
\textstyleChBold{Total} & \raggedleft 52 & \raggedleft 67 & \raggedleft 68 & \raggedleft 50 & \raggedleft 27 & \raggedleft \textstyleChBold{89} & \raggedleft 88 & \raggedleft\arraybslash \textstyleChBold{441}\\
\lspbottomrule
\end{tabular}
\begin{styleFigure}
  
%%please move the includegraphics inside the {figure} environment
%%\includegraphics[width=\textwidth]{kluge-img3.jpg}
 
\end{styleFigure}

\begin{styleCapFigure}
\label{bkm:Ref417129110}Figure ‎3.\stepcounter{Figure}{\theFigure}:  Token frequencies for \textitbf{\-ang}{}-suffixed lexemes with verbal, nominal, and quantifier bases by speakers, topics, and interlocutors
\end{styleCapFigure}


The data given in Table  ‎3 .8 and Figure  ‎3 .7 show that for the 84 \textitbf{\-ang}{}-suffixed lexemes, 352/441 tokens (80\%) can be explained in terms of speaker education levels, topics, and/or role-relations between the speakers and their interlocutors; this includes 323/403 tokens (80\%) with verbal bases, and 29/38 tokens (76\%) with nominal bases.
\end{styleBodyaftervbefore}


The better-educated speakers (\textsc{+edc-spk}) produced 209/441 tokens (47\%), while the less-educated speakers (\textsc{\-edc-spk}) produced 232/441 (53\%) tokens.
\end{styleBodyvafter}


In terms of topics (\textsc{top}), 187/441 tokens (42\%) occurred during conversations about \textsc{high} topics, that is, political, educational or religious affairs (\textsc{pol}, \textsc{edc} and \textsc{rel}, respectively). This includes 76/209 \textsc{+edc-spk} tokens (36\%) and 111/232 \textsc{\-edc-spk} tokens (48\%). Another 88/441 tokens (20\%) were produced during conversations with an outsider, namely the author (\textsc{outsd}), including 83/209 \textsc{+edc-spk} tokens (40\%) and 5/232 \textsc{\-edc-spk} tokens (0.2\%).
\end{styleBodyvafter}


This leaves 166/441 tokens (38\%) that were produced when the interlocutors discussed \textsc{low} topics. This includes 50/166 \textsc{+edc-spk} tokens (30\%) and 116/166 \textsc{\-edc-spk} tokens (70\%). The 116 \textsc{low} topic tokens produced by \textsc{\-edc-spk} are distributed as follows. When conversing with \textsc{+stat} Papuans, 27 tokens were produced (that is, 27/232 \textsc{\-edc-spk} tokens – 12\%). The remaining 89 tokens (that is, 89/232 \textsc{\-edc-spk} tokens – 38\%) occurred when \textsc{\-edc-spk} discussed \textsc{low} topics with \textsc{\-stat} Papuans, and therefore cannot be explained in terms of speaker education levels, topics, and/or role-relations. This total of 89 tokens refers to 20\% of all 441 \textitbf{\-ang} tokens. It includes 80/403 tokens (20\%) with verbal bases and 9/38 tokens (24\%) with nominal or quantifier bases.\footnote{\\
\\
\\
\\
\\
\\
\\
As for the 28 hapaxes (23 with verbal bases, and five with nominal bases), 17 appear to be conditioned by the variables of speaker education levels, topics, and/or role-relations, and therefore are best explained as code-switches with Indonesian. This leaves 11 hapaxes that are unaccounted for in terms of language external factors and that might be the result of a productive word-formation process. For 11 hapaxes, P=0.0249 as opposed to P=0.0635 for 28 hapaxes (N=441). The total of 11 hapaxes includes nine with verbal bases (P=0.0223) and two with nominal or quantifier bases (P=0.0526).\\
\\
\\
\\
\\
\\
\\
\\
}
\end{styleBodyvxvafter}

\paragraph[Summary and conclusions]{Summary and conclusions}
\label{bkm:Ref346647732}
Suffix \textitbf{\-ang} is polyfunctional in that it derives nouns from verbal, nominal, and quantifier bases. This polyfunctionality suggests that affixation with \textitbf{\-ang} is a somewhat productive process (see language internal factor (c in §3.1.1, p. \pageref{bkm:Ref364758123}).



Concerning \textitbf{\-ang}{}-affixation of verbal bases, four other observations support this conclusion: (1) the transparent form-function relationship between the derived nouns and their respective bases, (2) the large number of low frequency words and small number of high frequency words, (3) the relative token frequencies with most bases having higher frequencies than the affixed lexemes, and (4) the low number of derived lexemes tentatively classified as SI-borrowings.
\end{styleBodyvafter}


To a lesser extent, the same observations apply to \textitbf{\-ang}{}-affixation of nominal bases: (1) the form-function relationships between derived lexemes and bases is more or less transparent, (2) all derived lexemes are low frequency words, (3) most bases have higher token frequencies than the affixed lexemes, and (3) the number of derived lexemes tentatively classified as SI-borrowings is rather low. These findings suggest that \textitbf{\-ang}{}-affixation of nominal bases is also a somewhat productive process.
\end{styleBodyvafter}


With respect to the speech situations during which the derived nouns occurred, the following patterns emerge. For affixed nouns with verbal bases, one fifth of the attested tokens cannot be accounted for in terms of pertinent variables of the communicative event; that is, for these items there are no indications that they are code-switches with Indonesian. However, the vast majority of tokens with verbal bases (80\%) seem to be conditioned by the variables of speaker education levels, topics, and/or role-relations and are best explained as code-switches with Indonesian. The same applies to nouns with nominal bases for which most tokens also appear to be conditioned by the three mentioned variables of the communicative event. Hence, these items are also best explained as code-switches with Indonesian.
\end{styleBodyvafter}


These findings suggest that in Papuan Malay \textitbf{\-ang}{}-affixation is a productive process to derive nouns from verbal and nominal bases. The degree of productivity appears to be limited, however, as most tokens seem to be code-switches with Indonesian.
\end{styleBodyvxvafter}

\subsection{Prefix \textscItalBold{pe(n}\textscItalBold{)\-} ‘\textsc{ag}’}
\label{bkm:Ref376173294}\label{bkm:Ref354834872}\label{bkm:Ref347762867}\label{bkm:Ref347514353}
Affixation with \textscItalBold{pe(n)\-} ‘\textsc{ag}’ typically derives nominals from verbal bases. The derived nouns denote the agent or instrument of the action, event, or state specified by the verbal base, as in (0). Some lexemes are also derived from nominal bases. The affixation process appears to be marginally productive in Papuan Malay, at best, as discussed below.
\end{styleBodyxafter}

\begin{tabular}{lllll}
\lsptoprule
\label{bkm:Ref346799015}
\gll {pokoknya} {orang} {\bluebold{pen–datang}} {pulang}\\ %
& the.main.thing.is & person & \textsc{ag}–come & go.home\\
\lspbottomrule
\end{tabular}
\ea
\glt 
‘the main thing is (that) the \bluebold{strangers} return home’ (Lit. ‘\bluebold{the one who comes}’) \textstyleExampleSource{[081029-005-Cv.0048]}
\z


Suffix \textscItalBold{pe(n)\-} is a reflex of Proto-Malayic *\textitbf{p}\textscItalBold{an}\textitbf{\-}, which “formed deverbal nouns that were used attributively, predicatively, and in prepositional phrases, and that had a nominal as head or subject. They denoted a purpose or instrument when prefixed to VDIs and VTRs. Moreover, *\textitbf{p}\textscItalBold{an}\textitbf{\-} denoted an inclination or characteristic when prefixed to VSIs” {\citep[193]{Adelaar1992}}. In Standard Malay, derived lexemes with a monovalent base “denote a characteristic” while forms with a bivalent base “usually denote an actor or instrument” or “a goal or result, or they form an abstract noun. Furthermore \textitbf{pa}\textscItalBold{n}\textitbf{\-} forms are used attributively, and, on the basis of VSIs, they can function as VSIs” {(1992: 183)}.



In some of the eastern Malay varieties, the prefix is also found. In Ambon Malay, the prefix occurs but it is unproductive {(van Minde 1997: 109)}. In Manado Malay \textitbf{paŋ-} also occurs and is productive (in addition, a non-unproductive form \textitbf{pa}\textsc{\-} exists) {(Stoel 2005: 18, 24)}. Likewise, in North Moluccan / Ternate Malay \textitbf{pa(}\textscItalBold{n}\textitbf{)\-} occurs, but its status is uncertain. While {\citet[4]{Voorhoeve1983}} maintains that it “is no longer morphologically distinct”, {Litamahuputty}{ }{(2012: 30)} states that the prefix is productive. In these varieties, the prefix usually denotes the actor or instrument of the event expressed by the base. In addition, however, some of prefixed forms can also receive a verbal reading, as discussed in more detail in §3.1.4.2.
\end{styleBodyvafter}


The corpus contains 34 nouns (186 tokens) prefixed with \textscItalBold{pe(n)\-}:\footnote{\\
\\
\\
\\
\\
\\
\\
The 34 nouns include 11 hapaxes (P=0.0591); the 29 nouns with verbal bases include nine hapaxes (P=0.0588); the five nouns with nominal bases include two hapaxes (P=0.0606).\\
\\
\\
\\
\\
\\
\\
\\
}
\end{styleBodyvvafter}

%\setcounter{itemize}{0}
\begin{itemize}
\item \begin{styleIIndented}
Nouns with verbal bases (29 items with 153 tokens)
\end{styleIIndented}\item \begin{styleIvI}
Nouns with nominal bases (five items with 33 tokens)
\end{styleIvI}\end{itemize}

The corpus also contains nine formally complex words with non-compositional semantics, such as \textitbf{peserta} ‘participant’ or \textitbf{panggayu} ‘(a/to) paddle’.



Before discussing \textscItalBold{pe(n)\-}affixation of verbal bases in §3.1.4.2 and of nominal bases in §3.1.4.3, the allomorphy of \textscItalBold{pe(n)\-} is investigated in §3.1.4.1. Three variables of the communicative event that may impact the use of \textscItalBold{pe(n)\-} are explored in §3.1.4.4. The main points on prefix \textscItalBold{pe(n)\-} are summarized and evaluated in §3.1.4.5.
\end{styleBodyvxvafter}

\paragraph[Allomorphy of pe(n)\-]{Allomorphy of \textscItalBold{pe(n)\-}}
\label{bkm:Ref376173190}\label{bkm:Ref354566123}\label{bkm:Ref337036767}
Prefix \textscItalBold{pe(n)\-} has two allomorphs, \textitbf{pe(}\textscItalBold{n}\textitbf{)}\textitbf{\-} and \textitbf{pa(}\textscItalBold{n}\textitbf{)}\textitbf{\-} (small-caps \textscItalBold{n} represents the different realizations of the nasal). The allomorphs are not governed by phonological processes.



The form \textitbf{pe(}\textscItalBold{n}\textitbf{)}\textitbf{\-}, in turn, has seven allomorphs that result from morphologically conditioned phonological rules. More specifically, they are conditioned by the word-initial segment of the base word, as shown in Table  ‎3 .9: /\textstyleChCharisSIL{pɛm\-}/, /\textstyleChCharisSIL{pɛn\-}/, /\textstyleChCharisSIL{pɛɲ\-}/, \textstyleChCharisSIL{pɛŋ\-}/, /\textstyleChCharisSIL{pɛ\-}/, /\textstyleChCharisSIL{p\-}/, and /\textstyleChCharisSIL{pl\-}/. The prefix is realized as /\textstyleChCharisSIL{pɛm\-}/ when the initial segment of the base is a bilabial stop. Onset voiced stops are retained, while voiceless stops are deleted. With onset bilabial /\textstyleChCharisSIL{m}/, the prefix is realized as /\textstyleChCharisSIL{pɛ\-}/. With alveolar stops, the prefix is very commonly realized as /\textstyleChCharisSIL{pɛn\-}/. Again, the onset voiced stop is retained, while the onset voiceless stop is deleted. Alternatively, however, the onset voiceless stop can also be retained, in which case the prefix is realized as /\textstyleChCharisSIL{pɛ\-}/. With onset fricative /\textstyleChCharisSIL{s}/, the prefix is realized as /\textstyleChCharisSIL{pɛɲ\-}/, with /\textstyleChCharisSIL{s}/ being deleted. With onset palato-alveolar affricates, \textitbf{pe(}\textscItalBold{n}\textitbf{)}\textitbf{\-} is realized as /\textstyleChCharisSIL{pɛn\-}/. With onset rhotic /r/, the affix is realized as /\textstyleChCharisSIL{pɛ\-}/. With onset velar stops and onset vowels, the prefix is realized as /\textstyleChCharisSIL{pɛŋ\-}/. Finally, when prefixed to \textitbf{ajar} ‘teach’, \textitbf{pe(}\textscItalBold{n}\textitbf{)}\textitbf{\-} is realized as /\textstyleChCharisSIL{pl\-}/.


\begin{stylecaption}
\label{bkm:Ref335915563}Table ‎3.\stepcounter{Table}{\theTable}:  Realizations of allomorph \textitbf{pe(}\textscItalBold{n}\textitbf{)}\textitbf{\-}
\end{stylecaption}

\tablehead{
 \textitbf{pe(}\textscItalBold{n}\textitbf{)\-}base & Orthogr. & \arraybslash Gloss\\
}
\begin{tabular}{lll}
\lsptoprule
/\textstyleChCharisSIL{pɛm}–\textstyleChCharisSIL{bantu}/ & \textitbf{pembantu} & ‘house-helper’\\
/\textstyleChCharisSIL{pɛm}–\textstyleChCharisSIL{pili}/ & \textitbf{pemili} & ‘voter’\\
/\textstyleChCharisSIL{pɛ}–\textstyleChCharisSIL{muda}/ & \textitbf{pemuda} & ‘youth’\\
/\textstyleChCharisSIL{pɛn}–\textstyleChCharisSIL{dataŋ}/ & \textitbf{pendatang} & ‘new-comer’\\
/\textstyleChCharisSIL{pɛn}–\textstyleChCharisSIL{tumpaŋ}/ & \textitbf{penumpang} & ‘passenger’\\
/\textstyleChCharisSIL{pɛ}–\textstyleChCharisSIL{tugas}/ & \textitbf{petugas} & ‘official’\\
/\textstyleChCharisSIL{pɛɲ}–\textstyleChCharisSIL{sakit}/ & \textitbf{penyakit} & ‘disease’\\
/\textstyleChCharisSIL{pɛn}–\textstyleChCharisSIL{tʃuri}/ & \textitbf{pencuri} & ‘thief, to steal (\textsc{emph})’\\
/\textstyleChCharisSIL{pɛn}–\textstyleChCharisSIL{dʒaga}/ & \textitbf{penjaga} & ‘guard’\\
/\textstyleChCharisSIL{pɛ}–\textstyleChCharisSIL{rɛntʃana}/ & \textitbf{perencana} & ‘planner’\\
/\textstyleChCharisSIL{pɛŋ–acara}/ & \textitbf{pengacara} & ‘master of ceremony’\\
/\textstyleChCharisSIL{pɛŋ}–\textstyleChCharisSIL{ganti}/ & \textitbf{pengganti} & ‘replacement’\\
/\textstyleChCharisSIL{pl}–\textstyleChCharisSIL{adʒar}/ & \textitbf{plajar} & ‘teacher’\\
\lspbottomrule
\end{tabular}

The allomorph \textitbf{pa(}\textscItalBold{n}\textitbf{)}\textitbf{\-} occurs considerably less frequently. Attested are only the four items listed in Table  ‎3 .10 with a total of 18 \textitbf{pa(}\textscItalBold{n}\textitbf{)}\textitbf{\-} tokens. Form \textitbf{pa(}\textscItalBold{n}\textitbf{)}\textitbf{\-} has two attested allomorphs: /\textstyleChCharisSIL{pan\-}/ and /\textstyleChCharisSIL{pa\-}/. The phonological processes involved in the allomorphy are the same as those for \textitbf{pe(}\textscItalBold{n}\textitbf{)}\textitbf{\-}, discussed above. For two of the items, the prefix is alternatively realized as allomorph \textitbf{pe(}\textscItalBold{n}\textitbf{)}\textitbf{\-}. Therefore, for each item the token frequencies for \textitbf{pa(}\textscItalBold{n}\textitbf{)}\textitbf{\-} and for \textitbf{pe(}\textscItalBold{n}\textitbf{)}\textitbf{\-} are given. If the prefix is realized with /\textstyleChCharisSIL{pɛ(}\textsc{n}\textstyleChCharisSIL{)\-}/ in a greater number of tokens than with /\textstyleChCharisSIL{pa(}\textsc{n}\textstyleChCharisSIL{)\-}/, then its orthographic representation is \textscItalBold{pe(n)\-} as in \textitbf{pencuri} ‘thief, steal (\textsc{emph})’.


\begin{stylecaption}
\label{bkm:Ref346722692}Table ‎3.\stepcounter{Table}{\theTable}:  Realizations of allomorph \textitbf{pa(}\textscItalBold{n}\textitbf{)}\textitbf{\-}
\end{stylecaption}

\tablehead{
 \textitbf{pa(}\textscItalBold{n}\textitbf{)}\textitbf{{}-}base & Orthogr. & Gloss & \textitbf{pa(}\textscItalBold{n}\textitbf{)}\textitbf{\-} \# & \arraybslash \textitbf{pe(}\textscItalBold{n}\textitbf{)}\textitbf{\-} \#\\
}
\begin{tabular}{lllll}
\lsptoprule
/\textstyleChCharisSIL{pa}–\textstyleChCharisSIL{malas}/ & \textitbf{pamalas} & ‘listless person, be very listless’ & \raggedleft 12 & \raggedleft\arraybslash 2\\
/\textstyleChCharisSIL{pan}–\textstyleChCharisSIL{diam}/ & \textitbf{pandiam} & ‘taciturn person, be very quiet’ & \raggedleft 2 & \raggedleft\arraybslash 0\\
/\textstyleChCharisSIL{pan}–\textstyleChCharisSIL{takut}/ & \textitbf{panakut} & ‘coward, be very fearful (of)’ & \raggedleft 3 & \raggedleft\arraybslash 0\\
/\textstyleChCharisSIL{pan}–\textstyleChCharisSIL{tʃuri}/ & \textitbf{pencuri} & ‘thief, steal (\textsc{emph})’ & \raggedleft 1 & \raggedleft\arraybslash 11\\
\lspbottomrule
\end{tabular}

In realizing the prefix typically as \textitbf{pe(}\textscItalBold{n}\textitbf{)}\textitbf{\-} rather than as \textitbf{pa(}\textscItalBold{n}\textitbf{)}\textitbf{\-}, Papuan Malay differs from other eastern Malay varieties such as Ambon Malay {(van Minde 1997: 109)}, Manado Malay {\citep[23]{Stoel2005}}, and North Moluccan / Ternate Malay ({Voorhoeve 1983: 4;} {Litamahuputty 2012: 30}). In these varieties the prefix is always realized as \textitbf{pa(}\textscItalBold{n}\textitbf{)}\textitbf{\-}. Instead, the \textscItalBold{pe(n)\-}prefixed items have more resemblance with the corresponding items in Standard Indonesian where the prefix is realized as \textitbf{pe(}\textscItalBold{n}\textitbf{)}\textitbf{\-}. This is again an indication of the distinct history of Papuan Malay vis-à-vis the other eastern Malay varieties, discussed in §1.8.


\paragraph[Prefixed items derived from verbal bases]{Prefixed items derived from verbal bases}
\label{bkm:Ref337036769}
The corpus includes 29 \textscItalBold{pe(n)\-}prefixed nouns (with 153 tokens) with verbal bases, listed in Table  ‎3 .11. Included are items with biverbal bases such as \textitbf{curi} ‘steal’, monovalent dynamic bases such as \textitbf{duduk} ‘sit’, or monovalent stative bases such as \textitbf{muda} ‘be young’. The affixation process derives nouns that designate the subject of the action, event, or state specified by the verbal base.



All but one of the derived words are low frequency words (28 lexemes, attested with less than 20 tokens). In addition, the token frequencies for the respective bases are (much) higher for most of the derived words (24 lexemes). While the 29 prefixed items are structurally nouns, four of them also have verbal functions in their actual uses: \textitbf{pamalas} ‘listless person, be very listless’, \textitbf{pandiam} ‘taciturn person, be very quiet’, \textitbf{panakut} ‘coward, be very fearful (of)’, and \textitbf{pencuri} ‘thief, steal (\textsc{emph})’. These items are investigated in more detail in (0) to (0).
\end{styleBodyvafter}


Of the 29 derived lexemes, more than half (17 items) were tentatively classified as borrowings from Standard Indonesian (SI-borrowings) (for details see language internal factor (f in §3.1.1, p. \pageref{bkm:Ref346541608}); in Table  ‎3 .11 these items are underlined.
\end{styleBodyvvafter}

\begin{stylecaption}
\label{bkm:Ref335910032}Table ‎3.\stepcounter{Table}{\theTable}:  Affixation with \textscItalBold{pe(n)\-} of verbal bases
\end{stylecaption}

\tablehead{
 BW & Gloss & Item & Gloss & \textscItalBold{pe(n)\-} \# & \arraybslash BW \#\\
}
\begin{tabular}{llllll}
\lsptoprule
\textitbf{muda} & ‘be young’ & \textitbf{pemuda} & ‘youth’ & \raggedleft 46 & \raggedleft\arraybslash 24\\
\textitbf{malas} & ‘be list-less’ & \textitbf{pamalas} & ‘listless person, be very listless’ & \raggedleft 14 & \raggedleft\arraybslash 19\\
\textitbf{curi} & ‘steal’ & \textitbf{pencuri} & ‘thief, steal (\textsc{emph})’ & \raggedleft 12 & \raggedleft\arraybslash 4\\
\textitbf{pimping} & ‘lead’ & \textitbf{pemimping} & ‘leader’ & \raggedleft 11 & \raggedleft\arraybslash 8\\
\textitbf{datang} & ‘come’ & \textitbf{pendatang} & ‘new-comer’ & \raggedleft 10 & \raggedleft\arraybslash 447\\
\textitbf{sakit} & ‘be sick’ & \textitbf{penyakit} & ‘disease’ & \raggedleft 7 & \raggedleft\arraybslash 155\\
\textitbf{duduk} & ‘sit’ & \textitbf{penduduk} & ‘inhabitant’ & \raggedleft 5 & \raggedleft\arraybslash 167\\
\textitbf{tunggu} & ‘wait’ & \textitbf{penunggu} & ‘tutelary spirit’ & \raggedleft 5 & \raggedleft\arraybslash 92\\
\textitbf{pili} & ‘choose’ & \textitbfUndl{pemili} & ‘voter’ & \raggedleft 5 & \raggedleft\arraybslash 25\\
\textitbf{tanggung-jawap} & ‘be re\-sponsible’ & \textitbfUndl{penanggung-jawap} & ‘responsible per-son’ & \raggedleft 5 & \raggedleft\arraybslash 6\\
\textitbf{tumpang} & ‘join in’ & \textitbf{penumpang} & ‘passenger’ & \raggedleft 5 & \raggedleft\arraybslash 1\\
\textitbf{takut} & ‘feel afraid (of)’ & \textitbf{panakut} & ‘coward, be very fearful (of)’ & \raggedleft 3 & \raggedleft\arraybslash 154\\
\textitbf{tokok} & ‘pound’ & \textitbfUndl{penokok} & ‘pounder’ & \raggedleft 3 & \raggedleft\arraybslash 44\\
\textitbf{antar} & ‘bring’ & \textitbfUndl{pengantar} & ‘escort’ & \raggedleft 2 & \raggedleft\arraybslash 130\\
\textitbf{diam} & ‘be quiet’ & \textitbf{pandiam} & ‘taciturn person, be very quiet’ & \raggedleft 2 & \raggedleft\arraybslash 58\\
\textitbf{jaga} & ‘guard’ & \textitbfUndl{penjaga} & ‘guard’ & \raggedleft 2 & \raggedleft\arraybslash 41\\
\textitbf{ajar} & ‘teach’ & \textitbfUndl{plajar} & ‘teacher’ & \raggedleft 2 & \raggedleft\arraybslash 41\\
\textitbf{bantu} & ‘help’ & \textitbf{pembantu} & ‘house helper’ & \raggedleft 2 & \raggedleft\arraybslash 34\\
\textitbf{urus} & ‘arrange’ & \textitbfUndl{pengurus} & ‘manager’ & \raggedleft 2 & \raggedleft\arraybslash 28\\
\textitbf{bicara} & ‘speak’ & \textitbfUndl{pembicara} & ‘speaker’ & \raggedleft 1 & \raggedleft\arraybslash 332\\
\textitbf{ikut} & ‘follow’ & \textitbfUndl{pengikut} & ‘follower’ & \raggedleft 1 & \raggedleft\arraybslash 253\\
\textitbf{dengar} & ‘hear’ & \textitbfUndl{pendengar} & ‘listener’ & \raggedleft 1 & \raggedleft\arraybslash 130\\
\textitbf{pikir} & ‘think’ & \textitbfUndl{pemikir} & ‘thinker’ & \raggedleft 1 & \raggedleft\arraybslash 102\\
\textitbf{ganti} & ‘replace’ & \textitbfUndl{pengganti} & ‘replacement’ & \raggedleft 1 & \raggedleft\arraybslash 40\\
\textitbf{tolong} & ‘help’ & \textitbfUndl{penolong} & ‘helper’ & \raggedleft 1 & \raggedleft\arraybslash 39\\
\textitbf{tunjuk} & ‘show’ & \textitbfUndl{petunjuk} & ‘guide’ & \raggedleft 1 & \raggedleft\arraybslash 32\\
\textitbf{tendang} & ‘kick’ & \textitbfUndl{penendang} & ‘kicker’ & \raggedleft 1 & \raggedleft\arraybslash 4\\
\textitbf{iris} & ‘slice’ & \textitbfUndl{pengiris} & ‘slicer’ & \raggedleft 1 & \raggedleft\arraybslash 3\\
\textitbf{tinju} & ‘box’ & \textitbfUndl{petinju} & ‘boxer’ & \raggedleft 1 & \raggedleft\arraybslash 1\\
\lspbottomrule
\end{tabular}

Affixing verbal bases with \textscItalBold{pe(n)\-} derives nouns that denote the subject of the action, event, or state specified by the verbal base. The prefixed nouns include personal agents such as \textitbf{pendatang} ‘new comer’, impersonal agents such as \textitbf{penyakit} ‘disease’, or instruments such as \textitbf{penokok} ‘pounder’. This polysemy can be explained in terms of {Booij’s (1986: 509)} “extension scheme” which shows that “the conceptual category Agent […] derived from verbs with an Agent subject can be extended” to instruments such that “Personal Agent {\textgreater} Impersonal Agent {\textgreater} Instrument”. In Papuan Malay, this extension schema also includes less typical agents derived from stative verbs, so-called “attributants”, following {van Valin’s (2005: 55)} cross-linguistics definitions of thematic relations. Examples are \textitbf{pemuda} ‘youth’, derived from \textitbf{muda} ‘be young’. Hence, prefix \textscItalBold{pe(n)\-} is glossed as ‘\textsc{ag}’ (‘agent’) in the sense of ‘agents or instruments who/which habitually do \textsc{base} or have the characteristics of \textsc{base}’.
\end{styleBodyaftervbefore}


Two of the derived nouns together with their verbal bases are given in context: \textitbf{pemimping} ‘leader’ and its bivalent base \textitbf{pimping} ‘lead’ in (0) and (0), and \textitbf{pemuda} ‘youth’ and its monovalent base \textitbf{muda} ‘be young’ in (0) and (0), respectively.
\end{styleBodyvxafter}

\begin{styleExampleTitle}
Prefix \textscItalBold{pe(n)\-}: Semantics of verbal bases and derived lexemes
\end{styleExampleTitle}

\begin{tabular}{llllll}
\lsptoprule
\label{bkm:Ref346786615}
\gll {\bluebold{pemimping}} {mati,} {yo} {smua} {mati}\\ %
& pem–pimping &  &  &  & \\
& \textsc{ag}–lead & die & yes & all & die\\
\lspbottomrule
\end{tabular}
\ea
\glt 
‘(when) the \bluebold{leader} dies, yes, all die’ \textstyleExampleSource{[081010-001-Cv.0026]}
\z

\begin{tabular}{lllllllll}
\lsptoprule
\label{bkm:Ref346786617}
\gll {o} {kenal} {karna} {bapa} {kang} {biasa} {\bluebold{pimping}} {kor}\\ %
& oh! & know & because & father & you.know & usual & lead & choir\\
\lspbottomrule
\end{tabular}
\ea
\glt 
‘oh, (I) know (him), because, you know, the gentleman usually \bluebold{leads} the choir’ \textstyleExampleSource{[081011-022-Cv.0243]}
\z

\begin{tabular}{llllllll}
\lsptoprule
\label{bkm:Ref346786618}
\gll {sa} {liat} {\bluebold{pe–muda}} {di} {Takar} {banyak} {skali}\\ %
& \textsc{1sg} & see & \textsc{ag}–be.young & at & Takar & many & very\\
\lspbottomrule
\end{tabular}
\ea
\glt 
‘I see (there are) very many \bluebold{young people} in Takar’ \textstyleExampleSource{[080925-003-Cv.0176]}
\z

\begin{tabular}{llllll}
\lsptoprule
\label{bkm:Ref346786619}
\gll {kasi–ang} {masi} {\bluebold{muda}} {baru} {janda}\\ %
& love–\textsc{pat} & still & be.young & and.then & widow\\
\lspbottomrule
\end{tabular}
\ea
\glt 
‘poor thing, (she’s) still \bluebold{young} but now (she’s) a widow’ \textstyleExampleSource{[081006-015-Cv.0032]}
\z


Four of the prefixed lexemes listed in Table  ‎3 .11 are nouns that can also receive an intensified verbal reading: \textitbf{pamalas} ‘be very listless’ as in (0), \textitbf{pencuri} ‘steal (\textsc{emph})’ as in (0), \textitbf{panakut} ‘be very fearful (of)’ as in (0), and \textitbf{pandiam} ‘be very quiet’ as in (0). In (0) \textitbf{pamalas} ‘be very listless’ receives a verbal reading given that a nominal reading of \textitbf{pamalas kerja} ‘the lazy males work’ is inappropriate. In (0), \textitbf{pencuri} ‘steal (\textsc{emph})’ has verbal function as only verbs are negated with \textitbf{tra} ‘\textsc{neg}’ (see §5.3.6 and §13.1.1). In (0) \textitbf{panakut} ‘be very fearful(.of)’ functions as a verb, which is intensified with \textitbf{sampe} ‘reach’. The utterance in (0) is ambiguous, as \textitbf{pandiam} can receive the nominal reading ‘taciturn person’ or the verbal reading ‘be very quiet’.
\end{styleBodyxafter}

\begin{styleExampleTitle}
Prefix \textscItalBold{pe(n)\-}: Verbal reading of derived lexemes
\end{styleExampleTitle}

\begin{tabular}{lllllll}
\lsptoprule
\label{bkm:Ref339899144}
\gll {jadi} {sampe} {skarang} {laki{\Tilde}laki} {\bluebold{pa}\bluebold{–}\bluebold{malas}} {kerja}\\ %
& so & until & now & \textsc{rdp}{\Tilde}husband & \textsc{ag}–be.listless & work\\
\lspbottomrule
\end{tabular}
\ea
\glt 
‘so until now the men are \bluebold{too} \bluebold{listless} / \bluebold{don’t like it at all} to work’ \textstyleExampleSource{[081014-007-CvEx.0087]}
\z

\begin{tabular}{llll}
\lsptoprule
\label{bkm:Ref339899145}
\gll {dong} {tra} {\bluebold{pen}\bluebold{–}\bluebold{curi}}\\ %
& \textsc{3pl} & \textsc{neg} & \textsc{ag}–steal\\
\lspbottomrule
\end{tabular}
\ea
\glt 
‘(nowadays), they don’t \bluebold{steal (}\blueboldSmallCaps{emph}\bluebold{)}!’ \textstyleExampleSource{[081011-022-Cv.0298]}
\z

\begin{tabular}{lllllll}
\lsptoprule
\label{bkm:Ref339899146}
\gll {…} {i} {biasa–nya} {\bluebold{panakut}} {sampe} {bagemana}\\ %
&  &  &  & pan\bluebold{–}takut &  & \\
&  & ugh! & be.usual–\textsc{3possr} & \textsc{ag}–feel.afraid(.of) & reach & how\\
\lspbottomrule
\end{tabular}
\ea
\glt 
[About a frightening event at night:] ‘[she started (running) past (us),] ugh, usually (she’s) \bluebold{very fearful} beyond words’ \textstyleExampleSource{[081025-006-Cv.0328]}
\z

\begin{tabular}{llllllll}
\lsptoprule
\label{bkm:Ref439954517}\label{bkm:Ref339901701}
\gll {Sofia} {de} {bilang} {begini,} {sa} {ini} {\bluebold{pan}\bluebold{–}\bluebold{diam}}\\ %
& Sofia & \textsc{3sg} & say & like.this & \textsc{1sg} & \textsc{d.prox} & \textsc{ag}–be.quiet\\
\lspbottomrule
\end{tabular}
\ea
\glt 
‘Sofia said like this, ‘I’m a \bluebold{taciturn person} / I’m \bluebold{very quiet}’’ \textstyleExampleSource{[081115-001a-Cv.0190]}
\z


As discussed in the introductory remarks in §3.1.4, the corresponding prefix in Proto-Malayic and Standard Malay also has verbal function. That is, with monovalent stative bases, the derived lexemes “can function as VSIs” {\citep[183]{Adelaar1992}}. This prefix does not, however, have the intensifying verbal function that Papuan Malay \textscItalBold{pe(n)\-} has. This intensified verbal reading of mono- and bivalent verbal bases prefixed with \textscItalBold{pe(n)\-} could be an extension of the original functions of \textbf{\textit{pəN}}\- found in Standard Malay or of *\textitbf{p}\textscItalBold{an}\textitbf{\-} found in Proto-Malayic.



In other eastern Malay varieties, lexical items prefixed with \textitbf{pa}\- can also receive a verbal reading. For Ambon Malay, {van Minde}{ (1997: 109)} presents a number of examples, noting that “the word class of the \textitbf{pa(}\textscItalBold{n}\textitbf{)}\- formation varies between transitive verb, intransitive verb and noun”. For North Moluccan / Ternate Malay, {Voorhoeve}{ (1983: 4)} presents two prefixed items with a basic verbal reading: \textitbf{pamalas} ‘lazy’ and \textitbf{panggayung} ‘row’. Likewise, {\citet[40]{Litamahuputty1994}} presents two such items: \textitbf{pamalas} ‘lazy’ and \textitbf{panako} ‘afraid’; both “are considered to be monomorphemic”, however. For Manado Malay, {Stoel}{ (2005: 24)} also presents two such items: \textitbf{pancuri} ‘steal’ and \textitbf{pandusta} ‘lie’. As mentioned, though, prefix \textitbf{pa}\- is unproductive in Manado and North Moluccan / Ternate Malay.
\end{styleBodyvxvafter}

\paragraph[Prefixed items derived from nominal bases]{Prefixed items derived from nominal bases}
\label{bkm:Ref337036770}
The corpus contains five \textscItalBold{pe(n)\-}prefixed nouns (with 33 tokens), listed in Table  ‎3 .12, which are derived from nominal bases and denote abstract concepts. In general, the derived lexemes denote an ‘agent who executes what \textsc{base} indicates’. Four of the five lexemes are low frequency words, attested with less than 20 tokens. Moreover, the token frequencies for the respective bases are (much) higher for three of the five derived words. In addition, four items were tentatively classified as SI-borrowings (for details see language internal factor (f in §3.1.1, p. \pageref{bkm:Ref346541608}); in Table  ‎3 .12 these items are underlined.


\begin{stylecaption}
\label{bkm:Ref335898112}Table ‎3.\stepcounter{Table}{\theTable}:  Affixation with \textscItalBold{pe(n)\-} of nominal bases
\end{stylecaption}

\tablehead{
 BW & Gloss & Item & Gloss & \textscItalBold{pe(n)\-} \# & \arraybslash BW \#\\
}
\begin{tabular}{llllll}
\lsptoprule
\textitbf{printa} & ‘command’ & \textitbf{pemrinta} & ‘government’ & \raggedleft 23 & \raggedleft\arraybslash 5\\
\textitbf{tugas} & ‘duty’ & \textitbfUndl{petugas} & ‘official’ & \raggedleft 5 & \raggedleft\arraybslash 19\\
\textitbf{usaha} & ‘effort’ & \textitbfUndl{pengusaha} & ‘entrepreneur’ & \raggedleft 3 & \raggedleft\arraybslash 2\\
\textitbf{acara} & ‘ceremony’ & \textitbfUndl{pengacara} & ‘master of ceremony’ & \raggedleft 1 & \raggedleft\arraybslash 40\\
\textitbf{rencana} & ‘plan’ & \textitbfUndl{perencana} & ‘planner’ & \raggedleft 1 & \raggedleft\arraybslash 17\\
\lspbottomrule
\end{tabular}

In (0) and (0) one of the prefixed nouns and its nominal base are given in context, namely \textitbf{pemrinta} ‘government’ and \textitbf{printa} ‘command’, respectively.


\begin{tabular}{lllllllll}
\lsptoprule
\label{bkm:Ref346798720}
\gll {kalo} {de} {bilang} {spulu} {milyar} {\bluebold{pemrinta}} {sanggup} {bayar}\\ %
&  &  &  &  &  & pem–printa &  & \\
& if & \textsc{3sg} & say & ten & billion & \textsc{ag}–command & be.capable & pay\\
\lspbottomrule
\end{tabular}
\ea
\glt 
‘if he demands ten billion (then) the \bluebold{government} is capable of paying’ \textstyleExampleSource{[081029-004-Cv.0073]}
\z

\begin{tabular}{lllllll}
\lsptoprule
\label{bkm:Ref346798719}
\gll {masi} {banyak} {yang} {melangar} {\bluebold{printa{\Tilde}printa}} {Tuhang}\\ %
& still & many & \textsc{rel} & collide.with & \textsc{rdp}{\Tilde}command & God\\
\lspbottomrule
\end{tabular}
\ea
\glt
‘(there are) still many who violate God’s \bluebold{commands}’ \textstyleExampleSource{[081014-014-NP.0050]}
\end{styleFreeTranslEngxvpt}

\paragraph[Variables of the communicative event]{Variables of the communicative event}
\label{bkm:Ref346719142}
To examine the issue of productivity of \textscItalBold{pe(n)\-} in Papuan Malay from a different perspective, a domain analysis was conducted which focused on the variables of speaker education levels, topics, and/or role-relations (for details see ‘Language external factors’ in §3.1.1, p. \pageref{bkm:Ref346712254}). In all, 34 items prefixed with \textscItalBold{pe(n)\-}, totaling 186 tokens, were investigated:


\begin{itemize}
\item \begin{styleIIndented}
29 prefixed items with verbal bases (153 tokens)
\end{styleIIndented}\item \begin{styleIvI}
Five prefixed items with nominal bases (33 tokens)
\end{styleIvI}\end{itemize}

For the 34 prefixed lexemes, most tokens (167/186 – 90\%) can be accounted for in terms of speaker education levels, topics, and/or role-relations. The remaining 19/186 tokens (10\%) cannot be explained in terms of these variables of the communicative event. These tokens occurred when less-educated speakers (\textsc{\-edc-spk}) conversed with fellow-Papuans of equally low social standing (\textsc{\-stat}) about \textsc{low} topics, that is, casual daily-life issues.\footnote{\\
\\
\\
\\
\\
\\
\\
As mentioned under Factor 3 ‘Relationships between interlocutors’ in §1.5.1 (p. \pageref{bkm:Ref376429590}), all of the recorded less-educated speakers belong to the group of Papuans with lower social status (\textsc{\-stat}), while the recorded Papuans with higher social status (\textsc{+stat}), such as teachers, government officials, or pastors, are all better educated.\\
\\
\\
\\
\\
\\
\\
\\
} (See Table  ‎3 .13 and Figure  ‎3 .8.)



If the prefixed lexemes were the result of a productive affixation process, one would expect the percentage of tokens that cannot be explained in terms of speaker education levels, topics, and/or role-relations to be much higher than 10\%. Instead, most tokens (90\%) seem to be conditioned by these variables of the communicative event. These findings do not support the conclusion that the respective lexemes are the result of a productive derivation process. Instead, they seem to be code-switches with Indonesian.
\end{styleBodyvafter}


The data presented in Table  ‎3 .13 and Figure  ‎3 .8 is discussed in more detail below.
\end{styleBodyvvafter}

\begin{stylecaption}
\label{bkm:Ref341112625}Table ‎3.\stepcounter{Table}{\theTable}:  Token frequencies for \textscItalBold{pe(n)\-}prefixed lexemes with verbal and nominal bases by speakers, topics, and interlocutors (34 items)
\end{stylecaption}

\tablehead{ & \multicolumn{4}{l}{ Topics (\textsc{top})} & \multicolumn{3}{l}{ Interlocutors (\textsc{ilct})} & \arraybslash Tokens\\
}
\begin{tabular}{lllllllll}
\lsptoprule
\multicolumn{9}{l}{Prefixed lexemes with verbal bases (29 items)}\\
& \textsc{pol} & \textsc{edc} & \textsc{rel} & \textsc{low} & \textsc{+stat} & \textsc{\-stat} & \textsc{outsd} & \arraybslash Total\\
\textsc{+edc-spk} & \raggedleft 37 & \raggedleft 6 & \raggedleft 3 & \raggedleft 19 & \raggedleft {}-{}-{}- & \raggedleft {}-{}-{}- & \raggedleft 11 & \raggedleft\arraybslash 76\\
\textsc{\-edc-spk} & \raggedleft 11 & \raggedleft 2 & \raggedleft 37 & \raggedleft {}-{}-{}- & \raggedleft 9 & \raggedleft \textstyleChBold{\textmd{18}} & \raggedleft 0 & \raggedleft\arraybslash 77\\
Subtotal & \raggedleft 48 & \raggedleft 8 & \raggedleft 40 & \raggedleft 19 & \raggedleft 9 & \raggedleft \textstyleChBold{18} & \raggedleft 11 & \raggedleft\arraybslash 153\\
\multicolumn{9}{l}{Prefixed lexemes with nominal bases (5 items)}\\
& \textsc{pol} & \textsc{edc} & \textsc{rel} & \textsc{low} & \textsc{+stat} & \textsc{\-stat} & \textsc{outsd} & \arraybslash Total\\
\textsc{+edc-spk} & \raggedleft 10 & \raggedleft 0 & \raggedleft 12 & \raggedleft 5 & \raggedleft {}-{}-{}- & \raggedleft {}-{}-{}- & \raggedleft 0 & \raggedleft\arraybslash 27\\
\textsc{\-edc-spk} & \raggedleft 1 & \raggedleft 2 & \raggedleft 2 & \raggedleft {}-{}-{}- & \raggedleft 0 & \raggedleft \textstyleChBold{1} & \raggedleft 0 & \raggedleft\arraybslash 6\\
Subtotal & \raggedleft 11 & \raggedleft 2 & \raggedleft 14 & \raggedleft 5 & \raggedleft 0 & \raggedleft \textstyleChBold{1} & \raggedleft 0 & \raggedleft\arraybslash 33\\
\multicolumn{9}{l}{\textstyleChBold{TOTAL} (34 items)}\\
& \textsc{pol} & \textsc{edc} & \textsc{rel} & \textsc{low} & \textsc{+stat} & \textsc{\-stat} & \textsc{outsd} & \arraybslash Total\\
\textsc{+edc-spk} & \raggedleft 47 & \raggedleft 6 & \raggedleft 15 & \raggedleft 24 & \raggedleft {}-{}-{}- & \raggedleft {}-{}-{}- & \raggedleft 11 & \raggedleft\arraybslash 103\\
\textsc{\-edc-spk} & \raggedleft 12 & \raggedleft 4 & \raggedleft 39 & \raggedleft {}-{}-{}- & \raggedleft 9 & \raggedleft \textstyleChBold{19} & \raggedleft 0 & \raggedleft\arraybslash 83\\
\textstyleChBold{Total} & \raggedleft 59 & \raggedleft 10 & \raggedleft 54 & \raggedleft 24 & \raggedleft 9 & \raggedleft \textstyleChBold{19} & \raggedleft 11 & \raggedleft\arraybslash \textstyleChBold{186}\\
\lspbottomrule
\end{tabular}
\begin{styleFigure}
  
%%please move the includegraphics inside the {figure} environment
%%\includegraphics[width=\textwidth]{kluge-img4.jpg}
 
\end{styleFigure}

\begin{styleCapFigure}
\label{bkm:Ref417129179}Figure ‎3.\stepcounter{Figure}{\theFigure}:  Token frequencies for \textscItalBold{pe(n)\-}prefixed lexemes with verbal and nominal bases by speakers, topics, and interlocutors
\end{styleCapFigure}


The data given in Table  ‎3 .13 and Figure  ‎3 .8 shows that for the 34 \textscItalBold{pe(n)\-}prefixed lexemes, most tokens (167/186 – 90\%) can be explained in terms of speaker education levels, topics, and/or role-relations between the speakers and their interlocutors; this total includes 135/153 (88\%) tokens with verbal and 32/33 tokens (97\%) with nominal bases.
\end{styleBodyaftervbefore}


More than half of the tokens were produced by better-educated speakers (\textsc{+edc-spk}) (103/186 – 55\%), while less-educated speakers (\textsc{\-edc-spk}) produced 83/186 tokens (45\%).
\end{styleBodyvafter}


Two thirds of the 186 tokens (123/186 – 66\%) occurred during conversations about \textsc{high} topics, that is, political, educational or religious affairs (\textsc{pol}, \textsc{edc} and \textsc{rel}, respectively). This includes 68/103 tokens (66\%) produced by \textsc{+edc-spk} and 55/83 tokens (66\%) produced by \textsc{\-edc-spk}. In addition, 11/186 tokens (6\%) occurred during conversations with an outsider, namely the author (\textsc{outsd}), all of them being \textsc{+edc-spk} tokens (11/103 – 11\%).
\end{styleBodyvafter}


This leaves 52/186 tokens (28\%) that were produced when the interlocutors discussed \textsc{low} topics. This includes 24/103 \textsc{+edc-spk} tokens (23\%) and 28/83 \textsc{\-edc-spk} tokens (34\%). The 28 \textsc{low} topic tokens produced by \textsc{\-edc-spk} are distributed as follows. Nine tokens occurred during conversations with \textsc{+stat} Papuans (that is, 9/83 \textsc{\-edc-spk} tokens – 11\%). The remaining 19 tokens (that is, 19/83 \textsc{\-edc-spk} tokens – 23\%) occurred when \textsc{\-edc-spk} discussed \textsc{low} topics with \textsc{\-stat} Papuans, and therefore cannot be explained in terms of speaker education levels, topics, and/or role-relations. This total of 19 tokens refers to 10\% of all 186 \textscItalBold{pe(n)\-}tokens, including 18/153 tokens (12\%) with verbal bases and 1/33 tokens (3\%) with nominal bases.\footnote{\\
\\
\\
\\
\\
\\
\\
Concerning the 11 hapaxes (nine with verbal and two with nominal bases), the data suggests that seven are conditioned by the variables of speaker education levels, topics, and/or role-relations, and therefore are best explained as code-switches with Standard Indonesian. This leaves only four hapaxes (with verbal bases) that cannot be accounted for in terms of language external factors and that are likely to result from a productive word-formation process. For four hapaxes P=0.0376 as opposed to P=0.0591 for 11 hapaxes (N=186).\\
\\
\\
\\
\\
\\
\\
\\
}
\end{styleBodyvxvafter}

\paragraph[Summary and conclusions]{Summary and conclusions}
\label{bkm:Ref346719140}
Prefix \textscItalBold{pe(n)\-} is polyfunctional, in that it derives nouns from verbal and nominal bases. This polyfunctionality suggests that affixation with \textscItalBold{pe(n)\-} is a somewhat productive process (see language internal factor (c in §3.1.1, p. \pageref{bkm:Ref364758123}).



Concerning \textscItalBold{pe(n)\-}affixation of verbal bases, three other observations support this conclusion: (1) the transparent form-function relationship between the derived nouns and their respective bases, (2) the large number of low frequency words and small number of high frequency words, and (3) the relative token frequencies with most bases having higher frequencies than the affixed lexemes. On the other hand, more than half of the derived lexemes were tentatively classified as SI-borrowings. These observations suggest that productivity of the affixation process is rather limited.
\end{styleBodyvafter}


As for \textscItalBold{pe(n)\-}affixation of nominal bases, two observations suggest that this is a productive process: (1) most of the derived lexemes are low frequency words, and (2) most bases have higher token frequencies than the affixed lexemes. On the other hand, almost all derived lexemes were tentatively classified as SI-borrowings. These findings suggest that \textscItalBold{pe(n)\-}affixation of nominal bases has limited productivity
\end{styleBodyvafter}


As for the speech situations during which the derived nouns occurred, the vast majority of the attested tokens are conditioned by the variables of speaker education levels, topics, and/or role-relations. Hence, these items are best explained as code-switches with Indonesian.
\end{styleBodyvafter}


These findings suggest that in Papuan Malay\textscItalBold{ pe(n)\-}affixation has, at best, marginal productivity.
\end{styleBodyvxvafter}

\subsection{Prefix \textscItalBold{ber}\textscItalBold{\-} ‘\textsc{vblz}’}
\label{bkm:Ref347514355}
Prefix \textscItalBold{ber\-} ‘\textsc{vblz}’ is typically attached to verbal bases, as in (0), or to nominal bases. Besides, the corpus also includes a few lexical items with numeral and quantifier bases. The prefixed lexemes have a verbal reading. As shown throughout this section, however, affixation with \textscItalBold{ber\-} ‘\textsc{vblz}’ is not used as a productive derivation device in Papuan Malay.
\end{styleBodyxafter}

\begin{tabular}{lllllllll}
\lsptoprule
\label{bkm:Ref346868496}
\gll {…} {waktu} {saya} {\bluebold{ber–buru}} {saya} {perlu} {makang} {pinang}\\ %
&  & time & \textsc{1sg} & \textsc{vblz}–hunt & \textsc{1sg} & need & eat & betel.nut\\
\lspbottomrule
\end{tabular}
\ea
\glt 
‘… when I \bluebold{hunt} I need to chew betel nuts’ \textstyleExampleSource{[080919-004-NP.0011]}
\z


The corpus contains 62 derived nouns (602 tokens) prefixed with \textscItalBold{ber\-}:\footnote{\\
\\
\\
\\
\\
\\
\\
The 62 verbs include 25 hapaxes (P=0.0415); the 29 verbs with verbal bases include 11 hapaxes (P=0.0484); the 33 verbs with nominal, numeral, or quantifier bases include 14 hapaxes (P=0.0373).\\
\\
\\
\\
\\
\\
\\
\\
}


%\setcounter{itemize}{0}
\begin{itemize}
\item \begin{styleIIndented}
Verbs with verbal bases (29 items with 227 tokens)
\end{styleIIndented}\item \begin{styleIvI}
Verbs with nominal, numeral, and quantifier bases (33 items with 375 tokens)
\end{styleIvI}\end{itemize}

The corpus also includes 16 formally complex words with non-compositional semantics, such as \textitbf{bertriak} ‘scream’, \textitbf{berjuang} ‘struggle’, or \textitbf{berlabu} ‘anchor’.



Before discussing \textscItalBold{ber\-}affixation of verbal bases in §3.1.5.2 and of nominal bases in §3.1.5.3, the allomorphy of \textscItalBold{ber\-} is investigated in §3.1.5.1. Pertinent variables of the communicative event that may impact the use of \textscItalBold{ber\-} are explored in §3.1.8. The main findings on prefix \textscItalBold{ber\-} are summarized and evaluated in §3.1.5.4.
\end{styleBodyvxvafter}

\paragraph[Allomorphy of ber\-]{Allomorphy of \textscItalBold{ber\-}}
\label{bkm:Ref438906018}\label{bkm:Ref354566129}\label{bkm:Ref326303487}
Prefix \textscItalBold{ber\-} has two allomorphs, \textitbf{ber}\- and \textitbf{ba}\-. The allomorphs are not governed by phonological processes.



The form \textitbf{ber}\-, in turn, has four realizations that are effected by morphologically conditioned phonological rules. More specifically, the four allomorphs are conditioned by the word-initial segment of the base word, as illustrated in Table  ‎3 .14: /\textstyleChCharisSIL{bɛr\-}/, /\textstyleChCharisSIL{br\-}/, /\textstyleChCharisSIL{bl\-}/, and /\textstyleChCharisSIL{bɛ\-}/. The prefix is typically realized as /\textstyleChCharisSIL{bɛr\-}/. With an onset vowel, however, \textitbf{ber}\- is very commonly realized as /br\textstyleChCharisSIL{\-}/. When prefixed to \textitbf{ajar} ‘teach’ the prefix is realized as /\textstyleChCharisSIL{bl\-}/, while it is realized as /\textstyleChCharisSIL{bɛ\-}/ when affixed to \textitbf{kerja} ‘work’ or \textitbf{brapa} ‘several’.


\begin{stylecaption}
\label{bkm:Ref335755577}Table ‎3.\stepcounter{Table}{\theTable}:  Realizations of allomorph \textitbf{ber}\-
\end{stylecaption}

\tablehead{
 \textitbf{ber}\textsc{\-}base & Orthogr. & \arraybslash Gloss\\
}
\begin{tabular}{lll}
\lsptoprule
/\textstyleChCharisSIL{bɛr}–\textstyleChCharisSIL{dʒuaŋ}/ & \textitbf{berjuang} & ‘struggle (for)’\\
/\textstyleChCharisSIL{br}–\textstyleChCharisSIL{aŋkat}/ & \textitbf{brangkat} & ‘leave’\\
/\textstyleChCharisSIL{bl}–\textstyleChCharisSIL{adʒar}/ & \textitbf{blajar} & ‘study’\\
/\textstyleChCharisSIL{bɛ}–\textstyleChCharisSIL{kɛrdʒa}/ & \textitbf{bekerja} & ‘work’\\
/\textstyleChCharisSIL{bɛ}–\textstyleChCharisSIL{brapa}/ & \textitbf{bebrapa} & ‘be several’\\
\lspbottomrule
\end{tabular}

Allomorph \textitbf{ba}\- occurs much less frequently. Attested are only the 15 items listed in Table  ‎3 .15 with a total of 37 tokens. Some of these items are alternatively realized with allomorph \textitbf{ber}\-. Therefore, for each item the token frequencies for \textitbf{ba}\- and for \textitbf{ber}\- are given. If in a greater number of tokens the prefix is realized with /\textstyleChCharisSIL{ba\-}/ rather than with /\textstyleChCharisSIL{bɛr\-}/, then its orthographic representation is \textitbf{ba}\- as in \textitbf{bakalay} ‘fight’. If both realizations have the same token frequencies, then the orthographic representation follows its realization in the recorded texts, as in \textitbf{bergaya} ‘put on airs’.


\begin{stylecaption}
\label{bkm:Ref326244473}Table ‎3.\stepcounter{Table}{\theTable}:  Realizations of allomorph \textitbf{ba}\-
\end{stylecaption}

\tablehead{
 \textitbf{ba\-}base & Orthogr. & Gloss & \textitbf{ba\-} \# & \arraybslash \textitbf{ber\-} \#\\
}
\begin{tabular}{lllll}
\lsptoprule
/\textstyleChCharisSIL{ba}–\textstyleChCharisSIL{kalaj}/ & \textitbf{bakalay} & ‘fight’ & \raggedleft 19 & \raggedleft\arraybslash 0\\
/\textstyleChCharisSIL{ba}–\textstyleChCharisSIL{taria}/ & \textitbf{bertriak}\footnotemark{} & ‘scream’ & \raggedleft 3 & \raggedleft\arraybslash 18\\
/\textstyleChCharisSIL{ba}–\textstyleChCharisSIL{biŋuŋ}/ & \textitbf{babingung} & ‘be confused’ & \raggedleft 2 & \raggedleft\arraybslash 0\\
/\textstyleChCharisSIL{ba}–\textstyleChCharisSIL{diam}/ & \textitbf{badiam} & ‘be quiet’ & \raggedleft 2 & \raggedleft\arraybslash 0\\
/\textstyleChCharisSIL{ba}–\textstyleChCharisSIL{diri}/ & \textitbf{berdiri} & ‘stand’ & \raggedleft 1 & \raggedleft\arraybslash 54\\
/\textstyleChCharisSIL{ba}–\textstyleChCharisSIL{dara}/ & \textitbf{berdara} & ‘bloody & \raggedleft 1 & \raggedleft\arraybslash 1\\
/\textstyleChCharisSIL{ba}–duri/ & \textitbf{berduri} & ‘be thorny’ & \raggedleft 1 & \raggedleft\arraybslash 1\\
/\textstyleChCharisSIL{ba}–\textstyleChCharisSIL{gaja}/ & \textitbf{bergaya} & ‘put on airs’ & \raggedleft 1 & \raggedleft\arraybslash 1\\
/\textstyleChCharisSIL{ba}–gisi/ & \textitbf{bergisi} & ‘be nutritious’ & \raggedleft 1 & \raggedleft\arraybslash 1\\
/\textstyleChCharisSIL{ba}–jalang/ & \textitbf{berjalang} & ‘walk’ & \raggedleft 1 & \raggedleft\arraybslash 1\\
/\textstyleChCharisSIL{ba}–ribut/ & \textitbf{beribut} & ‘be noisy’ & \raggedleft 1 & \raggedleft\arraybslash 1\\
/\textstyleChCharisSIL{ba}–\textstyleChCharisSIL{gigit}/ & \textitbf{bagigit} & ‘bite’ & \raggedleft 1 & \raggedleft\arraybslash 0\\
/\textstyleChCharisSIL{ba}–\textstyleChCharisSIL{kumis}/ & \textitbf{bakumis} & ‘have a beard’ & \raggedleft 1 & \raggedleft\arraybslash 0\\
/\textstyleChCharisSIL{ba}–\textstyleChCharisSIL{isi}/ & \textitbf{baisi} & ‘be muscular’ & \raggedleft 1 & \raggedleft\arraybslash 0\\
/\textstyleChCharisSIL{ba}–\textstyleChCharisSIL{mɛkap}/ & \textitbf{bamekap} & ‘wear make-up’ & \raggedleft 1 & \raggedleft\arraybslash 0\\
\lspbottomrule
\end{tabular}
\footnotetext{\\
\\
\\
\\
\\
\\
\\
The root is realized as /\textstyleChCharisSILviiivpt{triak}/ when speakers employ allomorph \textitbf{ber}\-, whereas it is realized as /\textstyleChCharisSILviiivpt{taria}/ when speakers use allomorph \textitbf{ba}\-.\\
\\
\\
\\
\\
\\
\\
\\
}

In realizing prefix \textscItalBold{ber\-} most commonly as allomorph \textitbf{ber}\- rather than as \textitbf{ba}\-, Papuan Malay contrasts with other eastern Malay varieties such as Ambon Malay {(van Minde 1997: 95)}, Banda Malay {\citep[249]{Paauw2009}}, Kupang Malay {\citep[46]{Steinhauer1983}}, Manado Malay {\citep[18]{Stoel2005}}, and North Moluccan / Ternate Malay ({Taylor 1983: 18}{;} {Voorhoeve 1983: 4}{;} {Litamahuputty 2012: 125}). In these varieties the prefix is always realized as \textitbf{ba}\-. Instead, the items prefixed with \textscItalBold{ber\-} have more resemblance with the corresponding Indonesian items where the prefix is realized as \textitbf{ber}\-. In addition, in Larantuka Malay the prefix is also realized as \textitbf{bə(r)}\- {\citep[253]{Paauw2009}}. Again this difference between Papuan Malay and the other eastern Malay varieties points to the distinct histories of both, discussed in §1.8.


\paragraph[Prefixed items derived from verbal bases]{Prefixed items derived from verbal bases}
\label{bkm:Ref341105426}
The corpus includes 29 \textscItalBold{ber\-}prefixed lexemes (with 227 tokens) with verbal bases, as listed in Table  ‎3 .16. Of the 29 lexemes, 11 have monovalent bases such as stative \textitbf{diam} ‘be quiet’ or dynamic \textitbf{jalang} ‘walk’. The remaining 18 items have bivalent bases. Of these 18 prefixed lexemes, five have monotransitive as well as intransitive uses, while 11 lexemes have intransitive uses only. These 16 lexemes have the same semantics as their bivalent bases, as shown in (0) to (0). For the remaining two prefixed lexemes the semantics are distinct from those of their bases, as shown in (0) to (0). One of them has monotransitive as well as intransitive uses, while the other one has intransitive uses only.



Almost all of the derived lexemes are low frequency words (27 lexemes, attested with less than 20 tokens). Moreover, the token frequencies for the respective bases are (much) higher for most of the derived words (22 lexemes). This is due to the fact that the affixed lexemes and the bases have the same semantics and that, overall, speakers tend to use the bases rather than the prefixed forms, as shown in (0) to (0). Also, most of the 29 prefixed lexemes (24 items) were tentatively classified as borrowings from Standard Indonesian (SI-borrowings) (for details see language internal factor (f in §3.1.1, p. \pageref{bkm:Ref346541608}); in Table  ‎3 .16 these items are underlined.
\end{styleBodyvvafter}

\begin{stylecaption}
\label{bkm:Ref325989674}Table ‎3.\stepcounter{Table}{\theTable}:  Affixation with \textscItalBold{ber\-} of verbal bases
\end{stylecaption}

\tablehead{
\multicolumn{2}{l}{ BW} & \multicolumn{2}{l}{ Item} & Gloss & \textscItalBold{ber\-} \# & \arraybslash BW \#\\
}
\begin{tabular}{lllllll}
\lsptoprule
\multicolumn{7}{l}{Monovalent bases: Bases and prefixed lexemes with same semantics}\\
& \textitbf{tobat} & \multicolumn{2}{l}{\textitbfUndl{bertobat}} & ‘repent’ & \raggedleft 8 & \raggedleft\arraybslash 1\\
& \textitbf{beda} & \multicolumn{2}{l}{\textitbfUndl{berbeda}} & ‘be different’ & \raggedleft 7 & \raggedleft\arraybslash 34\\
& \textitbf{tanggung-jawap} & \multicolumn{2}{l}{\textitbfUndl{bertanggung-jawap}} & ‘be responsible’ & \raggedleft 5 & \raggedleft\arraybslash 6\\
& \textitbf{bahaya} & \multicolumn{2}{l}{\textitbfUndl{berbahaya}} & ‘be dangerous’ & \raggedleft 3 & \raggedleft\arraybslash 3\\
& \textitbf{diam} & \multicolumn{2}{l}{\textitbfUndl{badiam}} & ‘be quiet’ & \raggedleft 2 & \raggedleft\arraybslash 60\\
& \textitbf{bingung} & \multicolumn{2}{l}{\textitbfUndl{berbingung}} & ‘be confused’ & \raggedleft 2 & \raggedleft\arraybslash 30\\
& \textitbf{jalang} & \multicolumn{2}{l}{\textitbfUndl{berjalang}} & ‘walk’ & \raggedleft 1 & \raggedleft\arraybslash 480\\
& \textitbf{ibada} & \multicolumn{2}{l}{\textitbfUndl{beribada}} & ‘worship’ & \raggedleft 1 & \raggedleft\arraybslash 11\\
& \textitbf{sandar} & \multicolumn{2}{l}{\textitbfUndl{bersandar}} & ‘lean’ & \raggedleft 1 & \raggedleft\arraybslash 6\\
& \textitbf{hati{\Tilde}hati} & \multicolumn{2}{l}{\textitbfUndl{berhati{\Tilde}hati}} & ‘be careful’ & \raggedleft 1 & \raggedleft\arraybslash 5\\
& \textitbf{pisa} & \multicolumn{2}{l}{\textitbfUndl{berpisa}} & ‘be separate’ & \raggedleft 1 & \raggedleft\arraybslash 4\\
\multicolumn{7}{l}{Bivalent bases: Bases and prefixed lexemes with same semantics}\\
\multicolumn{7}{l}{Prefixed lexemes: Monotransitive and intransitive uses}\\
& \textitbf{buru} & \multicolumn{2}{l}{\textitbf{berburu}} & ‘hunt’ & \raggedleft \textstyleChBold{\textmd{10}} & \raggedleft\arraybslash 5\\
& \textitbf{buat} & \multicolumn{2}{l}{\textitbfUndl{berbuat}} & ‘make’ & \raggedleft \textstyleChBold{\textmd{7}} & \raggedleft\arraybslash 100\\
& \textitbf{pikir} & \multicolumn{2}{l}{\textitbfUndl{berpikir}} & ‘think’ & \raggedleft \textstyleChBold{\textmd{8}} & \raggedleft\arraybslash 102\\
& \textitbf{harap} & \multicolumn{2}{l}{\textitbfUndl{berharap}} & ‘hope’ & \raggedleft 1 & \raggedleft\arraybslash 8\\
& \textitbf{ribut} & \multicolumn{2}{l}{\textitbf{bribut}} & ‘trouble’ & \raggedleft 1 & \raggedleft\arraybslash 5\\
\multicolumn{7}{l}{Prefixed lexemes: Monotransitive uses}\\
& \textitbf{bicara} & \textitbfUndl{berbicara} & \multicolumn{2}{l}{‘speak’} & \raggedleft \textstyleChBold{\textmd{7}} & \raggedleft\arraybslash 333\\
& \textitbf{kerja} & \textitbfUndl{bekerja} & \multicolumn{2}{l}{‘work’} & \raggedleft \textstyleChBold{\textmd{5}} & \raggedleft\arraybslash 191\\
& \textitbf{tahang} & \textitbfUndl{bertahang} & \multicolumn{2}{l}{‘hold (out/back)’} & \raggedleft \textstyleChBold{\textmd{5}} & \raggedleft\arraybslash 48\\
& \textitbf{uba} & \textitbf{bruba} & \multicolumn{2}{l}{‘change’} & \raggedleft \textstyleChBold{\textmd{5}} & \raggedleft\arraybslash 9\\
& \textitbf{gabung} & \textitbfUndl{bergabung} & \multicolumn{2}{l}{‘join’} & \raggedleft \textstyleChBold{\textmd{4}} & \raggedleft\arraybslash 3\\
& \textitbf{maing} & \textitbfUndl{bermaing} & \multicolumn{2}{l}{‘play’} & \raggedleft 3 & \raggedleft\arraybslash 113\\
& \textitbf{tindak} & \textitbfUndl{bertindak} & \multicolumn{2}{l}{‘act’} & \raggedleft 2 & \raggedleft\arraybslash 1\\
& \textitbf{ikut} & \textitbfUndl{brikut} & \multicolumn{2}{l}{‘follow’} & \raggedleft 1 & \raggedleft\arraybslash 259\\
& \textitbf{kumpul} & \textitbfUndl{berkumpul} & \multicolumn{2}{l}{‘gather’} & \raggedleft 1 & \raggedleft\arraybslash 16\\
& \textitbf{bentuk} & \textitbfUndl{berbentuk} & \multicolumn{2}{l}{‘form’} & \raggedleft 1 & \raggedleft\arraybslash 12\\
& \textitbf{gigit} & \textitbfUndl{bergigit} & \multicolumn{2}{l}{‘bite’} & \raggedleft 1 & \raggedleft\arraybslash 10\\
\multicolumn{7}{l}{Bivalent bases: Bases and prefixed lexemes with distinct semantics}\\
\multicolumn{7}{l}{Prefixed lexeme: Monotransitive and intransitive uses}\\
& \textitbf{ajar} (‘teach’) & \multicolumn{2}{l}{\textitbf{blajar}} & ‘study’ & \raggedleft \textstyleChBold{\textmd{51}} & \raggedleft\arraybslash 41\\
\multicolumn{7}{l}{Prefixed lexeme: Monotransitive uses}\\
& \textitbf{angkat} (‘lift’) & \multicolumn{2}{l}{\textitbf{brangkat}} & ‘leave’ & \raggedleft \textstyleChBold{\textmd{82}} & \raggedleft\arraybslash 81\\
\lspbottomrule
\end{tabular}

Affixation with \textscItalBold{ber\-} of verbal bases derives lexemes that typically have the same semantics as their respective bases, with \textscItalBold{ber\-} being glossed as ‘\textsc{vblz}’ (‘verbalizer’). This applies to \textscItalBold{ber\-}prefixed lexemes with monovalent and with bivalent bases.
\end{styleBodyaftervbefore}


The fact that monovalent bases derive \textscItalBold{ber\-}prefixed lexemes with the same semantics is illustrated with stative \textitbf{bingung} ‘be confused’ and prefixed \textitbf{berbingung} ‘be confused’ in (0) and (0), and with dynamic \textitbf{ibada} ‘worship’ and prefixed \textitbf{beribada} ‘worship’ in (0) and (0), respectively.
\end{styleBodyvxafter}

\begin{styleExampleTitle}
Prefix \textscItalBold{ber\-}: Semantics of verbal bases and derived lexemes
\end{styleExampleTitle}

\begin{tabular}{llllllllll}
\lsptoprule
\label{bkm:Ref346902714}
\gll {memang} {sa} {punya} {ade} {sa} {juga} {\bluebold{bingung}} {dengang} {dia}\\ %
& indeed & \textsc{1sg} & \textsc{poss} & ySb & \textsc{1sg} & also & be.confused & with & \textsc{3sg}\\
\lspbottomrule
\end{tabular}
\ea
\glt 
‘indeed (he was) my younger cousin, I’m also \bluebold{confused} about him’ \textstyleExampleSource{[080918-001-CvNP.0014]}
\z

\begin{tabular}{lllllllll}
\lsptoprule
\label{bkm:Ref346902713}
\gll {\multicolumn{2}{l}{nanti}} {di} {skola} {baru} {kamu} {\bluebold{ba–bingung}} {dengang}\\ %
& \multicolumn{2}{l}{very.soon} & at & school & and.then & \textsc{2pl} & \textsc{vblz}–be.confused & with\\
& bahasa & \multicolumn{7}{l}{Inggris}\\
& language & \multicolumn{7}{l}{English}\\
\lspbottomrule
\end{tabular}
\ea
\glt 
[Addressing lazy students:] ‘later in school, then you’ll be \bluebold{confused} about English’ \textstyleExampleSource{[081115-001a-Cv.0151]}
\z

\begin{tabular}{lllllll}
\lsptoprule
\label{bkm:Ref346902712}
\gll {orang} {jalang} {itu} {mo} {pergi} {\bluebold{ibada}}\\ %
& person & walk & \textsc{d.dist} & want & go & worship\\
\lspbottomrule
\end{tabular}
\ea
\glt 
[About a youth retreat:] ‘people doing that traveling want to go (and) \bluebold{worship}’ \textstyleExampleSource{[081006-016-Cv.0017]}
\z

\begin{tabular}{lllllllll}
\lsptoprule
\label{bkm:Ref346902711}
\gll {\multicolumn{2}{l}{nanti}} {\multicolumn{2}{l}{kita}} {\bluebold{ber–ibada}} {selesay} {malam} {ka}\\ %
& \multicolumn{2}{l}{very.soon} & \multicolumn{2}{l}{\textsc{1pl}} & \textsc{vblz}–worship & finish & night & maybe\\
& baru & \multicolumn{2}{l}{sa} & \multicolumn{5}{l}{pergi}\\
& and.then & \multicolumn{2}{l}{\textsc{1sg}} & \multicolumn{5}{l}{go}\\
\lspbottomrule
\end{tabular}
\ea
\glt 
‘later, after we have \bluebold{worshipped}, maybe in the evening, and then I’ll go (there)’ \textstyleExampleSource{[080918-001-CvNP.0016]}\footnote{\\
\\
\\
\\
\\
\\
\\
The original recording says \textitbf{kita i beribada selesay}. Most likely the speaker wanted to say \textitbf{kita ibada selesay} ‘after we have worshipped’ but cut himself off to replace \textitbf{ibada} ‘worship’ with \textitbf{beribada} ‘worship’.\\
\\
\\
\\
\\
\\
\\
\\
}
\z


Bivalent bases also derive \textscItalBold{ber\-}prefixed lexemes that have the same semantics as their bases, as shown in (0) to (0). As discussed in §11.1.2, bivalent verbs have not only monotransitive but also intransitive uses. The same applies to some of the \textscItalBold{ber\-}prefixed lexemes, as illustrated in (0) to (0).
\end{styleBodyxafter}

\begin{styleExampleTitle}
Prefix \textscItalBold{ber\-}: Same semantics of verbal bases and derived lexemes
\end{styleExampleTitle}

\begin{tabular}{llllllll}
\lsptoprule
\label{bkm:Ref346954401}
\gll {jadi} {kitorang} {bingung} {\bluebold{pikir}} {itu} {pen–jaga} {kubur\bluebold{–}ang}\\ %
& so & \textsc{1pl} & be.confused & think & \textsc{d.dist} & \textsc{ag}–guard & bury–\textsc{pat}\\
\lspbottomrule
\end{tabular}
\ea
\glt 
‘so we’re confused to \bluebold{think (about)}, what’s-its-name, a guard (for) the grave’ \textstyleExampleSource{[080923-007-Cv.0024]}
\z

\begin{tabular}{llllllllll}
\lsptoprule
(\stepcounter{}{\the}) & … & tapi & ana{\Tilde}ana & ni & dong & tida & taw & \bluebold{ber–pikir} & itu\\
&  & but & \textsc{rdp}{\Tilde}child & \textsc{d.prox} & \textsc{3pl} & \textsc{neg} & know & \textsc{vblz}–think & \textsc{d.dist}\\
\lspbottomrule
\end{tabular}
\ea
\glt 
[About impolite teenagers:] ‘… but these kids they don’t know (how) to \bluebold{think (about)} those (feelings of mine)’ \textstyleExampleSource{[081115-001b-Cv.0037]}
\z

\begin{tabular}{llllllll}
\lsptoprule
(\stepcounter{}{\the}) & \multicolumn{2}{l}{skarang} & orang & su & tra & \bluebold{pikir} & tentang\\
& \multicolumn{2}{l}{now} & person & already & \textsc{neg} & think & about\\
& hal & \multicolumn{6}{l}{ke–benar–ang}\\
& thing & \multicolumn{6}{l}{\textsc{nmlz}–be.true–\textsc{nmlz}}\\
\lspbottomrule
\end{tabular}
\ea
\glt 
‘nowadays, the people already don’t \bluebold{think} about things (related to) truth’ \textstyleExampleSource{[081006-032-Cv.0016]}
\z

\begin{tabular}{lllllllll}
\lsptoprule
\label{bkm:Ref346954398}
\gll {…} {karna} {dia} {\bluebold{ber–pikir}} {tentang} {dia} {punya} {badang}\\ %
&  & because & \textsc{3sg} & \textsc{vblz}–think & about & \textsc{3sg} & \textsc{poss} & body\\
\lspbottomrule
\end{tabular}
\ea
\glt 
‘[she doesn’t think about serving my or her guests] because she \bluebold{thinks} about her body’ \textstyleExampleSource{[081006-032-Cv.0062]}
\z


Most \textscItalBold{ber\-}prefixed lexemes with bivalent bases, however, have intransitive uses only, while their bases can be used mono- or intransitively. This is illustrated with \textitbf{bicara} ‘speak’ and \textitbf{berbicara} ‘speak’ in (0) to (0).
\end{styleBodyxafter}

\begin{styleExampleTitle}
Prefix \textscItalBold{ber\-}: Semantics and distribution of verbal bases and derived lexemes
\end{styleExampleTitle}

\begin{tabular}{llllllll}
\lsptoprule
\label{bkm:Ref348359811}
\gll {baru} {de} {\bluebold{bicara}} {sa} {deng} {bahasa} {Inggris}\\ %
& and.then & \textsc{3sg} & speak & \textsc{1sg} & with & language & English\\
\lspbottomrule
\end{tabular}
\ea
\glt 
‘and then she \bluebold{talked (to)} me in English’ \textstyleExampleSource{[081115-001a-Cv.0229]}
\z

\begin{tabular}{llll}
\lsptoprule
(\stepcounter{}{\the}) & de & \bluebold{bicara} & trus\\
& \textsc{3sg} & speak & be.continuous\\
\lspbottomrule
\end{tabular}
\ea
\glt 
‘he kept \bluebold{talking}’ \textstyleExampleSource{[080922-010a-CvNF.0145]}
\z

\begin{tabular}{llllll}
\lsptoprule
\label{bkm:Ref348359809}
\gll {baru} {nanti} {\bluebold{ber–bicara}} {untuk} {nika}\\ %
& and.then & very.soon & \textsc{vblz}–speak & for & marry.officially\\
\lspbottomrule
\end{tabular}
\ea
\glt 
[About wedding customs:] ‘and then very soon (they’ll) \bluebold{talk} about marrying’ \textstyleExampleSource{[081110-006-CvEx.0050]}
\z


The corpus includes only two \textscItalBold{ber\-}prefixed lexemes that have distinct semantics vis-à-vis their bivalent bases, namely \textitbf{ajar} ‘teach’ and prefixed \textitbf{blajar} ‘study’, and \textitbf{angkat} ‘lift’ and prefixed \textitbf{brangkat} ‘leave’ as shown in (0) to (0). Both \textitbf{ajar} ‘teach’ and \textitbf{blajar} ‘study’ are used monotransitively as in (0) and (0), as well as intransitively as in (0) and (0), respectively; in each case both lexemes maintain their distinct semantics.
\end{styleBodyxafter}

\begin{styleExampleTitle}
Prefix \textscItalBold{ber\-}: Distinct semantics and same distribution of verbal bases and derived lexemes
\end{styleExampleTitle}

\begin{tabular}{lllllll}
\lsptoprule
\label{bkm:Ref346965799}
\gll {de} {\bluebold{ajar}} {dorang} {tu} {untuk} {baik}\\ %
& \textsc{3sg} & teach & \textsc{3pl} & \textsc{d.dist} & for & be.good\\
\lspbottomrule
\end{tabular}
\ea
\glt 
‘she \bluebold{teaches} them there for (their own) good’ \textstyleExampleSource{[081115-001a-Cv.0216]}
\z

\begin{tabular}{llllllllll}
\lsptoprule
\label{bkm:Ref346965800}
\gll {Ise} {de} {…} {\multicolumn{2}{l}{ikut}} {bahasa} {Inggris} {\bluebold{bl–ajar}} {kursus}\\ %
& Ise & \textsc{3sg} &  & \multicolumn{2}{l}{follow} & language & English & \textsc{vblz}–teach & course\\
& \multicolumn{2}{l}{bahasa} & \multicolumn{2}{l}{Inggris} & \multicolumn{5}{l}{dulu}\\
& \multicolumn{2}{l}{language} & \multicolumn{2}{l}{English} & \multicolumn{5}{l}{first}\\
\lspbottomrule
\end{tabular}
\ea
\glt 
‘Ise will participate in an English course, (she’ll) \bluebold{study} an English language course first’ \textstyleExampleSource{[081025-003-Cv.0223]}
\z

\begin{tabular}{llllllll}
\lsptoprule
\label{bkm:Ref346965801}
\gll {de} {suda} {\bluebold{ajar}} {bagus} {tiap} {sore} {itu}\\ %
& \textsc{3sg} & already & teach & be.good & every & afternoon & \textsc{d.dist}\\
\lspbottomrule
\end{tabular}
\ea
\glt 
‘she’s already been \bluebold{teaching} well, each and every afternoon’ \textstyleExampleSource{[081115-001a-Cv.0126]}
\z

\begin{tabular}{lllll}
\lsptoprule
\label{bkm:Ref357332087}
\gll {dong} {tida} {\bluebold{bl–ajar}} {baik}\\ %
& \textsc{3pl} & \textsc{neg} & \textsc{vblz}–teach & be.good\\
\lspbottomrule
\end{tabular}
\ea
\glt 
‘they don’t \bluebold{study} well’ \textstyleExampleSource{[081115-001b-Cv.0067]}
\z


Bivalent \textitbf{angkat} ‘lift’ and prefixed \textitbf{brangkat} ‘leave’ also have distinct semantics. In addition, they also have a distinct distribution. The base \textitbf{angkat} ‘lift’ is used monotransitively, as well as intransitively, as in (0) and (0), respectively. By contrast, \textitbf{brangkat} ‘leave’ is always used intransitively, as in (0).
\end{styleBodyxafter}

\begin{styleExampleTitle}
Prefix \textscItalBold{ber\-}: Distinct semantics and distinct distribution of verbal base and derived lexeme
\end{styleExampleTitle}

\begin{tabular}{llllll}
\lsptoprule
\label{bkm:Ref346965797}
\gll {bapa} {de} {\bluebold{angkat}} {rotang} {besar}\\ %
& father & \textsc{3sg} & lift & rattan & be.big\\
\lspbottomrule
\end{tabular}
\ea
\glt 
‘father \bluebold{picked-up} a big rattan (stick)’ \textstyleExampleSource{[080921-004a-CvNP.0084]}
\z

\begin{tabular}{llllll}
\lsptoprule
\label{bkm:Ref438916679}
\gll {sa} {sendiri} {tra} {bisa} {\bluebold{angkat}}\\ %
& \textsc{1sg} & be.alone & \textsc{neg} & be.able & lift\\
\lspbottomrule
\end{tabular}
\ea
\glt 
‘[the pig was very big,] I alone could not \bluebold{transport} (it)’ \textstyleExampleSource{[080919-003-NP.0008]}
\z

\begin{tabular}{lllll}
\lsptoprule
\label{bkm:Ref346965798}
\gll {skarang} {de} {mo} {\bluebold{br–angkat}}\\ %
& now & \textsc{3sg} & want & \textsc{vblz}–lift\\
\lspbottomrule
\end{tabular}
\ea
\glt 
‘then he wanted to \bluebold{leave}’ \textstyleExampleSource{[080919-007-CvNP.0023]}
\z


In sum, with the exception of the last two lexemes, \textscItalBold{ber\-}prefixed verbs have the same semantics as their respective bases. This suggests that in Papuan Malay affixation of verbal bases with prefix \textscItalBold{ber\-} is not a productive process. Instead, the attested prefixed lexemes and their bases are taken as pairs of words from two different speech varieties: the unaffixed items are native Papuan Malay lexemes whereas the corresponding affixed items are SI-borrowings.



Given these properties, Papuan Malay \textscItalBold{ber\-} contrasts with the corresponding prefix in other Malay varieties. In most eastern Malay varieties, the corresponding prefix \textitbf{ba}\- forms verbs with a variety of meanings. The most common ones are durative and reflexive meanings, which are reported for Ambon Malay {(van Minde 1997: 96–98)}, Banda Malay {(Paauw 2009: 249–250)},\footnote{\\
\\
\\
\\
\\
\\
\\
For Banda Malay, {\citet[249]{Paauw2009}} reports that \textitbf{ba}\- does not form verbs with reflexive meaning.\\
\\
\\
\\
\\
\\
\\
\\
} Manado Malay {(Stoel }{2005: 18–22)}, and North Moluccan / Ternate Malay ({Taylor 1983: 18;} {Litamahuputty 2012: 125–127}). In Kupang Malay {(Steinhauer 1983: 46–49)} and Larantuka Malay {(Paauw 2009: 249–254-255)}, the prefix typically signals durative and reciprocal meanings. In Standard Indonesian, the main function of the corresponding prefix \textitbf{ber\-} is to create monovalent verbs ({Englebretson 2003: 131;} {2007: 96}). When attached to verbal bases, the prefix indicates “that the subject of the utterance is the patient, that is, the experiencer of the action” {(Mintz 1994: 134–138)}.
\end{styleBodyvxvafter}

\paragraph[Prefixed items derived from nominal, numeral or quantifier bases]{Prefixed items derived from nominal, numeral or quantifier bases}
\label{bkm:Ref326244591}
The corpus contains 33 \textscItalBold{ber\-}prefixed lexemes (with 375 tokens), as listed in Table  ‎3 .17: 30 lexemes with nominal bases (362 tokens), two lexemes with numeral bases (7 tokens), and one lexeme with a quantifier base (6 tokens).



Most of the derived lexemes are low frequency words (29 lexemes, attested with less than 20 tokens). Besides, the token frequencies for the respective bases are (much) higher for most of the derived words (21 lexemes). This is due to the fact that Papuan Malay speakers typically use alternative analytical constructions to convey the meanings of the prefixed lexemes, as shown below in (0) to (0). Further, most of the 33 items (25 items) were tentatively classified as SI-borrowings (for details see language internal factor (f in §3.1.1, p. \pageref{bkm:Ref346541608}); in Table  ‎3 .17 these items are underlined.
\end{styleBodyvvafter}

\begin{stylecaption}
\label{bkm:Ref325989676}Table ‎3.\stepcounter{Table}{\theTable}:  Affixation with \textscItalBold{ber\-} of nominal, numeral, and quantifier bases
\end{stylecaption}

\tablehead{
 BW & Gloss & Item & Gloss & \textscItalBold{ber\-} \# & \arraybslash BW \#\\
}
\begin{tabular}{llllll}
\lsptoprule
\multicolumn{6}{l}{Nominal bases}\\
\textitbf{doa} & ‘prayer’ & \textitbf{berdoa} & ‘pray’ & \raggedleft 136 & \raggedleft\arraybslash 20\\
\textitbf{arti} & ‘meaning’ & \textitbf{brarti} & ‘mean’ & \raggedleft 89 & \raggedleft\arraybslash 7\\
\textitbf{diri} & ‘self’ & \textitbf{berdiri} & ‘stand’ & \raggedleft 55 & \raggedleft\arraybslash 14\\
\textitbf{usaha} & ‘effort’ & \textitbf{berusaha} & ‘attempt’ & \raggedleft 25 & \raggedleft\arraybslash 2\\
\textitbf{dosa} & ‘sin’ & \textitbf{berdosa} & ‘sin’ & \raggedleft 6 & \raggedleft\arraybslash 4\\
\textitbf{saksi} & ‘witness’ & \textitbfUndl{bersaksi} & ‘testify’ & \raggedleft 6 & \raggedleft\arraybslash 2\\
\textitbf{hasil} & ‘result’ & \textitbf{berhasil} & ‘succeed’ & \raggedleft 6 & \raggedleft\arraybslash 13\\
\textitbf{kwasa} & ‘power’ & \textitbfUndl{berkwasa} & ‘be powerful’ & \raggedleft 4 & \raggedleft\arraybslash 25\\
\textitbf{hak} & ‘right’ & \textitbfUndl{berhak} & ‘have right’ & \raggedleft 4 & \raggedleft\arraybslash 15\\
\textitbf{sodara} & ‘sibling’ & \textitbfUndl{bersodara} & ‘be siblings’ & \raggedleft 3 & \raggedleft\arraybslash 127\\
\textitbf{kebung} & ‘garden’ & \textitbfUndl{berkebung} & ‘do farming’ & \raggedleft 3 & \raggedleft\arraybslash 61\\
\textitbf{ade-kaka} & ‘siblings’ & \textitbfUndl{brade-kaka} & ‘be siblings’ & \raggedleft 2 & \raggedleft\arraybslash 26\\
\textitbf{malam} & ‘night’ & \textitbfUndl{bermalam} & ‘overnight’ & \raggedleft 2 & \raggedleft\arraybslash 191\\
\textitbf{bahasa} & ‘language’ & \textitbfUndl{berbahasa} & ‘speak’ & \raggedleft 2 & \raggedleft\arraybslash 136\\
\textitbf{temang} & ‘friend’ & \textitbfUndl{bertemang} & ‘be friends’ & \raggedleft 2 & \raggedleft\arraybslash 85\\
\textitbf{kluarga} & ‘family’ & \textitbfUndl{berkluarga} & ‘have family’ & \raggedleft 2 & \raggedleft\arraybslash 49\\
\textitbf{gaya} & ‘manner’ & \textitbf{bergaya} & ‘put on airs’ & \raggedleft 2 & \raggedleft\arraybslash 7\\
\textitbf{ana} & ‘child’ & \textitbf{brana} & ‘give birth’ & \raggedleft 1 & \raggedleft\arraybslash 739\\
\textitbf{bua} & ‘fruit’ & \textitbfUndl{berbua} & ‘have fruit’ & \raggedleft 1 & \raggedleft\arraybslash 38\\
\textitbf{dara} & ‘blood’ & \textitbfUndl{berdara} & ‘bleed’ & \raggedleft 1 & \raggedleft\arraybslash 27\\
\textitbf{sifat} & ‘character\-istic’ & \textitbfUndl{bersifat} & ‘have character\-istics of’ & \raggedleft 1 & \raggedleft\arraybslash 18\\
\textitbf{duri} & ‘thorn’ & \textitbfUndl{berduri} & ‘have thorns’ & \raggedleft 1 & \raggedleft\arraybslash 8\\
\textitbf{harga} & ‘value’ & \textitbfUndl{berharga} & ‘be valuable’ & \raggedleft 1 & \raggedleft\arraybslash 4\\
\textitbf{syukur} & ‘thanks’ & \textitbfUndl{bersyukur} & ‘give thanks’ & \raggedleft 1 & \raggedleft\arraybslash 2\\
\textitbf{fungsi} & ‘function’ & \textitbfUndl{berfungsi} & ‘function’ & \raggedleft 1 & \raggedleft\arraybslash 1\\
\textitbf{gisi} & ‘nutrient’ & \textitbfUndl{bergisi} & ‘be nutritious’ & \raggedleft 1 & \raggedleft\arraybslash 1\\
\textitbf{isi} & ‘content’ & \textitbfUndl{baisi} & ‘be muscular’ & \raggedleft 1 & \raggedleft\arraybslash 1\\
\textitbf{komunikasi} & ‘communica\-tion’ & \textitbfUndl{berkomuni\-kasi} & ‘communicate’ & \raggedleft 1 & \raggedleft\arraybslash 1\\
\textitbf{kumis} & ‘beard’ & \textitbfUndl{bakumis} & ‘have a beard’ & \raggedleft 1 & \raggedleft\arraybslash 1\\
\textitbf{mekap} & ‘make-up’ & \textitbfUndl{bamekap} & ‘wear make-up’ & \raggedleft 1 & \raggedleft\arraybslash 1\\
\multicolumn{6}{l}{Numeral bases}\\
\textitbf{satu} & ‘one’ & \textitbfUndl{bersatu} & ‘be one’ & \raggedleft 6 & \raggedleft\arraybslash 516\\
\textitbf{empat} & ‘four’ & \textitbfUndl{berempat} & ‘be four’ & \raggedleft 1 & \raggedleft\arraybslash 66\\
\multicolumn{6}{l}{Quantifier base}\\
\textitbf{brapa} & ‘several’ & \textitbfUndl{bebrapa} & ‘be several’ & \raggedleft 6 & \raggedleft\arraybslash 109\\
\lspbottomrule
\end{tabular}

Affixation with \textscItalBold{ber\-} derives monovalent verbs with the general meaning of ‘be/have/do \textsc{base}’. Examples are \textitbf{brarti} ‘have the meaning of’ or ‘mean’, \textitbf{berdoa} ‘do prayer’ or ‘pray’, \textitbf{bersatu} ‘be one’, or \textitbf{bebrapa} ‘be several’. The monovalent verb \textitbf{berdiri} ‘stand’ is an exception. Historically related to the noun \textitbf{diri} ‘self’, it does not have a transparent form-function relationship to its base. The transparent form-function relationship between the remaining 32 items and their bases suggests that these lexemes are the result of a productive affixation process. Two observations are made, however.
\end{styleBodyaftervbefore}


First, the data indicates that Papuan Malay speakers prefer to employ analytical constructions to express the meanings conveyed by the prefixed items, as illustrated in (0) to (0). To communicate ‘have \textsc{base}’, speakers typically use the existential verb \textitbf{ada} ‘exist’ rather than the prefixed form, as shown in (0) with \textitbf{ada duri} vs. \textitbf{berduri} ‘have thorns’.
\end{styleBodyvxafter}

\begin{tabular}{lllllllllllllllllll}
\lsptoprule
\label{bkm:Ref338927614}
\gll {\multicolumn{2}{l}{ada}} {\multicolumn{2}{l}{…}} {dua} {\multicolumn{2}{l}{macang}} {\multicolumn{2}{l}{jenis}} {\multicolumn{2}{l}{ada}} {\multicolumn{2}{l}{yang}} {\multicolumn{3}{l}{\bluebold{ber}\bluebold{–}\bluebold{duri}}} {ada} {yang}\\ %
& \multicolumn{2}{l}{exist} & \multicolumn{2}{l}{} & two & \multicolumn{2}{l}{variety} & \multicolumn{2}{l}{kind} & \multicolumn{2}{l}{exist} & \multicolumn{2}{l}{\textsc{rel}} & \multicolumn{3}{l}{\textsc{vblz}–thorn} & exist & \textsc{rel}\\
& tida & \multicolumn{2}{l}{…} & \multicolumn{3}{l}{kang} & \multicolumn{2}{l}{ada} & \multicolumn{2}{l}{sagu} & \multicolumn{2}{l}{yang} & \multicolumn{2}{l}{tida} & \bluebold{ada} & \multicolumn{3}{l}{\bluebold{duri}}\\
& \textsc{neg} & \multicolumn{2}{l}{} & \multicolumn{3}{l}{you.know} & \multicolumn{2}{l}{exist} & \multicolumn{2}{l}{sago} & \multicolumn{2}{l}{\textsc{rel}} & \multicolumn{2}{l}{\textsc{neg}} & exist & \multicolumn{3}{l}{thorn}\\
\lspbottomrule
\end{tabular}
\ea
\glt 
‘there are … two kinds (of sago palms), ones that \bluebold{have thorns} and ones that don’t (have thorns) … you know (there are) sago palms that don’t \bluebold{have thorns}’ \textstyleExampleSource{[081014-006-CvPr.0007/0009]}
\z


To express ‘be \textsc{base}’, speakers use a nominal predicate such as \textitbf{ade-kaka} ‘siblings’ in (0), rather than the respective prefixed form \textitbf{brade-kaka} ‘be siblings’ as in (0).
\end{styleBodyxafter}

\begin{tabular}{lllllll}
\lsptoprule
\label{bkm:Ref338927615}
\gll {jadi} {saya} {dengang} {dia} {\bluebold{ade-kaka}} {sunggu}\\ %
&  &  &  &  & ySb-oSb & \\
& so & \textsc{1sg} & with & \textsc{3sg} & siblings & be.true\\
\lspbottomrule
\end{tabular}
\ea
\glt 
‘so I and she \bluebold{are} full \bluebold{siblings}’ \textstyleExampleSource{[080927-009-CvNP.0044]}
\z

\begin{tabular}{llllllll}
\lsptoprule
\label{bkm:Ref338927616}
\gll {jadi} {saya} {dengang} {kaka} {Nofita} {masi} {\bluebold{br}\bluebold{–}\bluebold{ade-kaka}}\\ %
& so & \textsc{1sg} & with & oSb & Nofita & still & \textsc{vblz}–siblings\\
\lspbottomrule
\end{tabular}
\ea
\glt 
‘so I and older sister Nofita \bluebold{are} still \bluebold{siblings}’ \textstyleExampleSource{[080927-007-CvNP.0022]}
\z


To communicate ‘do \textsc{base}’, speakers typically employ alternative verbs. They tend to say, for example, \textitbf{biking kebung} ‘make/work a garden’ as in (0), rather than use prefixed \textitbf{berkebung} ‘do farming’ as in (0). Likewise, it is more common to say \textitbf{taw bahasa X} ‘speak language X’ than to use prefixed \textitbf{berbahasa X} ‘speak language X’ as in (0).
\end{styleBodyxafter}

\begin{tabular}{llllllll}
\lsptoprule
\label{bkm:Ref338927617}
\gll {kalo} {di} {Arbais} {prempuang} {bisa} {\bluebold{biking}} {\bluebold{kebung}}\\ %
& if & at & Arbais & woman & be.able & make & garden\\
\lspbottomrule
\end{tabular}
\ea
\glt 
‘as for Arbais, (there) the women can \bluebold{work a garden}’ \textstyleExampleSource{[081014-007-CvEx.0035]}
\z

\begin{tabular}{llllll}
\lsptoprule
\label{bkm:Ref338927618}
\gll {bapa} {pergi} {\bluebold{ber}\bluebold{–}\bluebold{kebung}} {saya} {ikut}\\ %
& father & go & \textsc{vblz}–garden & \textsc{1sg} & follow\\
\lspbottomrule
\end{tabular}
\ea
\glt 
‘(whenever my) father went to \bluebold{do farming} I went with (him)’ \textstyleExampleSource{[081110-008-CvNP.0002]}
\z

\begin{tabular}{lllllllllllllllll}
\lsptoprule
\label{bkm:Ref338927619}
\gll {jadi} {\multicolumn{2}{l}{tong}} {cuma} {\multicolumn{2}{l}{\bluebold{taw}}} {\multicolumn{2}{l}{\bluebold{bahasa}}} {\bluebold{Yali}} {\multicolumn{2}{l}{…}} {\multicolumn{2}{l}{tapi}} {sa} {\multicolumn{2}{l}{bilang}}\\ %
& so & \multicolumn{2}{l}{\textsc{1pl}} & just & \multicolumn{2}{l}{know} & \multicolumn{2}{l}{language} & Yali & \multicolumn{2}{l}{} & \multicolumn{2}{l}{but} & \textsc{1sg} & \multicolumn{2}{l}{say}\\
& \multicolumn{2}{l}{kamu} & \multicolumn{3}{l}{ber\bluebold{–}syukur} & \multicolumn{2}{l}{karna} & \multicolumn{3}{l}{bisa} & \multicolumn{2}{l}{…} & \multicolumn{3}{l}{\bluebold{ber}\bluebold{–}\bluebold{bahasa}} & \bluebold{Yali}\\
& \multicolumn{2}{l}{\textsc{2pl}} & \multicolumn{3}{l}{\textsc{vblz}\bluebold{–}thank.God} & \multicolumn{2}{l}{because} & \multicolumn{3}{l}{be.able} & \multicolumn{2}{l}{} & \multicolumn{3}{l}{\textsc{vblz}–language} & Yali\\
\lspbottomrule
\end{tabular}
\ea
\glt 
‘so we only \bluebold{spoke Yali} … but I said, ‘you (should) be grateful because (you) can \bluebold{speak Yali}’’ \textstyleExampleSource{[081011-022-Cv.0101/0184]}
\z


Second, the exchange in (0) suggests that the high frequency items listed in Table  ‎3 .17 may well have non-compositional semantics for Papuan Malay speakers. In a conversation about religious affairs, the speaker produced \textitbf{diberdoa} ‘be prayed for’. This item is ungrammatical in both Papuan Malay and Standard Indonesian. Papuan Malay does not have a morphologically marked undergoer voice. The Standard Indonesian undergoer voice marker \textitbf{di}\- cannot co-occur with prefix \textitbf{ber}\-, but always replaces it. This example shows that the speaker perceives \textitbf{berdoa} ‘pray’ as a monomorphemic word to which she affixed the Indonesian undergoer voice marker \textitbf{di}\- in an attempt to approximate Indonesian.
\end{styleBodyxafter}

\begin{tabular}{llllllllll}
\lsptoprule
\label{bkm:Ref338954753}
\gll {bebang} {masala} {de} {punya} {dia} {perlu} {…} {harus} {\bluebold{di}\bluebold{–}\bluebold{ber}\bluebold{–}\bluebold{doa}}\\ %
& burden & problem & \textsc{3sg} & \textsc{poss} & \textsc{3sg} & need &  & have.to & \textsc{uv}–\textsc{vblz}–prayer\\
\lspbottomrule
\end{tabular}
\ea
\glt
[Conversation about problems of a church congregation:] ‘(all) burdens (and) problems (that) it has, (the congregation) needs … has to \bluebold{be prayed for}’ \textstyleExampleSource{[080917-008-NP.0089/0091]}
\end{styleFreeTranslEngxvpt}

\paragraph[Summary and conclusions]{Summary and conclusions}
\label{bkm:Ref352332391}
Prefix \textscItalBold{ber\-} is a polyfunctional affix that derives lexemes from verbal and nominal bases. This polyfunctionality suggests that in Papuan Malay affixation with \textscItalBold{ber\-} is a somewhat productive process (see language internal factor (c in §3.1.1, p. \pageref{bkm:Ref364758123}). Two other observations support this conclusion: (1) the large number of low frequency words and small number of high frequency words, and (2) the relative token frequencies with most bases having higher frequencies than the affixed lexemes.



Four other observations, however, do not support the conclusion that affixation with \textscItalBold{ber\-} is a productive process: (1) for the prefixed lexemes with verbal bases, the derived lexemes have the same semantics as their bases, (2) for lexemes with nominal bases, speakers prefer to use alternative analytical constructions rather than the affixed lexemes, (3) high frequency items may well have non-compositional semantics for Papuan Malay speakers, and (4) most of the lexemes with verbal or nominal bases were tentatively classified as SI-borrowings.
\end{styleBodyvafter}


Taken together, these findings indicate that Papuan Malay speakers do not employ prefix \textscItalBold{ber\-} as a productive device to derive new words. This conclusion is also supported by the findings of a domain analysis which indicate that most of the attested tokens can be accounted for in terms of the variables of speaker education levels, topics, and/or role-relations (details are discussed in §3.1.8, together with the findings for suffix \textitbf{\-nya} ‘\textsc{3possr}’ and circumfix \textitbf{ke}\textitbf{\-}/\textitbf{\-}\textitbf{ang} ‘\textsc{nmlz}’). Therefore, these lexemes are best explained as code-switches with Indonesian. (For a detailed discussion of prefix \textitbf{ber}\- in Standard Indonesian and Standard Malay see {Adelaar 1992; Mintz 1994; Sneddon 2010}.)
\end{styleBodyvafter}


The conclusion that in Papuan Malay prefix \textscItalBold{ber\-} is unproductive again sets Papuan Malay apart from other eastern Malay varieties. In regional varieties such as Ambon Malay {(van Minde 1997: 96–98)}, Banda Malay {(Paauw 2009: 249–250)}, Larantuka Malay {(Paauw 2009: 253–255)}, Manado Malay {(Stoel 2005: 18–22)}, and North Moluccan / Ternate Malay ({Taylor 1983: 18;} {Litamahuputty 2012: 125–127}) the prefix is a productive derivational device.\footnote{\\
\\
\\
\\
\\
\\
\\
{\citet[4]{Voorhoeve1983}} considers prefix \textitbf{ba}\- to be unproductive.\\
\\
\\
\\
\\
\\
\\
\\
} This distinction between Papuan Malay and the other eastern Malay varieties once again hints at the separate histories of both, discussed in §1.8.
\end{styleBodyvxvafter}

\subsection{Suffix \textitbf{\-nya} ‘\textsc{3possr}’}
\label{bkm:Ref374464238}\label{bkm:Ref347514357}
Suffix \textitbf{\-nya} ‘\textsc{3possr}’ is typically attached to nominal bases to indicate possessive relations, as illustrated in (0). In addition, a considerable number of suffixed lexemes have verbal bases, while a small number of lexemes have prepositional, adverbial, locative, or demonstrative bases. However, affixation with \textitbf{\-nya} ‘\textsc{3possr}’ is not used as a productive derivation device in Papuan Malay.
\end{styleBodyxafter}

\begin{tabular}{llllllll}
\lsptoprule
\label{bkm:Ref347303696}
\gll {jadi} {\bluebold{ana–nya}} {hidup,} {ana} {itu} {masi} {ada}\\ %
& so & child–\textsc{3possr} & live & child & \textsc{d.dist} & still & exist\\
\lspbottomrule
\end{tabular}
\ea
\glt 
‘so \bluebold{her child} lives, that child still exists’ \textstyleExampleSource{[080921-005-CvNP.0007]}
\z


The corpus contains 123 lexical items (387 tokens) suffixed with \textitbf{\-nya}:\footnote{\\
\\
\\
\\
\\
\\
\\
The 123 suffixed lexemes include 68 hapaxes (P=0.1757); the 81 lexemes with nominal bases include 44 hapaxes (P=0.1549); the 36 lexemes with verbal bases include 21 hapaxes (P=0.2561); the five lexemes with other bases include three hapaxes (P=0.1500).\\
\\
\\
\\
\\
\\
\\
\\
}


%\setcounter{itemize}{0}
\begin{itemize}
\item \begin{styleIIndented}
Suffixed items with nominal bases (82 items with 285 tokens)
\end{styleIIndented}\item \begin{styleIIndented}
Suffixed items with verbal bases (36 items with 82 tokens)
\end{styleIIndented}\item \begin{styleIvI}
Suffixed items with other bases (five items with 20 tokens)
\end{styleIvI}\end{itemize}

The corpus also contains seven formally complex words with non-compositional semantics. All seven items have adverbial function, such as \textitbf{misalnya} ‘for example’ or \textitbf{akirnya} ‘finally’.



Suffixed lexemes with nominal bases are discussed in §3.1.6.1, those with verbal bases in §3.1.6.2, and those with other bases in §3.1.6.3. Pertinent variables of the communicative event that may impact the use of \textitbf{\-nya} are explored in §3.1.8. The main findings on suffix \textitbf{\-nya} are summarized and evaluated in §3.1.6.4.
\end{styleBodyvxvafter}

\paragraph[Suffixed items derived from nominal bases]{Suffixed items derived from nominal bases}
\label{bkm:Ref347239580}
The corpus contains 82 \textitbf{\-nya}{}-suffixed lexemes (with 285 tokens) with nominal bases, where \textitbf{\-nya} typically signals possession. As an extension of the possessive-marking function, some of the derived items listed in Table  ‎3 .18 function as sentence adverbs, namely \textitbf{maksutnya} ‘that is to say’ (literally ‘the purpose of’), \textitbf{katanya} ‘it is being said’ (literally ‘the word of’), and \textitbf{artinya} ‘that means’ (literally ‘the meaning of’).



Derived words with token frequencies of five or more are listed in Table  ‎3 .18. All but two of the derived lexemes are low frequency words (80 items, attested with less than 20 tokens). Besides, the token frequencies for the respective bases are (much) higher for most of the derived words (65 lexemes). Of the 82 suffixed lexemes, 76 were tentatively classified as borrowings from Standard Indonesian (SI-borrowings) (for details see language internal factor (f in §3.1.1, p. \pageref{bkm:Ref346541608}); in Table  ‎3 .18 these items are underlined. The exceptions are the three derived lexemes that function as sentence adverbs, two of which are presented in context in (0) and (0).
\end{styleBodyvafter}


The low token frequencies for the derived lexemes result from the fact that Papuan Malay speakers usually use an alternative strategy to express possessive relations. Instead of suffixing \textitbf{\-nya} to a nominal base, Papuan Malay encodes adnominal possession by an analytical construction with \textitbf{punya}, or reduced \textitbf{pu}, ‘\textsc{poss}’ (see Chapter 9). The ‘\textitbf{punya} \#’ column in Table  ‎3 .18 lists the token frequencies for adnominal possessive constructions with \textitbf{punya}/\textitbf{pu} ‘\textsc{poss}’. Examples are given in (0) and (0).
\end{styleBodyvvafter}

\begin{stylecaption}
\label{bkm:Ref336454218}Table ‎3.\stepcounter{Table}{\theTable}:  Affixation with \textitbf{\-nya} of nominal bases
\end{stylecaption}

\tablehead{
 BW & Gloss & Item & Gloss & \textitbf{\-nya} \# & \arraybslash \textitbf{punya} \#\\
}
\begin{tabular}{llllll}
\lsptoprule
\textitbf{nama} & ‘name’ & \textitbfUndl{namanya} & ‘the name of’ & \raggedleft 23 & \raggedleft\arraybslash 38\\
\textitbf{istri} & ‘wife’ & \textitbfUndl{istrinya} & ‘the wife of’ & \raggedleft 11 & \raggedleft\arraybslash 22\\
\textitbf{ana} & ‘child’ & \textitbfUndl{ananya} & ‘the child of’ & \raggedleft 7 & \raggedleft\arraybslash 119\\
\textitbf{orang} & ‘person’ & \textitbfUndl{orangnya} & ‘the person of’ & \raggedleft 6 & \raggedleft\arraybslash 8\\
\textitbf{ruma} & ‘house’ & \textitbfUndl{rumanya} & ‘the house of’ & \raggedleft 5 & \raggedleft\arraybslash 43\\
\textitbf{hasil} & ‘product’ & \textitbfUndl{hasilnya} & ‘the product of’ & \raggedleft 5 & \raggedleft\arraybslash 2\\
\textitbf{istila} & ‘term’ & \textitbfUndl{istilanya} & ‘the term of/for’ & \raggedleft 5 & \raggedleft\arraybslash 1\\
\textitbf{dalam} & ‘inside’ & \textitbfUndl{dalamnya} & ‘the inside of’ & \raggedleft 5 & \raggedleft\arraybslash {}-{}-{}-\\
\textitbf{maksut} & ‘purpose’ & \textitbf{maksutnya} & ‘that is to say’ & \raggedleft 70 & \raggedleft\arraybslash 3\\
\textitbf{kata} & ‘word’ & \textitbf{katanya} & ‘it is being said’ & \raggedleft 19 & \raggedleft\arraybslash {}-{}-{}-\\
\textitbf{arti} & ‘meaning’ & \textitbf{artinya} & ‘that means’ & \raggedleft 17 & \raggedleft\arraybslash {}-{}-{}-\\
\lspbottomrule
\end{tabular}

In (0), \textitbf{\-nya} is suffixed to the nominal base \textitbf{nama} ‘name’, giving the possessive reading \textitbf{namanya} ‘her name’. By contrast, (0) shows the inherited analytical strategy of expressing the same meaning with possessive marker \textitbf{pu} ‘\textsc{poss}’.


\begin{styleExampleTitle}
Suffix \textitbf{\-nya}: Possessive reading of derived lexemes
\end{styleExampleTitle}

\begin{tabular}{lll}
\lsptoprule
\label{bkm:Ref347158681}
\gll {\bluebold{nama–nya}} {Madga}\\ %
& name–\textsc{3possr} & Madga\\
\lspbottomrule
\end{tabular}
\ea
\glt 
‘\bluebold{her name} is Madga’ \textstyleExampleSource{[081011-005-Cv.0027]}
\z

\begin{tabular}{lllll}
\lsptoprule
\label{bkm:Ref347158683}
\gll {\bluebold{de}} {\bluebold{pu}} {\bluebold{nama}} {Martin}\\ %
& \textsc{3sg} & \textsc{poss} & name & Martin\\
\lspbottomrule
\end{tabular}
\ea
\glt 
‘\bluebold{his name} is Martin’ \textstyleExampleSource{[081011-022-Cv.0241]}
\z


The examples in (0) and (0) illustrate the uses of \textitbf{maksutnya} ‘that is to say’ and \textitbf{katanya} ‘it is being said’, respectively, as sentence adverbs.
\end{styleBodyxafter}

\begin{styleExampleTitle}
Suffix \textitbf{\-nya}: Adverbial reading of derived lexemes
\end{styleExampleTitle}

\begin{tabular}{lllllllll}
\lsptoprule
\label{bkm:Ref347225566}
\gll {…} {\bluebold{maksut–nya}} {saya} {harus} {dayung} {dulu} {dengang} {prahu}\\ %
&  & purpose–\textsc{3possr} & \textsc{1sg} & have.to & row & first & with & boat\\
\lspbottomrule
\end{tabular}
\ea
\glt 
‘[I’m getting ready, I take my bow and arrows and an oar,] \bluebold{that is to say}, I have to row first with a boat’ \textstyleExampleSource{[080919-004-NP.0008]}
\z

\begin{tabular}{lllll}
\lsptoprule
\label{bkm:Ref347225567}
\gll {\bluebold{kata–nya}} {orang} {Sulawesi} {smua}\\ %
& word–\textsc{3possr} & person & Sulawesi & all\\
\lspbottomrule
\end{tabular}
\ea
\glt
‘\bluebold{it’s being said} (that) they are all Sulawesi people’ (Lit. ‘(the) Sulawesi people (are) all’) \textstyleExampleSource{[081029-005-Cv.0106]}
\end{styleFreeTranslEngxvpt}

\paragraph[Suffixed items derived from verbal bases]{Suffixed items derived from verbal bases}
\label{bkm:Ref347157184}
The corpus contains 36 \textitbf{\-nya}{}-suffixed lexemes (with 82 tokens) with verbal bases. Affixation with \textitbf{\-nya} derives nominals from verbal bases. Shifting from the possessive reading of \textitbf{\-nya}, the derived nominals have the general meaning of ‘the \textsc{base} of’, such as \textitbf{ceritranya} ‘the telling of’ or ‘his/her telling’. As an extension of the nominalizing and possessive-marking function of \textitbf{\-nya}, eight of the derived lexemes function as adverbs, such as \textitbf{biasanya} ‘usually’ (literally ‘its being usual’) or \textitbf{kususnya} ‘especially’ (literally ‘its being special’). Derived words with token frequencies of three or more are listed in Table  ‎3 .19.



All 36 affixed lexemes are low frequency words, attested with less than 20 tokens. Moreover, the token frequencies for the respective bases are (much) higher for all but one of the derived words (35 lexemes). This is due to the fact that Papuan Malay speakers tend to use the respective bases, as in (0) to (0), rather than the suffixed forms. Of the 36 derived lexemes, nine were tentatively classified as SI-borrowings (for details see language internal factor (f in §3.1.1, p. \pageref{bkm:Ref346541608}); in Table  ‎3 .19 these items are underlined.
\end{styleBodyvvafter}

\begin{stylecaption}
\label{bkm:Ref336454220}Table ‎3.\stepcounter{Table}{\theTable}:  Affixation with \textitbf{\-nya} of verbal bases
\end{stylecaption}

\tablehead{
 BW & Gloss & Item & Gloss & \textitbf{\-nya} \# & \arraybslash BW \#\\
}
\begin{tabular}{llllll}
\lsptoprule
\textitbf{mo} & ‘want’ & \textitbfUndl{mawnya} & ‘the wanting of’ & \raggedleft 6 & \raggedleft\arraybslash 972\\
\textitbf{ceritra} & ‘tell’ & \textitbfUndl{ceritranya} & ‘the telling of’ & \raggedleft 6 & \raggedleft\arraybslash 162\\
\textitbf{pegang} & ‘hold’ & \textitbfUndl{pegangnya} & ‘the holding of’ & \raggedleft 3 & \raggedleft\arraybslash 114\\
\textitbf{hidup} & ‘live’ & \textitbfUndl{hidupnya} & ‘the living of’ & \raggedleft 3 & \raggedleft\arraybslash 74\\
\textitbf{biasa} & ‘be usual’ & \textitbfUndl{biasanya} & ‘usually’ & \raggedleft 18 & \raggedleft\arraybslash 181\\
\textitbf{harus} & ‘have to’ & \textitbf{harusnya}\footnotemark{} & ‘appropriately’ & \raggedleft 7 & \raggedleft\arraybslash 379\\
\textitbf{kusus} & ‘be special’ & \textitbf{kususnya} & ‘especially’ & \raggedleft 3 & \raggedleft\arraybslash 30\\
\lspbottomrule
\end{tabular}
\footnotetext{\\
\\
\\
\\
\\
\\
\\
Included in the six \textitbf{harusnya} ‘appropriately’ tokens is one \textitbf{seharusnya} token which also means ‘appropriately’. According to one consultant, \textitbf{harusnya} ‘appropriately’ is the more common form.\\
\\
\\
\\
\\
\\
\\
\\
}

In (0), \textitbf{\-nya} is suffixed to the verbal base \textitbf{mo} ‘want’ giving the nominalized form \textitbf{mawnya} ‘the wanting of’. The example in (0) illustrates the preferred strategy of expressing the same meaning in a verbal clause with the base \textitbf{mo} ‘want’.


\begin{styleExampleTitle}
Suffix \textitbf{\-nya}: Use patterns of base word \textitbf{mo} ‘want’ versus derived lexeme
\end{styleExampleTitle}

\begin{tabular}{lllllllll}
\lsptoprule
\label{bkm:Ref347164096}
\gll {\bluebold{maw–nya}} {ke} {kampung} {maw} {biking} {apa} {di} {sana}\\ %
& want–\textsc{3possr} & to & village & want & make & what & at & \textsc{l.dist}\\
\lspbottomrule
\end{tabular}
\ea
\glt 
[Addressing a teenager who plays hooky:] ‘\bluebold{your wish} (is to go) to the village, what do (you) want to do there?’ (Lit. ‘\bluebold{his wanting} (is) to the village’) \textstyleExampleSource{[081115-001a-Cv.0046]}
\z

\begin{tabular}{llllllll}
\lsptoprule
\label{bkm:Ref347164097}
\gll {\bluebold{ko}} {\bluebold{mo}} {ke} {kampung} {tapi} {ko} {skola}\\ %
& \textsc{2sg} & want & to & village & but & \textsc{2sg} & go.to.school\\
\lspbottomrule
\end{tabular}
\ea
\glt 
‘\bluebold{you want} (to go) to the village but you’re going to school’ \textstyleExampleSource{[080922-001a-CvPh.0734]}
\z


In (0), \textitbf{\-nya} is suffixed to the verbal base \textitbf{biasa} ‘be usual’ with adverbially used \textitbf{biasanya} ‘usually’ modifying the verb \textitbf{dansa} ‘dance’. More commonly, however, speakers employ the base \textitbf{biasa} ‘be usual’, as in (0) with adverbially used \textitbf{biasa} ‘be usual’ modifying the verb \textitbf{maing} ‘play’.
\end{styleBodyxafter}

\begin{styleExampleTitle}
Suffix \textitbf{\-nya}: Use patterns of base word \textitbf{biasa} ‘be usual’ versus derived lexeme
\end{styleExampleTitle}

\begin{tabular}{llllllll}
\lsptoprule
\label{bkm:Ref338835668}
\gll {…} {dansa} {lemon-nipis} {itu} {\bluebold{biasa–nya}} {dansa} {lemon-nipis}\\ %
&  & dance & citron & \textsc{d.dist} & be.usual–\textsc{3possr} & dance & citron\\
\lspbottomrule
\end{tabular}
\ea
\glt 
‘[they make a ceremony, they sing on and on,] (they) dance that citron (group dance), \bluebold{usually} (they) dance the citron (group dance)’ \textstyleExampleSource{[081110-005-CvPr.0098]}
\z

\begin{tabular}{llllllll}
\lsptoprule
\label{bkm:Ref338835669}
\gll {Herman} {dorang} {\bluebold{biasa}} {maing} {di} {sini} {tu}\\ %
& Herman & \textsc{3pl} & be.usual & play & at & \textsc{l.prox} & \textsc{d.dist}\\
\lspbottomrule
\end{tabular}
\ea
\glt
‘Herman and the others \bluebold{usually} play right here’ \textstyleExampleSource{[080923-009-Cv.0017]}
\end{styleFreeTranslEngxvpt}

\paragraph[Suffixed items derived from other bases]{Suffixed items derived from other bases}
\label{bkm:Ref347239584}
The corpus contains five lexemes (with 20 tokens) which are derived from a number of different bases. Two lexemes have prepositional bases and one has an adverbial base, listed in Table  ‎3 .20, with \textitbf{\-nya} having adverb-marking function. In addition, one lexeme has a demonstrative base and one a locative base, listed in Table  ‎3 .21, with \textitbf{\-nya} having emphasizing function.



The two lexemes with prepositional bases and the one with an adverbial base have distinct meanings vis-à-vis their bases. These items usually function as sentence adverbs as shown in (0) and (0). Again, the adverbial-marking function of \textitbf{\-nya} seems to be an extension of its nominalizing and possessive-marking function. For instance, \textitbf{spertinya} ‘it seems’ can be literally translated as ‘its being similar to’. All five affixed lexemes are low frequency words, attested with less than 20 tokens. In addition, the token frequencies for the respective bases are (much) higher for all of the derived words. All five suffixed lexemes were tentatively classified as SI-borrowings; in Table  ‎3 .20 these items are underlined.
\end{styleBodyvvafter}

\begin{stylecaption}
\label{bkm:Ref347237672}Table ‎3.\stepcounter{Table}{\theTable}:  Affixation with \textitbf{\-nya} of prepositional and adverbial bases
\end{stylecaption}

\tablehead{
 BW & Gloss & Item & Gloss & \textitbf{\-nya} \# & \arraybslash BW \#\\
}
\begin{tabular}{llllll}
\lsptoprule
\multicolumn{6}{l}{Prepositional base}\\
\textitbf{sperti} & ‘similar to’ & \textitbfUndl{spertinya} & ‘it seems’ & \raggedleft 12 & \raggedleft\arraybslash 217\\
\textitbf{kaya} & ‘like’ & \textitbfUndl{kayanya} & ‘it looks like’ & \raggedleft 5 & \raggedleft\arraybslash 61\\
\multicolumn{6}{l}{Adverbial bases}\\
\textitbf{memang} & ‘indeed’ & \textitbfUndl{memangnya} & ‘actually’ & \raggedleft 1 & \raggedleft\arraybslash 143\\
\lspbottomrule
\end{tabular}

The examples in (0) and (0) illustrate the respective uses of \textitbf{spertinya} ‘it seems’ and \textitbf{kayanya} ‘it looks like’ as sentence adverbs.


\begin{styleExampleTitle}
Suffix \textitbf{\-nya}: Adverbial reading of derived lexemes
\end{styleExampleTitle}

\begin{tabular}{llllllll}
\lsptoprule
\label{bkm:Ref347238674}
\gll {\bluebold{sperti–nya}} {de} {suda} {tinggalkang} {de} {punya} {orang-tua}\\ %
& similar.to–\textsc{3possr} & \textsc{3sg} & already & leave & \textsc{3sg} & \textsc{poss} & parent\\
\lspbottomrule
\end{tabular}
\ea
\glt 
‘\bluebold{it seems} she already left her parents behind’ \textstyleExampleSource{[081110-005-CvPr.0086]}
\z

\begin{tabular}{lll}
\lsptoprule
\label{bkm:Ref347238669}
\gll {\bluebold{kaya–nya}} {munta{\Tilde}munta}\\ %
& like–\textsc{3possr} & \textsc{rdp}{}-vomit\\
\lspbottomrule
\end{tabular}
\ea
\glt 
‘\bluebold{it looked like} (he was going to) vomit’ \textstyleExampleSource{[081025-008-Cv.0051]}
\z


When suffixed to demonstrative or locative bases, \textitbf{\-nya} functions as an emphasizer. This usage of \textitbf{\-nya} is very rare, however; attested are only the two lexemes listed in Table  ‎3 .21. Instead, to signal emphasis, Papuan Malay speakers typically employ a modifying demonstrative (see §7.1.2.3); this is shown with the token frequencies given in the ‘\textsc{dem} \#’ column, which refer to modification with a demonstrative. Examples are presented in (0) and (0).


\begin{stylecaption}
\label{bkm:Ref337049201}Table ‎3.\stepcounter{Table}{\theTable}:  Affixation with \textitbf{\-nya} of demonstrative and locative bases
\end{stylecaption}

\tablehead{
 BW & Gloss & Item & Gloss & \textitbf{\-nya} \# & \arraybslash \textsc{dem} \textitbf{\#}\\
}
\begin{tabular}{llllll}
\lsptoprule
\textitbf{itu} & ‘\textsc{d.dist}’ & \textitbfUndl{itunya} & ‘it!’ & \raggedleft 1 & \raggedleft\arraybslash 19\\
\textitbf{sini} & ‘\textsc{l.prox}’ & \textitbfUndl{sininya} & ‘right here’ & \raggedleft 1 & \raggedleft\arraybslash 18\\
\lspbottomrule
\end{tabular}

In (0), \textitbf{\-nya} is suffixed to the medial locative \textitbf{sini} ‘\textsc{l.prox}’, giving the emphatic reading \textitbf{sininya} ‘right here’. In (0) the same meaning is expressed with an analytical construction in which the distal demonstrative modifies the locative.


\begin{styleExampleTitle}
Suffix \textitbf{\-nya}: Emphatic reading of derived lexemes
\end{styleExampleTitle}

\begin{tabular}{lllllllll}
\lsptoprule
\label{bkm:Ref347167303}
\gll {jatu} {di} {sana,} {di} {sini} {di} {\bluebold{sini–nya}} {ter\bluebold{–}kupas}\\ %
& fall & at & \textsc{l.dist} & at & \textsc{l.prox} & at & \textsc{l.prox}–\textsc{3possr} & \textsc{acl}–peel\\
\lspbottomrule
\end{tabular}
\ea
\glt 
[About a motorbike accident:] ‘he fell (with his bike) over there, here, \bluebold{right here} (his skin) was peeled off’ \textstyleExampleSource{[081014-013-NP.0001]}
\z

\begin{tabular}{llllll}
\lsptoprule
\label{bkm:Ref347167304}
\gll {a} {di} {\bluebold{sini}} {\bluebold{tu}} {bahaya}\\ %
& ah! & at & \textsc{l.prox} & \textsc{d.dist} & be.dangerous\\
\lspbottomrule
\end{tabular}
\ea
\glt
‘ah, \bluebold{right here} it is dangerous’ \textstyleExampleSource{[081011-001-Cv.0138]}
\end{styleFreeTranslEngxvpt}

\paragraph[Summary and conclusions]{Summary and conclusions}
\label{bkm:Ref347157249}
Suffix \textitbf{\-nya} is a polyfunctional affix that derives lexemes from nominal, verbal and a number of other bases. Three observations indicate that in Papuan Malay affixation with \textitbf{\-nya} is a productive process: (1) the polyfunctionality of the suffix and the transparent form-function relationship between the derived lexemes and their respective bases, (2) the large number of low frequency words and small number of high frequency words, and (3) the relative token frequencies with most bases having higher frequencies than the affixed lexemes.



Two other observations, however, do not support this conclusion: (1) speakers usually employ alternative strategies that express the same meanings as the suffixed items, and (2) most of the suffixed items were tentatively classified as SI-borrowings. Also, the findings of a domain analysis suggest that most of the attested tokens can be accounted for in terms of the variables of speaker education levels, topics, and/or role-relations. (Details are discussed in §3.1.8, together with the findings for prefix \textscItalBold{ber\-} ‘\textsc{vblz}’ and circumfix \textitbf{ke}\textitbf{\-}/\textitbf{\-}\textitbf{ang} ‘\textsc{nmlz}’.)
\end{styleBodyvafter}


In considering these conflicting observations, two findings are given special weight, namely the fact that speakers prefer alternative strategies without affixation, and the findings of the domain analysis. Therefore, it is concluded that in Papuan Malay affixation with \textitbf{\-nya} is not used as a productive derivation device. Instead, the suffixed lexemes are best explained as code-switches with Indonesian. (For a detailed discussion of suffix \textitbf{\-nya} in Standard Indonesian and Standard Malay see {Mintz 1994; Sneddon 2010}.)
\end{styleBodyvxvafter}

\subsection{Circumfix \textitbf{ke}\textitbf{\-}/\textitbf{\-}\textitbf{an}\textitbf{g} ‘\textsc{nmlz}’}
\label{bkm:Ref347396558}
Circumfix \textitbf{ke}\textitbf{\-}/\textitbf{\-}\textitbf{ang} ‘\textsc{nmlz}’ is typically attached to verbs. The circumfixed lexemes have a nominal reading; usually they denote stable conditions or attributes, as in (0). Some lexical items also have nominal, numeral, or quantifier bases. Circumfixation with \textitbf{ke}\textitbf{\-}/\textitbf{\-}\textitbf{ang} ‘\textsc{nmlz}’, however, is not used as a productive derivation device in Papuan Malay, as discussed below
\end{styleBodyxafter}

\begin{tabular}{lllllll}
\lsptoprule
\label{bkm:Ref347307300}
\gll {jadi} {itu} {suda} {\bluebold{ke–biasa–ang}} {dari} {dulu}\\ %
& so & \textsc{d.dist} & already & \textsc{nmlz}–be.usual–\textsc{nmlz} & from & first\\
\lspbottomrule
\end{tabular}
\ea
\glt 
‘so already that (has become) a \bluebold{habit} from the past’ \textstyleExampleSource{[081014-007-CvEx.0063]}
\z


The corpus includes 65 lexical items (258 tokens) circumfixed with \textitbf{ke}\textitbf{\-}/\textitbf{\-}\textitbf{ang}:\footnote{\\
\\
\\
\\
\\
\\
\\
The 65 circumfixed lexemes include 22 hapaxes (P=0.0853); the 57 lexemes with verbal bases include 17 hapaxes (P=0.0711); the eight lexemes verbs with nominal, numeral, or quantifier bases include five hapaxes (P=0.2632).\\
\\
\\
\\
\\
\\
\\
\\
}


%\setcounter{itemize}{0}
\begin{itemize}
\item \begin{styleIIndented}
Circumfixed items with verbal bases (57 items with 239 tokens)
\end{styleIIndented}\item \begin{styleIvI}
Circumfixed items with nominal or numeral/quantifier bases (eight items with 19 tokens)
\end{styleIvI}\end{itemize}

The corpus also contains three formally complex words with non-compositional semantics, \textitbf{kebaktiang} ‘religious service’, \textitbf{kecelakaang} ‘accident’, and \textitbf{kegiatang} ‘activity’.



Circumfixed items with verbal bases are discussed in §3.1.7.1, and those with nominal, numeral, or quantifier bases in §3.1.7.2. Pertinent variables of the communicative event that may impact the use of \textitbf{ke}\textitbf{\-}/\textitbf{\-}\textitbf{ang} are examined in §3.1.8. The main findings on circumfix \textitbf{ke}\textitbf{\-}/\textitbf{\-}\textitbf{ang} are summarized and evaluated in §3.1.7.3.
\end{styleBodyvxvafter}

\paragraph[Circumfixed items derived from verbal bases]{Circumfixed items derived from verbal bases}
\label{bkm:Ref339888690}
The corpus includes 57 \textitbf{ke}\textitbf{\-}/\textitbf{\-}\textitbf{ang} -circumfixed lexemes (with 238 tokens) with verbal bases, such as bivalent \textitbf{turung} ‘descend’ or monovalent \textitbf{biasa} ‘be usual’. Of the 57 lexemes 52 are nouns and five are accidental verbs.



The 52 circumfixed nouns typically denote stable conditions or attributes in the sense of ‘state/quality of being \textsc{base}’. Derived words with token frequencies of four or more are listed in Table  ‎3 .22. Examples are presented in (0) and (0). All but one of the affixed lexemes are low frequency words (51 lexemes, attested with less than 20 tokens). Moreover, the token frequencies for the respective bases are (much) higher for most of the derived words (41 lexemes). Of the 52 circumfixed nouns, more than half (27 items) were tentatively classified as borrowings from Standard Indonesian (SI-borrowings) (for details see language internal factor (f in §3.1.1, p. \pageref{bkm:Ref346541608}); in Table  ‎3 .22 these items are underlined.
\end{styleBodyvvafter}

\begin{stylecaption}
\label{bkm:Ref337206396}Table ‎3.\stepcounter{Table}{\theTable}:  Affixation with \textitbf{ke}\textitbf{\-}/\textitbf{\-}\textitbf{ang} of verbal bases
\end{stylecaption}

\tablehead{
 BW & Gloss & Item & Gloss & \textitbf{ke}\textitbf{\-}/\textitbf{\-}\textitbf{ang} \# & \arraybslash BW \#\\
}
\begin{tabular}{llllll}
\lsptoprule
\textitbf{biasa} & ‘be usual’ & \textitbf{kebiasaang} & ‘habit’ & \raggedleft 21 & \raggedleft\arraybslash 185\\
\textitbf{merdeka} & ‘be indepen\-dent’ & \textitbf{kemerdekaang} & ‘freedom’ & \raggedleft 14 & \raggedleft\arraybslash 42\\
\textitbf{baik} & ‘be good’ & \textitbfUndl{kebaikang} & ‘goodness’ & \raggedleft 13 & \raggedleft\arraybslash 182\\
\textitbf{trang} & ‘be clear’ & \textitbfUndl{ketrangang} & ‘explana\-tion’ & \raggedleft 11 & \raggedleft\arraybslash 4\\
\textitbf{tindis} & ‘overlap’ & \textitbfUndl{ketindisang} & ‘k.o. trap’ & \raggedleft 10 & \raggedleft\arraybslash 13\\
\textitbf{turung} & ‘descend’ & \textitbf{keturungang} & ‘descent’ & \raggedleft 9 & \raggedleft\arraybslash 192\\
\textitbf{sempat} & ‘have enough time’ & \textitbf{kesempatang} & ‘opportuni\-ty’ & \raggedleft 9 & \raggedleft\arraybslash 2\\
\textitbf{benar} & ‘be true’ & \textitbfUndl{kebenarang} & ‘truth’ & \raggedleft 9 & \raggedleft\arraybslash 16\\
\textitbf{hidup} & ‘live’ & \textitbf{kehidupang} & ‘life’ & \raggedleft 8 & \raggedleft\arraybslash 74\\
\textitbf{nyata} & ‘be obvious’ & \textitbf{kenyataang} & ‘reality’ & \raggedleft 8 & \raggedleft\arraybslash 1\\
\textitbf{takut} & ‘feel afraid (of)’ & \textitbfUndl{ketakutang} & ‘fear’ &  & \\
\textitbf{sehat} & ‘be healthy’ & \textitbf{kesehatang} & ‘health’ & \raggedleft 7 & \raggedleft\arraybslash 11\\
\textitbf{jahat} & ‘be bad’ & \textitbf{kejahatang} & ‘evilness’ & \raggedleft 7 & \raggedleft\arraybslash 10\\
\textitbf{inging} & ‘wish’ & \textitbf{keingingang} & ‘wish’ & \raggedleft 6 & \raggedleft\arraybslash 6\\
\textitbf{laku} & ‘do’ & \textitbf{kelakuang} & ‘behavior’ & \raggedleft 6 & \raggedleft\arraybslash 5\\
\textitbf{mo} & ‘want’ & \textitbf{kemawang} & ‘will’ & \raggedleft 5 & \raggedleft\arraybslash 972\\
\textitbf{lebi} & ‘be more’ & \textitbf{kelebiang} & ‘surplus’ & \raggedleft 5 & \raggedleft\arraybslash 467\\
\textitbf{saksi} & ‘testify’ & \textitbfUndl{kesaksiang} & ‘testimony’ & \raggedleft 5 & \raggedleft\arraybslash 2\\
\textitbf{ada} & ‘exist’ & \textitbf{keadaang} & ‘condition’ & \raggedleft 4 & \raggedleft\arraybslash 1,742\\
\textitbf{betul} & ‘be true’ & \textitbfUndl{kebetulang} & ‘chance’ & \raggedleft 4 & \raggedleft\arraybslash 123\\
\textitbf{kurang} & ‘lack’ & \textitbf{kekurangang} & ‘shortage’ & \raggedleft 4 & \raggedleft\arraybslash 40\\
\lspbottomrule
\end{tabular}

One \textitbf{ke}\textitbf{\-}/\textitbf{\-}\textitbf{ang}{}-lexeme and its base are given in context: \textitbf{kebaikang} ‘goodness’ in (0) and its base \textitbf{baik} ‘be good’ in (0).


\begin{styleExampleTitle}
Circumfix \textitbf{ke}\textitbf{\-}/\textitbf{\-}\textitbf{ang}: Semantics of base words and derived lexemes
\end{styleExampleTitle}

\begin{tabular}{lllllll}
\lsptoprule
\label{bkm:Ref347318719}
\gll {dong} {masi} {ingat} {de} {pu} {\bluebold{ke–baik–ang}}\\ %
& \textsc{2pl} & pray & \textsc{1pl} & \textsc{3sg} & \textsc{poss} & \textsc{nmlz}–be.good–\textsc{nmlz}\\
\lspbottomrule
\end{tabular}
\ea
\glt 
‘they still remember his/her \bluebold{goodness}’ \textstyleExampleSource{[081110-008-CvNP.0261]}
\z

\begin{tabular}{lllllllll}
\lsptoprule
\label{bkm:Ref347318721}
\gll {knapa} {orang} {bilang,} {adu} {ko} {pu} {sifat} {\bluebold{baik}}\\ %
& why & person & say & oh.no! & \textsc{2sg} & \textsc{poss} & characteristic & be.good\\
\lspbottomrule
\end{tabular}
\ea
\glt 
‘why do people say, ‘oh no, your character is \bluebold{good}’’ \textstyleExampleSource{[081110-008-CvNP.0134]}
\z


As an extension of its function to derive nouns that denote stable states or attributes, five \textitbf{ke}\textitbf{\-}/\textitbf{\-}\textitbf{ang}{}-circumfixed lexemes with verbal bases receive an accidental verbal reading, as listed in Table  ‎3 .23.\footnote{\\
\\
\\
\\
\\
\\
\\
In discussions about \textitbf{ke}\textitbf{\-}/\textitbf{\-}\textitbf{ang}{}-circumfixed lexemes with a verbal reading in Standard Indonesian or Malay, the circumfix is typically glossed as ‘\textsc{advrs}’ (‘adversative’) ({Englebretson 2003}; {Kroeger 2005}) or ‘\textsc{nonvol}’ (‘nonvolitional’) {\citep{Englebretson2007}}.\\
\\
\\
\\
\\
\\
\\
\\
} That is, these items indicate that the referent has undergone an accidental or unintentional action or event, such as \textitbf{keliatang} ‘be visible’ or \textitbf{ketinggalang} ‘be left behind’. An example is presented in (0). All five affixed lexemes are low frequency words, attested with less than 20 tokens. Besides, the token frequencies for the respective bases are (much) higher for all of the derived words. Two of the five accidental verbs were tentatively classified as SI-borrowings; in Table  ‎3 .23 these items are underlined.


\begin{stylecaption}
\label{bkm:Ref347321075}Table ‎3.\stepcounter{Table}{\theTable}:  Verbs with circumfix \textitbf{ke}\textitbf{\-}/\textitbf{\-}\textitbf{ang} with verbal bases
\end{stylecaption}

\tablehead{
 BW & Gloss & Item & Gloss & \textitbf{ke}\textitbf{\-}/\textitbf{\-}\textitbf{ang} \# & \arraybslash BW \#\\
}
\begin{tabular}{llllll}
\lsptoprule
\textitbf{liat} & ‘see’ & \textitbf{keliatang} & ‘be visible’ & \raggedleft 6 & \raggedleft\arraybslash 467\\
\textitbf{tinggal} & ‘stay’ & \textitbf{ketinggalang} & ‘be left behind’ & \raggedleft 5 & \raggedleft\arraybslash 515\\
\textitbf{taw} & ‘know’ & \textitbf{ketawang} & ‘be found out’ & \raggedleft 1 & \raggedleft\arraybslash 603\\
\textitbf{lewat} & ‘pass by’ & \textitbfUndl{kelewatang} & ‘be overly abundant’ & \raggedleft 1 & \raggedleft\arraybslash 140\\
\textitbf{masuk} & ‘enter’ & \textitbfUndl{kemasukang} & ‘be possessed’ & \raggedleft 1 & \raggedleft\arraybslash 261\\
\lspbottomrule
\end{tabular}

One \textitbf{ke}\textitbf{\-}/\textitbf{\-}\textitbf{ang}{}-lexeme and its base are given in context: \textitbf{keliatang} ‘be visible’ in (0) and \textitbf{liat} ‘see’ in (0). The verbal status of \textitbf{keliatang} ‘be visible’ is evidenced by the fact that it is negated with \textitbf{tida} ‘\textsc{neg}’ (nominals cannot be negated with \textitbf{tida} ‘\textsc{neg}’; see §5.2 and §13.1.1).


\begin{styleExampleTitle}
Circumfix \textitbf{ke}\textitbf{\-}/\textitbf{\-}\textitbf{ang}: Verbal reading of derived lexemes
\end{styleExampleTitle}

\begin{tabular}{llllll}
\lsptoprule
\label{bkm:Ref349723492}
\gll {taw{\Tilde}taw} {orang} {itu} {tida} {\bluebold{ke–liat–ang}}\\ %
& suddenly & person & that & \textsc{neg} & \textsc{nmlz}–see–\textsc{nmlz}\\
\lspbottomrule
\end{tabular}
\ea
\glt 
‘suddenly, that person wasn’t \bluebold{visible} (any longer)’ \textstyleExampleSource{[080922-002-Cv.0123]}
\z

\begin{tabular}{lllllllll}
\lsptoprule
\label{bkm:Ref349723493}
\gll {tukang} {ojek} {ini} {dia} {tida} {\bluebold{liat}} {kolam} {ini}\\ %
& craftsman & motorbike.taxi & \textsc{d.prox} & \textsc{3sg} & \textsc{neg} & see & big.hole & \textsc{d.prox}\\
\lspbottomrule
\end{tabular}
\ea
\glt
‘this motorbike taxi driver, he didn’t \bluebold{see} this big hole’ \textstyleExampleSource{[081015-005-NP.0009]}
\end{styleFreeTranslEngxvpt}

\paragraph[Circumfixed items derived from nominal, numeral or quantifier bases]{Circumfixed items derived from nominal, numeral or quantifier bases}
\label{bkm:Ref339888691}
The corpus includes six \textitbf{ke}\textitbf{\-}/\textitbf{\-}\textitbf{ang}{}-circumfixed nouns with nominal bases (with eight tokens), and two nouns with numeral or quantifier bases (with 11 tokens), listed in Table  ‎3 .24. The lexemes with nominal bases express ‘abstract concepts associated with \textsc{base}’, while those with numeral or quantifier bases denote stable conditions in the sense of ‘state of being \textsc{base}’.



All eight affixed lexemes are low frequency words, attested with less than 20 tokens. Moreover, the token frequencies for the respective bases are (much) higher for all of the derived words. Seven of the eight derived lexemes were tentatively classified as SI-borrowings (for details see language internal factor (f in §3.1.1, p. \pageref{bkm:Ref346541608}); in Table  ‎3 .24 these items are underlined.
\end{styleBodyvvafter}

\begin{stylecaption}
\label{bkm:Ref339871234}Table ‎3.\stepcounter{Table}{\theTable}:  Affixation with \textitbf{ke}\textitbf{\-}/\textitbf{\-}\textitbf{ang} of nominal, numeral, and quantifier bases
\end{stylecaption}

\tablehead{
 BW & Gloss & Item & Gloss & \textitbf{ke}\textitbf{\-}/\textitbf{\-}\textitbf{ang} \# & \arraybslash BW \#\\
}
\begin{tabular}{llllll}
\lsptoprule
\multicolumn{6}{l}{Nominal bases}\\
\textitbf{budaya} & ‘culture’ & \textitbfUndl{kebudayaang} & ‘civilization’ & \raggedleft 2 & \raggedleft\arraybslash 18\\
\textitbf{untung} & ‘fortune’ & \textitbfUndl{keuntungang} & ‘advantage’ & \raggedleft 2 & \raggedleft\arraybslash 26\\
\textitbf{camat} & ‘subdistrict head’ & \textitbfUndl{kecamatang} & ‘subdistrict’ & \raggedleft 1 & \raggedleft\arraybslash 22\\
\textitbf{hutang} & ‘forest’ & \textitbfUndl{kehutangang} & ‘forestry’ & \raggedleft 1 & \raggedleft\arraybslash 42\\
\textitbf{pegaway} & ‘civil servant’ & \textitbfUndl{kepegawayang} & ‘civil service’ & \raggedleft 1 & \raggedleft\arraybslash 16\\
\textitbf{uang} & ‘money’ & \textitbfUndl{keuangang} & ‘finances’ & \raggedleft 1 & \raggedleft\arraybslash 139\\
\multicolumn{6}{l}{Numeral and quantifier bases}\\
\textitbf{banyak} & ‘many’ & \textitbf{kebanyakang} & ‘majority’ & \raggedleft 10 & \raggedleft\arraybslash 184\\
\textitbf{satu} & ‘one’ & \textitbfUndl{kesatuang} & ‘unity’ & \raggedleft 1 & \raggedleft\arraybslash 514\\
\lspbottomrule
\end{tabular}

One \textitbf{ke}\textitbf{\-}/\textitbf{\-}\textitbf{ang}\textitbf{{}-}item and its base are given in context: \textitbf{kebanyakang} ‘majority’ in (0) and \textitbf{banyak} ‘many’ in (0).


\begin{tabular}{llllllll}
\lsptoprule
\label{bkm:Ref347309024}
\gll {smua} {orang} {\bluebold{ke–banyak–ang}} {mempunyai} {masala} {tapi} {…}\\ %
& all & person & \textsc{nmlz}–many–\textsc{nmlz} & have & problem & but & \\
\lspbottomrule
\end{tabular}
\ea
\glt 
‘all people, the \bluebold{majority} have problems but …’ \textstyleExampleSource{[080917-010-CvEx.0162]}
\z

\begin{tabular}{llllll}
\lsptoprule
\label{bkm:Ref347309025}
\gll {bua} {apel} {di} {sini} {\bluebold{banyak}}\\ %
& fruit & apple & at & \textsc{l.prox} & many\\
\lspbottomrule
\end{tabular}
\ea
\glt
‘here are \bluebold{many} apples’ (Lit. ‘the apples here are \bluebold{many}’) \textstyleExampleSource{[080922-001a-CvPh.0408]}
\end{styleFreeTranslEngxvpt}

\paragraph[Summary and conclusions]{Summary and conclusions}
\label{bkm:Ref347306726}
Circumfix \textitbf{ke}\textitbf{\-}/\textitbf{\-}\textitbf{ang} is a polyfunctional affix that derives lexemes from verbal, nominal, numeral, and quantifier bases. Three observations suggest that in Papuan Malay affixation with \textitbf{ke}\textitbf{\-}/\textitbf{\-}\textitbf{ang} is a productive process: (1) the polyfunctionality of the circumfix and the transparent form-function relationship between the derived lexemes and their respective bases, (2) the large number of low frequency words and small number of high frequency words, and (3) the relative token frequencies with most bases having higher frequencies than the affixed lexemes. On the other hand, however, more than half of the circumfixed lexemes were tentatively classified as SI-borrowings.



These findings are further qualified by the results of a domain analysis. These results suggest that most of the attested tokens (with verbal, nominal and numeral/quantifier bases) can be accounted for in terms of the variables of speaker education levels, topics, and/or role-relations. Hence, it cannot be concluded that these items are the result of a productive derivation process. (Details are given in §3.1.8, together with the findings for prefix \textscItalBold{ber\-} ‘\textsc{vblz}’ and suffix \textitbf{\-nya} ‘\textsc{3possr}’.)
\end{styleBodyvafter}


Considering the conflicting observations, and taking the findings of the domain analysis as the main decisive factor, it is concluded that in Papuan Malay affixation with \textitbf{ke}\textitbf{\-}/\textitbf{\-}\textitbf{ang} is not used as a productive derivation device. Instead, the circumfixed lexemes are best explained as code-switches with Indonesian. (For a detailed discussion of circumfix \textitbf{ke}\textitbf{\-}/\textitbf{\-}\textitbf{ang} in Standard Indonesian and Standard Malay see {Adelaar 1992; Mintz 1994; Sneddon 2010}.)
\end{styleBodyvxvafter}

\subsection{Variables of the communicative event: Affixes \textscItalBold{ber\-} ‘\textsc{vblz}’, \textitbf{\-nya} ‘\textsc{3possr}’, and \textitbf{ke}\textitbf{\-}/\textitbf{\-}\textitbf{ang} ‘\textsc{nmlz}’}
\label{bkm:Ref347393336}
To further investigate the degrees of productivity for prefix \textscItalBold{ber\-} ‘\textsc{vblz}’, suffix \textitbf{\-nya} ‘\textsc{3possr}’, and circumfix \textitbf{ke}\textitbf{\-}/\textitbf{\-}\textitbf{ang} ‘\textsc{nmlz}’ in Papuan Malay, a domain analysis was conducted. This analysis focused on the variables of speaker education levels, topics, and/or role-relations. In all, 243 items\footnote{\\
\\
\\
\\
\\
\\
\\
Six items with high token frequencies are excluded from the analysis: five \textscItalBold{ber-}prefixed items with more than 50 tokens (in all 413 tokens) and one \textitbf{\-nya}{}-suffixed item with 70 tokens. Given their high token frequencies, it was assumed that speakers employ these items regardless of the variables of speaker education levels, topics, and/or role-relations.\\
In addition, the derivation \textbf{\textit{berusaha}} ‘attempt’ was excluded due to questions concerning the reliability of the recorded tokens. Of its 25 occurrences, 11 were produced by the same speaker during a phone conversation which was characterized by many repetitions due to a bad connection.\\
\\
\\
\\
\\
\\
\\
} with a total of 739 tokens were examined:


\begin{itemize}
\item \begin{styleIIndented}
27 items prefixed with \textscItalBold{ber\-} with verbal bases (94 tokens)
\end{styleIIndented}\item \begin{styleIIndented}
29 items prefixed with \textscItalBold{ber\-} with nominal, numeral, and quantifier bases (70 tokens)
\end{styleIIndented}\item \begin{styleIIndented}
81 items suffixed with \textitbf{\-nya} with nominal bases (215 tokens)
\end{styleIIndented}\end{itemize}
\begin{itemize}
\item \begin{styleIIndented}
36 items suffixed with \textitbf{\-nya} with verbal bases (82 tokens)
\end{styleIIndented}\item \begin{styleIIndented}
5 items suffixed with \textitbf{\-nya} with other bases (20 tokens)
\end{styleIIndented}\end{itemize}
\begin{itemize}
\item \begin{styleIIndented}
57 items circumfixed with \textitbf{ke}\textitbf{\-}/\textitbf{\-}\textitbf{ang} with verbal bases (239 tokens)
\end{styleIIndented}\end{itemize}
\begin{itemize}
\item \begin{styleIvI}
8 items circumfixed with \textitbf{ke}\textitbf{\-}/\textitbf{\-}\textitbf{ang} with nominal and numeral/quantifier bases (19 tokens)
\end{styleIvI}\end{itemize}

For the 243 affixed lexemes, most tokens (684/739 – 93\%) can be explained in terms of speaker education levels, topics, and/or role-relations. The remaining 55/739 tokens (7\%) cannot be accounted for in terms of these variables of the communicative event. These tokens occurred when less-educated speakers (\textsc{\-edc-spk}) conversed with fellow-Papuans of equally low social standing (\textsc{\-stat}) about \textsc{low} topics, that is, casual daily-life issues.\footnote{\\
\\
\\
\\
\\
\\
\\
\\
As mentioned under Factor 3 ‘Relationships between interlocutors’ in §1.5.1 (p. \pageref{bkm:Ref376429590}), all of the recorded less-educated speakers belong to the group of Papuans with lower social status (\textsc{\-stat}), while the recorded Papuans with higher social status (\textsc{+stat}), such as teachers, government officials, or pastors, are all better educated.\\
\\
\\
\\
\\
\\
\\
} (See Table  ‎3 .25 and Figure  ‎3 .9; see also Appendix A for detailed tables and figures for each of the three affixes.)



If the affixed items were the result of a productive affixation process, one would expect the percentage of tokens that cannot be explained in terms of speaker education levels, topics, and/or role-relations to be much higher than 7\%. Instead, most tokens (93\%) seem to be conditioned by these variables of the communicative event. In turn, these findings do not support the conclusion that the affixed lexemes are the result of a productive derivation process. Instead, they seem to be code-switches with Indonesian.
\end{styleBodyvafter}


The data presented in Table  ‎3 .25 and Figure  ‎3 .9 are discussed in more detail below.
\end{styleBodyvvafter}

\begin{stylecaption}
\label{bkm:Ref341105945}Table ‎3.\stepcounter{Table}{\theTable}:  Token frequencies for lexemes affixed with \textscItalBold{ber\-}, \textitbf{\-nya}, and \textitbf{ke}\textitbf{\-}/\textitbf{\-}\textitbf{ang} by speakers, topics, and interlocutors (246 items)
\end{stylecaption}

\tablehead{ & \multicolumn{4}{l}{ Topics (\textsc{top})} & \multicolumn{3}{l}{ Interlocutors (\textsc{ilct})} & \arraybslash Tokens\\
}
\begin{tabular}{lllllllll}
\lsptoprule
\multicolumn{9}{l}{Affixation with \textscItalBold{ber\-} (56 items)}\\
& \textsc{pol} & \textsc{edc} & \textsc{rel} & \textsc{low} & \textsc{+stat} & \textsc{\-stat} & \textsc{outsd} & \arraybslash Total\\
\textsc{+edc-spk} & \raggedleft 20 & \raggedleft 31 & \raggedleft 20 & \raggedleft 18 & \raggedleft {}-{}-{}- & \raggedleft {}-{}-{}- & \raggedleft 14 & \raggedleft\arraybslash 103\\
\textsc{\-edc-spk} & \raggedleft 4 & \raggedleft 9 & \raggedleft 11 & \raggedleft {}-{}-{}- & \raggedleft 11 & \raggedleft \textstyleChBold{1}\textstyleChBold{6} & \raggedleft 10 & \raggedleft\arraybslash 61\\
Subtotal & \raggedleft 24 & \raggedleft 40 & \raggedleft 31 & \raggedleft 18 & \raggedleft 11 & \raggedleft \textstyleChBold{16} & \raggedleft 24 & \raggedleft\arraybslash 164\\
\multicolumn{9}{l}{Affixation with \textitbf{\-nya} (122 items)}\\
& \textsc{pol} & \textsc{edc} & \textsc{rel} & \textsc{low} & \textsc{+stat} & \textsc{\-stat} & \textsc{outsd} & \arraybslash Total\\
\textsc{+edc-spk} & \raggedleft 33 & \raggedleft 35 & \raggedleft 12 & \raggedleft 41 & \raggedleft {}-{}-{}- & \raggedleft {}-{}-{}- & \raggedleft 57 & \raggedleft\arraybslash 178\\
\textsc{\-edc-spk} & \raggedleft 16 & \raggedleft 11 & \raggedleft 28 & \raggedleft {}-{}-{}- & \raggedleft 30 & \raggedleft \textstyleChBold{26} & \raggedleft 28 & \raggedleft\arraybslash 139\\
Subtotal & \raggedleft 49 & \raggedleft 46 & \raggedleft 40 & \raggedleft 41 & \raggedleft 30 & \raggedleft \textstyleChBold{26} & \raggedleft 85 & \raggedleft\arraybslash 317\\
\multicolumn{9}{l}{Affixation with \textitbf{ke}\textitbf{\-}/\textitbf{\-}\textitbf{ang} (65 items)}\\
& \textsc{pol} & \textsc{edc} & \textsc{rel} & \textsc{low} & \textsc{+stat} & \textsc{\-stat} & \textsc{outsd} & \arraybslash Total\\
\textsc{+edc-spk} & \raggedleft 16 & \raggedleft 46 & \raggedleft 14 & \raggedleft 38 & \raggedleft {}-{}-{}- & \raggedleft {}-{}-{}- & \raggedleft 45 & \raggedleft\arraybslash 159\\
\textsc{\-edc-spk} & \raggedleft 7 & \raggedleft 20 & \raggedleft 16 & \raggedleft {}-{}-{}- & \raggedleft 35 & \raggedleft \textstyleChBold{13} & \raggedleft 8 & \raggedleft\arraybslash 99\\
Subtotal & \raggedleft 23 & \raggedleft 66 & \raggedleft 30 & \raggedleft 38 & \raggedleft 35 & \raggedleft \textstyleChBold{13} & \raggedleft 53 & \raggedleft\arraybslash 258\\
\multicolumn{9}{l}{\textstyleChBold{TOTAL} (246 items)}\\
& \textsc{pol} & \textsc{edc} & \textsc{rel} & \textsc{low} & \textsc{+stat} & \textsc{\-stat} & \textsc{outsd} & \arraybslash Total\\
\textsc{+edc-spk} & \raggedleft 69 & \raggedleft 112 & \raggedleft 46 & \raggedleft 97 & \raggedleft {}-{}-{}- & \raggedleft {}-{}-{}- & \raggedleft 116 & \raggedleft\arraybslash 440\\
\textsc{\-edc-spk} & \raggedleft 27 & \raggedleft 40 & \raggedleft 55 & \raggedleft {}-{}-{}- & \raggedleft 76 & \raggedleft \textstyleChBold{55} & \raggedleft 46 & \raggedleft\arraybslash 299\\
\textstyleChBold{Total} & \raggedleft 96 & \raggedleft 152 & \raggedleft 101 & \raggedleft 97 & \raggedleft 76 & \raggedleft \textstyleChBold{55} & \raggedleft 162 & \raggedleft\arraybslash \textstyleChBold{739}\\
\lspbottomrule
\end{tabular}
\begin{styleFigure}
  
%%please move the includegraphics inside the {figure} environment
%%\includegraphics[width=\textwidth]{kluge-img5.jpg}
 
\end{styleFigure}

\begin{styleCapFigure}
\label{bkm:Ref417129343}Figure ‎3.\stepcounter{Figure}{\theFigure}:  Token frequencies for lexemes affixed with \textscItalBold{ber\-}, \textitbf{\-nya}, or \textitbf{ke}\textitbf{\-}/\textitbf{\-}\textitbf{ang} by speakers, topics, and interlocutors
\end{styleCapFigure}


The data presented in Table  ‎3 .25 and Figure  ‎3 .9 show that for the 243 affixed lexemes, most tokens (684/739 – 93\%) can be explained in terms of speaker education levels, topics, and/or role-relations between the speakers and their interlocutors (\textsc{ilct}).
\end{styleBodyaftervbefore}


Most tokens (440/739 – 60\%) were produced by \textsc{+edc-spk}, while 299/739 tokens (40\%) were produced by \textsc{\-edc-spk}.
\end{styleBodyvafter}


The \textsc{+edc-spk} produced about half of their tokens (227/440 – 52\%) when talking about \textsc{high} topics. Another 116/440 \textsc{+edc-spk} tokens (26\%) occurred during conversations with an outsider, namely the author (\textsc{outs}). The remaining 97/440 \textsc{+edc-spk} tokens (22\%) were produced during conversations with fellow-Papuans about \textsc{low} topics.
\end{styleBodyvafter}


The \textsc{\-edc-spk} produced 41\% of their tokens (122/299) while discussing \textsc{high} topics. Another 46/299 \textsc{\-edc-spk} tokens (15\%) occurred during conversations with the author. The remaining \textsc{\-edc-spk} tokens (131/299 – 44\%) were produced during conversations about \textsc{low} topics. Of these, 76/299 tokens (25\%) occurred when \textsc{\-edc-spk} discussed \textsc{low} topics with \textsc{+stat} Papuans. The remaining 55/299 \textsc{\-edc-spk} tokens (18\%) were produced during conversations with \textsc{\-stat} Papuans, and therefore cannot be explained in terms of speaker education levels, topics, and/or role-relations. This total of 55 tokens refers to 7\% of the 739 tokens attested in the corpus.\footnote{\\
\\
\\
\\
\\
\\
\\
\\
As for the attested hapaxes, the findings of this domain analysis suggest that most are conditioned by the variables of speaker education levels, topics, and/or role-relations: 22/25 hapaxes prefixed with \textscItalBold{ber\-}, 59/68 hapaxes suffixed with \textitbf{\-nya}, and 18/22 hapaxes circumfixed with \textitbf{ke\-}/\textitbf{\-ang}. These items are best explained as code-switches with Indonesian. This leaves only three \textscItalBold{ber-}hapaxes, nine \textitbf{\-nya}{}-hapaxes, and four \textitbf{ke\-}/\textitbf{\-ang}{}-hapaxes that are unaccounted for in terms of language external factors and that are likely to be the result of a productive word-formation process. This, in turn, decreases the respective P values:\\
\ea%1
    \label{ex:1}
    \langinfo{lg}{fam}{src}\\
    \gll\\
	\\
    \ea
\glt
    \z

	 for three \textscItalBold{ber-}hapaxes P=0.0183 (N=164) as opposed to P=0.0415 for 25 hapaxes (N=602) (N differs for the two P values, as six of the derived lexemes were excluded from the domain analysis),\\
\ea%2
    \label{ex:2}
    \langinfo{lg}{fam}{src}\\
    \gll\\
	\\
    \ea
\glt
    \z

	 for nine \textitbf{\-nya}{}-hapaxes P=0.0284 (N=317) as opposed to P=0.1758 for 68 hapaxes (N=387) (N differs for the two P values, as one derived lexeme was excluded from the domain analysis), and\\
\ea%3
    \label{ex:3}
    \langinfo{lg}{fam}{src}\\
    \gll\\
	\\
    \ea
\glt
    \z

	 for four \textitbf{ke\-}/\textitbf{\-ang}{}-hapaxes P=0.0155 as opposed to P=0.0853 for 22 hapaxes (N=258).\\
\\
\\
\\
}
\end{styleBodyvxvafter}

\section{Compounding}
\label{bkm:Ref341973806}
Compounding denotes the “formation of a new lexeme by adjoining two or more lexemes” {\citep[40]{Bauer2003}}. In Papuan Malay, however, the demarcation between compounds and phrasal expressions is unclear. That is, neither phonological, morphological, morphosyntactic, nor semantic criteria allow classifying word sequences unambiguously as either compounds or phrasal expressions, as shown in §3.2.1. The attested word combinations always have a binary structure in that they consist of two juxtaposed lexemes; the first component is always a noun. More specifically, three types of word sequences can be distinguished, namely endocentric, exocentric, and coordinative ones, as discussed in §3.2.2. The main points on compounding are summarized in §3.3.
\end{styleBodyxvafter}

\subsection{Demarcation of compounds from phrasal expressions}
\label{bkm:Ref374460181}\label{bkm:Ref339619435}
Four different criteria have been suggested to distinguish compounds from phrasal expressions: phonological, morphological, morphosyntactic, and semantic criteria {\citep[24]{Aikhenvald2007}}. They are discussed in turn in this section.



On phonological grounds, compounds can be distinguished from phrasal expressions in terms of their stress behavior. Compounds typically contain one primary stress, whereas in phrasal expressions each phonological word carries its own stress {\citep[25]{Aikhenvald2007}}. This criterion also applies to Papuan Malay, as shown in (0) and (0). In the compound \textitbf{kacang-hijow} ‘mung bean’ in (0), the penultimate syllable carries primary stress, while secondary stress is assigned to the alternating syllable preceding the one carrying the primary stress. By contrast, in the phrasal expression \textitbf{kacang hijow} ‘green bean’ in (0) each constituent carries its own stress. In fast speech, however, it is difficult to distinguish both constructions on phonological grounds. Instead, the context is the determining factor to establish the intended meaning.
\end{styleBodyvvafter}

\begin{styleExampleTitle}
Phonological criteria
\end{styleExampleTitle}

\begin{tabular}{llll}
\lsptoprule
\label{bkm:Ref339619911}
\gll {/\textstyleChCharisSIL{ˌka.tʃaŋ.ˈhi.dʒ}ɔ\textstyleChCharisSIL{w}/} {\bluebold{kacang-hijow}} {‘mung bean’}\\ %
&  & bean-be.green & \\
\label{bkm:Ref339619912}
\gll {/\textstyleChCharisSIL{ˈka.tʃaŋ ˈhi.dʒ}ɔ\textstyleChCharisSIL{w}/} {\bluebold{kacang hijow}} {‘green bean’}\\ %
&  & bean be.green & \\
\lspbottomrule
\end{tabular}

As for morphological criteria, compounds are typically distinct from phrasal expressions in that the former are marked with additional morphemes or have distinct constituent orders vis-à-vis phrasal expressions {\citep[26]{Aikhenvald2007}}. In terms of morphological criteria, however, Papuan Malay compounds are not distinct from phrasal expressions. As illustrated in (0) and (0), neither construction has an additional morpheme that would mark it as a compound or phrasal expression. Neither are the two constructions distinct in terms of their constituent order, as in each case the head precedes the modifier.
\end{styleBodyaftervbefore}


On morphosyntactic grounds, compounds are usually distinct from phrasal expressions in that the components of a compound cannot be separated by inserting other morphemes {\citep[26]{Aikhenvald2007}}. Such an insertion leads to the loss of their compound sense. This criterion also applies to Papuan Malay as shown in (0) and (0). When, for instance, the relativizer \textitbf{yang} ‘\textsc{rel}’ is inserted in the compound \textitbf{lemon-manis} ‘orange’ in (0), the compound sense is lost. The result is the phrasal expression \textitbf{lemon yang manis} ‘lemon which is sweet’ or ‘sweet lemon’ in (0).
\end{styleBodyvvafter}

\begin{styleExampleTitle}
Morphosyntactic criteria
\end{styleExampleTitle}

\begin{tabular}{lll}
\lsptoprule
\label{bkm:Ref339619913}
\gll {\bluebold{lemon-manis}} {‘orange’}\\ %
& lemon-be.sweet & \\
\label{bkm:Ref339619914}
\gll {\bluebold{lemon yang manis}} {‘lemon which is sweet’}\\ %
& lemon \textsc{rel} be.sweet & \\
\lspbottomrule
\end{tabular}

In cases such as the compound \textitbf{orang-tua} ‘parent’ in (0) or the phrasal expression \textitbf{orang tua} ‘old person’ in (0), however, it is difficult to distinguish both constructions on morphosyntactic grounds. Again, the context is the determining factor to establish the intended meaning.


\begin{styleExampleTitle}
Ambiguities with respect to morphosyntactic criteria
\end{styleExampleTitle}

\begin{tabular}{lll}
\lsptoprule
\label{bkm:Ref339619915}
\gll {\bluebold{orang-tua}} {‘parent’}\\ %
& person-be.old & \\
\label{bkm:Ref339619916}
\gll {\bluebold{orang tua}} {‘old person’}\\ %
& person be.old & \\
\lspbottomrule
\end{tabular}

Semantically, compounds and phrasal expressions can be arranged on a scale from less to more compositional {\citep[28]{Aikhenvald2007}}. The corpus, however, does not contain non-compositional compounds with idiosyncratic semantics.\footnote{\\
\\
\\
\\
\\
\\
\\
\\
\\
\\
\\
While {\citet[28]{Aikhenvald2007}} suggests that compounds can also be compositional, {\citet[175]{Dryer2007b}} maintains that compounds have “an idiosyncratic meaning not predictable from the meaning of the component parts, as compared with syntactic compounds, in which one noun is modifying a second noun in a productive syntactic construction”.\\
\\
\\
\\
} This is illustrated in (0) to (0). Less compositional compounds are expressions such as \textitbf{kampung-tana} ‘home village’ in (0), or \textitbf{paduang-swara} ‘choir’ in (0). Compounds that are more compositional are those whose meaning is predictable from the meanings of its parts, such as \textitbf{air-mata} ‘tears’ in (0) or \textitbf{tali-prut} ‘intestines’ in (0). Very transparent compounds blend into phrasal expressions such as \textitbf{uang jajang} ‘pocket money’ or ‘money for snacks’ in (0). On the one hand one could say that \textitbf{uang jajang} ‘pocket money’ is a compound with an idiosyncratic meaning. On the other hand one could argue that this construction has a phrasal structure that denotes a purpose relation between the nominal head \textitbf{uang} ‘money’ and its nominal modifier \textitbf{jajang} ‘snack’; hence, the construction \textitbf{uang jajang} ‘money for snacks’ is a phrasal expression and not a compound. Finally, there are phrasal expressions with clear compositional semantics, such as \textitbf{air sagu} ‘liquid of the sago palm tree’ in (0). (For details on noun phrases with nominal modifiers see §8.2.2.)


\begin{styleExampleTitle}
Semantic criteria
\end{styleExampleTitle}

\begin{tabular}{lll}
\lsptoprule
\label{bkm:Ref339619918}
\gll {\bluebold{kampung-tana}} {‘home village’}\\ %
& village-ground & \\
\label{bkm:Ref339619917}
\gll {\bluebold{paduang-swara}} {‘choir’}\\ %
& fusion-voice & \\
\label{bkm:Ref386138251}
\gll {\bluebold{air-mata}} {‘tears’}\\ %
& water-eye & \\
\label{bkm:Ref367869296}
\gll {\bluebold{tali-prut}} {‘intestines’}\\ %
& cord-stomach & \\
\label{bkm:Ref339619922}
\gll {\bluebold{uang jajang}} {‘pocket money’ / ‘money for snacks’}\\ %
& money snack & \\
\label{bkm:Ref339619921}
\gll {\bluebold{air sagu}} {‘liquid of the sago palm tree’}\\ %
& water sago & \\
\lspbottomrule
\end{tabular}

The data presented in this section shows that in Papuan Malay neither phonological, morphological, morphosyntactic, nor semantic criteria allow the unambiguous classification of word sequences as either compounds or phrasal expressions. Instead the data suggest that, following Lieber and Štekauer’s (2009: 14) definition of compounding, some Papuan Malay word combinations are “more compoundlike” while others are “less compoundlike […] with no clear categorical distinction” along this “cline”. The combinations range from less compositional two-word expressions such as \textitbf{kampung-tana} ‘home village’ to those with compositional transparent semantics such as \textitbf{air sagu} ‘liquid of the sago palm tree’. Given this lack of a clear demarcation between compounds and phrasal expressions, the term “collocation” rather than “compound” is used hereafter for such juxtaposed word sequences.\footnote{\\
\\
\\
\\
\\
\\
\\
\\
\\
\\
\\
Collocations are defined as “word combinations which have developed an idiomatic semantic relation based on their frequent co-occurrence” ({Bussmann 1996: 200}; see also {Krishnamurthy 2006}).\\
\\
\\
\\
}


\subsection{Types of collocations}
\label{bkm:Ref434690508}
In Papuan Malay, three types of collocations are found: endocentric, exocentric, and coordinative ones. In the following they are discussed one by one. 



In endocentric collocations, one component has head function while the subordinate component has modifying, content-specifying function, denoting “a sub-class of the items denoted by one of their elements” {\citep[42]{Bauer2003}}. In Papuan Malay endocentric collocations, the head component always precedes the modifier component which can be a noun or a stative verb. Semantically, these ‘\textsc{n} \textsc{n}’ or ‘\textsc{n} \textsc{v}’ collocations encode different types of relationships between their components such as ‘part-whole’, ‘subtype-of’, or ‘characteristic-of’ relations, as illustrated in Table  ‎3 .26. In addition, the corpus contains one collocation in which the modifying component is a numeral: \textitbf{segi-empat} ‘quadrangle’ (literally ‘side-four’).
\end{styleBodyvvafter}

\begin{stylecaption}
\label{bkm:Ref346103375}Table ‎3.\stepcounter{Table}{\theTable}:  Endocentric ‘\textsc{n} \textsc{n/v}’ collocations
\end{stylecaption}

\tablehead{
 Item & Gloss & Literal translation & \arraybslash Semantic relation\\
}
\begin{tabular}{llll}
\lsptoprule
\textitbf{tali-prut} & ‘intestines’ & ‘cord of the stomach’ & ‘Part-whole’\\
cord-stomach &  &  & \\
\textitbf{lemon-manis} & ‘orange’ & ‘sweet lemon’ & ‘Subtype-of’\\
lemon-be.sweet &  &  & \\
\textitbf{kreta-api} & ‘train’ & ‘carriage of fire’ & ‘Characteristic-of’\\
carriage-fire &  &  & \\
\lspbottomrule
\end{tabular}

In exocentric collocations, none of the constituents functions as its head. They “denote something which is not a sub-class” of either of their components; that is, “they are not hyponyms of either of their elements” {\citep[42]{Bauer2003}}, as shown in Table  ‎3 .27. In the collocation \textitbf{bapa-ade}, literally ‘father-younger.sibling’, for instance, neither of the two components serves as the content-specifying element. Likewise \textitbf{kepala-batu}, literally ‘head-stone’ does not refer to some kind of head. Instead, it denotes a ‘pig-headed person’. These examples also show that exocentric collocations typically consist of two juxtaposed nouns.


\begin{stylecaption}
\label{bkm:Ref346103376}Table ‎3.\stepcounter{Table}{\theTable}:  Exocentric ‘\textsc{n} \textsc{n}’ collocations
\end{stylecaption}

\tablehead{
 Item & \arraybslash Gloss\\
}
\begin{tabular}{ll}
\lsptoprule
\textitbf{bapa-ade} & ‘father’s younger brother’ (FyB) / ‘mother’s younger sister’s husband’ (MyZH)\\
father-ySb & \\
\textitbf{kepala-batu} & ‘pig-headed person’\\
head-stone & \\
\textitbf{mata-hari} & ‘sun’\\
eye-day & \\
\lspbottomrule
\end{tabular}

The distinction between endocentric and exocentric collocations is not always clear-cut, however, as shown in Table  ‎3 .28. The kinship terms \textitbf{bapa-tua} ‘uncle’ (literally ‘father-be.old’) and \textitbf{mama-tua} ‘aunt’ (literally ‘mother-be.old’) qualify as exocentric collocations on semantic grounds but as endocentric collocations on syntactic grounds. Both terms are exocentric in that they designate something which is not a sub-class of either of their components: \textitbf{bapa-tua} does not refer to an ‘old father’, neither does \textitbf{mama-tua} refer to an ‘old mother’. Instead, \textitbf{bapa-tua} denotes a ‘parent’s older brother’ (PoB) or a ‘parent’s older sister’s husband’ (PoZH), while \textitbf{mama-tua} designates a ‘parent’s older sister’ (PoZ) or a ‘parent’s older brother’s wife’ (PoBW). Syntactically, however, \textitbf{tua} ‘be old’ is subordinate to the head \textitbf{bapa}/\textitbf{mama} ‘father/mother’ and has modifying content-specifying function. Hence, both kinship terms also qualify as endocentric collocations.


\begin{stylecaption}
\label{bkm:Ref346115320}Table ‎3.\stepcounter{Table}{\theTable}:  Endocentric versus exocentric collocations: Ambiguities
\end{stylecaption}

\tablehead{
 Item & \arraybslash Gloss\\
}
\begin{tabular}{ll}
\lsptoprule
\textitbf{bapa-tua} & ‘parent’s older brother’ (PeB) / ‘parent’s older sister’s husband’ (PeZH)\\
father-be.old & \\
\textitbf{mama-tua} & ‘parent’s older sister’ (PeZ) / ‘parent’s older brother’s wife’ (PeBW)\\
mother-be.old & \\
\hhline{-~}
\lspbottomrule
\end{tabular}

Coordinative collocations designate entities made up of two nominal components that “can be interpreted as being joined by ‘and’” {\citep[351]{Bauer2009}}. That is, in such collocations both components “are of semantically equal weight” {\citep[221]{Bussmann1996}}. The nominal components can be antonyms, synonyms, or different parts or aspects of the designated concept, as shown in Table  ‎3 .29.


\begin{stylecaption}
\label{bkm:Ref346110435}Table ‎3.\stepcounter{Table}{\theTable}:  Coordinative ‘\textsc{n} \textsc{n}’ collocations
\end{stylecaption}

\tablehead{
 Item & Gloss & \arraybslash Semantic relation\\
}
\begin{tabular}{lll}
\lsptoprule
\textitbf{ade-kaka} & ‘siblings’ & Antonyms\\
ySb-oSb &  & \\
\textitbf{kasi-sayang} & ‘deep love’ & Synonyms\\
love-love &  & \\
\textitbf{guntur-kilat} & ‘thunderstorm’ & Different parts/aspects\\
thunder-lightning &  & \\
\textitbf{tete-moyang} & ‘ancestors’ & Different parts/aspects\\
grandfather-ancestor &  & \\
\lspbottomrule
\end{tabular}
\section{Summary}
\label{bkm:Ref341961694}
This section briefly summarizes the main points on affixation and compounding.


%\setcounter{itemize}{0}
\begin{itemize}
\item \begin{styleOvNvwnext}
Affixation
\end{styleOvNvwnext}\end{itemize}

Affixation in Papuan Malay has very limited productivity. This conclusion is based on an investigation of six affixes: the prefixes \textscItalBold{ter\-} ‘\textsc{acl}’, \textscItalBold{pe(n)\-} ‘\textsc{ag}’, and \textscItalBold{ber\-} ‘\textsc{vblz}’, the suffixes \-\textitbf{ang} ‘\textsc{pat}’ and \textitbf{\-nya} ‘\textsc{3possr}’, and the circumfix \textitbf{ke}\-/\-\textitbf{ang} ‘\textsc{nmlz}’. Given the sociolinguistic profile of Papuan Malay (substantial language contact between Papuan Malay and Indonesian with both languages being in a diglossic distribution, positive to somewhat ambivalent language attitudes toward Papuan Malay, and lack of language awareness of many Papuan Malay speakers) no productivity testing was conducted, as a substantial amount of interference from Indonesian was expected. This interference would have skewed testees’ naïve judgments. Instead, the six affixes were examined in terms of six language internal and three language external factors considered relevant in establishing the degree of productivity of these affixes.



The results of this investigation are as follows:
\end{styleBodyvvafter}

%\setcounter{itemize}{0}
\begin{itemize}
\item 
Papuan Malay \textscItalBold{ter\-} ‘\textsc{acl}’ has limited productivity; it indicates accidental or unintentional actions or events. In other eastern Malay varieties and in Standard Malay, the prefix is rather productive; here it likewise signals accidental or unintentional actions or events.
\end{styleBodyvafter}\item 
Papuan Malay \-\textitbf{ang} ‘\textsc{pat}’ has limited productivity; it typically designates the patient or result of an action, event or state. As for other eastern Malay varieties, the suffix is only mentioned for Ambon Malay; its degree of productivity is unclear. In Standard Malay the suffix is very productive. Both in Ambon and in Standard Malay, the suffix also indicates the patient or product of an action, event or state.
\end{styleBodyvafter}\item 
Papuan Malay \textscItalBold{pe(n)\-} ‘\textsc{ag}’ has marginal productivity, at best. It typically denotes the subject of the action, event, or state specified by the verbal base; some of the affixed lexemes also receive an intensified intransitive or monotransitive reading. As for other eastern Malay varieties, the prefix seems to have retained its productivity only in Ternate Malay. In Standard Malay, the suffix is very productive. In other Malay varieties the prefix likewise denotes the subject of the action, event, or state specified by the verbal base. A verbal interpretation, but not the intensified reading, is also reported for other eastern Malay varieties. In Standard Malay, by contrast, only derivations with monovalent stative bases can function as monovalent stative verbs.
\end{styleBodyvafter}\item 
In Papuan Malay, prefix \textscItalBold{ber\-} ‘\textsc{vblz}’ is unproductive, whereas in other eastern Malay varieties and Standard Malay the prefix is very productive.
\end{styleBodyvafter}\item 
In Papuan Malay, \textitbf{\-nya} ‘\textsc{3possr}’ and \textitbf{ke}\textitbf{\-}/\textitbf{\-}\textitbf{ang} ‘\textsc{nmlz}’ are unproductive. The same applies to other eastern Malay varieties, while both affixes are very productive in Standard Malay.
\end{styleBodyxafter}\end{itemize}
%\setcounter{itemize}{0}
\begin{itemize}
\item \begin{styleOvNvwnext}
Compounding
\end{styleOvNvwnext}\end{itemize}

In Papuan Malay, the demarcation between compounds and phrasal expressions is unclear. Neither phonological, morphological, morphosyntactic, nor semantic criteria allow the unambiguous classification of two juxtaposed nouns as compounds and phrasal expressions. Therefore, the term “collocation” is employed as a cover term for such word combinations that differ in transparency from non-compositional idiosyncratic semantics to compositional transparent semantics. Three different types of collocations are attested, endocentric, exocentric, and coordinative ones. Given the lack of a clear demarcation between compounds and phrasal expressions, it remains unclear to what degree compounding is a productive process.


%\setcounter{page}{1}\chapter[Reduplication]{Reduplication}
\label{bkm:Ref372621586}
Reduplication refers to “the morphological operation in which a new word (form) is created by copying a word or a part thereof, and affixing that copy to the base” {\citep[321]{Booij2007}}. In Papuan Malay, as in other Austronesian languages {(Himmelmann 2005: 121–125)}, reduplication is a very productive morphological device to derive new words.



With respect to lexeme formation, Papuan Malay makes use of three different types of reduplication: (1) full reduplication, (2) partial reduplication, and (3) imitative reduplication. Alternatively, {Wiltshire and Marantz}{ }{(1978: 558)} refer to these reduplication types as “exact total reduplication”, “exact partial reduplication”, and “inexact partial reduplication”, respectively. In terms of lexeme interpretation, a variety of meanings can be attributed to the reduplicated lexemes, such as plurality and diversity, intensity, or continuation and repetition.
\end{styleBodyvafter}


Reduplication in terms of lexeme formation is described in §4.1 while lexeme interpretation is discussed in §4.2. This discussion is followed by a comparison of reduplication across different eastern Malay varieties in §4.3. The main points of this chapter are summarized in §4.4.
\end{styleBodyvxvafter}

\section{Lexeme formation}
\label{bkm:Ref360096569}
A phonological approach to reduplication is {Marantz’s (1982: 436)} prosodic template model which views reduplication as “normal affixation” with “one unique feature”, namely “the resemblance of the added material to the stem being reduplicated”. More specifically, “every reduplication process may be characterized by a ‘skeleton’ of some sort”, either a phonemic melody, “a C-V skeleton, a syllabic skeleton, or a skeleton of morpheme symbols” {(1982: 439)}. The four-tiered representation in (0), taken from {\citet[437]{Marantz1982}}, illustrates how the segments of the four skeleta are linked to each other.
\end{styleBodyxafter}

\begin{tabular}{llllllllll}
\lsptoprule
\label{bkm:Ref346352741}
\gll {phonemic melody} {p1} {p2} {p3} {p4} {p5} {p6} {p7} {\arraybslash …}\\ %
&  & \textitbfxivpt{{\textbar}} & \textitbfxivpt{{\textbar}} & \textitbfxivpt{{\textbar}} & \textitbfxivpt{{\textbar}} & \textitbfxivpt{{\textbar}} & \textitbfxivpt{{\textbar}} & \textitbfxivpt{{\textbar}} & \\
& C-V skeleton & C & V & C & C & V & C & V & \\
&  & \multicolumn{3}{l}{   
%%please move the includegraphics inside the {figure} environment
%%\includegraphics[width=\textwidth]{kluge-img6.jpg}
 } & \multicolumn{2}{l}{   
%%please move the includegraphics inside the {figure} environment
%%\includegraphics[width=\textwidth]{kluge-img7.jpg}
 } & \multicolumn{2}{l}{   
%%please move the includegraphics inside the {figure} environment
%%\includegraphics[width=\textwidth]{kluge-img8.jpg}
 } & \\
& syllabic skeleton & \multicolumn{3}{l}{ $\sigma $} & \multicolumn{2}{l}{ $\sigma $} & \multicolumn{2}{l}{ $\sigma $} & \\
&  & \multicolumn{7}{l}{   
%%please move the includegraphics inside the {figure} environment
%%\includegraphics[width=\textwidth]{kluge-img9.jpg}
 } & \\
& morpheme symbol & \multicolumn{7}{l}{ $\mu $} & \\
& p1 = phoneme & \multicolumn{7}{l}{} & \\
& $\sigma $ = syllable & \multicolumn{7}{l}{} & \\
\lspbottomrule
\end{tabular}

During reduplication, an affixed skeleton receives its phonemic content by “the copying of the stem’s phonemic melody on the same tier as the melody and on the same side of the stem melody to which the affix is attached […] along with some specific constraints on the autosegmental association of the phonemes of the copied melody with the Cs and Vs of reduplicating morphemes” {\citep[445]{Marantz1982}}.
\end{styleBodyaftervbefore}


Full and partial reduplication use two different types of skeleta. In full reduplication, the affix is a morphemic skeleton or, more specifically, the morphological word. In partial reduplication, the added material is a syllabic skeleton. In Papuan Malay, this syllabic skeleton is a closed, heavy syllable which gets prefixed to the base. This shows, that in Papuan Malay reduplication in general is prefixal rather suffixal.
\end{styleBodyvafter}


Both types of lexeme formation are described in §4.1.1.1 and §4.1.1.2, respectively. Imitative reduplication is discussed in §4.1.1.3.
\end{styleBodyvxvafter}

\subsection{Full reduplication}

Full reduplication is very common in Papuan Malay. Cross-linguistically in full reduplication “the reduplicant matches the base from which it is copied without phoneme changes or additions” {\citep[2]{Rubino2011}}. That is, in terms of {Marantz’s (1982) }prosodic template model, full morpheme reduplication involves “the addition of a morphemic skeleton to a stem. The morphemic skeleton, lacking a syllabic skeleton, a C-V skeleton, and a phonemic melody, borrows all three from the stem to which it attaches” {(1982: 456)}.



Full reduplication of morphological words in Papuan Malay is illustrated with the two examples in (0): reduplication of the root \textitbf{dorang} ‘\textsc{3pl}’, resulting in \textitbf{dorang{\Tilde}dorang} ‘\textsc{rdp}{\Tilde}\textsc{3pl}’ in (0), and reduplication of the derived word \textitbf{tingkatang} ‘level’ (\textitbf{tingkat}\-\textitbf{ang} ‘floor-\textsc{pat}’), resulting in ‘\textitbf{tingkatang{\Tilde}tingkatang} ‘\textsc{rdp}{\Tilde}level’ in (0). In each case, the content of the reduplicative affix is obtained by copying the phonemic melody of the base over the morphemic skeleton of the reduplicating affix. This applies to roots as in (0) as well as to derived words as in (0).
\end{styleBodyvxafter}

\begin{tabular}{lllllllllllllllllll}
\lsptoprule
\label{bkm:Ref346365355}\label{bkm:Ref362083078}
\gll {\label{bkm:Ref362083091}} {\multicolumn{17}{l}{[\textstyleChCharisSIL{ˌdɔ.rɐŋ.ˈdɔ.rɐŋ}]}}\\ %
&  &  &  &  & d & ɔ & r & ɐ & ŋ &  & d & ɔ & r & ɐ & ŋ &  &  & \\
&  &  &  &  & \textitbfxivpt{{\textbar}} & \textitbfxivpt{{\textbar}} & \textitbfxivpt{{\textbar}} & \textitbfxivpt{{\textbar}} & \textitbfxivpt{{\textbar}} &  & \textitbfxivpt{{\textbar}} & \textitbfxivpt{{\textbar}} & \textitbfxivpt{{\textbar}} & \textitbfxivpt{{\textbar}} & \textitbfxivpt{{\textbar}} &  &  & \\
&  &  &  &  & C & V & C & V & C & + & C & V & C & V & C &  &  & \\
&  &  &  &  & \multicolumn{2}{l}{   
%%please move the includegraphics inside the {figure} environment
%%\includegraphics[width=\textwidth]{kluge-img10.jpg}
 } & \multicolumn{3}{l}{   
%%please move the includegraphics inside the {figure} environment
%%\includegraphics[width=\textwidth]{kluge-img11.jpg}
 } &  & \multicolumn{2}{l}{   
%%please move the includegraphics inside the {figure} environment
%%\includegraphics[width=\textwidth]{kluge-img12.jpg}
 } & \multicolumn{3}{l}{   
%%please move the includegraphics inside the {figure} environment
%%\includegraphics[width=\textwidth]{kluge-img13.jpg}
 } &  &  & \\
&  &  &  &  & \multicolumn{2}{l}{ $\sigma $} & \multicolumn{3}{l}{ $\sigma $} &  & \multicolumn{3}{l}{ $\sigma $} & \multicolumn{2}{l}{ $\sigma $} &  &  & \\
&  &  &  &  & \multicolumn{5}{l}{  
%%please move the includegraphics inside the {figure} environment
%%\includegraphics[width=\textwidth]{kluge-img14.jpg}
 } &  & \multicolumn{5}{l}{  
%%please move the includegraphics inside the {figure} environment
%%\includegraphics[width=\textwidth]{kluge-img15.jpg}
 } &  &  & \\
&  &  &  &  & \multicolumn{5}{l}{ $\mu $} &  & \multicolumn{5}{l}{ $\mu $} &  &  & \\
&  & \multicolumn{17}{l}{}\\
& \label{bkm:Ref362083094} & \multicolumn{17}{l}{[\textstyleChCharisSIL{tɪŋ.ˌka.tɐŋ.tɪŋ.ˈka.tɐŋ}]}\\
&  & t & ɪ & ŋ & k & a & t & ɐ & ŋ &  & t & ɪ & ŋ & k & a & t & ɐ & \arraybslash ŋ\\
&  & \textitbfxivpt{{\textbar}} & \textitbfxivpt{{\textbar}} & \textitbfxivpt{{\textbar}} & \textitbfxivpt{{\textbar}} & \textitbfxivpt{{\textbar}} & \textitbfxivpt{{\textbar}} & \textitbfxivpt{{\textbar}} & \textitbfxivpt{{\textbar}} &  & \textitbfxivpt{{\textbar}} & \textitbfxivpt{{\textbar}} & \textitbfxivpt{{\textbar}} & \textitbfxivpt{{\textbar}} & \textitbfxivpt{{\textbar}} & \textitbfxivpt{{\textbar}} & \textitbfxivpt{{\textbar}} & \arraybslash \textitbfxivpt{{\textbar}}\\
&  & C & V & C & C & V & C & V & C & + & C & V & C & C & V & C & V & \arraybslash C\\
&  & \multicolumn{3}{l}{   
%%please move the includegraphics inside the {figure} environment
%%\includegraphics[width=\textwidth]{kluge-img16.jpg}
 } & \multicolumn{2}{l}{   
%%please move the includegraphics inside the {figure} environment
%%\includegraphics[width=\textwidth]{kluge-img17.jpg}
 } & \multicolumn{3}{l}{   
%%please move the includegraphics inside the {figure} environment
%%\includegraphics[width=\textwidth]{kluge-img18.jpg}
 } &  & \multicolumn{3}{l}{   
%%please move the includegraphics inside the {figure} environment
%%\includegraphics[width=\textwidth]{kluge-img19.jpg}
 } & \multicolumn{2}{l}{   
%%please move the includegraphics inside the {figure} environment
%%\includegraphics[width=\textwidth]{kluge-img20.jpg}
 } & \multicolumn{3}{l}{   
%%please move the includegraphics inside the {figure} environment
%%\includegraphics[width=\textwidth]{kluge-img21.jpg}
 }\\
&  & \multicolumn{3}{l}{ $\sigma $} & \multicolumn{2}{l}{ $\sigma $} & \multicolumn{3}{l}{ $\sigma $} &  & \multicolumn{3}{l}{ $\sigma $} & \multicolumn{2}{l}{ $\sigma $} & \multicolumn{3}{l}{ $\sigma $}\\
&  & \multicolumn{8}{l}{   
%%please move the includegraphics inside the {figure} environment
%%\includegraphics[width=\textwidth]{kluge-img22.jpg}
 } &  & \multicolumn{8}{l}{   
%%please move the includegraphics inside the {figure} environment
%%\includegraphics[width=\textwidth]{kluge-img23.jpg}
 }\\
&  & \multicolumn{8}{l}{ $\mu $} &  & \multicolumn{8}{l}{ $\mu $}\\
\lspbottomrule
\end{tabular}

In Papuan Malay, only words are reduplicated; bound morphemes such as prefixes are never reduplicated (see Table  ‎4 .1 in §4.1.1.1). Full reduplication is attested for content words (§4.1.1.1) and some function words (§4.1.1.2). The corpus also includes a few reduplicated items that do not have an unreduplicated single base (§4.1.1.3). Reduplication of reduplicated bases is unattested.


\paragraph[Reduplication of content words]{Reduplication of content words}
\label{bkm:Ref339284930}
Full reduplication most commonly applies to content words. Attested are reduplicated nouns, verbs, adverbs, numerals, and quantifiers, as shown in Table  ‎4 .1.


\begin{stylecaption}
\label{bkm:Ref339283797}Table ‎4.\stepcounter{Table}{\theTable}:  Reduplication of content words\footnote{\\
\\
\\
\\
\\
\\
\\
\\
\\
\\
\\
As discussed in §4.2, reduplication conveys a variety of different meanings. Hence, a reduplicated item can receive different interpretations, depending on the context. The reduplicated noun \textitbf{tulang{\Tilde}tulang}, for instance, can receive the following readings: ‘any one of the bones’, ‘different kinds of bones’, ‘all of the bones’. Hence, no translation is given for the reduplicated items in Table  ‎4 .1. The same applies to Table  ‎4 .2 in §4.1.1.2.\\
\\
\\
\\
}
\end{stylecaption}

\tablehead{
 Word class & Base & Gloss & \arraybslash Reduplicated item\\
}
\begin{tabular}{llll}
\lsptoprule
Nouns & \textitbf{ade} & ‘younger sibling’ & \textitbf{ade{\Tilde}ade}\\
& \textitbf{bua} & ‘fruit’ & \textitbf{bua{\Tilde}buaang}\\
& \textitbf{tingkatang} & ‘level’ & \textitbf{tingkatang{\Tilde}tingkatang}\\
& \textitbf{tulang} & ‘bone’ & \textitbf{tulang{\Tilde}tulang}\\
Verbs & \textitbf{baik} & ‘be good’ & \textitbf{baik{\Tilde}baik}\\
& \textitbf{ceritra} & ‘tell’ & \textitbf{ceritra{\Tilde}ceritra}\\
& \textitbf{talipat} & ‘fold’ & \textitbf{talipat{\Tilde}talipat}\\
& \textitbf{tumpuk} & ‘pile’ & \textitbf{bertumpuk{\Tilde}tumpuk}\\
Adverbs & \textitbf{baru} & ‘recently’ & \textitbf{baru{\Tilde}baru}\\
& \textitbf{skarang} & ‘now’ & \textitbf{skarang{\Tilde} skarang}\\
& \textitbf{sring} & ‘often’ & \textitbf{sring{\Tilde}sring}\\
Numerals & \textitbf{satu} & ‘one’ & \textitbf{satu{\Tilde}satu}\\
& \textitbf{dua} & ‘two’ & \textitbf{dua{\Tilde}dua}\\
& \textitbf{lima} & ‘five’ & \textitbf{lima{\Tilde}lima}\\
Quantifiers & \textitbf{banyak} & ‘many’ & \textitbf{banyak{\Tilde}banyak}\\
& \textitbf{sedikit} & ‘few’ & \textitbf{sedikit{\Tilde}sedikit}\\
& \textitbf{sembarang} & ‘any (kind of)’ & \textitbf{sembarang{\Tilde}sembarang}\\
\lspbottomrule
\end{tabular}

Four of the content words listed in Table  ‎4 .1 involve affixation: \textitbf{bua} ‘fruit’ and reduplicated \textitbf{bua{\Tilde}bua\-ang} (suffix \textitbf{\-ang} ‘\textsc{pat}’), \textitbf{tumpuk} ‘pile’ and reduplicated \textitbf{ber\-tumpuk{\Tilde}tumpuk} (prefix \textscItalBold{ber\-} ‘\textsc{vblz}’), \textitbf{tingkat\-ang} ‘level’ and reduplicated \textitbf{tingkat}\textitbf{{}-}\textitbf{ang{\Tilde}tingkat}\textitbf{{}-}\textitbf{ang} (suffix \textitbf{\-ang} ‘\textsc{pat}’), and \textitbf{ta-lipat} ‘be folded’ and reduplicated \textitbf{ta-lipat{\Tilde}ta-lipat} (prefix \textscItalBold{ter\-} ‘\textsc{acl}’). The four lexeme pairs illustrate that reduplication may precede affixation as with \textitbf{bua} ‘fruit’ or \textitbf{tumpuk} ‘pile’ or may follow affixation as with \textitbf{tingkatang} ‘level’ or \textitbf{talipat} ‘be folded’. These examples also show that reduplication only affects free morphemes while affixes are never reduplicated.
\end{styleBodyaftervbefore}


Reduplication of content words is demonstrated with the three examples in (0) to (0). Reduplication of a noun is illustrated in (0); in this context reduplicated \textitbf{ade} ‘younger sibling’ conveys plurality. The utterance in (0) includes a reduplicated verb; in this context, \textitbf{lari} ‘run’ expresses continuation. And the elicited example in (0) illustrates reduplication of an adverb; in this context prohibitive \textitbf{jangang} ‘\textsc{neg.imp}, don’t’ denotes intensity. The three examples illustrate only three of the different meanings expressed with reduplication. Depending on the context, a reduplicated noun can also signal repetition, to name just one other meaning aspect. Along similar lines, a reduplicated verb can also express aimlessness, among other meanings. This variety of different meanings is discussed in detail in §4.2.
\end{styleBodyvxafter}

\begin{tabular}{lllllllllll}
\lsptoprule
\label{bkm:Ref339283695}
\gll {jadi} {saya,} {saya} {deng} {sa} {pu} {\bluebold{ade{\Tilde}ade}} {tinggal} {di} {ruma}\\ %
& so & \textsc{1sg} & \textsc{1sg} & with & \textsc{1sg} & \textsc{poss} & \textsc{rdp}{\Tilde}ySb & stay & at & house\\
\lspbottomrule
\end{tabular}
\ea
\glt 
Plurality: ‘so I, I and my \bluebold{younger siblings} stayed at the house’ \textstyleExampleSource{[081014-014-NP.0002]}
\z

\begin{tabular}{lllllll}
\lsptoprule
\label{bkm:Ref339283697}
\gll {kitong} {dua} {\bluebold{lari{\Tilde}lari}} {sampe} {di} {Martewar}\\ %
& \textsc{1pl} & two & \textsc{rdp}{\Tilde}run & reach & at & Martewar\\
\lspbottomrule
\end{tabular}
\ea
\glt 
Continuation: ‘the two of us \bluebold{kept running} all the way to Martewar’ \textstyleExampleSource{[080923-010-CvNP.0009]}
\z

\begin{tabular}{llllllll}
\lsptoprule
\label{bkm:Ref339283699}
\gll {…} {tapi} {\bluebold{jangang{\Tilde}jangang}} {hujang} {di} {tenga} {jalang}\\ %
&  & but & \textsc{rdp}{\Tilde}\textsc{neg.imp} & rain & at & middle & street\\
\lspbottomrule
\end{tabular}
\ea
\glt
Intensity: ‘[I want to go to (my) gardens,] but \bluebold{let’s hope} it \bluebold{won’t} rain in the middle of the way’ \textstyleExampleSource{[Elicited BR120813.031]}
\end{styleFreeTranslEngxvpt}

\paragraph[Reduplication of function words]{Reduplication of function words}
\label{bkm:Ref339284928}
Some Papuan Malay functions words can also be reduplicated. Attested are reduplicated personal pronouns, demonstratives, locatives,\footnote{\\
\\
\\
\\
\\
\\
\\
\\
\\
\\
\\
While reduplication of \textitbf{sana} ‘\textsc{l.dist}’ is unattested in the corpus, it does occur, following one consultant.\\
\\
\\
\\
} interrogatives, the causative verb \textitbf{kasi} ‘give’, and the reciprocity marker \textitbf{baku} ‘\textsc{recp}’, as listed in Table  ‎4 .2.


\begin{stylecaption}
\label{bkm:Ref342485297}Table ‎4.\stepcounter{Table}{\theTable}:  Reduplication of function words
\end{stylecaption}

\tablehead{
 Word class & Base & Gloss & \arraybslash Reduplicated item\\
}
\begin{tabular}{llll}
\lsptoprule
Personal pronouns & \textitbf{saya} & ‘\textsc{1sg}’ & \textitbf{saya{\Tilde}saya}\\
& \textitbf{kamu} & ‘\textsc{2pl}’ & \textitbf{kamu{\Tilde}kamu}\\
& \textitbf{dorang} & ‘\textsc{3pl}’ & \textitbf{dorang{\Tilde}dorang}\\
Demonstratives & \textitbf{ini} & ‘\textsc{d.prox}’ & \textitbf{ini{\Tilde}ini}\\
& \textitbf{itu} & ‘\textsc{d.dist}’ & \textitbf{itu{\Tilde}itu}\\
Locatives & \textitbf{sini} & ‘\textsc{l.prox}’ & \textitbf{sini{\Tilde}sini}\\
& \textitbf{situ} & ‘\textsc{l.med}’ & \textitbf{situ{\Tilde}situ}\\
& \textitbf{sana} & ‘\textsc{l.dist}’ & \textitbf{sana{\Tilde}sana}\\
Interrogatives & \textitbf{siapa} & ‘who’ & \textitbf{siapa{\Tilde}siapa}\\
& \textitbf{apa} & ‘what’ & \textitbf{apa{\Tilde}apa}\\
& \textitbf{kapang} & ‘when’ & \textitbf{kapang{\Tilde}kapang}\\
Causative verb & \textitbf{kasi} & ‘give’ & \textitbf{kas{\Tilde}kas}\\
Reciprocity marker & \textitbf{baku} & ‘\textsc{recp}’ & \textitbf{baku{\Tilde}baku}\\
\lspbottomrule
\end{tabular}

Reduplication of three different types of functions words and the different meaning aspects conveyed is illustrated in (0) to (0): personal pronouns in (0), locatives in the elicited example in (0), and interrogatives in (0).


\begin{tabular}{lllll}
\lsptoprule
\label{bkm:Ref339294830}
\gll {\bluebold{kamu{\Tilde}kamu}} {ini} {bangung,} {bangung}\\ %
& \textsc{rdp}{\Tilde}\textsc{2pl} & \textsc{d.prox} & wake.up & wake.up\\
\lspbottomrule
\end{tabular}
\ea
\glt 
Collectivity: ‘\bluebold{you all} here wake-up!, wake-up!’ \textstyleExampleSource{[081115-001a-Cv.0330]}
\z

\begin{tabular}{llllll}
\lsptoprule
\label{bkm:Ref339280594}
\gll {ko} {lari} {suda} {ke} {\bluebold{sana{\Tilde}sana}}\\ %
& \textsc{2sg} & run & already & to & \textsc{rdp}{\Tilde}\textsc{loc.dist}\\
\lspbottomrule
\end{tabular}
\ea
\glt 
Diversity: ‘you run to \bluebold{somewhere} over there’ \textstyleExampleSource{[Elicited BR120813.016]}
\z

\begin{tabular}{llllllll}
\lsptoprule
\label{bkm:Ref339280593}
\gll {…} {sa} {tra} {perna} {lari} {ke} {\bluebold{siapa{\Tilde}siapa}}\\ %
&  & \textsc{1sg} & \textsc{neg} & once & run & to & \textsc{rdp}{\Tilde}who\\
\lspbottomrule
\end{tabular}
\ea
\glt
Intensity: ‘[even (when) my children were already sick,] I’ve never run to \bluebold{anyone} (for black-magic help)’ \textstyleExampleSource{[081006-034-CvEx.0028]}
\end{styleFreeTranslEngxvpt}

\paragraph[Reduplication without corresponding single base]{Reduplication without corresponding single base}
\label{bkm:Ref360287666}\label{bkm:Ref339284927}
Across word classes, some reduplicated forms do not have an unreduplicated single base. Attested are four nouns, three verbs, one quantifier, and one conjunction, as listed in Table  ‎4 .3.


\begin{stylecaption}
\label{bkm:Ref342485347}Table ‎4.\stepcounter{Table}{\theTable}:  Reduplication without corresponding single base
\end{stylecaption}

\tablehead{
 Word class & Base & Reduplicated item & \arraybslash Gloss\\
}
\begin{tabular}{llll}
\lsptoprule
Nouns & \textitbf{*alang} & \textitbf{alang-alang} & ‘cogongrass’\\
& \textitbf{*kura} & \textitbf{kura-kura} & ‘turtle’\\
& \textitbf{*pori} & \textitbf{pori-pori} & ‘pore’\\
& \textitbf{*soa} & \textitbf{soa-soa} & ‘monitor lizard’\\
Verbs & \textitbf{*belit} & \textitbf{belit-belit} & ‘curve’\\
& \textitbf{*gong} & \textitbf{gong-gong} & ‘bark (at)’\\
& \textitbf{*tele} & \textitbf{tele-tele} & ‘talk excessively’\\
Quantifier & \textitbf{*masing} & \textitbf{masing-masing} & ‘each’\\
Conjunction & \textitbf{*gara} & \textitbf{gara-gara} & ‘because’\\
\lspbottomrule
\end{tabular}
\subsection{Partial reduplication}
\label{bkm:Ref438305007}\label{bkm:Ref438304972}
Partial reduplication is rare in Papuan Malay. Generally speaking, this type of reduplication “involves the reiteration of only part of the semantic-syntactic or phonetic-phonological constituent whose meaning is accordingly modified” {\citep[304]{Moravcsik1978}}.



That is, the added material is not a morphemic skeleton as in the case of full reduplication but the reduplicant is a C-V skeleton or a syllabic skeleton which gets prefixed to the base. If the reduplicant is a C-V skeleton, the “entire phonemic melody of the stem is copied over the affixed C-V skeleton and linked to C and V ‘slots’ in the skeleton” {\citep[437]{Marantz1982}} (concerning the principles involved in this linking see {Marantz 1982: 446–447}). A “syllabic skeleton, lacking a phonemic melody and a C-V skeleton, borrows both from the stem to which it attaches” {(1982: 437)}.
\end{styleBodyvafter}


In Papuan Malay, the reduplicant is a closed heavy syllable which is prefixed to the stem from which it borrows the phonemic melody and C-V skeleton, as shown in (0). In (0), for example, the initial closed syllable [\textstyleChCharisSIL{bɐp}] is copied over the reduplicating syllabic skeleton. With vowel-initial stems, the initial VC is copied over the reduplicating syllabic skeleton. This is shown in (0) with the initial VC [\textstyleChCharisSIL{an}] which is copied over the prefixed CVC syllable. These examples also show that the prefixed syllable does not take into account the syllable structure of the base.


\begin{tabular}{llllllllllllll}
\lsptoprule
\label{bkm:Ref368559144}\label{bkm:Ref368559169}
\gll {\label{bkm:Ref368559197}} {b} {a} {p} {a} {} {} {b} {a} {p} {a} {} {}\\ %
&  & \textitbfxivpt{{\textbar}} & \textitbfxivpt{{\textbar}} & \textitbfxivpt{{\textbar}} &  &  &  & \textitbfxivpt{{\textbar}} & \textitbfxivpt{{\textbar}} & \textitbfxivpt{{\textbar}} & \textitbfxivpt{{\textbar}} &  & \\
&  & C & V & C &  &  & + & C & V & C & V &  & \\
&  & \multicolumn{3}{l}{  
%%please move the includegraphics inside the {figure} environment
%%\includegraphics[width=\textwidth]{kluge-img24.jpg}
 } &  &  &  & \multicolumn{2}{l}{  
%%please move the includegraphics inside the {figure} environment
%%\includegraphics[width=\textwidth]{kluge-img25.jpg}
 } & \multicolumn{2}{l}{  
%%please move the includegraphics inside the {figure} environment
%%\includegraphics[width=\textwidth]{kluge-img26.jpg}
 } &  & \\
&  & \multicolumn{3}{l}{ $\sigma $} &  &  &  & \multicolumn{2}{l}{ $\sigma $} & \multicolumn{2}{l}{ $\sigma $} &  & = [\textstyleChCharisSIL{bɐp.ˈba.pa}]\\
&  &  &  &  &  &  &  &  &  &  &  &  & \\
&  & b & a & r & u &  &  & b & a & r & u &  & \\
&  & \textitbfxivpt{{\textbar}} & \textitbfxivpt{{\textbar}} & \textitbfxivpt{{\textbar}} &  &  &  & \textitbfxivpt{{\textbar}} & \textitbfxivpt{{\textbar}} & \textitbfxivpt{{\textbar}} & \textitbfxivpt{{\textbar}} &  & \\
&  & C & V & C &  &  & + & C & V & C & V &  & \\
&  & \multicolumn{3}{l}{  
%%please move the includegraphics inside the {figure} environment
%%\includegraphics[width=\textwidth]{kluge-img27.jpg}
 } &  &  &  & \multicolumn{2}{l}{  
%%please move the includegraphics inside the {figure} environment
%%\includegraphics[width=\textwidth]{kluge-img28.jpg}
 } & \multicolumn{2}{l}{  
%%please move the includegraphics inside the {figure} environment
%%\includegraphics[width=\textwidth]{kluge-img29.jpg}
 } &  & \\
&  & \multicolumn{3}{l}{ $\sigma $} &  &  &  & \multicolumn{2}{l}{ $\sigma $} & \multicolumn{2}{l}{ $\sigma $} &  & = [\textstyleChCharisSIL{bɐr.ˈba.ru}]\\
&  &  &  &  &  &  &  &  &  &  &  &  & \\
& \label{bkm:Ref368559199} &  & a & n & a &  &  &  & a & n & a &  & \\
&  &  & \textitbfxivpt{{\textbar}} & \textitbfxivpt{{\textbar}} &  &  &  &  & \textitbfxivpt{{\textbar}} & \textitbfxivpt{{\textbar}} & \textitbfxivpt{{\textbar}} &  & \\
&  & C & V & C &  &  & + & C & V & C & V &  & \\
&  & \multicolumn{3}{l}{  
%%please move the includegraphics inside the {figure} environment
%%\includegraphics[width=\textwidth]{kluge-img30.jpg}
 } &  &  &  & \multicolumn{2}{l}{  
%%please move the includegraphics inside the {figure} environment
%%\includegraphics[width=\textwidth]{kluge-img31.jpg}
 } & \multicolumn{2}{l}{  
%%please move the includegraphics inside the {figure} environment
%%\includegraphics[width=\textwidth]{kluge-img32.jpg}
 } &  & \\
&  & \multicolumn{3}{l}{ $\sigma $} &  &  &  & \multicolumn{2}{l}{ $\sigma $} & \multicolumn{2}{l}{ $\sigma $} &  & = [\textstyleChCharisSIL{a.ˈna.na}]\\
&  &  &  &  &  &  &  &  &  &  &  &  & \\
&  &  & a & p & a &  &  &  & a & p & a &  & \\
&  &  & \textitbfxivpt{{\textbar}} & \textitbfxivpt{{\textbar}} &  &  &  &  & \textitbfxivpt{{\textbar}} & \textitbfxivpt{{\textbar}} & \textitbfxivpt{{\textbar}} &  & \\
&  & C & V & C &  &  & + & C & V & C & V &  & \\
&  & \multicolumn{3}{l}{  
%%please move the includegraphics inside the {figure} environment
%%\includegraphics[width=\textwidth]{kluge-img33.jpg}
 } &  &  &  & \multicolumn{2}{l}{  
%%please move the includegraphics inside the {figure} environment
%%\includegraphics[width=\textwidth]{kluge-img34.jpg}
 } & \multicolumn{2}{l}{  
%%please move the includegraphics inside the {figure} environment
%%\includegraphics[width=\textwidth]{kluge-img35.jpg}
 } &  & \\
&  & \multicolumn{3}{l}{ $\sigma $} &  &  &  & \multicolumn{2}{l}{ $\sigma $} & \multicolumn{2}{l}{ $\sigma $} &  & = [\textstyleChCharisSIL{a.ˈpa.pa}]\\
\lspbottomrule
\end{tabular}

In Papuan Malay, partial reduplication is only attested for disyllabic lexical roots with penultimate stress. It always involves the partial reduplication of the stressed penultimate syllable of the base, as shown in Table  ‎4 .4. The results are trisyllabic words with penultimate stress. If the base has a CV.CV(C) syllable structure, stress in the reduplicated word remains on the penultimate syllable of the base, as in \textitbf{bapa{\Tilde}bapa} [\textstyleChCharisSIL{bɐp.ˈba.pa}] ‘fathers’. With vowel-initial stems, Papuan Malay copies the initial VC sequence, as in \textitbf{ana{\Tilde}ana} [\textstyleChCharisSIL{a.ˈ}\textstyleChCharisSIL{na.na}] ‘children’. In this case, the reduplicant’s segments do not originate from one and the same syllable of the base. Across languages this phenomenon is rather common. That is, as {Wiltshire and \citet[562]{Marantz1978}} note, partial reduplication can be “oblivious to the prosodic structure of the base from which it copies a melody”.
\end{styleBodyaftervbeforexivpt}


The partially reduplicated forms are alternants of fully reduplicated ones and have the same semantics; [\textstyleChCharisSIL{a.ˈna.na}] ‘children’, for instance, is an alternant of [\textstyleChCharisSIL{ˌa.na.ˈa.na}] ‘children’.


\begin{stylecaption}
\label{bkm:Ref368559107}Table ‎4.\stepcounter{Table}{\theTable}:  Partial reduplication
\end{stylecaption}

\tablehead{
 Base & Gloss & \multicolumn{2}{l}{ Reduplicated item}\\
}
\begin{tabular}{llll}
\lsptoprule
\textitbf{ana} & ‘child’ & \textitbf{ana{\Tilde}ana} & [\textstyleChCharisSIL{a.ˈna.na}]\\
\textitbf{apa} & ‘what’ & \textitbf{apa{\Tilde}apa} & [\textstyleChCharisSIL{a.ˈpa.pa}]\\
\textitbf{bapa} & ‘father’ & \textitbf{bapa{\Tilde}bapa} & [\textstyleChCharisSIL{bɐp.ˈba.pa}]\\
\textitbf{baru} & ‘be new’ & \textitbf{baru{\Tilde}baru} & [\textstyleChCharisSIL{bɐr.ˈba.ru}]\\
\lspbottomrule
\end{tabular}
\subsection{Imitative reduplication}

The third attested type of reduplication is imitative or rhyming reduplication. Cross-linguistically, this reduplication type is also being referred to as “echo construction”; it “involves reduplication with some different phonological material, such as a vowel or consonant change or addition, or morpheme order reversal” {\citep[2]{Rubino2011}}.



Imitative reduplication in Papuan Malay is unproductive and rare; attested are only the three lexemes listed in Table  ‎4 .5. The reduplicated component resembles the base in part but also differs from it, in that imitative reduplication involves a vowel change. For one of the attested lexemes, the bare base is also inexistent: \textitbf{*ngyaung}.
\end{styleBodyvvafter}

\begin{stylecaption}
\label{bkm:Ref358730367}Table ‎4.\stepcounter{Table}{\theTable}:  Imitative reduplication with vowel change
\end{stylecaption}

\tablehead{
 Reduplicated item & Gloss & Base & \arraybslash Gloss\\
}
\begin{tabular}{llll}
\lsptoprule
\textitbf{ngying{\Tilde}ngyaung} & ideophone: cockatoo call & \textitbf{*ngyaung} & {}-{}-{}-\\
\textitbf{tuk{\Tilde}tak} & ideophone: bang! & \textitbf{tak} & ideophone: bang!\\
\textitbf{bola{\Tilde}balik} & ‘move back and forth’ & \textitbf{balik} & ‘return’\\
\lspbottomrule
\end{tabular}
\section{Lexeme interpretation}
\label{bkm:Ref360096571}
In Papuan Malay, as in other languages, reduplication conveys a variety of different meanings, such as plurality and diversity, intensity, or continuation and repetition. Some of these meaning aspects tend to be limited to certain word classes, while others are conveyed by a variety of different word classes.



The meaning aspects of reduplicated Papuan Malay content words are examined in §4.2.1 to §4.2.4, those of reduplicated function words in §4.2.5. The underlying general meaning or gesamtbedeutung of reduplication is explored in §4.2.6.
\end{styleBodyvxvafter}

\subsection{Reduplication of nouns}
\label{bkm:Ref227393103}
Across languages, reduplication of nouns has been found to express a variety of meanings such as “number […], case, distributivity, indefiniteness, reciprocity, size (diminutive or augmentative), and associative qualities” {\citep{Rubino2011}}. In Papuan Malay, the following meaning aspects are attested: collectivity and diversity (§4.2.1.1), repetition (§4.2.1.2), and indefiniteness (§4.2.1.3). Reduplicated nouns can also undergo an interpretational shift and receive a verbal or adverbial reading (§4.2.1.4).
\end{styleBodyxvafter}

\paragraph[Collectivity and diversity]{Collectivity and diversity}
\label{bkm:Ref434690509}
A major function of noun reduplication is to signal plurality, given that in Papuan Malay bare nouns are not marked for number. Instead, speakers express plurality as deemed necessary. Depending on the context, the lexical item \textitbf{ana} ‘child’, for instance, could also be read as ‘children’. One strategy to express plurality overtly is reduplication of nouns. Overall, however, speakers use reduplication only when an unambiguous plural reading is important to them and when the context does not allow such an unambiguous interpretation. (Alternative strategies to indicate plurality are modification with a numeral or quantifier, or with a plural personal pronoun; see §8.2.3 and §6.2.2, respectively.)



Cross-linguistically, three types of plurality have been identified which are encoded by noun reduplication {(Wiltshire and Marantz 1978: 561)}: collectivity, diversity (or variety), and distributivity.\footnote{\\
\\
\\
\\
\\
\\
\\
\\
\\
\\
\\
{Wiltshire and \citet[561]{Marantz1978}} refer to “collectivity” as “simple plurality”.\\
\\
\\
\\
} Of these three types, Papuan Malay uses two, namely collectivity as in (0) and (0), and diversity as in (0) and (0). Another type of plurality is indefiniteness {\citep{Rubino2011}}, which is also found in Papuan Malay, as demonstrated in (0) and (0) in §4.2.1.3 (p. \pageref{bkm:Ref436750017}).
\end{styleBodyvafter}


Reduplication of nouns most often indicates collectivity in the sense of ‘all \textsc{base}’, as shown with \textitbf{ana{\Tilde}ana} ‘children’ in (0) and \textitbf{orang{\Tilde}orang} ‘people’ in (0).
\end{styleBodyvvafter}

\begin{styleExampleTitle}
Reduplicated nouns: Collectivity
\end{styleExampleTitle}

\begin{tabular}{llllllll}
\lsptoprule
\label{bkm:Ref359864170}
\gll {\bluebold{ana{\Tilde}ana}} {su} {pergi} {kerja,} {\bluebold{ana{\Tilde}ana}} {su} {kawing}\\ %
& \textsc{rdp}{\Tilde}child & already & go & work & \textsc{rdp}{\Tilde}child & already & marry.inofficially\\
\lspbottomrule
\end{tabular}
\ea
\glt 
[Complaint of a lonely couple:] ‘\bluebold{all (our) children} already went to work (elsewhere), \bluebold{all (our) children} are already married’ \textstyleExampleSource{[080917-010-CvEx.0071]}
\z

\begin{tabular}{llllll}
\lsptoprule
\label{bkm:Ref359864171}
\gll {e,} {\bluebold{orang{\Tilde}orang}} {itu} {dong} {mara{\Tilde}mara}\\ %
& hey! & \textsc{rdp}{\Tilde}person & \textsc{d.dist} & \textsc{3pl} & \textsc{rdp}{\Tilde}feel.angry(.about)\\
\lspbottomrule
\end{tabular}
\ea
\glt 
‘hey, \bluebold{all those people}, they’ll be really angry (with you)’ \textstyleExampleSource{[080917-008-NP.0053]}
\z


Less often, reduplicated nouns signal diversity such as \textitbf{bua{\Tilde}bua} ‘various fruit (trees)’ in (0), or \textitbf{pohong{\Tilde}pohong} ‘various trees’ in (0).


\begin{styleExampleTitle}
Reduplicated nouns: Diversity
\end{styleExampleTitle}

\begin{tabular}{lllll}
\lsptoprule
\label{bkm:Ref359864592}
\gll {\bluebold{bua{\Tilde}bua}} {di} {sini} {banyak}\\ %
& \textsc{rdp}{\Tilde}fruit & at & \textsc{l.prox} & many\\
\lspbottomrule
\end{tabular}
\ea
\glt 
‘there are a many \bluebold{different kinds of fruit (trees)} here’ (Lit. ‘\bluebold{the various fruit (trees)} here are many’) \textstyleExampleSource{[080922-001a-CvPh.0425]}
\z

\begin{tabular}{lllllllll}
\lsptoprule
\label{bkm:Ref359864593}
\gll {…} {ini} {suda} {tida} {begini} {lagi,} {suda} {ada}\\ %
&  & \textsc{d.prox} & already & \textsc{neg} & like.this & again & already & exist\\
& \multicolumn{8}{l}{\bluebold{pohong{\Tilde}pohong}}\\
& \multicolumn{8}{l}{\textsc{rdp}{\Tilde}tree}\\
\lspbottomrule
\end{tabular}
\ea
\glt
‘[in five years, yes,] this (garden) won’t be same (as) there will be already \bluebold{various trees} (here)’ \textstyleExampleSource{[081029-001-Cv.0007]}
\end{styleFreeTranslEngxvpt}

\paragraph[Repetition]{Repetition}
\label{bkm:Ref360203431}
Reduplication of nouns denoting periods of the day can indicate repetition. This is illustrated with \textitbf{pagi{\Tilde}pagi} ‘every morning’ in (0), and \textitbf{malam{\Tilde}malam} ‘every evening’ in (0). (For alternative readings of reduplicated nouns expressing time divisions, see (0) in §4.2.1.3, p. \pageref{bkm:Ref436750017}, and (0) and (0) in §4.2.1.4, p. \pageref{bkm:Ref436750344}.)
\end{styleBodyxafter}

\begin{tabular}{llll}
\lsptoprule
\label{bkm:Ref436750219}\label{bkm:Ref359865737}
\gll {\bluebold{pagi{\Tilde}pagi}} {biking} {te}\\ %
& \textsc{rdp}{\Tilde}morning & make & tea\\
\lspbottomrule
\end{tabular}
\ea
\glt 
‘\bluebold{every morning} (they) made tea’ \textstyleExampleSource{[081025-009a-Cv.0023]}
\z

\begin{tabular}{lllllll}
\lsptoprule
\label{bkm:Ref360104154}
\gll {ko} {jangang} {ikut{\Tilde}ikut} {orang} {tua} {\bluebold{malam{\Tilde}malam}}\\ %
& \textsc{2sg} & \textsc{neg.imp} & \textsc{rdp}{\Tilde}follow & person & be.old & \textsc{rdp}{\Tilde}night\\
\lspbottomrule
\end{tabular}
\ea
\glt
‘don’t keep hanging out with the grown-ups \bluebold{every evening}’ \textstyleExampleSource{[081013-002-Cv.0005]}
\end{styleFreeTranslEngxvpt}

\paragraph[Indefiniteness]{Indefiniteness}
\label{bkm:Ref360104409}
Depending on the context, reduplicated nouns may signal indefiniteness by referring to an unspecified group member, in the sense of ‘any’ or ‘some’. This is illustrated with \textitbf{om{\Tilde}om} ‘any one of the uncles’ in (0), and \textitbf{malam{\Tilde}malam} ‘at some point in the evening’ in (0). (For alternative interpretations of reduplicated nouns signaling time divisions, see (0) and (0) in §4.2.1.2, p. \pageref{bkm:Ref436750219}, and (0) and (0) in §4.2.1.4, p. \pageref{bkm:Ref436750344}.)
\end{styleBodyxafter}

\begin{tabular}{lllllll}
\lsptoprule
\label{bkm:Ref359864172}
\gll {baru} {titip} {di,} {ini,} {\bluebold{om{\Tilde}om}} {dorang}\\ %
& and.then & deposit & at & \textsc{d.prox} & \textsc{rdp}{\Tilde}uncle & \textsc{3pl}\\
\lspbottomrule
\end{tabular}
\ea
\glt 
‘leave (the letter) with, what’s-his-name, \bluebold{any one of the uncles} and his family’ \textstyleExampleSource{[080922-001a-CvPh.0602]}
\z

\begin{tabular}{lllll}
\lsptoprule
\label{bkm:Ref436750017}\label{bkm:Ref359864173}
\gll {dia} {lewat} {pante} {\bluebold{malam{\Tilde}malam}}\\ %
& \textsc{3sg} & pass.by & coast & \textsc{rdp}{\Tilde}night\\
\lspbottomrule
\end{tabular}
\ea
\glt
‘he drove along the beach \bluebold{at some (point in) the evening}’ \textstyleExampleSource{[081006-020-Cv.0016]}
\end{styleFreeTranslEngxvpt}

\paragraph[Interpretational shift]{Interpretational shift}
\label{bkm:Ref368577236}
Reduplicated nouns can also undergo, what generally-speaking \citet[212]{Booij2007} calls, an “interpretational shift” or “type coercion”. In Papuan Malay, such a shift can result in a stative verbal reading of reduplicated nouns as in (0) to (0), or in an adverbial reading as in (0) to (0), depending on the larger linguistic context.



Interpretational shift resulting in a stative verbal reading of reduplicated nouns usually applies to reduplicated kinship terms, taking the predicate slot in nonverbal clauses. This is illustrated with \textitbf{ana} ‘child’ in (0) and \textitbf{tete} ‘grandfather’ in the elicited example in (0). In this context, the reduplicated nouns receive a stative verbal rather than a nominal reading. That is, referring to specific age groups, they designate pertinent attributes of their base words, as in \textitbf{ana{\Tilde}ana} ‘be quite small’ (literally ‘\textsc{rdp}{\Tilde}child’) in (0), or \textitbf{tete{\Tilde}tete} ‘be quite old’ (literally ‘\textsc{rdp}{\Tilde}grandfather’) in (0). In addition, the corpus contains one example in which a non-kinship term, namely the common noun \textitbf{rawa} ‘swamp’, undergoes a similar interpretational shift, receiving the stative verbal reading in \textitbf{rawa{\Tilde}rawa} ‘be swampy’, as shown in (0).
\end{styleBodyvvafter}

\begin{styleExampleTitle}
Reduplicated nouns: Stative verbal reading
\end{styleExampleTitle}

\begin{tabular}{llllll}
\lsptoprule
\label{bkm:Ref359854153}
\gll {waktu} {itu} {sa} {masi} {\bluebold{ana{\Tilde}ana}}\\ %
& time & \textsc{d.dist} & \textsc{1sg} & still & \textsc{rdp}{\Tilde}child\\
\lspbottomrule
\end{tabular}
\ea
\glt 
‘at that time I was still \bluebold{quite small}’ \textstyleExampleSource{[080922-008-CvNP.0004]}
\z

\begin{tabular}{lllllllllll}
\lsptoprule
\label{bkm:Ref365892143}
\gll {pace} {\multicolumn{3}{l}{ni}} {de} {\multicolumn{2}{l}{su}} {\bluebold{tete{\Tilde}tete}} {tapi} {masi}\\ %
& man & \multicolumn{3}{l}{\textsc{d.prox}} & \textsc{3sg} & \multicolumn{2}{l}{already} & \textsc{rdp}{\Tilde}grandfather & but & still\\
& \multicolumn{2}{l}{maing} & deng & \multicolumn{3}{l}{ana{\Tilde}ana} & \multicolumn{4}{l}{muda}\\
& \multicolumn{2}{l}{play} & with & \multicolumn{3}{l}{\textsc{rdp}{\Tilde}child} & \multicolumn{4}{l}{be.young}\\
\lspbottomrule
\end{tabular}
\ea
\glt 
‘this guy, he’s already \bluebold{quite old} but he still hangs out with the young people’ \textstyleExampleSource{[Elicited BR120813.003]}
\z

\begin{tabular}{lll}
\lsptoprule
\label{bkm:Ref359854151}
\gll {masi} {\bluebold{rawa{\Tilde}rawa}}\\ %
& still & \textsc{rdp}{\Tilde}swamp\\
\lspbottomrule
\end{tabular}
\ea
\glt 
[About a road building project:] ‘(the area is) still \bluebold{swampy}’ \textstyleExampleSource{[081006-033-Cv.0027]}
\z


Interpretational shift can also affect reduplicated location nouns or nouns denoting periods of the day, with the reduplicated nouns receiving an intensified or emphatic adverbial reading. This is illustrated with the location nouns \textitbf{depang} ‘front’ in (0) and \textitbf{samping} ‘side’ in (0), and the temporal nouns \textitbf{pagi} ‘morning’ in (0) and \textitbf{malam} ‘night’ in (0). (For alternative readings of reduplicated nouns designating time divisions, see (0) and (0) in §4.2.1.2, p. \pageref{bkm:Ref436750219}, and (0) in §4.2.1.3, p. \pageref{bkm:Ref436750017}.)


\begin{styleExampleTitle}
Reduplicated nouns: Adverbial reading
\end{styleExampleTitle}

\begin{tabular}{llllllll}
\lsptoprule
\label{bkm:Ref359859962}
\gll {sa} {tunjuk} {\bluebold{depang{\Tilde}depang}} {muka,} {blajar} {untuk} {mandiri}\\ %
& \textsc{1sg} & show & \textsc{rdp}{\Tilde}front & front & study & for & stand.alone\\
\lspbottomrule
\end{tabular}
\ea
\glt 
‘I point \bluebold{right into} (their) faces (and tell them), ‘study to become independent’’ \textstyleExampleSource{[081115-001a-Cv.0054]}
\z

\begin{tabular}{lllllll}
\lsptoprule
\label{bkm:Ref359859963}
\gll {jalang} {di} {\bluebold{samping{\Tilde}samping}} {itu} {pagar} {itu}\\ %
& walk & at & \textsc{rdp}{\Tilde}side & \textsc{d.dist} & fence & \textsc{d.dist}\\
\lspbottomrule
\end{tabular}
\ea
\glt 
‘(he/she) walked \bluebold{right next to}, what’s-its-name, that fence’ \textstyleExampleSource{[081025-006-Cv.0094]}
\z

\begin{tabular}{llllllllll}
\lsptoprule
\label{bkm:Ref436750344}\label{bkm:Ref359854154}
\gll {…} {\bluebold{pagi{\Tilde}pagi}} {jam} {lima} {sa} {su} {masuk} {di} {kamar}\\ %
&  & \textsc{rdp}{\Tilde}morning & hour & five & \textsc{1sg} & already & enter & at & room\\
\lspbottomrule
\end{tabular}
\ea
\glt 
[About disciplining ill-behaved teenagers:] ‘[tonight I’ll still sleep,] (but tomorrow) \bluebold{early in the morning} at five o’clock I will already have gone into (their) room’ \textstyleExampleSource{[081115-001a-Cv.0325]}
\z

\begin{tabular}{lllllllllll}
\lsptoprule
\label{bkm:Ref359854155}
\gll {\multicolumn{3}{l}{\bluebold{malam{\Tilde}malam}}} {Ise} {bawa} {pulang} {dia} {pi} {tidor} {dengang}\\ %
& \multicolumn{3}{l}{\textsc{rdp}{\Tilde}night} & Ise & bring & go.home & \textsc{3sg} & go & sleep & with\\
& de & punya & \multicolumn{8}{l}{mama}\\
& \textsc{3sg} & \textsc{poss} & \multicolumn{8}{l}{mother}\\
\lspbottomrule
\end{tabular}
\ea
\glt
[About a crying child:] ‘\bluebold{late at night} Ise brought (her) home so that she (would) go and sleep with her mother’ \textstyleExampleSource{[081006-025-CvEx.0007]}
\end{styleFreeTranslEngxvpt}

\subsection{Reduplication of verbs}
\label{bkm:Ref374451130}\label{bkm:Ref362004280}
Cross-linguistically, reduplication of verbs tends to encode meaning aspects such as “distribution of an argument, tense, aspect (continued or repeated occurrence; completion; inchoativity), attenuation, intensity, transitivity (valence, object defocusing), or reciprocity” ({Rubino 2011}; see also {Wiltshire and Marantz 1978: 561}). In Papuan Malay, the following meaning aspects are attested: continuation, repetition, and habit (§4.2.2.1), collectivity and diversity (§4.2.2.2), intensity (§4.2.2.3), immediacy (§4.2.2.4), aimlessness (§4.2.2.5), attenuation (4.2.2.6), and imitation (§4.2.2.7). Reduplicated verbs can also undergo interpretational shift, in that they can receive a nominal or adverbial reading (§4.2.2.8).
\end{styleBodyxvafter}

\paragraph[Continuation, repetition, and habit]{Continuation, repetition, and habit}
\label{bkm:Ref360089613}
A major function of verb reduplication is to indicate continuation, repetition, or habit. The function of signaling continuation is demonstrated with a dynamic verb in (0) and a stative verb in the elicited example in (0). The function of signaling repetition of an action is shown in (0).


\begin{styleExampleTitle}
Reduplicated verbs: Continuation and repetition
\end{styleExampleTitle}

\begin{tabular}{lllllll}
\lsptoprule
\label{bkm:Ref360030250}
\gll {…} {ada} {setang} {datang} {\bluebold{ganggu{\Tilde}ganggu}} {kitorang}\\ %
&  & exist & evil.spirit & come & \textsc{rdp}{\Tilde}disturb & \textsc{1pl}\\
\lspbottomrule
\end{tabular}
\ea
\glt 
‘[when (we) sleep at night,] there is an evil spirit (who) comes and \bluebold{continuously bothers} (us)’ \textstyleExampleSource{[081006-022-CvEx.0168]}
\z

\begin{tabular}{llllllll}
\lsptoprule
\label{bkm:Ref360030252}
\gll {sa} {pu} {temang} {de} {\bluebold{sakit{\Tilde}sakit}} {di} {Dok-Dua}\\ %
& \textsc{1sg} & \textsc{poss} & friend & \textsc{3sg} & \textsc{rdp}{\Tilde}be.sick & at & Dok-Dua\\
\lspbottomrule
\end{tabular}
\ea
\glt 
‘my friend is being \bluebold{sick continuously} in the Dok-Dua (hospital)’ \textstyleExampleSource{[Elicited BR120813.036]}
\z

\begin{tabular}{lllllll}
\lsptoprule
\label{bkm:Ref360030254}
\gll {baru} {de} {pi} {\bluebold{bicara{\Tilde}bicara}} {sa} {begini}\\ %
& and.then & \textsc{3sg} & go & \textsc{rdp}{\Tilde}speak & \textsc{1sg} & like.this\\
\lspbottomrule
\end{tabular}
\ea
\glt 
‘but then he went to \bluebold{talk} about me like this \bluebold{again and again}’ \textstyleExampleSource{[081025-009b-Cv.0006]}
\z


As an extension of marking continuation or repetition, reduplicated verbs may also signal habit, as shown in (0).


\begin{styleExampleTitle}
Reduplicated verbs: Habit
\end{styleExampleTitle}

\begin{tabular}{llllllll}
\lsptoprule
\label{bkm:Ref360027781}
\gll {begitu} {de} {besar} {baru} {de} {\bluebold{nakal{\Tilde}nakal}} {begini}\\ %
& like.that & \textsc{3sg} & be.big & and.then & \textsc{3sg} & \textsc{rdp}{\Tilde}be.mischievous & like.this\\
\lspbottomrule
\end{tabular}
\ea
\glt
‘he grew up like that, and now he’s \bluebold{mischievous} like this \bluebold{all the time}’ \textstyleExampleSource{[080917-010-CvEx.0044]}
\end{styleFreeTranslEngxvpt}

\paragraph[Collectivity and diversity]{Collectivity and diversity}
\label{bkm:Ref360089617}
Verb reduplication may also indicate collectivity or diversity of the clausal subject. The function of signaling collectivity is illustrated with the examples in (0) and (0), while the diversity-marking function of reduplicated verbs is shown in the elicited examples in (0) and (0).
\end{styleBodyxafter}

\begin{tabular}{lllllllll}
\lsptoprule
\label{bkm:Ref360031981}
\gll {dong} {taru} {piring{\Tilde}piring} {kaleng} {yang} {\bluebold{piring}} {yang} {\bluebold{bagus{\Tilde}bagus}}\\ %
& \textsc{3pl} & put & \textsc{rdp}{\Tilde}plate & tin.can & \textsc{rel} & plate & \textsc{rel} & \textsc{rdp}{\Tilde}be.good\\
\lspbottomrule
\end{tabular}
\ea
\glt 
[About honoring guests:] ‘they place tin plates (in front of them) that are \bluebold{plates} that are \bluebold{all good}’ \textstyleExampleSource{[081014-010-CvEx.0015]}
\z

\begin{tabular}{llllllll}
\lsptoprule
\label{bkm:Ref360031982}
\gll {\bluebold{pisang}} {\bluebold{Sorong}} {sana} {tu,} {iii,} {\bluebold{besar{\Tilde}besar}} {manis}\\ %
& banana & Sorong & \textsc{l.dist} & \textsc{d.dist} & oh! & \textsc{rdp}{\Tilde}be.big & sweet\\
\lspbottomrule
\end{tabular}
\ea
\glt 
‘those \bluebold{bananas (from) Sorong} over there, oooh, (they) are \bluebold{all big} (and) sweet’ \textstyleExampleSource{[081011-003-Cv.0017]}
\z

\begin{tabular}{lllll}
\lsptoprule
\label{bkm:Ref360031983}
\gll {ko} {pu} {\bluebold{kwe}} {\bluebold{kras{\Tilde}kras}}\\ %
& \textsc{2sg} & \textsc{poss} & cake & \textsc{rdp}{\Tilde}be.harsh\\
\lspbottomrule
\end{tabular}
\ea
\glt 
‘your \bluebold{various cakes} are \bluebold{hard}’ \textstyleExampleSource{[Elicited BR120813.034]}
\z

\begin{tabular}{lllllll}
\lsptoprule
\label{bkm:Ref360031985}
\gll {\bluebold{mobil}} {di} {jalang} {\bluebold{rusak{\Tilde}rusak}} {karna} {banjir}\\ %
& car & at & street & \textsc{rdp}{\Tilde}be.damaged & because & flooding\\
\lspbottomrule
\end{tabular}
\ea
\glt
‘the \bluebold{various cars} in the street were \bluebold{broken} because of the flooding’ \textstyleExampleSource{[Elicited BR120813.035]}
\end{styleFreeTranslEngxvpt}

\paragraph[Intensity]{Intensity}
\label{bkm:Ref372540011}
Also, quite commonly reduplicated verbs signal intensity. In such cases, reduplicated dynamic verbs receive the reading ‘\textsc{base} intensely’, as in (0) and (0), while reduplication of stative verbs translates with ‘very \textsc{base}’, as in (0) and (0).


\begin{styleExampleTitle}
Reduplicated verbs: Intensity
\end{styleExampleTitle}

\begin{tabular}{llllll}
\lsptoprule
\label{bkm:Ref360026851}
\gll {baru} {dia} {tertawa,} {de} {\bluebold{tertawa{\Tilde}tertawa}}\\ %
& and.then & \textsc{3sg} & laugh & \textsc{3sg} & \textsc{rdp}{\Tilde}laugh\\
\lspbottomrule
\end{tabular}
\ea
\glt 
‘but then he laughed, he \bluebold{laughed intensely}’ \textstyleExampleSource{[080916-001-CvNP.0004]}
\z

\begin{tabular}{llll}
\lsptoprule
\label{bkm:Ref360026852}
\gll {orang} {\bluebold{bertriak{\Tilde}triak}} {tu}\\ %
& person & \textsc{rdp}{\Tilde}scream & \textsc{d.dist}\\
\lspbottomrule
\end{tabular}
\ea
\glt 
‘the people were really \bluebold{screaming} \bluebold{intensely}’ \textstyleExampleSource{[081006-022-CvEx.0007]}
\z

\begin{tabular}{lllllll}
\lsptoprule
\label{bkm:Ref360026853}
\gll {sa} {jalang} {sampe} {sa} {su} {\bluebold{swak{\Tilde}swak}}\\ %
& \textsc{1sg} & walk & until & \textsc{1sg} & already & \textsc{rdp}{\Tilde}be.exhausted\\
\lspbottomrule
\end{tabular}
\ea
\glt 
‘I walked until I was already \bluebold{very exhausted}’ \textstyleExampleSource{[081025-008-Cv.0038]}
\z

\begin{tabular}{lllll}
\lsptoprule
\label{bkm:Ref360026854}
\gll {…} {dong} {tu} {\bluebold{pintar{\Tilde}pintar}}\\ %
&  & \textsc{3pl} & \textsc{d.dist} & \textsc{rdp}{\Tilde}be.clever\\
\lspbottomrule
\end{tabular}
\ea
\glt 
‘they (\textsc{emph}) are \bluebold{very clever}’ \textstyleExampleSource{[081109-001-Cv.0117]}
\z


When reduplicated verbs are negated with \textitbf{tra} ‘\textsc{neg}’ or \textitbf{jangang} ‘don’t’, they express an intensified negative in the sense of ‘not \textsc{base} at all’, as shown in (0) and (0).


\begin{styleExampleTitle}
Negation of reduplicated verbs
\end{styleExampleTitle}

\begin{tabular}{llllll}
\lsptoprule
\label{bkm:Ref359953404}
\gll {sa} {\bluebold{tra}} {\bluebold{takut{\Tilde}takut}} {siapa} {pun}\\ %
& \textsc{1sg} & \textsc{neg} & \textsc{rdp}{\Tilde}feel.afraid(.of) & who & even\\
\lspbottomrule
\end{tabular}
\ea
\glt 
‘I’m \bluebold{not afraid at all} of anybody’ \textstyleExampleSource{[081006-034-CvEx.0026]}
\z

\begin{tabular}{llllllll}
\lsptoprule
\label{bkm:Ref359953405}
\gll {\bluebold{jangang}} {\bluebold{bli{\Tilde}bli}} {di} {sini,} {ini} {su} {malam}\\ %
& \textsc{neg.imp} & \textsc{rdp}{\Tilde}buy & at & \textsc{l.prox} & \textsc{d.prox} & already & night\\
\lspbottomrule
\end{tabular}
\ea
\glt
‘(you) \bluebold{shouldn’t buy} (your sweets at the kiosk) here \bluebold{at all} (because) it is already night’ \textstyleExampleSource{[080917-008-NP.0061]}
\end{styleFreeTranslEngxvpt}

\paragraph[Immediacy]{Immediacy}
\label{bkm:Ref360089619}
Reduplicated verbs can indicate immediacy in the sense of ‘as soon as \textsc{base}’. This is illustrated with the reduplicated dynamic verbs in the elicited examples in (0) and (0).
\end{styleBodyxafter}

\begin{tabular}{lllllll}
\lsptoprule
\label{bkm:Ref359950154}
\gll {\bluebold{pulang{\Tilde}pulang}} {dari} {kantor} {pace} {de} {tidor}\\ %
& \textsc{rdp}{\Tilde}go.home & from & office & man & \textsc{3sg} & sleep\\
\lspbottomrule
\end{tabular}
\ea
\glt 
‘\bluebold{as soon as} (he) \bluebold{came home} from the office, the man slept’ \textstyleExampleSource{[Elicited BR120813.007]}
\z

\begin{tabular}{lllllll}
\lsptoprule
\label{bkm:Ref359950156}
\gll {mace} {ni} {\bluebold{datang{\Tilde}datang}} {trus} {de} {makang}\\ %
& woman & \textsc{d.prox} & \textsc{rdp}{\Tilde}come & next & \textsc{3sg} & eat\\
\lspbottomrule
\end{tabular}
\ea
\glt
‘\bluebold{as soon as} this woman \bluebold{arrived}, she ate’ \textstyleExampleSource{[Elicited BR120813.008]}
\end{styleFreeTranslEngxvpt}

\paragraph[Aimlessness]{Aimlessness}
\label{bkm:Ref372539784}
Quite often, reduplication adds the connotation of aimlessness or casualness. That is, the reduplicated verb may signal that an activity is done repeatedly without a specific goal, as in (0) and (0).
\end{styleBodyxafter}

\begin{tabular}{lllllllll}
\lsptoprule
\label{bkm:Ref360276479}
\gll {sa} {itu} {sa} {pegang} {sagu} {sa} {makang} {\bluebold{jalang{\Tilde}jalang}}\\ %
& \textsc{1sg} & \textsc{d.dist} & \textsc{1sg} & hold & sago & \textsc{1sg} & eat & \textsc{rdp}{\Tilde}walk\\
\lspbottomrule
\end{tabular}
\ea
\glt 
‘as for me, I was holding (some) sago, I ate (it) while \bluebold{strolling around}’ \textstyleExampleSource{[081025-009a-Cv.0073]}
\z

\begin{tabular}{llllll}
\lsptoprule
\label{bkm:Ref360276481}
\gll {malam} {kitong} {\bluebold{duduk{\Tilde}duduk}} {kitong} {\bluebold{menyanyi{\Tilde}menyanyi}}\\ %
& night & \textsc{1pl} & \textsc{rdp}{\Tilde}sit & \textsc{1pl} & \textsc{rdp}{\Tilde}sing\\
\lspbottomrule
\end{tabular}
\ea
\glt
‘in the evening we were \bluebold{sitting around}, we were \bluebold{singing casually}’ \textstyleExampleSource{[081025-009a-Cv.0001]}
\end{styleFreeTranslEngxvpt}

\paragraph[Attenuation]{Attenuation}
\label{bkm:Ref438306105}
Depending on the context, reduplicated stative verbs may signal attenuation in the sense of ‘rather \textsc{base}’, as demonstrated in (0) and (0).
\end{styleBodyxafter}

\begin{tabular}{llllllllllll}
\lsptoprule
\label{bkm:Ref359951694}
\gll {…} {\multicolumn{2}{l}{biking}} {\multicolumn{2}{l}{macang}} {kam} {\multicolumn{2}{l}{pu}} {Jayapura} {pu} {sayur}\\ %
&  & \multicolumn{2}{l}{make} & \multicolumn{2}{l}{variety} & \textsc{2pl} & \multicolumn{2}{l}{\textsc{poss}} & Jayapura & \textsc{poss} & vegetable\\
& \multicolumn{2}{l}{gnemo} & \multicolumn{2}{l}{yang} & \multicolumn{3}{l}{\bluebold{pahit{\Tilde}pahit}} & \multicolumn{4}{l}{itu}\\
& \multicolumn{2}{l}{melinjo} & \multicolumn{2}{l}{\textsc{rel}} & \multicolumn{3}{l}{\textsc{rdp}{\Tilde}be.bitter} & \multicolumn{4}{l}{\textsc{d.dist}}\\
\lspbottomrule
\end{tabular}
\ea
\glt 
‘[then she asked, ‘you don’t fear the bitter (taste of melinjos)?, then mama Pawla said,] ‘do you think this (melinjo) is like your Jayapura melinjo vegetable which is \bluebold{somewhat bitter}?’’ \textstyleExampleSource{[080923-004-Cv.0016]}
\z

\begin{tabular}{lllllllm{-9.4015896E-4in}llll}
\lsptoprule
\label{bkm:Ref359951696}
\gll {badang} {\multicolumn{3}{l}{\bluebold{kurus{\Tilde}kurus},}} {rambut} {\multicolumn{2}{l}{ini}} {\multicolumn{2}{l}{tebal,}} {de} {pu}\\ %
& body & \multicolumn{3}{l}{\textsc{rdp}{\Tilde}be.thin} & hair & \multicolumn{2}{l}{\textsc{d.prox}} & \multicolumn{2}{l}{be.thick} & \textsc{3sg} & \textsc{poss}\\
& \multicolumn{2}{l}{kuku} & ini & \multicolumn{3}{l}{\bluebold{panjang{\Tilde}panjang},} & \multicolumn{2}{l}{kaki} & \multicolumn{3}{l}{\bluebold{kurus{\Tilde}kurus}}\\
& \multicolumn{2}{l}{digit.nail} & \textsc{d.prox} & \multicolumn{3}{l}{\textsc{rdp}{\Tilde}be.long} & \multicolumn{2}{l}{foot} & \multicolumn{3}{l}{\textsc{rdp}{\Tilde}be.thin}\\
\lspbottomrule
\end{tabular}
\ea
\glt
‘(his) body was \bluebold{somewhat thin}, (his) hair was thick, his fingernails were \bluebold{rather long}, (and his) legs were \bluebold{rather thin}’ \textstyleExampleSource{[081006-035-CvEx.0077]}
\end{styleFreeTranslEngxvpt}

\paragraph[Imitation]{Imitation}
\label{bkm:Ref372564434}
Reduplicated verbs may also mark imitation in the sense of ‘something is an imitation of X’ or ‘something is similar to X’. This is illustrated with the dynamic verbs in (0) and (0), and the stative verbs in the elicited examples in (0) and (0).
\end{styleBodyxafter}

\begin{tabular}{llllllllll}
\lsptoprule
\label{bkm:Ref359942058}
\gll {sa} {tendang} {dia} {di} {kaki} {sampe} {de} {\bluebold{lari{\Tilde}lari}} {\bluebold{babi}}\\ %
& \textsc{1sg} & kick & \textsc{3sg} & at & leg & until & \textsc{3sg} & \textsc{rdp}{\Tilde}run & pig\\
\lspbottomrule
\end{tabular}
\ea
\glt 
‘I kicked him against (his) lower leg with the result that he \bluebold{staggered}’ (Lit. ‘he \bluebold{ran{\Tilde}ran} (like) \bluebold{a pig} (which has been shot)’) \textstyleExampleSource{[Elicited BR120813.004]}
\z

\begin{tabular}{llll}
\lsptoprule
\label{bkm:Ref359942060}
\gll {dia} {\bluebold{mati{\Tilde}mati}} {\bluebold{ayam}}\\ %
& \textsc{3sg} & \textsc{rdp}{\Tilde}die & chicken\\
\lspbottomrule
\end{tabular}
\ea
\glt 
‘he had an \bluebold{epileptic seizure}’ (Lit. ‘he \bluebold{died{\Tilde}died} (like) \bluebold{a chicken}’; that is, he was shaking like a chicken with its head cut off) \textstyleExampleSource{[Elicited BR120813.006]}
\z

\begin{tabular}{lllllllllll}
\lsptoprule
\label{bkm:Ref359942061}
\gll {pace} {\multicolumn{2}{l}{ni}} {\multicolumn{2}{l}{de}} {su} {\bluebold{tua{\Tilde}tua}} {\bluebold{kladi}} {tapi} {suka}\\ %
& man & \multicolumn{2}{l}{\textsc{d.prox}} & \multicolumn{2}{l}{\textsc{3sg}} & already & \textsc{rdp}{\Tilde}be.old & taro.root & but & enjoy\\
& \multicolumn{2}{l}{cari} & \multicolumn{2}{l}{prempuang} & \multicolumn{6}{l}{muda}\\
& \multicolumn{2}{l}{search} & \multicolumn{2}{l}{woman} & \multicolumn{6}{l}{be.young}\\
\lspbottomrule
\end{tabular}
\ea
\glt 
‘this guy, he’s already \bluebold{very old} but (he) likes to have young women’ (Lit. ‘he’s \bluebold{old{\Tilde}old} (like) \bluebold{a taro root})’ \textstyleExampleSource{[Elicited BR120813.038]}
\z

\begin{tabular}{lllllllll}
\lsptoprule
\label{bkm:Ref359942795}
\gll {prempuang} {itu} {de} {pu} {kulit} {\bluebold{hitam{\Tilde}hitam}} {\bluebold{panta}} {\bluebold{blanga}}\\ %
& woman & \textsc{d.dist} & \textsc{3sg} & \textsc{poss} & skin & \textsc{rdp}{\Tilde}be.black & buttock & cooking.pot\\
\lspbottomrule
\end{tabular}
\ea
\glt
‘that woman, her skin is \bluebold{black} (like) the bottom of a frying pan’ (Lit. ‘her skin is \bluebold{black{\Tilde}black} (like) \bluebold{the bottom} …’) \textstyleExampleSource{[Elicited BR120813.046]}
\end{styleFreeTranslEngxvpt}

\paragraph[Interpretational shift]{Interpretational shift}
\label{bkm:Ref368577283}
Reduplicated verbs can also undergo an interpretational shift. Such a shift can result in a nominal reading of reduplicated verbs, as in the elicited examples in (0) and (0), or an adverbial reading, as in (0) to (0).



Reduplicated verbs with a nominal reading typically denote the instrument of the action specified by the verbal base, such as \textitbf{garo{\Tilde}garo} ‘rake’ (literally ‘\textsc{rdp}{\Tilde}scratch’) in (0) or \textitbf{gait{\Tilde}gait} ‘pole’ (literally ‘\textsc{rdp}{\Tilde}hook’) in (0).
\end{styleBodyvvafter}

\begin{styleExampleTitle}
Reduplicated verbs: Nominal reading
\end{styleExampleTitle}

\begin{tabular}{llllllll}
\lsptoprule
\label{bkm:Ref359941328}
\gll {tadi} {de} {pake} {\bluebold{garo{\Tilde}garo}} {buat} {garo} {rumput}\\ %
& earlier & \textsc{3sg} & use & \textsc{rdp}{\Tilde}scratch & for & scratch & grass\\
\lspbottomrule
\end{tabular}
\ea
\glt 
‘earlier he took a \bluebold{rake} to rake the grass’ \textstyleExampleSource{[Elicited BR120813.010]}
\z

\begin{tabular}{llllll}
\lsptoprule
\label{bkm:Ref359941330}
\gll {sa} {gait} {mangga} {deng} {\bluebold{gait{\Tilde}gait}}\\ %
& \textsc{1sg} & hook & mango & with & \textsc{rdp}{\Tilde}hook\\
\lspbottomrule
\end{tabular}
\ea
\glt 
‘I plucked mangoes with a \bluebold{pole}’ \textstyleExampleSource{[Elicited BR120813.033]}
\z


Reduplicated verbs can also receive an adverbial reading, as in (0) to (0). Certain reduplicated dynamic verbs may take on the function as modal adverbs, such as \textitbf{taw{\Tilde}taw} ‘suddenly’ (literally ‘\textsc{rdp}{\Tilde}know’) in (0). Some reduplicated stative verbs are used as temporal adverbs such as \textitbf{lama{\Tilde}lama} ‘gradually’ (literally ‘\textsc{rdp}{\Tilde}be.long (of.duration)’) in (0), while others are used as manner adverbs, such as \textitbf{cepat{\Tilde}cepat} ‘quickly’ (literally ‘\textsc{rdp}{\Tilde}be.fast’) in (0).


\begin{styleExampleTitle}
Reduplicated verbs: Adverbial reading
\end{styleExampleTitle}

\begin{tabular}{llllll}
\lsptoprule
\label{bkm:Ref360086611}
\gll {\bluebold{taw{\Tilde}taw}} {orang} {itu} {tida} {keliatang}\\ %
& \textsc{rdp}{\Tilde}think & person & \textsc{d.dist} & \textsc{neg} & be.visible\\
\lspbottomrule
\end{tabular}
\ea
\glt 
‘\bluebold{suddenly}, that person wasn’t visible (any longer)’ \textstyleExampleSource{[080922-002-Cv.0123]}
\z

\begin{tabular}{lllllll}
\lsptoprule
\label{bkm:Ref360088408}
\gll {\bluebold{lama{\Tilde}lama}} {de} {padat} {itu} {macang} {aspal}\\ %
& \textsc{rdp}{\Tilde}be.long & \textsc{3sg} & be.solid & \textsc{d.dist} & variety & asphalt\\
\lspbottomrule
\end{tabular}
\ea
\glt 
‘\bluebold{gradually}, it (the lime stone) becomes solid like asphalt’ \textstyleExampleSource{[081011-001-Cv.0304]}
\z

\begin{tabular}{lllllllll}
\lsptoprule
\label{bkm:Ref360088411}
\gll {yo,} {pak} {Hendrik} {ini} {de} {bilang,} {mandi} {\bluebold{cepat{\Tilde}cepat}}\\ %
& yes & father & Hendrik & \textsc{d.prox} & \textsc{3sg} & say & bathe & \textsc{rdp}{\Tilde}be.fast\\
\lspbottomrule
\end{tabular}
\ea
\glt
‘yes, Mr. Hendrik here, he said, ‘bathe \bluebold{quickly}’’ \textstyleExampleSource{[080917-008-NP.0133]}
\end{styleFreeTranslEngxvpt}

\subsection{Reduplication of adverbs}

Reduplication of adverbs typically signals intensity, similar to the reduplication of verbs, discussed in §4.2.2 (concerning the similarities between adverbs and verbs, see also §5.4). This is illustrated with grading adverb \textitbf{paling} ‘most’, the temporal adverb \textitbf{skarang} ‘now’, and the frequency adverb \textitbf{sring} ‘often’ in the three elicited examples in (0) to (0).
\end{styleBodyxafter}

\begin{tabular}{llllllllll}
\lsptoprule
\label{bkm:Ref359680610}
\gll {de} {bilang} {de} {mo} {kerja} {tapi} {\bluebold{paling{\Tilde}paling}} {de} {tidor}\\ %
& \textsc{3sg} & say & \textsc{3sg} & want & work & but & \textsc{rdp}{\Tilde}most & \textsc{3sg} & sleep\\
\lspbottomrule
\end{tabular}
\ea
\glt 
‘he says, he wants to work but \bluebold{most likely} he’ll sleep’ \textstyleExampleSource{[Elicited BR120813.015]}
\z

\begin{tabular}{llllll}
\lsptoprule
(\stepcounter{}{\the}) & \bluebold{skarang{\Tilde}skarang} & de & ada & di & polisi\\
& \textsc{rdp}{\Tilde}now & \textsc{3sg} & exist & at & police\\
\lspbottomrule
\end{tabular}
\ea
\glt 
‘\bluebold{right now} he/she is at the police (station)’ \textstyleExampleSource{[Elicited BR131231.002]}
\z

\begin{tabular}{lllllll}
\lsptoprule
\label{bkm:Ref359673859}
\gll {sa} {pu} {kaka} {\bluebold{sring{\Tilde}sring}} {ke} {Jayapura}\\ %
& \textsc{1sg} & \textsc{poss} & oSb & \textsc{rdp}{\Tilde}often & to & Jayapura\\
\lspbottomrule
\end{tabular}
\ea
\glt
‘my older sibling (travels) to Jayapura \bluebold{very often}’ \textstyleExampleSource{[Elicited BR131231.001]}
\end{styleFreeTranslEngxvpt}

\subsection{Reduplication of numerals and quantifiers}
\label{bkm:Ref374454714}\label{bkm:Ref374454591}\label{bkm:Ref227393111}
\textstyleti{Across languages, reduplication of numerals “has been found to express various categories including collectives, distributives, multiplicatives, and limitatives” }{\citep{Rubino2011}}\textstyleti{. In Papuan Malay, }reduplicated numerals typically express collectivity or distributiveness, while quantifiers signal distributiveness.



Reduplicated numerals have two meaning aspects. They may express collectivity in the sense of ‘all \textsc{base}’ as in (0) and the elicited example in (0), or signal distributiveness in the sense of ‘\textsc{base} by \textsc{base}’ as in (0) and (0).
\end{styleBodyvvafter}

\begin{styleExampleTitle}
Reduplication of numerals: Collectivity and distributiveness
\end{styleExampleTitle}

\begin{tabular}{lllll}
\lsptoprule
\label{bkm:Ref360091962}
\gll {yo,} {kas} {tinggal} {\bluebold{dua{\Tilde}dua}}\\ %
& yes & give & stay & \textsc{rdp}{\Tilde}two\\
\lspbottomrule
\end{tabular}
\ea
\glt 
‘yes, let \bluebold{both of them} stay’ \textstyleExampleSource{[080919-006-CvNP.0018]}
\z

\begin{tabular}{llllllllll}
\lsptoprule
\label{bkm:Ref360091963}
\gll {…} {karna} {pesta} {tu} {de} {su} {kasi} {mati} {\bluebold{tiga{\Tilde}tiga}}\\ %
&  & because & party & \textsc{d.dist} & \textsc{3sg} & already & give & die & \textsc{rdp}{\Tilde}three\\
\lspbottomrule
\end{tabular}
\ea
\glt 
‘[he/she has three pigs (but)] because of that festivity he/she already killed \bluebold{all three of them}’ \textstyleExampleSource{[Elicited BR120813.043]}
\z

\begin{tabular}{lllllll}
\lsptoprule
\label{bkm:Ref360091964}
\gll {…} {jadi} {lega} {ada} {lepas{\Tilde}lepas} {\bluebold{satu{\Tilde}satu}}\\ %
&  & so & be.relieved & exist & \textsc{rdp}{\Tilde}free & \textsc{rdp}{\Tilde}one\\
\lspbottomrule
\end{tabular}
\ea
\glt 
‘[fortunately, (the people) over there have already received Jesus,] so (you can feel) relieved, they were freed \bluebold{one-by-one}’ \textstyleExampleSource{[081025-007-Cv.0017]}
\z

\begin{tabular}{lllll}
\lsptoprule
\label{bkm:Ref360091965}
\gll {sa} {minum} {\bluebold{lima{\Tilde}lima}} {mangkok}\\ %
& \textsc{1sg} & drink & \textsc{rdp}{\Tilde}five & cup\\
\lspbottomrule
\end{tabular}
\ea
\glt 
[About the lack of water during a retreat:] ‘I drank \bluebold{five} cups (\bluebold{every morning})’ (Lit. ‘\bluebold{five-by-five} cups’) \textstyleExampleSource{[081025-009a-Cv.0070]}
\z


Reduplicated quantifiers signal distributiveness, as in (0) and (0).


\begin{styleExampleTitle}
Reduplication of quantifiers: Distributiveness
\end{styleExampleTitle}

\begin{tabular}{lllllllll}
\lsptoprule
\label{bkm:Ref363111436}
\gll {…} {kariawang} {dong} {\bluebold{banyak{\Tilde}banyak}} {dong} {baru} {turung} {ini}\\ %
&  & employee & \textsc{3pl} & \textsc{rdp}{\Tilde}many & \textsc{3pl} & recently & descend & \textsc{d.prox}\\
\lspbottomrule
\end{tabular}
\ea
\glt 
[Waiting for other boat passengers:] ‘the employees came recently down(stream) \bluebold{in groups of numerous people}’ (Lit. ‘\bluebold{many-by-many}’) \textstyleExampleSource{[080922-001a-CvPh.0812]}
\z

\begin{tabular}{lllllllll}
\lsptoprule
\label{bkm:Ref363111437}
\gll {dong} {blum} {isi,} {selaing} {dong} {isi} {\bluebold{sedikit{\Tilde}sedikit}} {to?}\\ %
& \textsc{3pl} & not.yet & fill & besides & \textsc{3pl} & fill & \textsc{rdp}{\Tilde}few & right?\\
\lspbottomrule
\end{tabular}
\ea
\glt
[About how best to distribute food during a retreat:] ‘they haven’t yet filled (their plates), moreover \bluebold{each one of them} (should) fill (their plates with) \bluebold{little} (food), right?’ (Lit. ‘\bluebold{little by little}’) \textstyleExampleSource{[081025-009a-Cv.0081]}
\end{styleFreeTranslEngxvpt}

\subsection{Reduplication of function words}
\label{bkm:Ref227393114}
Reduplication of function words occurs considerably less often than that of content words. This section describes reduplication of the following function words: personal pronouns (§4.2.5.1), demonstratives and locatives (§4.2.5.2), interrogatives (§4.2.5.3), and the causative verb \textitbf{kasi} ‘give’ and the reciprocity marker \textitbf{baku} ‘\textsc{recp}’ (§4.2.5.4).
\end{styleBodyxvafter}

\paragraph[Personal pronouns]{Personal pronouns}
\label{bkm:Ref359675945}
Personal pronouns can be reduplicated when use pronominally (for details on their different uses, see §5.5 and §6.1). Reduplicated personal pronouns have three meaning aspects. Depending on the context, they signal collectivity, disparagement, or imitation,\footnote{\\
\\
\\
\\
\\
\\
\\
\\
\\
\\
\\
As mentioned in §4.2.2.7, the term \textstyleChItalic{imitation} includes meanings such as ‘something is an imitation of X’ or ‘something is similar to X’.\\
\\
\\
\\
} as (0) and in the elicited examples in (0) and (0), respectively.
\end{styleBodyxafter}

\begin{tabular}{lllllll}
\lsptoprule
\label{bkm:Ref359674869}
\gll {\bluebold{kamu{\Tilde}kamu}} {ini} {bangung} {bangung} {bangung} {bangung}\\ %
& \textsc{rdp}{\Tilde}\textsc{2pl} & \textsc{d.prox} & wake.up & wake.up & wake.up & wake.up\\
\lspbottomrule
\end{tabular}
\ea
\glt 
‘\bluebold{all of you} here wake-up, wake-up, wake-up, wake-up’ \textstyleExampleSource{[081115-001a-Cv.0329]}
\z

\begin{tabular}{llllllll}
\lsptoprule
\label{bkm:Ref359674700}
\gll {knapa} {\bluebold{saya{\Tilde}saya}} {saja} {yang} {bapa} {kasi} {tugas}\\ %
& why & \textsc{rdp}{\Tilde}\textsc{1sg} & just & \textsc{rel} & father & give & duty\\
\lspbottomrule
\end{tabular}
\ea
\glt 
‘why is it (always) \bluebold{poor me} whom father gives chores’ \textstyleExampleSource{[Elicited BR120813.025]}
\z

\begin{tabular}{lllllll}
\lsptoprule
\label{bkm:Ref359675204}
\gll {\bluebold{dorang{\Tilde}dorang}} {tra} {perna} {kasi} {bersi} {halamang}\\ %
& \textsc{rdp}{\Tilde}\textsc{3pl} & \textsc{neg} & ever & give & be.clean & yard\\
\lspbottomrule
\end{tabular}
\ea
\glt
‘\bluebold{people like them} never clean (their) yard’ \textstyleExampleSource{[Elicited BR120813.024]}
\end{styleFreeTranslEngxvpt}

\paragraph[Demonstratives and locatives]{Demonstratives and locatives}
\label{bkm:Ref359675946}
Demonstratives and locatives can also be reduplicated when used pronominally (for details on their different uses, see §5.6 and §5.7, respectively). Reduplicated demonstratives express diversity as in (0) and (0).\footnote{\\
\\
\\
\\
\\
\\
\\
\\
\\
\\
\\
Demonstrative sequences such as \textitbf{itu tu} ‘\textsc{d.dist} \textsc{d.dist}’ also convey intensity or emphasis, as discussed in detail in§7.1.2.3. Given its phonological properties, however, juxtaposed \textitbf{itu tu} ‘\textsc{d.dist} \textsc{d.dist}’ is not taken as an instance of partial reduplication. As discussed in §4.1.2, partial reduplication of the stem \textitbf{itu} ‘\textsc{d.dist}’ should result in the reduplicated form \textitbf{it{\Tilde}itu} ‘\textsc{d.dist}{\Tilde}\textsc{d.dist}’. Therefore, \textitbf{itu tu} ‘\textsc{d.dist} \textsc{d.dist}’ is taken as an instance of demonstrative stacking (see §5.6.4).\\
\\
\\
\\
} Depending on the context, reduplicated locatives may signal diversity as in (0), or emphasize the core meaning of the respective locative, as in (0).
\end{styleBodyxafter}

\begin{tabular}{llllll}
\lsptoprule
\label{bkm:Ref359671772}
\gll {setela} {itu} {nanti} {buat} {\bluebold{ini{\Tilde}ini}}\\ %
& after & \textsc{d.dist} & very.soon & make & \textsc{rdp}{\Tilde}\textsc{d.prox}\\
\lspbottomrule
\end{tabular}
\ea
\glt 
‘soon after that (they) did \bluebold{these various} (things)’ \textstyleExampleSource{[080923-013-CvEx.0030]}
\z

\begin{tabular}{llllllll}
\lsptoprule
\label{bkm:Ref360113564}
\gll {…} {yang} {laing} {\bluebold{itu{\Tilde}itu}} {honorer} {smua} {itu}\\ %
&  & \textsc{rel} & be.different & \textsc{rdp}{\Tilde}\textsc{d.dist} & be.honorary & all & \textsc{d.dist}\\
\lspbottomrule
\end{tabular}
\ea
\glt 
‘[there are no school teachers, only him and Markus,] (as for) the others, \bluebold{those various} (teachers) are all honorary (teachers)’ \textstyleExampleSource{[081011-024-Cv.0054]}
\z

\begin{tabular}{llllllll}
\lsptoprule
\label{bkm:Ref359671774}
\gll {jadi} {de} {bapa} {ke} {Jayapura} {tinggal} {\bluebold{situ{\Tilde}situ}}\\ %
& so & \textsc{3sg} & father & to & Jayapura & stay & \textsc{rdp}{\Tilde}\textsc{l.med}\\
\lspbottomrule
\end{tabular}
\ea
\glt 
‘so her father (went) to Jayapura and lived \bluebold{there in a number of different places}’ \textstyleExampleSource{[081011-023-Cv.0163]}
\z

\begin{tabular}{llllllll}
\lsptoprule
\label{bkm:Ref359671773}
\gll {…} {di} {sini} {ada} {air,} {mari} {\bluebold{sini{\Tilde}sini}}\\ %
&  & at & \textsc{l.prox} & exist & water & hither & \textsc{rdp}{\Tilde}\textsc{l.prox}\\
\lspbottomrule
\end{tabular}
\ea
\glt
‘[(you) may fish from up here,] here is water, (come) here, \bluebold{right here}’ \textstyleExampleSource{[081025-003-Cv.0093]}
\end{styleFreeTranslEngxvpt}

\paragraph[Interrogatives]{Interrogatives}
\label{bkm:Ref359675947}
Likewise, interrogatives can be reduplicated when used pronominally (for details on their different uses, see §5.8). Reduplicated interrogatives signal indefiniteness by referring to an unspecified group member, in the sense of ‘any’ or ‘Wh-ever’. This is illustrated with the examples in (0) to (0); the example in (0) is elicited.
\end{styleBodyxafter}

\begin{tabular}{lllllll}
\lsptoprule
\label{bkm:Ref359673501}
\gll {yo,} {tida} {bole} {kas} {taw} {\bluebold{siapa{\Tilde}siapa}}\\ %
& yes & \textsc{neg} & may & give & know & \textsc{rdp}{\Tilde}who\\
\lspbottomrule
\end{tabular}
\ea
\glt 
‘yes, (you) must not tell \bluebold{anybody}’ \textstyleExampleSource{[080922-001a-CvPh.0288]}
\z

\begin{tabular}{lllllllll}
\lsptoprule
(\stepcounter{}{\the}) & saya & tida & biking & \bluebold{apa{\Tilde}apa} & karna & babi & suda & mati\\
& \textsc{1sg} & \textsc{neg} & make & \textsc{rdp}{\Tilde}what & because & pig & already & die\\
\lspbottomrule
\end{tabular}
\ea
\glt 
[About hunting a wild pig:] ‘I didn’t do \bluebold{anything} because the pig was already dead’ \textstyleExampleSource{[080919-004-NP.0023]}
\z

\begin{tabular}{llllllll}
\lsptoprule
\label{bkm:Ref359673504}
\gll {nanti} {\bluebold{kapang{\Tilde}kapang}} {ka} {ko} {jalang{\Tilde}jalang} {ke} {mari}\\ %
& very.soon & \textsc{rdp}{\Tilde}when & or & \textsc{2sg} & \textsc{rdp}{\Tilde}walk & to & hither\\
\lspbottomrule
\end{tabular}
\ea
\glt 
‘later \bluebold{whenever} (you have time) you come here’ \textstyleExampleSource{[Elicited BR120813.029]}
\z

\begin{tabular}{lllllllll}
\lsptoprule
\label{bkm:Ref438977437}
\gll {di} {\bluebold{mana{\Tilde}mana}} {smua} {pake} {ini,} {tajam} {besi} {ini}\\ %
& at & \textsc{rdp}{\Tilde}where & all & use & \textsc{d.prox} & be.sharp & metal & \textsc{d.prox}\\
\lspbottomrule
\end{tabular}
\ea
\glt 
[About sagu production:] ‘\bluebold{wherever} everybody uses it, this sharp metal’ \textstyleExampleSource{[081014-006-CvPr.0059]}
\z


Alternatively, speakers may use the bare interrogative followed by the focus adverb \textitbf{saja} ‘just’ to encode indefinite referents, as discussed in §5.8.8.
\end{styleBodyxvafter}

\paragraph[Causative verb kasi ‘give’ and reciprocity marker baku ‘recp’]{Causative verb \textitbf{kasi} ‘give’ and reciprocity marker \textitbf{baku} ‘\textsc{recp}’}
\label{bkm:Ref360095881}
Reduplication of the causative verb \textitbf{kasi} ‘give’ and reciprocity marker \textitbf{baku} ‘\textsc{recp}’, as in (0) and (0) respectively, signals repetition or continuation. (For more details on causative and reciprocal constructions, see §11.2 and §11.3, respectively.)
\end{styleBodyxafter}

\begin{tabular}{llllllllll}
\lsptoprule
\label{bkm:Ref359673860}
\gll {knapa} {kam} {\bluebold{kas{\Tilde}kas}} {bangung} {dia,} {de} {masi} {mo} {tidor}\\ %
& why & \textsc{2pl} & \textsc{rdp}{\Tilde}give & wake.up & \textsc{3sg} & \textsc{3sg} & still & want & sleep\\
\lspbottomrule
\end{tabular}
\ea
\glt 
‘why do you \bluebold{keep }waking him up?, he still wants to sleep’ (Lit. ‘\bluebold{give{\Tilde}give} to wake up’) \textstyleExampleSource{[080918-001-CvNP.0039]}
\z

\begin{tabular}{lllllllll}
\lsptoprule
\label{bkm:Ref359673861}
\gll {itu} {sampe} {tong} {\bluebold{baku{\Tilde}baku}} {tawar} {ini} {deng} {doseng}\\ %
& \textsc{d.dist} & until & \textsc{1pl} & \textsc{rdp}{\Tilde}\textsc{recp} & bargain & \textsc{d.prox} & with & lecturer\\
\lspbottomrule
\end{tabular}
\ea
\glt
‘it got to the point that we and the lecturer were arguing \bluebold{constantly with each other}’ \textstyleExampleSource{[080917-010-CvEx.0177]}
\end{styleFreeTranslEngxvpt}

\subsection{Gesamtbedeutung of reduplication}
\label{bkm:Ref360284130}
Reduplication in Papuan Malay conveys a number of different meaning aspects ranging from continuation and diversity to disparagement and imitation. This variety in meaning raises two questions: first, does reduplication have a general meaning or gesamtbedeutung, and second, is there a specific relation between the meaning and the syntactic class of the base word.



Table  ‎4 .6 lists the Papuan Malay word classes which attract reduplication and the meaning aspects they convey.
\end{styleBodyvvafter}

\begin{stylecaption}
\label{bkm:Ref360116475}Table ‎4.\stepcounter{Table}{\theTable}:  Word classes and meaning aspects in reduplication
\end{stylecaption}

\tablehead{
\multicolumn{2}{l}{ Dimension} & Meaning aspects & \arraybslash Word class of base\\
}
\begin{tabular}{llll}
\lsptoprule
\textsc{aug} & \textsc{quant} & Continuation/repetition/habit & \textsc{n}, \textsc{v}, \textsc{caus}, \textsc{recp}\\
& \textsc{quant} & Collectivity & \textsc{n}, \textsc{v}, \textsc{num}, \textsc{pro}\\
& \textsc{quant} & Diversity & \textsc{n}, \textsc{v}, \textsc{dem}, \textsc{loc}\\
& \textsc{quant} & Distributiveness & \textsc{num}, \textsc{qt}\\
& \textsc{intens} & Intensity & \textsc{v}, \textsc{adv}, \textsc{loc}\\
& \textsc{intens} & Immediacy & \textsc{v}\\
\textsc{dim} &  & Disparagement & \textsc{pro}\\
&  & Indefiniteness & \textsc{n}, \textsc{int}\\
&  & Aimlessness & \textsc{v}\\
&  & Attenuation & \textsc{v}\\
&  & Imitation & \textsc{v}, \textsc{pro}\\
\lspbottomrule
\end{tabular}

Some of the meaning aspects which reduplication in Papuan Malay conveys include, what cross-linguistically {\citet[130]{Moravcsik2013}} refers to as “contradictory senses”. The aspect ‘immediacy’, for instance, represents an increase in intensity, while the aspect ‘aimlessness’ refers to a decrease in intensity. This phenomenon that reduplication brings together a variety of meanings, some of which are opposite, is quite common cross-linguistically (see, for instance, {Regier 1994; Mattes 2007: 124–125; Kiyomi 2009: 1151; Moravcsik 2013: 129–133}).
\end{styleBodyaftervbefore}


Examining the “crosslinguistically recurrent semantic properties of reduplication”, {\citet[131]{Moravcsik2013}} comes to the conclusion that
\end{styleBodyvvafter}

\begin{styleIvI}
reduplication may be viewed as a marking device to indicate that the word is to be understood in an out-of-the-ordinary sense: the meaning deviates from the normal sense of the base either by being “more” or by being “less”.
\end{styleIvI}


These contradictory meaning aspects of augmentation and diminution have also been noted for Malayo-Polynesian languages. In her study on reduplication in 30 of these languages, {\citet[1151]{Kiyomi2009}} considers these two meanings of reduplication to be


\begin{styleIvI}
two manifestations of the same semantic principle of ‘a …er degree of …’, which is projected in the opposite directions. Then one can postulate that the prototypical meanings of reduplication represent the semantic principle ‘\textsc{a} \textsc{higher}/\textsc{lower} \textsc{degree} \textsc{of} …’
\end{styleIvI}


The overview presented in Table  ‎4 .6 indicates that this semantic principle of “a …er degree of …” in terms of augmentation or diminution also accounts for the different meaning aspects of reduplication in Papuan Malay.



In Papuan Malay, the notion of ‘higher degree of …’ involves augmentation in terms of quantity or intensity. Cross-linguistically, {Moravcsik (1978: 317, 321)} specifies that in the context of reduplication quantity can refer to the “participants of [an] event or the events themselves”, while intensity refers to the amount of “energy investment or size of effect”. In Papuan Malay, the augmentation of quantity includes meaning aspects such as collectivity or repetition, while the increase in intensity includes the meaning aspects of intensity and immediacy, as listed in Table  ‎4 .6.
\end{styleBodyvafter}


The notion of ‘lesser degree of …’ involves, generally speaking, diminution which typically “adds the meaning of smallness to the stem meaning” {\citep[1153]{Kiyomi2009}}. As {\citet[424]{Jurafsky1993}} points out, however, the diminutive exhibits a variety of “metaphorical extensions” which involve “meaning shifts from the physical world to the social domain, and from the physical world to the conceptual or category domain”. Such semantic extensions of the diminutive are also found in Papuan Malay, in that the semantic effect of diminution brings together the meaning aspects of disparagement, indefiniteness, aimlessness, attenuation, and imitation.
\end{styleBodyvafter}


The ‘disparagement’ sense is linked to the notion of diminution metaphorically in that it has to do with social importance or power. The ‘indefiniteness’ sense is also a metaphorical extension in that it conveys toned-down reference. Likewise, the ‘aimlessness’ sense is linked to the notion of diminution in that it denotes actions which are done with less intensity. The ‘attenuation’ sense is a metaphorical extension of the core sense ‘size’ in that it denotes properties which are weaker, or activities which are carried out less intensely. The ‘imitation’ sense refers to objects or activities which copy or imitate what the base denotes. This sense is linked to the core sense of diminution in that the objects and activities are not identical with their models but merely resemble them a little bit. (See {Jurafsky 1993: 426, 430; Mattes 2007: 125; V. Mattes, p.c. 2013; Moravcsik 2013: 129–130}.)
\end{styleBodyvafter}


In summarizing the above and in applying {Kiyomi’s (2009: 1151)} terminology, it is concluded that in Papuan Malay, the gesamtbedeutung of reduplication is ‘a \textsc{higher}/\textsc{lower} \textsc{degree} \textsc{of} …’. Table  ‎4 .7 gives examples for the two dimensions of augmentation and diminution conveyed by reduplication.
\end{styleBodyvvafter}

\begin{stylecaption}
\label{bkm:Ref360116476}Table ‎4.\stepcounter{Table}{\theTable}:  Gesamtbedeutung of reduplication
\end{stylecaption}

\tablehead{
\multicolumn{2}{l}{ Dimensions} & Item & \multicolumn{2}{l}{ Gloss}\\
}
\begin{tabular}{lllll}
\lsptoprule
Augmentation & Quantity & \textitbf{ana{\Tilde}ana} & \textsc{rdp}{\Tilde}child & ‘children’\\
& Intensity & \textitbf{pintar{\Tilde}pintar} & \textsc{rdp}{\Tilde}be.clever & ‘be very clever\\
Diminution & Attenuation & \textitbf{kurus{\Tilde}kurus} & \textsc{rdp}{\Tilde}be.thin & ‘be rather thin’\\
& Imitation & \textitbf{mati{\Tilde}mati} & \textsc{rdp}{\Tilde}die & ‘die like …’\\
\lspbottomrule
\end{tabular}

With respect to the relation between the meaning and the syntactic class of the base word, two major observations are made. First, across word classes, reduplicated lexemes differ in terms of the meaning aspects which they convey. Second, meaning aspects differ as regards the range of word classes they attract for reduplication. (See Table  ‎4 .6.)
\end{styleBodyaftervbefore}


First, concerning the reduplicated lexemes and the meaning aspects they convey, the gathered data indicates that within certain word classes reduplication tends to convey more than one specific meaning. Reduplication in certain verbs, for example, can express immediacy while in other verbs it signals continuation or repetition. It is notably content words which carry this variety of different meanings, that is, nouns, verbs, and numerals. In addition, reduplication within two classes of function words also conveys more than one meaning aspect, namely in the classes of personal pronouns and locatives. Reduplication within the other three classes of function words, by contrast, tends to carry specific meanings: reduplicated demonstratives express diversity, interrogatives indicate indefiniteness, and the causative and reciprocity markers signal continuation or repetition. In relating the word classes which attract reduplication to certain meaning aspects it is noted, however, that the meaning of a given reduplicated form is more than the meaning of its constituents. The fact that the entire reduplicated form and not its individual constituents carry this meaning, indicates, what {Booij (2013: 260–261)} calls, a “holistic” or constructional meaning of the reduplicated forms, applying.
\end{styleBodyvafter}


Second, regarding the meaning aspects and the range of word classes they attract for reduplication, three meaning aspects bring together the largest number of different word classes, namely four each. The continuation/repetition/habit meaning aspect brings together nouns, verbs, and the causative and reciprocity markers. The collectivity meaning aspect brings together nouns, verbs, numerals, and personal pronouns. And the diversity meaning aspect brings together nouns, verbs, demonstratives, and locatives. Another pertinent meaning aspect is intensity, which attracts three different word classes for reduplication, namely verbs, adverbs, and locatives. Three more meaning aspects, which attract two word classes each for reduplication, are distributiveness, indefiniteness, and imitation. The remaining meaning aspects attract only one word class each for reduplication, that is, verbs for immediacy, aimlessness, and attenuation, and personal pronouns for disparagement. These observations suggest that there is not a specific, one-to-one relation between the meaning and the syntactic class of the base word.
\end{styleBodyvxvafter}

\section{Reduplication across eastern Malay varieties}
\label{bkm:Ref360287459}
Reduplication is also very common in other eastern Malay varieties, such as Ambon Malay (AM) {(van Minde 1997: 112–140)}, Banda Malay (BM) {(Paauw 2009: 160, 206)}, Kupang Malay (KM) {(Paauw 2009: 160, 171–173, 206, 252–253)}, Larantuka Malay (LM) {(Paauw 2009: 161, 171–173, 206, 256–258)}, Manado Malay (MM) {(Stoel 2005: 25–28)}, and Ternate Malay (TM) {(Litamahuputty 2012: 136–139)}. This section compares reduplication across these Malay varieties in terms of lexeme formation (§4.3.1), lexeme interpretation (§4.3.2), and interpretational shift (§4.3.3), as far as mentioned in the literature. For comparison, reduplication in Papuan Malay (PM) is also included. Also included for comparison is Standard Indonesian (SI) ({MacDonald 1976; Mintz 2002; Sneddon 2010}).
\end{styleBodyxvafter}

\subsection{Lexeme formation}
\label{bkm:Ref360284484}
Similar to Papuan Malay, the above-mentioned six Malay varieties also employ full reduplication, as shown in Table  ‎4 .8. Typically, reduplication affects content words, while reduplication of function words is rarer. Manado and Ternate Malay also employ reduplication of bound morphemes. The data in Table  ‎4 .8 also shows which varieties use a combination of reduplication and affixation, and in which varieties reduplicated forms without corresponding base words are found. Besides Papuan Malay, only two of the six other eastern Malay varieties use partial and imitative reduplication, namely Ambon and Larantuka Malay.


\begin{stylecaption}
\label{bkm:Ref360204525}Table ‎4.\stepcounter{Table}{\theTable}:  Lexeme formation across eastern Malay varieties and Standard Indonesian
\end{stylecaption}

\begin{tabular}{llllllllllll}
\lsptoprule

\multicolumn{12}{l}{%\setcounter{itemize}{0}
\begin{itemize}
\item Full reduplication\end{itemize}
}\\
\multicolumn{12}{l}{\begin{itemize}
\item %\setcounter{itemize}{0}
\begin{itemize}
\item Content words (productive)\end{itemize}
\end{itemize}
}\\
& \textsc{n} & \multicolumn{2}{l}{PM} & \multicolumn{2}{l}{AM} & BM & KM & LM & MM & TM & SI\\
& \textsc{v} & \multicolumn{2}{l}{PM} & \multicolumn{2}{l}{AM} &  & KM & LM & MM & TM & SI\\
& \textsc{adv} & \multicolumn{2}{l}{PM} & \multicolumn{2}{l}{AM} &  &  &  & MM &  & SI\\
& \textsc{num} & \multicolumn{2}{l}{PM} & \multicolumn{2}{l}{AM} &  & KM &  & MM &  & SI\\
& \textsc{qt} & \multicolumn{2}{l}{PM} & \multicolumn{2}{l}{} &  &  & LM &  &  & \\
\multicolumn{12}{l}{\begin{itemize}
\item %\setcounter{itemize}{0}
\begin{itemize}
\item Function words (unproductive)\end{itemize}
\end{itemize}
}\\
& \textsc{pro} & \multicolumn{2}{l}{PM} & \multicolumn{2}{l}{AM} &  & KM & LM & MM &  & SI\\
& \textsc{dem} & \multicolumn{2}{l}{PM} & \multicolumn{2}{l}{AM} &  &  &  &  &  & \\
& \textsc{loc} & \multicolumn{2}{l}{PM} & \multicolumn{2}{l}{AM} &  &  &  &  &  & \\
& \textsc{int} & \multicolumn{2}{l}{PM} & \multicolumn{2}{l}{AM} &  & KM & LM &  &  & \\
& \textsc{caus} & \multicolumn{2}{l}{PM} & \multicolumn{2}{l}{} &  &  &  &  &  & \\
& \textsc{recp} & \multicolumn{2}{l}{PM} & \multicolumn{2}{l}{} &  &  &  & MM & TM & \\
\multicolumn{12}{l}{\begin{itemize}
\item %\setcounter{itemize}{0}
\begin{itemize}
\item Bound morphemes (unproductive)\end{itemize}
\end{itemize}
}\\
& \textsc{pfx} & \multicolumn{2}{l}{} & \multicolumn{2}{l}{} &  &  &  & MM & TM & \\
\multicolumn{12}{l}{\begin{itemize}
\item %\setcounter{itemize}{0}
\begin{itemize}
\item Reduplication and affixation (productive)\end{itemize}
\end{itemize}
}\\
& \multicolumn{2}{l}{\textsc{rdp} prec. \textsc{affx}} & \multicolumn{2}{l}{PM} & AM &  &  & LM & MM &  & SI\\
& \multicolumn{2}{l}{\textsc{affx} prec. \textsc{rdp}} & \multicolumn{2}{l}{PM} & AM &  & KM &  &  &  & SI\\
\multicolumn{12}{l}{\begin{itemize}
\item %\setcounter{itemize}{0}
\begin{itemize}
\item No corresponding base words (unproductive)\end{itemize}
\end{itemize}
}\\
& \textsc{n} & \multicolumn{2}{l}{PM} & \multicolumn{2}{l}{AM} &  &  &  &  &  & SI\\
& \textsc{v} & \multicolumn{2}{l}{PM} & \multicolumn{2}{l}{AM} &  &  &  &  &  & SI\\
& \textsc{qt} & \multicolumn{2}{l}{PM} & \multicolumn{2}{l}{} &  &  &  &  &  & \\
& \textsc{adv} & \multicolumn{2}{l}{} & \multicolumn{2}{l}{AM} &  &  &  &  &  & SI\\
& \textsc{cnj} & \multicolumn{2}{l}{PM} & \multicolumn{2}{l}{AM} &  &  &  &  &  & SI\\
\multicolumn{12}{l}{\begin{itemize}
\item Partial reduplication\end{itemize}
}\\
& productive & \multicolumn{2}{l}{PM} & \multicolumn{2}{l}{AM} &  &  & LM &  &  & \\
& unproductive & \multicolumn{2}{l}{} & \multicolumn{2}{l}{} &  &  &  &  &  & SI\\
\multicolumn{12}{l}{\begin{itemize}
\item Imitative reduplication (unproductive)\end{itemize}
}\\
&  & \multicolumn{2}{l}{PM} & \multicolumn{2}{l}{AM} &  &  & LM &  &  & SI\\
\lspbottomrule
\end{tabular}

The data given in Table  ‎4 .8 shows that reduplication in Ambon Malay is about as pervasive as in Papuan Malay, with both varieties sharing many features. This applies to the attested reduplication types (full, partial, and imitative), as well as to the attested morpheme types which can be reduplicated. For the five other eastern Malay varieties and Standard Indonesian, reduplication seems to play a much lesser role, as shown by the gaps in Table  ‎4 .8. For the eastern Malay varieties, this applies especially to the reduplication of function words; furthermore, these varieties appear not to have reduplicated forms which lack a corresponding unreduplicated base.
\end{styleBodyaftervbefore}


Two explanations present themselves for these observations. One explanation is that the commonalities between Papuan Malay and Ambon Malay, together with the lack of overlap with the five other eastern Malay varieties, are due to the distinct history of Papuan Malay, argued for in §1.8. An alternative explanation is that the differences among the eastern Malay varieties are due to differing degrees of depth with which the different authors describe reduplication in the Malay varieties presented in Table  ‎4 .8. This grammar on Papuan Malay, as well as that of Ambon Malay, and also those of Standard Indonesian, describe reduplication as a word-formation process rather thoroughly, while the descriptions of the five other eastern Malay varieties mention only the most salient features of reduplication in these varieties; hence, the rather large number of gaps in Table  ‎4 .8.
\end{styleBodyvxvafter}

\subsection{Lexeme interpretation}
\label{bkm:Ref360284485}
As in Papuan Malay, the gesamtbedeutung of reduplication in the six other eastern Malay varieties is ‘a \textsc{higher}/\textsc{lower} \textsc{degree} \textsc{of} …’. Table  ‎4 .9 gives examples for this gesamtbedeutung across the seven Malay varieties.


\begin{stylecaption}
\label{bkm:Ref360284730}Table ‎4.\stepcounter{Table}{\theTable}:  Gesamtbedeutung of reduplication across eastern Malay varieties
\end{stylecaption}

\tablehead{
 Dimensions & Malay & Item & \multicolumn{2}{l}{ Gloss}\\
}
\begin{tabular}{lllll}
\lsptoprule
\textsc{aug.quant} & PM & \textitbf{bua{\Tilde}bua} & \textsc{rdp}{\Tilde}fruit & ‘various fruits’\\
& AM & \textitbf{kata{\Tilde}kata} & \textsc{rdp}{\Tilde}word & ‘words’\\
& BM & \textitbf{orang{\Tilde}orang} & \textsc{rdp}{\Tilde}person & ‘people’\\
& KM & \textitbf{buku{\Tilde}buku} & \textsc{rdp}{\Tilde}book & ‘books’\\
& LM & \textitbf{ana{\Tilde}ana} & \textsc{rdp}{\Tilde}child & ‘children’\\
& MM & \textitbf{dua{\Tilde}dua} & \textsc{rdp}{\Tilde}two & ‘all two, both’\\
& TM & \textitbf{ular{\Tilde}ular} & \textsc{rdp}{\Tilde}snake & ‘snakes’\\
\textsc{aug.intens} & PM & \textitbf{pintar{\Tilde}pintar} & \textsc{rdp}{\Tilde}be.clever & ‘be very clever\\
& AM & \textitbf{biru{\Tilde}biru} & \textsc{rdp}{\Tilde}green & ‘be very green’\\
& LM & \textitbf{uma{\Tilde}ame} & \textsc{rdp}{\Tilde}chew & ‘chew wildly’\\
& MM & \textitbf{kita{\Tilde}kita} & \textsc{rdp}{\Tilde}\textsc{1sg} & ‘constantly me’\\
& TM & \textitbf{ba{\Tilde}ba}–\textitbf{diang} & \textsc{rdp}{\Tilde}\textsc{int}–be.quiet & ‘be very quiet’\\
\textsc{dim} & PM & \textitbf{kurus{\Tilde}kurus} & \textsc{rdp}{\Tilde}be.thin & ‘rather thin’\\
& AM & \textitbf{malu{\Tilde}malu} & \textsc{rdp}{\Tilde}ashamed & ‘shy as’\\
& KM & \textitbf{apa{\Tilde}apa} & \textsc{rdp}{\Tilde}what & ‘anything’\\
& LM & \textitbf{apa{\Tilde}apa} & \textsc{rdp}{\Tilde}what & ‘anything’\\
& MM & \textitbf{saki{\Tilde}saki} & \textsc{rdp}{\Tilde}be.sick & ‘sickly’\\
\lspbottomrule
\end{tabular}

Table  ‎4 .10 demonstrates in more detail which word classes attract reduplication and which meaning aspects they convey in all seven Malay varieties.
\end{styleBodyaftervbefore}


First, the data in Table  ‎4 .10 shows that across the seven Malay varieties, reduplication of content words tends to convey more than one meaning aspect, while reduplicated function words tend to carry specific meaning aspects, such as indefiniteness for interrogatives. The exception is Manado Malay, where reduplication of content words tends to carry a specific meaning, such as plurality for nouns.
\end{styleBodyvafter}


Second, the data in Table  ‎4 .10 illustrates that in the other eastern Malay varieties some meaning aspects also attract a wider range of word classes for reduplication than other meaning aspects. This applies to the plurality/diversity, the intensity, the continuation/repetition/habit, and the indefiniteness meaning aspects.
\end{styleBodyvafter}


Of all the eastern Malay varieties, the different meaning aspects attested in Papuan Malay attract the widest range of different word classes. For Ambon Malay, the range of attracted word classes is also rather large. In the other eastern Malay varieties, however, the attracted range of word classes is much smaller. At this point, it remains unclear, though, whether these smaller ranges are due to inherent properties of these varieties or due to an incomplete documentation in the respective literature.
\end{styleBodyvvafter}

\begin{stylecaption}
\label{bkm:Ref360274824}Table ‎4.\stepcounter{Table}{\theTable}:  Word classes and meaning aspects in reduplication across eastern Malay varieties\footnote{\\
\\
\\
\\
\\
\\
\\
\\
\\
\\
\\
In Table  ‎4 .10, the category of prefixes in Manado Malay {\citep[27]{Stoel2005}} and Ternate Malay {\citep[139]{Litamahuputty2012}} includes the reciprocal marker \textitbf{baku} ‘\textsc{rcp}’. Reduplicated Ternate Malay “activity words” {(Litamahuputty 2012: 136–138)} are included in the word class of verbs.\\
\\
\\
\\
}
\end{stylecaption}

\tablehead{ & PM & AM & BM & KM & LM & MM & \arraybslash TM\\
}
\begin{tabular}{llllllll}
\lsptoprule
\multicolumn{8}{l}{Augmentation (quantity)}\\
Continuation/ repetition/habit & \textsc{n}, \textsc{v}, \textsc{adv}, \textsc{recp}, \textsc{caus} & \textsc{v} &  & \textsc{v} & \textsc{v} & \textsc{v}, \textsc{pfx} & \textsc{v}, \textsc{pfx}\\
Plurality/ diversity & \textsc{n}, \textsc{v}, \textsc{dem}, \textsc{loc} & \textsc{n}, \textsc{v} & \textsc{n} & \textsc{n} & \textsc{n} & \textsc{n} & \textsc{n}, \textsc{v}\\
Collectivity & \textsc{num}, \textsc{pro} & \textsc{num} &  &  &  & \textsc{num} & \\
Distributiveness & \textsc{num}, \textsc{qt} &  &  &  &  &  & \\
Involvement &  &  &  &  &  & \textsc{pro} & \\
Totality &  & \textsc{n} & \textsc{n} &  &  &  & \\
Aimlessness & \textsc{v} &  &  &  & \textsc{v} &  & \textsc{v}\\
\multicolumn{8}{l}{Augmentation (intensity)}\\
Intensity & \textsc{v}, \textsc{adv}, \textsc{loc} & \textsc{v}, \textsc{adv} &  &  & \textsc{v} & \textsc{adv} & \textsc{v}, \textsc{pfx}\\
Immediacy & \textsc{v} &  &  &  &  &  & \\
\multicolumn{8}{l}{Diminution}\\
Disparagement & \textsc{pro} &  &  &  &  &  & \\
Indefiniteness & \textsc{n}, \textsc{int} & \textsc{dem}, \textsc{int} &  & \textsc{pro}, \textsc{int} & \textsc{pro}, \textsc{int} &  & \\
Attenuation & \textsc{v} & \textsc{v} &  &  &  &  & \\
Vagueness &  & \textsc{adv} &  &  &  &  & \\
Imitation & \textsc{v}, \textsc{pro} & \textsc{v}, \textsc{pro} &  &  &  &  & \\
\lspbottomrule
\end{tabular}

Overall, there is not a specific, one-to-one relation between the meaning aspects of the reduplicated lexemes and the syntactic class of the corresponding base words in any of the Malay varieties discussed here.


\subsection{Interpretational shift}
\label{bkm:Ref374440478}\label{bkm:Ref365986734}
Interpretational shift of reduplicated lexemes, as described for Papuan Malay (see §4.2.1.4 and §4.2.2.8), is also attested for Ambon Malay {(van Minde 1997: 118, 123, 125)}, Larantuka Malay {(Paauw 2009: 126, 270)}, Manado Malay {\citep[26]{Stoel2005}}, and Ternate Malay {\citep[220]{Litamahuputty2012}}.



With respect to the patterns of interpretational shift, two observations are made which are summarized in Table  ‎4 .11. First, in each of the varieties for which interpretational shift is mentioned, it is content words that may undergo such a shift. Second, the Malay varieties differ in terms of the syntactic categories of the base and the readings which the reduplicated forms can receive. In Papuan and Ambon Malay, nouns and verbs can undergo interpretational shift, while in Manado Malay only nouns and in Larantuka and Ternate Malay only verbs are affected. Most often, such shifts result in the reduplicated form receiving an adverbial reading. Such is the case in Papuan, Ambon, Larantuka, and Ternate Malay; the exception is Manado Malay. Considerably less often the shift results in a nominal reading (Papuan and Ambon Malay) or a verbal reading (Papuan Malay) of reduplicated lexemes.
\end{styleBodyvvafter}

\begin{stylecaption}
\label{bkm:Ref360200738}Table ‎4.\stepcounter{Table}{\theTable}:  Patterning of interpretational shift across eastern Malay varieties
\end{stylecaption}

\tablehead{
\multicolumn{2}{l}{ Syntactic category} & \multicolumn{2}{l}{ Reduplicated forms and their meanings} & \arraybslash Received reading\\
}
\begin{tabular}{lllll}
\lsptoprule
Nouns & PM & \textitbf{rawa{\Tilde}rawa} & ‘be swampy’ & verbal\\
&  & \textsc{rdp}{\Tilde}swamp &  & \\
&  & \textitbf{malam{\Tilde}malam} & ‘late at night’ & adverbial\\
&  & \textsc{rdp}{\Tilde}night &  & \\
& AM & \textitbf{malang{\Tilde}malang} & ‘during the night’ & adverbial\\
&  & \textsc{rdp}{\Tilde}night &  & \\
& MM & \textitbf{opa{\Tilde}opa} & ‘quite old’ & verbal\\
&  & \textsc{rdp}{\Tilde}grandfather &  & \\
Verbs\footnotemark{} & PM & \textitbf{gait{\Tilde}gait} & ‘pole’ & nominal\\
&  & \textsc{rdp}{\Tilde}hook &  & \\
&  & \textitbf{baru{\Tilde}baru} & ‘just now’ & adverbial\\
&  & \textsc{rdp}{\Tilde}be.new &  & \\
& AM & \textitbf{gai{\Tilde}gai} & ‘pole’ & nominal\\
&  & \textsc{rdp}{\Tilde}hook &  & \\
&  & \textitbf{kamuka{\Tilde}kamuka} & ‘formerly, earlier’ & adverbial\\
&  & \textsc{rdp}{\Tilde}go.first &  & \\
& LM & \textitbf{tiba{\Tilde}tiba} & ‘suddenly’ & adverbial\\
&  & \textsc{rdp}{\Tilde}arrive &  & \\
& TM & \textitbf{asik{\Tilde}asik} & ‘busily’ & adverbial\\
&  & \textsc{rdp}{\Tilde}busy &  & \\
\lspbottomrule
\end{tabular}
\footnotetext{\\
\\
\\
\\
\\
\\
\\
\\
\\
\\
\\
The ‘verb’ category includes Manado Malay adjectives and Ternate Malay “quality words” {(Litamahuputty 2012: 136–138)}.\\
\\
\\
\\
}

The ability of reduplicated lexemes to undergo interpretational shift seems to be best explained in terms of a slot filling-function of reduplication. Cross-linguistically, temporal noun phrases, for instance, are prone to fill adverbial slots; an example is the English sentence ‘she came home late at night’.
\end{styleBodyaftervbefore}


Hence, in this grammar of Papuan Malay, the interpretational shifts described in §4.2.1.4 and §4.2.2.8 are taken to result from a slot filling-function of reduplication. That is, reduplication enables base words to fill different syntactic slots, such as an adverbial or a nominal slot.
\end{styleBodyvafter}


In Ternate Malay, interpretational shifts also seem to be the results of a slot-filling function of reduplication, with {\citet[220]{Litamahuputty2012} }noting that “both reduplicated quality words and activity words may serve to express manner when they immediately follow an activity”. For Ambon Malay, by contrast, {van Minde (1997: 118, 123, 125)} considers the observed interpretational shifts as “transpositions” which result from “derivational” processes. For Manado Malay, {\citet[26]{Stoel2005}} notes that when kinship terms or similar words are reduplicated “then the reduplicated form is an adjective referring to a certain age group”. This statement suggests that {\citet{Stoel2005}} considers interpretational shifts to result from a category-changing function of reduplication. For Larantuka Malay, {Paauw (2009: 126, 270)} does not discuss the attested interpretational shifts.
\end{styleBodyvxvafter}

\section{Summary}
\label{bkm:Ref363111369}
Reduplication in Papuan Malay is a very productive morphological device for deriving new words. In terms of lexeme formation, three different types of reduplication are attested: full, partial, and imitative reduplication. The most common type is full reduplication, which involves the repetition of an entire root, stem, or word; bound morphemes are not reduplicated. Full reduplication usually applies to content words, although some function words can also be reduplicated. Partial and imitative reduplication are rare. The gesamtbedeutung of reduplication is “a \textsc{higher}/\textsc{lower} \textsc{degree} \textsc{of} …” in the sense of augmentation and diminution, applying {Kiyomi’s (2009: 1151)} terminology. There is, however, no specific, one-to-one relation between the meaning aspects of the reduplicated lexemes and the syntactic class of the corresponding base words.



A comparison of reduplication in Papuan Malay and five other eastern Malay varieties shows that Papuan Malay shares many features with Ambon Malay. In both varieties, reduplication plays an important role. In Banda, Manado, Larantuka, and Ternate Malay, by contrast, reduplication seems to be much less pervasive. These commonalities and differences may well point to the particular history of Papuan Malay, argued for in §1.8. The observed differences could, however, also result from gaps in the descriptions of Banda, Manado, Larantuka, and Ternate Malay.
\end{styleBodyvafter}

%\setcounter{page}{1}\chapter[Word classes]{Word classes}
\label{bkm:Ref361997822}\section{Introduction}

This chapter discusses the Papuan Malay word classes, or parts of speech. Some of the word classes are examined in more detail in separate chapters.



The notion of “word class” is defined as a class of “words that share morphological or syntactic properties” {\citep[5188]{Asher1994}}. In general, pertinent criteria for establishing class membership are a “word’s distribution, its range of syntactic functions, and the morphological or syntactic categories for which it is specifiable” {(Schachter and Shopen 2007: 1–2)}. In Papuan Malay, morphological criteria do not play a major role in distinguishing different word classes, given the lack of inflectional morphology and the rather limited productivity of derivational patterns (see §3.1). Instead the main criteria for defining distinct word classes are their syntactic properties.
\end{styleBodyvafter}


Based on their syntactic properties, three open and several closed lexical classes are distinguished. It is acknowledged, however, that Papuan Malay has membership overlap between a number of categories (see §5.14). Most of this variation involves verbs, including overlap between verbs and nouns, which is typical of Malay varieties and other western Austronesian languages. In discussing lexical and syntactic categories in western Austronesian languages,\footnote{\\
\\
\\
\\
\\
\\
\\
\\
\\
\\
\\
More specifically, {\citet[112]{Himmelmann2005}} refers to western Austronesian “symmetrical voice languages”, that is languages that have “at least two voice alternations marked on the verb, neither of which is clearly the basic form”.\\
\\
\\
\\
} {\citet[127]{Himmelmann2005}} points out, that “the syntactic distinction between nouns and verbs is often somewhat less clearly delineated in that word-forms which semantically appear to be verbs easily and without further morphological modification occur in nominal functions and vice versa”. This applies especially to languages with “multifunctional lexical bases, that is, “lexical bases which occur without further affixation in a variety of syntactic functions” {(2005: 129)}.
\end{styleBodyvafter}


Regarding the analytical consequences of such overlap, {\citet[128]{Himmelmann2005}} notes that most authors “assume underlying syntactic differences based on the semantics of the forms”, analyzing such instances of variation “as involving zero conversion”. As far as the description of regional Malay varieties is concerned, this approach is accepted, for example, by {van \citet{Minde1997}} in his grammar of Ambon Malay, {\citet{Stoel2005}} in his description of Manado Malay, and {\citet[250]{Paauw2009}} in his discussion of regional Malay varieties such as Banda Malay, Kupang Malay, or Larantuka Malay. Some authors, however, “argue for a basic lack of a morphosyntactic noun/verb distinction”, as {\citet[128]{Himmelmann2005}} points out. Examples for this alternative approach are {Gil’s (2013)} description of Riau Indonesian (see also {Gil 1994}), {Himmelmann’s (2008)} analysis of Tagalog (see also {Himmelmann 1991}), and {Litamahuputty’s (2012)} grammar of Ternate Malay.
\end{styleBodyvafter}


In discussing Papuan Malay lexical and syntactic categories in this grammar, nouns and verbs are analyzed as belonging to distinct word classes, in spite of the attested variation in membership, discussed in §5.14. This approach is chosen because of the distinct syntactic properties of the categories under discussion, as shown in more detail throughout this chapter. In cases of variation, the category membership of a given lexeme can usually be deduced from the context in which an utterance occurs. Rather than proposing additional special word classes for lexical items with dual distribution, the lexemes in question are analyzed as having dual class membership and the variation as involving zero conversion.
\end{styleBodyvafter}


In the next two sections, the two major open lexical classes of nouns and verbs are discussed. The class of nouns, described in §5.2, includes common nouns, proper nouns, location, and direction nouns. Verbs, discussed in §5.3, are divided into trivalent, bivalent, and monovalent verbs, with the class of monovalent verbs including dynamic and stative verbs. Adverbs, discussed in §5.4, constitute the third open word class. The closed word classes are then described, that is, personal pronouns in §5.5, demonstratives in §5.6, locatives in §5.7, interrogatives in §5.8, numerals in §5.9, quantifiers in §5.10, prepositions in §5.11, and conjunctions in §5.12. Tags, placeholder and hesitation makers, interjections, and idiophones are presented in §5.13. The final section of this chapter (§5.14) discusses the categories with variation in word class membership.
\end{styleBodyvxvafter}

\section{Nouns}
\label{bkm:Ref374448552}\label{bkm:Ref374433347}\label{bkm:Ref353282651}\label{bkm:Ref351643962}\label{bkm:Ref351643101}
Papuan Malay has a large open class of nouns which refer to persons, things, and places, as well as abstract concepts and ideas. Typically, nouns have head-function in noun phrases and argument function in verbal clauses.



Based on their syntactic properties, the nouns are divided into common nouns (§5.2.1), proper nouns (§5.2.2), location nouns (§5.2.3), and direction nouns (§5.2.4). Their defining syntactic and functional properties are discussed in more detail in the respective sections.
\end{styleBodyvafter}


Morphological properties do not play a major role in defining nouns as a distinct word class. This is due to the lack of inflectional morphology and the limited role of derivational processes. The latter include reduplication, and, to a limited extent, affixation with suffix \-\textitbf{ang} or prefix \textscItalBold{pe(n)\-} (for details see §3.1.3 and §3.1.4, respectively).
\end{styleBodyvafter}


Nouns are distinct from other word classes such as verbs (§5.3), adverbs (§5.4), personal pronouns (§5.5), and demonstratives (§5.6) in terms of the following distributional properties. Some of these properties, however, do not apply to all four noun types. The exceptions are mentioned below and discussed in more detail in the respective sections on the different noun types.
\end{styleBodyvvafter}

%\setcounter{itemize}{0}
\begin{itemize}
\item \begin{styleIIndented}
Nouns are distinct from verbs (a) in terms of their predominant functions as heads in noun phrases and as arguments in verbal clauses, (b) in that they can be quantified with numerals and quantifiers (this only applies to common and proper nouns), and (c) in that they are only negated with \textitbf{bukang} ‘\textsc{neg}’.
\end{styleIIndented}\item \begin{styleIIndented}
Unlike adverbs, nouns (a) have predicative uses, and (b) can modify other nouns.
\end{styleIIndented}\item \begin{styleIIndented}
Nouns are distinct from personal pronouns, in that nouns (a) can be modified with personal pronouns, while personal pronouns are not modified with nouns, (b) can be modified with numerals/quantifiers in pre- or post-head position, while personal pronouns are only modified with numerals/quantifiers in post-head position, and (c) can express the possessum in adnominal possessive constructions, while personal pronouns do not take this slot. (Most of these properties only apply to common and (to a lesser extent to) proper nouns but not to locative and direction nouns.)
\end{styleIIndented}\item \begin{styleIvI}
Nouns can be modified with demonstratives, whereas demonstratives cannot be modified with nouns.
\end{styleIvI}\end{itemize}

The following sections describe the four noun types in more detail: common nouns are discussed in §5.2.1, proper nouns in §5.2.2, location nouns in §5.2.3, and direction nouns in §5.2.4. Also included are brief descriptions of time-denoting nouns in §5.2.5, classifying nouns in §5.2.6, and kinship terms in §5.2.7.
\end{styleBodyxvafter}

\subsection{Common nouns}
\label{bkm:Ref342375954}
Common nouns have general reference, in that they “do not refer to individual entities (‘tokens’) but only connote classes (‘types) of entities” {(Givón 2001: 58)}. They have the following defining syntactic and functional properties:


%\setcounter{itemize}{0}
\begin{itemize}
\item \begin{styleIIndented}
Head function in noun phrases is predominant (Chapter 8); in addition, they also have predicative function in non-verbal clauses (Chapter 12).
\end{styleIIndented}\item \begin{styleIIndented}
Argument function (subject or object) in verbal clauses is predominant (Chapter 11).\footnote{\\
\\
\\
\\
\\
\\
\\
\\
\\
\\
\\
As {Givón (2001: 59)} points out, it is technically speaking “not the noun but rather the \textit{noun phrase }that assumes the various grammatical roles […] However, within the noun phrase, a noun is typically the syntactic and semantic \textit{head}, defining the type of entity involved. All other elements in the noun phrase are \textit{modifiers }of that head noun”.\\
\\
\\
\\
}
\end{styleIIndented}\item \begin{styleIIndented}
Quantification (with numerals and quantifiers) and modification with adnominal constituents (including other nouns, verbs, personal pronouns, demonstratives, locatives, interrogatives, noun phrases, prepositional phrases, and/or relative clauses) (Chapter 8).
\end{styleIIndented}\item \begin{styleIIndented}
Negation only with \textitbf{bukang} ‘\textsc{neg}’ (§13.1.2).
\end{styleIIndented}\item \begin{styleIvI}
In adnominal possessive constructions, common nouns can express the possessor and/or the possessum (Chapter 9).
\end{styleIvI}\end{itemize}

Cross-linguistically, two types of common nouns can be distinguished, count nouns and mass nouns. While a count noun designates “a separate, one of a number of such entities which can be counted”, a mass noun “denotes a quantity or mass of unindividuated material” {(Asher 1994: 5108, 5144)}. Examples of Papuan Malay count and mass nouns, both concrete and abstract, are presented in {Table  ‎5 .1}.


\begin{stylecaption}
\label{bkm:Ref350435184}Table ‎5.\stepcounter{Table}{\theTable}:  Count and mass nouns
\end{stylecaption}

\begin{tabular}{llll}
\lsptoprule

\multicolumn{2}{l}{ Concrete count nouns} & \multicolumn{2}{l}{ Abstract count nouns}\\
\textitbf{ana} & ‘child’ & \textitbf{adat} & ‘tradition’\\
\textitbf{bawang} & ‘onion’ & \textitbf{berkat} & ‘blessing’\\
\textitbf{celana} & ‘trouser’ & \textitbf{dosa} & ‘sin’\\
\textitbf{daung} & ‘leaf’ & \textitbf{jatwal} & ‘schedule’\\
\textitbf{hutang} & ‘forest’ & \textitbf{kwasa} & ‘power’\\
\textitbf{jaring} & ‘net’ & \textitbf{pamali} & ‘taboo’\\
\textitbf{sumur} & ‘well’ & \textitbf{tanggal} & ‘date’\\
\textitbf{tikus} & ‘rat’ & \textitbf{tuju} & ‘goal’\\
\multicolumn{2}{l}{ Concrete mass nouns} & \multicolumn{2}{l}{ Abstract mass nouns}\\
\textitbf{ampas} & ‘waste’ & \textitbf{cinta} & ‘love’\\
\textitbf{busa} & ‘foam’ & \textitbf{baw} & ‘smell’\\
\textitbf{dara} & ‘blood’ & \textitbf{dana} & ‘funds’\\
\textitbf{garam} & ‘salt’ & \textitbf{duka} & ‘grief’\\
\textitbf{minyak} & ‘oil’ & \textitbf{hikmat} & ‘wisdom’\\
\textitbf{nasi} & ‘cooked rice’ & \textitbf{iman} & ‘faith’\\
\textitbf{susu} & ‘milk’ & \textitbf{ongkos} & ‘expenses’\\
\textitbf{te} & ‘tea’ & \textitbf{umur} & ‘age’\\
\lspbottomrule
\end{tabular}

Count nouns can be modified with numerals as in (0) and (0), or with quantifiers as in (0) to (0). The numerals and quantifiers can occur in prehead position, as in (0), (0), or (0), or in post-head position as in (0), (0), or (0). (Concerning the position of adnominal numerals vis-à-vis their head nominal and their semantics, see §5.9 and §8.3.1.)


\begin{styleExampleTitle}
Count nouns\footnote{\\
\\
\\
\\
\\
\\
\\
\\
\\
\\
\\
Documentation: \textitbf{dua} ‘two’ 080919-001-Cv.0022, BR111017-002.003, \textitbf{banyak} ‘many’ 081006-023-CvEx.0007, 081029-004-Cv.0021, \textitbf{sedikit} ‘few’ BR111021.014, BR111021.015.\\
\\
\\
\\
}
\end{styleExampleTitle}

\begin{tabular}{lll}
\lsptoprule
\label{bkm:Ref353627613}
\gll {\textitbf{dua orang}} {‘two people’}\\ %
& two person & \\
\label{bkm:Ref353627623}
\gll {\textitbf{orang dua}} {‘both people’}\\ %
& person two & \\
\label{bkm:Ref353627625}
\gll {\textitbf{banyak orang}} {‘many people’}\\ %
& many person & \\
\label{bkm:Ref353627626}
\gll {\textitbf{orang banyak}} {‘many people’}\\ %
& person many & \\
\label{bkm:Ref353627627}
\gll {\textitbf{sedikit orang}} {‘few people’}\\ %
& few person & \\
\label{bkm:Ref353627628}
\gll {\textitbf{orang sedikit}} {‘few people’}\\ %
& person few & \\
\lspbottomrule
\end{tabular}

Mass nouns can be modified with quantifiers, which always occur in post-head position, as in (0) and (0). That is, the quantifiers cannot occur in pre-head position, as shown with the elicited ungrammatical constructions in (0) and (0). Also, mass nouns cannot co-occur with numerals, neither in pre- nor in post-head position, as shown with the elicited ungrammatical examples in (0) and (0). (As for the position of adnominal quantifiers vis-à-vis their head nominal and the semantics involved, see §5.10 and §8.3.2.)


\begin{styleExampleTitle}
Mass nouns\footnote{\\
\\
\\
\\
\\
\\
\\
\\
\\
\\
\\
Documentation: \textitbf{banyak} ‘many’ BR111021.015, BR111021.017, \textitbf{sedikit} ‘few’ BR111021.016, BR111021.018, \textitbf{dua} ‘two’ BR111021.019, BR111021.020.\\
\\
\\
\\
}
\end{styleExampleTitle}

\begin{tabular}{lll}
\lsptoprule
\label{bkm:Ref353628102}
\gll {\textitbf{sagu banyak}} {‘lots of sago’}\\ %
& sago many & \\
\label{bkm:Ref353628103}
\gll {\textitbf{sagu sedikit}} {‘little sago’}\\ %
& sago few & \\
\label{bkm:Ref353628104}
\gll {\textitbf{*banyak sagu}} {Intended reading: ‘lots of sago’}\\ %
& many sago & \\
\label{bkm:Ref353628105}
\gll {\textitbf{*sedikit sagu}} {Intended reading: ‘little sago’}\\ %
& few sago & \\
\label{bkm:Ref353628106}
\gll {\textitbf{*dua sagu}} {(‘two sago’)}\\ %
& two sago & \\
\label{bkm:Ref353628107}
\gll {\textitbf{*sagu dua}} {(‘two sago’)}\\ %
& sago two & \\
\lspbottomrule
\end{tabular}
\subsection{Proper nouns}
\label{bkm:Ref342375955}
Proper nouns have specific reference in that they “refer to individual entities (or specific groups)” {(Givón 2001: 58)}. Hence, proper nouns are distinct from common nouns, which have general reference. More specifically, proper nouns express the names of specific people and geographical places. In Papuan Malay proper nouns are distinct from common nouns in terms of the following properties:


%\setcounter{itemize}{0}
\begin{itemize}
\item \begin{styleIIndented}
Proper nouns can be modified with the following constituents: monovalent stative verbs, personal pronouns, demonstratives, locatives, interrogatives, numerals, quantifiers, or relative clauses (Chapter 8). Unlike common nouns, they are not readily modified with other nouns, noun phrases, or prepositional phrases.
\end{styleIIndented}\item \begin{styleIIndented}
Proper nouns always occur as bare nouns; reduplicated proper nouns are unattested (§4.1.1.1).
\end{styleIIndented}\item \begin{styleIvI}
Proper nouns typically express the possessor but not the possessum in adnominal possessive constructions (Chapter 9).
\end{styleIvI}\end{itemize}

Some examples of person and place names attested in the corpus are presented in Table  ‎5 .2. Original Papuan Malay names, however, do not exist as such. The person names are very commonly taken from the Bible or originate from European languages. Family or clan names and place names originate from local languages, such as the Papuan language Isirawa (see also §1.4). The examples in Table  ‎5 .2 also illustrate that person names with more than two syllables are most commonly shortened to two-syllable names.


\begin{stylecaption}
\label{bkm:Ref339707430}Table ‎5.\stepcounter{Table}{\theTable}:  Proper nouns: Person and place names
\end{stylecaption}

\tablehead{
\multicolumn{2}{l}{ Male person names} & \multicolumn{2}{l}{ Female person names}\\
 Long form & Short form & Long form & \arraybslash Short form\\
}
\begin{tabular}{llll}
\lsptoprule
\textitbf{Abimelek} & \textitbf{Abi} & \textitbf{Antonia} & \textitbf{Anto}\\
\textitbf{Benyamin} & \textitbf{Beni} & \textitbf{Fransiska} & \textitbf{Siska}\\
\textitbf{Dominggus} & \textitbf{Domi} & \textitbf{Gerice} & \textitbf{Ice}\\
\textitbf{Edwart} & \textitbf{Edo} & \textitbf{Hendrika} & \textitbf{Ika}\\
\textitbf{Hermanus} & \textitbf{Herman} & \textitbf{Isabela} & \textitbf{Ise}\\
\textitbf{Kornelius} & \textitbf{Kori} & \textitbf{Magdalena} & \textitbf{Magda}\\
\textitbf{Lodowik} & \textitbf{Lodo} & \textitbf{Pawlina} & \textitbf{Pawla}\\
\textitbf{Martinus} & \textitbf{Tinus} & \textitbf{Samalina} & \textitbf{Lina}\\
\textitbf{Pontius} & \textitbf{Ponti} & \textitbf{Sarlota} & \textitbf{Ota}\\
\textitbf{Sokarates} & \textitbf{Ates} & \textitbf{Yohana} & \textitbf{Hana}\\
\multicolumn{2}{l}{ Clan and family names} & \multicolumn{2}{l}{ Place names}\\
\textitbf{Aweta} & \textitbf{Manierong} & \textitbf{Arbais} & \textitbf{Mararena}\\
\textitbf{Cawem} & \textitbf{Merne} & \textitbf{Betaf} & \textitbf{Rotea}\\
\textitbf{Catwe} & \textitbf{Sefanya} & \textitbf{Dabe} & \textitbf{Sarmi}\\
\textitbf{Domanser} & \textitbf{Sope} & \textitbf{Karfasia} & \textitbf{Takar}\\
\textitbf{Kaywor} & \textitbf{Yapo} & \textitbf{Liki} & \textitbf{Webro}\\
\lspbottomrule
\end{tabular}

Modification of proper nouns with monovalent stative verbs, personal pronouns, demonstratives, locatives, interrogatives, numerals, quantifiers, and relative clauses is illustrated in (0) to (0), respectively.\footnote{\\
\\
\\
\\
\\
\\
\\
\\
\\
\\
\\
Documentation: verb 081011-024-Cv.0142, personal pronoun 080916-001-CvNP.0003, demonstrative 080917-008-NP.0043, locative 080917-008-NP.0118, interrogative 080922-001a-CvPh.1245, numeral 080922-002-Cv.0052, quantifier 080922-010a-NF.0269, relative clause 080919-006-CvNP.0017.\\
\\
\\
\\
}


\begin{tabular}{lll}
\lsptoprule
\label{bkm:Ref350414638}
\gll {\textitbf{Jayapura besar itu}} {‘that big (city of) Jayapura’}\\ %
& Jayapura be.big. \textsc{d.dist} & \\
(\stepcounter{}{\the}) & \textitbf{Iskia de} & ‘Iskia’ (Lit. ‘he Iskia’)\\
& \textsc{3sg} & \\
(\stepcounter{}{\the}) & \textitbf{Sarmi itu} & ‘that (city of) Sarmi’\\
& Sarmi \textsc{d.dist} & \\
(\stepcounter{}{\the}) & \textitbf{Paynete situ} & ‘Paynete there’\\
& Paynete \textsc{l.med} & \\
(\stepcounter{}{\the}) & \textitbf{Muay mana?} & ‘which Muay?’\\
& Muay where? & \\
(\stepcounter{}{\the}) & \textitbf{Suebu satu ni} & ‘this certain (member of the) Suebu (family)\\
& Suebu one \textsc{d.prox} & \\
(\stepcounter{}{\the}) & \textitbf{Sope banyak} & ‘many Sope (family members)’\\
& Sope many & \\
\label{bkm:Ref350414641}
\gll {\textitbf{Wili yang tinggal}} {‘Wili who’ll stay’}\\ %
& Wili \textsc{rel} stay & \\
\lspbottomrule
\end{tabular}

When addressing interlocutors or talking about others, speakers very commonly introduce person names with common nouns that indicate kinship relations or are used as honorifics, as shown in Table  ‎5 .3. Likewise, place names are often preceded by common nouns denoting geographical entities.


\begin{stylecaption}
\label{bkm:Ref339707474}Table ‎5.\stepcounter{Table}{\theTable}:  Introduced person and place names\footnote{\\
\\
\\
\\
\\
\\
\\
\\
\\
\\
\\
Documentation: 080922-001a-CvPh.1096, 081011-024-Cv.0123, 081014-005-Cv.0002, 081014-014-CvNP.0084.\\
Documentation: 080922-002-Cv.0049, 080917-008-NP.0018, 081025-008-Cv.0008, 080917-008-NP.0126.\\
\\
\\
}
\end{stylecaption}

\begin{tabular}{ll}
\lsptoprule

\multicolumn{2}{l}{ Introduced person names}\\
\textitbf{ade Aris} & ‘younger sibling Aris’\\
\textitbf{mama Sance} & ‘mama Sance’\\
\textitbf{bapa-tua Fredi} & ‘uncle Fredi’\\
\textitbf{tete Daut} & ‘grandfather Daut’\\
\textitbf{mace Agustina} & ‘Ms. Agustina’\\
\textitbf{pace Alpeus} & ‘Mr. Alpeus’\\
\multicolumn{2}{l}{ Introduced places names}\\
\textitbf{kampung Harapang} & ‘Harapang village’\\
\textitbf{kota Sarmi} & ‘Sarmi’ city\\
\textitbf{kali Biri} & ‘Biri’ river\\
\textitbf{pulow Sarmi} & ‘Sarmi island’\\
\lspbottomrule
\end{tabular}
\subsection{Location nouns}
\label{bkm:Ref322531172}
Location nouns, or locative nouns, designate locations rather than physical objects. The Papuan Malay location nouns are given in Table  ‎5 .4, together with their token frequencies in the corpus.


\begin{stylecaption}
\label{bkm:Ref322624897}Table ‎5.\stepcounter{Table}{\theTable}:  Papuan Malay location nouns
\end{stylecaption}

\tablehead{
 Item & Gloss & \arraybslash \# tokens\\
}
\begin{tabular}{lll}
\lsptoprule
\textitbf{atas} & ‘top’ & \raggedleft\arraybslash 146\\
\textitbf{bawa} & ‘bottom’ & \raggedleft\arraybslash 116\\
\textitbf{blakang} & ‘backside’ & \raggedleft\arraybslash 92\\
\textitbf{dalam} & ‘inside’ & \raggedleft\arraybslash 230\\
\textitbf{depang} & ‘front’ & \raggedleft\arraybslash 102\\
\textitbf{luar} & ‘outside’ & \raggedleft\arraybslash 79\\
\textitbf{pinggir} & ‘border’ & \raggedleft\arraybslash 23\\
\textitbf{samping} & ‘side’ & \raggedleft\arraybslash 24\\
\textitbf{sebla} & ‘side’ & \raggedleft\arraybslash 110\\
\textitbf{sekitar} & ‘vicinity’ & \raggedleft\arraybslash 17\\
\textitbf{tenga} & ‘middle’ & \raggedleft\arraybslash 42\\
\lspbottomrule
\end{tabular}

Location nouns are distinct from common nouns (§5.2.1) in terms of the following properties:


%\setcounter{itemize}{0}
\begin{itemize}
\item \begin{styleIIndented}
In their nominal uses, location nouns (a) only occur in prepositional phrases, (b) can be modified with nouns, demonstratives, or locatives, but with no other constituents, and (c) do not take the possessor or possessum slots in adnominal possessive constructions.\footnote{\\
\\
\\
\\
\\
\\
\\
\\
\\
\\
\\
\\
The exception is \textitbf{blakang} ‘backside’. It also has the body part meaning ‘back’. As such it can denote the possessum in an adnominal possessive construction such as \textitbf{sa pu blakang} ‘\textsc{1sg} \textsc{poss} backside’ ‘my back’ \textstyleExampleSource{[081015-005-NP.0032]}.\\
\\
\\
}
\end{styleIIndented}\item \begin{styleIvI}
In their adnominal uses, location nouns are juxtaposed to common nouns only; that is, unlike common nouns, they cannot be stacked.
\end{styleIvI}\end{itemize}

Location nouns are distinct from direction nouns (§5.2.4) in that they can be modified with juxtaposed adnominal nouns, while direction nouns cannot be modified in this way.



The nominal uses of the location nouns are discussed in §5.2.3.1 and their adnominal uses in §5.2.3.2.
\end{styleBodyvxvafter}

\paragraph[Nominal uses]{Nominal uses}
\label{bkm:Ref322520426}
In their nominal uses, the nouns always occur inside prepositional phrases and are typically modified with a juxtaposed adnominal noun such that ‘\textsc{prep} \textsc{n.loc} \textsc{n}’. Semantically, \textsc{n.loc} \textsc{n} noun phrases are characterized by the subordination of the adnominal noun in \textsc{N2} position under the head nominal location noun in N1 position (see also §8.2.2).



Generally speaking, the main function of location nouns is to specify the spatial relationship between a figure and the ground {(Levinson and Wilkins 2006: 3)}, with the ground being encoded by the juxtaposed adnominal noun. The same applies to the Papuan Malay location nouns, in that they more fully specify the spatial relationship between figure and ground than is achieved by a bare preposition that introduces the ground. This is illustrated with the contrastive examples in (0) to (0) and in (0) and (0).
\end{styleBodyvvafter}

\begin{styleExampleTitle}
‘\textsc{prep} \textsc{n.loc} \textsc{n}’ versus ‘\textsc{prep} \textsc{n}’ prepositional phrases
\end{styleExampleTitle}

\begin{tabular}{llllll}
\lsptoprule
\label{bkm:Ref368127620}
\gll {\textitbf{di}} {\textitbf{atas}} {\textitbf{pohong}} {‘in the top of the tree’} {\textstyleExampleSource{[081006-023-CvEx.0061]}}\\ %
& at & top & tree &  & \\
(\stepcounter{}{\the}) & \textitbf{di} & \textitbf{bawa} & \textitbf{pohong} & ‘under the tree’ & \textstyleExampleSource{[081109-002-JR.0002]}\\
& at & bottom & tree &  & \\
\label{bkm:Ref368127622}
\gll {\textitbf{di}} {} {\textitbf{pohong}} {‘in the tree} {\textstyleExampleSource{[081006-023-CvEx.0080]}}\\ %
& at &  & tree &  & \\
\label{bkm:Ref368127623}
\gll {\textitbf{di}} {\textitbf{pinggir}} {\textitbf{kali}} {‘alongside the river’} {\textstyleExampleSource{[081011-001-Cv.0167]}}\\ %
& at & border & river &  & \\
\label{bkm:Ref368127624}
\gll {\textitbf{di}} {} {\textitbf{kali}} {‘in the river’} {\textstyleExampleSource{[080919-004-NP.0030]}}\\ %
& at &  & river &  & \\
\lspbottomrule
\end{tabular}

More examples illustrating the nominal uses of locations nouns in prepositional phrases are given in (0) to (0).


\begin{styleExampleTitle}
Location nouns with nominal modifier\footnote{\\
\\
\\
\\
\\
\\
\\
\\
\\
\\
\\
\\
Documentation: \textitbf{atas} ‘top’ 081025-008-Cv.0162, \textitbf{bawa} ‘bottom’ 081025-009b-Cv.0018, \textitbf{blakang} ‘backside’ 081106-001-Ex.0002, \textitbf{dalam} ‘inside’ 081025-006-Cv.0039, \textitbf{depang} ‘front’ 081115-001a-Cv.0139, \textitbf{luar} ‘outside’ 081025-003-Cv.0159, \textitbf{pinggir} ‘border’ 080918-001-CvNP.0060, \textitbf{samping} ‘side’ 081014-014-CvNP.0046, \textitbf{sebla} ‘side’ 081109-001-Cv.0026, \textitbf{sekitar} ‘vicinity’ 081011-024-Cv.0140, \textitbf{tenga} ‘middle’ 080927-009-CvNP.0037.\\
\\
\\
}
\end{styleExampleTitle}

\begin{tabular}{lllll}
\lsptoprule
\label{bkm:Ref339711305}
\gll {\textitbf{atas}} {‘top’} {\textitbf{dari atas kursi}} {‘from the top of the chair’}\\ %
&  &  & from top chair & \\
(\stepcounter{}{\the}) & \textitbf{bawa} & ‘bottom’ & \textitbf{di bawa meja} & ‘below the table’\\
&  &  & at bottom table & \\
(\stepcounter{}{\the}) & \textitbf{blakang} & ‘backside’ & \textitbf{dengang blakang kapak} & ‘with the backside of the axe’\\
&  &  & with backside axe & \\
(\stepcounter{}{\the}) & \textitbf{dalam} & ‘inside’ & \textitbf{di dalam kamar} & ‘inside the room’\\
&  &  & at inside room & \\
(\stepcounter{}{\the}) & \textitbf{depang} & ‘front’ & \textitbf{di depang greja tu} & ‘in front of that church’\\
&  &  & at front church & \\
(\stepcounter{}{\the}) & \textitbf{luar} & ‘outside’ & \textitbf{ke luar negri} & ‘abroad’\\
&  &  & to outside country & \\
(\stepcounter{}{\the}) & \textitbf{pinggir} & ‘border’ & \textitbf{di pinggir jalang} & ‘alongside the road’\\
&  &  & at border road & \\
(\stepcounter{}{\the}) & \textitbf{samping} & ‘side’ & \textitbf{di samping ruma} & ‘beside the house’\\
&  &  & at side house & \\
(\stepcounter{}{\the}) & \textitbf{sebla} & ‘side’ & \textitbf{ke sebla darat} & ‘landwards’\\
&  &  & to side land & \\
(\stepcounter{}{\the}) & \textitbf{sekitar} & ‘vicinity’ & \textitbf{di sekitar Pante-Barat} & ‘in the vicinity of Pante-Barat’\\
&  &  & at vicinity Pante-Barat & \\
\label{bkm:Ref339711308}
\gll {\textitbf{tenga}} {‘middle’} {\textitbf{di tenga hutang}} {‘in the middle of the forest’}\\ %
&  &  & at middle forest & \\
\lspbottomrule
\end{tabular}

In the examples in (0) to (0), the ground, encoded by the adnominal noun in \textsc{N2} position, is mentioned overtly. If the ground is understood from the context, though, the adnominal noun denoting it can be omitted and the location noun is used as an independent nominal as in (0) to (0). In (0) the ground is understood from the speech situation: it is the house where the speech acts occurs. In (0) to (0) the ground is understood from the discourse: it is \textitbf{kitorang tiga} ‘we three’ in (0), \textitbf{sumur} ‘well’ in (0), and \textitbf{bandara} ‘airport’ in (0).


\begin{styleExampleTitle}
Location nouns with omitted nominal modifier
\end{styleExampleTitle}

\begin{tabular}{llllllllllll}
\lsptoprule
\label{bkm:Ref339637255}
\gll {tida} {\multicolumn{2}{l}{usa}} {\multicolumn{2}{l}{kamu}} {\multicolumn{2}{l}{duduk}} {\bluebold{di}} {\multicolumn{3}{l}{\bluebold{depang},}}\\ %
& \textsc{neg} & \multicolumn{2}{l}{need.to} & \multicolumn{2}{l}{\textsc{2pl}} & \multicolumn{2}{l}{sit} & at & \multicolumn{3}{l}{front}\\
& \multicolumn{2}{l}{ana} & \multicolumn{2}{l}{prempuang} & \multicolumn{2}{l}{itu} & \multicolumn{3}{l}{duduk} & \bluebold{di} & \bluebold{blakang}\\
& \multicolumn{2}{l}{child} & \multicolumn{2}{l}{woman} & \multicolumn{2}{l}{\textsc{d.dist}} & \multicolumn{3}{l}{sit} & at & backside\\
\lspbottomrule
\end{tabular}
\ea
\glt 
‘it’s not necessary that you sit \bluebold{in front (of the house)}, as for girls, (they) sit \bluebold{in the back (of the house)}’ \textstyleExampleSource{[081115-001a-Cv.0316]}
\z

\begin{tabular}{lllllllllll}
\lsptoprule
\label{bkm:Ref339637257}
\gll {kitorang} {tiga} {…} {naik} {di} {motor} {…} {Martina} {\bluebold{di}} {\bluebold{tenga}}\\ %
& \textsc{1pl} & three &  & ascend & at & motorbike &  & Martina & at & middle\\
\lspbottomrule
\end{tabular}
\ea
\glt 
‘we three … got onto the motorbike … Martina was \bluebold{in the middle}’ \textstyleExampleSource{[081015-005-NP.0020]}
\z

\begin{tabular}{lllllllllll}
\lsptoprule
\label{bkm:Ref339637258}
\gll {sumur} {itu} {masi} {ada} {…} {\bluebold{di}} {\bluebold{dalam}} {\bluebold{tu}} {ada} {senjata}\\ %
& well & \textsc{d.dist} & still & exist &  & at & inside & \textsc{d.dist} & exist & rifle\\
\lspbottomrule
\end{tabular}
\ea
\glt 
‘that well still exists … \bluebold{inside there} are rifles’ \textstyleExampleSource{[080922-010a-CvNF.0120-0121]}
\z

\begin{tabular}{lllllllllllllllll}
\lsptoprule
\label{bkm:Ref339637259}
\gll {\multicolumn{2}{l}{pas}} {\multicolumn{2}{l}{turung}} {\multicolumn{2}{l}{bandara}} {\multicolumn{2}{l}{Sentani}} {\multicolumn{3}{l}{pas}} {\multicolumn{2}{l}{de}} {\multicolumn{2}{l}{ketemu}} {dengang}\\ %
& \multicolumn{2}{l}{precisely} & \multicolumn{2}{l}{descend} & \multicolumn{2}{l}{airport} & \multicolumn{2}{l}{Sentani} & \multicolumn{3}{l}{precisely} & \multicolumn{2}{l}{\textsc{3sg}} & \multicolumn{2}{l}{meet} & with\\
\multicolumn{2}{l}{} & \multicolumn{2}{l}{Wamena} & \multicolumn{2}{l}{dorang,} & \multicolumn{2}{l}{pas} & \multicolumn{2}{l}{Wamena} & dong & \multicolumn{2}{l}{\bluebold{di}} & \multicolumn{2}{l}{\bluebold{pinggir}} & \multicolumn{2}{l}{\bluebold{situ}}\\
\multicolumn{2}{l}{} & \multicolumn{2}{l}{Wamena} & \multicolumn{2}{l}{\textsc{3pl}} & \multicolumn{2}{l}{precisely} & \multicolumn{2}{l}{Wamena} & \textsc{3pl} & \multicolumn{2}{l}{at} & \multicolumn{2}{l}{border} & \multicolumn{2}{l}{\textsc{l.med}}\\
\lspbottomrule
\end{tabular}
\ea
\glt 
‘the moment (he) landed (at) Sentani airport, he met the Wamena people, right then the Wamena people were (sitting) \bluebold{alongside (the airstrip)} \bluebold{there}’ \textstyleExampleSource{[081109-009-JR.0003]}
\z


The examples in (0) and (0) also illustrate that an independently used location noun can be modified with a demonstrative or a locative, respectively.



As shown so far, location nouns more fully specify the spatial relationship between a figure and the ground than is achieved by a bare preposition that introduces the ground. If the specific spatial relationship can be deduced from the context, though, the location noun can be omitted as illustrated with elided \textitbf{atas} ‘top’ in (0) and \textitbf{dalam} ‘inside’ in (0).
\end{styleBodyvvafter}

\begin{styleExampleTitle}
Omitted location nouns
\end{styleExampleTitle}

\begin{tabular}{llllllllll}
\lsptoprule
\label{bkm:Ref339637260}
\gll {de} {kas} {turung} {mama} {Petrus} {\bluebold{dari}} {\bluebold{atas}} {\bluebold{kursi}} {to?}\\ %
& \textsc{3sg} & give & descend & mother & Petrus & from & top & chair & right?\\
\lspbottomrule
\end{tabular}
\ea
\glt 
‘he (the evil spirit) threw mother Petrus \bluebold{from (the top of her) chair}, right?’ \textstyleExampleSource{[081025-008-Cv.0158]}
\z

\begin{tabular}{llllllll}
\lsptoprule
\label{bkm:Ref339637262}
\gll {dong} {mandi} {\bluebold{di}} {\bluebold{dalam}} {\bluebold{kamar}} {\bluebold{mandi}} {\bluebold{sana}}\\ %
& \textsc{3pl} & bathe & at & inside & room & bathe & \textsc{l.dist}\\
\lspbottomrule
\end{tabular}
\ea
\glt
‘they were bathing \bluebold{in(side of) the bathroom over there}’ \textstyleExampleSource{[081109-001-Cv.0081]}
\end{styleFreeTranslEngxvpt}

\paragraph[Adnominal uses]{Adnominal uses}
\label{bkm:Ref322520428}
In their adnominal uses, the location nouns are juxtaposed to common nouns or, although much less frequently, to common nouns with juxtaposed adnominal personal pronouns, such that ‘\textsc{n} (\textsc{pro}) \textsc{n.loc}’. In their adnominal uses, they signal locational relations. Overall, though, the adnominal uses of location nouns are marginal: of a total of 981 tokens, only 35 (4\%) have adnominal uses, whereas 946 have nominal uses (96\%).



In designating locational relations, the location nouns have restrictive function. That is, they signal that the referent encoded by the head nominal is precisely the one situated in the spatial location designated by the location noun. Thereby, the location noun aids the hearer in the identification of the referent, as in \textitbf{jalang atas} ‘upper road’ in (0), \textitbf{rem blakang} ‘rear brakes’ in (0), or \textitbf{tetangga dong sebla} ‘the neighbors next door’ in (0). The locational relation can also be figurative as in \textitbf{generasi bawa} ‘next generation’ in (0), or in \textitbf{dunia luar} ‘outside world’ in (0), or temporal as in \textitbf{bulang depang} ‘next month’ in (0). Adnominal uses for \textitbf{sekitar} ‘vicinity’ are unattested in the corpus.
\end{styleBodyvvafter}

\begin{styleExampleTitle}
Locational relations: Spatial and figurative\footnote{\\
\\
\\
\\
\\
\\
\\
\\
\\
\\
\\
\\
Documentation: \textitbf{atas} ‘top’ BR111031-001.005, \textitbf{blakang} ‘backside’ 081022-002-Cv.0013, \textitbf{dalam} ‘inside’ 081025-006-Cv.0023, \textitbf{luar} ‘outside’ 081029-002-Cv.0033, \textitbf{pinggir} ‘edge’ 080923-010-CvNP.0010, \textitbf{samping} ‘side’ BR111031-001.004, \textitbf{sebla} ‘side’ 081006-035-CvEx.0067, \textitbf{tenga} ‘middle’ 081014-006-Pr.0037, \textitbf{bawa} ‘bottom’ 081011-024-Cv.0148, \textitbf{depang} ‘front’ 080921-011-Cv.0012.\\
\\
\\
}
\end{styleExampleTitle}

\begin{tabular}{lllll}
\lsptoprule
\label{bkm:Ref339711499}
\gll {\textitbf{atas}} {‘top’} {\textbf{\textit{jalang atas}}} {‘the upper road’ (Lit. ‘the road on top’)}\\ %
&  &  & road top & \\
\label{bkm:Ref339711501}
\gll {\textitbf{blakang}} {‘backside’} {\textitbf{rem blakang}} {‘rear brakes’}\\ %
&  &  & brake backside & \\
(\stepcounter{}{\the}) & \textitbf{dalam} & ‘inside’ & \textitbf{kolor dalam} & ‘undershorts’\\
&  &  & shorts inside & \\
\label{bkm:Ref339711504}
\gll {\textitbf{luar}} {‘outside’} {\textitbf{dunia luar}} {‘outside world’}\\ %
&  &  & world outside & \\
(\stepcounter{}{\the}) & \textitbf{pinggir} & ‘border’ & \textitbf{tana pinggir} & ‘the ground along the side’\\
&  &  & ground border & \\
(\stepcounter{}{\the}) & \textitbf{samping} & ‘side’ & \textitbf{sak samping} & ‘side pocket\\
&  &  & bag side & \\
\label{bkm:Ref339711508}
\gll {\textitbf{sebla}} {‘side’} {\textitbf{tetangga dong sebla}} {‘the neighbors next door’}\\ %
&  &  & neighbor \textsc{3pl} side & \\
(\stepcounter{}{\the}) & \textitbf{tenga} & ‘middle’ & \textitbf{kolam tenga} & ‘the pond in the middle’\\
&  &  & big.hole middle & \\
\label{bkm:Ref339711510}
\gll {\textitbf{bawa}} {‘bottom’} {\textitbf{generasi bawa}} {‘next generation’ (Lit. ‘gen\-eration at the bottom’)}\\ %
&  &  & generation bottom & \\
\label{bkm:Ref339711511}
\gll {\textitbf{depang}} {‘front’} {\textitbf{bulang depang}} {‘next month’ (Lit. ‘month in front’)}\\ %
&  &  & month front & \\
\lspbottomrule
\end{tabular}
\subsection{Direction nouns}
\label{bkm:Ref373936913}
Direction nouns express cardinal directions and relative directions. The former designate the \textstylest{four principal compass points, while the latter express left-right orientation. }The Papuan Malay direction nouns are presented in Table  ‎5 .5, together with their token frequencies in the corpus (given their low token frequencies, most examples in this section are elicited).


\begin{stylecaption}
\label{bkm:Ref401392176}Table ‎5.\stepcounter{Table}{\theTable}:  Papuan Malay cardinal and relative directions
\end{stylecaption}

\tablehead{
 Item & Gloss & \arraybslash \# tokens\\
}
\begin{tabular}{lll}
\lsptoprule
\textitbf{utara} & ‘north’ & \raggedleft\arraybslash {}-{}-{}-\\
\textitbf{slatang} & ‘south’ & \raggedleft\arraybslash {}-{}-{}-\\
\textitbf{barat} & ‘west’ & \raggedleft\arraybslash 10\\
\textitbf{timur} & ‘east’ & \raggedleft\arraybslash 5\\
\textitbf{kiri} & ‘left’ & \raggedleft\arraybslash 1\\
\textitbf{kanang} & ‘right & \raggedleft\arraybslash 2\\
\lspbottomrule
\end{tabular}

Direction nouns have the following distributional properties:


%\setcounter{itemize}{0}
\begin{itemize}
\item \begin{styleIIndented}
Direction nouns occur in prepositional phrases as independent heads of the noun phrase within the prepositional phrase; they are unattested as head nominals in unembedded noun phrases.
\end{styleIIndented}\item \begin{styleIIndented}
Direction nouns have adnominal uses; that is, they occur in noun phrases with a preceding noun as nominal head.
\end{styleIIndented}\item \begin{styleIvI}
Direction nouns can be modified with adnominally used demonstratives or locatives.
\end{styleIvI}\end{itemize}

Direction nouns are distinct from common nouns (§5.2.1) and location nouns (§5.2.3) in terms of the following properties:


%\setcounter{itemize}{0}
\begin{itemize}
\item \begin{styleIIndented}
Contrasting with common nouns, direction nouns (a) are unattested as heads of unembedded noun phrases, (b) are only modified with demonstratives and locatives, and (c) are unattested in adnominal possessive constructions, neither as the possessor nor as the possessum.
\end{styleIIndented}\item \begin{styleIvI}
Contrasting with location nouns, direction nouns with juxtaposed adnominal nouns are unattested when employed as nominals in prepositional phrases.
\end{styleIvI}\end{itemize}

Direction nouns typically occur as complements in prepositional phrases, as shown with the four cardinal directions in (0) to (0) and the two relative directions in (0) and (0). Direction nouns can be modified with demonstratives as in \textitbf{utara ini} ‘this north’ in (0) or \textitbf{kiri ini} ‘this left’ in (0), or with locatives as in \textitbf{slatang sana} ‘south over there’ in (0) or \textitbf{kanang sana} ‘right over there’ in (0).


\begin{styleExampleTitle}
Direction nouns as complements in prepositional phrases
\end{styleExampleTitle}

\begin{tabular}{lllllllll}
\lsptoprule
\label{bkm:Ref350444849}
\gll {sa} {pu} {prahu} {hanyut} {sampe} {ke} {\bluebold{utara}} {\bluebold{ini}}\\ %
& \textsc{1sg} & \textsc{poss} & boat & drift & reach & to & north & \textsc{d.prox}\\
\lspbottomrule
\end{tabular}
\ea
\glt 
‘my boat drifted up to the \bluebold{north here}’ \textstyleExampleSource{[Elicited BR130103.018]}
\z

\begin{tabular}{lllllllll}
\lsptoprule
\label{bkm:Ref350441454}
\gll {pohong} {gaharu} {tu} {paling} {banyak} {di} {\bluebold{slatang}} {\bluebold{sana}}\\ %
& tree & agarwood & \textsc{d.dist} & most & many & at & south & \textsc{l.dist}\\
\lspbottomrule
\end{tabular}
\ea
\glt 
‘agarwood trees are most common in the \bluebold{south over there}’ \textstyleExampleSource{[Elicited BR130103.017]}
\z

\begin{tabular}{llllllll}
\lsptoprule
(\stepcounter{}{\the}) & de & blang, & a & sa & datang & dari & \bluebold{barat}\\
& \textsc{3sg} & say & ah! & \textsc{1sg} & come & from & west\\
\lspbottomrule
\end{tabular}
\ea
\glt 
‘he said, ‘ah, I come from the \bluebold{west}’’ \textstyleExampleSource{[080922-010a-CvNF.0237]}
\z

\begin{tabular}{llllllll}
\lsptoprule
\label{bkm:Ref350441456}
\gll {pesawat} {ini} {de} {terbang} {ke} {\bluebold{timur}} {dulu}\\ %
& airplane & \textsc{d.prox} & \textsc{3sg} & fly & to & east & first\\
\lspbottomrule
\end{tabular}
\ea
\glt 
‘this plane it flies to the \bluebold{east} first’ \textstyleExampleSource{[Elicited BR130103.014]}
\z

\begin{tabular}{llllllllllll}
\lsptoprule
\label{bkm:Ref350438739}
\gll {\multicolumn{2}{l}{pesawat}} {\multicolumn{2}{l}{de}} {\multicolumn{2}{l}{terbang}} {dari} {\bluebold{kiri}} {\bluebold{ini},} {baru} {lewat}\\ %
& \multicolumn{2}{l}{airplane} & \multicolumn{2}{l}{\textsc{3sg}} & \multicolumn{2}{l}{fly} & from & left & \textsc{d.prox} & and.then & pass.by\\
& sana & \multicolumn{2}{l}{trus} & \multicolumn{2}{l}{ke} & \multicolumn{6}{l}{Wamena}\\
& \textsc{l.dist} & \multicolumn{2}{l}{next} & \multicolumn{2}{l}{to} & \multicolumn{6}{l}{Wamena}\\
\lspbottomrule
\end{tabular}
\ea
\glt 
‘the plane flies from the \bluebold{left here} and passes by over there (and) and then (it flies on) to Wamena’ \textstyleExampleSource{[Elicited BR130103.022]}
\z

\begin{tabular}{lllllllll}
\lsptoprule
\label{bkm:Ref350438741}
\gll {\multicolumn{2}{l}{ko}} {jalang} {\multicolumn{2}{l}{trus,}} {baru} {ko} {putar}\\ %
& \multicolumn{2}{l}{\textsc{2sg}} & walk & \multicolumn{2}{l}{be.continuous} & and.then & \textsc{2sg} & turn.around\\
& ke & \multicolumn{3}{l}{\bluebold{kanang}} & \multicolumn{4}{l}{\bluebold{sana}}\\
& to & \multicolumn{3}{l}{right} & \multicolumn{4}{l}{\textsc{l.dist}}\\
\lspbottomrule
\end{tabular}
\ea
\glt 
‘you walk on, only then you turn to the \bluebold{right over there}’ \textstyleExampleSource{[Elicited BR130103.005]}
\z


In (0) and (0) the preposition is obligatory. With motion verbs that also express direction, however, the allative preposition \textitbf{ke} ‘to’ may also be omitted. This is illustrated in (0) with the motion verb \textitbf{belok} ‘turn’. (For details on the elision of prepositions encoding location, see §10.1.5.)


\begin{styleExampleTitle}
Elision of the preposition
\end{styleExampleTitle}

\begin{tabular}{lllllllllllllll}
\lsptoprule
\label{bkm:Ref350438844}
\gll {di} {\multicolumn{2}{l}{jembatang}} {\multicolumn{2}{l}{depang}} {\multicolumn{2}{l}{ko}} {\multicolumn{2}{l}{belok}} {Ø} {\bluebold{kanang}} {trus} {di} {jembatang}\\ %
& at & \multicolumn{2}{l}{bridge} & \multicolumn{2}{l}{front} & \multicolumn{2}{l}{\textsc{2sg}} & \multicolumn{2}{l}{turn} &  & right & next & at & bridge\\
& \multicolumn{2}{l}{depang} & lagi & ko & \multicolumn{2}{l}{belok} & \multicolumn{2}{l}{Ø} & \multicolumn{6}{l}{\bluebold{kiri}}\\
& \multicolumn{2}{l}{front} & again & \textsc{2sg} & \multicolumn{2}{l}{turn} & \multicolumn{2}{l}{} & \multicolumn{6}{l}{left}\\
\lspbottomrule
\end{tabular}
\ea
\glt 
‘at the bridge ahead you turn \bluebold{right}, and then at the next bridge you turn \bluebold{left}’ \textstyleExampleSource{[Elicited BR130103.002]}
\z


In their adnominal uses, the direction nouns are juxtaposed to a head nominal. Semantically, these noun phrases designate ‘subtype-of’ relations as in \textitbf{bagiang barat} ‘western part’ and \textitbf{bagiang timur} ‘eastern part’ in (0), or they denote locational relations as in \textitbf{sebla kiri} ‘left side’ in (0), or in \textitbf{tangang kanang} ‘right hand/arm’ in (0).


\begin{styleExampleTitle}
Adnominal uses of direction nouns
\end{styleExampleTitle}

\begin{tabular}{llllllllllllll}
\lsptoprule
\label{bkm:Ref350441457}
\gll {kalo} {\multicolumn{3}{l}{\bluebold{bagiang}}} {\multicolumn{2}{l}{\bluebold{barat}}} {\multicolumn{2}{l}{itu}} {\multicolumn{2}{l}{kasiang}} {prempuang} {tokok} {prempuang}\\ %
& if & \multicolumn{3}{l}{part} & \multicolumn{2}{l}{west} & \multicolumn{2}{l}{\textsc{d.dist}} & \multicolumn{2}{l}{pity} & woman & tap & woman\\
& \multicolumn{2}{l}{ramas} & tapi & \multicolumn{2}{l}{kalo} & \multicolumn{2}{l}{\bluebold{bagiang}} & \multicolumn{2}{l}{\bluebold{timur}} & \multicolumn{4}{l}{tida}\\
& \multicolumn{2}{l}{press} & but & \multicolumn{2}{l}{if} & \multicolumn{2}{l}{part} & \multicolumn{2}{l}{east} & \multicolumn{4}{l}{\textsc{neg}}\\
\lspbottomrule
\end{tabular}
\ea
\glt 
[About regional differences within the regency:] ‘as for the \bluebold{western part} there, (it’s a) pity, the women tap (and) the women press (the sagu) but as for the \bluebold{eastern part} (it’s) not (like that)’ \textstyleExampleSource{[081014-007-CvEx.0025-0026]}
\z

\begin{tabular}{llllllll}
\lsptoprule
\label{bkm:Ref350438845}
\gll {lapangang} {bola} {kaki} {ada} {di} {\bluebold{sebla}} {\bluebold{kiri}}\\ %
& field & ball & foot & exist & at & side & left\\
\lspbottomrule
\end{tabular}
\ea
\glt 
‘the football field is on the \bluebold{left side}’ \textstyleExampleSource{[Elicited BR130103.011]}
\z

\begin{tabular}{llllllllll}
\lsptoprule
\label{bkm:Ref350438846}
\gll {tulang} {yang} {\bluebold{tangang}} {\bluebold{kanang}} {ini} {su} {kluar} {ke} {samping}\\ %
& bone & \textsc{rel} & hand & right & \textsc{d.prox} & already & go.out & to & side\\
\lspbottomrule
\end{tabular}
\ea
\glt
[About an accident:] ‘the bone of the \bluebold{right arm} here already stuck out sideways’ \textstyleExampleSource{[081108-003-JR.0006]}
\end{styleFreeTranslEngxvpt}

\subsection{Time-denoting nouns}
\label{bkm:Ref351641496}
The label ‘time-denoting nouns’ refers to nouns which denote time units (§5.2.5.1), the periods of the day (§5.2.5.2), the days of the week and months of the year (§5.2.5.3), and relative time (§5.2.5.4). Time-denoting nouns have the same syntactic properties as common nouns (for details see §5.2.1.)
\end{styleBodyxvafter}

\paragraph[Time units]{Time units}
\label{bkm:Ref350878933}
Table  ‎5 .6 lists the different time-denoting nouns that divide a year into smaller units.


\begin{stylecaption}
\label{bkm:Ref322686912}Table ‎5.\stepcounter{Table}{\theTable}:  Time units
\end{stylecaption}

\tablehead{
 Item & Gloss & Item & \arraybslash Gloss\\
}
\begin{tabular}{llll}
\lsptoprule
\textitbf{titik} & ‘second’ & \textitbf{minggu} & ‘week’\\
\textitbf{minit} & ‘minute’ & \textitbf{bulang} & ‘month’\\
\textitbf{jam} & ‘hour’ & \textitbf{taung} & ‘year’\\
\textitbf{hari} & ‘day’ &  & \\
\lspbottomrule
\end{tabular}

The time units listed in Table  ‎5 .6 are count nouns that can be modified with numerals or quantifiers, as illustrated in (0) to (0). In addition to designating a time unit, \textitbf{minggu} ‘week’ also denotes a day of the week, namely ‘Sunday’ (see Table  ‎5 .8).


\begin{tabular}{lllllllll}
\lsptoprule
\label{bkm:Ref350789850}
\gll {bapa} {bilang} {begini,} {tunggu} {\bluebold{lima}} {\bluebold{blas}} {\bluebold{minit}} {to?}\\ %
& father & say & like.this & wait & five & teens & minute & right?\\
\lspbottomrule
\end{tabular}
\ea
\glt 
‘father said like this, ‘wait \bluebold{fifteen minutes}, right?!’’ \textstyleExampleSource{[081025-006-Cv.0173]}
\z

\begin{tabular}{lllllllllll}
\lsptoprule
(\stepcounter{}{\the}) & jadi & baru & \bluebold{sembilang} & \bluebold{bulang} & sa & pi & layani & di & greja & itu\\
& so & be.new & nine & month & \textsc{1sg} & go & serve & at & church & \textsc{d.dist}\\
\lspbottomrule
\end{tabular}
\ea
\glt 
‘so it’s just been \bluebold{nine months} (that) I’ve been serving in that church’ \textstyleExampleSource{[080927-006-CvNP.0010]}
\z

\begin{tabular}{lllll}
\lsptoprule
\label{bkm:Ref350789852}
\gll {\bluebold{setiap}} {\bluebold{hari}} {dong} {latiang}\\ %
& every & day & \textsc{pl} & practice\\
\lspbottomrule
\end{tabular}
\ea
\glt
‘\bluebold{every day} they practice’ \textstyleExampleSource{[081109-006-JR.0001]}
\end{styleFreeTranslEngxvpt}

\paragraph[Periods of the day]{Periods of the day}
\label{bkm:Ref350878934}
Table  ‎5 .7 presents the time-denoting nouns for the four periods of the day. More specifically, \textitbf{pagi} ‘morning’ designates the period from just after midnight until about eleven o’clock, while \textitbf{siang} ‘midday’ refers to the time from about eleven o’clock until about fourteen hours. The next period, \textitbf{sore} ‘afternoon’, lasts until about eighteen hours when darkness sets in, while \textitbf{malam} ‘night’ denotes nighttime.


\begin{stylecaption}
\label{bkm:Ref350790221}Table ‎5.\stepcounter{Table}{\theTable}:  Periods of the day
\end{stylecaption}

\tablehead{
 Item & Gloss & Item & \arraybslash Gloss\\
}
\begin{tabular}{llll}
\lsptoprule
\textitbf{pagi} & ‘morning’ & \textitbf{sore} & ‘afternoon’\\
\textitbf{siang} & ‘midday’ & \textitbf{malam} & ‘night’\\
\lspbottomrule
\end{tabular}

The four periods-of-the-day expressions are count nouns that can be modified with numerals or quantifiers as shown in (0) and (0). In addition, these expressions are also used as modifiers within noun phrases as in (0) to (0).


\begin{styleExampleTitle}
Head and modifier functions
\end{styleExampleTitle}

\begin{tabular}{lllllllllll}
\lsptoprule
\label{bkm:Ref350791591}
\gll {saya} {hanya} {bisa} {makang} {kasi} {makang} {dorang} {\bluebold{satu}} {\bluebold{malam}} {saja}\\ %
& \textsc{1sg} & only & be.able & eat & give & eat & \textsc{3pl} & one & night & just\\
\lspbottomrule
\end{tabular}
\ea
\glt 
‘I can only eat, feed them just \bluebold{one night}’ \textstyleExampleSource{[081011-020-Cv.0080]}
\z

\begin{tabular}{llllllll}
\lsptoprule
\label{bkm:Ref350791592}
\gll {ko} {harus} {\bluebold{setiap}} {\bluebold{pagi}} {harus} {jalang} {trus}\\ %
& \textsc{2sg} & have.to & every & morning & have.to & walk & be.continuous\\
\lspbottomrule
\end{tabular}
\ea
\glt 
[About attending school:] ‘you have to (go to school) \bluebold{every morning}, (you) have to go regularly’ \textstyleExampleSource{[080917-007-CvHt.0004]}
\z

\begin{tabular}{lllll}
\lsptoprule
\label{bkm:Ref350842668}
\gll {tra} {ada} {\bluebold{snek}} {\bluebold{pagi}}\\ %
& \textsc{neg} & exist & snack & morning\\
\lspbottomrule
\end{tabular}
\ea
\glt 
‘there was no \bluebold{morning snack}’ \textstyleExampleSource{[081025-008-Cv.0079]}
\z

\begin{tabular}{llllllll}
\lsptoprule
(\stepcounter{}{\the}) & \bluebold{hari} & \bluebold{sening} & \bluebold{sore} & \bluebold{itu} & smua & harus & hadir\\
& day & Monday & afternoon & \textsc{d.dist} & all & have.to & attend\\
\lspbottomrule
\end{tabular}
\ea
\glt 
[About volleyball training:] ‘\bluebold{next Monday afternoon} everyone has to attend’ \textstyleExampleSource{[081109-001-Cv.0053]}
\z

\begin{tabular}{lllllllllll}
\lsptoprule
\label{bkm:Ref350842670}
\gll {dari} {jam} {dua} {blas} {tong} {makang} {sampe} {\bluebold{jam}} {\bluebold{satu}} {\bluebold{siang}}\\ %
& from & hour & two & teens & \textsc{1pl} & eat & until & hour & one & midday\\
\lspbottomrule
\end{tabular}
\ea
\glt 
‘we ate from twelve o’clock until \bluebold{one o’clock midday}’ \textstyleExampleSource{[081025-008-Cv.0085]}
\z


Within the clause, the four expressions typically occur at clause boundaries. Most often, they occur in clause-initial position where they set the temporal stage for the entire clause. Alternatively, although less often, the temporal expressions occur in clause-final position, where they are less prominent. This is illustrated in (0) to (0) with near contrastive examples. The time expression \textitbf{pagi} ‘morning’ occurs in clause-initial position in (0) and in clause-final position in (0). Likewise, \textitbf{malam} ‘night’ occurs in clause-initial position in (0) and in clause-final position in (0).


\begin{styleExampleTitle}
Positions within the clause
\end{styleExampleTitle}

\begin{tabular}{llllllll}
\lsptoprule
\label{bkm:Ref350793704}
\gll {\bluebold{pagi}} {kitong} {datang} {lagi} {dong} {kasi} {makang}\\ %
& morning & \textsc{1pl} & come & again & \textsc{3pl} & give & eat\\
\lspbottomrule
\end{tabular}
\ea
\glt 
[About a youth retreat:] ‘\bluebold{in the morning}, we came again, they fed (us)’ \textstyleExampleSource{[081025-009a-Cv.0024]}
\z

\begin{tabular}{lllll}
\lsptoprule
\label{bkm:Ref350793705}
\gll {kemaring} {sa} {datang} {\bluebold{pagi}}\\ %
& yesterday & \textsc{1sg} & come & morning\\
\lspbottomrule
\end{tabular}
\ea
\glt 
‘yesterday, I came \bluebold{in the morning}’ \textstyleExampleSource{[080922-002-Cv.0021]}
\z

\begin{tabular}{lllll}
\lsptoprule
\label{bkm:Ref362433428}
\gll {…} {\bluebold{malam}} {sa} {berdoa}\\ %
&  & night & \textsc{1sg} & pray\\
\lspbottomrule
\end{tabular}
\ea
\glt 
‘[when they said (that) he was very very sick,] \bluebold{in the evening} I prayed (for him)’ \textstyleExampleSource{[080923-015-CvEx.0010]}
\z

\begin{tabular}{lllllllll}
\lsptoprule
\label{bkm:Ref350793707}
\gll {pas} {bapa} {berdoa} {\bluebold{malam}} {\bluebold{itu},} {pagi} {de} {meninggal}\\ %
& precisely & father & pray & night & \textsc{d.dist} & morning & \textsc{3sg} & die\\
\lspbottomrule
\end{tabular}
\ea
\glt 
‘(my) father prayed \bluebold{that evening}, and right away in the morning he (the boy) died’ \textstyleExampleSource{[081025-009b-Cv.0039]}
\z


The periods-of-the-day expressions are also used in greetings, as illustrated in (0) to (0).


\begin{styleExampleTitle}
Usage in greetings
\end{styleExampleTitle}

\begin{tabular}{llll}
\lsptoprule
\label{bkm:Ref350843319}
\gll {slamat} {\bluebold{pagi}} {pak}\\ %
& be.safe & morning & father\\
\lspbottomrule
\end{tabular}
\ea
\glt 
‘good \bluebold{morning} Sir’ \textstyleExampleSource{[080923-011-Cv.0002]}
\z

\begin{tabular}{llll}
\lsptoprule
(\stepcounter{}{\the}) & slamat & \bluebold{siang} & ana\\
& be.safe & midday & child\\
\lspbottomrule
\end{tabular}
\ea
\glt 
‘good \bluebold{midday} child’ \textstyleExampleSource{[080922-001a-CvPh.1260]}
\z

\begin{tabular}{llll}
\lsptoprule
(\stepcounter{}{\the}) & slamat & \bluebold{sore} & smua\\
& be.safe & afternoon & all\\
\lspbottomrule
\end{tabular}
\ea
\glt 
‘good \bluebold{afternoon} you all’ \textstyleExampleSource{[081110-002-Cv.0001]}
\z

\begin{tabular}{lllll}
\lsptoprule
\label{bkm:Ref350843320}
\gll {slamat} {\bluebold{malam}} {pak} {pendeta}\\ %
& be.safe & night & father & pastor\\
\lspbottomrule
\end{tabular}
\ea
\glt
‘good \bluebold{evening} Mr. Pastor’ \textstyleExampleSource{[080925-003-Cv.0240]}
\end{styleFreeTranslEngxvpt}

\paragraph[Days of the week and months of the year]{Days of the week and months of the year}
\label{bkm:Ref350878935}
The seven days of the week and the twelve months of the year are listed in Table  ‎5 .8.


\begin{stylecaption}
\label{bkm:Ref350790222}Table ‎5.\stepcounter{Table}{\theTable}:  Days of the week and months of the year
\end{stylecaption}

\begin{tabular}{llll}
\lsptoprule

\multicolumn{4}{l}{ Days of the week}\\
 Item & Gloss & Item & \arraybslash Gloss\\
\textitbf{sening} & ‘Monday’ & \textitbf{jumat} & ‘Friday’\\
\textitbf{slasa} & ‘Tuesday’ & \textitbf{saptu} & ‘Saturday’\\
\textitbf{rabu} & ‘Wednesday’ & \textitbf{minggu} & ‘Sunday’\\
\textitbf{kamis} & ‘Thursday’ &  & \\
\multicolumn{4}{l}{ Months of the year}\\
 Item & Gloss & Item & \arraybslash Gloss\\
\textitbf{januari} & ‘January’ & \textitbf{juli} & ‘July’\\
\textitbf{februari} & ‘February’ & \textitbf{agustus} & ‘August’\\
\textitbf{maret} & ‘March’ & \textitbf{september} & ‘September’\\
\textitbf{april} & ‘April’ & \textitbf{oktober} & ‘October’\\
\textitbf{mey} & ‘May’ & \textitbf{nofember} & ‘November’\\
\textitbf{juni} & ‘Juni’ & \textitbf{desember} & ‘December’\\
\lspbottomrule
\end{tabular}

Typically, the days of the week and the months of the year occur in \textsc{n1n2} noun phrases, headed by the common nouns \textitbf{hari} ‘day’ and \textitbf{bulang} ‘month’, respectively (see Table  ‎5 .6; see also §8.2.2). Examples for the days of the week are given in (0) and (0) and for the months of the year in (0). Occasionally, however, speakers omit \textitbf{hari} ‘day’ or \textitbf{bulang} ‘month’ as with \textitbf{rabu} ‘Wednesday’ in (0) and with \textitbf{oktober} ‘October’ and \textitbf{januari} ‘January’ in (0), respectively.


\begin{tabular}{lllllll}
\lsptoprule
\label{bkm:Ref350795218}
\gll {yo} {bapa,} {\bluebold{hari}} {\bluebold{minggu}} {sa} {datang}\\ %
& yes & father & day & Sunday & \textsc{1sg} & come\\
\lspbottomrule
\end{tabular}
\ea
\glt 
‘yes father, \bluebold{on Sunday} I’ll come’ \textstyleExampleSource{[080922-001a-CvPh.0344]}
\z

\begin{tabular}{lllllllllllllll}
\lsptoprule
\label{bkm:Ref350795219}
\gll {\multicolumn{2}{l}{\bluebold{hari}}} {\multicolumn{2}{l}{\bluebold{slasa}}} {\multicolumn{2}{l}{itu}} {…} {\multicolumn{2}{l}{de}} {pu} {\multicolumn{2}{l}{ana}} {prempuang} {meninggal}\\ %
& \multicolumn{2}{l}{day} & \multicolumn{2}{l}{Tuesday} & \multicolumn{2}{l}{\textsc{d.dist}} &  & \multicolumn{2}{l}{\textsc{3sg}} & \textsc{poss} & \multicolumn{2}{l}{child} & woman & die\\
& jadi & \multicolumn{2}{l}{tong} & \multicolumn{2}{l}{tinggal} & di & \multicolumn{2}{l}{ruma} & \multicolumn{3}{l}{sampe} & \multicolumn{3}{l}{\bluebold{rabu}}\\
& so & \multicolumn{2}{l}{\textsc{1pl}} & \multicolumn{2}{l}{stay} & at & \multicolumn{2}{l}{house} & \multicolumn{3}{l}{until} & \multicolumn{3}{l}{Wednesday}\\
\lspbottomrule
\end{tabular}
\ea
\glt 
‘that \bluebold{Monday} … his daughter died, so we stayed at home until \bluebold{Wednesday}’ \textstyleExampleSource{[080925-003-Cv.0001]}
\z

\begin{tabular}{llllllll}
\lsptoprule
\label{bkm:Ref350844219}
\gll {ko} {pu} {alpa} {banyak} {di} {\bluebold{bulang}} {\bluebold{oktober}}\\ %
& \textsc{2sg} & \textsc{poss} & be.absent & many & at & month & October\\
\lspbottomrule
\end{tabular}
\ea
\glt 
‘you have lots of (unexcused) absences in \bluebold{October}’ \textstyleExampleSource{[081023-004-Cv.0015]}
\z

\begin{tabular}{lllllllll}
\lsptoprule
\label{bkm:Ref350844220}
\gll {o} {nanti} {\bluebold{oktober}} {e} {\bluebold{januari}} {baru} {kitong} {antar}\\ %
& oh! & very.soon & October & uh & January & and.then & \textsc{1pl} & bring\\
\lspbottomrule
\end{tabular}
\ea
\glt
[About wedding customs:] ‘oh later in \bluebold{October} uh \bluebold{January}, and then we’ll bring (our daughter to your house)’ \textstyleExampleSource{[081110-005-CvPr.0049]}
\end{styleFreeTranslEngxvpt}

\paragraph[Relative time]{Relative time}
\label{bkm:Ref350878936}
Relative time is expressed with the three time-denoting nouns and two phrasal expressions presented in Table  ‎5 .9.


\begin{stylecaption}
\label{bkm:Ref350788411}Table ‎5.\stepcounter{Table}{\theTable}:  Relative time
\end{stylecaption}

\tablehead{
 Item & \arraybslash Gloss\\
}
\begin{tabular}{ll}
\lsptoprule
\textitbf{kemaring dulu} & ‘the day before yesterday’\\
yesterday be.prior & \\
\textitbf{kemaring} & ‘yesterday, sometime ago’\\
\textitbf{hari ini} & ‘today’\\
day \textsc{d.prox} & \\
\textitbf{besok} & ‘tomorrow, sometime in the future’\\
\textitbf{lusa} & ‘the day after tomorrow’\\
\lspbottomrule
\end{tabular}

Within the clause, the relative-time denoting expressions typically occur in clause-initial position. Here they set the temporal stage for the entire clause, similar to the nouns denoting periods of the day, discussed in §5.2.5.2. This is illustrated with the examples in (0) to (0). Alternatively, but less often, the relative-time expressions directly precede the predicate where they are less prominent, as shown in (0). The contrast in meaning conveyed by the different positions within the clause is illustrated with \textitbf{besok} ‘tomorrow’ in the near contrastive examples in (0) and (0). By fronting \textitbf{besok} ‘tomorrow’ in (0), the speaker accentuates the temporal setting of the entire clause. This is not the case in (0), where \textitbf{besok} ‘tomorrow’ directly precedes the predicate, where it is less salient.
\end{styleBodyaftervbefore}


The examples in (0) and (0) also illustrate that the temporal scope of \textitbf{kemaring} ‘yesterday’ and \textitbf{besok} ‘tomorrow’ is larger than the preceding or following 24-hour period, respectively. Generally speaking \textitbf{kemaring} ‘yesterday’ denotes a past point in time such as \textitbf{kemaring} ‘some time ago’ in (0). Along similar lines, \textitbf{besok} ‘tomorrow’ refers to a future point in time which in (0) is \textitbf{besok} ‘next year’.
\end{styleBodyvvafter}

\begin{styleExampleTitle}
Positions within the clause
\end{styleExampleTitle}

\begin{tabular}{llllllllllll}
\lsptoprule
\label{bkm:Ref350868459}
\gll {\multicolumn{2}{l}{\bluebold{kemaring}}} {\multicolumn{2}{l}{\bluebold{dulu}}} {\multicolumn{2}{l}{sa}} {deng} {nene} {nene} {jam} {dua}\\ %
& \multicolumn{2}{l}{yesterday} & \multicolumn{2}{l}{be.prior} & \multicolumn{2}{l}{\textsc{1sg}} & with & grandmother & grandmother & hour & two\\
& malam & \multicolumn{2}{l}{datang} & \multicolumn{2}{l}{deng} & \multicolumn{6}{l}{menangis}\\
& night & \multicolumn{2}{l}{come} & \multicolumn{2}{l}{with} & \multicolumn{6}{l}{cry}\\
\lspbottomrule
\end{tabular}
\ea
\glt 
‘\bluebold{the day before yesterday} I and grandmother, at two in the morning grandmother came crying …’ \textstyleExampleSource{[081014-008-CvNP.0001]}
\z

\begin{tabular}{llllllll}
\lsptoprule
(\stepcounter{}{\the}) & yo, & \bluebold{hari} & \bluebold{ini} & suda & ko & su & skola\\
& yes & day & \textsc{d.prox} & already & \textsc{2sg} & already & go.to.school\\
\lspbottomrule
\end{tabular}
\ea
\glt 
‘yes, \bluebold{today} you already went to school’ \textstyleExampleSource{[080917-003a-CvEx.0006]}
\z

\begin{tabular}{llllllllllll}
\lsptoprule
\label{bkm:Ref362447896}
\gll {kalo} {\multicolumn{3}{l}{\bluebold{besok}}} {de} {\multicolumn{2}{l}{itu}} {hadir} {ke} {sana} {tu}\\ %
& if & \multicolumn{3}{l}{tomorrow} & \textsc{3sg} & \multicolumn{2}{l}{\textsc{d.dist}} & attend & to & \textsc{l.dist} & \textsc{d.dist}\\
& \multicolumn{2}{l}{biking} & de & \multicolumn{3}{l}{sperti} & \multicolumn{5}{l}{bos}\\
& \multicolumn{2}{l}{make} & \textsc{3sg} & \multicolumn{3}{l}{similar.to} & \multicolumn{5}{l}{boss}\\
\lspbottomrule
\end{tabular}
\ea
\glt 
[About an event planned for the next year:] ‘if \bluebold{next year} he (the mayor), what’s-its-name, (comes and) attends (the retreat) over there, treat him like a boss’ \textstyleExampleSource{[081025-009a-Cv.0172]}
\z

\begin{tabular}{lllllllll}
\lsptoprule
\label{bkm:Ref350875874}
\gll {bapa} {nanti} {\bluebold{besok}} {hadir} {di} {ini} {retrit} {pemuda}\\ %
& father & very.soon & tomorrow & attend & at & \textsc{d.prox} & retreat & youth\\
\lspbottomrule
\end{tabular}
\ea
\glt 
[About an event planned for the next year:] ‘you (‘father’) (have to) attend, what’s-its-name, the youth retreat \bluebold{next year}’ \textstyleExampleSource{[081025-009a-Cv.0175]}
\z


In addition, the corpus includes a small number of utterances in which the nouns designating relative-time occur as subjects in nonverbal clauses. This is illustrated with \textitbf{besok} ‘tomorrow’ and \textitbf{lusa} ‘the day after tomorrow’ in (0).


\begin{styleExampleTitle}
Subject-function in nonverbal clauses
\end{styleExampleTitle}

\begin{tabular}{lllllllll}
\lsptoprule
\label{bkm:Ref350870146}
\gll {\bluebold{besok}} {hari} {kamis} {\bluebold{lusa}} {hari} {jumat} {baru} {…}\\ %
& tomorrow & day & Thursday & day.after.tomorrow & day & Friday & and.then & \\
\lspbottomrule
\end{tabular}
\ea
\glt 
‘\bluebold{tomorrow} is Thursday, \bluebold{the day after tomorrow} is Friday and then …’ \textstyleExampleSource{[080917-003a-CvEx.0006]}
\z


Like other nouns, relative-time denoting nouns also have adnominal uses as shown in (0) and (0). In their adnominal uses, they occur in post-head position and have restrictive function. That is, they specify whether the period or point in time encoded by the head nominal is situated in the future or in the past, as in \textitbf{hari minggu besok} ‘next Sunday’ in (0) or \textitbf{taung kemaring} ‘a few years back’ in (0).


\begin{styleExampleTitle}
Adnominal uses
\end{styleExampleTitle}

\begin{tabular}{llllllll}
\lsptoprule
\label{bkm:Ref350878181}
\gll {yo} {memang} {\bluebold{hari}} {\bluebold{minggu}} {\bluebold{besok}} {sa} {datang}\\ %
& yes & indeed & day & Sunday & tomorrow & \textsc{1sg} & come\\
\lspbottomrule
\end{tabular}
\ea
\glt 
‘yes, indeed, \bluebold{next Sunday} I’ll come’ \textstyleExampleSource{[080922-001a-CvPh.0346]}
\z

\begin{tabular}{lllllllllllll}
\lsptoprule
\label{bkm:Ref350878182}
\gll {\multicolumn{2}{l}{banyak}} {\multicolumn{2}{l}{mati}} {di} {\multicolumn{2}{l}{lautang}} {kas} {tenggelam} {sampe} {\bluebold{taung}} {\bluebold{kemaring}}\\ %
& \multicolumn{2}{l}{many} & \multicolumn{2}{l}{die} & at & \multicolumn{2}{l}{ocean} & give & sink & until & year & yesterday\\
& taung & \multicolumn{2}{l}{…} & dua & \multicolumn{2}{l}{ribu} & \multicolumn{6}{l}{dua}\\
& year & \multicolumn{2}{l}{} & two & \multicolumn{2}{l}{thousand} & \multicolumn{6}{l}{two}\\
\lspbottomrule
\end{tabular}
\ea
\glt 
[About people in a container who died in the ocean:] ‘many died in the (open) ocean, (the murderers) sank (the containers), (many died in the open ocean) until \bluebold{a few years back}, (until) the year 2002’ \textstyleExampleSource{[081029-002-Cv.0025]}
\z


Relative-time expressions also occur as complements in prepositional phrases as, for instance, in \textitbf{sampe besok} ‘until the next day’ (literally ‘until tomorrow’) in (0). This example also illustrates that \textitbf{besok} ‘tomorrow’ denotes relative time. As the events described here happened in the past, \textitbf{besok} ‘tomorrow’ refers to a future point in time relative to the narrated events. Hence, \textitbf{besok} translates as ‘the next day’. (Prepositions encoding time are discussed in more detail in §10.1.)


\begin{styleExampleTitle}
Complements in prepositional phrases
\end{styleExampleTitle}

\begin{tabular}{lllllllll}
\lsptoprule
\label{bkm:Ref350873187}
\gll {sa} {minum} {lagi} {trus} {sa} {tinggal} {\bluebold{sampe}} {\bluebold{besok}}\\ %
& \textsc{1sg} & drink & again & next & \textsc{1sg} & stay & until & tomorrow\\
\lspbottomrule
\end{tabular}
\ea
\glt
[About recovering from an accident:] ‘I took (medicine) again, then I stayed \bluebold{until the next day}’ (Lit. ‘\bluebold{until tomorrow}’) \textstyleExampleSource{[081015-005-NP.0042-0043]}
\end{styleFreeTranslEngxvpt}

\subsection{Classifying nouns}
\label{bkm:Ref351641492}
Papuan Malay has a very reduced inventory of classifying nouns. Attested is only one, namely the common noun \textitbf{ekor} ‘tail’ which is also used to count animals. In this function, it always follows a post-head numeral, as shown in (0).


\begin{styleExampleTitle}
Enumeration of animals
\end{styleExampleTitle}

\begin{tabular}{llllllllllllll}
\lsptoprule
\label{bkm:Ref339637338}
\gll {dong} {\multicolumn{2}{l}{dua}} {\multicolumn{2}{l}{dapat}} {\multicolumn{2}{l}{ikang}} {\multicolumn{3}{l}{ini}} {\multicolumn{2}{l}{\bluebold{tiga}}} {\bluebold{ekor}}\\ %
& \textsc{3pl} & \multicolumn{2}{l}{two} & \multicolumn{2}{l}{get} & \multicolumn{2}{l}{fish} & \multicolumn{3}{l}{\textsc{d.prox}} & \multicolumn{2}{l}{three} & tail\\
& \multicolumn{2}{l}{dapat} & \multicolumn{2}{l}{\bluebold{ikang}} & \multicolumn{2}{l}{\bluebold{tiga}} & \multicolumn{2}{l}{\bluebold{ekor}} & dong & \multicolumn{2}{l}{dua} & \multicolumn{2}{l}{…}\\
& \multicolumn{2}{l}{get} & \multicolumn{2}{l}{fish} & \multicolumn{2}{l}{three} & \multicolumn{2}{l}{tail} & \textsc{3pl} & \multicolumn{2}{l}{two} & \multicolumn{2}{l}{}\\
\lspbottomrule
\end{tabular}
\ea
\glt 
‘the two of them get these fish, \bluebold{three (of them)}, having gotten \bluebold{three fish}, the two of them …’ (Lit. ‘\bluebold{three tails}’) \textstyleExampleSource{[081109-011-JR.0003]}
\z


As a classifying noun, \textitbf{ekor} ‘tail’ does not refer to the entities themselves being counted but rather to their form, as is rather common in Malay and other Austronesian varieties; see for instance Ambon Malay {(van Minde 1997: 153)}, Ternate Malay {\citep[62]{Litamahuputty1994}}, Tetun {(van Klinken 1999)}, or Standard Malay {(Mintz 2002: 321–323)}.



Enumeration of people and objects, by contrast, is done without a classifier as illustrated in (0) and (0), respectively.
\end{styleBodyvvafter}

\begin{styleExampleTitle}
Enumeration of people and objects
\end{styleExampleTitle}

\begin{tabular}{lllll}
\lsptoprule
\label{bkm:Ref339637339}
\gll {jadi} {saya} {\bluebold{empat}} {\bluebold{ana}}\\ %
& so & \textsc{1sg} & four & child\\
\lspbottomrule
\end{tabular}
\ea
\glt 
‘so I (have) \bluebold{four children}’ \textstyleExampleSource{[081006-024-CvEx.0002]}
\z

\begin{tabular}{lllllllllllll}
\lsptoprule
\label{bkm:Ref339637340}
\gll {\multicolumn{2}{l}{orang}} {\multicolumn{2}{l}{Sarmi}} {\multicolumn{2}{l}{harus}} {\multicolumn{2}{l}{siap}} {\multicolumn{2}{l}{untuk}} {orang} {Sorong}\\ %
& \multicolumn{2}{l}{person} & \multicolumn{2}{l}{Sarmi} & \multicolumn{2}{l}{have.to} & \multicolumn{2}{l}{provide} & \multicolumn{2}{l}{for} & person & Sorong\\
& \bluebold{spulu} & \multicolumn{2}{l}{\bluebold{kaing}} & \multicolumn{2}{l}{\bluebold{itu}} & \multicolumn{2}{l}{\bluebold{kaing}} & \multicolumn{2}{l}{\bluebold{adat}} & \multicolumn{3}{l}{\bluebold{itu}}\\
& ten & \multicolumn{2}{l}{cloth} & \multicolumn{2}{l}{\textsc{d.dist}} & \multicolumn{2}{l}{cloth} & \multicolumn{2}{l}{tradition} & \multicolumn{3}{l}{\textsc{d.dist}}\\
\lspbottomrule
\end{tabular}
\ea
\glt
[About bride-prices:] ‘a Sarmi person has to provide a Sorong person with \bluebold{those ten cloths}, \bluebold{those traditional cloths}’ \textstyleExampleSource{[081006-029-CvEx.0012]}
\end{styleFreeTranslEngxvpt}

\subsection{Kinship terms}
\label{bkm:Ref435630031}
This section presents the most common Papuan Malay terms for consanguineal and affinal kin. An initial investigation of the kinship system indicates that Papuan Malay uses a combination of Iroquois and Hawaiian terminologies and makes a relative age discrimination.



Before presenting the Papuan Malay kinship terms, Table  ‎5 .10 lists the standard symbols used to abbreviate basic terms.
\end{styleBodyvvafter}

\begin{stylecaption}
\label{bkm:Ref359570228}Table ‎5.\stepcounter{Table}{\theTable}:  Symbols for kinship terms
\end{stylecaption}

\tablehead{
 Terms & Symbols & Terms & Symbols & Terms & \arraybslash Symbols\\
}
\begin{tabular}{llllll}
\lsptoprule
father & F & brother & B & husband & H\\
mother & M & sister & Z & wife & W\\
parent & P & sibling & Sb & spouse & Sp\\
son & S & older & o &  & \\
daughter & D & younger & y &  & \\
child & C &  &  &  & \\
\lspbottomrule
\end{tabular}

More complex kinship terms are expressed by chains of these abbreviations, such as FZ for ‘father’s sister’ or MF for ‘mother’s father’.


\paragraph[Consanguineal kin]{Consanguineal kin}
\label{bkm:Ref359606628}
The kinship system is Iroquois, in that Papuan Malay makes a distinction in the first ascending generation between same-sex and cross-sex parents’ siblings in a bifurcate merging pattern, as demonstrated in Table  ‎5 .11. Contrasting with typical Iroquois systems, however, the cross-parallel distinction only applies to parents’ younger siblings. That is, only parents’ same-sexed younger siblings are considered as consanguines: \textitbf{bapa-ade} ‘uncle’ (literally ‘younger father’) and \textitbf{mama-ade} ‘aunt’ (literally ‘younger mother’). Parents’ opposite-sexed younger siblings are called \textitbf{om} ‘uncle’ and \textitbf{tanta} ‘aunt’; both terms are loan words from Dutch. By contrast, Papuan Malay does not distinguish between parents’ older siblings of opposite sex. That is, all parents’ older siblings are considered as consanguines regardless of their sex: \textitbf{bapa-tua} ‘uncle’ (literally ‘old father’) and \textitbf{mama-tua} ‘aunt’ (literally ‘old mother’). The six consanguineal terms also extend to affinal kin, as discussed in §5.2.7.2.



With respect to other generations, the kinship system is Hawaiian, in that it extends bilaterally, without making distinctions between lineal and collateral consanguines, or between cross and parallel consanguines. Consequently, Papuan Malay does not distinguish between siblings and cousins, as shown in Table  ‎5 .11. That is, children of parents’ siblings are also classified as siblings. In addition, the system makes a relative age discrimination. Older siblings and children of parents’ older siblings are called \textitbf{kaka} ‘older sibling’ while younger siblings and children of parents’ younger siblings are called \textitbf{ade} ‘younger sibling’. The same relative age discrimination applies to cousins in the second degree of collaterality: their relative ages are determined by the ages of the linking grandparents. With the exception of the reference term \textitbf{orang-tua} ‘parent’, speakers use the consanguineal terms, listed in Table  ‎5 .11, both for reference and for address.
\end{styleBodyvvafter}

\begin{stylecaption}
\label{bkm:Ref359570239}Table ‎5.\stepcounter{Table}{\theTable}:  Papuan Malay kinship terms: Consanguineal kin
\end{stylecaption}

\tablehead{
 Item & Gloss & Symbol & \arraybslash Relation\\
}
\begin{tabular}{llll}
\lsptoprule
\textitbf{bapa} & ‘father’ & F & father\\
\textitbf{mama} & ‘mother’ & M & mother\\
\textitbf{orang-tua} & ‘parent’ & P & parent\\
\textitbf{ana} & ‘child’ & C & child\\
\textitbf{kaka} & ‘older sibling’ & oSb & older sibling\\
&  & PoSbC & parent’s older sibling’s child\\
\textitbf{ade} & ‘younger sibling’ & ySb & younger sibling\\
&  & PySbC & parent’s younger sibling’s child\\
\textitbf{bapa-tua} & ‘uncle’ & PoB & parent’s older brother\\
\textitbf{bapa-ade} & ‘uncle’ & FyB & father’s younger brother\\
\textitbf{om} & ‘uncle’ & MyB & mother’s younger brother\\
\textitbf{mama-tua} & ‘aunt’ & PoZ & parent’s older sister\\
\textitbf{mama-ade} & ‘aunt’ & MyZ & mother’s younger sister\\
\textitbf{tanta} & ‘aunt’ & FyZ & father’s younger sister\\
\textitbf{tete} & ‘grandfather’ & PF & parent’s father\\
&  & PPB & parent’s parent’s brother\\
\textitbf{nene} & ‘grandmother’ & PM & parent’s mother\\
&  & PPZ & parent’s parent’s sister\\
\textitbf{cucu} & ‘grandchild’ & CC & child’s child\\
\lspbottomrule
\end{tabular}

To signal the gender of a sibling or child, the kinship terms \textitbf{kaka} ‘older sibling’, \textitbf{ade} ‘younger sibling’, and \textitbf{ana} ‘child’ are modified with the common nouns \textitbf{laki{\Tilde}laki} ‘man’ or \textitbf{prempuang} ‘woman’, giving \textitbf{kaka laki{\Tilde}laki} ‘older brother’, \textitbf{ade prempuang} ‘younger sister’, or \textitbf{ana laki{\Tilde}laki} ‘son’.


\paragraph[Affinal kin]{Affinal kin}
\label{bkm:Ref359606523}
The Papuan Malay affinal terms, listed in Table  ‎5 .12, include two terms for spouse, that is, \textitbf{paytua} ‘husband’ and \textitbf{maytua} ‘wife’, and two terms for in-laws, namely \textitbf{mantu} ‘parent/child in-law’ and \textitbf{ipar} ‘sibling in-law’. Speakers employ these terms for both reference and address.



Papuan Malay distinguishes between in-laws belonging to different generations and those belonging to the same generation, as illustrated in Table  ‎5 .12.
\end{styleBodyvafter}


The expression for in-laws belonging to the first ascending or descending generation is the self-reciprocal term \textitbf{mantu} ‘parent/child in-law’. This term, however, is unattested on its own. It is always modified with the common nouns \textitbf{bapa} ‘father’, \textitbf{mama} ‘mother’, or \textitbf{ana} ‘child’ to specify the affinal relationship, giving \textitbf{bapa mantu} ‘father in-law’, \textitbf{mama mantu} ‘mother in-law’, or \textitbf{ana mantu} ‘child in-law’.
\end{styleBodyvafter}


The term for same-generation in-laws is \textitbf{ipar} ‘sibling in-law’. This self-reciprocal term extends to spouses’ siblings and those siblings’ spouses, as well as to children’s spouses’ parents (co-parents-in-law). Again, a relative age discrimination is made similar to that for siblings: \textitbf{kaka ipar} ‘older sibling in-law’ and \textitbf{ade ipar} ‘younger sibling in-law’.
\end{styleBodyvvafter}

\begin{stylecaption}
\label{bkm:Ref359570240}Table ‎5.\stepcounter{Table}{\theTable}:  Papuan Malay kinship terms: Affinal kin
\end{stylecaption}

\tablehead{
 Item & Gloss & Symbol & \arraybslash Relation\\
}
\begin{tabular}{llll}
\lsptoprule
\textitbf{paytua} & ‘husband’ & H & husband\\
\textitbf{maytua} & ‘wife’ & W & wife\\
\textitbf{mantu} & ‘parent/child in-law’ & SpP & spouse’s parents\\
&  & CSp & child’s spouse\\
\textitbf{ipar} & ‘sibling in-law’ & SbSp & sibling’s spouse\\
&  & SpSb & spouse’s sibling\\
&  & SpSbSp & spouse’s sibling’s spouse\\
&  & CSpP & child’s spouse’s parents\\
\lspbottomrule
\end{tabular}

The six consanguineal terms that distinguish between same-sex and cross-sex parents’ siblings in the first ascending generation, mentioned in §5.2.7.1, also extend to affinal kin, as shown in Table  ‎5 .13.


\begin{stylecaption}
\label{bkm:Ref359607222}Table ‎5.\stepcounter{Table}{\theTable}:  Papuan Malay consanguineal terms extending to affinal kin
\end{stylecaption}

\tablehead{
 Item & Gloss & Symbol & \arraybslash Relation\\
}
\begin{tabular}{llll}
\lsptoprule
\textitbf{bapa-tua} & ‘uncle’ & PoZH & parent’s older sister’s husband\\
\textitbf{bapa-ade} & ‘uncle’ & MyZH & mother’s younger sister’s husband\\
\textitbf{om} & ‘uncle’ & FyZH & father’s younger sister’s husband\\
\textitbf{mama-tua} & ‘aunt’ & PoBW & parent’s older brother’s wife\\
\textitbf{mama-ade} & ‘aunt’ & FyBW & father’s younger brother’s wife\\
\textitbf{tanta} & ‘aunt’ & MyBW & mother’s younger brother’s wife\\
\lspbottomrule
\end{tabular}
\section{Verbs}
\label{bkm:Ref435775768}\label{bkm:Ref435775367}
Papuan Malay has a large open class of verbs which express actions, events, and processes, as well as states or more time-stable properties. They have the following defining syntactic and functional properties:


%\setcounter{itemize}{0}
\begin{itemize}
\item \begin{styleIIndented}
Valency: each verb takes a specific number of arguments (§5.3.1).
\end{styleIIndented}\item \begin{styleIIndented}
Predicative function is predominant; besides, verbs also have attributive uses in noun phrases (§5.3.2).
\end{styleIIndented}\item \begin{styleIIndented}
Modification with adverbs, including intensification and grading (§5.3.4 and §5.3.5).
\end{styleIIndented}\item \begin{styleIIndented}
Negation only with \textitbf{tida} ‘\textsc{neg}’ or \textitbf{tra} ‘\textsc{neg}’ (§5.3.6).\footnote{\\
\\
\\
\\
\\
\\
\\
\\
\\
\\
\\
\\
As for the occurrence of \textitbf{bukang} ‘\textsc{neg}’ in verbal clauses, see Footnote 155 in §5.3.6 (p. \pageref{bkm:Ref434860809}).\\
\\
\\
}
\end{styleIIndented}\item \begin{styleIvI}
Occurrence in causative and in reciprocal constructions (§5.3.7 and §5.3.8).
\end{styleIvI}\end{itemize}

Morphological properties play only a minor role in defining verbs as a distinct word class, due to the lack of inflectional morphology and the limited role of derivational processes. The latter include reduplication (for details see §4.2.2), and, to a limited extent, affixation with prefix \textscItalBold{ter\-} or suffix \-\textitbf{ang} (§5.4; see also §3.1).



Verbs are divided into three classes on the basis of their valency and their tendency to function predicatively, namely trivalent, bivalent, and monovalent verbs. In turn, monovalent verbs are further divided into dynamic and stative verbs. That is, Papuan Malay does not have a distinct class of adjectives. Instead, monovalent stative verbs encode, what \citet[4]{Dixon2004} calls “the four core semantic types” of dimension, age, value, and color which are “typically associated with the word class adjective”. The two criteria of valency and prevalent predicative function also account for the other properties of verbs, listed above and discussed in more detail in the following sections.
\end{styleBodyvafter}


Verbs are distinct from nouns (§5.2) and adverbs (§5.4) in terms of the following distributional properties:
\end{styleBodyvvafter}

%\setcounter{itemize}{0}
\begin{itemize}
\item \begin{styleIIndented}
Contrasting with nouns, verbs (a) have valency,\footnote{\\
\\
\\
\\
\\
\\
\\
\\
\\
\\
\\
\\
It is acknowledged that some authors maintain that nouns have valence. {Van Valin and La\citet{Polla1997}}, for instance, discuss the “layered structure of adpositional and noun phrases” {(1997: 52–67)} and the “semantic representation of nouns and noun phrases” {(1997: 184–195)}, and {van Valin (2001: 89–92)} examines “[t]ypes of dependencies”. See also Croft’s (1991: 62–79) discussion on “Structural markedness and the semantic prototypes”, as well as {\citet{Allerton2006}, Sommerfeldt and \citet{Schreiber1983}, and van Durme and Institut for Sprog og \citet{Kommunikation1997}}.\\
\\
\\
} and (b) are negated with \textitbf{tida}/\textitbf{tra} ‘\textsc{neg}’. In addition (c) verbs, except for monovalent stative ones, occur as predicates in reciprocal constructions, and (d) monovalent stative and bivalent verbs occur as predicates in comparative constructions.
\end{styleIIndented}\item \begin{styleIvI}
Unlike adverbs, verbs (a) are used predicatively, and (b) can modify nouns.
\end{styleIvI}\end{itemize}

The following sections explore the characteristics and properties of verbs in more detail. As for their syntactic properties the following topics are discussed: valency in §5.3.1, predicative and attributive functions in §5.3.2, adverbial modification in §5.3.3, intensification in §5.3.4, grading in §5.3.5, negation in §5.3.6, occurrences in causative constructions in §5.3.7, and uses in reciprocal constructions in §5.3.8. Finally, the morphological properties of verbs are briefly examined in §5.4. In each section, dynamic verbs are discussed first, and stative verbs second. Dynamic verbs, in turn, are described in order from those with three arguments to those with one argument. Each section also discusses the type and token frequencies in the corpus for the respective properties and summarizes these frequencies in a table. These tables form the basis for the summary in §5.4.11.
\end{styleBodyxvafter}

\subsection{Valency}
\label{bkm:Ref350344076}
Valency is defined as a “weighting or quantification of verbs in terms of the number of dependents (or arguments or valents) they take” {\citep[5185]{Asher1994}}. Papuan Malay verbs are classified into three classes on the basis of their valency, namely verbs with one, two, or three core arguments. Examples are given in Table  ‎5 .14: verbs that have two or three arguments are listed first, followed by verbs with one argument. Monovalent verbs are further distinguished according to their semantics into dynamic and stative verbs, and other properties, discussed in the following sections.


\begin{stylecaption}
\label{bkm:Ref350499276}Table ‎5.\stepcounter{Table}{\theTable}:  Tri-, bi-, and monovalent verbs
\end{stylecaption}

\begin{tabular}{llll}
\lsptoprule

\multicolumn{4}{l}{ Trivalent verbs}\\
\textitbf{ambil} & ‘fetch’ & \textitbf{kasi} & ‘give’\\
\textitbf{bawa} & ‘bring’ & \textitbf{kirim} & ‘send’\\
\textitbf{bli} & ‘buy’ & \textitbf{minta} & ‘request’\\
\textitbf{ceritra} & ‘tell’ &  & \\
\multicolumn{4}{l}{ Bivalent verbs}\\
\textitbf{antar} & ‘bring’ & \textitbf{kubur} & ‘bury’\\
\textitbf{bunu} & ‘kill’ & \textitbf{lawang} & ‘oppose’\\
\textitbf{cabut} & ‘pull out’ & \textitbf{maki} & ‘abuse (verbally)’\\
\textitbf{dorong} & ‘push’ & \textitbf{mara} & ‘feel angry (about)’\\
\textitbf{ejek} & ‘mock’ & \textitbf{naik} & ‘ascend’\\
\textitbf{ganas} & ‘feel furious (about)’ & \textitbf{pake} & ‘use’\\
\textitbf{ganggu} & ‘disturb’ & \textitbf{rabik} & ‘tear’\\
\textitbf{hela} & ‘haul’ & \textitbf{simpang} & ‘store’\\
\textitbf{ikut} & ‘follow’ & \textitbf{tarik} & ‘pull’\\
\textitbf{jual} & ‘sell’ & \textitbf{usir} & ‘chase away’\\
\multicolumn{4}{l}{ Monovalent dynamic verbs}\\
\textitbf{bernang} & ‘swim’ & \textitbf{kembali} & ‘return’\\
\textitbf{bocor} & ‘leak’ & \textitbf{lari} & ‘run’\\
\textitbf{datang} & ‘come’ & \textitbf{maju} & ‘advance’\\
\textitbf{duduk} & ‘sit’ & \textitbf{mandi} & ‘bathe’\\
\textitbf{gementar} & ‘tremble’ & \textitbf{oleng} & ‘shake’\\
\textitbf{guling} & ‘roll over’ & \textitbf{pergi} & ‘go’\\
\textitbf{hidup} & ‘live’ & \textitbf{sandar} & ‘lean’\\
\textitbf{hosa} & ‘pant’ & \textitbf{sante} & ‘relax’\\
\textitbf{jalang} & ‘walk’ & \textitbf{terbang} & ‘fly’\\
\textitbf{jatu} & ‘fall’ & \textitbf{tinggal} & ‘stay’\\
\multicolumn{4}{l}{ Monovalent stative verbs}\\
\textitbf{abu} & ‘be dusty’ & \textitbf{muda} & ‘be young’\\
\textitbf{bagus} & ‘be good’ & \textitbf{nyamang} & ‘be comfortable’\\
\textitbf{cantik} & ‘be beautiful’ & \textitbf{panas} & ‘be hot’\\
\textitbf{dinging} & ‘be cold’ & \textitbf{puti} & ‘be white’\\
\textitbf{enak} & ‘be pleasant’ & \textitbf{renda} & ‘be low’\\
\textitbf{gila} & ‘be crazy’ & \textitbf{sakit} & ‘be sick’\\
\textitbf{hijow} & ‘be green’ & \textitbf{swak} & ‘be exhausted’\\
\textitbf{jahat} & ‘be bad’ & \textitbf{tinggi} & ‘be tall’\\
\textitbf{kecil} & ‘be small’ & \textitbf{tua} & ‘be old’\\
\textitbf{lema} & ‘be weak’ & \textitbf{waras} & ‘be sane’\\
\lspbottomrule
\end{tabular}

Trivalent verbs have three core arguments, that is, a subject and two grammatical objects. This is illustrated with \textitbf{kasi} ‘give’ in (0). It is important to note, however, that the attested trivalent verbs allow but do not require three syntactic arguments. (For details, see §11.1.3.)


\begin{styleExampleTitle}
Trivalent verbs with three core arguments
\end{styleExampleTitle}

\begin{tabular}{lllll}
\lsptoprule
\label{bkm:Ref346026842}
\gll {dia} {\bluebold{kasi}} {kitong} {daging}\\ %
& \textsc{3sg} & give & \textsc{1pl} & meat\\
\lspbottomrule
\end{tabular}
\ea
\glt 
‘he \bluebold{gave} us (fish) meat’ \textstyleExampleSource{[080919-004-NP.0061]}
\z


Bivalent verbs have two core arguments, a subject and one grammatical object. This is shown with \textitbf{pukul} ‘hit’ in (0) and \textitbf{mara} ‘feel angry (about)’ in (0). Bivalent verbs also allow, but do not require two syntactic arguments. (For details see §11.1.2.)


\begin{styleExampleTitle}
Bivalent verbs with two core arguments
\end{styleExampleTitle}

\begin{tabular}{lllllll}
\lsptoprule
\label{bkm:Ref346026841}
\gll {bapa} {de} {\bluebold{pukul}} {sa} {deng} {pisow}\\ %
& father & \textsc{3sg} & hit & \textsc{1sg} & with & knife\\
\lspbottomrule
\end{tabular}
\ea
\glt 
‘(my) husband \bluebold{hit }me with a knife’ \textstyleExampleSource{[081011-023-Cv.0167]}
\z

\begin{tabular}{llllll}
\lsptoprule
\label{bkm:Ref350762245}
\gll {…} {jadi} {sa} {\bluebold{mara}} {dia}\\ %
&  & so & \textsc{1sg} & feel.angry(.about) & \textsc{3sg}\\
\lspbottomrule
\end{tabular}
\ea
\glt 
‘[he doesn’t report to me in a good way,] so I \bluebold{feel angry about} him’ \textstyleExampleSource{[081011-020-Cv.0107]}
\z


Monovalent verbs have only one core argument. They are further divided into dynamic and stative verbs. Dynamic verbs such as \textitbf{lari} ‘run’ in (0) denote actions involving one participant, while stative verbs, such as \textitbf{besar} ‘be big’ or \textitbf{kecil} ‘be small’ in (0), express states or more time-stable properties.


\begin{styleExampleTitle}
Monovalent verbs with one core argument
\end{styleExampleTitle}

\begin{tabular}{llllll}
\lsptoprule
(\stepcounter{}{\the}) & Nofita & de & \bluebold{lari} & dari & saya\\
& Nofita & \textsc{3sg} & run & from & \textsc{1sg}\\
\lspbottomrule
\end{tabular}
\ea
\glt 
‘Nofita \bluebold{ran (away)} from me’ \textstyleExampleSource{[081025-006-Cv.0322]}
\z

\begin{tabular}{llllllll}
\lsptoprule
\label{bkm:Ref350268140}
\gll {kepala} {ni} {\bluebold{besar}} {baru} {badang} {ni} {\bluebold{kecil}}\\ %
& head & \textsc{d.prox} & be.big & and.then & body & \textsc{d.prox} & be.small\\
\lspbottomrule
\end{tabular}
\ea
\glt 
‘(his) head here \bluebold{is big} but (his) body here \bluebold{is small}’ \textstyleExampleSource{[081025-006-Cv.0278]}
\z


In the corpus, the class of trivalent verbs is the smallest one with seven, as shown in Table  ‎5 .15. A small majority of attested verbs are bivalent with 535 entries (52\%), while 490 verbs are monovalent (48\%). Most of the monovalent verbs are stative (351/490 – 72\%), while 139 verbs are dynamic (28\%).


\begin{stylecaption}
\label{bkm:Ref350271558}Table ‎5.\stepcounter{Table}{\theTable}:  Verb type frequencies
\end{stylecaption}

\begin{tabular}{lll} & \multicolumn{2}{l}{ Frequencies}\\
\lsptoprule
Verb class & \raggedleft \# & \raggedleft\arraybslash \%\\
\textsc{v.tri} & \raggedleft 7 & \raggedleft\arraybslash 0.7\%\\
\textsc{v.bi} & \raggedleft 535 & \raggedleft\arraybslash 51.8\%\\
\textsc{v.mo} & \raggedleft 490 & \raggedleft\arraybslash 47.5\%\\
\textsc{v.mo}(\textsc{dy}) & \raggedleft (139) & \raggedleft\arraybslash (28.4\%)\\
\textsc{v.mo}(\textsc{st}) & \raggedleft (351) & \raggedleft\arraybslash (71.6\%)\\
Total & \raggedleft 1,032 & \raggedleft\arraybslash 100\%\\
\lspbottomrule
\end{tabular}

In addition, the corpus contains 43 derived monovalent verbs prefixed with \textscItalBold{ter\-} that denote accidental or unintentional actions or events (167 tokens). These lexemes are examined in detail in §3.1.2, and briefly reviewed in §5.4; therefore, they are not further discussed in this section.


\subsection{Predicative and attributive functions}
\label{bkm:Ref350353875}
Verbs can function predicatively as well as attributively. The identified verb classes display clear distributional preferences, however. Dynamic verbs usually function predicatively, and less frequently attributively. Monovalent stative verbs, by contrast, typically occur as adnominal modifiers in noun phrases, although they also have predicative function.



In their predicative uses, verbs act “as ‘comment’ on a given noun as ‘topic’”, using {Dixon’s (1994: 31)} terminology. This typical function of dynamic verbs is demonstrated with bivalent \textitbf{bunu} ‘kill’ in (0). The predicative use of monovalent stative verbs is illustrated with \textitbf{tinggi} ‘be high’ in (0). In the corpus, all dynamic verbs have predicative function, while only 40\% of the stative verbs (139/351) are used predicatively.
\end{styleBodyvvafter}

\begin{styleExampleTitle}
Predicative uses
\end{styleExampleTitle}

\begin{tabular}{llllll}
\lsptoprule
\label{bkm:Ref350345027}
\gll {bapa} {Iskia} {dong} {\bluebold{bunu}} {\bluebold{babi}}\\ %
& father & Iskia & \textsc{3pl} & kill & pig\\
\lspbottomrule
\end{tabular}
\ea
\glt 
‘father Iskia and his companions \bluebold{killed a pig}’ \textstyleExampleSource{[080917-008-NP.0120]}
\z

\begin{tabular}{llll}
\lsptoprule
\label{bkm:Ref350345028}
\gll {glombang} {itu} {\bluebold{tinggi}}\\ %
& wave & \textsc{d.dist} & be.high\\
\lspbottomrule
\end{tabular}
\ea
\glt 
‘that wave \bluebold{was high}’ \textstyleExampleSource{[080923-015-CvEx.0016]}
\z


In their attributive function within noun phrases, the modifying verbs serve to specify or restrict “the reference of the noun”, {in Dixon’s (1994: 31)} terminology. Cross-linguistically, this is achieved in one of two ways, as {\citet{Dixon1994} points out}. One option is verb-via-juxtaposition modification; that is, the modifying verb is directly juxtaposed to a noun in a noun phrase. The second option is “verb-via-relative-clause modification” {\citep[19]{Dixon2004}; that is, modification is achieved by means of a relative clause. }Papuan Malay also makes use of these two options, as illustrated in (0) to (0)



The first option of verb-via-juxtaposition modification is illustrated in (0). The examples show that all verb types can occur in noun phrases as adnominal modifiers in post-head position, both with agentive and non-agentive head nominals (the examples in (0) and (0) are elicited).
\end{styleBodyvvafter}

\begin{styleExampleTitle}
Attributive uses: Verb-via-juxtaposition modification
\end{styleExampleTitle}

\begin{tabular}{llllllll} & \multicolumn{7}{l}{Trivalent verbs (elicited)}\\
\lsptoprule
\label{bkm:Ref353442066}\label{bkm:Ref353442103}
\gll {\label{bkm:Ref348015466}} {\bluebold{sifat}} {\bluebold{kasi}} {} {\label{bkm:Ref348015467}} {\bluebold{tukang}} {\bluebold{bli}}\\ %
&  & spirit & give &  &  & craftsman & buy\\
&  & \multicolumn{2}{l}{‘disposition of giving’} &  &  & \multicolumn{2}{l}{‘one who likes to buy’}\\
& \multicolumn{7}{l}{Bivalent verbs}\\
&  & \bluebold{ana} & \bluebold{angkat} &  &  & \bluebold{tukang} & \bluebold{minum}\\
&  & child & lift &  &  & craftsman & drink\\
&  & \multicolumn{2}{l}{‘adopted child’} &  &  & \multicolumn{2}{l}{‘drunkard’}\\
& \multicolumn{7}{l}{Monovalent dynamic verbs}\\
&  & \bluebold{sabung} & \bluebold{mandi} &  &  & \bluebold{tukang} & \bluebold{jalang}\\
&  & soap & bathe &  &  & craftsman & walk\\
&  & \multicolumn{2}{l}{‘bathing soap’} &  &  & \multicolumn{2}{l}{‘one who likes to walk around’}\\
& \multicolumn{7}{l}{Monovalent stative verbs}\\
&  & \bluebold{bua} & \bluebold{mera} &  &  & \bluebold{orang} & \bluebold{tua}\\
&  & fruit & be.red &  &  & person & be.old\\
&  & \multicolumn{2}{l}{‘red fruit’} &  &  & \multicolumn{2}{l}{‘old person’}\\
\lspbottomrule
\end{tabular}

The second option of modifying nouns within a noun phrase is by placing the verb within a relative clause, as illustrated in (0) to (0). This verb-via-relative-clause modification typically applies to dynamic verbs, such as (monotransitively used) trivalent \textitbf{bawa} ‘bring’ in the elicited example in (0), bivalent \textitbf{kawing} ‘marry unofficially’ in (0), or monovalent dynamic \textitbf{tinggal} ‘stay’ in (0).


\begin{styleExampleTitle}
Attributive uses: Verb-via-relative-clause modification
\end{styleExampleTitle}

\begin{tabular}{llllllll}
\lsptoprule
\label{bkm:Ref350247444}
\gll {ojek} {\bluebold{yang}} {\bluebold{bawa}} {tete} {tu} {su} {pulang}\\ %
& motorbike.taxi & \textsc{rel} & bring & grandfather & \textsc{d.dist} & already & go.home\\
\lspbottomrule
\end{tabular}
\ea
\glt 
‘that motorbike taxi driver \bluebold{who brought} grandfather has already returned home’ \textstyleExampleSource{[Elicited MY131119.001]}
\z

\begin{tabular}{llllllllll}
\lsptoprule
\label{bkm:Ref350247443}
\gll {orang} {Papua} {\bluebold{yang}} {\bluebold{kawing}} {orang} {pendatang} {de} {tinggal} {…}\\ %
& person & Papua & \textsc{rel} & marry.inofficially & person & stranger & \textsc{3sg} & stay & \\
\lspbottomrule
\end{tabular}
\ea
\glt 
‘a Papuan person \bluebold{who married} a stranger, he/she’ll stay (in Papua)’ \textstyleExampleSource{[081029-005-Cv.0046]}
\z

\begin{tabular}{llllllll}
\lsptoprule
\label{bkm:Ref350247442}
\gll {…} {buat} {sodara{\Tilde}sodara} {\bluebold{yang}} {\bluebold{tinggal}} {di} {kampung}\\ %
&  & for & \textsc{rdp}{\Tilde}sibling & \textsc{rel} & stay & at & village\\
\lspbottomrule
\end{tabular}
\ea
\glt 
‘[we cut (the pig meat) up that day, we divided (it) for us who cut (it) up that day, (and) then] for the relatives and friends \bluebold{who live} in the village’ \textstyleExampleSource{[080919-003-NP.0014]}
\z

\begin{tabular}{lllllllll}
\lsptoprule
\label{bkm:Ref350247441}
\gll {de} {ada} {potong} {ikang} {\bluebold{yang}} {\bluebold{besar}} {di} {pante}\\ %
& \textsc{3sg} & exist & cut & fish & \textsc{rel} & be.big & at & coast\\
\lspbottomrule
\end{tabular}
\ea
\glt 
‘at the beach he was cutting up a fish \bluebold{that was big}’ \textstyleExampleSource{[080919-004-NP.0061]}
\z


All verb types can be used attributively. However, when comparing the attested attributively used verb tokens across the two types of noun phrase modification, a clear pattern emerges, shown in Table  ‎5 .16. The vast majority of attributively used monovalent stative verbs occurs in noun phrases involving verb-via-juxtaposition modification, although stative verbs also occur in verb-via-relative-clause modification, such as \textitbf{besar} ‘be big’ in (0). The vast majority of attributively used dynamic verbs, by contrast, occurs in noun phrases involving verb-via-relative-clause modification. Cross-linguistically these preferences are rather common {\citep[31]{Dixon1994}}.



So far 612 noun phrases have been identified which involve verb-via-juxtaposition modification, and 834 noun phrases with verb-via-relative-clause modification. This total of 1,446 noun phrases involves 36 noun phrases (2.5\%) which are formed with seven distinct trivalent verbs, 432 noun phrases (29.9\%) formed with 146 distinct bivalent verbs, 170 noun phrases (11.8\%) formed with 37 distinct monovalent dynamic verbs, and 808 noun phrases (55.9\%) are formed with 146 distinct monovalent stative verbs. About two thirds of the attested 808 attributively used monovalent stative verb tokens, occur in noun phrases with verb-via-juxtaposition modification (520/808 – 64\%), while only about one third occurs in noun phrases with verb-via-relative-clause modification (288/808 – 36\%). The opposite holds for dynamic verbs. The vast majority of attributively used verb tokens occurs in noun phrases with verb-via-relative-clause modification: 35/36 trivalent verb tokens (97\%), 371/432 bivalent verb tokens (86\%), and 140/170 monovalent dynamic verb tokens (82\%). By contrast only few verbs are used in noun phrases with verb-via-juxtaposition modification: 1/36 trivalent verb tokens (3\%), 61/432 bivalent verb tokens (14\%), and 30/170 monovalent dynamic verb tokens (18\%).
\end{styleBodyvvafter}

\begin{stylecaption}
\label{bkm:Ref335933693}Table ‎5.\stepcounter{Table}{\theTable}:  Attributive uses of verbs within noun phrases\footnote{\\
\\
\\
\\
\\
\\
\\
\\
\\
\\
\\
\\
As percentages are rounded to one decimal place, they do not always add up to 100\%.\\
\\
\\
}
\end{stylecaption}

\tablehead{ & \multicolumn{2}{l}{ Token frequencies} & \multicolumn{2}{l}{ Type frequencies}\\
}
\begin{tabular}{lllll} & \multicolumn{2}{l}{ Via juxtaposition} & \multicolumn{2}{l}{ Different verbs}\\
\lsptoprule
Verb class & \raggedleft \# & \raggedleft \% & \raggedleft \# & \raggedleft\arraybslash \%\\
\textsc{v.tri} & \raggedleft 1 & \raggedleft 0.2\% & \raggedleft 1 & \raggedleft\arraybslash 1.0\%\\
\textsc{v.bi} & \raggedleft 61 & \raggedleft 10.0\% & \raggedleft 27 & \raggedleft\arraybslash 26.0\%\\
\textsc{v.mo}(\textsc{dy}) & \raggedleft 30 & \raggedleft 4.9\% & \raggedleft 10 & \raggedleft\arraybslash 9.6\%\\
\textsc{v.mo}(\textsc{st}) & \raggedleft 520 & \raggedleft 85.0\% & \raggedleft 66 & \raggedleft\arraybslash 63.5\%\\
Total & \raggedleft 612 & \raggedleft 100.0\% & \raggedleft 104 & \raggedleft\arraybslash 100.0\%\\
& \multicolumn{2}{l}{ Via relative-clause} & \multicolumn{2}{l}{ Different verbs}\\
Verb class & \raggedleft \# & \raggedleft \% & \raggedleft \# & \raggedleft\arraybslash \%\\
\textsc{v.tri} & \raggedleft 35 & \raggedleft 4.2\% & \raggedleft 5 & \raggedleft\arraybslash 2.2\%\\
\textsc{v.bi} & \raggedleft 371 & \raggedleft 44.5\% & \raggedleft 119 & \raggedleft\arraybslash 51.5\%\\
\textsc{v.mo}(\textsc{dy}) & \raggedleft 140 & \raggedleft 16.8\% & \raggedleft 27 & \raggedleft\arraybslash 11.7\%\\
\textsc{v.mo}(\textsc{st}) & \raggedleft 288 & \raggedleft 34.5\% & \raggedleft 80 & \raggedleft\arraybslash 34.6\%\\
Total & \raggedleft 834 & \raggedleft 100.0\% & \raggedleft 231 & \raggedleft\arraybslash 100.0\%\\
& \multicolumn{2}{l}{ Overall totals} & \multicolumn{2}{l}{ Overall totals}\\
Verb class & \raggedleft \# & \raggedleft \% & \raggedleft \# & \raggedleft\arraybslash \%\\
\textsc{v.tri} & \raggedleft 36 & \raggedleft 2.5\% & \raggedleft 6 & \raggedleft\arraybslash 1.8\%\\
\textsc{v.bi} & \raggedleft 432 & \raggedleft 29.9.\% & \raggedleft 146 & \raggedleft\arraybslash 43.6\%\\
\textsc{v.mo}(\textsc{dy}) & \raggedleft 170 & \raggedleft 11.8\% & \raggedleft 37 & \raggedleft\arraybslash 11.0\%\\
\textsc{v.mo}(\textsc{st}) & \raggedleft 808 & \raggedleft 55.9\% & \raggedleft 146 & \raggedleft\arraybslash 43.6\%\\
Total & \raggedleft 1,446 & \raggedleft 100.0\% & \raggedleft 335 & \raggedleft\arraybslash 100.0\%\\
\lspbottomrule
\end{tabular}
\subsection{Adverbial modification}
\label{bkm:Ref350353876}
In their predicative uses, tri-, bi-, and monovalent verbs can be modified with adverbs, as shown in (0) to (0). In (0) to (0), the temporal adverb \textitbf{langsung} ‘immediately’ modifies trivalent \textitbf{kasi} ‘give’, bivalent \textitbf{tanya} ‘ask’, monovalent dynamic \textitbf{pulang} ‘go home’, and stative \textitbf{basa} ‘be wet’, respectively; the example in (0) is elicited.


\begin{styleExampleTitle}
Adverbial modification with temporal adverb \textitbf{langsung} ‘immediately’
\end{styleExampleTitle}

\begin{tabular}{lllllll}
\lsptoprule
\label{bkm:Ref345673731}
\gll {pace} {dong} {\bluebold{langsung}} {\bluebold{kasi}} {dia} {senter}\\ %
& man & \textsc{3pl} & immediately & give & \textsc{3sg} & flashlight\\
\lspbottomrule
\end{tabular}
\ea
\glt 
‘the men \bluebold{immediately gave} him a flashlight’ \textstyleExampleSource{[Elicited BR130221.013]}
\z

\begin{tabular}{lllll}
\lsptoprule
(\stepcounter{}{\the}) & sa & \bluebold{langsung} & \bluebold{tanya} & dorang\\
& \textsc{1sg} & immediately & ask & \textsc{3pl}\\
\lspbottomrule
\end{tabular}
\ea
\glt 
I \bluebold{immediately asked} them’ \textstyleExampleSource{[080919-007-CvNP.0045]}
\z

\begin{tabular}{llll}
\lsptoprule
(\stepcounter{}{\the}) & sa & \bluebold{langsung} & \bluebold{pulang}\\
& \textsc{1sg} & immediately & go.home\\
\lspbottomrule
\end{tabular}
\ea
\glt 
‘I \bluebold{went home immediately}’ \textstyleExampleSource{[081014-008-CvNP.0018]}
\z

\begin{tabular}{llll}
\lsptoprule
\label{bkm:Ref345664034}
\gll {bapa} {\bluebold{langsung}} {\bluebold{diam}}\\ %
& father & immediately & be.quiet\\
\lspbottomrule
\end{tabular}
\ea
\glt 
‘the gentleman \bluebold{was immediately quiet}’ \textstyleExampleSource{[080917-010-CvEx.0213]}
\z


Along similar lines, frequency adverb \textitbf{lagi} ‘again, also’ modifies the verbs in (0) to (0); the example in (0) is elicited. (For more details on adverbs see §5.4.)


\begin{styleExampleTitle}
Adverbial modification with frequency adverb \textitbf{lagi} ‘again, also’
\end{styleExampleTitle}

\begin{tabular}{llllll}
\lsptoprule
\label{bkm:Ref349400613}
\gll {Dodo} {\bluebold{ambil}} {Agus} {air} {\bluebold{lagi}}\\ %
& Dodo & fetch & Agus & water & again\\
\lspbottomrule
\end{tabular}
\ea
\glt 
‘Dodo \bluebold{fetched} water for Agus \bluebold{again}’ \textstyleExampleSource{[Elicited BR130409.001]}
\z

\begin{tabular}{lllll}
\lsptoprule
(\stepcounter{}{\the}) & sa & \bluebold{tampeleng} & dia & \bluebold{lagi}\\
& \textsc{1sg} & slap.on.face.or.ears & \textsc{3sg} & again\\
\lspbottomrule
\end{tabular}
\ea
\glt 
‘I \bluebold{slapped} him \bluebold{across the face again}’ \textstyleExampleSource{[081013-002-Cv.0007]}
\z

\begin{tabular}{llllll}
\lsptoprule
(\stepcounter{}{\the}) & nanti & Lodia & dong & \bluebold{datang} & \bluebold{lagi}\\
& very.soon & Lodia & \textsc{3pl} & come & again\\
\lspbottomrule
\end{tabular}
\ea
\glt 
‘very soon Lodia and her companions will \bluebold{also come}’ \textstyleExampleSource{[081006-016-Cv.0010]}
\z

\begin{tabular}{lllllllll}
\lsptoprule
\label{bkm:Ref349400610}
\gll {…} {sampe} {mungking} {dua} {taung} {baru} {\bluebold{rame}} {\bluebold{lagi}}\\ %
&  & until & maybe & two & year & and.then & be.crowded & again\\
\lspbottomrule
\end{tabular}
\ea
\glt
‘[it goes on like that] for maybe two years before (the situation gets) \bluebold{lively again}’ \textstyleExampleSource{[081025-004-Cv.0102]}
\end{styleFreeTranslEngxvpt}

\subsection{Intensification}
\label{bkm:Ref350357624}
In their predicative uses, monovalent stative and bivalent verbs can be intensified with the degree adverbs \textitbf{skali} ‘very’ or \textitbf{terlalu} ‘too’, as shown in (0) to (0). While \textitbf{skali} ‘very’ follows the verb as in (0) and (0), \textitbf{terlalu} ‘too’ precedes it as in (0) and (0). Intensification of predicatively used monovalent dynamic and trivalent verbs is unattested in the corpus. Furthermore, intensification of attributively used verbs is unattested. (For details on degree adverbs see §5.4.7.)


\begin{styleExampleTitle}
Intensification
\end{styleExampleTitle}

\begin{tabular}{llllllll}
\lsptoprule
\label{bkm:Ref345664040}
\gll {sa} {\bluebold{snang}} {\bluebold{skali}} {dong} {pu} {cara} {masak}\\ %
& \textsc{1sg} & feel.happy(.about) & very & \textsc{3pl} & \textsc{poss} & manner & cook\\
\lspbottomrule
\end{tabular}
\ea
\glt 
‘I \bluebold{very (much) enjoy} their way of cooking’ \textstyleExampleSource{[081014-017-CvPr.0029]}
\z

\begin{tabular}{llll}
\lsptoprule
\label{bkm:Ref345664039}
\gll {Aris} {\bluebold{tinggi}} {\bluebold{skali}}\\ %
& Aris & be.high & very\\
\lspbottomrule
\end{tabular}
\ea
\glt 
‘Aris\bluebold{ is very tall}’ \textstyleExampleSource{[080922-001b-CvPh.0026]}
\z

\begin{tabular}{lllllll}
\lsptoprule
\label{bkm:Ref345664042}
\gll {…} {ade} {kecil} {\bluebold{terlalu}} {\bluebold{menangis}} {kitorang}\\ %
&  & ySb & be.small & too & cry & \textsc{1pl}\\
\lspbottomrule
\end{tabular}
\ea
\glt 
‘[Hana’s husband didn’t come along,] the small younger sibling \bluebold{cried too much} (for) us’ \textstyleExampleSource{[080921-002-Cv.0008]}
\z

\begin{tabular}{llllll}
\lsptoprule
\label{bkm:Ref345664041}
\gll {sa} {liat} {mama} {\bluebold{terlalu}} {\bluebold{baik}}\\ %
& \textsc{1sg} & see & mother & exceedingly & be.good\\
\lspbottomrule
\end{tabular}
\ea
\glt 
‘I see you (‘mother’) \bluebold{are too good}’ \textstyleExampleSource{[081115-001a-Cv.0324]}
\z


As mentioned, intensification of monovalent dynamic verbs is unattested in the corpus. According to one consultant, though, it is possible to intensify them with the expressions \textitbf{terlalu banyak} ‘too much’ or \textitbf{terlalu sedikit} ‘too little’, as in \textitbf{terlalu banyak tidor} ‘sleep too much’ in the elicited example in (0) and \textitbf{terlalu sedikit lari} in the elicited example in (0).


\begin{styleExampleTitle}
Grading of monovalent dynamic verbs with \textitbf{terlalu banyak}/\textitbf{sedikit} ‘too much/little’
\end{styleExampleTitle}

\begin{tabular}{llllll}
\lsptoprule
\label{bkm:Ref353788110}
\gll {Dodo} {de} {\bluebold{terlalu}} {\bluebold{banyak}} {\bluebold{tidor}}\\ %
& Dodo & \textsc{3sg} & too & many & sleep\\
\lspbottomrule
\end{tabular}
\ea
\glt 
‘Dodo \bluebold{sleeps too much}’ \textstyleExampleSource{[Elicited BR130410.005]}
\z

\begin{tabular}{llllll}
\lsptoprule
\label{bkm:Ref353788112}
\gll {Dodo} {de} {\bluebold{terlalu}} {\bluebold{sedikit}} {\bluebold{lari}}\\ %
& Dodo & \textsc{3sg} & too & few & run\\
\lspbottomrule
\end{tabular}
\ea
\glt 
‘Dodo \bluebold{runs too little}’ \textstyleExampleSource{[Elicited BR130410.008]}
\z


In addition, one of the consultants came up with the two examples in (0) and (0), respectively, in which dynamic \textitbf{lari} ‘run’ and \textitbf{tunduk} ‘bow’ are directly modified with \textitbf{terlalu} ‘too’. In (0), however, \textitbf{lari} means ‘deviate’ rather than ‘run’, and \textitbf{tunduk} ‘bow’ in (0) receives the stative reading ‘be obedient’.


\begin{styleExampleTitle}
Grading of monovalent dynamic verbs with \textitbf{terlalu} ‘too’
\end{styleExampleTitle}

\begin{tabular}{lllllllllll}
\lsptoprule
\label{bkm:Ref353783792}
\gll {\multicolumn{2}{l}{prahu}} {\multicolumn{2}{l}{ini}} {pu} {ukurang} {\bluebold{terlalu}} {\bluebold{lari}} {dari} {ukurang}\\ %
& \multicolumn{2}{l}{boat} & \multicolumn{2}{l}{\textsc{d.prox}} & \textsc{poss} & measurement & too & run & from & measurement\\
& yang & \multicolumn{2}{l}{ko} & \multicolumn{7}{l}{kasi}\\
& \textsc{rel} & \multicolumn{2}{l}{\textsc{2sg}} & \multicolumn{7}{l}{give}\\
\lspbottomrule
\end{tabular}
\ea
\glt 
‘the size of this boat \bluebold{deviates too much} from the size that you gave’ \textstyleExampleSource{[Elicited BR130410.017]}
\z

\begin{tabular}{lllll}
\lsptoprule
\label{bkm:Ref354989511}
\gll {Agus} {de} {\bluebold{terlalu}} {\bluebold{tunduk}}\\ %
& Agus & \textsc{3sg} & too & bow\\
\lspbottomrule
\end{tabular}
\ea
\glt 
‘Agus \bluebold{is too obedient}’ \textstyleExampleSource{[Elicited BR130410.004]}
\z


When examining the attested intensified monovalent stative and bivalent verb tokens as to whether they are intensified with \textitbf{skali} ‘very’ or with \textitbf{terlalu} ‘too’, the data show clear distributional preferences, presented in Table  ‎5 .17. The corpus contains 155 verb phrases, made up of 80 different verbs, in which \textitbf{skali} ‘very’ intensifies a verb. Most of these verbs are stative ones (81\%), accounting for 80\% of the \textitbf{skali}{}-intensification tokens. The corpus also contains 33 verb phrases, formed with 27 different verbs, in which \textitbf{terlalu} ‘too’ intensifies a verb. Again, most of the intensified verbs are stative ones (74\%) accounting for 73\% of the \textitbf{terlalu}{}-intensification tokens.


\begin{stylecaption}
\label{bkm:Ref349744834}Table ‎5.\stepcounter{Table}{\theTable}:  Intensification of verbs
\end{stylecaption}

\tablehead{ & \multicolumn{2}{l}{ Token frequencies} & \multicolumn{2}{l}{ Type frequencies}\\
}
\begin{tabular}{lllll} & \multicolumn{2}{l}{ \textitbf{skali}{}-intensification} & \multicolumn{2}{l}{ Different verbs}\\
\lsptoprule
Verb class & \raggedleft \# & \raggedleft \% & \raggedleft \# & \raggedleft\arraybslash \%\\
\textsc{v.tri} & \raggedleft 0 & \raggedleft {}-{}-{}- & \raggedleft 0 & \raggedleft\arraybslash {}-{}-{}-\\
\textsc{v.bi} & \raggedleft 31 & \raggedleft 20.0\% & \raggedleft 15 & \raggedleft\arraybslash 18.7\%\\
\textsc{v.mo}(\textsc{dy}) & \raggedleft 0 & \raggedleft {}-{}-{}- & \raggedleft 0 & \raggedleft\arraybslash {}-{}-{}-\\
\textsc{v.mo}(\textsc{st}) & \raggedleft 124 & \raggedleft 80.0\% & \raggedleft 65 & \raggedleft\arraybslash 81.3\%\\
Total & \raggedleft 155 & \raggedleft 100.0\% & \raggedleft 80 & \raggedleft\arraybslash 100.0\%\\
& \multicolumn{2}{l}{ \textitbf{terlalu}{}-intensification} & \multicolumn{2}{l}{ Different verbs}\\
Verb class & \raggedleft \# & \raggedleft \% & \raggedleft \# & \raggedleft\arraybslash \%\\
\textsc{v.tri} & \raggedleft 0 & \raggedleft {}-{}-{}- & \raggedleft 0 & \raggedleft\arraybslash {}-{}-{}-\\
\textsc{v.bi} & \raggedleft 9 & \raggedleft 27.3\% & \raggedleft 7 & \raggedleft\arraybslash 25.9\%\\
\textsc{v.mo}(\textsc{dy}) & \raggedleft 0 & \raggedleft {}-{}-{}- & \raggedleft 0 & \raggedleft\arraybslash {}-{}-{}-\\
\textsc{v.mo}(\textsc{st}) & \raggedleft 24 & \raggedleft 72.7\% & \raggedleft 20 & \raggedleft\arraybslash 74.1\%\\
Total & \raggedleft 33 & \raggedleft 100.0\% & \raggedleft 27 & \raggedleft\arraybslash 100.0\%\\
\lspbottomrule
\end{tabular}
\subsection{Grading}
\label{bkm:Ref348020285}
In their predicative uses, monovalent stative and bivalent verbs can occur with grading adverbs, as shown in (0) to (0), whereas grading of monovalent dynamic and trivalent verbs is unattested. The comparative degree is marked with the grading adverb \textitbf{lebi} ‘more’ and the superlative degree with \textitbf{paling} ‘most’; both adverbs precede the verb. (For details on degree adverbs see §5.4.7; for details on comparative clauses see §11.5.)


\begin{styleExampleTitle}
Grading of bivalent verbs
\end{styleExampleTitle}

\begin{tabular}{lllllll}
\lsptoprule
\label{bkm:Ref348018007}
\gll {a,} {dong} {mala} {\bluebold{lebi}} {\bluebold{sayang}} {saya}\\ %
& ah! & \textsc{3pl} & in.fact & more & love & \textsc{1sg}\\
\lspbottomrule
\end{tabular}
\ea
\glt 
‘ah, they actually \bluebold{loved} me \bluebold{more}’ \textstyleExampleSource{[Elicited BR130221.034]}\footnote{\\
\\
\\
\\
\\
\\
\\
\\
\\
\\
\\
\\
The elicited example in (0) is the corrected version of the original recording \textitbf{dong mana lebi sayang saya} ‘they actually[\textsc{spm}] loved me more’ [081110-008-NPHt.0021]. That is, the speaker mispronounced \textitbf{mala} ‘in.fact’, realizing it as \textitbf{mana}.\\
\\
\\
}
\z

\begin{tabular}{llllll}
\lsptoprule
(\stepcounter{}{\the}) & tempat & itu & sa & \bluebold{paling} & \bluebold{takut}\\
& place & \textsc{d.dist} & \textsc{1sg} & most & feel.afraid(.of)\\
\lspbottomrule
\end{tabular}
\ea
\glt 
‘that place I \bluebold{feel most afraid of}’ \textstyleExampleSource{[081025-006-Cv.0285]}
\z

\begin{styleExampleTitle}
Grading of monovalent stative verbs
\end{styleExampleTitle}

\begin{tabular}{lllllllll}
\lsptoprule
(\stepcounter{}{\the}) & yo & kaka, & itu & yang & \bluebold{lebi} & \bluebold{baik} & untuk & saya\\
& yes & oSb & \textsc{d.dist} & \textsc{rel} & more & be.good & for & \textsc{1sg}\\
\lspbottomrule
\end{tabular}
\ea
\glt 
[Talking about her husband:] ‘yes older sibling, that (is the one) who is \bluebold{better} for me’ \textstyleExampleSource{[081110-008-CvNP.0178]}
\z

\begin{tabular}{lllll}
\lsptoprule
\label{bkm:Ref345596393}
\gll {puri} {tu} {\bluebold{paling}} {\bluebold{besar}}\\ %
& anchovy-like.fish & \textsc{d.dist} & most & be.big\\
\lspbottomrule
\end{tabular}
\ea
\glt 
‘that anchovy-like fish is \bluebold{the biggest}’ \textstyleExampleSource{[080927-003-Cv.0002]}
\z


Again, monovalent dynamic verbs differ from monovalent stative and bivalent verbs in that they are not directly modified with a grading adverb. Instead they are modified with \textitbf{lebi banyak} to indicate comparative degree, as in \textitbf{lebi banyak bertriak} ‘scream more’ in the elicited example in (0), or with \textitbf{paling banyak} to indicate superlative degree, as in \textitbf{paling banyak tertawa} ‘laugh most’ in the elicited example in (0).


\begin{styleExampleTitle}
Grading of monovalent dynamic verbs
\end{styleExampleTitle}

\begin{tabular}{lllllll}
\lsptoprule
\label{bkm:Ref345596392}
\gll {Dodo} {\bluebold{lebi}} {\bluebold{banyak}} {\bluebold{bertriak}} {dari} {Agus}\\ %
& Dodo & more & many & scream & with & Agus\\
\lspbottomrule
\end{tabular}
\ea
\glt 
‘Dodo \bluebold{screams more} than Agus’ \textstyleExampleSource{[Elicited BR130221.025]}
\z

\begin{tabular}{lllll}
\lsptoprule
\label{bkm:Ref348018005}
\gll {Dodo} {\bluebold{paling}} {\bluebold{banyak}} {\bluebold{tertawa}}\\ %
& Dodo & most & many & scream\\
\lspbottomrule
\end{tabular}
\ea
\glt 
‘Dodo \bluebold{laughs most}’ \textstyleExampleSource{[Elicited BR130221.030]}
\z


With respect to the frequencies of the monovalent stative and bivalent verbs in comparative constructions, the data indicate a clear pattern, presented in Table  ‎5 .18. The vast majority of graded verbs are monovalent stative ones. The corpus contains 54 \textitbf{lebi}{}-comparative constructions, formed with 22 different verbs. Of these, 77\% are monovalent stative, accounting for 89\% of the attested comparative constructions. In addition, the corpus contains 46 \textitbf{paling}{}-superlative constructions, formed with 30 different verbs. Again, most of these verbs are monovalent stative (80\%) which account for 83\% of the superlative constructions. Cross-linguistically, this distributional pattern corresponds to the “prototypical comparative scheme” in which the parameter of comparison “is typically expressed by an adjective, in a language with a large open class of adjectives; or else by a stative verb (with an adjective-like meaning)” {\citep[787]{Dixon2008}}.


\begin{stylecaption}
\label{bkm:Ref349737405}Table ‎5.\stepcounter{Table}{\theTable}:  Grading of verbs
\end{stylecaption}

\tablehead{ & \multicolumn{2}{l}{ Token frequencies} & \multicolumn{2}{l}{ Type frequencies}\\
}
\begin{tabular}{lllll} & \multicolumn{2}{l}{ \textsc{cmpr}{}-constructions} & \multicolumn{2}{l}{ Different verbs}\\
\lsptoprule
Verb class & \raggedleft \# & \raggedleft \% & \raggedleft \# & \raggedleft\arraybslash \%\\
\textsc{v.tri} & \raggedleft 0 & \raggedleft {}-{}-{}- & \raggedleft 0 & \raggedleft\arraybslash {}-{}-{}-\\
\textsc{v.bi} & \raggedleft 6 & \raggedleft 11.1\% & \raggedleft 5 & \raggedleft\arraybslash 22.7\%\\
\textsc{v.mo}(\textsc{dy}) & \raggedleft 0 & \raggedleft {}-{}-{}- & \raggedleft 0 & \raggedleft\arraybslash {}-{}-{}-\\
\textsc{v.mo}(\textsc{st}) & \raggedleft 48 & \raggedleft 88.9\% & \raggedleft 17 & \raggedleft\arraybslash 77.3\%\\
Total & \raggedleft 54 & \raggedleft 100\% & \raggedleft 22 & \raggedleft\arraybslash 100\%\\
& \multicolumn{2}{l}{ \textsc{supl}{}-constructions} & \multicolumn{2}{l}{ Different verbs}\\
Verb class & \raggedleft \# & \raggedleft \% & \raggedleft \# & \raggedleft\arraybslash \%\\
\textsc{v.tri} & \raggedleft 0 & \raggedleft {}-{}-{}- & \raggedleft 0 & \raggedleft\arraybslash {}-{}-{}-\\
\textsc{v.bi} & \raggedleft 8 & \raggedleft 17.4\% & \raggedleft 6 & \raggedleft\arraybslash 20.0\%\\
\textsc{v.mo}(\textsc{dy}) & \raggedleft 0 & \raggedleft {}-{}-{}- & \raggedleft 0 & \raggedleft\arraybslash {}-{}-{}-\\
\textsc{v.mo}(\textsc{st}) & \raggedleft 38 & \raggedleft 82.6\% & \raggedleft 24 & \raggedleft\arraybslash 80.0\%\\
Total & \raggedleft 46 & \raggedleft 100\% & \raggedleft 30 & \raggedleft\arraybslash 100\%\\
\lspbottomrule
\end{tabular}
\subsection{Negation}
\label{bkm:Ref350357337}
Verbs are negated with \textitbf{tida} ‘\textsc{neg}’ or \textitbf{tra} ‘\textsc{neg}’.\footnote{\\
\\
\\
\\
\\
\\
\\
\\
\\
\\
\\
\\
\label{bkm:Ref434860809}The negator \textitbf{bukang} ‘\textsc{neg}’ also occurs in verbal clauses. However, it does not negate the verb as \textitbf{tida/tra} ‘\textsc{neg}’ does. Instead \textitbf{bukang} ‘\textsc{neg}’ has scope over the entire proposition and expresses contrastive negation of that proposition as a whole (for details see §13.1.2).\\
\\
\\
} This is demonstrated with trivalent \textitbf{kasi} ‘give’ in (0), bivalent \textitbf{pake} ‘use’ in (0), monovalent dynamic \textitbf{datang} ‘come’ in (0), and monovalent stative \textitbf{baik} ‘be good’ in (0). These examples also illustrate that both negators are used interchangeably (for more details on negation see §13.1).
\end{styleBodyxafter}

\begin{tabular}{lllllllllll}
\lsptoprule
\label{bkm:Ref350256905}
\gll {kaka} {su} {bilang} {de} {begitu,} {sa} {\bluebold{tra}} {\bluebold{kasi}} {ko} {jempol}\\ %
& oSb & already & say & \textsc{3sg} & like.that & \textsc{1sg} & \textsc{neg} & give & \textsc{2sg} & thumb\\
\lspbottomrule
\end{tabular}
\ea
\glt 
‘I (‘older sibling’) already told him like that, ‘I \bluebold{won’t give} you a thumbs up’’ \textstyleExampleSource{[081115-001a-Cv.0042]}
\z

\begin{tabular}{lllllllllll}
\lsptoprule
\label{bkm:Ref350256904}
\gll {kalo} {saya} {berburu} {\bluebold{tida}} {\bluebold{pake}} {anjing} {malam} {hari} {saya} {kluar}\\ %
& if & \textsc{1sg} & hunt & \textsc{neg} & use & dog & night & day & \textsc{1sg} & go.out\\
\lspbottomrule
\end{tabular}
\ea
\glt 
‘if I hunt and \bluebold{don’t take} dogs, I leave at night’ \textstyleExampleSource{[080919-004-NP.0002]}
\z

\begin{tabular}{llllllll}
\lsptoprule
\label{bkm:Ref350256903}
\gll {de} {\bluebold{tra}} {datang} {…} {de} {\bluebold{tida}} {datang}\\ %
& \textsc{3sg} & \textsc{neg} & come &  & \textsc{3sg} & \textsc{neg} & come\\
\lspbottomrule
\end{tabular}
\ea
\glt 
‘she did \bluebold{not} come … she did \bluebold{not} come’ \textstyleExampleSource{[081010-001-Cv.0204-0205]}
\z

\begin{tabular}{lllllll}
\lsptoprule
\label{bkm:Ref350256902}
\gll {nanti} {dia} {pikir} {saya} {\bluebold{tida}} {\bluebold{baik}}\\ %
& very.soon & \textsc{3sg} & think & \textsc{1sg} & \textsc{neg} & be.good\\
\lspbottomrule
\end{tabular}
\ea
\glt
‘later he’ll think (that) I’m \bluebold{not good}’ \textstyleExampleSource{[080919-004-NP.0052]}
\end{styleFreeTranslEngxvpt}

\subsection{Causative constructions}
\label{bkm:Ref350353877}
Papuan Malay syntactic causatives are monoclausal V\textsubscript{1}V\textsubscript{2} constructions. A causative verb V\textsubscript{1}, encodes the notion of cause, while the V\textsubscript{2} denotes the notion of effect. Two full verbs both of which are still used synchronically function as causative verbs, namely trivalent \textitbf{kasi} ‘give’, with its short form \textitbf{kas}, and bivalent \textitbf{biking} ‘make’. Syntactic causatives have monovalent or bivalent bases, while causative constructions with trivalent verbs are unattested.



In \textitbf{kasi}{}-causatives the V\textsubscript{2} can be bivalent or monovalent, while in \textitbf{biking}{}-causatives the V\textsubscript{2} is always monovalent. (See §11.2 for a detailed discussion of causative constructions.)
\end{styleBodyvafter}


Causative constructions with \textitbf{kasi} ‘give’ are presented in (0) to (0). The V\textsubscript{2} is bivalent \textitbf{masuk} ‘enter’ in (0), monovalent dynamic \textitbf{bangung} ‘wake up’ in (0), and stative \textitbf{sembu} ‘be healed’ in (0). (For more details on \textitbf{kasi}{}-causatives, see §11.2.1.2.)
\end{styleBodyvvafter}

\begin{styleExampleTitle}
Causative constructions with \textitbf{kasi} ‘give’
\end{styleExampleTitle}

\begin{tabular}{lllllll}
\lsptoprule
\label{bkm:Ref349582483}
\gll {dong} {\bluebold{kas}} {\bluebold{masuk}} {korek} {di} {sini}\\ %
& \textsc{3pl} & give & enter & matches & at & \textsc{l.prox}\\
\lspbottomrule
\end{tabular}
\ea
\glt 
‘they \bluebold{inserted} matches here’ (Lit. ‘\bluebold{give to enter}’) \textstyleExampleSource{[081025-006-Cv.0180]}
\z

\begin{tabular}{lllllllll}
\lsptoprule
\label{bkm:Ref349582482}
\gll {sa} {takut} {skali} {jadi} {sa} {\bluebold{kas}} {\bluebold{bangung}} {mama}\\ %
& \textsc{1sg} & feel.afraid(.of) & very & so & \textsc{1sg} & give & wake.up & mother\\
\lspbottomrule
\end{tabular}
\ea
\glt 
‘I felt very afraid, so I \bluebold{woke up} you (‘mother’)’ (Lit. ‘\bluebold{give to wake up}’) \textstyleExampleSource{[080917-008-NP.0031]}
\z

\begin{tabular}{llllllll}
\lsptoprule
\label{bkm:Ref349582481}
\gll {ko} {\bluebold{kasi}} {\bluebold{sembu}} {sa} {punya} {ana} {ini}\\ %
& \textsc{2sg} & give & be.healed & \textsc{1sg} & \textsc{poss} & child & \textsc{d.prox}\\
\lspbottomrule
\end{tabular}
\ea
\glt 
‘[Addressing an evil spirit:] ‘you \bluebold{heal} this child of mine!’ (Lit. ‘\bluebold{give to be healed}’) \textstyleExampleSource{[081006-023-CvEx.0031]}
\z


In causatives with \textitbf{biking} ‘make’, the V\textsubscript{2} is always monovalent. Most often, the monovalent verb is stative, such as \textitbf{pusing} ‘be dizzy, be confused’ in (0). However, \textitbf{biking}{}-causatives can also be formed with non-agentive dynamic bases, such as \textitbf{tenggelam} ‘sink’ in the elicited example in (0). If the causee is inanimate, or animate but helpless, the base can also be agentive dynamic, such as \textitbf{hidup} ‘live’ in the elicited example in (0). (For more details on \textitbf{biking}{}-causatives, see §11.2.1.3; see also examples (0) and (0) in §11.2.1.2, p. \pageref{bkm:Ref436750473}.)


\begin{styleExampleTitle}
Causative constructions with \textitbf{biking} ‘make’
\end{styleExampleTitle}

\begin{tabular}{llllllllll}
\lsptoprule
\label{bkm:Ref388290810}
\gll {yo,} {dong} {dua} {deng} {Wili} {tu} {\bluebold{biking}} {\bluebold{pusing}} {mama}\\ %
& yes & \textsc{3pl} & two & with & Wili & \textsc{d.dist} & make & be.dizzy & mother\\
\lspbottomrule
\end{tabular}
\ea
\glt 
‘yes! he and Wili there \bluebold{worried} (their) mother’ (Lit. ‘\bluebold{make to be dizzy/confused}’) \textstyleExampleSource{[081011-003-Cv.0002]}
\z

\begin{tabular}{lllllll}
\lsptoprule
\label{bkm:Ref349582480}
\gll {banyak} {mati} {di} {lautang,} {\bluebold{biking}} {\bluebold{tenggelam}}\\ %
& many & die & at & ocean & make & sink\\
\lspbottomrule
\end{tabular}
\ea
\glt 
[About people in a container who died in the ocean:] ‘many died in the (open) ocean, (the murderers) \bluebold{sank} (the containers)’ \textstyleExampleSource{[Elicited BR131103.003]}
\z

\begin{tabular}{lllllllllll}
\lsptoprule
\label{bkm:Ref377643831}
\gll {…} {tapi} {dong} {\bluebold{biking}} {bangkit} {dia} {lagi,} {\bluebold{biking}} {\bluebold{hidup}} {dia}\\ %
&  & but & \textsc{3pl} & make & be.resurrected & \textsc{3sg} & again & make & live & \textsc{3sg}\\
\lspbottomrule
\end{tabular}
\ea
\glt 
[About sorcerers who can resurrect the dead:] ‘[he’s already (dead),] but they \bluebold{resurrect} him again, \bluebold{make} him \bluebold{live}’ \textstyleExampleSource{[Elicited BR131103.005]}
\z


Concerning the frequencies of mono- and bivalent verbs in causative constructions, the following pattern emerges. Most of the attested verb tokens in \textitbf{kasi}{}-causative constructions are bivalent or monovalent dynamic ones, whereas the attested verbs in \textitbf{biking}{}-causatives are always monovalent stative ones, as shown in Table  ‎5 .19. The corpus contains 478 \textitbf{kasi}{}-causative constructions, formed with 81 different verbs. Most verbs in \textitbf{kasi}{}-constructions are dynamic one (78\%), including 48\% bivalent and 22\% monovalent dynamic verbs. Together the attested dynamic verbs account for 92\% of the \textitbf{kasi}{}-causatives. By contrast, \textitbf{biking}{}-causatives are always formed with monovalent stative verbs. In all, the corpus contains 25 \textitbf{biking}{}-causatives, formed with 16 different verbs.


\begin{stylecaption}
\label{bkm:Ref349730686}Table ‎5.\stepcounter{Table}{\theTable}:  Causative constructions with \textitbf{kasi} ‘give’ and with \textitbf{biking} ‘make’
\end{stylecaption}

\tablehead{ & \multicolumn{2}{l}{ Token frequencies} & \multicolumn{2}{l}{ Type frequencies}\\
}
\begin{tabular}{lllll} & \multicolumn{2}{l}{ \textitbf{kasi}{}-causatives} & \multicolumn{2}{l}{ Different verbs}\\
\lsptoprule
Verb class & \raggedleft \# & \raggedleft \% & \raggedleft \# & \raggedleft\arraybslash \%\\
\textsc{v.tri} & \raggedleft 0 & \raggedleft {}-{}-{}- & \raggedleft 0 & \raggedleft\arraybslash {}-{}-{}-\\
\textsc{v.bi} & \raggedleft 327 & \raggedleft 68.4\% & \raggedleft 39 & \raggedleft\arraybslash 48.1\%\\
\textsc{v.mo}(\textsc{dy}) & \raggedleft 115 & \raggedleft 24.1\% & \raggedleft 18 & \raggedleft\arraybslash 22.2\%\\
\textsc{v.mo}(\textsc{st}) & \raggedleft 36 & \raggedleft 7.5\% & \raggedleft 24 & \raggedleft\arraybslash 29.6\%\\
Total & \raggedleft 478 & \raggedleft 100\% & \raggedleft 81 & \raggedleft\arraybslash 100\%\\
& \multicolumn{2}{l}{ \textitbf{biking}{}-causatives} & \multicolumn{2}{l}{ Different verbs}\\
Verb class & \raggedleft \# & \raggedleft \% & \raggedleft \# & \raggedleft\arraybslash \%\\
\textsc{v.tri} & \raggedleft 0 & \raggedleft {}-{}-{}- & \raggedleft 0 & \raggedleft\arraybslash {}-{}-{}-\\
\textsc{v.bi} & \raggedleft 0 & \raggedleft {}-{}-{}- & \raggedleft 0 & \raggedleft\arraybslash {}-{}-{}-\\
\textsc{v.mo}(\textsc{dy}) & \raggedleft 0 & \raggedleft {}-{}-{}- & \raggedleft 0 & \raggedleft\arraybslash {}-{}-{}-\\
\textsc{v.mo}(\textsc{st}) & \raggedleft 25 & \raggedleft 100\% & \raggedleft 16 & \raggedleft\arraybslash 100\%\\
Total & \raggedleft 25 & \raggedleft 100.0\% & \raggedleft 16 & \raggedleft\arraybslash 100.0\%\\
\lspbottomrule
\end{tabular}
\subsection{Reciprocal constructions}
\label{bkm:Ref350353878}
Verbs can occur in reciprocal constructions in which the reciprocity marker \textitbf{baku} ‘\textsc{recp}’ precedes the verb (for more details on reciprocal constructions, see §11.3). This is illustrated with trivalent \textitbf{ceritra} ‘tell’ in the elicited example in (0), bivalent \textitbf{gendong} ‘hold’ in (0), and monovalent dynamic \textitbf{saing} ‘compete’ in (0). Reciprocal constructions with monovalent stative verbs are unattested.
\end{styleBodyxafter}

\begin{tabular}{lllllll}
\lsptoprule
\label{bkm:Ref439093922}
\gll {Markus} {deng} {Yan} {dong} {\bluebold{baku}} {\bluebold{ceritra}}\\ %
& Markus & with & Yan & \textsc{3sg} & \textsc{recp} & tell\\
\lspbottomrule
\end{tabular}
\ea
\glt 
‘Markus and Yan were \bluebold{talking to each other}’ \textstyleExampleSource{[Elicited BR130601.001]}\footnote{\\
\\
\\
\\
\\
\\
\\
\\
\\
\\
\\
\\
The corpus contains one reciprocal construction formed with trivalent \textitbf{ceritra} ‘tell’, similar to the elicited example in (0). Most of the utterance is unclear, however, as the speaker mumbles.\\
\\
\\
}
\z

\begin{tabular}{lllllll}
\lsptoprule
\label{bkm:Ref350270690}
\gll {kitong} {\bluebold{baku}} {\bluebold{gendong}} {to?} {baku} {gendong}\\ %
& \textsc{1pl} & \textsc{recp} & hold & right? & \textsc{recp} & hold\\
\lspbottomrule
\end{tabular}
\ea
\glt 
‘we’ll \bluebold{hold each other}, right?, (we’ll) \bluebold{hold each other}’ \textstyleExampleSource{[080922-001a-CvPh.0695]}
\z

\begin{tabular}{llll}
\lsptoprule
\label{bkm:Ref349576736}
\gll {ade-kaka} {\bluebold{baku}} {\bluebold{saing}}\\ %
& ySb-oSb &  & \\
& siblings & \textsc{recp} & compete\\
\lspbottomrule
\end{tabular}
\ea
\glt 
‘the siblings were \bluebold{competing with each other}’ \textstyleExampleSource{[080919-006-CvNP.0001]}
\z


The data in the corpus indicates the following frequency patterns for reciprocal constructions, as shown in Table  ‎5 .20. The corpus contains 101 reciprocal constructions formed with 42 different verbs. Most of these verbs are bivalent (88\%), accounting for 94\% of the reciprocal constructions.


\begin{stylecaption}
\label{bkm:Ref349736777}Table ‎5.\stepcounter{Table}{\theTable}:  Reciprocal constructions
\end{stylecaption}

\tablehead{ & \multicolumn{2}{l}{ Token frequencies} & \multicolumn{2}{l}{ Type frequencies}\\
}
\begin{tabular}{lllll} & \multicolumn{2}{l}{ \textsc{recp}{}-constructions} & \multicolumn{2}{l}{ Different verbs}\\
\lsptoprule
Verb class & \raggedleft \# & \raggedleft \% & \raggedleft \# & \raggedleft\arraybslash \%\\
\textsc{v.tri} & \raggedleft 1 & \raggedleft 1.0\% & \raggedleft 1 & \raggedleft\arraybslash 2.4\%\\
\textsc{v.bi} & \raggedleft 95 & \raggedleft 94.1\% & \raggedleft 37 & \raggedleft\arraybslash 88.1\%\\
\textsc{v.mo}(\textsc{dy}) & \raggedleft 5 & \raggedleft 5.0\% & \raggedleft 4 & \raggedleft\arraybslash 9.5\%\\
\textsc{v.mo}(\textsc{st}) & \raggedleft 0 & \raggedleft {}-{}-{}- & \raggedleft 0 & \raggedleft\arraybslash {}-{}-{}-\\
Total & \raggedleft 101 & \raggedleft 100\% & \raggedleft 42 & \raggedleft\arraybslash 100\%\\
\lspbottomrule
\end{tabular}
\section{Adverbs}
\label{bkm:Ref350361213}\label{bkm:Ref358367282}
Papuan Malay has a large open class of adverbs, which modify constituents other than nouns. Their main function is to indicate aspect, frequency, affirmation and negation, modality, time, focus, and degree. Within the clause, the adverbs most commonly occur in pre-predicate position. Unlike the other two open lexical classes of nouns and verbs, Papuan Malay adverbs are not used predicatively.



Cross-linguistically, {Haser and \citet[66]{Kortmann2006}} note that in terms of their semantics and morphology, “adverbs are most closely related to adjectives, from which they are often derived”. With the restriction that Papuan Malay has a class of monovalent stative verbs instead of adjectives (see §5.3.1), this observation also seems to apply to the Papuan Malay adverbs. First, a number of adverbs are related to monovalent stative verbs, such as the temporal adverb \textitbf{dulu} ‘first, in the past’ which is related to stative \textitbf{dulu} ‘be prior’ (see §5.4.5), or the focus adverb \textitbf{pas} ‘precisely’ which is related to stative \textitbf{pas} ‘be exact’ (see §5.4.6; see also §5.14). Second, manner is expressed through stative verbs (see §5.4.8). Third, reduplicated verbs can receive an adverbial reading due to an interpretational shift. Examples are \textitbf{baru{\Tilde}baru} ‘just now’ with its stative base \textitbf{baru} ‘be new’ (see §5.4.5; see also §4.2.2.8). In Papuan Malay this link with verbs extends to dynamic verbs, in that reduplicated dynamic verbs can also receive an adverbial reading. Examples are the modal adverbs \textitbf{kira{\Tilde}kira} ‘probably’ and \textitbf{taw{\Tilde}taw} ‘unexpectedly’ which are related to the respective dynamic verbs \textitbf{kira} ‘think’ and \textitbf{taw} ‘know’ (see §5.4.4; see also §5.14).
\end{styleBodyvafter}


In addition to this prominent link with verbs, Papuan Malay adverbs are also related to nouns, although this link appears to be less prominent. First, a number of modal adverbs are historically derived from nouns by unproductive affixation with \-\textitbf{nya} ‘\textsc{3possr}’. Examples are \textitbf{artinya} ‘that means’ (literally ‘the meaning of’), \textitbf{katanya} ‘it is being said’ (literally ‘the word of’), or \textitbf{maksutnya} ‘that is to say’ (literally ‘the purpose of’). Second, reduplicated nouns can receive an adverbial reading due to an interpretational shift (see §4.2.1.4).
\end{styleBodyvafter}


The adverbs occur in different positions within the clause. They can take a pre-predicate or post-predicate position, with the pre-predicate position being the most common. There are also a fair number of adverbs which can occur in both positions. For the pre-predicate adverbs two positions are possible, directly preceding the predicate and preceding the subject. Likewise, two positions are possible for the post-predicate adverbs, directly following the predicate and, in clauses with peripheral adjuncts, following the adjunct. Depending on their positions within the clause, the adverbs differ in terms of their semantic effect. Generally speaking, pre-predicate adverbs which precede the subject have scope over the entire proposition. The semantic effect of pre-predicate adverbs which directly precede the predicate, and of post-predicate adverbs is more limited. On the whole, however, these distinctions are subtle, as shown with the temporal adverb \textitbf{langsung} ‘immediately’ in §5.4.5.
\end{styleBodyvafter}


The following sections describe the adverbs in terms of their positions within the clause and their overall semantic functions. Aspect adverbs are discussed in §5.4.1, frequency adverbs in §5.4.2, affirmation and negation adverbs in §5.4.3, modal adverbs in §5.4.4, temporal adverbs in §5.4.5, focus adverbs in §5.4.6, and degree adverbs in §5.4.7. Papuan Malay does not have manner adverbs; instead, manner is expressed through stative verbs which always follow the main verb, as briefly discussed in §5.4.8. Each of these sections includes a table which lists the different adverbs and indicates whether they take a pre-predicate (\textsc{pre-pred}) and/or post-predicate (\textsc{post-pred}) position within the clause (empty cells signal unattested constituent combinations). The different positions are also illustrated with (near) contrastive examples. An investigation of the semantic effects encoded by these positions, however, is left for future research. Also left for future research is the question of which adverbs can co-occur and in which positions.
\end{styleBodyvafter}


Following the description of the different types of adverbs, §5.4.9 summarizes the main points of this section, especially with respect to the interplay between syntactic properties and functions of the adverbs.
\end{styleBodyvxvafter}

\subsection{Aspectual adverbs}
\label{bkm:Ref358725502}\label{bkm:Ref358227558}
Aspectual adverbs provide temporal information about the event or state denoted by the verb in terms of their “duration or completion” {\citep[5094]{Asher1994}}. Thereby they differ from the temporal adverbs which designate temporal points {(Givón 2001: 91–92)}. The Papuan Malay aspectual adverbs are presented in Table  ‎5 .21.



Aspectual \textitbf{blum} ‘not yet’ and \textitbf{masi} ‘still’ have prospective meanings; that is, they point “forward to possible transitions in the future”, using {Smessaert and ter Meulen’s (2004: 221) terminology}. More specifically, \textitbf{blum} ‘not yet’ indicates that the event or state denoted by the verb is not yet completed or has not yet occurred, while \textitbf{masi} ‘still’ signals that the event or state is still continuing. Aspectual \textitbf{suda} ‘already’, by contrast, has a retrospective meaning; that is, it marks “a realized transition in the past”, again employing {Smessaert and ter Meulen’s (2004: 221) terminology} (\textitbf{suda} ‘already’ is very often shortened to \textitbf{su}). Besides, \textitbf{suda} ‘already’ can signal imperative mood, in which case it occurs in clause-final position, as discussed in §13.3. Progressive aspect is not encoded by an adverb but with the existential verb \textitbf{ada} ‘exist’; for expository reasons, however, the progressive marking function of \textitbf{ada} ‘exist’ is discussed here (existential clauses are discussed in §11.4).
\end{styleBodyvafter}


The three adverbs always occur in pre-predicate position, as shown in Table  ‎5 .21. This applies to their uses in verbal clauses, as in (0) and (0), and in nonverbal clauses, as in (0) to (0). Likewise, adverbially used \textitbf{ada} ‘exist’ precedes the predicate, as shown in (0), (0), and (0).
\end{styleBodyvvafter}

\begin{stylecaption}
\label{bkm:Ref362710969}Table ‎5.\stepcounter{Table}{\theTable}:  Aspectual adverbs and adverbially used \textitbf{ada} ‘exist’ and their positions within the clause
\end{stylecaption}

\tablehead{
 Item & Gloss & \multicolumn{2}{l}{ Position within the clause}\\
&  & \textsc{pre-pred} & \arraybslash \textsc{post-pred}\\
}
\begin{tabular}{llll}
\lsptoprule
\textitbf{blum} & ‘not yet’ & X & \\
\textitbf{masi} & ‘still’ & X & \\
\textitbf{suda} & ‘already’ & X & \\
\textitbf{ada} & ‘exist’ & X & \\
\lspbottomrule
\end{tabular}

In verbal predicate clauses, the aspectual adverbs and adverbially used \textitbf{ada} ‘exist’ modify dynamic verbs, as in (0) and (0), or stative verbs as in (0) and (0).


\begin{styleExampleTitle}
Aspectual adverbs and adverbially used \textitbf{ada} ‘exist’ modifying verbal predicates
\end{styleExampleTitle}

\begin{tabular}{lllllllll}
\lsptoprule
\label{bkm:Ref358025668}
\gll {a} {mama} {\bluebold{blum}} {mandi,} {mama} {\bluebold{masi}} {bangung} {tidor}\\ %
& ah! & mother & not.yet & bathe & mother & still & wake.up & sleep\\
\lspbottomrule
\end{tabular}
\ea
\glt 
‘ah, I (‘mother’) have \bluebold{not yet} bathed, I (‘mother’) am \bluebold{still} waking up’ \textstyleExampleSource{[080924-002-Pr.0007]}
\z

\begin{tabular}{lllllllllll}
\lsptoprule
\label{bkm:Ref358025672}
\gll {ana} {itu} {de} {\bluebold{suda}} {besar} {betul,} {de} {\bluebold{suda}} {besar} {…}\\ %
& child & \textsc{d.dist} & \textsc{3sg} & already & be.big & be.true & \textsc{3sg} & already & be.big & \\
\lspbottomrule
\end{tabular}
\ea
\glt 
‘(when) that child is \bluebold{already} really grown-up, (when) he/she’s \bluebold{already} grown-up, …’ \textstyleExampleSource{[081006-025-CvEx.0005]}
\z

\begin{tabular}{llllllllll}
\lsptoprule
\label{bkm:Ref364507777}
\gll {sa} {pu} {maytua} {\bluebold{ada}} {\bluebold{tidor}} {karna} {hari} {blum} {siang}\\ %
& \textsc{1sg} & \textsc{poss} & wife & exist & sleep & because & day & not.yet & day\\
\lspbottomrule
\end{tabular}
\ea
\glt 
‘my wife \bluebold{was sleeping} because it wasn’t daylight yet’ \textstyleExampleSource{[080919-004-NP.0026]}
\z

\begin{tabular}{lllllll}
\lsptoprule
\label{bkm:Ref364507778}
\gll {dong} {bilang,} {a} {de} {\bluebold{ada}} {\bluebold{sakit}}\\ %
& \textsc{3pl} & say & ah! & \textsc{3sg} & exist & be.sick\\
\lspbottomrule
\end{tabular}
\ea
\glt 
‘they said, ‘ah, he’s \bluebold{being sick}’’ \textstyleExampleSource{[080919-007-CvNP.0025]}
\z


The examples in (0) to (0) demonstrate the uses of the aspectual adverbs and adverbially used \textitbf{ada} ‘exist’ in nonverbal predicate clauses. (An alternative analysis of clauses with \textitbf{ada} ‘exist’, such as the one in (0), is presented in §11.4.1.)


\begin{styleExampleTitle}
Aspectual adverbs modifying nonverbal predicates
\end{styleExampleTitle}

\begin{tabular}{llllll}
\lsptoprule
\label{bkm:Ref358024861}
\gll {itu} {kang} {\bluebold{blum}} {musim} {ombak}\\ %
& \textsc{d.dist} & you.know & not.yet & season & wave\\
\lspbottomrule
\end{tabular}
\ea
\glt 
[About traveling by high or low tide:] ‘that is \bluebold{not yet} the wavy season, you know’ \textstyleExampleSource{[080927-003-Cv.0020]}
\z

\begin{tabular}{lllllll}
\lsptoprule
(\stepcounter{}{\the}) & Roni & \bluebold{masi} & deng & de & pu & temang{\Tilde}temang\\
& Roni & still & with & \textsc{3sg} & \textsc{poss} & \textsc{rdp}{\Tilde}friend\\
\lspbottomrule
\end{tabular}
\ea
\glt 
‘Roni is \bluebold{still} with his friends’ \textstyleExampleSource{[081006-031-Cv.0011]}
\z

\begin{tabular}{lllllll}
\lsptoprule
\label{bkm:Ref362704189}
\gll {sa} {\bluebold{su}} {di} {Arare} {sama} {Pawla}\\ %
& \textsc{1sg} & already & at & Arare & to & Pawla\\
\lspbottomrule
\end{tabular}
\ea
\glt 
‘I (would) \bluebold{already} be in Arare with Pawla’ \textstyleExampleSource{[081025-009a-Cv.0110]}
\z

\begin{tabular}{lllllll}
\lsptoprule
\label{bkm:Ref371443108}
\gll {ana{\Tilde}ana} {prempuang} {dong} {\bluebold{ada}} {di} {depang}\\ %
& \textsc{rdp}{\Tilde}child & woman & 3\textsc{pl} & exist & at & front\\
\lspbottomrule
\end{tabular}
\ea
\glt
‘the girls \bluebold{are being} in front’ \textstyleExampleSource{[080921-004a-CvNP.0066]}
\end{styleFreeTranslEngxvpt}

\subsection{Frequency adverbs}
\label{bkm:Ref358227560}
Frequency adverbs “typically indicate the number of times something happened” during a given time interval {\citep[688]{Doetjes2007}}. The Papuan Malay frequency adverbs are listed in Table  ‎5 .22; they always occur in pre-predicate position.\footnote{\\
\\
\\
\\
\\
\\
\\
\\
\\
\\
\\
\\
In the corpus only \textitbf{biasanya} ‘usually’ and \textitbf{perna} ‘ever’ are attested in the clause-initial position; for the remaining frequency adverbs, their uses in this position were established by means of elicitation.\\
\\
\\
}


\begin{stylecaption}
\label{bkm:Ref362709873}Table ‎5.\stepcounter{Table}{\theTable}:  Frequency adverbs and their positions within the clause
\end{stylecaption}

\tablehead{
 Item & Gloss & \multicolumn{2}{l}{ Position within the clause}\\
&  & \textsc{pre-pred} & \arraybslash \textsc{post-pred}\\
}
\begin{tabular}{llll}
\lsptoprule
\textitbf{biasanya}\footnotemark{} & ‘usually’ & X & \\
\textitbf{perna} & ‘ever, once’ & X & \\
\textitbf{jarang} & rarely’ & X & \\
\textitbf{kadang({\Tilde}kadang)} & ‘sometimes’ & X & \\
\textitbf{slalu} & ‘always’ & X & \\
\textitbf{sring} & ‘often’ & X & \\
\lspbottomrule
\end{tabular}
\footnotetext{\\
\\
\\
\\
\\
\\
\\
\\
\\
\\
\\
\\
The adverb \textitbf{biasanya} ‘usually’ is historically derived: \textitbf{biasa-nya} ‘be.usual-\textsc{3poss}’ (for details on suffixation with \textitbf{{}-nya} ‘\textsc{3poss}’, see §3.1.6).\\
\\
\\
}

The pre-predicate position of the frequency adverbs is illustrated in (0) to (0). The adverbs can directly precede the predicate, such as \textitbf{kadang{\Tilde}kadang} ‘sometimes’ in (0) or \textitbf{perna} ‘ever’ in (0), or they can precede the subject, such as \textitbf{kadang({\Tilde}kadang)} ‘sometimes’ in (0) or \textitbf{perna} ‘ever’ in (0). These examples also show that frequency adverbs not only modify verbal predicates as in (0) to (0), but also nonverbal predicates as in (0). The semantics conveyed by the different positions have to do with scope.


\begin{styleExampleTitle}
Frequency adverbs in clause-initial and pre-predicate positions
\end{styleExampleTitle}

\begin{tabular}{lllllll}
\lsptoprule
\label{bkm:Ref371776406}
\gll {yo,} {de} {\bluebold{kadang{\Tilde}kadang}} {terlalu,} {ini,} {egois}\\ %
& yes & \textsc{3sg} & sometimes & too & \textsc{d.prox} & be.egoistic\\
\lspbottomrule
\end{tabular}
\ea
\glt 
‘yes, she’s \bluebold{sometimes} too, what’s-its-name, egoistic’ \textstyleExampleSource{[081115-001a-Cv.0218/0220]}
\z

\begin{tabular}{lllllllllll}
\lsptoprule
\label{bkm:Ref358034951}
\gll {\bluebold{kadang}} {sa} {sa} {buang} {bola} {sama} {Wili} {deng} {Klara} {to?}\\ %
& sometimes & \textsc{1sg} & \textsc{1sg} & discard & ball & to & Wili & with & Klara & right?\\
\lspbottomrule
\end{tabular}
\ea
\glt 
‘\bluebold{sometimes} I, I threw the ball to Wili and Klara, right?’ \textstyleExampleSource{[081006-014-Cv.0005]}
\z

\begin{tabular}{llllllll}
\lsptoprule
\label{bkm:Ref358367798}\label{bkm:Ref358034954}
\gll {de} {\bluebold{perna}} {kasi} {makang} {sa} {pu} {ana}\\ %
& \textsc{3sg} & ever & give & eat & \textsc{1sg} & \textsc{poss} & child\\
\lspbottomrule
\end{tabular}
\ea
\glt 
‘she \bluebold{once} fed my child’ \textstyleExampleSource{[081110-008-CvNP.0050]}
\z

\begin{tabular}{lllllllll}
\lsptoprule
\label{bkm:Ref371776409}
\gll {…} {\bluebold{perna}} {kitong} {dua} {di} {apa} {kantor} {Golkar}\\ %
&  & ever & \textsc{1pl} & two & at & what & office & Golkar\\
\lspbottomrule
\end{tabular}
\ea
\glt
‘[so I and, what’s-his-name, Noferus here,] \bluebold{once} the two of us were at, what-is-it, the Golkar office’ \textstyleExampleSource{[080923-009-Cv.0050]}
\end{styleFreeTranslEngxvpt}

\subsection{Affirmation and negation adverbs}
\label{bkm:Ref358227563}
The affirmation and negation adverbs listed in Table  ‎5 .23 indicate general affirmation, negation, or prohibition, and provide responses to polar questions (see also Chapter 13).


\begin{stylecaption}
\label{bkm:Ref367351274}Table ‎5.\stepcounter{Table}{\theTable}:  Papuan Malay affirmation and negation adverbs
\end{stylecaption}

\tablehead{
 Item & Gloss & \multicolumn{2}{l}{ Position within the clause}\\
&  & \textsc{pre-pred} & \arraybslash \textsc{post-pred}\\
}
\begin{tabular}{llll}
\lsptoprule
\textitbf{yo} & ‘yes’ & X & \\
\textitbf{bukang} & ‘\textsc{neg}, no’ & X & \\
\textitbf{tida}/\textitbf{tra} & ‘\textsc{neg}, no’ & X & \\
\textitbf{jangang} & ‘\textsc{neg.imp}, don’t’ & X & \\
\lspbottomrule
\end{tabular}

The four adverbs always take a pre-predicate position. Affirmative \textitbf{yo} ‘yes’ is always fronted, while the negative and prohibitive adverbs directly precede the predicate. Affirmative \textitbf{yo} ‘yes’ is often realized as \textitbf{ya}, and negative \textitbf{jangang} ‘\textsc{neg.imp}’ is quite commonly shortened to \textitbf{jang}. Examples are provided in (0) to (0): affirmation with \textitbf{yo} ‘yes’ in (0), negation with interchangeably used \textitbf{tra} ‘\textsc{neg}’ and \textitbf{tida} ‘\textsc{neg}’ in (0), and with \textitbf{bukang} ‘\textsc{neg}’ (0), and prohibition with \textitbf{jangang} ‘\textsc{neg.imp}’ in (0). Negation and prohibition are discussed in more detail in §13.1 and §13.3.3, respectively.


\begin{styleExampleTitle}
Affirmation and negation adverbs: Examples
\end{styleExampleTitle}

\begin{tabular}{llllllll}
\lsptoprule
\label{bkm:Ref358297262}
\gll {\bluebold{yo},} {tikus} {de} {loncat} {ke} {klapa} {lagi}\\ %
& yes & rat & \textsc{3sg} & jump & to & coconut & again\\
\lspbottomrule
\end{tabular}
\ea
\glt 
‘\bluebold{yes}, the rat also jumped over to the coconut tree’ \textstyleExampleSource{[080917-003b-CvEx.0025]}
\z

\begin{tabular}{llllllll}
\lsptoprule
\label{bkm:Ref358297263}
\gll {de} {\bluebold{tra}} {datang} {…} {de} {\bluebold{tida}} {datang}\\ %
& \textsc{3sg} & \textsc{neg} & come &  & \textsc{3sg} & \textsc{neg} & come\\
\lspbottomrule
\end{tabular}
\ea
\glt 
‘she did \bluebold{not} come … she did \bluebold{not} come’ \textstyleExampleSource{[081010-001-Cv.0204-0205]}
\z

\begin{tabular}{lllll}
\lsptoprule
\label{bkm:Ref358297264}
\gll {saya} {\bluebold{bukang}} {anjing} {hitam}\\ %
& \textsc{1sg} & \textsc{neg} & dog & be.black\\
\lspbottomrule
\end{tabular}
\ea
\glt 
‘(the situation is) \bluebold{not} (that) ‘I am a black dog’ \textstyleExampleSource{[081115-001a-Cv.0266]}
\z

\begin{tabular}{llllllll}
\lsptoprule
\label{bkm:Ref358297265}
\gll {Nofi} {\bluebold{jangang}} {ganggu} {kaka,} {ade} {tu,} {e?}\\ %
& Nofi & \textsc{neg.imp} & disturb & oSb & ySb & \textsc{d.dist} & e?\\
\lspbottomrule
\end{tabular}
\ea
\glt
‘Nofi \bluebold{don’t} bother that older relative, younger relative, eh?’ \textstyleExampleSource{[081011-009-Cv.0013]}
\end{styleFreeTranslEngxvpt}

\subsection{Modal adverbs}
\label{bkm:Ref358227562}
Modal adverbs “express the subjective evaluation of the speaker toward a state of affairs” {\citep[751]{Bussmann1996}}. This includes “epistemic” adverbs which “denote the speaker’s attitude toward the truth, certainty or probability of the state or event” and “evaluative” adverbs which express “the speaker’s \textstyleChItalic{evaluative} attitudes, i.e. judgments of \textstyleChItalic{preference} for or \textstyleChItalic{desirability} of a state or event” {(Givón 2001: 92–93)}.



The Papuan Malay modal adverbs are presented in Table  ‎5 .24. Most of them are historically derived by (unproductive) affixation (for details on affixation see §3.1). All Papuan Malay modal adverbs take a pre-predicate position. Besides the adverbs listed in Table  ‎5 .24, degree adverb \textitbf{paling} ‘most’ also has an epistemic function when it precedes the subject, as discussed in §5.4.7.
\end{styleBodyvvafter}

\begin{stylecaption}
\label{bkm:Ref358210734}Table ‎5.\stepcounter{Table}{\theTable}:  Papuan Malay modal adverbs and their positions within the clause
\end{stylecaption}

\tablehead{
\multicolumn{2}{l}{ Item} & \multicolumn{2}{l}{ Literal} & \multicolumn{2}{l}{ Gloss} & \multicolumn{2}{l}{ Position}\\
\multicolumn{2}{l}{} & \multicolumn{2}{l}{} & \multicolumn{2}{l}{} & \textsc{pre-pred} & \arraybslash \textsc{post-pred}\\
}
\begin{tabular}{llllllll}
\lsptoprule
\multicolumn{8}{l}{Epistemic adverbs}\\
& \textitbf{kata-nya} & \multicolumn{2}{l}{‘word-\textsc{3possr}’} & \multicolumn{2}{l}{‘it is being said’} & X & \\
& \textitbf{kira{\Tilde}kira} & \multicolumn{2}{l}{‘\textsc{rdp{\Tilde}}think’} & \multicolumn{2}{l}{‘probably’} & X & \\
& \textitbf{memang} & \multicolumn{2}{l}{} & \multicolumn{2}{l}{‘indeed’} & X & \\
& \textitbf{misal-nya} & \multicolumn{2}{l}{‘example-\textsc{3possr}’} & \multicolumn{2}{l}{‘for example’} & X & \\
& \textitbf{mungking} & \multicolumn{2}{l}{} & \multicolumn{2}{l}{‘maybe’} & X & \\
& \textitbf{pasti} & \multicolumn{2}{l}{} & \multicolumn{2}{l}{‘definitely’} & X & \\
& \textitbf{pokok-nya} & \multicolumn{2}{l}{‘main-\textsc{3possr}’} & \multicolumn{2}{l}{‘the main thing is’} & X & \\
& \textitbf{sebenar-nya} & \multicolumn{2}{l}{‘one:be.true-\textsc{3possr}’} & \multicolumn{2}{l}{‘actually’} & X & \\
& \textitbf{sperti-nya} & \multicolumn{2}{l}{‘similar.to-\textsc{3possr}’} & \multicolumn{2}{l}{‘it seems’} & X & \\
& \textitbf{arti-nya} & \multicolumn{2}{l}{‘meaning-\textsc{3possr}’} & \multicolumn{2}{l}{‘that means’} & X & \\
& \textitbf{maksut-nya} & \multicolumn{2}{l}{‘purpose-\textsc{3possr}’} & \multicolumn{2}{l}{‘that is to say’} & X & \\
\multicolumn{8}{l}{Evaluative adverbs}\\
& \multicolumn{2}{l}{\textitbf{akir-nya}} & \multicolumn{2}{l}{‘end-\textsc{3possr}’} & ‘finally’ & X & \\
& \multicolumn{2}{l}{\textitbf{coba}} & \multicolumn{2}{l}{‘try’} & ‘if only’ & X & \\
& \multicolumn{2}{l}{\textitbf{harus-nya}} & \multicolumn{2}{l}{‘have.to-\textsc{3possr}’} & ‘appropriately’ & X & \\
& \multicolumn{2}{l}{\textitbf{muda{\Tilde}muda-an}} & \multicolumn{2}{l}{‘\textsc{rdp}{\Tilde}be.easy-\textsc{pat}’} & ‘hopefully’ & X & \\
& \multicolumn{2}{l}{\textitbf{taw{\Tilde}taw}} & \multicolumn{2}{l}{‘\textsc{rdp}{\Tilde}know’} & ‘suddenly’ & X & \\
\lspbottomrule
\end{tabular}

The pre-predicate position of the modal adverbs is demonstrated in (0) to (0). Typically, they precede the subject. This is illustrated with epistemic \textitbf{memang} ‘indeed’ in (0) and \textitbf{pasti} ‘definitely’ in (0), and with evaluative \textitbf{akirnya} ‘finally’ in (0) and \textitbf{taw{\Tilde}taw} ‘suddenly’ in (0). Functioning at clause level, the epistemic adverbs introduce propositions which offer explanations and clarifications for the events depicted in the preceding discourse, while the evaluative adverbs provide an evaluation of the events described in the preceding discourse.


\begin{styleExampleTitle}
Modal adverbs in pre-predicate position preceding the subject
\end{styleExampleTitle}

\begin{tabular}{llllll}
\lsptoprule
\label{bkm:Ref367440558}
\gll {kas} {tinggal,} {\bluebold{memang}} {de} {nakal}\\ %
& give & stay & indeed & \textsc{3sg} & be.mischievous\\
\lspbottomrule
\end{tabular}
\ea
\glt 
‘let it be, \bluebold{indeed}, he is mischievous’ \textstyleExampleSource{[081015-001-Cv.0027]}
\z

\begin{tabular}{llll}
\lsptoprule
\label{bkm:Ref362880983}
\gll {\bluebold{pasti}} {de} {pulang}\\ %
& definitely & \textsc{3sg} & go.home\\
\lspbottomrule
\end{tabular}
\ea
\glt 
‘\bluebold{certainly}, she’ll come home’ \textstyleExampleSource{[081006-019-Cv.0010]}
\z

\begin{tabular}{lllllll}
\lsptoprule
\label{bkm:Ref358211478}
\gll {\bluebold{akirnya}} {asap{\Tilde}asap} {naik,} {langsung} {api} {menyala}\\ %
& finally & \textsc{rdp}{\Tilde}smoke & ascend & immediately & fire & flame\\
\lspbottomrule
\end{tabular}
\ea
\glt 
‘\bluebold{finally} smoke ascended, immediately the fire flared up’ \textstyleExampleSource{[080922-010a-CvNF.0079]}
\z

\begin{tabular}{llllll}
\lsptoprule
\label{bkm:Ref362880984}
\gll {\bluebold{taw{\Tilde}taw}} {orang} {itu} {tida} {keliatang}\\ %
& suddenly & person & \textsc{d.dist} & \textsc{neg} & be.visible\\
\lspbottomrule
\end{tabular}
\ea
\glt 
‘\bluebold{suddenly}, that person wasn’t visible (any longer)’ \textstyleExampleSource{[080922-002-Cv.0123]}
\z


While evaluative modal adverbs always precede the subject, most epistemic adverbs can take two pre-predicate positions. Besides preceding the subject, as in (0) and (0), they can also directly precede the predicate. The exceptions are \textitbf{artinya} ‘that means’ and \textitbf{maksutnya} ‘that is to say’, both of which always precede the subject. This position directly preceding the predicate is illustrated with \textitbf{memang} ‘indeed’ in (0) and with \textitbf{pasti} ‘definitely’ in (0) (compare both examples with the examples in (0) and (0), respectively). Both examples also show that modal adverbs not only occur in verbal clauses as in (0), but also in non-verbal clauses, as in (0).


\begin{styleExampleTitle}
Modal adverbs in pre-predicate position preceding the subject or the predicate
\end{styleExampleTitle}

\begin{tabular}{llllllll}
\lsptoprule
\label{bkm:Ref358211475}
\gll {jangang} {ko} {singgung,} {tapi} {ini} {\bluebold{memang}} {bukti}\\ %
& \textsc{neg.imp} & \textsc{2sg} & offend & but & this & indeed & proof\\
\lspbottomrule
\end{tabular}
\ea
\glt 
[About problems with the local elections:] ‘don’t feel offended but this is \bluebold{indeed} the proof’ \textstyleExampleSource{[081011-024-Cv.0150]}
\z

\begin{tabular}{lllllll}
\lsptoprule
\label{bkm:Ref362894494}
\gll {…} {tapi} {de} {\bluebold{pasti}} {kasi} {swara}\\ %
&  & but & \textsc{3sg} & definitely & give & voice\\
\lspbottomrule
\end{tabular}
\ea
\glt
[About meeting strangers in remote areas:] ‘[most likely, he/she won’t know your name yet,] but he/she’ll \bluebold{definitely} call (you)’ \textstyleExampleSource{[080919-004-NP.0078]}
\end{styleFreeTranslEngxvpt}

\subsection{Temporal adverbs}
\label{bkm:Ref358129559}
Temporal adverbs designate temporal points {(Givón 2001: 91–92). }Thereby they differ from aspectual adverbs which provide temporal information about the event or state denoted by the verb in terms of their completion or duration {\citep[5094]{Asher1994}. }The Papuan Malay temporal adverbs are listed in Table  ‎5 .25. Within the clause, almost all of them occur in pre-predicate or in post-predicate position. The exceptions are \textitbf{baru} ‘recently’ and \textitbf{baru}{\Tilde}\textitbf{baru} ‘just now’ which only occur in pre-predicate position.\footnote{\\
\\
\\
\\
\\
\\
\\
\\
\\
\\
\\
\\
Three of the adverbs listed in Table  ‎5 .25 have dual word class membership with monovalent stative verbs: \textitbf{baru} ‘recently’, \textitbf{dulu} ‘be prior’, and \textitbf{skarang} ‘now’ (variation in word class membership is discussed in §5.14).\\
\\
\\
}


\begin{stylecaption}
\label{bkm:Ref358123180}Table ‎5.\stepcounter{Table}{\theTable}:  Temporal adverbs and their positions within the clause
\end{stylecaption}

\tablehead{
 Item & Gloss & \multicolumn{2}{l}{ Position within the clause}\\
&  & \textsc{pre-pred} & \arraybslash \textsc{post-pred}\\
}
\begin{tabular}{llll}
\lsptoprule
\textitbf{dulu} & ‘first, in the past’ & X & \arraybslash X\\
\textitbf{lama{\Tilde}lama} & ‘gradually’ & X & \arraybslash X\\
\textitbf{langsung} & ‘immediately’ & X & \arraybslash X\\
\textitbf{nanti} & ‘very soon’ & X & \arraybslash X\\
\textitbf{sebentar} & ‘in/for a moment’ & X & \arraybslash X\\
\textitbf{skarang} & ‘now’ & X & \arraybslash X\\
\textitbf{tadi} & ‘earlier’ & X & \arraybslash X\\
\textitbf{baru} & ‘recently’ & X & \\
\textitbf{baru{\Tilde}baru} & ‘just now’ & X & \\
\lspbottomrule
\end{tabular}

Examples for the pre-predicate position are given in (0) to (0), and for the post-predicate position in (0) to (0). The different meaning aspects conveyed by both positions are discussed in connection with the examples in (0) to (0).
\end{styleBodyaftervbefore}


In pre-predicate position, the adverbs can directly precede the predicate, such as \textitbf{langsung} ‘immediately’ in (0) and \textitbf{nanti} ‘very soon’ in (0), or precede the subject, such as \textitbf{langsung} ‘immediately’ in (0) and \textitbf{nanti} ‘very soon’ in (0).
\end{styleBodyvvafter}

\begin{styleExampleTitle}
Temporal adverbs in pre-predicate position
\end{styleExampleTitle}

\begin{tabular}{llllllll}
\lsptoprule
\label{bkm:Ref363050572}
\gll {de} {\bluebold{langsung}} {ke} {asrama} {polisi} {cari} {bapa}\\ %
& \textsc{3sg} & immediately & to & dormitory & police & search & father\\
\lspbottomrule
\end{tabular}
\ea
\glt 
‘he (went) \bluebold{immediately} to the police dormitory to look for father’ \textstyleExampleSource{[081011-022-Cv.0242]}
\z

\begin{tabular}{lllllll}
\lsptoprule
\label{bkm:Ref358194490}
\gll {wa,} {ko} {datang,} {\bluebold{langsung}} {ko} {lapar?}\\ %
& wow! & \textsc{2sg} & come & immediately & \textsc{2sg} & be.hungry\\
\lspbottomrule
\end{tabular}
\ea
\glt 
‘wow, you come (here, and) \bluebold{immediately} you’re hungry?’ \textstyleExampleSource{[081110-002-Cv.0049]}
\z

\begin{tabular}{lllllll}
\lsptoprule
\label{bkm:Ref363050574}
\gll {…} {dang} {ko} {\bluebold{nanti}} {kena} {picaang}\\ %
&  & and & \textsc{2sg} & very.soon & hit & splinter\\
\lspbottomrule
\end{tabular}
\ea
\glt 
‘[don’t (go down to the beach, (it’s) dirty,] and \bluebold{later} you’ll run into broken glass and cans’ \textstyleExampleSource{[080917-004-CvHt.0002]}
\z

\begin{tabular}{lllllllllll}
\lsptoprule
\label{bkm:Ref358194357}
\gll {\multicolumn{2}{l}{\bluebold{nanti}}} {\multicolumn{2}{l}{bapa}} {mo} {brangkat,} {\bluebold{nanti}} {bapa} {kas} {taw}\\ %
& \multicolumn{2}{l}{very.soon} & \multicolumn{2}{l}{father} & want & leave & very.soon & father & give & know\\
& sama & \multicolumn{2}{l}{bapa-ade} & \multicolumn{7}{l}{pendeta}\\
& with & \multicolumn{2}{l}{uncle} & \multicolumn{7}{l}{pastor}\\
\lspbottomrule
\end{tabular}
\ea
\glt 
‘\bluebold{very soon} I (‘father’) will leave (and) \bluebold{then} I (‘father’) will inform uncle pastor’ \textstyleExampleSource{[080922-001a-CvPh.0339]}
\z


The post-predicate position is illustrated in (0) to (0). In clauses with peripheral adjuncts, the adverb follows the predicate and precedes the adjunct, such as \textitbf{nanti} ‘very soon’ in the elicited example in (0) and \textitbf{langsung} ‘immediately’ in (0). Clauses, in which the temporal adverb follows the peripheral adjunct are either ungrammatical, such as \textitbf{nanti} ‘very soon’ in the elicited examples in (0), or only marginally grammatical such as \textitbf{langsung} ‘immediately’ in the elicited contrastive examples in (0).


\begin{styleExampleTitle}
Temporal adverbs in post-predicate position
\end{styleExampleTitle}

\begin{tabular}{llllll}
\lsptoprule
\label{bkm:Ref363050573}
\gll {tong} {pergi} {\bluebold{nanti}} {ke} {Sarmi}\\ %
& \textsc{1pl} & go & very.soon & to & Sarmi\\
\lspbottomrule
\end{tabular}
\ea
\glt 
‘we’ll go \bluebold{very soon}’ to Sarmi’ \textstyleExampleSource{[Elicited MY131113.001]}
\z

\begin{tabular}{lllllll}
\lsptoprule
\label{bkm:Ref371766874}
\gll {*} {tong} {pergi} {ke} {Sarmi} {\bluebold{nanti}}\\ %
&  & \textsc{1pl} & come & to & Sarmi & very.soon\\
\lspbottomrule
\end{tabular}
\ea
\glt 
Intended reading: ‘we’ll go to Sarmi \bluebold{very soon}’ \textstyleExampleSource{[Elicited MY131113.002]}
\z

\begin{tabular}{lllllllll}
\lsptoprule
\label{bkm:Ref363206763}
\gll {…} {tak!,} {masuk} {\bluebold{langsung}} {di} {bawa} {meja} {sana}\\ %
&  & bang! & enter & immediately & at & bottom & table & \textsc{l.dist}\\
\lspbottomrule
\end{tabular}
\ea
\glt 
[About a small boy who had a collision with an evil spirit:] ‘whump! \bluebold{immediately} (the kid) went under the table over there’ \textstyleExampleSource{[081025-009b-Cv.0029]}
\z

\begin{tabular}{llllllllll}
\lsptoprule
\label{bkm:Ref363050571}
\gll {??} {…} {tak!,} {masuk} {di} {bawa} {meja} {sana} {\bluebold{langsung}}\\ %
&  &  & bang! & enter & at & bottom & table & \textsc{l.dist} & immediately\\
\lspbottomrule
\end{tabular}
\ea
\glt 
Intended reading: ‘whump! (the kid) went under the table over there \bluebold{immediately}’ \textstyleExampleSource{[Elicited MY131113.003]}
\z


The meaning aspects conveyed by the different positions of the temporal adverbs have to do with scope. This is demonstrated with \textitbf{langsung} ‘immediately’ in three (near) contrastive examples: the pre-predicate position following the subject is shown in (0), the pre-predicate position preceding the subject in (0), and the post-predicate position in (0).


\begin{styleExampleTitle}
Positions and scope of temporal adverbs
\end{styleExampleTitle}

\begin{tabular}{llll}
\lsptoprule
\label{bkm:Ref363208514}
\gll {bapa} {\bluebold{langsung}} {diam}\\ %
& father & immediately & be.quiet\\
\lspbottomrule
\end{tabular}
\ea
\glt 
‘the gentleman was \bluebold{immediately} quiet’ \textstyleExampleSource{[080917-010-CvEx.0186]}
\z

\begin{tabular}{llll}
\lsptoprule
\label{bkm:Ref363208513}
\gll {\bluebold{langsung}} {dong} {diam}\\ %
& immediately & \textsc{3pl} & be.quiet\\
\lspbottomrule
\end{tabular}
\ea
\glt 
‘\bluebold{immediately} they were quiet’ \textstyleExampleSource{[080922-003-Cv.0085]}
\z

\begin{tabular}{lllll}
\lsptoprule
\label{bkm:Ref363208515}
\gll {bapa} {de} {diam} {\bluebold{langsung}}\\ %
& father & \textsc{3sg} & be.quiet & immediately\\
\lspbottomrule
\end{tabular}
\ea
\glt 
‘the gentleman was quiet \bluebold{immediately}’ \textstyleExampleSource{[080917-010-CvEx.0191]}
\z


Only one temporal adverb has clear distinct meanings depending on its positions, namely \textitbf{dulu} ‘first, in the past’. Pre-predicate \textitbf{dulu} translates with ‘in the past’, whereas post-predicate \textitbf{dulu} translates with ‘first’, as shown in (0).


\begin{styleExampleTitle}
Temporal \textitbf{dulu} ‘in the past, first’ in clause-initial and post-predicate positions
\end{styleExampleTitle}

\begin{tabular}{lllllllll}
\lsptoprule
\label{bkm:Ref363050575}
\gll {\bluebold{dulu}} {kitong} {pu} {\multicolumn{2}{l}{orang-tua}} {itu} {tida} {bisa}\\ %
& first & \textsc{1pl} & \textsc{poss} & \multicolumn{2}{l}{parent} & \textsc{d.dist} & \textsc{neg} & be.able\\
& \multicolumn{4}{l}{berhubungang} & \multicolumn{4}{l}{\bluebold{dulu}}\\
& \multicolumn{4}{l}{have.sexual.intercourse} & \multicolumn{4}{l}{first}\\
\lspbottomrule
\end{tabular}
\ea
\glt 
‘\bluebold{in the past} our parents couldn’t have sex \bluebold{first} (before getting married)’ \textstyleExampleSource{[081110-006-CvEx.0012]}
\z


Temporal \textitbf{baru} ‘recently’ and \textitbf{baru{\Tilde}baru} ‘just now’ only occur in pre-predicate position, as in (0) and (0). While \textitbf{baru} ‘recently’ directly precedes the predicate, \textitbf{baru{\Tilde}baru} ‘just now’ precedes the subject.


\begin{styleExampleTitle}
Temporal \textitbf{baru} ‘recently’ and \textitbf{baru{\Tilde}baru} ‘just now’ in pre-predicate position only
\end{styleExampleTitle}

\begin{tabular}{lllll}
\lsptoprule
\label{bkm:Ref358194359}
\gll {kariawang} {dong} {\bluebold{baru}} {lewat}\\ %
& employee & \textsc{3pl} & recently & pass.by\\
\lspbottomrule
\end{tabular}
\ea
\glt 
‘the employees \bluebold{recently} walked by’ \textstyleExampleSource{[080922-001a-CvPh.0830]}
\z

\begin{tabular}{lllll}
\lsptoprule
\label{bkm:Ref371758441}
\gll {\bluebold{baru{\Tilde}baru}} {de} {masuk} {ruma-sakit}\\ %
& just.now & \textsc{3sg} & enter & hospital\\
\lspbottomrule
\end{tabular}
\ea
\glt
‘\bluebold{just now}, he got into hospital’ \textstyleExampleSource{[081115-001a-Cv.0070]}
\end{styleFreeTranslEngxvpt}

\subsection{Focus adverbs}
\label{bkm:Ref358227561}
Focus adverbs indicate “an accentual peak or stress which is used to contrast or compare [… an] item either explicitly or implicitly with a set of alternatives” {(Hoeksema and Zwarts 1991: 52)}. That is, focus adverbs highlight information and signal some kind of restriction, thereby adding emphasis to an utterance. Hence, they are also known as “emphatic” adverbs {(Givón 2001: 94)}. In Papuan Malay, almost all focus adverbs take a pre-predicate position, as shown in Table  ‎5 .26. The exceptions are \textitbf{juga} ‘also’, \textitbf{lagi} ‘again, also’, and \textitbf{saja} ‘just’ which take a post-predicate position. While the latter two only occur in post-predicate position, \textitbf{juga} ‘also’ also takes a pre-predicate position.


\begin{stylecaption}
\label{bkm:Ref375581299}Table ‎5.\stepcounter{Table}{\theTable}:  Focus adverbs and their positions within the clause\footnote{\\
\\
\\
\\
\\
\\
\\
\\
\\
\\
\\
\\
The focus adverb \textitbf{pas} ‘precisely’ has dual word class membership with the monovalent stative verb \textitbf{pas} ‘be exact’ (variation in word class membership is discussed in §5.14).\\
\\
\\
}
\end{stylecaption}

\tablehead{
 Item & Gloss & \multicolumn{2}{l}{ Position within the clause}\\
&  & \textsc{pre-pred} & \arraybslash \textsc{post-pred}\\
}
\begin{tabular}{llll}
\lsptoprule
\textitbf{apalagi} & ‘moreover’ & X & \\
\textitbf{kecuali} & ‘except’ & X & \\
\textitbf{kususnya}\footnotemark{} & ‘especially’ & X & \\
\textitbf{cuma} & ‘just’ & X & \\
\textitbf{hanya} & ‘only’ & X & \\
\textitbf{justru} & ‘precisely’ & X & \\
\textitbf{mala} & ‘instead’ & X & \\
\textitbf{pas} & ‘precisely’ & X & \\
\textitbf{juga} & ‘also’ & X & \arraybslash X\\
\textitbf{lagi} & ‘again, also’ &  & \arraybslash X\\
\textitbf{saja} & ‘just’ &  & \arraybslash X\\
\lspbottomrule
\end{tabular}
\footnotetext{\\
\\
\\
\\
\\
\\
\\
\\
\\
\\
\\
\\
The adverb \textitbf{kususnya} ‘especially’ is historically derived: \textitbf{kusus-nya} ‘be.special-\textsc{3poss}’ (for details on suffixation with \textitbf{{}-nya} ‘\textsc{3poss}’, see §3.1.6).\\
\\
\\
}

The pre-predicate position of the focus adverbs is illustrated in (0) to (0). Focus adverbs typically precede the subject. This is shown with \textitbf{cuma} ‘just’ in (0) and \textitbf{hanya} ‘only’ in (0). Most of them can also take a pre-predicate position directly preceding the predicate; the exceptions are \textitbf{apalagi} ‘moreover’, \textitbf{kecuali} ‘except’ and \textitbf{kususnya} ‘especially’ which always precede the subject. The position directly preceding the predicate is shown with \textitbf{cuma} ‘just’ in (0) and \textitbf{hanya} ‘only’ in (0), respectively. Another exception is pre-predicate \textitbf{juga} ‘also’, which always directly precedes the predicate, as in (0); for its post-predicate uses see (0). These examples also illustrate that focus adverbs not only modify verbal predicates, as in (0), and (0) to (0), but also nonverbal predicates, such as the numeral predicate \textitbf{dua} ‘two’ in (0).


\begin{styleExampleTitle}
Focus adverbs in clause-initial and pre-predicate positions
\end{styleExampleTitle}

\begin{tabular}{lllllllll}
\lsptoprule
\label{bkm:Ref362874870}
\gll {baru{\Tilde}baru} {de} {su} {turung,} {\bluebold{cuma}} {de} {su} {pulang}\\ %
& just.now & \textsc{3sg} & already & descend & just & \textsc{3sg} & already & go.home\\
\lspbottomrule
\end{tabular}
\ea
\glt 
[Reply to an interlocutor who is looking for someone:] ‘just now he already came by, (it’s) \bluebold{just} (that) he already went home’ \textstyleExampleSource{[080922-001a-CvPh.0554]}
\z

\begin{tabular}{llllllll}
\lsptoprule
\label{bkm:Ref358051249}
\gll {…} {tapi} {[sa} {pu} {alpa]} {\bluebold{cuma}} {dua}\\ %
&  & but & \textsc{1sg} & \textsc{poss} & be.absent & just & two\\
\lspbottomrule
\end{tabular}
\ea
\glt 
[About unexcused school absences:] ‘[I was absent many times,] but I had just two (official) absences’ (Lit. ‘my being absent was \bluebold{just} two’) \textstyleExampleSource{[081023-004-Cv.0014]}
\z

\begin{tabular}{llllllllllll}
\lsptoprule
\label{bkm:Ref362874872}
\gll {jadi} {\multicolumn{2}{l}{kalo}} {nika} {di} {kantor} {itu} {begitu,} {\bluebold{hanya}} {dong} {bilang}\\ %
& so & \multicolumn{2}{l}{if} & marry & at & office & \textsc{d.dist} & like.that & only & \textsc{3pl} & say\\
& \multicolumn{2}{l}{nika} & \multicolumn{9}{l}{sipil}\\
& \multicolumn{2}{l}{marry} & \multicolumn{9}{l}{be.civil}\\
\lspbottomrule
\end{tabular}
\ea
\glt 
[About marrying civically:] ‘so if (one) marries at the office it’s like that, \bluebold{only} (that) they call (it) ‘marrying civically’’ \textstyleExampleSource{[081110-007-CvPr.0030]}
\z

\begin{tabular}{lllllll}
\lsptoprule
\label{bkm:Ref358051250}
\gll {prempuang} {\bluebold{hanya}} {duduk} {makang} {pinang} {saja}\\ %
& woman & only & sit & eat & betel.nut & just\\
\lspbottomrule
\end{tabular}
\ea
\glt 
‘the girls \bluebold{just} sit (around) and eat betel nut’ \textstyleExampleSource{[081014-007-CvEx.0045]}
\z

\begin{tabular}{lllllll}
\lsptoprule
\label{bkm:Ref358046050}
\gll {Ise} {dong} {\bluebold{juga}} {duduk} {di} {sana}\\ %
& Ise & \textsc{3pl} & also & sit & at & \textsc{l.dist}\\
\lspbottomrule
\end{tabular}
\ea
\glt 
‘Ise and the others are \bluebold{also} sitting over there’ \textstyleExampleSource{[081025-009b-Cv.0075]}
\z


Three focus adverbs take a post-predicate position, namely \textitbf{juga} ‘also’, \textitbf{lagi} ‘again, also’, and \textitbf{saja} ‘just’. This is demonstrated with the examples in (0) to (0). (As shown in (0), \textitbf{juga} ‘also’ can also take a pre-predicate position.) In clauses with peripheral adjuncts, the three adverbs can directly follow the predicate, such as the first \textitbf{juga} ‘also’ token in (0) and \textitbf{lagi} ‘again, also’ in (0). Alternatively, they can follow the adjunct, such as the second \textitbf{juga} ‘also’ token in (0) and \textitbf{lagi} ‘again, also’ in (0). Focus adverb \textitbf{saja} ‘just’ has the same distributional properties as \textitbf{lagi} ‘again, also’. The semantics expressed with the different positions again have to do with scope.


\begin{styleExampleTitle}
Focus adverbs in post-predicate position
\end{styleExampleTitle}

\begin{tabular}{llllllllllllllll}
\lsptoprule
\label{bkm:Ref358046049}
\gll {\multicolumn{2}{l}{dari}} {\multicolumn{2}{l}{sini}} {\multicolumn{2}{l}{deng}} {\multicolumn{2}{l}{Papua-Lima,}} {\multicolumn{2}{l}{kembali}} {\multicolumn{2}{l}{\bluebold{juga}}} {deng} {\multicolumn{2}{l}{Papua-Lima}}\\ %
& \multicolumn{2}{l}{from} & \multicolumn{2}{l}{\textsc{l.prox}} & \multicolumn{2}{l}{with} & \multicolumn{2}{l}{Papua-Lima} & \multicolumn{2}{l}{return} & \multicolumn{2}{l}{also} & with & \multicolumn{2}{l}{Papua-Lima}\\
& … & ke & sana & \multicolumn{2}{l}{deng} & \multicolumn{2}{l}{Papua-Lima} & \multicolumn{2}{l}{kembali} & \multicolumn{2}{l}{deng} & \multicolumn{3}{l}{Papua-Lima} & \bluebold{juga}\\
&  & to & \textsc{l.dist} & \multicolumn{2}{l}{with} & \multicolumn{2}{l}{Papua-Lima} & \multicolumn{2}{l}{return} & \multicolumn{2}{l}{with} & \multicolumn{3}{l}{Papua-Lima} & also\\
\lspbottomrule
\end{tabular}
\ea
\glt 
‘(I’ll leave) from here with the Papua-Lima (ship) and return \bluebold{also} with the Papua-Lima (ship) … (I’ll get) over there with the Papua-Lima (ship and) return with the Papua-Lima (ship) \bluebold{also}’ \textstyleExampleSource{[080922-001a-CvPh.0483/0493]}
\z

\begin{tabular}{llllll}
\lsptoprule
\label{bkm:Ref363134152}
\gll {de} {kembali} {\bluebold{lagi}} {ke} {Papua}\\ %
& \textsc{3sg} & return & again & to & Papua\\
\lspbottomrule
\end{tabular}
\ea
\glt 
‘he came back \bluebold{again} to Papua’ \textstyleExampleSource{[081025-004-Cv.0008]}
\z

\begin{tabular}{llllll}
\lsptoprule
\label{bkm:Ref363134153}
\gll {sa} {pulang} {ke} {Waim} {\bluebold{lagi}}\\ %
& \textsc{1sg} & go.home & to & Waim & again\\
\lspbottomrule
\end{tabular}
\ea
\glt
‘I went home to Waim \bluebold{again}’ \textstyleExampleSource{[081015-005-NP.0051]}
\end{styleFreeTranslEngxvpt}

\subsection{Degree adverbs}
\label{bkm:Ref358227559}
Degree adverbs “describe the extent of a characteristic”, that is, they “emphasize that a characteristic is either greater or less than some typical level” {(Biber et al. 2002: 209). }Amplifiers or intensifiers “increase intensity”, while diminishers or downtoners “decrease the effect of the modified item” {(2002: 209–210).}



The Papuan Malay degree adverbs are presented in Table  ‎5 .27. The table includes four amplifiers/intensifiers and two diminishers/downtoners. Most of the adverbs occur in pre-predicate position. The exception is \textitbf{skali} ‘very’, which takes a post-predicate position. Two of the amplifiers modify gradable verbs, namely \textitbf{lebi} ‘more’ and \textitbf{paling} ‘most’. The former signals comparative degree while the latter marks superlative degree.


\begin{stylecaption}
\label{bkm:Ref362710241}Table ‎5.\stepcounter{Table}{\theTable}:  Degree adverbs and their positions within the clause
\end{stylecaption}

\tablehead{
\multicolumn{2}{l}{ Item} & Gloss & \multicolumn{2}{l}{ Position within the clause}\\
\multicolumn{2}{l}{} &  & \textsc{pre-pred} & \arraybslash \textsc{post-pred}\\
}
\begin{tabular}{lllll}
\lsptoprule
\multicolumn{5}{l}{Amplifiers/intensifiers}\\
& \textitbf{lebi} & ‘more’ & X & \\
& \textitbf{paling} & ‘most’ & X & \\
& \textitbf{terlalu} & ‘too’ & X & \\
& \textitbf{skali} & ‘very’ &  & \arraybslash X\\
\multicolumn{5}{l}{Diminishers/downtoners}\\
& \textitbf{agak} & ‘rather’ & X & \\
& \textitbf{hampir} & ‘almost’ & X & \\
\lspbottomrule
\end{tabular}

The four amplifiers modify monovalent stative and bivalent verbs, as discussed in §5.3.4 and §5.3.5 (comparative constructions are discussed in §11.5). The amplifiers occur in pre-predicate position, following the subject, such as \textitbf{paling} ‘most’ in (0). Furthermore, \textitbf{paling} ‘most’ can precede the subject, although not very often. In this clause-initial position it functions as an epistemic modal adverb which has scope over the entire proposition, as in (0) (modal adverbs are discussed in §5.4.4).


\begin{styleExampleTitle}
Amplifier degree adverbs
\end{styleExampleTitle}

\begin{tabular}{lllll}
\lsptoprule
\label{bkm:Ref362855798}
\gll {ana} {ini} {\bluebold{paling}} {bodo}\\ %
& child & \textsc{d.prox} & most & be.stupid\\
\lspbottomrule
\end{tabular}
\ea
\glt 
‘this child is \bluebold{most} stupid’ \textstyleExampleSource{[081011-005-Cv.0035]}
\z

\begin{tabular}{lllllllllllllll}
\lsptoprule
\label{bkm:Ref362855804}
\gll {\multicolumn{2}{l}{waktu}} {\multicolumn{2}{l}{saya}} {bilang} {\multicolumn{2}{l}{sa}} {\multicolumn{2}{l}{mo}} {biking} {acara,} {\bluebold{paling}} {sa} {tra}\\ %
& \multicolumn{2}{l}{time} & \multicolumn{2}{l}{\textsc{1sg}} & say & \multicolumn{2}{l}{\textsc{1sg}} & \multicolumn{2}{l}{want} & make & ceremony & most & \textsc{1sg} & \textsc{neg}\\
& kerja, & \multicolumn{2}{l}{sa} & \multicolumn{3}{l}{sebagey} & \multicolumn{2}{l}{kepala} & \multicolumn{6}{l}{acara}\\
& work & \multicolumn{2}{l}{\textsc{1sg}} & \multicolumn{3}{l}{as} & \multicolumn{2}{l}{head} & \multicolumn{6}{l}{ceremony}\\
\lspbottomrule
\end{tabular}
\ea
\glt 
‘when I say, I want to hold a festivity, \bluebold{most likely} I won’t (have to) work, I’ll be the head of the festivity’ \textstyleExampleSource{[080919-004-NP.0068]}
\z


The intensifier \textitbf{terlalu} ‘too’ also occurs in pre-predicate position, as in (0). By contrast, \textitbf{skali} ‘very’ takes a post-predicate position, as illustrated in (0) to (0). In clauses with peripheral adjuncts, as in (0), \textitbf{skali} ‘very’ follows the predicate, such as \textitbf{enak} ‘be pleasant’ in (0). Clauses in which \textitbf{skali} ‘very’ follows the peripheral adjunct, as in the elicited example in (0), are ungrammatical.


\begin{styleExampleTitle}
Intensifier degree adverbs
\end{styleExampleTitle}

\begin{tabular}{lllll}
\lsptoprule
\label{bkm:Ref362855800}
\gll {a,} {ko} {\bluebold{terlalu}} {bodo}\\ %
& ah! & \textsc{2sg} & too & be.stupid\\
\lspbottomrule
\end{tabular}
\ea
\glt 
‘ah, you are \bluebold{too} stupid’ \textstyleExampleSource{[080917-003a-CvEx.0009]}
\z

\begin{tabular}{lllllllll}
\lsptoprule
\label{bkm:Ref362855802}
\gll {ade} {bongso} {jadi} {ko} {sayang} {dia} {\bluebold{skali}} {e?}\\ %
& ySb & youngest.offspring & so & \textsc{2sg} & love & \textsc{3sg} & very & eh\\
\lspbottomrule
\end{tabular}
\ea
\glt 
‘(your) youngest sibling, so you love her \bluebold{very much}, eh?’ \textstyleExampleSource{[080922-001a-CvPh.0302]}
\z

\begin{tabular}{lllllll}
\lsptoprule
\label{bkm:Ref371769368}
\gll {kamu} {orang-tua} {enak} {\bluebold{skali}} {di} {sana}\\ %
& \textsc{2pl} & parent & be.pleasant & very & at & \textsc{l.dist}\\
\lspbottomrule
\end{tabular}
\ea
\glt 
‘you, the parents, (have) \bluebold{very pleasant} (lives) over there’ (Lit. ‘you … are \bluebold{very pleasant}’) \textstyleExampleSource{[081115-001a-Cv.0106]}
\z

\begin{tabular}{llllllll}
\lsptoprule
\label{bkm:Ref363124188}
\gll {*} {kamu} {orang-tua} {enak} {di} {sana} {\bluebold{skali}}\\ %
&  & \textsc{2pl} & parent & pleasant & at & \textsc{l.dist} & very\\
\lspbottomrule
\end{tabular}
\ea
\glt 
Intended reading: ‘you, the parents, (have) \bluebold{very pleasant} (lives) over there’ \textstyleExampleSource{[Elicited MY131113.004]}
\z


The diminishers \textitbf{agak} ‘rather’ and \textitbf{hampir} ‘almost’ also occur in pre-predicate position, as illustrated in (0) to (0). Always directly preceding the verb, \textitbf{agak} ‘rather’ modifies stative verbs, as in (0). Clauses in which \textitbf{agak} ‘rather’ precedes the subject, as in the elicited example in (0), are ungrammatical. Diminisher \textitbf{hampir} ‘almost’ typically modifies dynamic verbs, as in (0) and (0).\footnote{\\
\\
\\
\\
\\
\\
\\
\\
\\
\\
\\
\\
According to one consultant, some Papuan Malay speakers also use \textitbf{hampir} ‘almost’ to modify stative verbs. Much more often though they employ a construction with \textitbf{su mulay} ‘already start to’ as in (i) below:
[After an accident:] ‘and then we went (back) to school, (our wounds) \bluebold{were almost healed}’ (Lit. ‘\bluebold{already started to be healed}’) [081014-012-NP.0005]\par \\
\\
\\
} The adverb can directly precede the predicate, as in the elicited example in (0), or precede the subject, as in (0). In the corpus, \textitbf{hampir} ‘almost’ always occurs in the latter position, where the adverb has scope over the entire proposition.


\begin{styleExampleTitle}
Diminisher/downtoner degree adverbs
\end{styleExampleTitle}

\begin{tabular}{lllll}
\lsptoprule
\label{bkm:Ref358118747}
\gll {sa} {su} {\bluebold{agak}} {besar}\\ %
& \textsc{1sg} & already & rather & be.big\\
\lspbottomrule
\end{tabular}
\ea
\glt 
[About the speaker’s childhood:] ‘I was already \bluebold{rather} big’ \textstyleExampleSource{[080922-008-CvNP.0025]}
\z

\begin{tabular}{llllll}
\lsptoprule
\label{bkm:Ref371770751}
\gll {*} {\bluebold{agak}} {sa} {su} {besar}\\ %
&  & rather & \textsc{1sg} & already & be.big\\
\lspbottomrule
\end{tabular}
\ea
\glt 
Intended reading: ‘I was already \bluebold{rather} big’ \textstyleExampleSource{[Elicited MY131113.006]}
\z

\begin{tabular}{lllll}
\lsptoprule
\label{bkm:Ref358118749}
\gll {dong} {\bluebold{hampir}} {bunu} {bapa}\\ %
& \textsc{3pl} & almost & kill & father\\
\lspbottomrule
\end{tabular}
\ea
\glt 
‘they \bluebold{almost} killed (my) father’ \textstyleExampleSource{[Elicited MY131113.005]}
\z

\begin{tabular}{lllll}
\lsptoprule
\label{bkm:Ref358118748}
\gll {\bluebold{hampir}} {dong} {bunu} {bapa}\\ %
& almost & \textsc{3pl} & kill & father\\
\lspbottomrule
\end{tabular}
\ea
\glt
‘(it) \bluebold{almost} (happened that) they killed (my) father’ \textstyleExampleSource{[081011-022-Cv.0210]}
\end{styleFreeTranslEngxvpt}

\subsection{Expressing manner}
\label{bkm:Ref358227566}
Papuan Malay does not have manner adverbs. Instead, manner is expressed through stative verbs, as shown in (0) to (0). The modifying stative verbs always take a post-predicate position. In (0), for instance, post-predicate stative \textitbf{kras} ‘be harsh’ modifies stative \textitbf{sakit} ‘be sick’, and in (0) \textitbf{trus} ‘be continuous’ modifies \textitbf{tatap dia} ‘observe him’. In verbal clauses with peripheral adjuncts, the modifying stative verb can directly follow the predicate as in (0), or follow the adjunct, as in (0).
\end{styleBodyxafter}

\begin{tabular}{lllllll}
\lsptoprule
\label{bkm:Ref357794300}
\gll {baru} {satu} {kali} {sa} {sakit} {\bluebold{kras}}\\ %
& and.then & one & time & \textsc{1sg} & be.sick & be.harsh\\
\lspbottomrule
\end{tabular}
\ea
\glt 
‘but then one time I was \bluebold{badly} sick’ \textstyleExampleSource{[080922-008-CvNP.0009]}
\z

\begin{tabular}{llllll}
\lsptoprule
(\stepcounter{}{\the}) & e, & kam & mandi & \bluebold{cepat} & suda!\\
& hey! & \textsc{2pl} & bathe & be.fast & already\\
\lspbottomrule
\end{tabular}
\ea
\glt 
‘hey, you bathe \bluebold{quickly}!’ \textstyleExampleSource{[080917-008-NP.0128]}
\z

\begin{tabular}{lllll}
\lsptoprule
(\stepcounter{}{\the}) & dong & dua & lari & \bluebold{trus}\\
& \textsc{3pl} & two & run & be.continuous\\
\lspbottomrule
\end{tabular}
\ea
\glt 
[About a motorbike trip:] ‘the two of them drove \bluebold{continuously}’ \textstyleExampleSource{[081015-005-NP.0011]}
\z

\begin{tabular}{llllll}
\lsptoprule
\label{bkm:Ref363136730}
\gll {langsung} {sa} {tatap} {dia} {\bluebold{trus}}\\ %
& immediately & \textsc{1sg} & gaze.at & \textsc{3sg} & be.continuous\\
\lspbottomrule
\end{tabular}
\ea
\glt 
‘immediately I gazed at him \bluebold{continuously}’ \textstyleExampleSource{[081006-035-CvEx.0071]}
\z

\begin{tabular}{llllll}
\lsptoprule
\label{bkm:Ref363136731}
\gll {de} {buka} {\bluebold{trus}} {siang} {malam}\\ %
& \textsc{3sg} & open & be.continuous & day & night\\
\lspbottomrule
\end{tabular}
\ea
\glt 
[About opening hours of an office] ‘it is open \bluebold{continuously} day and night’ \textstyleExampleSource{[081005-001-Cv.0003]}
\z

\begin{tabular}{llllll}
\lsptoprule
\label{bkm:Ref363136732}
\gll {…} {terendam} {di} {air} {\bluebold{trus}}\\ %
&  & be.soaked & at & water & be.continuous\\
\lspbottomrule
\end{tabular}
\ea
\glt
[About a motorbike that got stuck in a river:] ‘[(the motorbike) is still there …,] (it) is immersed in water \bluebold{continuously}’ \textstyleExampleSource{[081008-003-Cv.0026]}
\end{styleFreeTranslEngxvpt}

\subsection{Summary}
\label{bkm:Ref363225033}
The Papuan Malay adverbs take different positions within the clause, that is, they can occur in pre-predicate or in post-predicate position. The most common position, however, is the pre-predicate one. There are also a fair number of adverbs which can occur in both positions.



For the pre-predicate adverbs two positions are attested, one directly preceding the predicate and one preceding the subject. A fair number of pre-predicate adverbs can occur in both positions. Likewise, for the post-predicate adverbs two positions are attested, one directly following the predicate and, in clauses with peripheral adjuncts, one following the adjunct. Most post-predicate adverbs can occur in both positions. In terms of their functions, the adverbs designate aspect, frequency, affirmation and negation, modality, time, focus, and degree; manner is expressed through stative verbs in post-predicate position.
\end{styleBodyvafter}


Listed according to their semantic functions, the adverbs have the following distributional preferences.
\end{styleBodyvvafter}

%\setcounter{itemize}{0}
\begin{itemize}
\item \begin{styleOvNvwnext}
Aspect adverbs
\end{styleOvNvwnext}\end{itemize}
\begin{styleIiI}
They only occur in pre-predicate position, directly preceding the predicate.
\end{styleIiI}

\begin{itemize}
\item \begin{styleOvNvwnext}
Frequency adverbs
\end{styleOvNvwnext}\end{itemize}
\begin{styleIiI}
They only occur in pre-predicate position where they directly precede the predicate or the subject.
\end{styleIiI}

\begin{itemize}
\item \begin{styleOvNvwnext}
Affirmation and negation adverbs
\end{styleOvNvwnext}\end{itemize}
\begin{styleIiI}
They always occur in a predicate position. The affirmation adverb always precedes the subject, while the three negation adverbs directly precede the predicate.
\end{styleIiI}

\begin{itemize}
\item \begin{styleOvNvwnext}
Modal adverbs
\end{styleOvNvwnext}\end{itemize}
\begin{styleIiI}
All epistemic and evaluative adverbs take a pre-predicate position, preceding the subject. Besides, most of the epistemic adverbs can also directly precede the predicate; the exceptions are \textitbf{artinya} ‘that means’ and \textitbf{maksutnya} ‘that is to say’ which always precede the subject.
\end{styleIiI}

\begin{itemize}
\item \begin{styleOvNvwnext}
Temporal adverbs
\end{styleOvNvwnext}\end{itemize}
\begin{styleIiI}
All but two can occur in pre- or in post-predicate position. In pre-predicate position, the adverbs can directly precede the predicate or the subject. In post-predicate position, they always follow the predicate and, in clauses with peripheral adjuncts, precede the adjunct. Two adverbs only occur in pre-predicate position, namely \textitbf{baru} ‘recently’ and \textitbf{baru{\Tilde}baru} ‘just now’.
\end{styleIiI}

\begin{itemize}
\item \begin{styleOvNvwnext}
Focus adverbs
\end{styleOvNvwnext}\end{itemize}
\begin{styleIiI}
All but three only occur in pre-predicate position where they can directly precede the predicate or the subject. The exceptions are \textitbf{juga} ‘also’, \textitbf{lagi} ‘again, also’, and \textitbf{saja} ‘just’, which take a post-predicate position. While \textitbf{lagi} ‘again, also’, and \textitbf{saja} ‘just’ only occur in post-predicate position, \textitbf{juga} ‘also’ also takes a pre-predicate position. In post-predicate position, the three adverbs can either directly follow the predicate or, in clauses with peripheral adjuncts, follow the adjunct.
\end{styleIiI}

\begin{itemize}
\item \begin{styleOvNvwnext}
Degree adverbs
\end{styleOvNvwnext}\end{itemize}
\begin{styleIiI}
All but one only take a pre-predicate position, where most of them directly precede the predicate. The exception is \textitbf{hampir} ‘almost’ which can also precede the subject. The one degree adverb which is unattested in pre-predicate position is \textitbf{skali} ‘very’. It only occurs in post-predicate position, directly following the predicate.
\end{styleIiI}


These distributional preferences are summarized in Table  ‎5 .28.


\begin{stylecaption}
\label{bkm:Ref363146853}Table ‎5.\stepcounter{Table}{\theTable}:  Papuan Malay adverbs and their positions within the clause
\end{stylecaption}

\tablehead{
 Adverb type & \multicolumn{2}{l}{ Positions within the clause}\\
\hhline{~--} & \textsc{pre-pred} & \arraybslash \textsc{post-pred}\\
}
\begin{tabular}{lll}
\lsptoprule
Aspect & all \textsc{adv} & none\\
Frequency & all \textsc{adv} & none\\
Affirmation/negation & all \textsc{adv} & none\\
Modal & all \textsc{adv} & none\\
Temporal & all \textsc{adv} & most \textsc{adv}\\
Focus & most \textsc{adv} & three \textsc{adv}\\
Degree & most \textsc{adv} & one \textsc{adv}\\
\lspbottomrule
\end{tabular}

As for those adverbs which can take more than one position within the clause, the semantic distinctions conveyed by the different positions have to do with scope. Overall, however, these distinctions are subtle and require further investigation.
\end{styleBodyaftervbefore}


Papuan Malay does not have manner adverbs. Instead, manner is expressed with monovalent stative verbs which always take a post-predicate position.
\end{styleBodyvxvafter}

\subsection{Morphological properties}

Papuan Malay has only two somewhat productive affixes, as discussed in Chapter 3, prefix \textscItalBold{ter\-} ‘\textsc{acl}’ and suffix \-\textitbf{ang} ‘\textsc{nmlz}’. Mono- and bivalent verbs can be prefixed with \textscItalBold{ter\-} ‘\textsc{acl}’ to derive verbs which denote accidental or unintentional actions or events. Examples are given in Table  ‎5 .29, such as bivalent \textitbf{angkat} ‘lift’ and \textitbf{lempar} ‘throw’, monovalent dynamic \textitbf{jatu} ‘fall’, and stative \textitbf{lambat} ‘be slow’ and \textitbf{sala} ‘be wrong’. Likewise, mono- and bivalent verbs can be suffixed with \-\textitbf{ang} ‘\textsc{nmlz}’ to derive nouns, such as bivalent \textitbf{jual} ‘sell’ and \textitbf{pake} ‘use’, monovalent dynamic \textitbf{jalang} ‘walk’ and \textitbf{libur} ‘take vacation’, and stative \textitbf{pica} ‘be broken’ and \textitbf{sial} ‘be unfortunate’. Some lexemes suffixed with \-\textitbf{ang} ‘\textsc{nmlz}’ also function as verbs, such as \textitbf{jualang} ‘merchandise, to sell’. Affixation of trivalent verbs is unattested. (For details on affixation with \textscItalBold{ter\-} ‘\textsc{acl}’ and \-\textitbf{ang} ‘\textsc{nmlz}’ see §3.1.2 and §3.1.3, respectively).


\begin{stylecaption}
\label{bkm:Ref336254478}Table ‎5.\stepcounter{Table}{\theTable}:  Affixation of verbs
\end{stylecaption}

\begin{tabular}{llll}
\lsptoprule

 BF & Gloss & Item & \arraybslash Gloss\\
\multicolumn{4}{l}{Prefix \textscItalBold{ter\-}: Derived verbs denoting accidental actions}\\
\textitbf{angkat} & ‘lift’ & \textitbf{trangkat} & ‘be lifted up’\\
\textitbf{lempar} & ‘throw’ & \textitbf{talempar} & ‘be thrown’\\
\textitbf{jatu} & ‘fall’ & \textitbf{terjatu} & ‘be dropped, fall’\\
\textitbf{lambat} & ‘be slow’ & \textitbf{terlambat} & ‘be late’\\
\textitbf{sala} & ‘be wrong’ & \textitbf{tasala} & ‘be mistaken’\\
\multicolumn{4}{l}{Suffix \-\textitbf{ang}: Derived nouns}\\
\textitbf{jual} & ‘sell’ & \textitbf{jualang} & ‘merchandise, to sell’\\
\textitbf{pake} & ‘use’ & \textitbf{pakeang} & ‘clothes’\\
\textitbf{jalang} & ‘walk’ & \textitbf{jalangang} & ‘route’\\
\textitbf{libur} & ‘take vacation’ & \textitbf{liburang} & ‘vacation’\\
\textitbf{pica} & ‘be broken’ & \textitbf{picaang} & ‘splinter’\\
\textitbf{sial} & ‘be unfortunate’ & \textitbf{sialang} & ‘s.o. unfortunate/ill-fated’\\
\lspbottomrule
\end{tabular}

In the corpus, affixation of bivalent bases occurs much more often than that of monovalent bases, as shown in Table  ‎5 .30. Regarding prefix \textscItalBold{ter\-} ‘\textsc{acl}’, the corpus includes 43 lexemes derived from verbal bases with a total of 166 tokens. Most of them are bivalent verbs (88\%), accounting for 92\% of all \textscItalBold{ter\-}tokens. As for suffix \-\textitbf{ang} ‘\textsc{nmlz}’, the corpus contains 69 lexemes with verbal bases, with a total of 403 tokens. Again, most of the verbal bases are bivalent (90\%), accounting for 89\% of all \-\textitbf{ang}{}-tokens.


\begin{stylecaption}
\label{bkm:Ref367965433}Table ‎5.\stepcounter{Table}{\theTable}:  Affixation of verbs
\end{stylecaption}

\tablehead{ & \multicolumn{2}{l}{ Token frequencies} & \multicolumn{2}{l}{ Type frequencies}\\
}
\begin{tabular}{lllll} & \multicolumn{2}{l}{ \textscItalBold{ter\-}affixation} & \multicolumn{2}{l}{ Different verbs}\\
\lsptoprule
Verb class & \raggedleft \# & \raggedleft \% & \raggedleft \# & \raggedleft\arraybslash \%\\
\textsc{v.tri} & \raggedleft 0 & \raggedleft {}-{}-{}- & \raggedleft 0 & \raggedleft\arraybslash {}-{}-{}-\\
\textsc{v.bi} & \raggedleft 153 & \raggedleft 91.6\% & \raggedleft 38 & \raggedleft\arraybslash 88.4\%\\
\textsc{v.mo}(\textsc{dy}) & \raggedleft 1 & \raggedleft 0.6\% & \raggedleft 1 & \raggedleft\arraybslash 2.3\%\\
\textsc{v.mo}(\textsc{st}) & \raggedleft 13 & \raggedleft 7.8\% & \raggedleft 4 & \raggedleft\arraybslash 9.3\%\\
Total & \raggedleft 167 & \raggedleft 100.0\% & \raggedleft 43 & \raggedleft\arraybslash 100.0\%\\
& \multicolumn{2}{l}{ \textitbf{{}-ang} affixation} & \multicolumn{2}{l}{ Different verbs}\\
Verb class & \raggedleft \# & \raggedleft \% & \raggedleft \# & \raggedleft\arraybslash \%\\
\textsc{v.tri} & \raggedleft 0 & \raggedleft {}-{}-{}- & \raggedleft 0 & \raggedleft\arraybslash {}-{}-{}-\\
\textsc{v.bi} & \raggedleft 357 & \raggedleft 88.6\% & \raggedleft 62 & \raggedleft\arraybslash 89.9\%\\
\textsc{v.mo}(\textsc{st}) & \raggedleft 12 & \raggedleft 3.0\% & \raggedleft 3 & \raggedleft\arraybslash 4.3\%\\
\textsc{v.mo}(\textsc{dy}) & \raggedleft 34 & \raggedleft 8.4\% & \raggedleft 4 & \raggedleft\arraybslash 5.8\%\\
Total & \raggedleft 403 & \raggedleft 100.0\% & \raggedleft 69 & \raggedleft\arraybslash 100.0\%\\
\lspbottomrule
\end{tabular}
\subsection{Summary}
\label{bkm:Ref353805297}
Tri-, bi-, and monovalent verbs have partially distinct and partially overlapping properties, which are summarized in Table  ‎11 .5 (in this table bi- and trivalent verbs are listed summarily in the column ‘Valency of 2 or 3’). They are distinct from each other in terms of two main criteria, namely their valency and their function, which is mainly predicative. Related to the criterion on valency is the ability of verbs to occur in causative and reciprocal expressions and to be affixed. Therefore Table  ‎11 .5 lists these characteristic under the label ‘valency’. The criterion of function has to do with the predicative (\textsc{pred}) and attributive (\textsc{attr}) uses of the verbs, their negation, and adverbial modification. Hence, Table  ‎11 .5 lists these characteristic under the label ‘function’.


\begin{stylecaption}
Table ‎5.\stepcounter{Table}{\theTable}:  Properties of tri-, bi-, and monovalent verbs\footnote{\\
\\
\\
\\
\\
\\
\\
\\
\\
\\
\\
\\
\par See {van Klinken (1999: 51–53)} for a similar approach to distinguishing different verb classes.\\
\\
\\
}
\end{stylecaption}

\begin{tabular}{lllll}
\lsptoprule

 Main criteria & Properties & Valency of 2 or 3 & \multicolumn{2}{l}{ Valency of 1}\\
\hhline{---~~} &  &  & dynamic & \arraybslash stative\\
 Function & Adverbial modification & Yes & Yes & \arraybslash Yes\\
\hhline{-~~~~} & Negation (\textitbf{tida} or \textitbf{tra}) & Yes & Yes & \arraybslash Yes\\
& \textsc{pred} uses & Most often & Most often & \arraybslash Less often\\
& \textsc{attr} uses (via relative clause) & Most often & Less often & \arraybslash Less often\\
 Valency & Base for \textscItalBold{ter\-} affixation & Most often & Less often & \arraybslash Less often\\
\hhline{-~~~~} & Base for \-\textitbf{ang} affixation & Most often & Less often & \arraybslash Less often\\
& Causative (\textitbf{kasi}) & Most often & Less often & \arraybslash Less often\\
& Reciprocal & Most often & Less often & \arraybslash No\\
& Valency {\textgreater}1 & Yes & No & \arraybslash No\\
& Causative (\textitbf{biking}) & Less often & No & \arraybslash Most often\\
 Function & \textsc{attr} uses (via juxtaposition) & Less often & No & \arraybslash Most often\\
\hhline{-~~~~} & Intensification (\textitbf{skali}) & Less often & No & \arraybslash Most often\\
& Intensification (\textitbf{terlalu}) & Less often & No & \arraybslash Most often\\
& Grading (\textitbf{lebi}) & Less often & No & \arraybslash Most often\\
& Grading (\textitbf{paling}) & Less often & No & \arraybslash Most often\\
\hhline{~----}
\lspbottomrule
\end{tabular}

In terms of valency, Papuan Malay has three verb classes, mono-, bi- and trivalent verbs. Related to the valency criterion is the ability of verbs to be used in causative constructions. All three verb types occur in causatives formed with \textitbf{kasi} ‘give’. Most often, however, \textitbf{kasi}{}-causatives are formed with bi- or trivalent verbs. By contrast, causative constructions with \textitbf{biking} ‘make’ are typically formed with stative verbs; dynamic verbs are unattested in \textitbf{biking}{}-causatives. Also related to the valency criterion is the ability of bi- and trivalent verbs to occur in reciprocal expressions. Monovalent dynamic verbs, by contrast, occur only rarely in such expressions, while reciprocal constructions with stative verbs are unattested. Finally, with respect to affixation, it is typically bivalent verbs that form the bases for lexemes prefixed with \textscItalBold{ter\-} or suffixed with \-\textitbf{ang}.
\end{styleBodyaftervbefore}


With respect to their function, all verbs are used predicatively, dynamic verbs much more often though, than stative verbs. In their predicate uses, all three verb types can be modified adverbially and all verbs are negated with \textitbf{tida}/\textitbf{tra} ‘\textsc{neg}’. Less often, verbs have attributive function in noun phrases. Verb-via-juxtaposition modification most commonly applies to stative verbs, while modification with dynamic verbs typically involves verb-via-relative-clause modification. Related to their attributive uses is the intensification and grading of verbs. Typically, this applies to monovalent stative verbs, while intensification and grading of bivalent verbs occurs much less often. Monovalent dynamic and trivalent verbs are neither intensified nor graded.
\end{styleBodyvxvafter}

\section{Personal pronouns}
\label{bkm:Ref351641095}\label{bkm:Ref350501171}
The Papuan Malay personal pronoun system distinguishes singular and plural numbers and three persons; the personal pronouns do not mark case, clusivity, gender, or politeness. Referring to animate and inanimate entities, they allow the unambiguous identification of their referents. They do so by signaling not only the person-number values of their referents, but also their definiteness.



The Papuan Malay personal pronouns are presented in Table  ‎5 .32.
\end{styleBodyvvafter}

\begin{stylecaption}
\label{bkm:Ref401858945}Table ‎5.\stepcounter{Table}{\theTable}:  Personal pronoun system with long and short forms
\end{stylecaption}

\begin{tabular}{lll} & Long forms & \arraybslash Short forms\\
\lsptoprule
\textsc{1sg} & \textitbf{saya} & \textitbf{sa}\\
\textsc{2sg} & \textitbf{ko} & \textitbf{{}-{}-{}-}\\
\textsc{3sg} & \textitbf{dia} & \textitbf{de}\\
\textsc{1pl} & \textitbf{kitong} & \textitbf{tong}\\
& \textitbf{kita} & \textitbf{ta}\\
& \textitbf{kitorang} & \textitbf{torang}\\
\textsc{2pl} & \textitbf{kamu} & \textitbf{kam}\\
\textsc{3pl} & \textitbf{dorang} & \textitbf{dong}\\
\lspbottomrule
\end{tabular}

Each personal pronoun, except for \textsc{2sg}, has at least one long and one short form. The use of the long and short pronoun forms does not mark grammatical distinctions but represents speaker preferences. These distributional preferences are discussed in detail in §6.1.1.
\end{styleBodyaftervbefore}


The Papuan Malay personal pronouns have the following distributional properties:
\end{styleBodyvvafter}

%\setcounter{itemize}{0}
\begin{itemize}
\item \begin{styleIIndented}
Substitution for noun phrases (pronominal uses) (§6.1).
\end{styleIIndented}\item \begin{styleIIndented}
Modification with demonstratives, locatives, numerals, quantifiers, prepositional phrases, and/or relative clauses (pronominal uses) (§6.1).
\end{styleIIndented}\item \begin{styleIvI}
Co-occurrence with noun phrases (adnominal uses): \textsc{n/np} \textsc{pro} (§6.2)\textsc{.}
\end{styleIvI}\end{itemize}

Personal pronouns are distinct from other word classes such as nouns (§5.2) and demonstratives (§5.6) in terms of the following distributional properties:


%\setcounter{itemize}{0}
\begin{itemize}
\item \begin{styleIIndented}
Personal pronouns are distinct from nouns in that personal pronouns (a) very commonly modify nouns, while nouns do not modify personal pronouns, (b) are modified with numerals/quantifiers in post-head position, while with nouns the modifying numerals/quantifiers can also occur in pre-head position, and (c) only designate the possessor in adnominal possessive constructions, while nouns can also express the possessum.
\end{styleIIndented}\item \begin{styleIvI}
Unlike demonstratives, personal pronouns (a) express person and number, (b) signal definiteness, while demonstratives indicate specificity,\footnote{\\
\\
\\
\\
\\
\\
\\
\\
\\
\\
\\
\\
\par According to {\citet[148]{Andrews2007}}, definiteness indicates that “an \textsc{np} has [...] a referent uniquely identifiable to the hearer”. Hence, the hearer is expected to be in a position to identify the referent. Specificity, by contrast, signals that “the speaker is referring to a particular instance of an entity as opposed to any instance of it” {(2007: 148)}. That is, the identifiability of the referent is not presupposed. Instead, the speaker makes the entity under discussion identifiable to the hearer by pointing out “a particular instance of an entity” among other possible referents {(2007: 148)}. (See also {Abbot 2006}.)\\
\\
\\
} and (c) cannot be stacked.
\end{styleIvI}\end{itemize}

The personal pronouns have pronominal and adnominal uses. This is illustrated with two examples. The utterance in (0) demonstrates the pronominal uses of short \textitbf{sa} ‘\textsc{1sg}’ and long \textitbf{dia} ‘\textsc{3sg}’, while the example in (0) shows the adnominal uses of short \textitbf{dong} ‘\textsc{3pl}’. The personal pronouns are discussed in detail in Chapter 6.
\end{styleBodyxafter}

\begin{tabular}{llllllllll}
\lsptoprule
\label{bkm:Ref353279710}
\gll {ana} {itu} {\bluebold{sa}} {paling} {sayang} {\bluebold{dia}} {\bluebold{tu}} {ana} {itu}\\ %
& child & \textsc{d.dist} & \textsc{1sg} & most & love & \textsc{3sg} & \textsc{d.dist} & child & \textsc{d.dist}\\
\lspbottomrule
\end{tabular}
\ea
\glt 
‘that child, \bluebold{I} love \bluebold{her (}\blueboldSmallCaps{emph}\bluebold{)} most, that child’ \textstyleExampleSource{[081011-023-Cv.0097]}
\z

\begin{tabular}{llll}
\lsptoprule
\label{bkm:Ref353281436}
\gll {\bluebold{Natanael}} {\bluebold{dong}} {menang}\\ %
& Natanael & \textsc{3pl} & win\\
\lspbottomrule
\end{tabular}
\ea
\glt
[About a volleyball game:] ‘\bluebold{Natanael and his friends} won’ \textstyleExampleSource{[081109-001-Cv.0002]}
\end{styleFreeTranslEngxvpt}

\section{Demonstratives}
\label{bkm:Ref350501736}
Papuan Malay has a two-term demonstrative system: proximal \textitbf{ini} ‘\textsc{d.prox}’ and distal \textitbf{itu} ‘\textsc{d.dist}’, together with their reduced fast-speech forms \textitbf{ni} ‘\textsc{d.prox}’ and \textitbf{tu} ‘\textsc{d.dist}’. As deictic expressions they orient the hearers and signal specificity. That is, they draw the hearers’ their attention to particular occurrences of an entity in the surrounding situation or in the discourse. While \textitbf{ini} ‘\textsc{d.prox}’ indicates proximity of this entity, \textitbf{itu} ‘\textsc{d.dist}’ signals its distance – in spatial and in non-spatial terms.



Papuan Malay demonstratives have the following distributional properties:
\end{styleBodyvvafter}

%\setcounter{itemize}{0}
\begin{itemize}
\item \begin{styleIIndented}
Co-occurrence with noun phrases (adnominal uses): \textsc{n}/\textsc{np} \textsc{dem} (§5.6.1).
\end{styleIIndented}\item \begin{styleIIndented}
Substitution for noun phrases (pronominal uses) (§5.6.2).
\end{styleIIndented}\item \begin{styleIIndented}
Modification with relative clauses (pronominal uses (§5.6.2).
\end{styleIIndented}\item \begin{styleIIndented}
Co-occurrence with verbs or adverbs (adverbial uses) (§5.6.3).
\end{styleIIndented}\item \begin{styleIvI}
Stacking of demonstratives: \textsc{dem} \textsc{dem} and \textsc{n} \textsc{dem} \textsc{dem} (§5.6.4).
\end{styleIvI}\end{itemize}

Demonstratives are distinct from other word classes such as personal pronouns (§5.5) and locatives (§5.7) in terms of the following syntactic properties:


%\setcounter{itemize}{0}
\begin{itemize}
\item \begin{styleIIndented}
Demonstratives are distinct from personal pronouns, in that demonstratives (a) have adverbial uses, (b) can be stacked, (c) can take the possessum slot in adnominal possessive constructions, and (d) signal specificity, while personal pronouns express definiteness.\footnote{\\
\\
\\
\\
\\
\\
\\
\\
\\
\\
\\
\\
\par Concerning the semantic distinctions between the notion of definiteness and the notion of specificity see Footnote 164 in §5.5 (p. \pageref{bkm:Ref438387261}).\\
\\
\\
}
\end{styleIIndented}\item \begin{styleIvI}
Contrasting with locatives, demonstratives (a) are employed as independent nominals in unembedded noun phrases, (b) occur in adnominal possessive constructions either as the possessor or the possessum, and (c) can be stacked.
\end{styleIvI}\end{itemize}

The adnominal uses of the demonstratives are discussed in §5.6.1, their pronominal uses in §5.6.2, their adverbial uses in §5.6.3, and stacking of demonstratives in §5.6.4. A full discussion of the Papuan Malay demonstratives is presented in §7.1.
\end{styleBodyxvafter}

\subsection{Adnominal uses}
\label{bkm:Ref350500971}
Adnominally used demonstratives occur in post-head position at the right periphery of the noun phrase. That is, all noun phrase constituents occur to the left of the demonstrative, with the demonstrative having scope over the entire noun phrase as illustrated in (0) to (0). Constituents occurring to the right of the demonstratives such as \textitbf{liar} ‘be wild’ in (0) are not part of the noun phrase: \textitbf{liar} ‘be wild’ is a clausal predicate. The examples in (0) and (0) show that the demonstratives signal specificity (and not definiteness). The noun phrase \textitbf{tanta dia itu} ‘that aunt’ (literally ‘that she aunt’) designates a specific and definite referent with distal \textitbf{itu} ‘\textsc{d.dist}’ indicating specificity while adnominally used \textitbf{dia} ‘\textsc{3sg}’ signals definiteness (§5.5). By contrast, the noun phrase \textitbf{ana kecil satu ini} ‘this particular small child’ in (0) denotes a specific but indefinite referent with proximal \textitbf{ini} ‘\textsc{d.prox}’ again indicating specificity while post-head \textitbf{satu} ‘one’ signals indefiniteness (see also §5.9.4).


\begin{styleExampleTitle}
Post-head demonstratives: Scope
\end{styleExampleTitle}

\begin{tabular}{llllllll}
\lsptoprule
\label{bkm:Ref350501071}
\gll {Wili} {ko} {jangang} {gara{\Tilde}gara} {\bluebold{tanta}} {\bluebold{dia}} {\bluebold{itu}}\\ %
& Wili & \textsc{2sg} & \textsc{neg.imp} & \textsc{rdp}{\Tilde}irritate & aunt & \textsc{3sg} & \textsc{d.dist}\\
\lspbottomrule
\end{tabular}
\ea
\glt 
‘you Wili don’t irritate \bluebold{that aunt}’ \textstyleExampleSource{[081023-001-Cv.0038]}
\z

\begin{tabular}{lllllllllllllll}
\lsptoprule
\label{bkm:Ref350501073}
\gll {\multicolumn{2}{l}{baru}} {\multicolumn{2}{l}{\bluebold{ana}}} {\multicolumn{2}{l}{\bluebold{kecil}}} {\multicolumn{2}{l}{\bluebold{satu}}} {\multicolumn{3}{l}{\bluebold{ini}}} {de} {tra} {gambar}\\ %
& \multicolumn{2}{l}{and.then} & \multicolumn{2}{l}{child} & \multicolumn{2}{l}{be.small} & \multicolumn{2}{l}{one} & \multicolumn{3}{l}{\textsc{d.prox}} & \textsc{3sg} & \textsc{neg} & draw\\
& \bluebold{ana} & \multicolumn{2}{l}{\bluebold{murit}} & \multicolumn{2}{l}{\bluebold{satu}} & \multicolumn{2}{l}{\bluebold{ni}} & \multicolumn{2}{l}{de} & tra & \multicolumn{4}{l}{gambar}\\
& child & \multicolumn{2}{l}{pupil} & \multicolumn{2}{l}{one} & \multicolumn{2}{l}{\textsc{d.prox}} & \multicolumn{2}{l}{\textsc{3sg}} & \textsc{neg} & \multicolumn{4}{l}{draw}\\
\lspbottomrule
\end{tabular}
\ea
\glt 
‘but then \bluebold{this particular small child}, he doesn’t draw, \bluebold{this particular school kid}, he doesn’t draw’ \textstyleExampleSource{[081109-002-JR.0002]}
\z

\begin{tabular}{llllllll}
\lsptoprule
\label{bkm:Ref350501074}
\gll {Papua-Satu} {ada} {muncul} {dari} {\bluebold{laut}} {\bluebold{sana}} {\bluebold{itu}}\\ %
& Papua-Satu & exist & appear & from & sea & \textsc{l.dist} & \textsc{d.dist}\\
\lspbottomrule
\end{tabular}
\ea
\glt 
‘(the ship) Papua-Satu is appearing from \bluebold{the sea over there (}\blueboldSmallCaps{emph}\bluebold{)}’ \textstyleExampleSource{[080917-008-NP.0129]}
\z

\begin{tabular}{llllll}
\lsptoprule
\label{bkm:Ref350501075}
\gll {…} {karna} {\bluebold{babi}} {\bluebold{ini}} {liar}\\ %
&  & because & pig & \textsc{d.prox} & be.wild\\
\lspbottomrule
\end{tabular}
\ea
\glt 
‘… because \bluebold{this pig is wild}’ \textstyleExampleSource{[080919-004-NP.0019]}
\z


Demonstratives can also modify constituents other than nouns, namely personal pronouns as in (0), interrogatives as in (0), or locatives as in (0).


\begin{styleExampleTitle}
Post-head demonstratives: Modifying personal pronouns, interrogatives, or locatives
\end{styleExampleTitle}

\begin{tabular}{lllllllll}
\lsptoprule
\label{bkm:Ref350501076}
\gll {\bluebold{ko}} {\bluebold{itu}} {manusia} {yang} {tra} {taw} {bicara} {temang}\\ %
& \textsc{2sg} & \textsc{d.dist} & human.being & \textsc{rel} & \textsc{neg} & know & speak & friend\\
\lspbottomrule
\end{tabular}
\ea
\glt 
‘\bluebold{you (}\blueboldSmallCaps{emph}\bluebold{)} are a human being who doesn’t know how to talk (badly about) friends’ \textstyleExampleSource{[081115-001a-Cv.0245]}
\z

\begin{tabular}{lllllllll}
\lsptoprule
\label{bkm:Ref350501077}
\gll {ana} {laki{\Tilde}laki} {ini} {de} {mo} {ke} {\bluebold{mana}} {\bluebold{ni}}\\ %
& child & \textsc{rdp}{\Tilde}husband & \textsc{d.prox} & \textsc{3sg} & want & to & where & \textsc{d.prox}\\
\lspbottomrule
\end{tabular}
\ea
\glt 
‘this boy, \bluebold{where (}\blueboldSmallCaps{emph}\bluebold{)} does he want to (go)?’ \textstyleExampleSource{[080922-004-Cv.0017]}
\z

\begin{tabular}{llllllll}
\lsptoprule
\label{bkm:Ref350501078}
\gll {di} {\bluebold{sini}} {\bluebold{tu}} {ada} {orang} {swanggi} {satu}\\ %
& at & \textsc{l.prox} & \textsc{d.dist} & exist & person & nocturnal.evil.spirit & one\\
\lspbottomrule
\end{tabular}
\ea
\glt
‘\bluebold{here (}\blueboldSmallCaps{emph}\bluebold{)} is a certain evil sorcerer’ \textstyleExampleSource{[081006-022-CvEx.0150]}
\end{styleFreeTranslEngxvpt}

\subsection{Pronominal uses}
\label{bkm:Ref350500973}
In their pronominal uses, the demonstratives stand for noun phrases, as illustrated in (0) to (0). They occur in all syntactic positions within the clause. In (0), a demonstrative takes the subject slot, in (0) the direct object slot, and in (0) the oblique object slot.


\begin{styleExampleTitle}
Pronominal uses in argument position
\end{styleExampleTitle}

\begin{tabular}{lllll}
\lsptoprule
\label{bkm:Ref350501079}
\gll {yo,} {\bluebold{itu}} {mo} {putus}\\ %
& yes & \textsc{d.dist} & want & break\\
\lspbottomrule
\end{tabular}
\ea
\glt 
[About redirecting a river for a street building project:] ‘yes, \bluebold{it} (the river) is going to get dispersed’ \textstyleExampleSource{[081006-033-Cv.0064]}
\z

\begin{tabular}{lllll}
\lsptoprule
\label{bkm:Ref350501082}
\gll {ko} {suka} {makang} {\bluebold{ini}}\\ %
& \textsc{2sg} & like & eat & \textsc{d.prox}\\
\lspbottomrule
\end{tabular}
\ea
\glt 
[About fried bananas:] ‘do you like to eat \bluebold{these}?’ \textstyleExampleSource{[081006-023-CvEx.0071-0072]}
\z

\begin{tabular}{lllll}
\lsptoprule
\label{bkm:Ref350501083}
\gll {dong} {percaya} {\bluebold{sama}} {\bluebold{itu}}\\ %
& \textsc{3pl} & trust & to & \textsc{d.dist}\\
\lspbottomrule
\end{tabular}
\ea
\glt 
[About believing in evil spirits:] ‘they believe \bluebold{in those}’ \textstyleExampleSource{[081006-023-CvEx.0001]}
\z


In their pronominal uses, the demonstratives can be modified with relative clauses, as in the elicited example in (0).


\begin{styleExampleTitle}
Modification of pronominally used demonstratives with relative clauses
\end{styleExampleTitle}

\begin{tabular}{lllllllllll}
\lsptoprule
\label{bkm:Ref362511882}
\gll {sa} {\multicolumn{2}{l}{pili}} {\bluebold{ini}} {yang} {mera,} {ade} {pili} {\bluebold{itu}} {yang}\\ %
& \textsc{1sg} & \multicolumn{2}{l}{choose} & \textsc{d.prox} & \textsc{rel} & be.red & ySb & choose & \textsc{d.dist} & \textsc{rel}\\
& \multicolumn{2}{l}{warna} & \multicolumn{8}{l}{puti}\\
& \multicolumn{2}{l}{color} & \multicolumn{8}{l}{be.white}\\
\lspbottomrule
\end{tabular}
\ea
\glt 
[About buying new shirts:] ‘I chose \bluebold{this} (one) which is red, (my) younger sibling chose \bluebold{that} (one) which is (of) white color’ \textstyleExampleSource{[Elicited MY131119.004]}
\z


Pronominally used demonstratives also occur in adnominal possessive constructions (see Chapter 9). They can designate the possessor as in (0) or the possessum as in (0).


\begin{styleExampleTitle}
Pronominal uses in adnominal possessive constructions
\end{styleExampleTitle}

\begin{tabular}{llllllll}
\lsptoprule
\label{bkm:Ref350501084}
\gll {bapa} {masi} {kenal} {…} {\bluebold{ini}} {\bluebold{pu}} {\bluebold{muka}}\\ %
& father & still & know &  & \textsc{d.prox} & \textsc{poss} & face\\
\lspbottomrule
\end{tabular}
\ea
\glt 
‘do you (‘father’) still know … \bluebold{this (one)’s face}?’ \textstyleExampleSource{[080922-001a-CvPh.1123]}
\z

\begin{tabular}{lllllll}
\lsptoprule
\label{bkm:Ref350501085}
\gll {ko} {ambil} {dulu} {\bluebold{ade}} {\bluebold{pu}} {\bluebold{itu}}\\ %
& \textsc{2sg} & fetch & first & ySb & \textsc{poss} & \textsc{d.dist}\\
\lspbottomrule
\end{tabular}
\ea
\glt
‘you pick up (the fish) first, \bluebold{that (fish) of the younger sister}’ (Lit. ‘\bluebold{younger sibling’s that}’) \textstyleExampleSource{[081006-019-Cv.0002]}
\end{styleFreeTranslEngxvpt}

\subsection{Adverbial uses}
\label{bkm:Ref350500974}
In their adverbial uses, the demonstratives co-occur with verbs as in \textitbf{pikir ni} ‘think (\textsc{emph})’ in (0) or with adverbs as in \textitbf{pasti tu} ‘exactly (\textsc{emph})’ in (0).
\end{styleBodyxafter}

\begin{tabular}{llllllll}
\lsptoprule
\label{bkm:Ref350501086}
\gll {de} {\bluebold{pikir}} {\bluebold{ni},} {dong} {ribut} {apa} {ka}\\ %
& \textsc{3sg} & think & \textsc{d.prox} & \textsc{3pl} & trouble & what & or\\
\lspbottomrule
\end{tabular}
\ea
\glt 
‘he \bluebold{thought} \bluebold{(}\blueboldSmallCaps{emph}\bluebold{)}, ‘what are they troubled about?’’ \textstyleExampleSource{[081014-005-Cv.0036]}
\z

\begin{tabular}{llllllll}
\lsptoprule
\label{bkm:Ref350501087}
\gll {yo,} {brita} {\bluebold{pasti}} {\bluebold{tu}} {yang} {sa} {bilang}\\ %
& yes & new & definitely & \textsc{d.dist} & \textsc{rel} & \textsc{1sg} & say\\
\lspbottomrule
\end{tabular}
\ea
\glt
‘yes, the news are \bluebold{exactly} \bluebold{(}\blueboldSmallCaps{emph}\bluebold{)} as I told (you)’ \textstyleExampleSource{[080922-001a-CvPh.0767]}
\end{styleFreeTranslEngxvpt}

\subsection{Stacking of demonstratives}
\label{bkm:Ref350500975}
Papuan Malay also allows the stacking of demonstratives. Typically, only identical demonstratives are stacked, as in (0) and (0). To some degree, however, non-identical demonstratives can also be stacked, as shown in (0) and (0).\footnote{\\
\\
\\
\\
\\
\\
\\
\\
\\
\\
\\
\\
\par Juxtaposed \textitbf{ini ni} ‘\textsc{d.prox d.prox}’ and \textitbf{itu tu} ‘\textsc{d.dist d.dist}’ are not taken as instances of partial reduplication. As discussed in §4.1.2, partial reduplication of the stems \textitbf{ini} ‘\textsc{d.prox}’ and \textitbf{itu} ‘\textsc{d.dist}’ should result in the reduplicated forms \textitbf{in{\Tilde}ini} ‘\textsc{d.prox{\Tilde}d.prox}’ and \textitbf{it{\Tilde}itu} ‘\textsc{d.dist}{\Tilde}\textsc{d.dist}’, respectively. Therefore, \textitbf{ini ni} ‘\textsc{d.prox d.prox}’ and \textitbf{itu tu} ‘\textsc{d.dist} \textsc{d.dist}’ are taken as an instance of demonstrative stacking. Reduplication of demonstratives does occur, however, as discussed in §4.2.5.2.\\
\\
\\
}



In (0), short proximal \textitbf{ni} ‘\textsc{d.prox}’ modifies the pronominally used long proximal demonstrative, such that ‘\textsc{dem} \textsc{dem}’. In (0), short distal \textitbf{tu} ‘\textsc{d.dist}’ modifies a nested noun phrase with the adnominally used long distal demonstrative, such that ‘[[\textsc{n} \textsc{dem}] \textsc{dem}]’.
\end{styleBodyvxafter}

\begin{tabular}{lllllllll}
\lsptoprule
\label{bkm:Ref350501088}
\gll {ada} {segala} {macang} {tulang} {dia} {buang} {[\bluebold{ini}} {\bluebold{ni}]}\\ %
& exist & all & variety & bone & \textsc{3sg} & throw(.away) & \textsc{d.prox} & \textsc{d.prox}\\
\lspbottomrule
\end{tabular}
\ea
\glt 
‘there were all kinds of bones, he threw away \bluebold{these very (ones)}’ \textstyleExampleSource{[080922-010a-CvNF.0101]}
\z

\begin{tabular}{lllllllll}
\lsptoprule
\label{bkm:Ref373943895}
\gll {waktu} {kitorang} {masuk} {di} {[[\bluebold{ruma}} {\bluebold{itu}]} {\bluebold{tu}]} {…}\\ %
& when & \textsc{1pl} & go.in & at & house & \textsc{d.dist} & \textsc{d.dist} & \\
\lspbottomrule
\end{tabular}
\ea
\glt 
‘when we moved into \bluebold{that very house}, …’ \textstyleExampleSource{[081006-022-CvEx.0167]}
\z


While unattested in the corpus, speakers do allow one combination of non-identical demonstrative stacking in elicitation. Acceptable is the order of proximal \textitbf{ini} ‘\textsc{d.prox}’ followed by short distal \textitbf{itu} ‘\textsc{d.dist}’, as shown in the elicited example in (0). The reverse order is not permitted by speakers even in elicitation, as illustrated in (0). At this point in the research on Papuan Malay, however, the semantics of ‘\textsc{n} \textitbf{ini tu}’ constructions as compared to ‘\textsc{n} \textitbf{ini ni}’ and ‘\textsc{n} \textitbf{itu tu}’ constructions remain uncertain.


\begin{styleExampleTitle}
Non-identical stacked demonstratives\footnote{\\
\\
\\
\\
\\
\\
\\
\\
\\
\\
\\
\\
\par The elicited examples are based on the example in (0) in §7.1.2.3 (p. \pageref{bkm:Ref439954271}).\\
\\
\\
}
\end{styleExampleTitle}

\begin{tabular}{llllllll}
\lsptoprule
\label{bkm:Ref338955288}
\gll {\bluebold{orang}} {\bluebold{ini}} {\bluebold{tu}} {percaya} {sama} {Tuhang} {Yesus}\\ %
& person & \textsc{d.prox} & \textsc{d.dist} & trust & to & God & Jesus\\
\lspbottomrule
\end{tabular}
\ea
\glt 
‘\bluebold{that person here} believes in God Jesus’ \textstyleExampleSource{[Elicited BR111017.009]}
\z

\begin{tabular}{lllllllll}
\lsptoprule
\label{bkm:Ref338955289}
\gll {*} {\bluebold{orang}} {\bluebold{itu}} {\bluebold{ni}} {percaya} {sama} {Tuhang} {Yesus}\\ %
&  & person & \textsc{d.dist} & \textsc{d.prox} & trust & to & God & Jesus\\
\lspbottomrule
\end{tabular}
\ea
\glt
Intended reading: ‘\bluebold{this person there} believes in God Jesus’ \textstyleExampleSource{[Elicited BR111017.010]}
\end{styleFreeTranslEngxvpt}

\section{Locatives}
\label{bkm:Ref350501176}
Papuan Malay has a distance oriented three-term locative system: proximal \textitbf{sini} ‘\textsc{l.prox}’, medial \textitbf{situ} ‘\textsc{l.med}’, and distal \textitbf{sana} ‘\textsc{l.dist}’. The locatives provide orientation to the hearer in the outside world and in the speech situation by signaling distance, both spatial and non-spatial. Hence, they are similar to the demonstratives. The demonstratives, however, draw the hearer’s attention to specific entities in the discourse or surrounding situation. The locatives, by contrast, focus the hearer’s attention to the specific location of these entities and the relative distance of this location to the deictic center.



The distributional properties of the locatives are as follows:
\end{styleBodyvvafter}

%\setcounter{itemize}{0}
\begin{itemize}
\item \begin{styleIIndented}
Substitution for noun phrases embedded in prepositional phrases (pronominal uses) (§5.7.1).
\end{styleIIndented}\item \begin{styleIIndented}
Modification with demonstratives or relative clauses (pronominal uses) (§5.7.1).
\end{styleIIndented}\item \begin{styleIvI}
Co-occurrence with noun phrases (adnominal uses): \textsc{n}/\textsc{np} \textsc{loc} (§5.7.2).
\end{styleIvI}\end{itemize}

Locatives are distinct from other word classes such as personal pronouns (§5.5) or demonstratives (§5.6) in terms of the following syntactic properties:


%\setcounter{itemize}{0}
\begin{itemize}
\item \begin{styleIIndented}
They only occur in prepositional phrases; that is, they are unattested as nominal heads in unembedded noun phrases.
\end{styleIIndented}\item \begin{styleIIndented}
They can be modified with adnominally used demonstratives and relatives clauses, but with no other adnominal modifier; hence, locatives cannot be stacked.
\end{styleIIndented}\item \begin{styleIvI}
They are unattested in adnominal possessive constructions as possessor or as possessum.
\end{styleIvI}\end{itemize}

The pronominal uses of the locatives are discussed in §5.7.1 and their adnominal uses in §5.7.2. Generally speaking, the pronominally used locatives provide additional information about the location of an entity, information non-essential for its identification. Adnominally used locatives, by contrast, limit the referential scope of their head nominals and thereby assist in the identification of their referents. A full discussion of the Papuan Malay locatives is found in §7.2.
\end{styleBodyxvafter}

\subsection{Pronominal uses}
\label{bkm:Ref350501723}
In their pronominal uses the locatives substitute or stand for noun phrases, as illustrated with the four elicited contrastive examples in (0). Distal \textitbf{sana} ‘\textsc{l.dist}’ in (0) substitutes for the noun phrase \textitbf{ruma yang paling di bawa itu} ‘that house that’s the furthest down’ in (0). The ungrammatical construction in (0) shows that the locative replaces the entire noun phrase and not only its nominal head \textitbf{ruma} ‘house’.


\begin{styleExampleTitle}
Pronominal uses: Substitution for noun phrases
\end{styleExampleTitle}

\begin{tabular}{lllllllllll}
\lsptoprule
\label{bkm:Ref339637288}\label{bkm:Ref319487134}
\gll {\label{bkm:Ref319487158}} {sa} {tinggal} {di} {\bluebold{ruma}} {\bluebold{yang}} {\bluebold{paling}} {\bluebold{di}} {\bluebold{bawa}} {\bluebold{itu}}\\ %
&  & \textsc{1sg} & stay & at & house & \textsc{rel} & most & at & bottom & \textsc{d.dist}\\
\lspbottomrule
\end{tabular}
\begin{styleFreeTranslIndentiicmEng}
‘I live in \bluebold{the house that’s the furthest down there}’ \textstyleExampleSource{[Elicited FS120314-001.007]}
\end{styleFreeTranslIndentiicmEng}

\begin{tabular}{llllll} & \label{bkm:Ref255574651} & sa & tinggal & di & \bluebold{sana}\\
\lsptoprule
&  & \textsc{1sg} & stay & at & \textsc{l.dist}\\
\lspbottomrule
\end{tabular}
\begin{styleFreeTranslIndentiicmEng}
‘I live \bluebold{over there}’ \textstyleExampleSource{[Elicited FS120314-001.008]}
\end{styleFreeTranslIndentiicmEng}

\begin{tabular}{llllllllllll} & \label{bkm:Ref319487162} & * & sa & tinggal & di & \bluebold{sana} & \bluebold{yang} & \bluebold{paling} & \bluebold{di} & \bluebold{bawa} & \bluebold{itu}\\
\lsptoprule
&  &  & \textsc{1sg} & stay & at & \textsc{l.dist} & \textsc{rel} & most & at & bottom & \textsc{d.dist}\\
\lspbottomrule
\end{tabular}
\begin{styleFreeTranslIndentiicmEng}
Intended reading: ‘I live \bluebold{over there that’s the furthest down}’ \textstyleExampleSource{[Elicited FS120314-001.010]}
\end{styleFreeTranslIndentiicmEng}


Locatives are always embedded in prepositional phrases. The prepositional phrase can be a peripheral adjunct, as in the first clause in (0) or in (0), a prepositional predicate, as in the second clause in (0), or an adnominal prepositional phrase, as in (0). Usually, the locatives are introduced with an overt preposition as in (0) or (0). The preposition may, however, also be elided as in (0): the omitted preposition is allative \textitbf{ke} ‘to’ (the elision of prepositions is discussed in §10.1.5).


\begin{styleExampleTitle}
Pronominal uses in prepositional phrases
\end{styleExampleTitle}

\begin{tabular}{llllllllll}
\lsptoprule
\label{bkm:Ref339637289}
\gll {ko} {datang} {\bluebold{ke}} {\bluebold{sini},} {nanti} {bapa} {\bluebold{ke}} {\bluebold{situ}} {…}\\ %
& \textsc{2sg} & come & to & \textsc{l.prox} & very.soon & father & to & \textsc{l.med} & \\
\lspbottomrule
\end{tabular}
\ea
\glt 
‘you come \bluebold{here}, later I (‘father’) (go) \bluebold{there} …’ \textstyleExampleSource{[080922-001a-CvPh.0462]}
\z

\begin{tabular}{llllllllll}
\lsptoprule
\label{bkm:Ref339637290}
\gll {\bluebold{orang}} {\bluebold{dari}} {\bluebold{sana}} {\bluebold{itu}} {…} {dorang} {itu} {kerja} {sendiri}\\ %
& person & from & \textsc{l.dist} & \textsc{d.dist} &  & \textsc{3pl} & \textsc{d.dist} & work & be.alone\\
\lspbottomrule
\end{tabular}
\ea
\glt 
‘\bluebold{those people from over there}, …. they work by themselves’ \textstyleExampleSource{[081014-007-CvEx.0050]}
\z

\begin{tabular}{lllllllll}
\lsptoprule
\label{bkm:Ref339637291}
\gll {kam} {datang} {\bluebold{Ø}} {\bluebold{sini},} {kam} {biking} {kaco} {saja}\\ %
& \textsc{2pl} & come &  & \textsc{l.prox} & \textsc{2pl} & make & be.confused & just\\
\lspbottomrule
\end{tabular}
\ea
\glt 
‘you come \bluebold{here}, you’re just stirring up trouble’ \textstyleExampleSource{[081025-007-Cv.0013]}
\z


The pronominally used locatives can be modified with demonstratives or relative clauses. Modification with the demonstratives typically involves short distal \textitbf{tu} ‘\textsc{d.dist}’ as in (0), while modification with long distal \textitbf{itu} ‘\textsc{d.dist}’ is only attested for the non-proximal locatives, as in \textitbf{sana itu} ‘over there (\textsc{emph})’ in (0). These distributional patterns still require further investigation. Modification with proximal \textitbf{ini} ‘\textsc{d.prox}’ is unattested but possible, as shown in the elicited example in (0). Modification with relative clauses is also possible, as illustrated for proximal \textitbf{sini} ‘\textsc{l.prox}’ in (0) and medial \textitbf{situ} ‘\textsc{l.med}’ in (0). In the corpus, however, such modification is rare and unattested for distal \textitbf{sana} ‘\textsc{l.dist}’.


\begin{styleExampleTitle}
Modification of pronominally used locatives
\end{styleExampleTitle}

\begin{tabular}{llllllllll}
\lsptoprule
\label{bkm:Ref339637292}
\gll {sampe} {di} {\bluebold{sini}} {\bluebold{tu}} {dia} {langsung} {sakit} {karna} {…}\\ %
& reach & at & \textsc{l.prox} & \textsc{d.dist} & \textsc{3sg} & immediately & be.sick & because & \\
\lspbottomrule
\end{tabular}
\ea
\glt 
‘having arrived \bluebold{here (}\blueboldSmallCaps{emph}\bluebold{)}, he was sick immediately because (he hadn’t eaten)’ \textstyleExampleSource{[081025-008-Cv.0050]}
\z

\begin{tabular}{llllll}
\lsptoprule
\label{bkm:Ref339637293}
\gll {dong} {lobe} {ke} {\bluebold{sana}} {\bluebold{itu}}\\ %
& \textsc{3pl} & walk.searchingly.with.lamp & to & \textsc{l.dist} & \textsc{d.dist}\\
\lspbottomrule
\end{tabular}
\ea
\glt 
‘they walk searchingly with lights \bluebold{to over there (}\blueboldSmallCaps{emph}\bluebold{)}’ \textstyleExampleSource{[081108-001-JR.0002]}
\z

\begin{tabular}{lllllllll}
\lsptoprule
\label{bkm:Ref339637294}
\gll {di} {\bluebold{sini}} {\bluebold{ni}} {orang} {tida} {taw} {makang} {pinang}\\ %
& at & \textsc{l.prox} & \textsc{d.prox} & person & \textsc{neg} & know & eat & betel.nut\\
\lspbottomrule
\end{tabular}
\ea
\glt 
‘\bluebold{here (}\blueboldSmallCaps{emph}\bluebold{)} people don’t habitually eat betel nuts’ \textstyleExampleSource{[Elicited BR111017.001]}
\z

\begin{tabular}{llllll}
\lsptoprule
\label{bkm:Ref362513109}
\gll {di} {\bluebold{sini}} {\bluebold{yang}} {tra} {banyak}\\ %
& at & \textsc{l.prox} & \textsc{rel} & \textsc{neg} & many\\
\lspbottomrule
\end{tabular}
\ea
\glt 
[About logistic problems:] ‘(it’s) \bluebold{here where} there weren’t many (passengers)’ \textstyleExampleSource{[081025-008-Cv.0140]}
\z

\begin{tabular}{lllllllllll}
\lsptoprule
\label{bkm:Ref362513110}
\gll {…} {sa} {mandi} {di} {situ,} {di} {\bluebold{situ}} {\bluebold{yang}} {mungking} {nangka}\\ %
&  & \textsc{1sg} & bathe & at & \textsc{l.med} & at & \textsc{l.med} & \textsc{rel} & maybe & jackfruit\\
\lspbottomrule
\end{tabular}
\ea
\glt
‘[I saw (the poles),] I bathed there, \bluebold{there where} (there are) maybe jackfruits’ \textstyleExampleSource{[080922-010a-CvNF.0298]}
\end{styleFreeTranslEngxvpt}

\subsection{Adnominal uses}
\label{bkm:Ref350501726}
Adnominally used locatives always occur in post-head position. Most commonly, they occur in noun phrases embedded in prepositional phrases, as illustrated in (0). In (0), proximal \textitbf{sini} ‘\textsc{l.prox}’ modifies the locational noun \textitbf{sebla} ‘side’; the noun phrases is introduced with allative \textitbf{ke} ‘to’. In (0) distal \textitbf{sana} ‘\textsc{l.dist}’ modifies the noun \textitbf{laut} ‘sea’; the preposition is locative \textitbf{di} ‘at, in’.


\begin{styleExampleTitle}
Adnominal uses in embedded noun phrases
\end{styleExampleTitle}

\begin{tabular}{llllllllll}
\lsptoprule
\label{bkm:Ref339637299}\label{bkm:Ref319046839}
\gll {\label{bkm:Ref320620692}} {\bluebold{ke}} {\bluebold{sebla}} {\bluebold{sini}} {} {\label{bkm:Ref320620694}} {\bluebold{di}} {\bluebold{laut}} {\bluebold{sana}}\\ %
&  & to & side & \textsc{l.prox} &  &  & at & sea & \textsc{l.dist}\\
&  & \multicolumn{3}{l}{\bluebold{‘to the side here}’\\
\textstyleExampleSource{[081011-001-Cv.0148]}} &  &  & \multicolumn{3}{l}{‘\bluebold{in the sea over there}’\\
\textstyleExampleSource{[080917-006-CvHt.0004]}}\\
\lspbottomrule
\end{tabular}

Adnominally used locatives also occur in unembedded noun phrases as in (0), although considerably less frequently. In (0), proximal \textitbf{sini} ‘\textsc{l.prox}’ modifies the personal pronoun \textitbf{dong} ‘\textsc{3pl}’, while in (0) medial \textitbf{situ} ‘\textsc{l.med}’ modifies the noun phrase \textitbf{orang kantor} ‘office employees’.


\begin{styleExampleTitle}
Adnominal uses in unembedded noun phrases
\end{styleExampleTitle}

\begin{tabular}{llllllllll}
\lsptoprule
\label{bkm:Ref339637300}\label{bkm:Ref320620988}
\gll {\label{bkm:Ref320620695}} {\bluebold{dong}} {\bluebold{sini}} {} {} {\label{bkm:Ref320620697}} {\bluebold{orang}} {\bluebold{kantor}} {\bluebold{situ}}\\ %
&  & \textsc{3pl} & \textsc{l.prox} &  &  &  & person & office & \textsc{l.med}\\
&  & \multicolumn{3}{l}{‘\bluebold{they here}’\\
\textstyleExampleSource{[080922-001a-CvPh.0556]}} &  &  & \multicolumn{3}{l}{‘\bluebold{the office employees there}’ \textstyleExampleSource{[081005-001-Cv.0018]}}\\
\lspbottomrule
\end{tabular}
\section{Interrogatives}
\label{bkm:Ref358367283}
Papuan Malay has six interrogatives which serve to form content questions. That is, marking a clause as a question, they signal to the hearer which piece of information is being asked for.



The Papuan Malay interrogatives and their functions within the clause are presented in Table  ‎5 .33. All of them are used pronominally. Most of them also have predicative uses; the exception is \textitbf{kapang} ‘when’. Besides, the majority of interrogatives also have adnominal uses, except for \textitbf{bagemana} ‘how’, \textitbf{kapang} ‘when’, and \textitbf{knapa} ‘why’ which are unattested. Furthermore, three interrogatives are used as placeholders. In their pronominal and adnominal uses, the interrogatives typically remain in-situ, that is, in the position of the constituents they replace.
\end{styleBodyvafter}


Besides, the mid-range quantifier \textitbf{brapa} ‘several’ (§5.10) also functions as an interrogative, which questions quantities in the sense of ‘how many’. For expository reasons, this interrogative function of quantifier \textitbf{brapa} ‘several, how many’ is discussed here.
\end{styleBodyvvafter}

\begin{stylecaption}
\label{bkm:Ref362697609}Table ‎5.\stepcounter{Table}{\theTable}:  Papuan Malay interrogatives and their functions within the clause
\end{stylecaption}

\tablehead{
 Item & Gloss & \multicolumn{4}{l}{ Functions within the clause}\\
&  & \textsc{pronom} & \textsc{adnom} & \textsc{pred} & \arraybslash \textsc{pl-hold}\\
}
\begin{tabular}{llllll}
\lsptoprule
\textitbf{siapa} & ‘who’ & X & X & X & \arraybslash X\\
\textitbf{apa} & ‘what’ & X & X & X & \arraybslash X\\
\textitbf{mana} & ‘where, which’ & X & X & X & \\
\textitbf{bagemana} & ‘how’ & X &  & X & \arraybslash X\\
\textitbf{kapang} & ‘when’ & X &  &  & \\
\textitbf{knapa} & ‘why’ & X &  & X & \\
\textitbf{brapa} & ‘how many’ &  & X & X & \\
\lspbottomrule
\end{tabular}

In their predicative uses, most of the five interrogatives can take two positions, as shown in Table  ‎5 .34; the same applies to quantifier \textitbf{brapa} ‘several, how many’. That is, all of them can remain in-situ, in the unmarked clause-final position, following the clausal subject. Besides, most of them can also be fronted to the marked clause-initial position, preceding the subject; the exception is \textitbf{knapa} ‘why’. In its interrogative uses, quantifier \textitbf{brapa} ‘several, how many’ can also take two positions. That is, it can remain in-situ, in the clause-final position, or be fronted to the marked clause-initial position. These positions and the semantics they convey are discussed in detail in the following sections.


\begin{stylecaption}
\label{bkm:Ref362697611}Table ‎5.\stepcounter{Table}{\theTable}:  Predicatively used interrogatives and their positions within the clause
\end{stylecaption}

\tablehead{
 Item & Gloss & \multicolumn{2}{l}{ Position within the clause}\\
&  & \textsc{cl-initial} & \arraybslash \textsc{cl-final}\\
}
\begin{tabular}{llll}
\lsptoprule
\textitbf{siapa} & ‘who’ & X & \arraybslash X\\
\textitbf{apa} & ‘what’ & X & \arraybslash X\\
\textitbf{mana} & ‘where, which’ & X & \arraybslash X\\
\textitbf{bagemana} & ‘how’ & X & \arraybslash X\\
\textitbf{knapa} & ‘why’ &  & \arraybslash X\\
\textitbf{brapa} & ‘how many’ & X & \arraybslash X\\
\lspbottomrule
\end{tabular}

In the following, the interrogatives are described in turn, \textitbf{siapa} ‘who’ in §5.8.1, \textitbf{apa} ‘what’ in §5.8.2, \textitbf{mana} ‘where, which’ in §5.8.3, \textitbf{bagemana} ‘how’ in §5.8.4, \textitbf{kapang} ‘when’ in §5.8.5, and \textitbf{knapa} ‘why’ in §5.8.6, and quantifier \textitbf{brapa} ‘how many’ in §5.8.7. Some of the interrogatives also express non-interrogative indefinite meanings; this function is summarily discussed in §5.8.8.


\subsection{\textitbf{siapa} ‘who’}
\label{bkm:Ref358363769}
The interrogative \textitbf{siapa} ‘who’ questions the identity of human referents. Its pronominal uses are illustrated in (0) to (0), its adnominal uses in (0) and (0), and its predicative uses in (0) and (0). In addition, \textitbf{siapa} ‘who’ serves as a placeholder as shown in (0). Furthermore, the interrogative is also used in one-word utterances.



In its pronominal uses, \textitbf{siapa} ‘who’ occurs in all syntactic positions, as shown (0) to (0), typically remaining in-situ. In the verbal clause in (0), \textitbf{siapa} ‘who’ takes the subject slot. In the corpus, however, verbal clauses with \textitbf{siapa} ‘who’ in the subject slot are rare. Typically, speakers use equative nominal clauses when they want to question the identity of the clausal subject. In such nonverbal clauses, \textitbf{siapa} ‘who’ takes the subject slot while a headless relative clause takes the predicate slot. This is shown with the elicited contrastive example in (0). In this equative clause, \textitbf{siapa} ‘who’ is the subject, while the headless relative clause \textitbf{yang suru …} ‘(the one) who told …’ is the predicate. Likewise in (0), the interrogative takes the subject slot while the headless relative clause \textitbf{yang datang …} ‘(the one) who came …’ takes the predicate slot.\footnote{\\
\\
\\
\\
\\
\\
\\
\\
\\
\\
\\
\\
\label{bkm:Ref436404513}\par Alternatively, as one anonymous reviewer points out, one could argue that in (0) and (0) the typical subject-predicate word order is inverted and that \textitbf{siapa} ‘who’ does not take the subject but the predicate slot, whereas the relative clause introduced with \textitbf{yang} ‘\textsc{rel}’ takes the subject slot.\\
\\
\\
} (For details on relative clauses see §14.3.2.)
\end{styleBodyvvafter}

\begin{styleExampleTitle}
Pronominal uses of \textitbf{siapa} ‘who’: Subject slot
\end{styleExampleTitle}

\begin{tabular}{llllllll}
\lsptoprule
\label{bkm:Ref362533083}
\gll {e,} {\bluebold{siapa}} {suru} {kam} {minum{\Tilde}minum} {di} {sini?}\\ %
& hey & who & order & \textsc{2pl} & \textsc{rdp}{\Tilde}drink & at & \textsc{l.prox}\\
\lspbottomrule
\end{tabular}
\ea
\glt 
‘hey, \bluebold{who} told you to keep drinking here’ \textstyleExampleSource{[081014-005-Cv.0006]}
\z

\begin{tabular}{lllllllll}
\lsptoprule
\label{bkm:Ref371684418}
\gll {e,} {\bluebold{siapa}} {yang} {suru} {kam} {minum{\Tilde}minum} {di} {sini?}\\ %
& hey & who & \textsc{rel} & order & \textsc{2pl} & \textsc{rdp}{\Tilde}drink & at & \textsc{l.prox}\\
\lspbottomrule
\end{tabular}
\ea
\glt 
‘hey, \bluebold{who} (is the one) who told you to keep drinking here’ \textstyleExampleSource{[Elicited MY131112.004]}
\z

\begin{tabular}{llllll}
\lsptoprule
\label{bkm:Ref356912635}
\gll {\bluebold{siapa}} {yang} {datang} {jemput} {saya?}\\ %
& who & \textsc{rel} & come & pick.up & \textsc{1sg}\\
\lspbottomrule
\end{tabular}
\ea
\glt 
‘\bluebold{who} (is the one) who came (and) picked me up?’ \textstyleExampleSource{[080918-001-CvNP.0001]}
\z


Interrogative \textitbf{siapa} ‘who’ also takes non-subject slots. In (0), \textitbf{siapa} ‘who’ takes the direct object slot in a monotransitive clause, in (0) a direct object slot in a double-object construction, in (0) the oblique object slot, and in (0) the peripheral adjunct slot.\footnote{\\
\\
\\
\\
\\
\\
\\
\\
\\
\\
\\
\\
\par Alternatively, one could argue that in (0) \textitbf{siapa} ‘who’ does not take an oblique object but an adjunct slot.\\
\\
\\
} In addition, \textitbf{siapa} ‘who’ questions the possessor’s identity in adnominal possessive constructions, as in (0).


\begin{styleExampleTitle}
Pronominal uses of \textitbf{siapa} ‘who’: Non-subject slots
\end{styleExampleTitle}

\begin{tabular}{llll}
\lsptoprule
\label{bkm:Ref356912637}
\gll {dong} {cari} {\bluebold{siapa}?}\\ %
& \textsc{3pl} & search & who\\
\lspbottomrule
\end{tabular}
\ea
\glt 
‘for \bluebold{whom} are they looking?’ \textstyleExampleSource{[080921-010-Cv.0010]}
\z

\begin{tabular}{llllllll}
\lsptoprule
\label{bkm:Ref356912638}
\gll {kwe} {mo} {pi} {kasi} {\bluebold{siapa}} {di} {sana?}\\ %
& cake & want & go & give & who & at & \textsc{l.dist}\\
\lspbottomrule
\end{tabular}
\ea
\glt 
‘as for the cake, \bluebold{to whom} do (you) want to go and give (it) over there?’ \textstyleExampleSource{[080922-001a-CvPh.0670]}
\z

\begin{tabular}{llllllll}
\lsptoprule
\label{bkm:Ref356912639}
\gll {…} {ke} {mana?,} {ke} {kampung?,} {deng} {\bluebold{siapa}?}\\ %
&  & to & which & to & village & with & who\\
\lspbottomrule
\end{tabular}
\ea
\glt 
[Talking to her young son:] ‘[do you want to leave today?,] where to?, to the village?, with \bluebold{whom}?’ \textstyleExampleSource{[080917-003a-CvEx.0048-0044]}
\z

\begin{tabular}{llllllll}
\lsptoprule
\label{bkm:Ref356978841}
\gll {baru} {nanti} {minggu} {keduanya} {sembayang} {di} {\bluebold{siapa}?}\\ %
& and.then & very.soon & week & second:\textsc{3possr} & worship & at & who\\
\lspbottomrule
\end{tabular}
\ea
\glt 
‘then later in the second week (of this month), (we’ll) worship at \bluebold{whose} (place)?’ (Lit. ‘at \bluebold{who}’) \textstyleExampleSource{[081011-005-Cv.0037]}
\z

\begin{tabular}{llllll}
\lsptoprule
\label{bkm:Ref356912640}
\gll {\bluebold{siapa}} {\bluebold{pu}} {mata} {yang} {buta?}\\ %
& who & \textsc{poss} & eye & \textsc{rel} & be.blind\\
\lspbottomrule
\end{tabular}
\ea
\glt 
‘\bluebold{whose} eyes are blind?’ \textstyleExampleSource{[080922-001a-CvPh.0142]}
\z


As a nominal modifier, \textitbf{siapa} ‘who’ takes the position of a modifying noun which it replaces. This illustrated with the interrogative clauses (0) and (0).


\begin{styleExampleTitle}
Adnominal uses of \textitbf{siapa} ‘who’
\end{styleExampleTitle}

\begin{tabular}{lllllll}
\lsptoprule
\label{bkm:Ref356911726}
\gll {[\bluebold{prempuang}} {\bluebold{siapa}]} {biking} {sa} {jadi} {bingung?}\\ %
& woman & who & make & \textsc{1sg} & become & be.confused\\
\lspbottomrule
\end{tabular}
\ea
\glt 
‘\bluebold{which woman} made me become confused?’ \textstyleExampleSource{[080922-004-Cv.0028]}
\z

\begin{tabular}{llllllll}
\lsptoprule
\label{bkm:Ref361827990}
\gll {skarang} {sa} {tanya,} {[\bluebold{orang}} {\bluebold{siapa}]} {yang} {benar?}\\ %
& now & \textsc{1sg} & ask & person & who & \textsc{rel} & be.true\\
\lspbottomrule
\end{tabular}
\ea
\glt 
‘now I asked, ‘\bluebold{which person} (is the one) who is right?’ \textstyleExampleSource{[080917-010-CvEx.0197]}
\z


In its predicative uses, \textitbf{siapa} ‘who’ occurs in equative nominal predicate clauses where it questions the identity of the clausal subject, as shown in (0) and (0) (for more details on nominal clauses see §12.2). The interrogative can remain in-situ, in the clause-final position, following the subject as in (0), or it can be fronted to the marked clause-initial position, preceding the subject, as in (0). In the corpus, the token frequencies for both positions are about the same. When speakers want to accentuate the subject, such as \textitbf{ini} ‘\textsc{d.prox}’ in (0), the interrogative remains in-situ, where it is less prominent. When, by contrast, speakers want to stress the questioning of the subject’s identity, they front \textitbf{siapa} ‘who’ to the clause-initial position, as in (0). Besides their different functions, the contrastive examples in (0) and (0) also have distinct intonation contours. When \textitbf{siapa} ‘who’ remains in-situ, as in (0), the clause has a rising intonation, typical of interrogatives. When it is fronted, the clause has a falling intonation, typical of declaratives. In both cases, \textitbf{siapa} ‘who’ is marked with a slight increase in pitch of its stressed penultimate syllable (“~\'{~}~”).


\begin{styleExampleTitle}
Predicative uses of \textitbf{siapa} ‘who’
\end{styleExampleTitle}

\begin{tabular}{lllll}
\lsptoprule
\label{bkm:Ref356912337}
\gll {\textstyleChBold{{}---{}---}} {\textstyleChBold{\textsuperscript{{}---}}\textstyleChBold{\textsubscript{{}---}}} {\textstyleChBold{{}---{}---}} {\textstyleChBold{\textsuperscript{{}---}}\textstyleChBold{\textsubscript{{}---}}}\\ %
& ini & \bluebold{siápa}? & ini & \bluebold{siápa}?\\
& \textsc{d.prox} & what & \textsc{d.prox} & what\\
\lspbottomrule
\end{tabular}
\ea
\glt 
‘\bluebold{who} is this? \bluebold{who} is this?’ \textstyleExampleSource{[080916-001-CvNP.0006]}
\z

\begin{tabular}{lll}
\lsptoprule
\label{bkm:Ref356912336}
\gll {\textstyleChBold{{}---{}---}} {\textstyleChBold{{}---}\textbf{\textsubscript{{\textbackslash}}}}\\ %
& \bluebold{siápa} & ini?\\
& what & \textsc{d.prox}\\
\lspbottomrule
\end{tabular}
\ea
\glt 
‘\bluebold{who} is this?’ \textstyleExampleSource{[081011-023-Cv.0104]}
\z


In addition, \textitbf{siapa} ‘who’ functions as a placeholder when speakers do not recall a referent’s name, as in (0).


\begin{styleExampleTitle}
Placeholder uses of \textitbf{siapa} ‘who’
\end{styleExampleTitle}

\begin{tabular}{lllll}
\lsptoprule
\label{bkm:Ref362537382}
\gll {Sarles} {antar} {\bluebold{siapa},} {Bolikarfus}\\ %
& Sarles & bring & who & Bolikarfus\\
\lspbottomrule
\end{tabular}
\ea
\glt 
‘Sarles gave a ride to, \bluebold{who-is-it}, Bolikarfus’ \textstyleExampleSource{[081002-001-CvNP.0031]}
\z


Rather commonly, speakers also employ the interrogative in one-word utterances, when they question the identity of a referent, in the sense of ‘who (do you mean)?’
\end{styleBodyxvafter}

\subsection{\textitbf{apa} ‘what’}
\label{bkm:Ref358363770}
The interrogative \textitbf{apa} ‘what’ questions the identity of non-human referents, namely entities and events; in addition, it can question reason. The pronominal uses of \textitbf{apa} ‘what’ are illustrated in (0) to (0), its adnominal uses in (0) and (0), and its predicative uses in (0) and (0). The interrogative is also used as a placeholder as shown in (0). Besides, \textitbf{apa} ‘what’ is also used in one-word utterances.



In its pronominal uses, \textitbf{apa} ‘what’ occurs in all syntactic positions, as demonstrated in (0) to (0), always remaining in-situ. In the elicited verbal clause in (0), \textitbf{apa} ‘what’ takes the subject slot. While this construction is grammatically correct and acceptable, verbal clauses with \textitbf{apa} ‘what’ in the subject slot are unattested in the corpus. Instead, speakers always use equative clauses when they want to question the identity of the clausal subject, similar to the questions formed with \textitbf{siapa} ‘who’ in (0) and (0) (§5.8.1). This is demonstrated with the contrastive equative clause in (0). In this example, \textitbf{apa} ‘what’ takes the subject slot, while the headless relative clause \textitbf{yang su gigit …} ‘(the one) who has already bitten …’ takes the predicate slot.\footnote{\\
\\
\\
\\
\\
\\
\\
\\
\\
\\
\\
\\
\par Similar to the comment in Footnote 168 in §5.8.1 (p. \pageref{bkm:Ref436404513}), one could argue that in (0) the subject-predicate word order is inverted and that \textitbf{apa} ‘what’ fills the predicate rather than the subject slot, while the relative clause introduced with \textitbf{yang} ‘\textsc{rel}’ takes the subject slot.\\
\\
\\
} (For details on relative clauses see §14.3.2.)
\end{styleBodyvvafter}

\begin{styleExampleTitle}
Pronominal uses of \textitbf{apa} ‘what’: Subject slot
\end{styleExampleTitle}

\begin{tabular}{lllllll}
\lsptoprule
\label{bkm:Ref436404564}\label{bkm:Ref356985743}
\gll {\bluebold{apa}} {su} {gigit} {sa} {pu} {lutut}\\ %
& what & already & bite & \textsc{1sg} & \textsc{poss} & knee\\
\lspbottomrule
\end{tabular}
\ea
\glt 
‘\bluebold{what} has bitten my knee?’ \textstyleExampleSource{[Elicited MY131112.005]}
\z

\begin{tabular}{llllllll}
\lsptoprule
\label{bkm:Ref371594185}
\gll {\bluebold{apa}} {yang} {su} {gigit} {sa} {pu} {lutut?}\\ %
& what & \textsc{rel} & already & bite & \textsc{1sg} & \textsc{poss} & knee\\
\lspbottomrule
\end{tabular}
\ea
\glt 
‘\bluebold{what} (is it) that has bitten my knee?’ \textstyleExampleSource{[080916-001-CvNP.0004]}
\z


Interrogative \textitbf{apa} ‘what’ also takes non-subject slots. In (0), \textitbf{apa} ‘what’ takes the direct object slot, in (0) the oblique object slot, and in (0) the peripheral adjunct slot.\footnote{\\
\\
\\
\\
\\
\\
\\
\\
\\
\\
\\
\\
\par Alternatively, one could argue that in (0) \textitbf{apa} ‘what’ does not take an oblique object but an adjunct slot.\\
\\
\\
} Besides, speakers use \textitbf{apa} ‘what’ to question reasons or motives, as in (0).


\begin{styleExampleTitle}
Pronominal uses of \textitbf{apa} ‘what’: Non-subject slots
\end{styleExampleTitle}

\begin{tabular}{llll}
\lsptoprule
\label{bkm:Ref356985744}
\gll {kam} {cari} {\bluebold{apa}?}\\ %
& \textsc{2pl} & search & what\\
\lspbottomrule
\end{tabular}
\ea
\glt 
‘\bluebold{what} are you looking for?’ \textstyleExampleSource{[080917-006-CvHt.0001]}
\z

\begin{tabular}{lllllll}
\lsptoprule
\label{bkm:Ref356985745}
\gll {tokok} {sagu} {tu} {deng} {\bluebold{apa}} {ini?}\\ %
& tap & sago & \textsc{d.dist} & with & what & \textsc{d.prox}\\
\lspbottomrule
\end{tabular}
\ea
\glt 
‘\bluebold{what} are you pounding that sagu with?’ \textstyleExampleSource{[081014-006-CvPr.0014]}
\z

\begin{tabular}{llllllll}
\lsptoprule
\label{bkm:Ref356985746}
\gll {sa} {tra} {taw} {tugu} {ini} {dari} {\bluebold{apa}?}\\ %
& \textsc{1sg} & \textsc{neg} & know & monument & \textsc{d.prox} & from & what\\
\lspbottomrule
\end{tabular}
\ea
\glt 
‘I don’t’ know from \bluebold{where} this monument comes?’ (Lit. ‘from \bluebold{what} (place)’) \textstyleExampleSource{[080917-008-NP.0003]}
\z

\begin{tabular}{llllll}
\lsptoprule
\label{bkm:Ref356985747}
\gll {de} {bilang,} {ko} {tidor} {\bluebold{apa}?}\\ %
& \textsc{3sg} & say & \textsc{2sg} & sleep & what\\
\lspbottomrule
\end{tabular}
\ea
\glt 
‘he said, ‘\bluebold{why} are you sleeping?’’ \textstyleExampleSource{[081006-034-CvEx.0022]}
\z


In its adnominal uses, \textitbf{apa} ‘what’ takes the position of a nominal modifier which it replaces, such as the name of a weekday in (0), or the name of a clan in (0).


\begin{styleExampleTitle}
Adnominal uses of \textitbf{apa} ‘what’
\end{styleExampleTitle}

\begin{tabular}{lllllll}
\lsptoprule
\label{bkm:Ref356985748}
\gll {[\bluebold{hari}} {\bluebold{apa}]} {baru} {sa} {minta} {ijing?}\\ %
& day & what & and.then & \textsc{1sg} & request & permission\\
\lspbottomrule
\end{tabular}
\ea
\glt 
[A school boy asking his mother:] ‘on \bluebold{which day} will I ask for permission (to be absent)?’ \textstyleExampleSource{[080917-003a-CvEx.0003]}
\z

\begin{tabular}{lllllll}
\lsptoprule
\label{bkm:Ref356985749}
\gll {dia} {tanya} {saya,} {ko} {[\bluebold{marga}} {\bluebold{apa}]?}\\ %
& \textsc{3sg} & ask & \textsc{1sg} & \textsc{2sg} & clan & what\\
\lspbottomrule
\end{tabular}
\ea
\glt 
‘he asked me, ‘\bluebold{which clan} do you (belong to)?’’ (Lit. ‘you are \bluebold{what clan}’) \textstyleExampleSource{[080922-010a-CvNF.0281]}
\z


In its predicative uses, \textitbf{apa} ‘what’ questions the identity or pertinent characteristics of the clausal subject (for more details on nominal predicate clauses see §12.2). Like \textitbf{siapa} ‘who’ (§5.8.1), \textitbf{apa} ‘what’ can remain in the unmarked clause-final position, as in (0). Alternatively, it can be fronted to the marked clause-initial position, as in (0), where it stresses the questioning of the subject’s identity or characteristics. The contrastive clauses in (0) and (0) have the same distinct intonation contours as the corresponding questions with \textitbf{siapa} ‘who’ in (0) and (0) (§5.8.1). Clauses with in-situ \textitbf{apa} ‘what’, as in (0), have the typical rising interrogative intonation. Clauses with fronted \textitbf{apa} ‘what’ have the typical falling declarative intonation. Like \textitbf{siapa} ‘who’, \textitbf{apa} ‘what’ is marked with a slight increase in pitch of its stressed penultimate syllable (“~\'{~}~”).


\begin{styleExampleTitle}
Predicative uses of \textitbf{apa} ‘what’
\end{styleExampleTitle}

\begin{tabular}{lll}
\lsptoprule
\label{bkm:Ref356985750}
\gll {\textstyleChBold{{}---{}---}} {\textstyleChBold{\textsuperscript{{}---}}\textstyleChBold{\textsubscript{{}---}}}\\ %
& ini & \bluebold{ápa}?\\
& \textsc{d.prox} & what\\
\lspbottomrule
\end{tabular}
\ea
\glt 
‘\bluebold{what} is this?’ \textstyleExampleSource{[081109-001-Cv.0012]}
\z

\begin{tabular}{llll}
\lsptoprule
\label{bkm:Ref356985752}
\gll {\textstyleChBold{{}---}\textbf{\textsubscript{{\textbackslash}}}} {\textstyleChBold{{}---{}---}} {\textstyleChBold{{}---}\textbf{\textsubscript{{\textbackslash}}}}\\ %
& adu, & \bluebold{ápa} & ini?\\
& oh.no! & what & \textsc{d.prox}\\
\lspbottomrule
\end{tabular}
\ea
\glt 
‘oh no, \bluebold{what} is this?’ \textstyleExampleSource{[081109-001-Cv.0012]}
\z


In addition, \textitbf{apa} ‘what’ functions as a placeholder, when speakers do not recall the name of a lexical item, as in (0).


\begin{styleExampleTitle}
Placeholder uses of \textitbf{apa} ‘what’
\end{styleExampleTitle}

\begin{tabular}{lllllll}
\lsptoprule
\label{bkm:Ref356985755}
\gll {de} {bisa} {bantu} {deng} {\bluebold{apa},} {ijasa}\\ %
& \textsc{3sg} & be.able & help & with & what & diploma\\
\lspbottomrule
\end{tabular}
\ea
\glt 
‘he can help (us) with, \bluebold{what-is-it}, the diploma’ \textstyleExampleSource{[081011-023-Cv.0107]}
\z


Speakers also employ \textitbf{apa} ‘what’ in one-word utterances to question the overall situation in the sense of ‘what (is wrong)?’, or to signal lack of understanding, in the sense of ‘what?’, for example during phone conversations with a bad connection.
\end{styleBodyxvafter}

\subsection{\textitbf{mana} ‘where, which’}
\label{bkm:Ref358363771}
The interrogative \textitbf{mana} ‘where, which’ questions locations and single items. Its pronominal uses are illustrated in (0) and (0), its adnominal uses in (0) and (0), and its predicative uses in (0) to (0). The interrogative is also used in one-word utterances, as shown in (0).



In its pronominal uses as the head of a noun phrase, \textitbf{mana} ‘where, which’ questions locations, as in (0) and (0). More specifically, it functions as the complement in a prepositional phrase which is headed by a preposition encoding location (details on prepositional phrases are provided in Chapter 10).
\end{styleBodyvvafter}

\begin{styleExampleTitle}
Pronominal uses of \textitbf{mana} ‘where, which’
\end{styleExampleTitle}

\begin{tabular}{lllll}
\lsptoprule
\label{bkm:Ref356841023}
\gll {ko} {tinggal} {\bluebold{di}} {\bluebold{mana}?}\\ %
& \textsc{2sg} & stay & at & where\\
\lspbottomrule
\end{tabular}
\ea
\glt 
‘\bluebold{where} do you live?’ \textstyleExampleSource{[080922-010a-CvNF.0237]}
\z

\begin{tabular}{lllll}
\lsptoprule
\label{bkm:Ref371600825}
\gll {ko} {datang} {\bluebold{dari}} {\bluebold{mana}?}\\ %
& \textsc{2sg} & come & from & where\\
\lspbottomrule
\end{tabular}
\ea
\glt 
‘\bluebold{from where} do you come?’ \textstyleExampleSource{[080922-010a-CvNF.0236]}
\z


In its adnominal uses, \textitbf{mana} ‘where, which’ questions single entities among larger numbers of identical or similar entities expressed by its referents, as in (0) and (0).


\begin{styleExampleTitle}
Adnominal uses of \textitbf{mana} ‘where, which’
\end{styleExampleTitle}

\begin{tabular}{lllllllllllllll}
\lsptoprule
\label{bkm:Ref356841026}
\gll {kalo} {\multicolumn{2}{l}{[\bluebold{ana}}} {\multicolumn{2}{l}{\bluebold{mana}]}} {\multicolumn{2}{l}{yang}} {\multicolumn{2}{l}{sa}} {\multicolumn{2}{l}{duduk}} {ceritra} {deng} {dia,}\\ %
& if & \multicolumn{2}{l}{child} & \multicolumn{2}{l}{where} & \multicolumn{2}{l}{\textsc{rel}} & \multicolumn{2}{l}{\textsc{1sg}} & \multicolumn{2}{l}{sit} & tell & with & \textsc{3sg}\\
& \multicolumn{2}{l}{itu} & \multicolumn{2}{l}{ana} & \multicolumn{2}{l}{itu,} & \multicolumn{2}{l}{de} & \multicolumn{2}{l}{hormat} & \multicolumn{4}{l}{torang}\\
& \multicolumn{2}{l}{\textsc{d.dist}} & \multicolumn{2}{l}{child} & \multicolumn{2}{l}{\textsc{d.dist}} & \multicolumn{2}{l}{\textsc{3sg}} & \multicolumn{2}{l}{respect} & \multicolumn{4}{l}{\textsc{1pl}}\\
\lspbottomrule
\end{tabular}
\ea
\glt 
[Conversation about a certain teenager:] ‘as for \bluebold{which kid} with whom I sit and talk with, that is that kid, she respects us’ \textstyleExampleSource{[081115-001a-Cv.0282]}
\z

\begin{tabular}{lllllll}
\lsptoprule
\label{bkm:Ref371598794}
\gll {dong} {bilang,} {[\bluebold{badang}} {\bluebold{mana}]} {yang} {sakit?}\\ %
& \textsc{3pl} & say & body & where & \textsc{rel} & be.sick\\
\lspbottomrule
\end{tabular}
\ea
\glt 
‘they said, \bluebold{which (part of your) body} (is the one) that is hurting?’’ \textstyleExampleSource{[081015-005-NP.0031]}
\z


In its predicative uses, \textitbf{mana} ‘where, which’ occurs in prepositional predicates which question the subject’s location (for details on prepositional predicates see §12.4). Like predicate clauses with \textitbf{siapa} ‘who’ (§5.8.1) and \textitbf{apa} ‘what’ (§5.8.2), prepositional predicates with \textitbf{mana} ‘where, which’ can take two positions. They can remain in-situ, following the clausal subject, as in (0). Alternatively, they can be fronted to the marked clause-initial position, where they stress the questioning of the subject’s location, as in the elicited example in (0). In the corpus, though, the preposition is always omitted from fronted prepositional predicates, as in (0).


\begin{styleExampleTitle}
Predicative uses of \textitbf{mana} ‘where, which’
\end{styleExampleTitle}

\begin{tabular}{lllll}
\lsptoprule
\label{bkm:Ref356837974}
\gll {sabung} {mandi} {di} {\bluebold{mana}?}\\ %
& soap & bathe & at & where\\
\lspbottomrule
\end{tabular}
\ea
\glt 
‘\bluebold{where} is (our) soap?’ \textstyleExampleSource{[081025-006-Cv.0026]}
\z

\begin{tabular}{lllll}
\lsptoprule
\label{bkm:Ref371599772}
\gll {di} {\bluebold{mana}} {sabung} {mandi?}\\ %
& at & where & soap & bathe\\
\lspbottomrule
\end{tabular}
\ea
\glt 
‘\bluebold{where} is (our) soap?’ \textstyleExampleSource{[Elicited MY131112.006]}
\z

\begin{tabular}{lllllll}
\lsptoprule
\label{bkm:Ref356837972}
\gll {Nofi,} {${\varnothing}$} {\bluebold{mana}} {kitong} {pu} {ikang{\Tilde}ikang?}\\ %
& Nofi &  & where & \textsc{1pl} & \textsc{poss} & \textsc{rdp}{\Tilde}fish\\
\lspbottomrule
\end{tabular}
\ea
\glt 
‘Nofi, \bluebold{where} are our fish?’ \textstyleExampleSource{[080917-006-CvHt.0002]}
\z


Quite commonly, the interrogative is used to form one-word utterances in which case it questions an entire proposition, as in (0).


\begin{styleExampleTitle}
One-word utterances with \textitbf{mana} ‘where, which’
\end{styleExampleTitle}

\begin{tabular}{llll}
\lsptoprule
\label{bkm:Ref356837305}
\gll {Speaker-2:} {di} {\bluebold{mana}?}\\ %
&  & at & where\\
\lspbottomrule
\end{tabular}
\ea
\glt
[Speaker-1: ‘(I used to) stay with my aunt Marta’]\\
Speaker-2: ‘\bluebold{where}?’ \textstyleExampleSource{[080922-002-Cv.0029-0030]}
\end{styleFreeTranslEngxvpt}

\subsection{\textitbf{bagemana} ‘how’}
\label{bkm:Ref358363773}
The interrogative \textitbf{bagemana} ‘how’ questions manner or circumstance in the sense of ‘how, what (is it) like’. The interrogative has pronominal uses, as illustrated in (0) and (0), and predicative uses, as shown in (0) to (0). In addition, \textitbf{bagemana} ‘how’ has placeholder uses as in (0). It also occurs in one-word utterances, as in (0) and (0).



In its pronominal uses, \textitbf{bagemana} ‘how’ can remain in-situ, in the unmarked clause-final position, or can occur in the marked clause-initial position. In the clause-final position, the interrogative questions the specific manner of an event or activity such as the best way of transporting a pig in (0). In the clause-initial position, the scope of \textitbf{bagemana} ‘how’ is larger. Here it questions an entire proposition, as in (0) and (0), and not only a specific manner, as in (0). The example in (0) also shows that, depending on the context, fronted \textitbf{bagemana} ‘how’ also question reasons.
\end{styleBodyvvafter}

\begin{styleExampleTitle}
Pronominal uses of \textitbf{bagemana} ‘how’
\end{styleExampleTitle}

\begin{tabular}{lllllllll}
\lsptoprule
\label{bkm:Ref356901224}
\gll {…} {adu,} {babi} {ni} {sa} {harus} {angkat} {\bluebold{bagemana}?}\\ %
&  & oh.no! & pig & \textsc{d.prox} & \textsc{1sg} & have.to & lift & how\\
\lspbottomrule
\end{tabular}
\ea
\glt 
‘[the pig was very big, I alone could not transport it, I thought,] ‘oh no!, this pig, \bluebold{how} am I going to transport it?’’ \textstyleExampleSource{[080919-003-NP.0008]}
\z

\begin{tabular}{llllll}
\lsptoprule
\label{bkm:Ref362686892}
\gll {\bluebold{bagemana}} {kitong} {mo} {dapat} {uang?}\\ %
& how & \textsc{1pl} & want & get & money\\
\lspbottomrule
\end{tabular}
\ea
\glt 
‘\bluebold{how} are we going to get money?’ \textstyleExampleSource{[080927-006-CvNP.0041]}
\z

\begin{tabular}{llllllllll}
\lsptoprule
\label{bkm:Ref356901223}
\gll {de} {tanya} {juga,} {\bluebold{bagemana}} {ko} {bisa} {kasi} {ana} {ini?}\\ %
& \textsc{3sg} & ask & also & how & \textsc{2sg} & be.able & give & child & \textsc{d.prox}\\
\lspbottomrule
\end{tabular}
\ea
\glt 
[About bride-price children:] ‘she also asked (me), ‘\bluebold{how} can you give this child (of yours away)?’’ \textstyleExampleSource{[081006-026-CvEx.0003]}
\z


When used predicatively, \textitbf{bagemana} ‘how’ can remain in-situ, as in (0) and (0), or can be fronted, as in (0) and (0). Similar to the predicative uses of the interrogatives discussed in the previous sections, the clause-final in-situ position is the unmarked one where the interrogative is less prominent in comparison to the clause-initial subject, as shown in (0). When placed in the marked clause-initial position, by contrast, \textitbf{bagemana} ‘how’ accentuates the questioning of the subject’s circumstance, as in (0). In addition, predicatively used \textitbf{bagemana} ‘how’ inquires about the well-being of one’s interlocutor(s) as in (0) and (0).


\begin{styleExampleTitle}
Predicative uses of \textitbf{bagemana} ‘how’
\end{styleExampleTitle}

\begin{tabular}{llllllllll}
\lsptoprule
\label{bkm:Ref356901296}
\gll {dong} {tida} {taw} {itu,} {Yesus} {itu,} {injil} {itu} {\bluebold{bagemana}?}\\ %
& \textsc{3pl} & \textsc{neg} & know & \textsc{d.dist} & Jesus & \textsc{d.dist} & Gospel & \textsc{d.dist} & how\\
\lspbottomrule
\end{tabular}
\ea
\glt 
‘they don’t know, what’s-his-name, Jesus, (they don’t know) \bluebold{what} the Gospel (is like)’ (Lit. ‘the gospel is \bluebold{how}?’) \textstyleExampleSource{[081006-023-CvEx.0005]}
\z

\begin{tabular}{lllllllll}
\lsptoprule
\label{bkm:Ref356901295}
\gll {…} {susa} {liat} {setang} {itu,} {\bluebold{bagemana}} {rupa} {setang}\\ %
&  & be.difficult & see & evil.spirit & \textsc{d.dist} & how & form & evil.spirit\\
\lspbottomrule
\end{tabular}
\ea
\glt 
[About evil spirits:] ‘[but for us who … already believe in Jesus, we can’t,] (for us) it is difficult to see that evil spirit, \bluebold{what} the evil spirit’s face (is like)’ (Lit. ‘\bluebold{how} (is) the evil spirit’s form?’) \textstyleExampleSource{[081006-022-CvEx.0069]}
\z

\begin{tabular}{lllll}
\lsptoprule
\label{bkm:Ref362691642}
\gll {yo,} {ko} {Herman} {\bluebold{bagemana}?}\\ %
& yes & \textsc{2sg} & Herman & how\\
\lspbottomrule
\end{tabular}
\ea
\glt 
[Greeting a visitor:] ‘yes, \bluebold{how} are you, Herman?’ \textstyleExampleSource{[081014-011-CvEx.0072]}
\z

\begin{tabular}{lllllll}
\lsptoprule
\label{bkm:Ref356901227}
\gll {eh,} {\bluebold{bagemana}} {ipar?,} {sore,} {dari} {Jayapura?}\\ %
& hey! & how & sibling-in-law & afternoon & from & Jayapura\\
\lspbottomrule
\end{tabular}
\ea
\glt 
[Greeting a visitor:] ‘hey, \bluebold{how} (is it going) brother-in-law?, (good) afternoon! (did you just get here) from Jayapura?’ \textstyleExampleSource{[081110-002-Cv.0003]}
\z


Another use of \textitbf{bagemana} ‘how’ is that of a placeholder, as shown in (0).


\begin{styleExampleTitle}
Placeholder uses of \textitbf{bagemana} ‘how’
\end{styleExampleTitle}

\begin{tabular}{lllllllllll}
\lsptoprule
\label{bkm:Ref356901230}
\gll {…} {sa} {macang,} {sa} {macang} {\bluebold{bagemana},} {e,} {rasa} {sa} {…}\\ %
&  & \textsc{1sg} & variety & \textsc{1sg} & variety & how & uh & feel & \textsc{1sg} & \\
\lspbottomrule
\end{tabular}
\ea
\glt 
‘[so when I (went) to Biak there, I felt very strange] I kind of, I kind of, \bluebold{what-is-it}, uh, felt (that) I …’ \textstyleExampleSource{[081011-013-Cv.0009]}
\z


In one-word utterances, \textitbf{bagemana} ‘how’ questions the circumstances of an event or state, as in (0), or signals lack of understanding as in (0).


\begin{styleExampleTitle}
One-word utterances with \textitbf{bagemana} ‘how’
\end{styleExampleTitle}

\begin{tabular}{lllllll}
\lsptoprule
\label{bkm:Ref356901226}
\gll {saya} {tanya} {saya} {punya} {bapa,} {\bluebold{bagemana}?}\\ %
& \textsc{1sg} & ask & \textsc{1sg} & \textsc{poss} & father & how\\
\lspbottomrule
\end{tabular}
\ea
\glt 
‘I asked my father, ‘\bluebold{how} (did this happen)?’’ \textstyleExampleSource{[080921-011-Cv.0012]}
\z

\begin{tabular}{lll}
\lsptoprule
\label{bkm:Ref356901228}
\gll {\bluebold{bagemana}?} {\bluebold{bagemana}?}\\ %
& how & how\\
\lspbottomrule
\end{tabular}
\ea
\glt
[During a phone conversation with a bad connection:] ‘\bluebold{what}?, \bluebold{what}?’ \textstyleExampleSource{[080922-001b-CvPh.0027]}
\end{styleFreeTranslEngxvpt}

\subsection{\textitbf{kapang} ‘when’}
\label{bkm:Ref358363775}
The interrogative \textitbf{kapang} ‘when’ questions time. Always used pronominally, \textitbf{kapang} ‘when’ usually occurs in clause-initial position, as shown with its first and third occurrences in (0). Here, \textitbf{kapang} ‘when’ questions the temporal setting of the events or states expressed by the entire clause. When the temporal setting is less important, \textitbf{kapang} ‘when’ occurs in clause-final position, as shown with the second \textitbf{kapang} ‘when’ token in (0). Hence, the different positions of \textitbf{kapang} ‘when’ within the clause have functions which parallel those of the time-denoting nouns which the interrogative replaces (see §5.2.5). Alternatively, but rarely, the interrogative occurs between the subject and the predicate, as in (0). According to one consultant, this position of \textitbf{kapang} ‘when’ is acceptable, although the semantics conveyed by this position are still ill understood.


\begin{styleExampleTitle}
Pronominal uses of \textitbf{kapang} ‘when’
\end{styleExampleTitle}

\begin{tabular}{lllllllllll}
\lsptoprule
\label{bkm:Ref356813998}
\gll {\bluebold{kapang}} {kita} {\multicolumn{2}{l}{mo}} {\multicolumn{2}{l}{antar?,}} {kitong} {antar} {\bluebold{kapang}?} {…}\\ %
& when & \textsc{1pl} & \multicolumn{2}{l}{want} & \multicolumn{2}{l}{deliver} & \textsc{1pl} & deliver & when & \\
& \bluebold{kapang} & \multicolumn{2}{l}{kitong} & \multicolumn{2}{l}{antar} & \multicolumn{5}{l}{dia?}\\
& when & \multicolumn{2}{l}{\textsc{1pl}} & \multicolumn{2}{l}{bring} & \multicolumn{5}{l}{\textsc{3sg}}\\
\lspbottomrule
\end{tabular}
\ea
\glt 
[Discussing when the bride’s parents will bring their daughter to the groom’s parents:] ‘[they (the bride’s parents) start asking, ‘…,] \bluebold{when} should we bring her? we bring her \bluebold{when}?, … \bluebold{when} do we bring her?’’ \textstyleExampleSource{[081110-005-CvPr.0043-0044]}
\z

\begin{tabular}{lllllllll}
\lsptoprule
\label{bkm:Ref356813999}
\gll {kasiang,} {sa} {\bluebold{kapang}} {mandi} {deng} {dorang} {lagi} {e?}\\ %
& pity & \textsc{1sg} & when & bathe & with & \textsc{3pl} & again & eh\\
\lspbottomrule
\end{tabular}
\ea
\glt
[About a sick boy:] ‘what a pity, \bluebold{when} will I bathe with them (my friends) again, eh?’ \textstyleExampleSource{[081025-009b-Cv.0044]}
\end{styleFreeTranslEngxvpt}

\subsection{\textitbf{knapa} ‘why’}
\label{bkm:Ref358363776}
The interrogative \textitbf{knapa} ‘why’ questions reasons and motives. Its pronominal uses are illustrated in (0) to (0), its predicative uses in (0), and its uses in one-word utterances in (0).



Typically, \textitbf{knapa} ‘why’ is used pronominally. Most often it occurs in clause-initial position, as in (0). In clauses marked with an initial conjunction, \textitbf{knapa} ‘why’ follows the conjunction as in (0). Alternatively, but rarely, \textitbf{knapa} ‘why’ occurs between the subject and the predicate, as in (0). According to one consultant, this position of \textitbf{knapa} ‘why’ is acceptable; the semantics of this position still need to be investigated, though.
\end{styleBodyvvafter}

\begin{styleExampleTitle}
Pronominal uses of \textitbf{knapa} ‘why’
\end{styleExampleTitle}

\begin{tabular}{lllllll}
\lsptoprule
\label{bkm:Ref356811063}
\gll {e,} {\bluebold{knapa}} {kam} {kas{\Tilde}kas} {bangung} {dia?}\\ %
& hey! & why & \textsc{2pl} & \textsc{rdp}{\Tilde}give & wake.up & \textsc{3sg}\\
\lspbottomrule
\end{tabular}
\ea
\glt 
‘hey, \bluebold{why} do you keep waking him up?’ \textstyleExampleSource{[080918-001-CvNP.0039]}
\z

\begin{tabular}{llllll}
\lsptoprule
\label{bkm:Ref356811065}
\gll {tapi} {\bluebold{knapa}} {ana} {ini} {sakit?}\\ %
& but & why & child & \textsc{d.prox} & be.sick\\
\lspbottomrule
\end{tabular}
\ea
\glt 
‘but \bluebold{why} is this child sick’ \textstyleExampleSource{[080917-010-CvEx.0133]}
\z

\begin{tabular}{llllllll}
\lsptoprule
\label{bkm:Ref362694371}
\gll {…} {Matius} {itu} {dia} {\bluebold{knapa}} {maju} {begitu?}\\ %
&  & Matius & \textsc{d.dist} & \textsc{3sg} & why & advance & like.that\\
\lspbottomrule
\end{tabular}
\ea
\glt 
‘[as for Matius, I’m very surprised,] Matius there, \bluebold{how come} he could advance like that?’ \textstyleExampleSource{[081006-032-Cv.0025]}
\z


Interrogative \textitbf{knapa} ‘what’ can also be used predicatively. In this case, \textitbf{knapa} ‘what’ remains in-situ, in the clause-final position, following the subject, as in (0).


\begin{styleExampleTitle}
Predicative uses of \textitbf{knapa} ‘why’
\end{styleExampleTitle}

\begin{tabular}{llll}
\lsptoprule
\label{bkm:Ref356811068}
\gll {bapa} {ko} {\bluebold{knapa}?}\\ %
& father & \textsc{2sg} & why\\
\lspbottomrule
\end{tabular}
\ea
\glt 
[After an accident]: ‘Sir, \bluebold{what happened}?’ (Lit. ‘you father (are) \bluebold{why}?’) \textstyleExampleSource{[081108-001-JR.0005]}
\z


The interrogative can also form one-word utterances in which case it questions an entire proposition, as in (0).


\begin{styleExampleTitle}
One-word utterances with \textitbf{knapa} ‘why’
\end{styleExampleTitle}

\begin{tabular}{llll}
\lsptoprule
\label{bkm:Ref362542723}
\gll {Speaker-2:} {e,} {\bluebold{knapa}?}\\ %
&  & hey! & why\\
\lspbottomrule
\end{tabular}
\ea
\glt
[About the birth of twins] [Speaker-1: ‘… as for the girl, they say it’s an evil spirit, so they kill (her)’]\\
Speaker-2: ‘hey, \bluebold{why}?’ \textstyleExampleSource{[081011-022-Cv.0147-0151]}
\end{styleFreeTranslEngxvpt}

\subsection{Interrogative uses of mid-range quantifier \textitbf{brapa} ‘several’}
\label{bkm:Ref358363772}
In its interrogative uses, the mid-range quantifier \textitbf{brapa} ‘several’ receives the reading ‘how many’. It questions quantities of countable entities and, in combination with the mid-range quantifier \textitbf{banyak} ‘many’, of non-countable entities. Its adnominal uses are shown in (0) to (0), and its predicative uses in (0) and (0).



Most often, \textitbf{brapa} ‘several, how many’ functions as a nominal modifier which takes the position of the numeral or quantifier it replaces. Corresponding to the syntax of adnominally used numerals and quantifiers, it precedes or follows its head nominal, as in (0) to (0). In pre-head position of countable referents, \textitbf{brapa} ‘several, how many’ questions the absolute numbers of items denoted by the head nominals, as in (0). In post-head position of countable referents, it questions unique positions within series, as in (0). When following mass nouns, the interrogative questions the non-numeric amounts of its referents, as in the elicited example in (0). Like other quantifiers, \textitbf{brapa} ‘several, how many’ is unattested in pre-head position of mass nouns. If the referent’s identity is known from the context, the head nominal can be omitted, as with numerals and other quantifiers. This is illustrated in (0), where the omitted head is \textitbf{rupia} ‘rupiah’. (Details on numerals and quantifiers are given in §5.9 and §5.10, respectively.)
\end{styleBodyvvafter}

\begin{styleExampleTitle}
Adnominal uses of \textitbf{brapa} ‘how many’
\end{styleExampleTitle}

\begin{tabular}{llllll}
\lsptoprule
\label{bkm:Ref356824613}
\gll {\bluebold{brapa}} {\bluebold{bulang}} {dorang} {skola} {ka?}\\ %
& several & month & \textsc{3pl} & go.to.school & or\\
\lspbottomrule
\end{tabular}
\ea
\glt 
‘(for) \bluebold{how many months} will they go to school?’ \textstyleExampleSource{[081025-003-Cv.0207]}
\z

\begin{tabular}{lllllll}
\lsptoprule
\label{bkm:Ref356824614}
\gll {jadi} {mama,} {mama} {pulang} {\bluebold{jam}} {\bluebold{brapa}?}\\ %
& so & mother & mother & go.home & hour & several\\
\lspbottomrule
\end{tabular}
\ea
\glt 
‘so mama, \bluebold{what time} will you (‘mother’) come home?’ (Lit. ‘\bluebold{how manyeth hour}’) \textstyleExampleSource{[080924-002-Pr.0002]}
\z

\begin{tabular}{llllll}
\lsptoprule
\label{bkm:Ref356824615}
\gll {ko} {minta} {\bluebold{minyak}} {\bluebold{brapa}} {\bluebold{banyak}?}\\ %
& \textsc{2pl} & request & cooked.rice & several & many\\
\lspbottomrule
\end{tabular}
\ea
\glt 
‘\bluebold{how much oil} do you request’ \textstyleExampleSource{[Elicited BR120520.001]}
\z

\begin{tabular}{lllll}
\lsptoprule
\label{bkm:Ref356820004}
\gll {kemaring} {dapat} {\bluebold{brapa}} {\bluebold{Ø}?}\\ %
& yesterday & get & several & \\
\lspbottomrule
\end{tabular}
\ea
\glt 
[Collecting money for a project:] ‘\bluebold{how many (rupiah)} did (you) get yesterday?’ \textstyleExampleSource{[080925-003-Cv.0090]}
\z


The predicative uses of \textitbf{brapa} ‘several, how many’ are shown in (0) to (0). Like \textitbf{siapa} ‘who’ (§5.8.1), \textitbf{apa} ‘what’ (§5.8.2), and \textitbf{mana} ‘where, which’ (§5.8.3), \textitbf{brapa} ‘several, how many’ can remain in the unmarked clause-final position, as in (0) and (0), or it can be fronted to the marked clause-initial position, as in the elicited example in (0) and (0). Again, the fronting of the interrogative serves to emphasize the questioning, namely of numeric quantities in (0), and of non-numeric quantities in (0). These two examples are elicited, though, as interrogatives with fronted \textitbf{brapa} ‘several, how many’ are unattested in the corpus. (See also §12.3 for details on numeral and quantifier predicate clauses.)


\begin{styleExampleTitle}
Predicative uses of \textitbf{brapa} ‘how many’
\end{styleExampleTitle}

\begin{tabular}{lllll}
\lsptoprule
\label{bkm:Ref356820006}
\gll {bapa} {pu} {ana{\Tilde}ana} {\bluebold{brapa}?}\\ %
& father & \textsc{poss} & \textsc{rdp}{\Tilde}child & several\\
\lspbottomrule
\end{tabular}
\ea
\glt 
‘\bluebold{how many} children do you (‘father’) have?’ (Lit. ‘father’s children are \bluebold{how many}?’) \textstyleExampleSource{[080923-009-Cv.0010]}
\z

\begin{tabular}{lllll}
\lsptoprule
\label{bkm:Ref371605901}
\gll {\bluebold{brapa}} {bapa} {pu} {ana{\Tilde}ana?}\\ %
& several & father & \textsc{poss} & \textsc{rdp}{\Tilde}child\\
\lspbottomrule
\end{tabular}
\ea
\glt 
‘\bluebold{how many} children do you (‘father’) have?’ \textstyleExampleSource{[Elicited MY131112.007]}
\z

\begin{tabular}{lllll}
\lsptoprule
\label{bkm:Ref356820007}
\gll {tong} {pu} {uang} {\bluebold{brapa}?}\\ %
& \textsc{1pl} & \textsc{poss} & money & several\\
\lspbottomrule
\end{tabular}
\ea
\glt 
‘\bluebold{how much} money do we have?’ (Lit. ‘our money is \bluebold{how many}?’) \textstyleExampleSource{[081006-017-Cv.0015]}
\z

\begin{tabular}{lllll}
\lsptoprule
\label{bkm:Ref371605902}
\gll {\bluebold{brapa}} {tong} {pu} {uang?}\\ %
& several & \textsc{1pl} & \textsc{poss} & money\\
\lspbottomrule
\end{tabular}
\ea
\glt
‘\bluebold{how much} money do we have?’ \textstyleExampleSource{[Elicited MY131112.008]}
\end{styleFreeTranslEngxvpt}

\subsection{Interrogatives denoting indefinite referents}
\label{bkm:Ref358363777}
Cross-linguistically, interrogatives may also function as “general indefinites” by referring “to a general population, of unknown size” {\citep[401]{Dixon2010b}}. In this case, the interrogatives translate with ‘whoever’, ‘whatever’, ‘wherever’, etc.



In Papuan Malay, the indefinite reading is achieved by juxtaposing the focus adverb \textitbf{saja} ‘just’ to the interrogative, as illustrated in (0) to (0). In the corpus, this function of the interrogatives is only attested for \textitbf{siapa} ‘who’, \textitbf{apa} ‘what’, and \textitbf{mana} ‘where, which’, as shown in (0) to (0). The elicited respective examples in (0) to (0) illustrate, however, that \textitbf{bagemana} ‘how’, \textitbf{kapang} ‘when’, and \textitbf{knapa} ‘why’ can also have this function.
\end{styleBodyvvafter}

\begin{tabular}{llllllllllll}
\lsptoprule
\label{bkm:Ref356832010}
\gll {kalo} {ko} {liat} {\bluebold{ko}} {\bluebold{pu}} {\bluebold{sodara}} {\bluebold{siapa}} {\bluebold{saja},} {kalo} {dia} {…}\\ %
& if & \textsc{2sg} & see & \textsc{2sg} & \textsc{poss} & sibling & who & just & if & \textsc{3sg} & \\
\lspbottomrule
\end{tabular}
\ea
\glt 
‘when you see \bluebold{your relatives whoever} (they are), when he/she …’ \textstyleExampleSource{[080919-004-NP.0078]}
\z

\begin{tabular}{llllll}
\lsptoprule
(\stepcounter{}{\the}) & bicara & \bluebold{apa} & \bluebold{saja}, & bicara & saja\\
& speak & what & just & speak & just\\
\lspbottomrule
\end{tabular}
\ea
\glt 
‘speak (to me about) \bluebold{whatever}, just speak (to me)’ \textstyleExampleSource{[080922-001a-CvPh.1174]}
\z

\begin{tabular}{lllllllllll}
\lsptoprule
(\stepcounter{}{\the}) & di & \bluebold{mana} & \bluebold{saja} & bapa & bisa & tinggal, & di & \bluebold{tempat} & \bluebold{mana} & \bluebold{saja}\\
& at & where & just & father & be.able & stay & at & place & where & just\\
\lspbottomrule
\end{tabular}
\ea
\glt 
‘I (‘father’) can live \bluebold{wherever}, (I can) live in \bluebold{whatever place}’ \textstyleExampleSource{[080922-001a-CvPh.1116]}
\z

\begin{tabular}{lllllllllll}
\lsptoprule
\label{bkm:Ref371607237}
\gll {sa} {tra} {mo} {taw!,} {\bluebold{bagemana}} {\bluebold{saja}} {ko} {harus} {pigi} {skola!}\\ %
& \textsc{1sg} & \textsc{neg} & want & know & how & just & \textsc{2sg} & have.to & go & school\\
\lspbottomrule
\end{tabular}
\ea
\glt 
[Addressing a child who does not want to go to school for various reasons:] ‘I don’t want to know!, you have to go to school, \bluebold{no matter what}!’ (Lit. ‘\bluebold{however}’) \textstyleExampleSource{[Elicited MY131112.001]}
\z

\begin{tabular}{llllll}
\lsptoprule
(\stepcounter{}{\the}) & \bluebold{kapang} & \bluebold{saja} & ko & bisa & datang\\
& when & just & \textsc{2sg} & be.able & come\\
\lspbottomrule
\end{tabular}
\ea
\glt 
‘you can come \bluebold{whenever}’ \textstyleExampleSource{[Elicited MY131112.002]}
\z

\begin{tabular}{llllllllll}
\lsptoprule
\label{bkm:Ref371607239}
\gll {\bluebold{knapa}} {\bluebold{saja}} {sa} {pu} {kaka} {de} {mo} {pulang} {Jayapura}\\ %
& why & just & \textsc{1sg} & \textsc{poss} & oSb & \textsc{3sg} & want & go.home & Jayapura\\
\lspbottomrule
\end{tabular}
\ea
\glt
‘my older sibling wants to return to Jayapura, \bluebold{for whatever reason}’ (Lit. ‘\bluebold{whyever}’) \textstyleExampleSource{[Elicited MY131112.002]}
\end{styleFreeTranslEngxvpt}

\subsection{Summary}

In requesting specific types of information, the Papuan Malay interrogatives have a variety of functions within the clause. All of them have pronominal uses. In addition, five them also have predicative uses; the exception is \textitbf{kapang} ‘when’. Furthermore, three of them also have adnominal and/or placeholder uses.



In their pronominal and adnominal uses, the interrogatives typically remain in-situ. Besides, pronominally used \textitbf{kapang} ‘when’ and \textitbf{knapa} ‘why’ may also occur in a clause-internal position, between the subject and the predicate. This position, however, is rare and the semantics conveyed still need to be investigated. In their predicative uses, four of the interrogatives can take two positions. That is, they can remain in situ, that is, in the clause-final position, or they can be fronted to the clause-initial position. When speakers want to accentuate the subject, the interrogative remains in-situ in the unmarked clause-final position. When, by contrast, speakers want to emphasize the fact that they are requesting specific types of information, such as the identity of the subject or its location, they front the interrogative to the marked clause-initial position where it is more salient. The exceptions is \textitbf{knapa} ‘why’, which is unattested clause-initially.
\end{styleBodyvafter}


The mid-range quantifier \textitbf{brapa} ‘several’ also functions as an interrogative, questioning quantities in the sense of ‘how many’. It has, however, only adnominal and predicative, and no pronominal uses. In its adnominal uses, it remains in-situ, while in its predicative uses it can take two positions: it can remain in-situ or be fronted to the clause-initial position.
\end{styleBodyvxvafter}

\section{Numerals}
\label{bkm:Ref351641490}
As numeric expressions, the Papuan Malay numerals designate countable divisions of their referents. Cardinal numbers are presented in §5.9.1, ordinal numbers in §5.9.2, and distributive numbers in §5.9.3. In §5.9.4 an additional non-enumerating function of the numeral \textitbf{satu} ‘one’ is presented.
\end{styleBodyxvafter}

\subsection{Cardinal numerals}
\label{bkm:Ref350504715}
Papuan Malay has a decimal numeral system. The basic cardinal numerals, along with some examples of how they are combined, are presented in Table  ‎5 .35.


\begin{stylecaption}
\label{bkm:Ref272485973}Table ‎5.\stepcounter{Table}{\theTable}:  Basic Papuan Malay cardinal numerals\footnote{\\
\\
\\
\\
\\
\\
\\
\\
\\
\\
\\
\\
\par The numerals \textitbf{sratus} ‘one hundred’ and \textitbf{sribu} ‘one thousand’ are historically derived by unproductive affixation with the prefix \textitbf{s(e)\-}.\\
\\
\\
}
\end{stylecaption}

\tablehead{
\raggedleft \# & Numbers & \raggedleft \# & \arraybslash Numbers\\
}
\begin{tabular}{llll}
\lsptoprule
\raggedleft 1 & \textitbf{satu} & \raggedleft 100 & \textitbf{sratus}\\
&  &  & one:hundred\\
\raggedleft 2 & \textitbf{dua} & \raggedleft 102 & \textitbf{sratus dua}\\
&  &  & one:hundred two\\
\raggedleft 3 & \textitbf{tiga} & \raggedleft 200 & \textitbf{dua ratus}\\
&  &  & two hundred\\
\raggedleft 4 & \textitbf{empat} & \raggedleft 234 & \textitbf{dua ratus tiga pulu empat}\\
&  &  & two hundred three tens four\\
\raggedleft 5 & \textitbf{lima} & \raggedleft 1.000 & \textitbf{sribu}\\
&  &  & one:thousand\\
\raggedleft 6 & \textitbf{enam} & \raggedleft 1.004 & \textitbf{sribu empat}\\
&  &  & one:thousand four\\
\raggedleft 7 & \textitbf{tuju} & \raggedleft 2.000 & \textitbf{dua ribu}\\
&  &  & two thousand\\
\raggedleft 8 & \textitbf{dlapang} & \raggedleft 2.013 & \textitbf{dua ribu tiga blas}\\
&  &  & two thousand three teens\\
\raggedleft 9 & \textitbf{sembilang} & \raggedleft 10.000 & \textitbf{spulu ribu}\\
&  &  & one:tens thousand\\
\raggedleft 10 & \textitbf{spulu} & \raggedleft 32.000 & \textitbf{tiga pulu dua ribu}\\
& one:tens &  & three tens two thousand\\
\raggedleft 11 & \textitbf{seblas} & \raggedleft 980.000 & \textitbf{sembilang ratus dlapang pulu ribu}\\
& one:teens &  & nine hundreds eight tens thousand\\
\raggedleft 12 & \textitbf{dua blas} & \raggedleft 1.000.000 & \textitbf{satu juta}\\
& two teens &  & one million\\
\raggedleft 20 & \textitbf{dua pulu} & \raggedleft 1.000.000.000 & \textitbf{satu milyar}\\
& two tens &  & one billion\\
\raggedleft 21 & \textitbf{dua pulu satu} & \raggedleft zero & \textitbf{kosong}\\
& two tens one &  & be empty\\
\raggedleft 30 & \textitbf{tiga pulu} &  & \\
& three tens &  & \\
\lspbottomrule
\end{tabular}

As illustrated in Table  ‎5 .35, complex numerals are formed by indicating the number of units of the highest power of ten, followed by the number of units of the next lower power down to the simple units or digits of one to ten. The individual components of complex numbers are combined by juxtaposition. The formulas for forming complex numerals are presented in (0) and (0):


\begin{styleExampleTitle}
Formulas for complex numerals
\end{styleExampleTitle}

\begin{tabular}{ll}
\lsptoprule
\label{bkm:Ref439245169}
\gll {Complex numerals with tens (\textitbf{pulu})}\\ %
& (\textsc{digit} \textitbf{juta}) (\textsc{digit} \textitbf{ribu}) (\textsc{digit} \textitbf{ratus}) (\textsc{digit} \textitbf{pulu}) \textsc{digit}\\
\label{bkm:Ref339713015}
\gll {Complex numerals with teens (\textitbf{blas})}\\ %
& (\textsc{digit} \textitbf{juta}) (\textsc{digit} \textitbf{ribu}) (\textsc{digit} \textitbf{ratus}) \textsc{digit} \textitbf{blas}\\
\lspbottomrule
\end{tabular}

Most often, cardinal numerals are used adnominally to enumerate entities. In this function they may precede or follow their head nominal. With a preposed numeral, the noun phrase signals the absolute number of items denoted by the head nominal, as in \textitbf{lima orang} ‘lima people’ in (0). Thereby the composite nature of countable referents is underlined. Post-head numerals, by contrast, express exhaustivity of definite referents such as \textitbf{pace dua ini} ‘both of these men’ in (0), or denote unique positions within a series. (For details on the adnominal uses of numerals see §8.3.1.)


\begin{styleExampleTitle}
Adnominally used numerals
\end{styleExampleTitle}

\begin{tabular}{lllll}
\lsptoprule
\label{bkm:Ref339637301}
\gll {mungking} {\bluebold{lima}} {\bluebold{orang}} {mati}\\ %
& maybe & five & person & die\\
\lspbottomrule
\end{tabular}
\ea
\glt 
‘about \bluebold{five people} died’ \textstyleExampleSource{[081025-004-Cv.0033]}
\z

\begin{tabular}{llllllll}
\lsptoprule
\label{bkm:Ref339637302}
\gll {\bluebold{pace}} {\bluebold{dua}} {\bluebold{ini}} {dong} {dua} {dari} {pedalamang}\\ %
& man & two & \textsc{d.dist} & \textsc{3pl} & two & from & interior\\
\lspbottomrule
\end{tabular}
\ea
\glt 
‘\bluebold{both these men}, the two of them are from the interior’ \textstyleExampleSource{[081109-010-JR.0001]}
\z


When the identity of the referent was established earlier or can be deduced from the context, the head nominal can be omitted, as in (0).


\begin{styleExampleTitle}
Numerals with omitted head nominal
\end{styleExampleTitle}

\begin{tabular}{lllllllll}
\lsptoprule
\label{bkm:Ref339637303}
\gll {Ika} {biasa} {angkat} {itu} {\bluebold{dlapang}} {\bluebold{pulu}} {\bluebold{sembilang}} {Ø}\\ %
& Ika & be.usual & pick-up & \textsc{d.dist} & eight & ten & nine & \\
\lspbottomrule
\end{tabular}
\ea
\glt 
‘Ika usually lifts, what’s-its-name, \bluebold{eighty-nine} (kilogram)’ \textstyleExampleSource{[081023-003-Cv.0004]}
\z


The examples in (0) and (0) also illustrate that numerals can be used with countable nouns that are animate or inanimate, respectively.



In addition to their adnominal uses, numerals are used predicatively. In (0), for example, the numeral \textitbf{dua blas} ‘twelve’ functions as a predicate that provides information about the numeric quantity of its subject \textitbf{de} ‘\textsc{3sg}’ (‘the moon’). (For details on numeral predicate clauses see §12.3).
\end{styleBodyvvafter}

\begin{styleExampleTitle}
Predicatively used numerals
\end{styleExampleTitle}

\begin{tabular}{llllll}
\lsptoprule
\label{bkm:Ref339637306}
\gll {di} {kalender} {de} {\bluebold{dua}} {\bluebold{blas}}\\ %
& at & calendar & \textsc{3sg} & two & tens\\
\lspbottomrule
\end{tabular}
\ea
\glt 
‘in the calendar\bluebold{ }there are\bluebold{ twelve} (moons)’ (Lit. ‘it (the moon) \bluebold{is twelve}’) \textstyleExampleSource{[081109-007-JR.0002]}
\z


The basic mathematical functions of the cardinal numerals are presented in Table  ‎5 .36.


\begin{stylecaption}
\label{bkm:Ref339713199}Table ‎5.\stepcounter{Table}{\theTable}:  Mathematical functions
\end{stylecaption}

\tablehead{
 Item & Sign & \arraybslash Gloss\\
}
\begin{tabular}{lll}
\lsptoprule
\textitbf{tamba} & + & ‘plus’\\
add &  & \\
\textitbf{kurang} & – & ‘minus’\\
lack &  & \\
\textitbf{kali} & x & ‘times’\\
time &  & \\
\textitbf{bagi} & / & ‘divide’\\
divide &  & \\
\lspbottomrule
\end{tabular}

In natural conversations, however, calculations occur only very rarely. Therefore, the following examples are elicited: the function of addition is presented in (0), subtraction in (0), multiplication in (0), and division in (0).


\begin{styleExampleTitle}
Addition
\end{styleExampleTitle}

\begin{tabular}{llllllllll}
\lsptoprule
\label{bkm:Ref339637308}
\gll {dua} {babi} {\bluebold{tamba}} {tiga} {babi} {sama} {dengang} {lima} {babi}\\ %
& two & pig & add & three & pig & be.same & with & five & pig\\
\lspbottomrule
\end{tabular}
\ea
\glt 
‘two pigs \bluebold{plus} three pigs are five pigs’ \textstyleExampleSource{[Elicited BR120820.001]}
\z

\begin{styleExampleTitle}
Subtraction
\end{styleExampleTitle}

\begin{tabular}{llllllllll}
\lsptoprule
\label{bkm:Ref339637309}
\gll {lima} {babi} {\bluebold{kurang}} {tiga} {babi} {sama} {dengang} {dua} {babi}\\ %
& five & pig & lack & three & pig & be.same & with & two & pig\\
\lspbottomrule
\end{tabular}
\ea
\glt 
‘five pigs \bluebold{minus} three pigs are two pigs’ \textstyleExampleSource{[Elicited BR120820.002]}
\z

\begin{styleExampleTitle}
Multiplication
\end{styleExampleTitle}

\begin{tabular}{llllllllll}
\lsptoprule
\label{bkm:Ref339637310}
\gll {dua} {babi} {\bluebold{kali}} {tiga} {babi} {sama} {dengang} {enam} {babi}\\ %
& two & pig & time & three & pig & be.same & with & six & pig\\
\lspbottomrule
\end{tabular}
\ea
\glt 
‘two pigs \bluebold{times} three pigs are six pigs’ \textstyleExampleSource{[Elicited BR120820.003]}
\z

\begin{styleExampleTitle}
Division
\end{styleExampleTitle}

\begin{tabular}{llllllllll}
\lsptoprule
\label{bkm:Ref339637311}
\gll {enam} {babi} {\bluebold{bagi}} {tiga} {babi} {sama} {dengang} {dua} {babi}\\ %
& six & pig & divide & three & pig & be.same & with & two & pig\\
\lspbottomrule
\end{tabular}
\ea
\glt
‘six pigs \bluebold{divided by} three pigs are two pigs’ \textstyleExampleSource{[Elicited BR120820.004]}
\end{styleFreeTranslEngxvpt}

\subsection{Ordinal numerals}
\label{bkm:Ref350504740}
Papuan Malay employs two strategies to express the notion of ordinal numerals. For kinship terms the concept of ordinal numerals is encoded by a ‘NNum’ noun phrase headed by the noun \textitbf{nomor} ‘number’, as shown in (0) and (0). This noun phrase ‘\textitbf{nomor} Num’ gives the ordinal reading ‘Num-th’ such as \textitbf{yang nomor tiga} ‘third’ in (0) or \textitbf{nomor empat} ‘fourth’ in the elicited example in (0).


\begin{styleExampleTitle}
Inherited strategy
\end{styleExampleTitle}

\begin{tabular}{lllllllllllllll}
\lsptoprule
\label{bkm:Ref339637315}
\gll {saya} {\multicolumn{2}{l}{tida}} {bole} {\multicolumn{2}{l}{kasi}} {\multicolumn{2}{l}{sama}} {\multicolumn{2}{l}{bapa}} {\multicolumn{2}{l}{punya}} {\multicolumn{2}{l}{sodara}}\\ %
& \textsc{1sg} & \multicolumn{2}{l}{\textsc{neg}} & may & \multicolumn{2}{l}{give} & \multicolumn{2}{l}{to} & \multicolumn{2}{l}{father} & \multicolumn{2}{l}{\textsc{poss}} & \multicolumn{2}{l}{sibling}\\
& \multicolumn{2}{l}{ana} & \multicolumn{3}{l}{prempuang} & \multicolumn{2}{l}{yang} & \multicolumn{2}{l}{sa} & \multicolumn{2}{l}{bilang} & \multicolumn{2}{l}{\bluebold{nomor}} & \bluebold{tiga}\\
& \multicolumn{2}{l}{child} & \multicolumn{3}{l}{woman} & \multicolumn{2}{l}{\textsc{rel}} & \multicolumn{2}{l}{\textsc{1sg}} & \multicolumn{2}{l}{say} & \multicolumn{2}{l}{number} & three\\
\lspbottomrule
\end{tabular}
\ea
\glt 
[About bride-price children:] ‘I shouldn’t have given to father’s sibling the daughter that, as I said, was \bluebold{(my) third (child)}’ (Lit. ‘\bluebold{number three}’) \textstyleExampleSource{[081006-024-CvEx.0088]}
\z

\begin{tabular}{llllllllll}
\lsptoprule
\label{bkm:Ref339637316}
\gll {Aleks} {ini} {sa} {pu} {tete} {pu} {ade} {\bluebold{nomor}} {\bluebold{empat}}\\ %
& Aleks & \textsc{d.prox} & \textsc{1sg} & \textsc{poss} & grandfather & \textsc{poss} & ySb & number & four\\
\lspbottomrule
\end{tabular}
\ea
\glt 
‘Aleks here is my grandfather’s \bluebold{fourth} youngest sibling’ (Lit. ‘\bluebold{number four}’) \textstyleExampleSource{[Elicited BR120821.002]}
\z


According to one consultant, the strategy presented in (0) and (0) is the inherited Papuan Malay strategy to express the notion of ordinal numbers. This strategy used to be employed not only for kinship terms but for countable nouns in general. With the increasing influence of Standard Indonesian, however, Papuan Malay speakers have started employing ordinal numbers of Indonesian origins more frequently. Hence, in the corpus the ordinal numbers for countable nouns other than kinship terms are of Standard Indonesian origins, such as \textitbf{kedua} ‘second’ in (0) or \textitbf{ketiga} ‘third’ in the elicited example in (0).\footnote{\\
\\
\\
\\
\\
\\
\\
\\
\\
\\
\\
\\
\par In Standard Indonesian, ordinal numerals, with the exception of \textitbf{pertama} ‘first’, are derived by prefixing \textitbf{ke-} to the cardinal number (for details see {Mintz 1994: 293}).\\
\\
\\
}


\begin{styleExampleTitle}
Borrowed strategy
\end{styleExampleTitle}

\begin{tabular}{lllll}
\lsptoprule
\label{bkm:Ref333249523}
\gll {distrik} {\bluebold{kedua}} {di} {mana}\\ %
& district & second & at & where\\
\lspbottomrule
\end{tabular}
\ea
\glt 
‘where is the \bluebold{second} district?’ \textstyleExampleSource{[081010-001-Cv.0071]}
\z

\begin{tabular}{lllllllll}
\lsptoprule
\label{bkm:Ref333249525}
\gll {ini} {bibit} {nangka} {yang} {\bluebold{ketiga}} {yang} {sa} {bli}\\ %
& \textsc{d.prox} & \textsc{1sg} & \textsc{poss} & \textsc{rel} & third & \textsc{rel} & \textsc{1sg} & buy\\
\lspbottomrule
\end{tabular}
\ea
\glt
‘this is the \bluebold{third} jackfruit seedling that I bought’ \textstyleExampleSource{[Elicited BR120821.003]}
\end{styleFreeTranslEngxvpt}

\subsection{Distributive numerals}
\label{bkm:Ref350504741}
Distributive numerals express that “a property or action” applies “to the individual members of a group, as opposed to the group as a whole” {\citep[154]{Crystal2008}}.In Papuan Malay, this notion of ‘one by one’ or ‘two by two’ is expressed through reduplication of the numeral. This is illustrated with \textitbf{satu{\Tilde}satu} ‘one by one’ or ‘in groups of one each’ in (0), and with \textitbf{dua{\Tilde}dua} ‘two by two’ or ‘in groups of two’ in (0). (See also §4.2.4.)
\end{styleBodyxafter}

\begin{tabular}{llllll}
\lsptoprule
\label{bkm:Ref339637319}
\gll {tong} {tiga} {cari} {jalang} {\bluebold{satu{\Tilde}satu}}\\ %
& \textsc{1pl} & three & search & street & \textsc{rdp}{\Tilde}one\\
\lspbottomrule
\end{tabular}
\ea
\glt 
‘the three of us looked for a path (through the river) \bluebold{one-by-one}’ \textstyleExampleSource{[081013-003-Cv.0003]}
\z

\begin{tabular}{llllllll}
\lsptoprule
\label{bkm:Ref339637320}
\gll {tong} {minum} {\bluebold{dua{\Tilde}dua}} {\bluebold{glas}} {ato} {\bluebold{tiga{\Tilde}tiga}} {\bluebold{glas}}\\ %
& \textsc{1pl} & drink & \textsc{rdp}{\Tilde}two & glass & or & \textsc{rdp}{\Tilde}three & glass\\
\lspbottomrule
\end{tabular}
\ea
\glt
[About the lack of water during a retreat:] ‘we drank \bluebold{two glasses each} or \bluebold{three glasses each} (per day)’ (Lit. ‘\bluebold{two by two} or \bluebold{three by three}’) \textstyleExampleSource{[081025-009a-Cv.0069]}
\end{styleFreeTranslEngxvpt}

\subsection{Additional function of \textitbf{satu} ‘one’}
\label{bkm:Ref350501509}
In addition to its enumerating function in postposed position, adnominally used \textitbf{satu} ‘one’ is employed to encode “specific indefiniteness” {\citep[444]{Crystal2008}}. That is, in \textsc{nn}um-\textsc{np}s adnominal \textitbf{satu} ‘one’ denotes specific but nonidentifiable referents, giving the specific indefinite reading ‘\textsc{n} \textitbf{satu}’ ‘a certain \textsc{n}’. The specific indefinite referent may be animate human such as \textitbf{ade satu} ‘a certain younger sibling’ in (0) or inanimate such as \textitbf{kampung satu} ‘a certain village’ in (0). The referent of \textitbf{ojek satu} in (0) can be interpreted as the animate referent ‘motorbike taxi driver’, or as the inanimate referent ‘motorbike taxi’.
\end{styleBodyxafter}

\begin{tabular}{llllll}
\lsptoprule
\label{bkm:Ref339637322}
\gll {ada} {\bluebold{ade}} {\bluebold{satu}} {di} {situ}\\ %
& exist & ySb & one & at & \textsc{l.med}\\
\lspbottomrule
\end{tabular}
\ea
\glt 
‘(there) is \bluebold{a certain younger sibling} there’ \textstyleExampleSource{[080922-004-Cv.0018]}
\z

\begin{tabular}{lllllllllllll}
\lsptoprule
\label{bkm:Ref339637325}
\gll {sa} {\multicolumn{2}{l}{pas}} {\multicolumn{2}{l}{jalang}} {\multicolumn{2}{l}{kaki}} {sampe} {di} {\bluebold{kampung}} {\bluebold{satu}} {Wareng}\\ %
& \textsc{1sg} & \multicolumn{2}{l}{precisely} & \multicolumn{2}{l}{walk} & \multicolumn{2}{l}{foot} & reach & at & village & one & Wareng\\
& \multicolumn{2}{l}{ada} & \multicolumn{2}{l}{\bluebold{ojek}} & \multicolumn{2}{l}{\bluebold{satu}} & \multicolumn{6}{l}{turung}\\
& \multicolumn{2}{l}{exist} & \multicolumn{2}{l}{motorbike.taxi} & \multicolumn{2}{l}{one} & \multicolumn{6}{l}{descend}\\
\lspbottomrule
\end{tabular}
\ea
\glt
‘right at the moment when I was walking on foot as far as \bluebold{a certain village} (named) Wareng, there was \bluebold{a certain motorbike taxi (driver)} that(/who) came down (the road)’ \textstyleExampleSource{[080923-010-CvNP.0001]}
\end{styleFreeTranslEngxvpt}

\section{Quantifiers}
\label{bkm:Ref351641491}
As non-numeric expressions, the Papuan Malay quantifiers denote definite or indefinite quantities of their referents. The universal and mid-range quantifiers are discussed in §5.10.1, and distributive quantifiers in §5.10.2.
\end{styleBodyxvafter}

\subsection{Universal and mid-range quantifiers}
\label{bkm:Ref350504986}
The Papuan Malay quantifiers are listed in Table  ‎5 .37, following {Gil’s (2001b)} distinction of universal and mid-range quantifiers.\footnote{\\
\\
\\
\\
\\
\\
\\
\\
\\
\\
\\
\\
\par Following {\citet[1]{Gil2011}}, the expression \textitbf{sembarang} ‘any’ is a “free-choice universal quantifier”.\\
\\
\\
}


\begin{stylecaption}
\label{bkm:Ref339717564}Table ‎5.\stepcounter{Table}{\theTable}:  Papuan Malay quantifiers
\end{stylecaption}

\begin{tabular}{ll}
\lsptoprule

\multicolumn{2}{l}{ Universal quantifiers}\\
\textitbf{masing-masing} & ‘all’\\
\textitbf{segala} & ‘all’\\
\textitbf{sembarang} & ‘any (kind of)’\\
\textitbf{(se)tiap} & ‘every’\\
\textitbf{smua} & ‘all’\\
\multicolumn{2}{l}{ Mid-range quantifiers}\\
\textitbf{banyak} & ‘many’\\
\textitbf{brapa} & ‘several’\\
\textitbf{sedikit} & ‘few’\\
\textitbf{stenga} & ‘half’\\
\lspbottomrule
\end{tabular}

Noun phrases with adnominal quantifiers have syntactic properties similar to those with adnominal numerals, as illustrated in (0) to (0). Noun phrases with pre-head quantifiers (‘\textsc{q}t\textsc{n-np}’) express non-numeric amounts or quantities of the items indicated by their head nominals. Thereby, the composite nature of countable referents is accentuated. Post-head quantifiers, by contrast, may denote exhaustivity of indefinite referents or signal unknown positions within series or sequences; they modify countable as well as uncountable referents.
\end{styleBodyaftervbefore}


The data in (0) to (0) show that not all quantifiers occur in all positions. Only five quantifiers occur in either pre- or post-head position, namely \textitbf{banyak} ‘many’, \textitbf{brapa} ‘several’, \textitbf{masing-masing} ‘each’, \textitbf{sedikit} ‘few’, and \textitbf{smua} ‘all’. While \textitbf{banyak} ‘many’, \textitbf{sedikit} ‘few’, and \textitbf{smua} ‘all’ can modify both count and mass nouns, \textitbf{brapa} ‘several’ and \textitbf{masing-masing} ‘each’ only modify count nouns. The remaining four quantifiers occur in pre-head position only, namely \textitbf{segala} ‘all’, \textitbf{sembarang} ‘any (kind of)’, \textitbf{(se)tiap} ‘every’, and \textitbf{stenga} ‘half’. These quantifiers modify count nouns only.
\end{styleBodyvafter}


Five of the quantifiers are used with either animate or inanimate referents, namely \textitbf{banyak} ‘many’, \textitbf{brapa} ‘several’, \textitbf{masing-masing} ‘each’, \textitbf{sedikit} ‘few’, and \textitbf{smua} ‘all’. By contrast, \textitbf{sembarang} ‘any (kind of)’ is only used with animate referents, and \textitbf{(se)tiap} ‘every’ and \textitbf{stenga} ‘half’ only with inanimate referents.\footnote{\\
\\
\\
\\
\\
\\
\\
\\
\\
\\
\\
\\
\par To express the notion of ‘every person’, speakers prefer quantification with \textitbf{masing-masing} ‘each’.\\
\\
\\
} Universal \textitbf{segala} ‘all’ is only used in combination with the noun \textitbf{macang} ‘variety’. (The mid-range quantifier \textitbf{brapa} ‘several’ also functions as an interrogative, which questions quantities in the sense of ‘how many’, as discussed in §5.8.7). (For details on the adnominal uses of the quantifiers see §8.3.2.)
\end{styleBodyvvafter}

\begin{styleExampleTitle}
Adnominal quantifiers in preposed and postposed position\footnote{\\
\\
\\
\\
\\
\\
\\
\\
\\
\\
\\
\\
\par Documentation: \textitbf{banyak} ‘many’ 081006-023-CvEx.0007, 081029-004-Cv.0021, \textstyleExampleSource{081011-001-Cv.0240; }\textitbf{brapa} ‘several’ 080919-001-Cv.0049, 080923-008-Cv.0012\textstyleExampleSource{; }\textitbf{masing-masing} ‘each’ BR111021.010, BR111021.009, \textitbf{sedikit} ‘few’ \textstyleExampleSource{BR111021-001.004}, \textstyleExampleSource{BR111021-001.006, }081006-035-CvEx.0050; \textitbf{segala} ‘all’ 081006-032-Cv.0017; \textitbf{sembarang} ‘any’ 080927-006-CvNP.0035; \textitbf{setiap} ‘every’ 080923-016-CvNP.0002; \textitbf{smua} ‘all’ 081006-030-CvEx.0009, 080921-004b-Cv.0026, BR111021.012; \textitbf{stenga} 081115-001b-Cv.0056.\\
\\
\\
}
\end{styleExampleTitle}

\tablehead{ &  &  & Pre-head position & Post-head position\\
}
\begin{tabular}{lllll}
\lsptoprule
\label{bkm:Ref406521700}
\gll { & Count N & \bluebold{\textmd{banyak}} orang & orang \bluebold{\textmd{banyak}}}\\ %
&  &  & many person & person many\\
&  &  & ‘\bluebold{many} people’ & ‘\bluebold{many} people’\\
&  & Mass N &  & te \bluebold{banyak}\\
&  &  &  & tea many\\
&  &  &  & ‘\bluebold{lots of} tea’\\
(\stepcounter{}{\the}) &  & Count N & \bluebold{sedikit} orang & kladi \bluebold{sedikit}\\
&  &  & few person & taro.root few\\
&  &  & ‘\bluebold{\textmd{few}} people’ & ‘\bluebold{\textmd{few}} taro roots’\\
&  & Mass N &  & air \bluebold{\textmd{sedikit}}\\
&  &  &  & water few\\
&  &  &  & ‘\bluebold{\textmd{little}} water’\\
(\stepcounter{}{\the}) &  & Count N & \bluebold{smua} masala & pemuda \bluebold{smua}\\
&  &  & all problem & youth all\\
&  &  & ‘\bluebold{all} problems’ & ‘\bluebold{all of} the young people’\\
&  & Mass N &  & gula \bluebold{smua}\\
&  &  &  & sugar all\\
&  &  &  & ‘\bluebold{all} (of the) sugar’\\
(\stepcounter{}{\the}) &  & Count N & \bluebold{brapa} orang & dorang \bluebold{brapa}\\
&  &  & several person & \textsc{3pl} several\\
&  &  & ‘\bluebold{several} people’ & ‘\bluebold{several} of them’\\
(\stepcounter{}{\the}) &  & Count N & \bluebold{masing-masing} trek & trek \bluebold{masing-masing}\\
&  &  & each truck & truck each\\
&  &  & ‘\bluebold{each} truck’ & ‘\bluebold{each} truck’\\
(\stepcounter{}{\the}) &  & Count N & \bluebold{segala} macang & \\
&  &  & all variety & \\
&  &  & ‘\bluebold{every}thing, what\bluebold{ever}’ & \\
(\stepcounter{}{\the}) &  & Count N & \bluebold{sembarang} orang & \\
&  &  & any person & \\
&  &  & ‘\bluebold{any} person, \bluebold{any}body’ & \\
(\stepcounter{}{\the}) &  & Count N & \bluebold{setiap} lagu & \\
&  &  & every song & \\
&  &  & ‘\bluebold{every} song’ & \\
\label{bkm:Ref339718055}
\gll { & Count N & \bluebold{stenga} jam & }\\ %
&  &  & half hour & \\
&  &  & ‘\bluebold{half} an hour’ & \\
\lspbottomrule
\end{tabular}

When the identity of the referent was established earlier or can be deduced from the context, the head nominal can be omitted. Not all quantifiers, however, are used in noun phrases with elided head nominal. Attested are only \textitbf{banyak} ‘many’ as in (0), \textitbf{brapa} ‘several’ as in (0), \textitbf{sedikit} ‘few’ as in (0), and \textitbf{smua} ‘all’ as in (0).


\begin{styleExampleTitle}
Quantifiers with omitted head nominal
\end{styleExampleTitle}

\begin{tabular}{llllllllllllll}
\lsptoprule
\label{bkm:Ref339637327}
\gll {\bluebold{banyak}} {Ø} {mati} {di,} {e,} {di} {di} {pulow{\Tilde}pulow,} {\bluebold{banyak}} {Ø} {mati} {di} {lautang}\\ %
& many &  & die & at & uh & at & at & \textsc{rdp}{\Tilde}island & many &  & die & at & ocean\\
\lspbottomrule
\end{tabular}
\ea
\glt 
‘[there are many Papuans who died,] \bluebold{many} (Papuans) died on, uh, on on the islands, \bluebold{many} (Papuans) died on the ocean’ \textstyleExampleSource{[081029-002-Cv.0024-0025]}
\z

\begin{tabular}{lllllllllll}
\lsptoprule
\label{bkm:Ref405213690}
\gll {kalo} {\multicolumn{3}{l}{suda}} {ambil} {satu} {Ø,} {kasiang,} {kitong} {hanya}\\ %
& if & \multicolumn{3}{l}{already} & take & one &  & pity & \textsc{1pl} & only\\
& \multicolumn{2}{l}{\bluebold{brapa}} & Ø & \multicolumn{7}{l}{saja}\\
& \multicolumn{2}{l}{several} &  & \multicolumn{7}{l}{just}\\
\lspbottomrule
\end{tabular}
\ea
\glt 
‘once (they) have taken one (of our children), what a pity, we (have) just a \bluebold{few} (children left)’ (Lit. ‘only \bluebold{several}’) \textstyleExampleSource{[081006-024-CvEx.0070]}
\z

\begin{tabular}{llllllllll}
\lsptoprule
\label{bkm:Ref339637329}
\gll {di} {sini} {yo} {fam} {Yapo} {ini} {ada} {\bluebold{sedikit}} {Ø}\\ %
& at & \textsc{l.prox} & yes & family.name & Yapo & \textsc{d.prox} & exist & few & \\
\lspbottomrule
\end{tabular}
\ea
\glt 
‘here, yes, there are (only) \bluebold{few} Yapo family (members)’ (Lit. ‘this Yapo family is \bluebold{few} (people)’) \textstyleExampleSource{[080922-010a-CvNF.0274]}
\z

\begin{tabular}{lllllllll}
\lsptoprule
\label{bkm:Ref339637330}
\gll {…} {mobil} {blakos,} {Ø} {\bluebold{smua}} {naik} {di} {blakang}\\ %
&  & car & pick-up.truck &  & all & climb & at & backside\\
\lspbottomrule
\end{tabular}
\ea
\glt 
‘[we took] a pick-up truck, \bluebold{all} (of the passengers) got onto its loading space’ \textstyleExampleSource{[081006-017-Cv.0001]}
\z


In addition to their adnominal uses, quantifies are also used predicatively. In (0), for instance, predicatively used \textitbf{banyak} ‘many’ conveys information about the non-numeric quantity of its subject \textitbf{picaang} ‘splinter’. (For details on quantifier predicate clauses see §12.3).


\begin{styleExampleTitle}
Predicatively used quantifier
\end{styleExampleTitle}

\begin{tabular}{lllll}
\lsptoprule
\label{bkm:Ref287689067}
\gll {…} {picaang} {juga} {\bluebold{banyak}}\\ %
&  & splinter & also & many\\
\lspbottomrule
\end{tabular}
\ea
\glt
‘[at the beach] there are also\bluebold{ lots of} splinters’ (Lit. ‘the splinters (are) also\bluebold{ many}’) \textstyleExampleSource{[080917-006-CvHt.0008]}
\end{styleFreeTranslEngxvpt}

\subsection{Distributive quantifiers}
\label{bkm:Ref350504987}
The notion of ‘little by little’ or ‘many by many’ is expressed through reduplication of the quantifier, similar to the formation of distributive numerals presented in (0) and (0) (§5.9.3; see also §4.2.4). Distributive quantifiers denote events that affect an indefinite number of members of a group or set at different points in time. In (0), for example, \textitbf{uang banyak{\Tilde}banyak} denotes ‘sets of lots of money’, while in (0) \textitbf{sedikit{\Tilde}sedikit} designates ‘sets of little (food)’.
\end{styleBodyxafter}

\begin{tabular}{lllll}
\lsptoprule
\label{bkm:Ref339637333}
\gll {bapa} {kirim} {\bluebold{uang}} {\bluebold{banyak{\Tilde}banyak}}\\ %
& father & send & money & \textsc{rdp}{\Tilde}many\\
\lspbottomrule
\end{tabular}
\ea
\glt 
[Phone conversation:] ‘father send \bluebold{lots of money at regular intervals}’ (Lit. ‘\bluebold{lots by lots of money}’) \textstyleExampleSource{[080922-001a-CvPh.0440]}
\z

\begin{tabular}{lllllllll}
\lsptoprule
\label{bkm:Ref339637334}
\gll {dong} {blum} {isi} {selaing} {dong} {isi} {\bluebold{sedikit{\Tilde}sedikit}} {to?}\\ %
& \textsc{3pl} & not.yet & fill & besides & \textsc{3pl} & fill & \textsc{rdp}{\Tilde}few & right?\\
\lspbottomrule
\end{tabular}
\ea
\glt
[About organizing the food distribution during a retreat:] ‘they haven’t yet filled (their plates), moreover they’ll fill (their plates only) with \bluebold{little} (food), right?’ (Lit. ‘\bluebold{little by little (food)}’) \textstyleExampleSource{[081025-009a-Cv.0081]}
\end{styleFreeTranslEngxvpt}

\section{Prepositions}
\label{bkm:Ref351641493}
Papuan Malay has eleven prepositions which serve to denote grammatical and semantic relations between their complements and the predicate. The prepositions have the following defining characteristics:


%\setcounter{itemize}{0}
\begin{itemize}
\item \begin{styleIIndented}
Prepositions introduce prepositional phrases with an overt noun phrase complement which may neither be fronted nor omitted (see Chapter 10).
\end{styleIIndented}\item \begin{styleIIndented}
All prepositions introduce peripheral adjuncts within the clause (see Chapter 10).
\end{styleIIndented}\item \begin{styleIIndented}
Most of the prepositions also introduce oblique arguments and/or nonverbal predicates (see§11.1.3.2 and §12.4, respectively).
\end{styleIIndented}\item \begin{styleIvI}
Some of the prepositions also introduce prepositional phrases that function as modifiers within noun phrases (see §8.2.7).
\end{styleIvI}\end{itemize}

The Papuan Malay prepositions are presented in Table  ‎5 .38. Three groups of prepositions are distinguished according to the semantic relations between their complements and the predicate: prepositions encoding (1) location in space and time, (2) accompaniment/instruments, goals, and benefaction, and (3) comparisons.


\begin{stylecaption}
\label{bkm:Ref350443584}Table ‎5.\stepcounter{Table}{\theTable}:  Papuan Malay prepositions according to the semantic relations between their complements and the predicate\footnote{\\
\\
\\
\\
\\
\\
\\
\\
\\
\\
\\
\\
\par Both \textitbf{untuk} ‘for’ and \textitbf{buat} ‘for’ introduce beneficiaries and benefactive recipients. Benefactive \textitbf{untuk} ‘for’, however, has a wider distribution and more functions than \textitbf{buat} ‘for’ in that \textitbf{untuk} ‘for’ (1) combines with demonstratives, (2) introduces inanimate referents, and (3) introduces circumstance. For details see §10.2.3 and §10.2.4.\\
Both \textitbf{kaya} ‘like’ and \textitbf{sperti} ‘similar to’ signal likeness in terms of appearance or behavior. They differ in scope, however. Similative \textitbf{kaya} ‘like’ signals overall resemblance between the two bases of comparison. The scope of \textitbf{sperti} ‘similar to’, by contrast, is more limited: it signals likeness or resemblance in some, typically implied, respect. For details see §10.3.1 and §10.3.2.\\
\\
}
\end{stylecaption}

\begin{tabular}{llllll}
\lsptoprule

\multicolumn{6}{l}{%\setcounter{itemize}{0}
\begin{itemize}
\item Prepositions encoding location in space and time (§10.1)\end{itemize}
}\\
& \multicolumn{2}{l}{Preposition} & Gloss & \multicolumn{2}{l}{Semantic relations}\\
\multicolumn{2}{l}{} & \textitbf{di} & ‘at, in’ &  & static location\\
\multicolumn{2}{l}{} & \textitbf{ke} & ‘to’ &  & movement toward a referent\\
\multicolumn{2}{l}{} & \textitbf{dari} & ‘from’ &  & movement from a source location\\
\multicolumn{2}{l}{} & \textitbf{sampe} & ‘until’ &  & movement toward a non-spatial temporal endpoint\\
\multicolumn{6}{l}{\begin{itemize}
\item Prepositions encoding accompaniment/instruments, goals, and benefaction (§10.2)\end{itemize}
}\\
& \multicolumn{2}{l}{Preposition} & Gloss & \multicolumn{2}{l}{Semantic relations}\\
\multicolumn{2}{l}{} & \textitbf{dengang} & ‘with’ &  & accompaniment\\
\multicolumn{2}{l}{} & \textitbf{sama} & ‘to’ &  & goal\\
\multicolumn{2}{l}{} & \textitbf{untuk} & ‘for’ &  & benefaction\\
\multicolumn{2}{l}{} & \textitbf{buat} & ‘for’ &  & benefaction\\
\multicolumn{6}{l}{\begin{itemize}
\item Prepositions encoding comparison (§10.3)\end{itemize}
}\\
& \multicolumn{2}{l}{Preposition} & Gloss & \multicolumn{2}{l}{Semantic relations}\\
\multicolumn{2}{l}{} & \textitbf{sperti} & ‘similar to’ &  & similarity\\
\multicolumn{2}{l}{} & \textitbf{kaya} & ‘like’ &  & similarity\\
\multicolumn{2}{l}{} & \textitbf{sebagey} & ‘as’ &  & equatability\\
\lspbottomrule
\end{tabular}

The complement in a prepositional phrase is obligatory. If the semantic relationship between this complement and the predicate can be deduced from the context, two of the prepositions of location may be omitted, locative \textitbf{di} ‘at, in’ and allative \textitbf{ke} ‘to’. A full discussion of the Papuan Malay prepositions and prepositional phrases is given in Chapter 10.
\end{styleBodyaftervbefore}


A considerable number of the prepositions have dual word class membership, two have trial class membership. That is, three prepositions are also used as verbs: \textitbf{buat} ‘for’, \textitbf{sama} ‘to’, and \textitbf{sampe} ‘until’ (see §5.3). Six prepositions are also used as conjunctions: \textitbf{dengang} ‘with’, \textitbf{kaya} ‘like’, \textitbf{sama} ‘to’, \textitbf{sampe} ‘until’, \textitbf{sperti} ‘similar to’, and \textitbf{untuk} ‘for’ (see §5.12 and Chapter 14). (Variation in word class membership is discussed in §5.14.)
\end{styleBodyvxvafter}

\section{Conjunctions}
\label{bkm:Ref351641494}
Papuan Malay has 23 conjunctions which serve to connect words, phrases, or clauses. They have the following defining characteristics:


%\setcounter{itemize}{0}
\begin{itemize}
\item \begin{styleIIndented}
Conjunctions combine different constituents, namely clauses, noun phrases, prepositional phrases, and verbs; they do not head phrases.
\end{styleIIndented}\item \begin{styleIIndented}
Conjunctions occur at the periphery of the constituents they mark.
\end{styleIIndented}\item \begin{styleIvI}
Conjunctions form intonation units with the constituents they mark, although they do not belong to them semantically.
\end{styleIvI}\end{itemize}

The Papuan Malay conjunctions can be divided into two major groups, namely those combining same-type constituents, such as clauses with clauses, and those linking different-type constituents, such as verbs with clauses.



Conjunctions combining clauses are traditionally divided into coordinating and subordinating ones {(Schachter and Shopen 2007: 45)}. With respect to clause linking in Papuan Malay, however, there is no formal marking of this distinction. That is, in terms of their morphosyntax and word order, clauses marked with a conjunction are not distinct from those which do not have a conjunction. (For details see Chapter14.)
\end{styleBodyvafter}


Table  ‎5 .39 gives an overview of the Papuan Malay conjunctions attested in the corpus. They are grouped in terms of the types of constituents they combine and the semantic relations they signal. Two of the conjunctions are listed twice as they mark more than one type of semantic relation, namely \textitbf{baru} ‘and then, after all’, and \textitbf{sampe} ‘until’.
\end{styleBodyvvafter}

\begin{stylecaption}
\label{bkm:Ref356475667}Table ‎5.\stepcounter{Table}{\theTable}:  Papuan Malay conjunctions
\end{stylecaption}

\tablehead{
\multicolumn{8}{l}{%\setcounter{itemize}{0}
\begin{itemize}
\item Conjunctions combining same-type constituents (§14.2)\end{itemize}
}\\
}
\begin{tabular}{llllllll} & \multicolumn{7}{l}{\begin{itemize}
\lsptoprule
\item %\setcounter{itemize}{0}
\begin{itemize}
\item Conjunctions marking addition (§14.2.1)\end{itemize}
\end{itemize}
}\\
\multicolumn{2}{l}{} & \multicolumn{2}{l}{Conjunction} & \multicolumn{2}{l}{Gloss} & \multicolumn{2}{l}{Semantic relations}\\
\multicolumn{3}{l}{} & \multicolumn{2}{l}{\textitbf{dengang}} & \multicolumn{2}{l}{‘with’} & Addition\\
\multicolumn{3}{l}{} & \multicolumn{2}{l}{\textitbf{dang}} & \multicolumn{2}{l}{‘and’} & Addition\\
\multicolumn{3}{l}{} & \multicolumn{2}{l}{\textitbf{sama}} & \multicolumn{2}{l}{‘to’} & Addition\\
& \multicolumn{7}{l}{\begin{itemize}
\item %\setcounter{itemize}{0}
\begin{itemize}
\item Conjunctions marking alternative (§14.2.2)\end{itemize}
\end{itemize}
}\\
\multicolumn{2}{l}{} & \multicolumn{2}{l}{Conjunction} & \multicolumn{2}{l}{Gloss} & \multicolumn{2}{l}{Semantic relations}\\
\multicolumn{3}{l}{} & \multicolumn{2}{l}{\textitbf{ato}} & \multicolumn{2}{l}{‘or’} & Alternative\\
\multicolumn{3}{l}{} & \multicolumn{2}{l}{\textitbf{ka}} & \multicolumn{2}{l}{‘or’} & Alternative\\
& \multicolumn{7}{l}{\begin{itemize}
\item %\setcounter{itemize}{0}
\begin{itemize}
\item Conjunctions marking time and/or condition (§14.2.3)\end{itemize}
\end{itemize}
}\\
\multicolumn{2}{l}{} & \multicolumn{2}{l}{Conjunction} & \multicolumn{2}{l}{Gloss} & \multicolumn{2}{l}{Semantic relations}\\
\multicolumn{3}{l}{} & \multicolumn{2}{l}{\textitbf{trus}} & \multicolumn{2}{l}{‘next’} & Sequence (neutral)\\
\multicolumn{3}{l}{} & \multicolumn{2}{l}{\textitbf{baru}} & \multicolumn{2}{l}{‘and then’} & Sequence (contrastive)\\
\multicolumn{3}{l}{} & \multicolumn{2}{l}{\textitbf{sampe}} & \multicolumn{2}{l}{‘until’} & Anteriority\\
\multicolumn{3}{l}{} & \multicolumn{2}{l}{\textitbf{seblum}} & \multicolumn{2}{l}{‘before’} & Anteriority\\
\multicolumn{3}{l}{} & \multicolumn{2}{l}{\textitbf{kalo}} & \multicolumn{2}{l}{‘if, when’} & Posteriority / Condition\\
& \multicolumn{7}{l}{\begin{itemize}
\item %\setcounter{itemize}{0}
\begin{itemize}
\item Conjunctions marking consequence (§14.2.4)\end{itemize}
\end{itemize}
}\\
\multicolumn{2}{l}{} & \multicolumn{2}{l}{Conjunction} & \multicolumn{2}{l}{Gloss} & \multicolumn{2}{l}{Semantic relations}\\
\multicolumn{3}{l}{} & \multicolumn{2}{l}{\textitbf{jadi}} & \multicolumn{2}{l}{‘so, since’} & Result / Cause\\
\multicolumn{3}{l}{} & \multicolumn{2}{l}{\textitbf{supaya}} & \multicolumn{2}{l}{‘so that’} & Purpose\\
\multicolumn{3}{l}{} & \multicolumn{2}{l}{\textitbf{untuk}} & \multicolumn{2}{l}{‘for’} & Purpose\\
\multicolumn{3}{l}{} & \multicolumn{2}{l}{\textitbf{sampe}} & \multicolumn{2}{l}{‘with the result that’} & Result\\
\multicolumn{3}{l}{} & \multicolumn{2}{l}{\textitbf{karna}} & \multicolumn{2}{l}{‘because’} & Cause (neutral)\\
\multicolumn{3}{l}{} & \multicolumn{2}{l}{\textitbf{gara-gara}} & \multicolumn{2}{l}{‘because’} & Cause (emotive)\\
& \multicolumn{7}{l}{\begin{itemize}
\item %\setcounter{itemize}{0}
\begin{itemize}
\item Conjunctions marking contrast (§14.2.5)\end{itemize}
\end{itemize}
}\\
\multicolumn{2}{l}{} & \multicolumn{2}{l}{Conjunction} & \multicolumn{2}{l}{Gloss} & \multicolumn{2}{l}{Semantic relations}\\
\multicolumn{3}{l}{} & \multicolumn{2}{l}{\textitbf{tapi}} & \multicolumn{2}{l}{‘but’} & Contrast\\
\multicolumn{3}{l}{} & \multicolumn{2}{l}{\textitbf{habis}} & \multicolumn{2}{l}{‘after all’} & Contrast\\
\multicolumn{3}{l}{} & \multicolumn{2}{l}{\textitbf{baru}} & \multicolumn{2}{l}{‘after all’} & Contrast\\
\multicolumn{3}{l}{} & \multicolumn{2}{l}{\textitbf{padahal}} & \multicolumn{2}{l}{‘but actually’} & Contrast\\
\multicolumn{3}{l}{} & \multicolumn{2}{l}{\textitbf{biar}} & \multicolumn{2}{l}{‘although’} & Concession\\
& \multicolumn{7}{l}{\begin{itemize}
\item %\setcounter{itemize}{0}
\begin{itemize}
\item Conjunctions marking similarity (§14.2.6)\end{itemize}
\end{itemize}
}\\
\multicolumn{2}{l}{} & \multicolumn{2}{l}{Conjunction} & \multicolumn{2}{l}{Gloss} & \multicolumn{2}{l}{Semantic relations}\\
\multicolumn{3}{l}{} & \multicolumn{2}{l}{\textitbf{sperti}} & \multicolumn{2}{l}{‘similar to’} & Similarity (partial)\\
\multicolumn{3}{l}{} & \multicolumn{2}{l}{\textitbf{kaya}} & \multicolumn{2}{l}{‘like’} & Similarity (overall)\\
\multicolumn{8}{l}{\begin{itemize}
\item Conjunctions combining different-type constituents (§14.3)\end{itemize}
}\\
\multicolumn{2}{l}{} & \multicolumn{2}{l}{Conjunction} & \multicolumn{2}{l}{Gloss} & \multicolumn{2}{l}{Syntactic function}\\
\multicolumn{3}{l}{} & \multicolumn{2}{l}{\textitbf{bahwa}} & \multicolumn{2}{l}{‘that’} & Complementizer\\
\multicolumn{3}{l}{} & \multicolumn{2}{l}{\textitbf{yang}} & \multicolumn{2}{l}{‘\textsc{rel}’} & Relativizer\\
\lspbottomrule
\end{tabular}

A substantial number of the conjunctions have dual word class membership, two have trial class membership. More specifically, seven of them are also used as verbs, namely \textitbf{biar} ‘although’, ‘\textitbf{buat} ‘for’, \textitbf{habis} ‘after all’, \textitbf{jadi} ‘so, since’, \textitbf{sama} ‘to’, \textitbf{sampe} ‘until’, and \textitbf{trus} ‘next’ (see §5.3). Six conjunctions are also used as prepositions, namely \textitbf{dengang} ‘with’, \textitbf{kaya} ‘like’, \textitbf{sama} ‘to’, \textitbf{sampe} ‘until’, \textitbf{sperti} ‘similar to’, and \textitbf{untuk} ‘for’ (see §5.11 and Chapter 10). Besides, alternative-marking \textitbf{ka} ‘or’ is also used to mark interrogative clauses (see §13.2.3). (For details on variation in word class membership see §5.14.)


\section{Tags, placeholders and hesitation markers, interjections, and onomatopoeia}
\label{bkm:Ref358716377}\subsection{Tags}
\label{bkm:Ref374463316}\label{bkm:Ref374463262}
Papuan Malay has three tags, \textitbf{to} ‘right?’, \textitbf{e} ‘eh?’ and \textitbf{kang} ‘you know?’, as shown in (0) to (0). They are short questions that are “tagged” onto the end of an utterance and have a rising intonation. Their main function is to confirm what is being said.



With \textitbf{to} ‘right?’, speakers ask for agreement or disagreement, as in (0) and (0),\footnote{\\
\\
\\
\\
\\
\\
\\
\\
\\
\\
\\
\\
\\
\par The tag \textitbf{to} ‘right?’ is a loan word from Dutch, which uses \textitbf{toch} ‘right?’ as a tag.\\
\\
} while with \textitbf{kang} ‘you know?’ speakers assume their interlocutors to agree with their statements, as in (0) and (0). Speakers use \textitbf{to} ‘right?’ at the end of an utterance. When employing \textitbf{kang} ‘you know?’, by contrast, they usually continue their utterance and add further information related to the issue under discussion. In this context, \textitbf{kang} ‘you know?’ quite often co-occurs with \textitbf{to} ‘right?’, as in (0).
\end{styleBodyvvafter}

\begin{styleExampleTitle}
Tags: \textitbf{to} ‘right?’ and \textitbf{kang} ‘you know’
\end{styleExampleTitle}

\begin{tabular}{llllllll}
\lsptoprule
\label{bkm:Ref358368951}
\gll {sebentar} {pasti} {hujang} {karna} {awang} {hitam} {\bluebold{to}?}\\ %
& in.a.moment & definitely & rain & because & cloud & be.black & right?\\
\lspbottomrule
\end{tabular}
\ea
\glt 
‘in a bit it will certainly rain because of the black clouds, \bluebold{right?}’ \textstyleExampleSource{[080919-005-Cv.0016]}
\z

\begin{tabular}{llllllll}
\lsptoprule
\label{bkm:Ref362111471}
\gll {de} {suda} {tidor,} {\bluebold{kang}?,} {dia} {hosa} {\bluebold{to}?}\\ %
& \textsc{3sg} & already & sleep & you.know & \textsc{3sg} & pant & right?\\
\lspbottomrule
\end{tabular}
\ea
\glt 
‘she was already sleeping, \bluebold{you know?}, she has breathing difficulties, \bluebold{right?}’ \textstyleExampleSource{[080916-001-CvNP.0005]}
\z

\begin{tabular}{llllllll}
\lsptoprule
\label{bkm:Ref362114103}
\gll {dong} {bilang} {soa-soa} {\bluebold{kang}?,} {kaya} {buaya} {begitu}\\ %
& \textsc{3pl} & say & monitor.lizard & you.know & like & crocodile & like.that\\
\lspbottomrule
\end{tabular}
\ea
\glt 
‘they call (it) a monitor lizard, \bluebold{you know?}, (it’s) like a crocodile’ \textstyleExampleSource{[080922-009-CvNP.0053]}
\z


Like \textitbf{to} ‘right?’, \textitbf{e} ‘eh?’ occurs at the end of an utterance, and like \textitbf{kang} ‘you know’, it assumes agreement. Its uses seem to be more restricted, though, than those of the two other tags. Speakers tend to employ \textitbf{e} ‘eh?’ as a marker of assurance, that is, when they want to give assurance, as in (0), or ask for assurance as in (0) and (0). As an extension of this assurance-marking function, \textitbf{e} ‘eh?’ is also used to mark imperatives, as in (0) (see also §13.3.1).


\begin{styleExampleTitle}
Tags: \textitbf{e} ‘eh?’
\end{styleExampleTitle}

\begin{tabular}{llllllll}
\lsptoprule
\label{bkm:Ref362111472}
\gll {saya} {cabut} {ko} {dari} {skola} {itu} {\bluebold{e}?}\\ %
& \textsc{1sg} & pull.out & \textsc{2sg} & from & school & \textsc{d.dist} & eh\\
\lspbottomrule
\end{tabular}
\ea
\glt 
‘I’ll take you out of school there, \bluebold{eh?}’ \textstyleExampleSource{[080922-001a-CvPh.0199]}
\z

\begin{tabular}{lllllll}
\lsptoprule
\label{bkm:Ref363640760}
\gll {bapa} {datang} {\bluebold{e}?} {bapa} {datang} {\bluebold{e}?}\\ %
& father & come & eh & father & come & eh\\
\lspbottomrule
\end{tabular}
\ea
\glt 
‘you (‘father’) will come (here), \bluebold{eh?}, you (‘father’) will come (here), \bluebold{eh?}’ \textstyleExampleSource{[080922-001a-CvPh.1072]}
\z

\begin{tabular}{lllllllll}
\lsptoprule
\label{bkm:Ref363640762}
\gll {ade} {bongso} {jadi} {ko} {sayang} {dia} {skali} {\bluebold{e}?}\\ %
& ySb & youngest.offspring & so & \textsc{2sg} & love & \textsc{3sg} & very & eh\\
\lspbottomrule
\end{tabular}
\ea
\glt 
‘(your) youngest sibling, so you love her very much, \bluebold{eh?}’ \textstyleExampleSource{[080922-001a-CvPh.0302]}
\z

\begin{tabular}{llllllllllll}
\lsptoprule
\label{bkm:Ref362111473}
\gll {hari} {minggu} {ko} {ke} {ruma} {\bluebold{e}?} {ke} {Siduas} {punya} {ruma} {\bluebold{e}?}\\ %
& day & Sunday & \textsc{2sg} & to & house & eh & to & Siduas & \textsc{poss} & house & eh\\
\lspbottomrule
\end{tabular}
\ea
\glt
‘on Sunday you go to the house, \bluebold{eh?!}, to Siduas’ house, \bluebold{eh?!}’ \textstyleExampleSource{[080922-001a-CvPh.0341]}
\end{styleFreeTranslEngxvpt}

\subsection{Placeholders and hesitation markers}
\label{bkm:Ref439514015}
Papuan Malay has five placeholders, namely the three interrogatives \textitbf{siapa} ‘who’, \textitbf{apa} ‘what’ and \textitbf{bagemana} ‘how’ and the two demonstratives \textitbf{ini} ‘\textsc{d.prox}’ and \textitbf{itu} ‘\textsc{d.prox}’. Their main function is to substitute for lexical items that the speaker has temporarily forgotten. The five placeholders are discussed in the respective sections on interrogatives (§5.8) and demonstratives (§7.1.2.6).



Hesitation markers, by contrast, have no lexical meaning. As vocal indicators they mainly serve to fill pauses. The main Papuan Malay hesitation marker is \textitbf{e(m)} ‘uh’, as in (0); alternative realizations are \textitbf{u(m)} ‘uh’ as in (0), or \textitbf{a(m)}, \textitbf{mmm}, or \textitbf{nnn} ‘uh’.
\end{styleBodyvvafter}

\begin{tabular}{llllllllll}
\lsptoprule
\label{bkm:Ref358386454}
\gll {kalo} {sa} {su} {pake,} {\bluebold{em},} {kaca-mata} {tu} {mungking} {…}\\ %
& if & \textsc{1sg} & already & use & uh & glasses & \textsc{d.dist} & maybe & \\
\lspbottomrule
\end{tabular}
\ea
\glt 
‘if I’d been wearing, \bluebold{uh}, those (sun)glasses, maybe …’ \textstyleExampleSource{[080919-005-Cv.0007]}
\z

\begin{tabular}{lllllllll}
\lsptoprule
\label{bkm:Ref358386456}
\gll {pace} {Oktofernus} {de,} {\bluebold{u},} {masi} {urus} {dorang} {sana}\\ %
& man & Oktofernus & \textsc{3sg} & uh & still & arrange & \textsc{3pl} & \textsc{l.dist}\\
\lspbottomrule
\end{tabular}
\ea
\glt
‘Mr. Oktofernus, \bluebold{uh}, was still taking care of them over there’ \textstyleExampleSource{[081025-008-Cv.0121]}
\end{styleFreeTranslEngxvpt}

\subsection{Interjections}

Interjections typically “constitute utterances by themselves and express a speaker’s current mental state or reaction toward an element in the linguistic or extralinguistic context” {\citep[743]{Ameka2006}}. Hence, “interjections are context-bound linguistic signs” {(2006: 743)}. That is, their interpretation depends on the specific context in which they are uttered. This also applies to Papuan Malay, as illustrated with the interjection \textitbf{adu} ‘ouch!, oh no!’. Depending on the context, the interjection expresses pain, ‘ouch!’, or disappointed surprise, ‘oh no!’.



Cross-linguistically, two major types of interjections are distinguished, namely primary and secondary interjections {\citep{Ameka2006}}. Primary interjections are defined as “little words or ‘non-words’, which […] do not normally enter into construction with other word classes” {(2006: 744)}. Secondary interjections, by contrast, are defined as “words that have an independent semantic value but which can be used conventionally as nonelliptical utterances by themselves to express a mental attitude or state” {(2006: 744)}.
\end{styleBodyvafter}


Papuan Malay primary interjections are presented Table  ‎5 .40 and in the examples in (0) to (0), and secondary interjections in Table  ‎5 .39 and in the examples in (0) to (0).
\end{styleBodyvafter}


The primary interjections, listed in Table  ‎5 .40, include words used for expressing emotions such as \textitbf{ba} ‘humph!’, getting attention such as \textitbf{e} ‘hey’, or addressing animals, such as \textitbf{ceh} ‘shoo’.
\end{styleBodyvvafter}

\begin{stylecaption}
\label{bkm:Ref358470054}Table ‎5.\stepcounter{Table}{\theTable}:  Papuan Malay primary interjections
\end{stylecaption}

\tablehead{
 Item & Gloss & \arraybslash Semantics: Interjection used …\\
}
\begin{tabular}{lll}
\lsptoprule
\textitbf{a} & ‘ah!, oh boy!, ugh!’ & to express emotions ranging from contentment to acute discomfort or annoyance\\
\textitbf{adu} & ‘ouch!, oh no’ & to express pain or disappointed surprise\\
\textitbf{ale} & ‘wow!’ & to express surprise or to attract attention\\
\textitbf{ay} & ‘aah!, aw!’ & to express surprise or affection\\
\textitbf{ba} & ‘humph!’ & to express disgust or denigration\\
\textitbf{ceh} & ‘shoo!’ & to chase something away\\
\textitbf{e} & ‘hey!, ha!, eh?’ & to express emphasis or astonishment or to attract attention\\
\textitbf{ha} & ‘huh?’ & to express surprise, disbelief, or confusion\\
\textitbf{hm} & ‘pfft’ & to express sarcasm or disagreement\\
\textitbf{hura} & ‘hooray!’ & to express joy, approval, or encouragement\\
\textitbf{i} & ‘ugh!, oh no!, oh!’ & to express disgust, irritation or disappointed surprise\\
\textitbf{isss} & ‘stop!’ & to stop someone/-thing or to attract attention\\
\textitbf{mpfff} & ‘ugh!’ & to express displeasure, or incredulity\\
\textitbf{na} & ‘well’ & to introduce a comment or statement, or to resume a conversation\\
\textitbf{o} & ‘oh!’ & to express surprise\\
\textitbf{oke} & ‘OK’ & to express agreement\\
\textitbf{prrrt} & ‘pfft!’ & to express sarcasm or disagreement\\
\textitbf{sio} & ‘alas!’ & to express sorrow or pity\\
\textitbf{sss} & ‘pfft!’ & to express sarcasm or disagreement\\
\textitbf{sssyyyt} & ‘shhh!’ & to silence someone\\
\textitbf{tsk-tsk} & ‘tsk-tsk’ & to express disapproval\\
\textitbf{uy} & ‘o boy!’ & to express surprise or to attract attention\\
\textitbf{wa} & ‘wow!’ & to express surprise or exasperation\\
\lspbottomrule
\end{tabular}

Examples of primary interjections in natural discourse are presented in (0) to (0).


\begin{styleExampleTitle}
Primary interjections
\end{styleExampleTitle}

\begin{tabular}{llllllllll}
\lsptoprule
\label{bkm:Ref358473772}
\gll {\bluebold{a},} {saya} {bisa} {pulang} {karna} {sa} {su} {dapat} {babi}\\ %
& ah! & \textsc{1sg} & be.able & go.home & because & \textsc{1sg} & already & get & pig\\
\lspbottomrule
\end{tabular}
\ea
\glt 
‘\bluebold{ah!}, I can return home because I already got the pig’ \textstyleExampleSource{[080919-004-NP.0024]}
\z

\begin{tabular}{llllllllll}
\lsptoprule
(\stepcounter{}{\the}) & \bluebold{mpfff}, & Yonece & de & liat\bluebold{{\Tilde}}liat & sa & smes & di & net & to?\\
& ugh! & Yonece & \textsc{3sg} & \textsc{rdp}{\Tilde}see & \textsc{1sg} & smash & at & (sport.)net & right?\\
\lspbottomrule
\end{tabular}
\ea
\glt 
[About a volleyball game:] ‘\bluebold{ugh!}, Yonece saw (that) I was going to smash, right?’ \textstyleExampleSource{[081109-001-Cv.0160]}
\z

\begin{tabular}{lllll}
\lsptoprule
\label{bkm:Ref358473774}
\gll {\bluebold{o},} {dong} {mara} {\bluebold{e}?}\\ %
& oh! & \textsc{3pl} & feel.angry(.about) & eh\\
\lspbottomrule
\end{tabular}
\ea
\glt 
‘\bluebold{oh!}, they’ll be angry, \bluebold{eh?}’ \textstyleExampleSource{[080917-008-NP.0054]}
\z


Examples of the Papuan Malay secondary interjections, listed in Table  ‎5 .41, include words for expressing emotions such as \textitbf{sunggu} ‘good grief’, as well as routine expressions for thanking, greetings, or leave-taking, such as \textitbf{da} ‘goodbye’. Some of them also have independent uses in Papuan Malay, such as \textitbf{bahaya} ‘be dangerous’ (see the column ‘Basic meaning’ in Table  ‎5 .41). Others, by contrast, are only used as interjections, such as \textitbf{ayo} ‘come on!’. Remarkably, many secondary interjections are loan words, such as \textitbf{bahaya} ‘great!, be dangerous’ (Sanskrit), \textitbf{mama} ‘oh boy, mother’ (Dutch), or \textitbf{sip} ‘that’s fine’ (English).


\begin{stylecaption}
\label{bkm:Ref362360780}Table ‎5.\stepcounter{Table}{\theTable}:  Papuan Malay secondary interjections
\end{stylecaption}

\tablehead{
 Item & Gloss & Basic meaning & \arraybslash Source language\\
}
\begin{tabular}{llll}
\lsptoprule
\textitbf{bahaya} & ‘great!’ & ‘be dangerous’ & Sanskrit\\
\textitbf{damay} & ‘my goodness’ & ‘peace’ & \\
\textitbf{mama} & ‘oh boy!’ & ‘mother’ & Dutch\\
\textitbf{sialang} & ‘damn it!’ & ‘bad luck’ & \\
\textitbf{sunggu} & ‘good grief!’ & ‘be true’ & \\
\textitbf{tobat} & ‘go to hell’ & ‘repent’ & Arabic\\
\textitbf{tolong} & ‘please!’ & ‘help!’ & \\
\textitbf{amin} & ‘amen’ &  & Arabic\\
\textitbf{ayo} & ‘come on!’ &  & \\
\textitbf{da} & ‘goodbye’ &  & Dutch\\
\textitbf{enta} & ‘who knows’ &  & \\
\textitbf{haleluya} & ‘hallelujah’ &  & Hebrew via Dutch\\
\textitbf{halow} & ‘hello’ &  & Dutch\\
\textitbf{shalom} & ‘peace be with you!’ &  & Hebrew via Dutch\\
\textitbf{sori} & ‘excuse me!’ &  & English\\
\textitbf{sip} & ‘that’s fine!’ &  & English\\
\textitbf{trima-kasi} & ‘thank you!’ &  & \\
\lspbottomrule
\end{tabular}

Examples of secondary interjections are presented in (0) to (0).


\begin{styleExampleTitle}
Secondary interjections
\end{styleExampleTitle}

\begin{tabular}{lllllllllll}
\lsptoprule
\label{bkm:Ref358462101}
\gll {\bluebold{damay},} {sa} {bulang} {oktober} {sa} {pu} {alpa} {cuma} {dua} {saja}\\ %
& peace & \textsc{1sg} & month & October & \textsc{1sg} & \textsc{poss} & be.absent & just & two & just\\
\lspbottomrule
\end{tabular}
\ea
\glt 
‘\bluebold{my goodness!}, in October I, I had just only two absences’ \textstyleExampleSource{[081023-004-Cv.0014]}
\z

\begin{tabular}{lllllllllllll}
\lsptoprule
(\stepcounter{}{\the}) & sa & \multicolumn{2}{l}{bilang,} & \multicolumn{2}{l}{o} & \multicolumn{2}{l}{\bluebold{sunggu}} & ini & kalo & Hendro & ini & de\\
& \textsc{1sg} & \multicolumn{2}{l}{say} & \multicolumn{2}{l}{oh!} & \multicolumn{2}{l}{be.true} & \textsc{d.prox} & if & Hendro & \textsc{d.prox} & \textsc{3sg}\\
& \multicolumn{2}{l}{su} & \multicolumn{2}{l}{angkat} & \multicolumn{2}{l}{deng} & \multicolumn{6}{l}{piring}\\
& \multicolumn{2}{l}{already} & \multicolumn{2}{l}{lift} & \multicolumn{2}{l}{with} & \multicolumn{6}{l}{plate}\\
\lspbottomrule
\end{tabular}
\ea
\glt 
‘I said, ‘oh \bluebold{good grief!}, what’s-his-name, as for this Hendro, he would already have taken (all the cake) with the plate’ \textstyleExampleSource{[081011-005-Cv.0028]}
\z

\begin{tabular}{lllll}
\lsptoprule
\label{bkm:Ref358462103}
\gll {kasi} {nasi} {suda,} {\bluebold{ayo}}\\ %
& give & cooked.rice & already & come.on!\\
\lspbottomrule
\end{tabular}
\ea
\glt
‘give me rice!, \bluebold{come on!}’ \textstyleExampleSource{[080922-001a-CvPh.1208]}
\end{styleFreeTranslEngxvpt}

\subsection{Onomatopoeia}

Papuan Malay has a large set of onomatopoeic words which serve to imitate the natural sounds associated with their referents. Quite a few of the onomatopoeic words presented in Table  ‎5 .42 emulate the sound of a sudden percussion, such as \textitbf{cekkk} ‘wham’. Other words are \textitbf{fuuu} ‘fooo’ which imitates the sound of blowing air, or \textitbf{piiip} ‘beep’ which emulates the blowing of a horn.


\begin{stylecaption}
\label{bkm:Ref358466532}Table ‎5.\stepcounter{Table}{\theTable}:  Papuan Malay onomatopoeic words
\end{stylecaption}

\tablehead{
 Item & \arraybslash Semantics\\
}
\begin{tabular}{ll}
\lsptoprule
\textitbf{cekkk} & Sound of a heavy blow\\
\textitbf{dederet} & Sound of a drum\\
\textitbf{fuuu} & Sound of blowing air\\
\textitbf{kkkhkh} & Sound of an object falling or collapsing with a dull or heavy sound\\
\textitbf{mmmuat} & Sound of kissing\\
\textitbf{ngying-ngyaung} & Sound of a cockatoo calling\\
\textitbf{pak, tak, tang, wreeek} & Sound of banging, of a punch to the jaw, or of colliding bodies, slamming objects\\
\textitbf{piiip} & Sound of blowing a horn\\
\textitbf{syyyt} & Sound of an object moving through air or water\\
\textitbf{srrrt} & Sound of pulling, tearing or cutting\\
\textitbf{ssst} & Sound of vomiting\\
\textitbf{tak} & Sound of knocking\\
\textitbf{tpf} & Sound of spitting out a mouthful of liquid\\
\textitbf{trrrt} & Sound of running feet\\
\textitbf{wruaw} & Sound of heavy breathing or suffocation\\
\textitbf{wuuu} & Sound of shouting\\
\lspbottomrule
\end{tabular}

Examples of onomatopoeic words in context are presented in (0) to (0).


\begin{tabular}{lllllllll}
\lsptoprule
\label{bkm:Ref358466481}
\gll {sa} {ayung} {dia} {tiga} {kali,} {\bluebold{pak}} {\bluebold{pak}} {\bluebold{pak}}\\ %
& \textsc{1sg} & hit & \textsc{3sg} & three & time & bang! & bang! & bang!\\
\lspbottomrule
\end{tabular}
\ea
\glt 
‘I hit him three times, \bluebold{bang!}, \bluebold{bang!}, \bluebold{bang!}’ \textstyleExampleSource{[080923-010-CvNP.0018]}
\z

\begin{tabular}{lllllllllllll}
\lsptoprule
(\stepcounter{}{\the}) & … & \multicolumn{2}{l}{kitong} & \multicolumn{2}{l}{liat,} & uy & cahaya & \multicolumn{2}{l}{\bluebold{syyyt}} & de & datang & sperti\\
&  & \multicolumn{2}{l}{we} & \multicolumn{2}{l}{see} & boy! & glow & \multicolumn{2}{l}{swish} & \textsc{3sg} & come & similar.to\\
& \multicolumn{2}{l}{lampu} & \multicolumn{2}{l}{itu} & \multicolumn{4}{l}{petromaks} & \multicolumn{4}{l}{itu}\\
& \multicolumn{2}{l}{lamp} & \multicolumn{2}{l}{\textsc{d.dist}} & \multicolumn{4}{l}{kerosene.lantern} & \multicolumn{4}{l}{\textsc{d.dist}}\\
\lspbottomrule
\end{tabular}
\ea
\glt 
‘[when the evil spirit comes from afar,] we see, oh boy!, a glow, \bluebold{swish!}, he/she comes (with a noise) like that, what’s-its-name, kerosene pressure lantern’ \textstyleExampleSource{[081006-022-CvEx.0153]}
\z

\begin{tabular}{llllllllllll}
\lsptoprule
\label{bkm:Ref358466485}
\gll {de} {\multicolumn{2}{l}{pegang}} {di} {\multicolumn{2}{l}{batang}} {leher} {baru} {de} {ramas} {tete,}\\ %
& \textsc{3sg} & \multicolumn{2}{l}{hold} & at & \multicolumn{2}{l}{stick} & neck & and.then & \textsc{3sg} & press & grandfather\\
& \multicolumn{2}{l}{tete} & \multicolumn{3}{l}{\bluebold{wruaw}} & \multicolumn{6}{l}{\bluebold{wruaw}}\\
& \multicolumn{2}{l}{grandfather} & \multicolumn{3}{l}{wheeze!} & \multicolumn{6}{l}{wheeze!}\\
\lspbottomrule
\end{tabular}
\ea
\glt 
‘he held (grandfather) by (his) throat, and then he pressed grandfather(’s throat and) grandfather (went) ‘\bluebold{wheeze!}, \bluebold{wheeze!}’’ \textstyleExampleSource{[081015-001-Cv.0012/0014]}
\z


Across languages, Onomatopoeic words belong to the larger class of idiophones which “report an extralinguistic event like a sound, a smell, a taste, a visual impression, a movement, or a psychic emotion” {(Kilian-Hatz 2006: 510)}. As for the Papuan Malay corpus, however, extralinguistic events other than the onomatopoeic sound imitations presented in Table  ‎5 .42 have not been identified.
\end{styleBodyxvafter}

\section{Variation in word class membership}
\label{bkm:Ref351641485}
Papuan Malay has variation in word class membership between (1) verbs and nouns, (2) verbs and adverbs, (3) verbs and conjunctions, (4) verbs and prepositions, and (5) prepositions and conjunctions.



Cross-linguistically, the shift of word categories occurs quite commonly. Generally speaking, it “is a unidirectional process; that is, it leads from less grammatical to more grammatical forms and constructions” {(Heine and Kuteva 2002: 4)}. Or in other words, “the shift from major categories to minor ones (N {\textgreater} Preposition/Conjunction, V {\textgreater} Auxiliary/Preposition) is much more frequent crosslinguistically than its opposite”, as {\citet[133]{Wischer2006}} points out.
\end{styleBodyvafter}


Therefore, in discussing variation in Papuan Malay word class membership between verbs and adverbs, verbs and prepositions, and verbs and conjunctions, the verbs are taken as the source forms from which the respective adverbs, prepositions, and conjunctions derived. As for variation between prepositions and conjunctions, {Heine and \citet[4]{Kuteva2002}} note that cross-linguistically “[p]repositions often develop into conjunctions”. Very likely, this observation also applies to the variation between prepositions and conjunctions in Papuan Malay. The dual membership of lexemes as verbs and nouns, however, is less clear-cut, as discussed in Paragraph 1 below.
\end{styleBodyvxafter}

%\setcounter{itemize}{0}
\begin{itemize}
\item \begin{styleOvNvwnext}
\label{bkm:Ref353554588}Verbs and nouns (see §5.2 and §5.3)
\end{styleOvNvwnext}\end{itemize}

A number of lexemes have dual membership as verbs and nouns. So far, 41 such lexemes have been identified, some of which are listed in Table  ‎5 .43, together with the token frequencies of their uses as verbs and nouns.



The identified lexemes fall into two groups. First, verbs and their associated instrument, result, patient, agent, or location nouns. The corpus contains 32 such verb-noun pairs. In most cases, the verb is bivalent (29 verbs), while only few are monovalent (3 verbs). Table  ‎5 .43 presents eight of these verb-noun pairs. The first four lexemes are most often used as verbs, that is, \textitbf{gambar} ‘draw’, \textitbf{jalang} ‘walk’, \textitbf{jubi} ‘bow shoot’, and \textitbf{skola} ‘go to school’. The remaining four lexemes are most often used as nouns, that is, \textitbf{dayung} ‘paddle’, \textitbf{musu} ‘enemy’, \textitbf{pana} ‘arrow’, and \textitbf{senter} ‘flashlight’.
\end{styleBodyvafter}


The second group of lexemes with dual membership are affixed items: two items suffixed with \-\textitbf{ang} and four prefixed with \textscItalBold{pe(n)\-}. Structurally, the six lexemes are nouns. In their actual uses, however, four of them are (more) often used as verbs (for a detailed discussion on affixation see §3.1).
\end{styleBodyvvafter}

\begin{stylecaption}
\label{bkm:Ref353523891}Table ‎5.\stepcounter{Table}{\theTable}:  Variation in word class membership between nouns and verbs
\end{stylecaption}

\tablehead{ & \multicolumn{3}{l}{ \textsc{verb}} &  & \multicolumn{3}{l}{ \textsc{noun}}\\
 Item & Gloss &  & \# & {\textgreater}/{\textless} & Gloss &  & \arraybslash \#\\
}
\begin{tabular}{llllllll}
\lsptoprule
\textitbf{gambar} & ‘draw’ & \textsc{v.bi} & \raggedleft 21 & {\textgreater} & ‘drawing’ & \textsc{res} & \raggedleft\arraybslash 2\\
\textitbf{jalang} & ‘walk’ & \textsc{v.mo} & \raggedleft 398 & {\textgreater} & ‘road’ & \textsc{loc} & \raggedleft\arraybslash 71\\
\textitbf{jubi} & ‘bow shoot’ & \textsc{v.bi} & \raggedleft 20 & {\textgreater} & ‘bow and arrow’ & \textsc{ins} & \raggedleft\arraybslash 14\\
\textitbf{skola} & ‘go to school’ & \textsc{v.mo} & \raggedleft 148 & {\textgreater} & ‘school’ & \textsc{loc} & \raggedleft\arraybslash 94\\
\textitbf{dayung} & ‘paddle’ & \textsc{v.bi} & \raggedleft 3 & {\textless} & ‘paddle’ & \textsc{ins} & \raggedleft\arraybslash 8\\
\textitbf{musu} & ‘hate’ & \textsc{v.bi} & \raggedleft 3 & {\textless} & ‘enemy’ & \textsc{pat} & \raggedleft\arraybslash 7\\
\textitbf{pana} & ‘bow shoot’ & \textsc{v.bi} & \raggedleft 13 & {\textless} & ‘arrow’ & \textsc{ins} & \raggedleft\arraybslash 39\\
\textitbf{senter} & ‘light with flashlight’ & \textsc{v.bi} & \raggedleft 5 & {\textless} & ‘flashlight’ & \textsc{ins} & \raggedleft\arraybslash 11\\
\textitbf{jualang} & ‘sell’ & \textsc{v.bi} & \raggedleft 7 & {\textgreater} & ‘merchandise’ & \textsc{pat} & \raggedleft\arraybslash 1\\
\textitbf{latiang} & ‘practice’ & \textsc{v.bi} & \raggedleft 12 & {\textgreater} & ‘practice’ & \textsc{pat} & \raggedleft\arraybslash 5\\
\textitbf{pamalas} & ‘be very list\-less’ & \textsc{v.mo} & \raggedleft 12 & {\textgreater} & ‘lazy person’ & \textsc{agt} & \raggedleft\arraybslash 2\\
\textitbf{panakut} & ‘be very fear\-ful (of)’ & \textsc{v.bi} & \raggedleft 2 & {\textgreater} & ‘coward’ & \textsc{agt} & \raggedleft\arraybslash 1\\
\textitbf{pandiam} & ‘be very quiet’\footnotemark{} & \textsc{v.mo} & \raggedleft (1) & OR & ‘taciturn person’ & \textsc{agt} & \raggedleft\arraybslash (1)\\
\textitbf{pencuri} & ‘steal (\textsc{emph})’ & \textsc{v.bi} & \raggedleft 5 & {\textless} & ‘thief’ & \textsc{agt} & \raggedleft\arraybslash 7\\
\lspbottomrule
\end{tabular}
\footnotetext{\\
\\
\\
\\
\\
\\
\\
\\
\\
\\
\\
\\
\\
The corpus includes only one token of \textitbf{pandiam} ‘taciturn person / be very quiet’; its reading is ambiguous, that is, it can receive a verbal or a nominal reading (see example (0) in §3.1.4.2, p. \pageref{bkm:Ref439954517}).\\
\\
}
\begin{itemize}
\item \begin{styleOvNvwnextxvbefore}
Verbs and adverbs (see §5.3 and §5.4)
\end{styleOvNvwnextxvbefore}\end{itemize}

Some verbs also have adverbial function. Five such lexemes have been identified so far, as listed in Table  ‎5 .44. All but one of them are more often used as adverbs than as verbs. The exception is bivalent \textitbf{coba} ‘try’ which is more often used as a verb and less often as an evaluative modal adverb (§5.4.4).


\begin{stylecaption}
\label{bkm:Ref439839853}Table ‎5.\stepcounter{Table}{\theTable}:  Variation in word class membership between verbs and adverbs
\end{stylecaption}

\tablehead{ & \multicolumn{3}{l}{ Source form: \textsc{verb}} & {\textgreater} & \multicolumn{2}{l}{ Derived form: \textsc{adv}}\\
 Item & Gloss &  & \# &  & Gloss & \arraybslash \#\\
}
\begin{tabular}{lllllll}
\lsptoprule
\textitbf{baru} & ‘be new’ & \textsc{v.mo} & \raggedleft 24 & {\textless} & ‘recently’ & \raggedleft\arraybslash 66\\
\textitbf{dulu} & ‘be prior’ & \textsc{v.mo} & \raggedleft 63 & {\textless} & ‘in the past, first’ & \raggedleft\arraybslash 286\\
\textitbf{pas} & ‘be exact’ & \textsc{v.mo} & \raggedleft 26 & {\textless} & ‘precisely’ & \raggedleft\arraybslash 110\\
\textitbf{skarang} & ‘be current’ & \textsc{v.mo} & \raggedleft 21 & {\textless} & ‘now’ & \raggedleft\arraybslash 282\\
\textitbf{coba} & ‘try’ & \textsc{v.bi} & \raggedleft 36 & {\textgreater} & ‘if only’ & \raggedleft\arraybslash 14\\
\lspbottomrule
\end{tabular}

In addition, the corpus includes six adverbs which are reduplicated verbs: \textitbf{baru{\Tilde}baru} ‘just now’, \textitbf{kira{\Tilde}kira} ‘probably’, \textitbf{lama{\Tilde}lama} ‘gradually’, \textitbf{muda{\Tilde}mudaang} ‘hopefully’, and \textitbf{taw{\Tilde}taw} ‘suddenly’. Their respective base words are \textitbf{baru} ‘be new’, \textitbf{kira} ‘think’, \textitbf{lama} ‘be long (of duration), \textitbf{muda} ‘be easy’, and \textitbf{taw} ‘know’


\begin{itemize}
\item \begin{styleOvNvwnextxvbefore}
Verbs and conjunctions (see §5.3 and §5.12)
\end{styleOvNvwnextxvbefore}\end{itemize}

Some verbs are zero-derived into the conjunction class, namely four monovalent stative and three bivalent verbs, as listed in Table  ‎5 .45. Again, the lexemes differ in terms of the relative token frequencies of the source forms and the derived conjunctional forms. For the first three items, the verbal source forms have higher token frequencies, whereas the last four lexemes are predominantly used as conjunctions.


\begin{stylecaption}
\label{bkm:Ref353523894}Table ‎5.\stepcounter{Table}{\theTable}:  Variation in word class membership between verbs and conjunctions
\end{stylecaption}

\tablehead{ & \multicolumn{3}{l}{ Source form: \textsc{verb}} & {\textgreater} & \multicolumn{2}{l}{ Derived form: \textsc{cnj}}\\
 Item & Gloss &  & \# &  & Gloss & \arraybslash \#\\
}
\begin{tabular}{lllllll}
\lsptoprule
\textitbf{biar} & ‘let’ & \textsc{v.bi} & \raggedleft 67 & {\textgreater} & ‘although’ & \raggedleft\arraybslash 39\\
\textitbf{habis} & ‘be used up’ & \textsc{v.mo} & \raggedleft 48 &  & ‘after all’ & \raggedleft\arraybslash 21\\
\textitbf{sama} & ‘be same’ & \textsc{v.mo} & \raggedleft 60 & {\textgreater} & ‘to’ & \raggedleft\arraybslash 8\\
\textitbf{baru} & ‘be new’ & \textsc{v.mo} & \raggedleft 24 & {\textless} & ‘and then, after all’ & \raggedleft\arraybslash 986\\
\textitbf{jadi} & ‘become’ & \textsc{v.bi} & \raggedleft 173 & {\textless} & ‘so, since’ & \raggedleft\arraybslash 1,213\\
\textitbf{sampe} & ‘reach’ & \textsc{v.bi} & \raggedleft 251 & {\textless} & ‘until’ & \raggedleft\arraybslash 257\\
\textitbf{trus} & ‘be continuous’ & \textsc{v.mo} & \raggedleft 70 & {\textless} & ‘next’ & \raggedleft\arraybslash 396\\
\lspbottomrule
\end{tabular}
\begin{itemize}
\item \begin{styleOvNvwnextxvbefore}
Verbs and prepositions (see §5.3 and §5.11)
\end{styleOvNvwnextxvbefore}\end{itemize}

One preposition is derived from a monovalent verb and two from bivalent verbs:


\begin{itemize}
\item \begin{styleIIndented}
The goal preposition \textitbf{sama} ‘to’ is derived from monovalent \textitbf{sama} ‘be same’.
\end{styleIIndented}\item \begin{styleIIndented}
The benefactive preposition \textitbf{buat} ‘for’ is derived from bivalent \textitbf{buat} ‘make’.
\end{styleIIndented}\end{itemize}
\begin{itemize}
\item \begin{styleIiI}
The temporal preposition \textitbf{sampe} ‘until’ is derived from bivalent \textitbf{sampe} ‘reach’.
\end{styleIiI}\end{itemize}
%\setcounter{itemize}{0}
\begin{itemize}
\item \begin{styleOvNvwnext}
Prepositions and conjunctions (see §5.11 and §5.12)
\end{styleOvNvwnext}\end{itemize}

Six Papuan Malay prepositions are also used as conjunctions:


\begin{itemize}
\item \begin{styleIIndented}
Temporal \textitbf{sampe} ‘until’ also functions as a conjunction that introduces temporal or result clauses.
\end{styleIIndented}\item \begin{styleIIndented}
Comitative \textitbf{dengang} ‘with’ and goal preposition \textitbf{sama} ‘to’ also function as conjunctions that combine noun phrases; occasionally, \textitbf{dengang} ‘with’ also links verb phrases.
\end{styleIIndented}\item \begin{styleIIndented}
Benefactive \textitbf{untuk} ‘for’ also functions as a conjunction that introduces purpose clauses.
\end{styleIIndented}\end{itemize}
\begin{itemize}
\item \begin{styleIiI}
Similative \textitbf{sperti} ‘similar to’ and \textitbf{kaya} ‘like’ also function as conjunctions that introduce simulative clauses.
\end{styleIiI}\end{itemize}

Papuan Malay displays variation in word class membership, most of which involves verbs. Overall, the observed variation corresponds to similar processes observed cross-linguistically, in that it involves a shift of word categories from major ones to minor ones (see {Heine and Kuteva 2002: 4;} {Wischer 2006: 133}). The exception is the dual membership of lexemes as verbs and nouns, which is typical, though, for Malay varieties and other western Austronesian languages.
\end{styleBodyxvafter}

\section{Summary}

In Papuan Malay, the main criteria for defining distinct word classes are their syntactic properties, due to the lack of inflectional morphology and the rather limited productivity of derivational patterns. Three open and a number of closed lexical classes can be distinguished. The open word classes are nouns, verbs, and adverbs. The major closed word classes are personal pronouns, interrogatives, demonstratives, locatives, numerals, quantifiers, prepositions, and conjunctions. At the same time, however, Papuan Malay has membership overlap between a number of categories, most of which involve verbs. This includes overlap between verbs and nouns which is typical of Malay varieties and other western Austronesian languages. However, nouns, verbs, and adverbs have distinct syntactic properties which warrant their analysis as distinct word classes.



Papuan Malay nouns and verbs are distinct in terms of the following syntactic properties: (a) nouns canonically function as heads in noun phrases and as arguments in verbal clauses; (b) verbs canonically function as predicates and have valency; (c) nouns are negated with \textitbf{bukang} ‘\textsc{neg}’, whereas verbs are negated with \textitbf{tida}/\textitbf{tra} ‘\textsc{neg}’; (d) only nouns can be quantified with numerals and quantifiers; and (e) only verbs occur as predicates in comparative constructions, and in reciprocal constructions. Based on their syntactic properties, nouns are divided into four groups, namely common, proper, location, and direction nouns. Verbs fall into four groups, namely trivalent, bivalent, monovalent dynamic and monovalent stative verbs which have partially distinct and partially overlapping properties. The four groups of verbs can be distinguished in terms of two main criteria which also account for most of their other properties, namely their valency and their function which is mainly predicative.
\end{styleBodyvafter}


Adverbs are distinct from nouns and verbs in that adverbs, unlike nouns and verbs, (a) cannot be used predicatively; and (b) cannot modify nouns. Overall, adverbs are most closely related to verbs; some adverbs, however, are more closely linked with nouns than with verbs. Within the clause, adverbs can take different positions. The semantic effects of these positions are yet to be investigated, however.
\end{styleBodyvafter}


Personal pronouns, interrogatives, demonstratives, and locatives are distinct from nouns in that (a) all four of them can modify nouns, while the opposite does not hold; and (b) in adnominal possessive constructions, personal pronouns and interrogatives only take the possessor slot while nouns also take the possessum slot. Personal pronouns, interrogatives, and demonstratives are distinct in that (a) personal pronouns express number and person, while interrogatives and demonstratives do not; (b) personal pronouns indicate definiteness, while demonstratives signal specificity; (c) only interrogatives can express indefinite referents; and (d) only demonstratives can be stacked. Demonstratives are distinct from locatives, in that demonstratives (a) are used as independent nominals in unembedded noun phrases while locatives always occur in prepositional phrases; (b) can take the possessor or the possessum slot in adnominal possessive constructions while locatives do no occur in these constructions; and (c) can be stacked.
\end{styleBodyvafter}

%\setcounter{page}{1}\chapter[Personal pronouns]{Personal pronouns}
\label{bkm:Ref354490894}
This chapter describes the personal pronoun system in Papuan Malay. Generally speaking, personal pronouns are defined as “inherent referential and definite expressions”; their main function is to signal definiteness and person-number values, whereby they allow the unambiguous identification of their referents ({Helmbrecht 2004: 26; see also Abbot 2006}).



This main function also applies to the Papuan Malay personal pronouns, (henceforth ‘pronouns’). In addition to expressing person and number values, they also mark their referents’ definiteness; the pronouns do not mark case, clusivity, gender, or politeness.
\end{styleBodyvafter}


The pronouns have the following distributional properties:
\end{styleBodyvvafter}

%\setcounter{itemize}{0}
\begin{itemize}
\item \begin{styleIIndented}
Substitution for noun phrases (pronominal uses) (§6.1).
\end{styleIIndented}\item \begin{styleIIndented}
Modification with demonstratives, locatives, numerals, quantifiers, and/or relative clauses (pronominal uses) (§6.1).
\end{styleIIndented}\item \begin{styleIvI}
Co-occurrence with noun phrases (adnominal uses): \textsc{n/np} \textsc{pro} (§6.2)\textsc{.}
\end{styleIvI}\end{itemize}

The Papuan Malay pronoun system, presented in Table  ‎6 .1, distinguishes singular and plural numbers and three persons by the person and number values in, what {\citet[3]{Daniel2011}} calls, “an unanalyzable person-number stem”. Hence, in terms of {Daniel’s (2011: 3)} typology of personal pronouns, Papuan Malay is a ‘Type 4’ language.



The pronoun system does not mark case, clusivity, gender, or politeness. Also, the third person pronouns are unrelated to the demonstratives \textitbf{ini} ‘\textsc{d.prox}’ and \textitbf{itu} ‘\textsc{d.dist}’.\footnote{\\
\\
\\
\\
\\
\\
\\
\\
\\
\\
\\
\\
\\
\par For detailed discussions of these otherwise rather common features of pronouns see the following studies: case {\citep{Bhat2007}}, clusivity {\citep{Filimonova2005}}, gender {\citep{Siewierska2011}}, politeness {\citep{Helmbrecht2011}}, third person pronouns and demonstratives {\citep{Bhat2011}}.\\
\\
}
\end{styleBodyvvafter}

\begin{stylecaption}
\label{bkm:Ref363302767}Table ‎6.\stepcounter{Table}{\theTable}:  Pronoun system with long and short forms and token frequencies
\end{stylecaption}

\begin{tabular}{llllllll} & \multicolumn{3}{l}{ Long pronoun forms} & \multicolumn{3}{l}{ Short pronoun forms} & \arraybslash Total\\
\lsptoprule
&  & \# & \% &  & \# & \% & \arraybslash \#\\
\textsc{1sg} & \textitbf{saya} & \raggedleft 1,014 & \raggedleft 23\% & \textitbf{sa} & \raggedleft 3,465 & \raggedleft \textstyleChUnderl{77\%} & \raggedleft\arraybslash 4,479\\
\textsc{2sg} & \textitbf{{}-{}-{}-} & \raggedleft {}-{}-{}- & \raggedleft {}-{}-{}- & \textitbf{ko} & \raggedleft 1,338 & \raggedleft 100\% & \raggedleft\arraybslash 1,338\\
\textsc{3sg} & \textitbf{dia} & \raggedleft 1,285 & \raggedleft 28\% & \textitbf{de} & \raggedleft 3,347 & \raggedleft \textstyleChUnderl{72\%} & \raggedleft\arraybslash 4,632\\
\textsc{1pl} & \textitbf{kitong} & \raggedleft 604 & \raggedleft 50\% & \textitbf{tong} & \raggedleft 594 & \raggedleft 50\% & \raggedleft\arraybslash 1,198\\
& \textitbf{kita} & \raggedleft 391 & \raggedleft \textstyleChUnderl{95\%} & \textitbf{ta} & \raggedleft 11 & \raggedleft 5\% & \raggedleft\arraybslash 402\\
& \textitbf{kitorang} & \raggedleft 112 & \raggedleft \textstyleChUnderl{77\%} & \textitbf{torang} & \raggedleft 34 & \raggedleft 23\% & \raggedleft\arraybslash 146\\
\textsc{2pl} & \textitbf{kamu} & \raggedleft 337 & \raggedleft \textstyleChUnderl{53\%} & \textitbf{kam} & \raggedleft 300 & \raggedleft 47\% & \raggedleft\arraybslash 637\\
\textsc{3pl} & \textitbf{dorang} & \raggedleft 464 & \raggedleft 23\% & \textitbf{dong} & \raggedleft 1,526 & \raggedleft \textstyleChUnderl{77\%} & \raggedleft\arraybslash 1,990\\
\lspbottomrule
\end{tabular}

Each pronoun has at least one long and one short form, except for second person singular \textitbf{ko} ‘\textsc{2sg}’. The token frequencies and percentages given in Table  ‎6 .1 indicate clear preferences for most of the pronoun forms (the percentages for the most frequent forms are underlined). As for the first person singular and the third person singular pronouns, the short forms are used much more often than the respective long forms: for the first person singular a total of 3,465 short form tokens (77\%) versus a total of 1,014 long form tokens (23\%) versus, and for the third person singular a total of 3,347 short form tokens (72\%) versus a total of 1,285 long form tokens (28\%). By contrast, for the first and second person plural pronouns, the long forms are used more frequently than the respective short forms, that is, for the first person plural a total of 1,107 long form tokens (63\%) versus a total of 639 short form tokens (37\%), and for the second person plural a total of 337 long form tokens (53\%) versus a total of 300 short form tokens (47\%).\footnote{\\
\\
\\
\\
\\
\\
\\
\\
\\
\\
\\
\\
\\
\label{bkm:Ref435099242}\par First person plural: Alternatively, one could argue that long \textitbf{kitong} ‘\textsc{1pl}’ and \textitbf{kitorang} ‘\textsc{1pl}’ and short \textitbf{tong} ‘\textsc{1pl}’ and \textitbf{torang} ‘\textsc{1pl}’ are not distinct forms but allomorphs. As for short \textitbf{ta} ‘\textsc{1pl}’, one could argue that, given its low token numbers, this is not a phonologically distinct form but the result of a phonetic deletion of the first syllable. {(U. Tadmor, p.c. 2013)}\\
Second person plural: In addition, the corpus contains one token of an alternative long form, namely \textitbf{kamorang} ‘\textsc{2pl}’. Its origins are yet to be established.\\
} These distributional distinctions are not grammatically determined. Instead they represent speaker preferences which are discussed in more detail in the following two sections.\footnote{\\
\\
\\
\\
\\
\\
\\
\\
\\
\\
\\
\\
\\
\\
\par A topic for further investigation is whether these distributional distinctions are possibly phonologically determined.\\
}
\end{styleBodyaftervbefore}


Papuan Malay pronouns very often co-occur with nouns or noun phrases, as shown in (0). This chapter argues that ‘\textsc{pro} \textsc{np}’ constructions in which a pronoun precedes a noun or noun phrase, as in \textitbf{ko }[\textitbf{sungay ko}] ‘you, [you river]’, constitute appositional constructions, with the pronouns having pronominal function. ‘\textsc{np} \textsc{pro}’ constructions in which the pronoun follows a noun or noun phrase, as in \textitbf{sungay ko} ‘you river’, by contrast, are analyzed as noun phrases with adnominally used pronouns in post-head position. To demonstrate this distinction, appositional ‘\textsc{pro} \textsc{np}’ constructions and adnominal ‘\textsc{np} \textsc{pro}’ are discussed in some detail in §6.1.6 and §6.2, respectively.
\end{styleBodyvxafter}

\begin{tabular}{lllllllllll}
\lsptoprule
\label{bkm:Ref357668613}
\gll {…} {tida} {perna} {dia} {liat,} {\bluebold{ko}} {\bluebold{sungay}} {\bluebold{ko}} {bisa} {terbuka}\\ %
&  & \textsc{neg} & once & \textsc{3sg} & see & \textsc{2sg} & river & \textsc{2sg} & be.able & be.opened\\
& \multicolumn{10}{l}{begini}\\
& \multicolumn{10}{l}{like.this}\\
\lspbottomrule
\end{tabular}
\ea
\glt 
[Seeing the ocean for the first time:] ‘[never before has he seen, what, a river that is so very big like this ocean,] never before has he seen \bluebold{you}, \bluebold{you river} can be wide like this?’ \textstyleExampleSource{[080922-010a-CvNF.0212-0213]}\footnote{\\
\\
\\
\\
\\
\\
\\
\\
\\
\\
\\
\\
\\
\\
\par Addressing a non-speech-act participant such as \textitbf{sungay} ‘river’ with second person \textitbf{ko} ‘\textsc{2sg}’ serves as a rhetorical figure of speech (for details see ‘‘np 2sg’ noun phrases as rhetorical figures of speech (“apostrophes”)’ in §6.2.1.1).\\
}
\z


The following sections discuss the pronouns in more detail. Their pronominal uses are examined in §6.1, and their adnominal uses in §6.2. The main points of this chapter are summarized in §6.3.
\end{styleBodyxvafter}

\section{Pronominal uses}
\label{bkm:Ref352678462}
This section explores three major aspects with respect to the pronominal uses of the pronouns: (1) the distribution of the long and short pronoun forms within the clause (§6.1.1), (2) their modification (§6.1.2), and (3) their uses in different constructions, namely adnominal possessive constructions (§6.1.3), inclusory conjunction constructions (§6.1.4), summary conjunction constructions (§6.1.5), and appositional constructions (§6.1.6).
\end{styleBodyxvafter}

\subsection{Distribution of personal pronouns within the clause}
\label{bkm:Ref351452338}
Regarding the distribution of the long and short pronoun forms within the clause, two topics are examined in more detail: (1) the syntactic slots that the pronouns take (§6.1.1.1), and (2) their positions within the clause (§6.1.1.2).
\end{styleBodyxvafter}

\paragraph[Personal pronouns in different syntactic slots]{Personal pronouns in different syntactic slots}
\label{bkm:Ref352671470}
Both the long and the short pronoun forms occur in all syntactic positions within the clause, as illustrated in Table  ‎6 .2 to Table  ‎6 .5.\footnote{\\
\\
\\
\\
\\
\\
\\
\\
\\
\\
\\
\\
\\
\\
\par The free translations in Table  ‎6 .2 to Table  ‎6 .5 are taken from the glossed recorded texts. Therefore, the tenses may vary; likewise, the translations for \textitbf{dia}/\textitbf{de} ‘\textsc{3sg}’ vary.\\
} In the corpus, all long pronoun forms can take the subject, direct object, and oblique object slots. Only one form is unattested: in double-object constructions \textitbf{kita} ‘\textsc{1pl}’ is unattested in a direct object slot. As for the short pronoun forms, all of them are attested for the subject slot. For the direct object slot, however, speakers much more often use the long rather than the short forms. This distinction in distribution is even more pronounced for the oblique object slot. As a result, not all short pronoun forms are attested in these positions. These preferences interrelate with the distributional patterns of the pronouns within the clause, as discussed in detail in §6.1.1.2.



Table  ‎6 .2 shows the uses of the pronouns in the subject slot.
\end{styleBodyvvafter}

\begin{stylecaption}
\label{bkm:Ref363302522}Table ‎6.\stepcounter{Table}{\theTable}:  Pronouns in the subject slot\footnote{\\
\\
\\
\\
\\
\\
\\
\\
\\
\\
\\
\\
\\
\\
\par Documentation: Long pronoun forms – 081006-025-CvEx.0006, 080917-003b-CvEx.0017, 080916-001-CvNP.0004, 081006-022-CvEx.0116, 080917-008-NP.0113, 080919-004-NP.0033, 081115-001a-Cv.0160, 081011-023-Cv.0296; short pronoun forms – 080916-001-CvNP.0001, 080916-001-CvNP.0004, 081029-005-Cv.0007, 080917-008-NP.0113, 080919-004-NP.0036, 081011-022-Cv.0242, 081015-005-NP.0039.\\
}
\end{stylecaption}

\tablehead{
 Example & Literal translation & \arraybslash Free translation\\
}
\begin{tabular}{lll}
\lsptoprule
\multicolumn{3}{l}{Long pronoun forms}\\
\textitbfUndl{saya}\textitbf{ tidor} & \textsc{1sg} sleep & ‘\textstyleChUnderl{I} slept’\\
\textitbfUndl{ko}\textitbf{ ana mama} & \textsc{2sg} child mother & ‘\textstyleChUnderl{you}’re mama’s child’\\
\textitbfUndl{dia}\textitbf{ tertawa} & \textsc{3sg} laugh & ‘\textstyleChUnderl{he} laughed’\\
\textitbfUndl{kitorang}\textitbf{ bunu dorang} & \textsc{1pl} kill \textsc{3pl} & ‘\textstyleChUnderl{we} killed them’\\
\textitbfUndl{kitong}\textitbf{ kembali dari sana} & \textsc{1pl} return from \textsc{l.dist} & ‘\textstyleChUnderl{we} returned from there’\\
\textitbfUndl{kita}\textitbf{ jalang} & \textsc{1pl} walk & ‘\textstyleChUnderl{we} walked’\\
\textitbfUndl{kamu}\textitbf{ bisa blajar} & \textsc{2pl} be.able study & ‘\textstyleChUnderl{you} can study’\\
\textitbfUndl{dorang}\textitbf{ mara} & \textsc{3pl} feel.angry(.about) & ‘\textstyleChUnderl{they} felt angry’\\
\multicolumn{3}{l}{Short pronoun forms}\\
\textitbfUndl{sa}\textitbf{ bilang} & \textsc{1sg} say & ‘\textstyleChUnderl{I} said’\\
\textitbfUndl{de}\textitbf{ tertawa} & \textsc{3sg} laugh & ‘\textstyleChUnderl{he} laughed’\\
\textitbfUndl{torang}\textitbf{ berdoa} & \textsc{1pl} pray & ‘\textstyleChUnderl{we} prayed’\\
\textitbfUndl{tong}\textitbf{ jalang kaki} & \textsc{1pl} walk foot & ‘\textstyleChUnderl{we} walked on foot’\\
\textitbfUndl{ta}\textitbf{ potong babi} & \textsc{1pl} cut pig & ‘\textstyleChUnderl{we} cut up the pig’\\
\textitbfUndl{kam}\textitbf{ cari bapa} & \textsc{2pl} search father & ‘\textstyleChUnderl{you}’ll look for father’\\
\textitbfUndl{dong}\textitbf{ bilang} & \textsc{3pl} say & ‘\textstyleChUnderl{they} said’\\
\lspbottomrule
\end{tabular}

Table  ‎6 .3 shows the uses of the pronouns in the direct object slot in monotransitive constructions. In this position only short \textitbf{ta} ‘\textsc{1pl}’ is unattested, due to the overall low token frequencies for \textitbf{kita}/\textitbf{ta} ‘\textsc{1pl}’ (see Table  ‎6 .1; see also Footnote 181, p. \pageref{bkm:Ref435099242}; for details on monotransitive clauses see §11.1.2).


\begin{stylecaption}
\label{bkm:Ref351211269}Table ‎6.\stepcounter{Table}{\theTable}:  Pronouns in the direct object slot in monotransitive constructions\footnote{\\
\\
\\
\\
\\
\\
\\
\\
\\
\\
\\
\\
\\
\\
\par Documentation: Long pronoun forms – \textstyleExampleSource{080922-010a-CvNF.0281, }080917-007-CvHt.0005, 081025-006-Cv.0150\textstyleExampleSource{, 081110-008-CvNP.0106}, 080922-001a-CvPh.0143\textstyleExampleSource{, 081115-001a-Cv.0169, }081010-001-Cv.0161, 081006-009-Cv.0010; short pronoun forms – 081011-023-Cv.0167, 081014-016-Cv.0001\textstyleExampleSource{, 081115-001a-Cv.0283, }080925-003-Cv.0221\textstyleExampleSource{, 081025-009a-Cv.0026, }081006-009-Cv.0017.\\
}
\end{stylecaption}

\tablehead{
 Example & \arraybslash Free translation\\
}
\begin{tabular}{ll}
\lsptoprule
\multicolumn{2}{l}{Long pronoun forms}\\
\textitbf{de tanya }\textitbfUndl{saya} & ‘he asked \textstyleChUnderl{me}’\\
\textsc{3pl} ask \textsc{1sg} & \\
\textitbf{nanti guru{\Tilde}guru cari }\textitbfUndl{ko} & ‘very soon the teachers will look for \textstyleChUnderl{you}’\\
very.soon \textsc{rdp}{\Tilde}teacher search \textsc{2sg} & \\
\textitbf{sa tanya }\textitbfUndl{dia}\textitbf{ begini} & ‘I asked \textstyleChUnderl{him} like this’\\
\textsc{1sg} ask \textsc{3sg} like.this & \\
\textitbf{bapa de pukul }\textitbfUndl{kitorang}\textitbf{ di muka} & ‘father hit \textstyleChUnderl{us} in the face’\\
father \textsc{3sg} hit \textsc{1pl} at front & \\
\textitbf{dong tipu }\textitbfUndl{kitong} & ‘they cheated \textstyleChUnderl{us}’\\
\textsc{3pl} cheat \textsc{1pl} & \\
\textitbf{dong suru }\textitbfUndl{kita}\textitbf{ begitu} & ‘they order \textstyleChUnderl{us} like that’\\
\textsc{3pl} order \textsc{1pl} like.that & \\
\textitbf{sa masi tunggu }\textitbfUndl{kamu} & ‘I still wait for \textstyleChUnderl{you}’\\
\textsc{1sg} still wait \textsc{2pl} & \\
\textitbf{sa memang titip }\textitbfUndl{dorang}\textitbf{ sama tanta Defretes} & ‘I indeed left \textstyleChUnderl{them} with aunt Defretes’\\
\textsc{1sg} indeed deposit \textsc{3pl} to aunt Defretes & \\
\multicolumn{2}{l}{Short pronoun forms}\\
\textitbf{de pukul }\textitbfUndl{sa} & ‘he hit \textstyleChUnderl{me}’\\
\textsc{3sg} hit \textsc{1sg} & \\
\textitbf{sa tanya }\textitbfUndl{de}\textitbf{ begini} & ‘I asked \textstyleChUnderl{her} like this’\\
\textsc{1sg} ask \textsc{3sg} like.this & \\
\textitbf{bapa bawa }\textitbfUndl{torang} ke Biak & ‘father brought \textstyleChUnderl{us} to Biak’\\
father bring \textsc{1pl} to Biak & \\
\textitbf{dong antar }\textitbfUndl{tong}\textitbf{ sampe muara Tor} & ‘they brought \textstyleChUnderl{us} as far as the mouth of the Tor river’\\
\textsc{3pl} bring \textsc{1pl} reach river.mouth Tor & \\
\textitbf{sa tunggu }\textitbfUndl{kam} & ‘I’ll await \textstyleChUnderl{you}’\\
\textsc{1sg} wait \textsc{2pl} & \\
\textitbf{sa titip }\textitbfUndl{dong}\textitbf{ sama Defretes} & ‘I left \textstyleChUnderl{them} with Defretes’\\
\textsc{1sg} deposit \textsc{3pl} to Defretes & \\
\lspbottomrule
\end{tabular}

Table  ‎6 .4 illustrates the uses of the pronouns in a direct object slot in double-object constructions. All long forms but one are attested; the exception is \textitbf{kita} ‘\textsc{1pl}’. As for the short forms, only three are attested, namely \textitbf{sa} ‘\textsc{1sg}’, \textitbf{tong} ‘\textsc{1pl}’, and \textitbf{dong} ‘\textsc{3pl}’. (For details on double-object constructions see §11.1.3.1.)


\begin{stylecaption}
\label{bkm:Ref403202445}Table ‎6.\stepcounter{Table}{\theTable}:  Pronouns in a direct object slot in in double-object constructions{ }\footnote{\\
\\
\\
\\
\\
\\
\\
\\
\\
\\
\\
\\
\\
\\
\par Documentation: Long pronoun forms – 081006-024-CvEx.0030, 080925-003-Cv.0209, 081108-003-JR.0002, 081025-008-Cv.0145, 080919-004-NP.0061, 081011-020-Cv.0045, 081010-001-Cv.0195; short pronoun forms – \textsc{080922-001a-CvPh.1010, }080922-002-Cv.0127, 080922-001a-CvPh.0339, 081006-023-CvEx.0074.\\
}
\end{stylecaption}

\tablehead{
 Example & \arraybslash Free translation\\
}
\begin{tabular}{ll}
\lsptoprule
\multicolumn{2}{l}{Long pronoun forms}\\
\textitbf{kasi }\textitbfUndl{saya}\textitbf{ ana satu} & ‘give \textstyleChUnderl{me} a certain child!’\\
give \textsc{1sg} child one & \\
\textitbf{mama bisa kasi ijing }\textitbfUndl{ko} & ‘I (‘mother’) can give \textstyleChUnderl{you} permission’\\
mother be.able give permission \textsc{2sg} & \\
\textitbf{skarang dong kasi }\textitbfUndl{dia}\textitbf{ senter} & ‘now they give \textstyleChUnderl{him} a flashlight’\\
now \textsc{3pl} give \textsc{3sg} flashlight & \\
\textitbf{mace kasi nasihat }\textitbfUndl{kitorang} & ‘the woman gave \textstyleChUnderl{us} advice’\\
woman give advice\textsc{ 3pl} & \\
\textitbf{dia kasi }\textitbfUndl{kitong}\textitbf{ daging} & ‘he gave \textstyleChUnderl{us} meat’\\
\textsc{3sg} give \textsc{1pl} meat & \\
\textitbf{minta{\Tilde}minta }\textitbfUndl{kamu} \textitbf{uang} & ‘(who) keeps asking \textstyleChUnderl{you} for money?’\\
\textsc{rdp{\Tilde}}request \textsc{2pl} money & \\
\textitbf{baru kasi }\textitbfUndl{dorang}\textitbf{ makangang} & ‘and then (you) give \textstyleChUnderl{them} food’\\
and.then give \textsc{3pl} food & \\
\multicolumn{2}{l}{Short pronoun forms}\\
\textitbf{bli }\textitbfUndl{sa}\textitbf{ boneka} & ‘buy \textstyleChUnderl{me} a doll!’\\
buy\textsc{ 1sg} doll & \\
\textitbf{dong kasi }\textitbfUndl{tong}\textitbf{ playangang} & ‘they’ll give \textstyleChUnderl{us} a service’\\
\textsc{3pl} give \textsc{1pl} service & \\
\textitbf{bawa }\textitbfUndl{dong}\textitbf{ pakeang} & ‘(the pastors) brought \textstyleChUnderl{them} clothes’\\
come bring \textsc{3pl} clothes & \\
\hhline{-~}
\lspbottomrule
\end{tabular}

Table  ‎6 .5 shows the uses of the pronouns in the oblique object slot. In this position, only three short forms are unattested, namely \textitbf{sa} ‘\textsc{1sg}’, \textitbf{kam} ‘\textsc{2pl}’, \textitbf{dong} ‘\textsc{3pl}’.


\begin{stylecaption}
\label{bkm:Ref351211270}Table ‎6.\stepcounter{Table}{\theTable}:  Pronouns in the oblique object slot\footnote{\\
\\
\\
\\
\\
\\
\\
\\
\\
\\
\\
\\
\\
\\
\par Documentation: Long pronoun forms – 080917-008-NP.0004, 080922-010a-CvNF.0089, 080922-010a-CvNF.0061, 081006-024-CvEx.0021, 081110-006-Pr.0014, 081006-024-CvEx.0021, 080922-001a-CvPh.0010, 080918-001-CvNP.0050; short pronoun forms – 080922-010a-CvNF.0209-sa, 080922-001a-CvPh.0339-kam, 080919-006-CvNP.0011-dong.\\
}
\end{stylecaption}

\tablehead{
 Example & \arraybslash Free translation\\
}
\begin{tabular}{ll}
\lsptoprule
\multicolumn{2}{l}{Long pronoun forms}\\
\textitbf{baru dia yang ceritra }\textitbfUndl{sama saya} & ‘and then (it was) him who told this story \textstyleChUnderl{to me}’\\
and.then \textsc{3sg} \textsc{rel} tell to \textsc{1sg} & \\
\textitbf{tida bisa sa kas taw }\textitbfUndl{untuk ko} & ‘it’s impossible that I inform \textstyleChUnderl{you} (about this issue)’\\
\textsc{neg} be.able \textsc{1sg} \textsc{caus} know for \textsc{2sg} & \\
\textitbf{… yang Aris dia kasi }\textitbfUndl{sama dia itu} & ‘[Oten’s wife] (is the one) that Aris gave \textstyleChUnderl{to him}’\\
… \textsc{rel} Aris \textsc{3sg} give to \textsc{3sg} & \\
\textitbf{de minta sama Ida, }\textitbfUndl{sama kitorang} & ‘he requested (a child) from Ida, \textstyleChUnderl{from us}’\\
\textsc{3sg} request to Ida to \textsc{1pl} & \\
\textitbf{de datang kas taw }\textitbfUndl{sama kitong} & ‘he’ll come (and) inform \textstyleChUnderl{us}’\\
\textsc{3sg} come give know to \textsc{1pl} & \\
\textitbf{jadi Raymon minta }\textitbfUndl{sama kita} & ‘so Raymon requested (a child) \textstyleChUnderl{from us}’\\
so Raymon request to \textsc{1pl} & \\
\textitbf{… kasi hadia itu }\textitbfUndl{untuk kamu itu} & ‘[immediately the government] will give that gift \textstyleChUnderl{to you}’\\
… give gift\textitbf{\textmd{\textup{ }}}\textsc{d.dist}\textitbf{\textmd{\textup{ for }}}\textsc{2pl d.dist} & \\
\textitbf{baru de ceritra apa }\textitbfUndl{sama dorang}\textitbf{ ka} & ‘and then maybe she told something \textstyleChUnderl{to them}’\\
and.then \textsc{3sg} tell what to \textsc{3pl} or & \\
\multicolumn{2}{l}{Short pronoun forms}\\
\textitbf{bapa-tua itu de ceritra }\textitbfUndl{sama sa}\textitbf{ begini} & ‘that uncle, he told \textstyleChUnderl{me} like this’\\
uncle \textsc{d.dist} \textsc{3sg} tell to \textsc{1sg}\textitbf{\textmd{\textup{ like.this}}} & \\
\textitbf{… yang telpon }\textitbfUndl{sama kam}\textitbf{ dua} & ‘[very soon it’ll be uncle pastor] who’ll phone \textstyleChUnderl{you} two’\\
…\textitbf{ }\textsc{rel}\textitbf{\textmd{\textup{ phone to }}}\textsc{2pl}\textitbf{\textmd{\textup{ two}}} & \\
\textitbf{tete ini bilang }\textitbfUndl{sama dong} & ‘this grandfather spoke \textstyleChUnderl{to them}’\\
grandfather \textsc{d.prox} say to \textsc{3pl} & \\
\hhline{-~}
\lspbottomrule
\end{tabular}
\paragraph[Personal pronouns within the clause]{Personal pronouns within the clause}
\label{bkm:Ref352671471}
Concerning the syntactic slots that the pronouns take, the distributional distinctions between the long and short pronoun forms interrelate with the distributional pattern of the pronouns within the clause.



The data in the corpus show a clear preference for the ‘heavy’ long pronoun forms to occur in clause-final position, regardless of their grammatical functions. This preference does not apply to other positions. That is, in clause-initial or clause-internal position, the long and the short pronoun forms occur, regardless of their grammatical functions and their positions vis-à-vis the predicate. This observed distributional pattern is a reflection of the cross-linguistic tendency for the clause-final position to be “the preferred site for ‘heavy’ constituents” which has to do “with processing considerations” ({Butler 2003: 179}; see also {Hawkins 1983: 88–114}).
\end{styleBodyvafter}


So far 710 clauses with clause-final pronouns have been identified in the corpus. In 62 clauses, \textitbf{ko} ‘\textsc{2sg}’ takes the clause-final position; given that for the second person singular pronoun only one form exists, it is excluded from further analysis. This leaves 648 clauses with a clause-final pronoun. In almost all clauses, it is a long pronoun form that occurs in clause-final position (97\% – 630/648), as shown in Table  ‎6 .6. Only rarely, a short pronoun form occurs in this position (3\% – 18/648); two of the short forms are not attested at all in clause-final position, namely \textitbf{tong} ‘\textsc{1pl}’ and \textitbf{ta} ‘\textsc{1pl}’.
\end{styleBodyvvafter}

\begin{stylecaption}
\label{bkm:Ref351224393}Table ‎6.\stepcounter{Table}{\theTable}:  Pronouns in clause-final position
\end{stylecaption}

\tablehead{ & \multicolumn{3}{l}{ Long pronoun forms} & \multicolumn{3}{l}{ Short pronoun forms} & \arraybslash Total\\
&  & \# & \% &  & \# & \% & \arraybslash \#\\
}
\begin{tabular}{llllllll}
\lsptoprule
\textsc{1sg} & \textitbf{saya} & \raggedleft 210 & \raggedleft 97\% & \textitbf{sa} & \raggedleft 7 & \raggedleft 3\% & \raggedleft\arraybslash 217\\
\textsc{3sg} & \textitbf{dia} & \raggedleft 236 & \raggedleft 99\% & \textitbf{de} & \raggedleft 2 & \raggedleft 1\% & \raggedleft\arraybslash 238\\
\textsc{1pl} & \textitbf{kitorang} & \raggedleft 18 & \raggedleft 82\% & \textitbf{torang} & \raggedleft 4 & \raggedleft 18\% & \raggedleft\arraybslash 22\\
\textsc{1pl} & \textitbf{kitong} & \raggedleft 15 & \raggedleft 100\% & \textitbf{tong} & \raggedleft {}-{}-{}- & \raggedleft {}-{}-{}- & \raggedleft\arraybslash 15\\
\textsc{1pl} & \textitbf{kita} & \raggedleft 7 & \raggedleft 100\% & \textitbf{ta} & \raggedleft {}-{}-{}- & \raggedleft {}-{}-{}- & \raggedleft\arraybslash 14\\
\textsc{2pl} & \textitbf{kamu} & \raggedleft 49 & \raggedleft 98\% & \textitbf{kam} & \raggedleft 1 & \raggedleft 2\% & \raggedleft\arraybslash 50\\
\textsc{3pl} & \textitbf{dorang} & \raggedleft 95 & \raggedleft 96\% & \textitbf{dong} & \raggedleft 4 & \raggedleft 4\% & \raggedleft\arraybslash 99\\
& Total & \raggedleft 630 & \raggedleft 97\% &  & \raggedleft 18 & \raggedleft 3\% & \raggedleft\arraybslash 648\\
{\textsc{2sg}} & \textitbf{ko} &  &  &  &  &  & \raggedleft\arraybslash 62\\
& Total &  &  &  &  &  & \raggedleft\arraybslash 710\\
\lspbottomrule
\end{tabular}

This tendency for the clause-final position to be the preferred site for the ‘heavy’ long pronoun forms affects the choice of the pronoun form for the different object slots, as shown in (0) to (0).
\end{styleBodyaftervbefore}


In the examples in (0) to (0), the pronouns take the direct object slot in monotransitive clauses. When the direct object occurs in clause-internal position, both the long and the short pronoun forms are used, as shown with long \textitbf{dia} ‘\textsc{3sg}’ in (0) and short \textitbf{de} ‘\textsc{3sg}’ in (0). When the direct object occurs in clause-final position, speakers typically take the long pronoun form, such as \textitbf{saya} ‘\textsc{1sg}’ in (0). Only rarely do speakers employ a short pronoun form in clause-final position, such as \textitbf{sa} ‘\textsc{1sg}’ in (0).
\end{styleBodyvvafter}

\begin{styleExampleTitle}
Pronouns in the direct object slot in monotransitive clauses
\end{styleExampleTitle}

\begin{tabular}{lllllll}
\lsptoprule
\label{bkm:Ref351371891}
\gll {sa} {su} {pukul} {\bluebold{dia}} {di} {kamar}\\ %
& \textsc{1sg} & already & hit & \textsc{3sg} & at & room\\
\lspbottomrule
\end{tabular}
\ea
\glt 
‘I already hit \bluebold{her} in (her) room’ \textstyleExampleSource{[081115-001a-Cv.0271]}
\z

\begin{tabular}{lllll}
\lsptoprule
\label{bkm:Ref403393157}
\gll {sa} {tanya} {\bluebold{de}} {begini}\\ %
& \textsc{1sg} & ask & \textsc{3sg} & like.this\\
\lspbottomrule
\end{tabular}
\ea
\glt 
‘I asked her like this’ \textstyleExampleSource{[081014-016-Cv.0001]}
\z

\begin{tabular}{lllll}
\lsptoprule
\label{bkm:Ref350421572}
\gll {nanti} {ko} {kejar} {\bluebold{saya}}\\ %
& very.soon & \textsc{2sg} & chase & \textsc{1sg}\\
\lspbottomrule
\end{tabular}
\ea
\glt 
‘in a moment you chase \bluebold{me}’ \textstyleExampleSource{[080917-004-CVHT.0001]}
\z

\begin{tabular}{llllllll}
\lsptoprule
\label{bkm:Ref386816327}
\gll {dulu} {bole} {bapa} {gendong} {\bluebold{sa},} {skarang} {…}\\ %
& first & may & father & hold & \textsc{1sg} & now & \\
\lspbottomrule
\end{tabular}
\ea
\glt 
[Talking to her father:] ‘in former times you (‘father’) were allowed to hold \bluebold{me}, now …’ \textstyleExampleSource{[080922-001a-CvPh.0699]}
\z


In the examples in (0) to (0), the pronouns take a direct object slot in double-object constructions. In this position the mentioned, distributional preferences are even more pronounced. In clause-internal position, both the long and the short pronoun forms occur, such as long \textitbf{saya} ‘\textsc{1sg}’ in (0) and short \textitbf{dong} ‘\textsc{3pl}’ in (0). In clause-final position, by contrast, only the long pronoun forms are attested, such as \textitbf{dorang} ‘\textsc{3pl}’ in (0). (Double-object constructions are discussed in detail in §11.1.3.1.)


\begin{styleExampleTitle}
Pronouns in a direct object slot in double-object constructions
\end{styleExampleTitle}

\begin{tabular}{lllll}
\lsptoprule
\label{bkm:Ref403393507}
\gll {kasi} {\bluebold{saya}} {ana} {satu!}\\ %
& kasi & \textsc{1sg} & child & one\\
\lspbottomrule
\end{tabular}
\ea
\glt 
‘give \bluebold{me} a certain child!’’ \textstyleExampleSource{[081006-024-CvEx.0030]}
\z

\begin{tabular}{lllll}
\lsptoprule
\label{bkm:Ref363301368}
\gll {kaka} {kirim} {\bluebold{dong}} {uang}\\ %
& oSb & send & \textsc{2pl} & money\\
\lspbottomrule
\end{tabular}
\ea
\glt 
‘the older sibling sent \bluebold{them} money’ \textstyleExampleSource{[080922-001a-CvPh.0860]}
\z

\begin{tabular}{llllll}
\lsptoprule
\label{bkm:Ref403393511}
\gll {sa} {mulay} {kasi} {nasihat} {\bluebold{dorang}}\\ %
& \textsc{1sg} & \textsc{start} & give & advice & \textsc{2pl}\\
\lspbottomrule
\end{tabular}
\ea
\glt 
‘I started giving \bluebold{them} advice’ \textstyleExampleSource{[081115-001a-Cv.0100]}
\z


As for pronouns in oblique object slots, again both the long and the short pronoun forms are used, such as long \textitbf{dorang} ‘\textsc{3pl}’ in (0) or short \textitbf{sa} ‘\textsc{1sg}’ in (0). In clause-final position, however, typically the long pronoun forms are used, such as \textitbf{dorang} ‘\textsc{3pl}’ in (0), while short pronoun forms such as \textitbf{dong} ‘\textsc{3pl}’ in (0) are very rare.


\begin{styleExampleTitle}
Pronouns in the oblique object slot
\end{styleExampleTitle}

\begin{tabular}{llllllll}
\lsptoprule
\label{bkm:Ref357509850}
\gll {sa} {bilang} {\bluebold{sama}} {\bluebold{dorang}} {yang} {di} {kampung}\\ %
& \textsc{1sg} & speak & to & \textsc{3pl} & \textsc{rel} & at & village\\
\lspbottomrule
\end{tabular}
\ea
\glt 
‘I told \bluebold{them} who are in the village’ \textstyleExampleSource{[080919-001-Cv.0157]}
\z

\begin{tabular}{lllllll}
\lsptoprule
\label{bkm:Ref357509851}
\gll {de} {bilang} {\bluebold{sama}} {\bluebold{sa}} {begini,} {…}\\ %
& \textsc{3sg} & say & to & \textsc{1sg} & like.this & \\
\lspbottomrule
\end{tabular}
\ea
\glt 
‘he said \bluebold{to me} like this, …’ \textstyleExampleSource{[080917-008-NP.0163]}
\z

\begin{tabular}{llllllll}
\lsptoprule
\label{bkm:Ref357509852}
\gll {itu} {yang} {sa} {kas} {taw} {\bluebold{sama}} {\bluebold{dorang}}\\ %
& \textsc{d.dist} & \textsc{rel} & \textsc{1sg} & already & know & to & \textsc{3pl}\\
\lspbottomrule
\end{tabular}
\ea
\glt 
‘that (is) what I let \bluebold{them} know’ \textstyleExampleSource{[081006-009-Cv.0010]}
\z

\begin{tabular}{llllll}
\lsptoprule
\label{bkm:Ref403394473}
\gll {tete} {ini} {bilang} {\bluebold{sama}} {\bluebold{dong}}\\ %
& grandfather & \textsc{d.prox} & say & to & \textsc{3pl}\\
\lspbottomrule
\end{tabular}
\ea
\glt
‘this grandfather spoke \bluebold{to them}’ \textstyleExampleSource{[080919-006-CvNP.0011]}
\end{styleFreeTranslEngxvpt}

\subsection{Modification of personal pronouns}
\label{bkm:Ref351452340}
Pronouns are readily modified with a number of different constituents, namely demonstratives, locatives, numerals, quantifiers, prepositional phrases, and/or relative clauses, as illustrated with the examples in (0) to (0).



Proximal demonstrative \textitbf{ini} ‘\textsc{d.prox}’ modifies long \textitbf{saya} ‘\textsc{1sg}’ in (0), while distal \textitbf{itu} ‘\textsc{d.dist}’ modifies \textitbf{ko} ‘\textsc{2sg}’ in (0). In both examples, the demonstratives signal the speakers’ psychological involvement with the events talked about. In (0), distal locative \textitbf{sana} ‘\textsc{l.dist}’ modifies short \textitbf{dong} ‘\textsc{3sg}’, designating the referent’s location relative to that of the speaker. In the corpus, pronouns are quite often modified with demonstratives, while modification with locatives is rare. (For details on demonstratives and locatives and their different functions see Chapter 7.)
\end{styleBodyvvafter}

\begin{styleExampleTitle}
Modification of pronouns with demonstratives or locatives
\end{styleExampleTitle}

\begin{tabular}{llllll}
\lsptoprule
\label{bkm:Ref362511363}
\gll {jadi} {\bluebold{saya}} {\bluebold{ini}} {ana} {mas-kawing}\\ %
& so & \textsc{1sg} & \textsc{d.prox} & child & bride.price\\
\lspbottomrule
\end{tabular}
\ea
\glt 
‘so \bluebold{I (}\blueboldSmallCaps{emph}\bluebold{)} am a bride-price child’ \textstyleExampleSource{[081006-028-CvEx.0016]}
\z

\begin{tabular}{lllllllllll}
\lsptoprule
\label{bkm:Ref368297930}
\gll {a,} {ko} {ke} {laut} {dulu,} {dong} {ada} {tunggu} {\bluebold{ko}} {\bluebold{itu}}\\ %
& ah! & \textsc{2sg} & to & sea & first & \textsc{3pl} & exist & wait & \textsc{2sg} & \textsc{d.dist}\\
\lspbottomrule
\end{tabular}
\ea
\glt 
‘ah, you (go down) to the sea first, they are waiting for \bluebold{you (}\blueboldSmallCaps{emph}\bluebold{)}!’ \textstyleExampleSource{[081015-003-Cv.0003]}
\z

\begin{tabular}{lllll}
\lsptoprule
\label{bkm:Ref340861573}
\gll {\bluebold{dong}} {\bluebold{sana}} {cari} {anging}\\ %
& \textsc{3pl} & \textsc{l.dist} & search & wind\\
\lspbottomrule
\end{tabular}
\ea
\glt 
‘\bluebold{they over there} are looking for a breeze’ \textstyleExampleSource{[081025-009b-Cv.0076]}
\z


Modification with numerals typically involves the numeral \textitbf{dua} ‘two’, as with short \textitbf{tong} ‘\textsc{1pl}’ in (0), but constructions with \textitbf{tiga} ‘three’ are also found. In the corpus, modification with quantifiers is limited to universal \textitbf{smua} ‘all’ and mid-range \textitbf{brapa} ‘several’, as shown with long \textitbf{kamu} ‘\textsc{2pl}’ in (0) and long \textitbf{dorang} ‘\textsc{3pl}’ in (0), respectively. Modification with other quantifiers is also possible, though, as shown with midrange \textitbf{banyak} ‘many’ in the elicited example in (0). The examples in (0) to (0) also demonstrate that the numerals and quantifiers always occur in post-head position. That is, they cannot occur in pre-head position as illustrated with the elicited ungrammatical constructions in (0) and (0). (In this respect pronouns differ from nouns in that noun phrases with adnominally used numerals or quantifiers can have an \textsc{n-mod} or a \textsc{mod-n} structure, as discussed in §8.3.)


\begin{styleExampleTitle}
Modification of pronouns with numerals or quantifiers
\end{styleExampleTitle}

\begin{tabular}{llllllll}
\lsptoprule
\label{bkm:Ref351456680}
\gll {\bluebold{tong}} {\bluebold{dua}} {mandi,} {pas} {Nofita} {de} {datang}\\ %
& \textsc{1pl} & two & bathe & precisely & Nofita & \textsc{3sg} & come\\
\lspbottomrule
\end{tabular}
\ea
\glt 
‘\bluebold{the two of us} were bathing, at that moment Nofita came’ \textstyleExampleSource{[081025-006-Cv.0326]}
\z

\begin{tabular}{llllllll}
\lsptoprule
\label{bkm:Ref351456681}
\gll {saya} {liat} {\bluebold{kamu}} {\bluebold{smua}} {tapi} {kamu} {…}\\ %
& \textsc{1sg} & see & \textsc{2pl} & all & but & \textsc{2pl} & \\
\lspbottomrule
\end{tabular}
\ea
\glt 
‘I see \bluebold{all of you} but you …’ \textstyleExampleSource{[080921-006-CvNP.0006]}
\z

\begin{tabular}{llllll}
\lsptoprule
\label{bkm:Ref340905830}
\gll {sa} {maki} {\bluebold{dorang}} {\bluebold{brapa}} {\bluebold{itu}}\\ %
& \textsc{1sg} & abuse.verbally & \textsc{3pl} & several & \textsc{d.dist}\\
\lspbottomrule
\end{tabular}
\ea
\glt 
‘I verbally abused \bluebold{several of them there}’ \textstyleExampleSource{[080923-008-Cv.0012]}
\z

\begin{tabular}{llllll}
\lsptoprule
\label{bkm:Ref367089218}
\gll {sa} {maki} {\bluebold{dorang}} {\bluebold{banyak}} {\bluebold{itu}}\\ %
& \textsc{1sg} & abuse.verbally & \textsc{3pl} & many & \textsc{d.dist}\\
\lspbottomrule
\end{tabular}
\ea
\glt 
‘I verbally abused \bluebold{many of them there}’ \textstyleExampleSource{[Elicited BR111021.024]}
\z

\begin{tabular}{lllllllll}
\lsptoprule
\label{bkm:Ref434916704}
\gll {*} {\bluebold{dua}} {\bluebold{tong}} {mandi,} {pas} {Nofita} {de} {datang}\\ %
&  & two & \textsc{1pl} & bathe & precisely & Nofita & \textsc{3sg} & come\\
\lspbottomrule
\end{tabular}
\ea
\glt 
Intended reading: ‘\bluebold{the two of us} were bathing, at that moment Nofita came’ \textstyleExampleSource{[Elicited ME151112.001]}
\z

\begin{tabular}{lllllllll}
\lsptoprule
\label{bkm:Ref434916707}
\gll {*} {saya} {liat} {\bluebold{smua}} {\bluebold{kamu}} {tapi} {kamu} {…}\\ %
&  & \textsc{1sg} & see & all & \textsc{2pl} & but & \textsc{2pl} & \\
\lspbottomrule
\end{tabular}
\ea
\glt 
Intended reading: ‘I see \bluebold{all of you} but you …’ \textstyleExampleSource{[Elicited ME151112.002]}
\z


In the corpus, the numerals and quantifiers typically form constituents with the quantified pronouns. That is, floating numerals or quantifiers are unattested, with one exception though. Quantifier \textitbf{smua} ‘all’ can also float to a clause-final position, as shown in (0). (This observation that among the numerals and quantifiers only quantifier \textitbf{smua} ‘all’ floats also applies to the modification of nouns, as discussed in §8.3.)


\begin{styleExampleTitle}
Floating quantifier \textitbf{smua} ‘all’
\end{styleExampleTitle}

\begin{tabular}{lllllll}
\lsptoprule
\label{bkm:Ref434913625}
\gll {langsung} {mandi,} {\bluebold{kitong}} {mulay} {mandi} {\bluebold{smua}}\\ %
& immediately & bathe & \textsc{1pl} & start & bathe & all\\
\lspbottomrule
\end{tabular}
\ea
\glt 
‘[We arrived here, arrived by motorboat,] immediately (we) bathed, \bluebold{we} started \bluebold{all} bathing’ \textstyleExampleSource{[080917-008-NP.0131]}
\z


Further, pronouns can be modified with prepositional phrases as illustrated with \textitbf{dong} ‘\textsc{3pl}’ in (0), or with relative clauses as shown with short \textitbf{sa} ‘\textsc{1sg}’ in (0).


\begin{styleExampleTitle}
Modification of pronouns with prepositional phrases or relative clauses
\end{styleExampleTitle}

\begin{tabular}{lllllllll}
\lsptoprule
\label{bkm:Ref351456683}
\gll {tapi} {\bluebold{dong}} {\bluebold{di}} {\bluebold{sana}} {\bluebold{tu}} {tida} {taw} {pencuri}\\ %
& but & \textsc{3pl} & at & \textsc{l.dist} & \textsc{d.dist} & \textsc{neg} & know & thief/steal\\
\lspbottomrule
\end{tabular}
\ea
\glt 
‘but \bluebold{them over there (}\blueboldSmallCaps{emph}\bluebold{)} never steal’ (Lit. ‘don’t know to steal’) \textstyleExampleSource{[081011-022-Cv.0293]}
\z

\begin{tabular}{lllllllll}
\lsptoprule
\label{bkm:Ref351456685}
\gll {waktu} {de} {kawing} {mas-kawing} {itu} {\bluebold{sa}} {\bluebold{yang}} {\bluebold{ambil}}\\ %
& when & \textsc{3sg} & marry.unofficially & bride.price & that & \textsc{1sg} & \textsc{rel} & get\\
\lspbottomrule
\end{tabular}
\ea
\glt
‘when she marries, that bride-price, (it’s) \bluebold{me who’ll get} (it)’ \textstyleExampleSource{[081006-025-CvEx.0024]}
\end{styleFreeTranslEngxvpt}

\subsection{Personal pronouns in adnominal possessive constructions}
\label{bkm:Ref351452339}
Pronouns also take the possessor slot in adnominal possessive constructions; overall, the short forms are preferred over the long forms, as shown in Table  ‎6 .7 (the percentages for the most frequent forms are underlined).



The corpus contains a total of 1,692 adnominal possessive constructions. In 160 constructions, \textitbf{ko} ‘\textsc{2sg}’ takes the possessor slot; again, it is excluded from further analysis given that it has only one form. This leaves 1,532 adnominal possessive constructions. In 1,097 constructions the possessor slot is filled with a short pronoun (72\%) as compared to only 435 constructions (28\%) in which a long pronoun takes the possessor slot. The exception is first person plural \textitbf{kitong}/\textitbf{tong} ‘\textsc{1pl}’: speakers employ long \textitbf{kitong} ‘\textsc{1pl}’ almost as often as short \textitbf{tong} ‘\textsc{1pl}’.
\end{styleBodyvvafter}

\begin{stylecaption}
\label{bkm:Ref351392324}Table ‎6.\stepcounter{Table}{\theTable}:  Pronominally used pronouns in adnominal possessive constructions
\end{stylecaption}

\tablehead{ & \multicolumn{3}{l}{ Long pronoun forms} & \multicolumn{3}{l}{ Short pronoun forms} & \arraybslash Total\\
&  & \# & \% &  & \# & \% & \arraybslash \#\\
}
\begin{tabular}{llllllll}
\lsptoprule
\textsc{1sg} & \textitbf{saya} & \raggedleft 83 & \raggedleft 16\% & \textitbf{sa} & \raggedleft 422 & \raggedleft \textstyleChUnderl{84\%} & \raggedleft\arraybslash 505\\
\textsc{3sg} & \textitbf{dia} & \raggedleft 106 & \raggedleft 17\% & \textitbf{de} & \raggedleft 508 & \raggedleft \textstyleChUnderl{83\%} & \raggedleft\arraybslash 614\\
\textsc{1pl} & \textitbf{kitorang} & \raggedleft 9 & \raggedleft \textstyleChUnderl{90\%} & \textitbf{torang} & \raggedleft 1 & \raggedleft 10\% & \raggedleft\arraybslash 10\\
\textsc{1pl} & \textitbf{kitong} & \raggedleft 40 & \raggedleft 49\% & \textitbf{tong} & \raggedleft 42 & \raggedleft 51\% & \raggedleft\arraybslash 82\\
\textsc{1pl} & \textitbf{kita} & \raggedleft 17 & \raggedleft \textstyleChUnderl{93\%} & \textitbf{ta} & \raggedleft 1 & \raggedleft 7\% & \raggedleft\arraybslash 29\\
\textsc{2pl} & \textitbf{kamu} & \raggedleft 12 & \raggedleft 27\% & \textitbf{kam} & \raggedleft 32 & \raggedleft \textstyleChUnderl{73\%} & \raggedleft\arraybslash 44\\
\textsc{3pl} & \textitbf{dorang} & \raggedleft 8 & \raggedleft 8\% & \textitbf{dong} & \raggedleft 91 & \raggedleft \textstyleChUnderl{92\%} & \raggedleft\arraybslash 99\\
& Total & \raggedleft 435 & \raggedleft 28\% &  & \raggedleft 1,097 & \raggedleft \textstyleChUnderl{72\%} & \raggedleft\arraybslash 1,532\\
{\textsc{2sg}} & \textitbf{ko} &  &  &  &  &  & \raggedleft\arraybslash 160\\
& Total &  &  &  &  &  & \raggedleft\arraybslash 1,692\\
\lspbottomrule
\end{tabular}

In (0), one possessive construction is presented in context with long \textitbf{dia} ‘\textsc{3sg}’ taking the possessor slot in (0). (For a detailed discussion of adnominal possessive constructions see Chapter 9.)


\begin{tabular}{lllllllllll}
\lsptoprule
\label{bkm:Ref351393879}
\gll {nanti} {\bluebold{dia}} {\bluebold{pu}} {\bluebold{maytua}} {tanya,} {ko} {dapat} {ikang} {di} {mana}\\ %
& very.soon & \textsc{3sg} & \textsc{poss} & wife & ask & \textsc{2sg} & get & fish & at & where\\
\lspbottomrule
\end{tabular}
\ea
\glt
‘later \bluebold{his wife} will ask, ‘where did you get the fish?’’ \textstyleExampleSource{[080919-004-NP.0062]}
\end{styleFreeTranslEngxvpt}

\subsection{Personal pronouns in inclusory conjunction constructions}
\label{bkm:Ref288671633}
Papuan Malay also employs plural pronouns in inclusory conjunction constructions, such that ‘\textsc{pro-pl} \textitbf{(dua) dengang} \textsc{np}’ or ‘\textsc{pro-pl} (two) with \textsc{np}’. The conjunct that designates the entire set is encoded by a plural pronoun. This conjunct is inclusory in that it “identifies a set of participants that includes the one or those referred to by the lexical noun phrase”, adopting {Lichtenberk’s (2000: 1)} terminology; hence it is an “inclusory pronoun” {(2000: 2)} or, as {\citet[33]{Haspelmath2007a}} calls it, an “inclusory conjunct”. In Papuan Malay, both conjuncts are linked by means of overt coordination with the comitative marker \textitbf{dengang} ‘with’, with its short form \textitbf{deng}. The inclusory conjunct precedes the included conjunct, as shown in (0) to (0).



Typically, the inclusory conjunct is encoded by a dual construction formed with a plural pronoun and the adnominally used numeral \textitbf{dua} ‘two’, such that ‘\textsc{pro-pl} \textitbf{dua}’. In (0), for instance, the speaker talks about herself and her husband. That is, the entire set consists of two referents with the inclusory conjunct \textitbf{tong dua} ‘we two’ including the conjunct \textitbf{bapa} ‘father’ in its reference. Only rarely is the inclusory conjunct encoded by a bare plural pronoun, as in (0). In this example, the entire set consists of the speaker, his wife, and their children, with the included conjunct \textitbf{ana{\Tilde}ana} ‘children’ being subsumed under the inclusory conjunct \textitbf{tong} ‘\textsc{1pl}’.
\end{styleBodyvvafter}

\begin{styleExampleTitle}
Plural and dual inclusory conjunction constructions with the first person plural pronoun
\end{styleExampleTitle}

\begin{tabular}{lllllllll}
\lsptoprule
\label{bkm:Ref341983671}
\gll {…} {\bluebold{tong}} {\bluebold{dua}} {\bluebold{deng}} {\bluebold{bapa}} {\bluebold{tu}} {sayang} {dia}\\ %
&  & \textsc{1pl} & two & with & father & \textsc{d.dist} & love & \textsc{3sg}\\
\lspbottomrule
\end{tabular}
\ea
\glt 
‘[but this child] \bluebold{I and (my) husband (}\blueboldSmallCaps{emph}\bluebold{)} love her’ \textstyleExampleSource{[081115-001a-Cv.0251]}
\z

\begin{tabular}{llllllll}
\lsptoprule
\label{bkm:Ref321135342}
\gll {malam} {hari} {atur} {\bluebold{tong}} {\bluebold{deng}} {\bluebold{ana{\Tilde}ana}} {makang}\\ %
& night & day & arrange & \textsc{1pl} & with & \textsc{rdp}{\Tilde}child & eat\\
\lspbottomrule
\end{tabular}
\ea
\glt 
‘in the evening (my wife) arranges (the food), \bluebold{we and the children} eat’ \textstyleExampleSource{[080919-004-NP.0007]}
\z


All three plural pronouns can take the inclusory conjunct slot, such as first person plural \textitbf{tong} ‘\textsc{1pl}’ in (0) and (0), second plural \textitbf{kam} ‘\textsc{2pl}’ in (0) and third person plural \textitbf{dong} ‘\textsc{3pl}’ in (0). Most often the included conjunct is encoded by a proper noun as in (0), or, although less frequently, by a noun phrase as in (0), or also in (0).


\begin{styleExampleTitle}
Inclusory conjunction constructions formed with the second and third person plural pronouns
\end{styleExampleTitle}

\begin{tabular}{llllllllll}
\lsptoprule
\label{bkm:Ref352599385}
\gll {\bluebold{kam}} {\bluebold{dua}} {\bluebold{deng}} {\bluebold{Isabela}} {pergi} {cek} {kapal} {di} {plabuang}\\ %
& \textsc{2pl} & two & with & Isabela & go & check & ship & at & harbor\\
\lspbottomrule
\end{tabular}
\ea
\glt 
‘\bluebold{you(}\bluebold{\textsc{sg}}\bluebold{) and Isabela} go check the ship at the harbor’ \textstyleExampleSource{[080922-001a-CvPh.0035]}
\z

\begin{tabular}{lllllllll}
\lsptoprule
\label{bkm:Ref321135340}
\gll {\multicolumn{2}{l}{\bluebold{dong}}} {\bluebold{dua}} {\bluebold{dengang}} {\bluebold{Natanael}} {\bluebold{pu}} {\bluebold{maytua}} {langsung}\\ %
& \multicolumn{2}{l}{\textsc{3pl}} & two & with & Natanael & \textsc{poss} & wife & immediately\\
& pake & \multicolumn{7}{l}{spit}\\
& use & \multicolumn{7}{l}{speedboat}\\
\lspbottomrule
\end{tabular}
\ea
\glt 
‘\bluebold{he/she and Natanael’s wife} immediately took the speedboat’ \textstyleExampleSource{[081014-008-CvNP.0006]}
\z


In addition, the corpus contains two inclusory conjunction constructions, presented in (0) and (0), in which the inclusory conjuncts are used for joining two noun phrases. Such inclusory conjunction constructions have also been described for other languages, especially in Polynesia, as {\citet{Haspelmath2007a} points out}. More specifically, {\citet[35]{Haspelmath2007a}} notes that in such a construction the “first conjunct precedes the inclusory pronoun, which is then followed by the other included conjunct(s) in the usual way”.


\begin{styleExampleTitle}
Inclusory conjunction constructions conjoining two noun phrases
\end{styleExampleTitle}

\begin{tabular}{llllllll}
\lsptoprule
\label{bkm:Ref352602477}
\gll {\bluebold{Dodo}} {\bluebold{kam}} {\bluebold{dua}} {\bluebold{deng}} {\bluebold{Waim}} {ceritrakang} {dulu}\\ %
& Dodo & \textsc{2pl} & two & with & Waim & tell & first\\
\lspbottomrule
\end{tabular}
\ea
\glt 
‘\bluebold{you (}\blueboldSmallCaps{sg}\bluebold{) Dodo and Waim} talk first’ \textstyleExampleSource{[081011-001-Cv.0001]}
\z

\begin{tabular}{llllllllllllllll}
\lsptoprule
\label{bkm:Ref352602478}
\gll {\bluebold{Tinus}} {\multicolumn{2}{l}{\bluebold{dorang}}} {\bluebold{dua}} {\multicolumn{3}{l}{\bluebold{dengang}}} {\multicolumn{2}{l}{\bluebold{Martina}}} {\multicolumn{3}{l}{\bluebold{ini},}} {dong} {dua} {lari}\\ %
& Tinus & \multicolumn{2}{l}{\textsc{3pl}} & two & \multicolumn{3}{l}{with} & \multicolumn{2}{l}{Martina} & \multicolumn{3}{l}{\textsc{d.prox}} & \textsc{3pl} & two & run\\
& \multicolumn{2}{l}{trus,} & \multicolumn{3}{l}{dorang} & dua & \multicolumn{2}{l}{lari} & \multicolumn{2}{l}{sampe} & di & \multicolumn{4}{l}{kali}\\
& \multicolumn{2}{l}{be.continuous} & \multicolumn{3}{l}{\textsc{3pl}} & two & \multicolumn{2}{l}{run} & \multicolumn{2}{l}{reach} & at & \multicolumn{4}{l}{river}\\
\lspbottomrule
\end{tabular}
\ea
\glt 
‘\bluebold{Tinus and Martina here}, the two of them drove continuously, the two of them drove all the way to the river’ \textstyleExampleSource{[081015-005-NP.0011]}
\z


The inclusory conjunction constructions presented in (0) to (0) contrast with, what {\citet[33]{Haspelmath2007a}} calls, “comitative conjunction constructions” which denote additive relations. In Papuan Malay, such comitative constructions are formed with comitative \textitbf{dengang} ‘with’. The inclusory conjunction constructions in (0) to (0) also contrast with ‘\textsc{n} \textsc{pro-pl}’ noun phrases with an associative inclusory reading. Both contrasts are illustrated with the examples in (0) and (0).



As for the distinction of comitative and inclusory conjunction constructions, \citet[33]{Haspelmath2007a} makes the following cross-linguistic observations. In a comitative conjunction construction, the conjunction of “two set-denoting \textsc{np}s […] ‘\{A, B\} and \{C, D\}’ yields the set \{A, B, C, D\}” {(2007a: 33)}. In inclusory conjunction constructions, by contrast, “some members of the second conjunct set are already included in the first conjunct set”; hence the result of the coordination is not the “union, but the \textitbf{unification}\textit{ }of the sets [such that] ‘\{A, B, C\} and \{B\}’ yields the set \{A, B, C\}” {(2007a: 33)}. This distinction also applies to Papuan Malay. While the constructions in (0) to (0) receive an inclusory reading, the comitative ‘N1 \textitbf{dengang} ‘with’ \textsc{N2}’ conjunction construction in (0) receives an additive reading. (Comitative conjunction constructions with \textitbf{dengang} ‘with’ are discussed in see §14.2.1.1.)
\end{styleBodyvvafter}

\begin{styleExampleTitle}
Comitative ‘N1 \textitbf{dengang} ‘with’ \textsc{N2}’ conjunction construction
\end{styleExampleTitle}

\begin{tabular}{lllllllll}
\lsptoprule
\label{bkm:Ref352600030}
\gll {\multicolumn{2}{l}{baru}} {siapa} {\bluebold{Sarles}} {\bluebold{dengang}} {\bluebold{dong}} {\bluebold{dua}} {turung}\\ %
& \multicolumn{2}{l}{and.then} & who & Sarles & with & \textsc{3pl} & two & descend\\
& bli & \multicolumn{7}{l}{ni}\\
& buy & \multicolumn{7}{l}{\textsc{d.prox}}\\
\lspbottomrule
\end{tabular}
\ea
\glt 
‘and then, who-is-it, \bluebold{Sarles and the two of them} came down and bought this’ \textstyleExampleSource{[081022-003-Cv.0012]}
\z


Papuan Malay inclusory conjunction constructions, that is, ‘\textsc{pro-pl} \textitbf{(dua) dengang} \textsc{np}’ constructions, are also distinct from ‘\textsc{n} \textsc{pro-pl}’ noun phrases with an associative inclusory plural reading. The pragmatic differences between the constituent order found in (0) to (0) as compared to that found in (0) are similar to the differences found in Toqabaqit, another Austronesian language, as observed by {\citep[27]{Lichtenberk2000}: the contrast }concerns “the relative degrees of discourse salience of the two sets of participants, the overtly and the covertly encoded ones”. This contrast also applies to Papuan Malay. In (0) to (0), the covertly encoded participants subsumed under the adnominal dual constructions are more salient and therefore mentioned first. The overtly encoded participants, by contrast, are less salient and therefore mentioned second. In the ‘\textsc{n} \textsc{pro-pl}’ noun phrase in (0), by contrast, the overtly encoded participant \textitbf{bapa} ‘father’ is more salient and therefore mentioned first. The covertly encoded participants subsumed under the adnominal dual construction \textitbf{dorang dua} ‘they two’ are less salient and of subordinate status. (For details on ‘\textsc{n} \textsc{pro-pl}’ noun phrases with an associative inclusory plural reading, see §6.2.2.2.)


\begin{styleExampleTitle}
‘\textsc{n} \textsc{pro-pl}’ noun phrase with an associative reading
\end{styleExampleTitle}

\begin{tabular}{llllllll}
\lsptoprule
\label{bkm:Ref321135343}
\gll {\bluebold{bapa}} {\bluebold{dorang}} {\bluebold{dua}} {pulang} {hari} {minggu} {cepat}\\ %
& father & \textsc{3pl} & two & go.home & day & Sunday & be.fast\\
\lspbottomrule
\end{tabular}
\ea
\glt
‘\bluebold{father and he} returned home quickly on Sunday’ \textstyleExampleSource{[080925-003-Cv.0163]}
\end{styleFreeTranslEngxvpt}

\subsection{Personal pronouns in summary conjunctions}
\label{bkm:Ref352741968}
The plural pronouns also occur in “summary conjunction” constructions, a term adopted from {Haspelmath’s (2007a: 36) cross-linguistic study on coordination: }following a set of conjoined noun phrases, a final constituent “sums up the set of conjuncts and thereby indicates that they belong together and that the list is complete”. According to {\citet[36]{Haspelmath2007a}}, however, this final constituent is a “numeral or quantifier”.



In Papuan Malay, by contrast, the final constituent that sums up the set of conjuncts is a plural pronoun. This is illustrated in the examples in (0) to (0). The set can consist of just two conjuncts as in (0), or of three or more as in (0). Typically the conjuncts are conjoined without an overt coordinator, as in (0) and (0). When the set of conjuncts is limited to two, as in (0), the conjuncts may also be linked with an overt coordinator, usually comitative \textitbf{dengang} ‘with’. (For details on the combining of noun phrases, see §14.2.)
\end{styleBodyvvafter}

\begin{styleExampleTitle}
Resumptive plural pronouns in summary conjunction constructions
\end{styleExampleTitle}

\begin{tabular}{llllll}
\lsptoprule
\label{bkm:Ref352658324}
\gll {\bluebold{mama}} {\bluebold{bapa}} {\bluebold{tong}} {mo} {sembayang}\\ %
& mother & father & \textsc{1pl} & want & worship\\
\lspbottomrule
\end{tabular}
\ea
\glt 
‘\bluebold{we mother and father} want to worship’ \textstyleExampleSource{[080917-003b-CvEx.0020]}
\z

\begin{tabular}{lllllllllll}
\lsptoprule
\label{bkm:Ref352658323}
\gll {…} {\bluebold{Hurki}} {e} {\bluebold{Herman}} {\bluebold{Nusa},} {em,} {\bluebold{Oktofina}} {\bluebold{kamu}} {duduk} {situ}\\ %
&  & Hurki & uh & Herman & Nusa & uh & Oktofina & \textsc{2pl} & sit & \textsc{l.med}\\
\lspbottomrule
\end{tabular}
\ea
\glt 
‘[in the evening (I said),] ‘\bluebold{you (}\blueboldSmallCaps{pl}\bluebold{) Hurki}, uh \bluebold{Herman, Nusa}, uh \bluebold{Oktofina} sit there’’ \textstyleExampleSource{[081115-001a-Cv.0085]}
\z

\begin{tabular}{lllllll}
\lsptoprule
\label{bkm:Ref352658325}
\gll {\bluebold{mama}} {\bluebold{deng}} {\bluebold{bapa}} {\bluebold{dong}} {su} {meninggal}\\ %
& mother & with & father & \textsc{3pl} & already & die\\
\lspbottomrule
\end{tabular}
\ea
\glt 
‘\bluebold{they mother and father} have already died’ \textstyleExampleSource{[080919-006-CvNP.0012]}
\z


When the number of conjuncts is limited to two, Papuan Malay speakers often employ a dual construction in which the adnominal pronoun is modified with the numeral \textitbf{dua} ‘two’ as in (0) and (0). In such a “dual conjunction” construction, a term also adopted from {\citet[36]{Haspelmath2007a}}, the conjuncts are most often conjoined with an overt coordinator, as in (0), although coordination without an overt coordinator is also possible, as in (0).


\begin{styleExampleTitle}
Resumptive plural pronouns in dual conjunction constructions
\end{styleExampleTitle}

\begin{tabular}{llllllllll}
\lsptoprule
\label{bkm:Ref352658326}
\gll {\bluebold{sa}} {\bluebold{deng}} {\bluebold{Eferdina}} {\bluebold{kitong}} {\bluebold{dua}} {pi} {berdoa} {tugu} {itu}\\ %
& \textsc{1sg} & with & Eferdina & \textsc{1pl} & two & go & pray & monument & \textsc{d.dist}\\
\lspbottomrule
\end{tabular}
\ea
\glt 
‘\bluebold{the two of us, I and Eferdina} go (and) pray over that statue’ \textstyleExampleSource{[080917-008-NP.0003]}
\z

\begin{tabular}{llllllllll}
\lsptoprule
\label{bkm:Ref352658327}
\gll {Rahab} {de} {bilang,} {\bluebold{bapa}} {\bluebold{mama}} {\bluebold{kam}} {\bluebold{dua}} {liat} {dulu}\\ %
& Rahab & \textsc{3sg} & say & father & mother & \textsc{2pl} & two & see & first\\
\lspbottomrule
\end{tabular}
\ea
\glt
‘Rahab said, \bluebold{the two of you, father} (and) \bluebold{mother}, have a look!’’ \textstyleExampleSource{[081006-035-CvEx.0044]}
\end{styleFreeTranslEngxvpt}

\subsection{Personal pronouns in appositional constructions}
\label{bkm:Ref352671416}
Pronouns very commonly occur in ‘\textsc{pro} \textsc{np}’ constructions in which a pronominally used pronoun precedes a noun or noun phrase. These constructions are analyzed as appositional constructions, with appositions being defined as “two or more noun phrases having the same referent and standing in the same syntactical relation to the rest of the sentence” {\citep[5193]{Asher1994}}. Such ‘\textsc{pro} \textsc{np}’ constructions are distinct from the ‘\textsc{np} \textsc{pro}’ constructions discussed in §6.2, in which an adnominally used pronoun follows its head nominal. To validate this distinction, appositional ‘\textsc{pro} \textsc{np}’ constructions are described in some detail in this section.



Appositions may be restrictive or nonrestrictive. In restrictive apposition, the second appositive limits or clarifies the first unit. In nonrestrictive apposition, by contrast, the second appositive is added as an optional additional piece of information {(Morley 2000: 182–188)}. The same applies to Papuan Malay ‘\textsc{pro} \textsc{np}’ appositions; that is, depending on their semantic function within the clause, the appositions may be restrictive or nonrestrictive. The referent is typically human with consultants agreeing that ‘\textsc{pro} \textsc{np}’ expressions with nonhuman referents are unacceptable. The corpus contains only one exception in which the referent is an inanimate entity, presented in (0), repeated as (0). As discussed in ‘‘np 2sg’ noun phrases as rhetorical figures of speech (“apostrophes”)’ in §6.2.1.1, however, the construction in (0) involves “a \textstyleChItalic{personification}\textit{ }of the nonhuman object that is addressed” {(Abrams and Harpham 2009: 314)}.
\end{styleBodyvafter}


Appositional ‘\textsc{pro} \textsc{np}’ constructions are formed with all persons and number; those with singular pronouns are presented in (0) to (0) and those with plural pronouns in (0) to (0). Dual constructions are also possible, as shown in (0). Appositions can be bare nouns as in (0), noun phrases with modifiers as in (0), or coordinate noun phrases as in (0). In terms of intonation, the data in the corpus does not indicate a clear pattern: the apposition can be set off from the preceding pronoun by a comma intonation (“{\textbar}”), as in (0), or can follow it with no intonation break as in (0).
\end{styleBodyvafter}


The appositional constructions with singular pronouns in (0), (0) and (0) are nonrestrictive with the appositions \textitbf{mama} ‘mother’ in (0), \textitbf{prempuang cantik} ‘beautiful woman’ in (0), and \textitbf{ana} ‘child’ in (0) providing additional optional information not needed for the identification of their pronominal referents. The constructions in (0) and (0), by contrast, are restrictive with the appositions \textitbf{sungay ko} ‘you river’ and \textitbf{Agus ni} ‘this Agus’ giving information needed for the identification of the referents ko ‘\textsc{2sg}’ and \textitbf{dia} ‘\textsc{3sg}’, respectively.
\end{styleBodyvvafter}

\begin{styleExampleTitle}
Appositions with singular pronouns: ‘\textsc{pro-sg} \textsc{np}’
\end{styleExampleTitle}

\begin{tabular}{lllllllll}
\lsptoprule
\label{bkm:Ref352667688}
\gll {…} {yo,} {akirnya} {\bluebold{sa}} {{\textbar}} {\bluebold{mama}} {berdoa} {berdoa}\\ %
&  & yes & finally & \textsc{1sg} &  & mother & pray & pray\\
\lspbottomrule
\end{tabular}
\ea
\glt 
‘[so in fifth grade she broke-off school,] yes, finally \bluebold{I}, \bluebold{(a/her) mother}, prayed (and) prayed’ \textstyleExampleSource{[081011-023-Cv.0178]}
\z

\begin{tabular}{lllllllll}
\lsptoprule
\label{bkm:Ref341895811}
\gll {kalo} {\multicolumn{2}{l}{ko}} {tida} {skola} {\bluebold{ko}} {\bluebold{prempuang}} {\bluebold{cantik}}\\ %
& if & \multicolumn{2}{l}{\textsc{2sg}} & \textsc{neg} & go.to.school & \textsc{2sg} & woman & be.beautiful\\
& \multicolumn{2}{l}{nanti} & \multicolumn{6}{l}{…}\\
& \multicolumn{2}{l}{very.soon} & \multicolumn{6}{l}{}\\
\lspbottomrule
\end{tabular}
\ea
\glt 
‘if you don’t go to school, later \bluebold{you}, \bluebold{a beautiful woman}, …’ \textstyleExampleSource{[081110-008-CvNP.0043]}
\z

\begin{tabular}{lllllllllll}
\lsptoprule
\label{bkm:Ref354558053}
\gll {…} {tida} {perna} {dia} {liat,} {\bluebold{ko}} {\bluebold{sungay}} {\bluebold{ko}} {bisa} {terbuka}\\ %
&  & \textsc{neg} & once & \textsc{3sg} & see & \textsc{2sg} & river & \textsc{2sg} & be.able & be.opened\\
& \multicolumn{10}{l}{begini}\\
& \multicolumn{10}{l}{like.this}\\
\lspbottomrule
\end{tabular}
\ea
\glt 
[Seeing the ocean for the first time:] ‘[never before has he seen, what, a river that is so very big like this ocean,] never before has he seen \bluebold{you}, \bluebold{you river} can be wide like this?’ \textstyleExampleSource{[080922-010a-CvNF.0212-0213]}
\z

\begin{tabular}{lllllllllll}
\lsptoprule
\label{bkm:Ref341895813}
\gll {dia} {tanya} {\bluebold{dia}} {{\textbar}} {\bluebold{Agus}} {\bluebold{ni},} {ko} {ada} {kapur} {ka}\\ %
& \textsc{3sg} & ask & \textsc{3sg} &  & Agus & \textsc{d.prox} & \textsc{2sg} & exist & lime & or\\
\lspbottomrule
\end{tabular}
\ea
\glt 
‘he asked \bluebold{him}, \bluebold{Agus here}, ‘do you have lime (powder)?’’ \textstyleExampleSource{[080922-010a-CvNF.0034]}
\z

\begin{tabular}{lllllllll}
\lsptoprule
\label{bkm:Ref363303009}\label{bkm:Ref363302680}
\gll {…} {tapi} {\bluebold{de}} {\bluebold{ana}} {juga} {cepat} {ikut} {terpengaru}\\ %
&  & but & \textsc{3sg} & child & also & be.fast & follow & be.influenced\\
\lspbottomrule
\end{tabular}
\ea
\glt 
‘… but \bluebold{he/she}, \bluebold{a kid}, also quickly follows (others) to be influenced’ \textstyleExampleSource{[080917-010-CvEx.0001]}
\z


Most often appositional constructions are formed with plural pronouns, such that ‘\textsc{pro-pl} \textsc{np}’. Semantically, ‘\textsc{pro-pl} \textsc{np}’ are distinct from ‘\textsc{pro-sg} \textsc{np}’ constructions in that they not only indicate the definiteness of the apposited noun phrases, but also their plurality, as shown in (0) to (0). For instance, \textitbf{pemuda} ‘youth’ in (0) or \textitbf{IPA satu} ‘Natural Science I (student)’ in (0) receive their plural reading from the preceding plural pronouns. If deemed necessary, speakers can specify the number of the apposited noun phrases with an adnominal numeral or quantifier, as in \textitbf{tiga orang itu} ‘those three people’ in (0), or in \textitbf{brapa prempuang} ‘several women’ in (0).


\begin{styleExampleTitle}
Appositions with plural pronouns: ‘\textsc{pro-pl} \textsc{np}’
\end{styleExampleTitle}

\begin{tabular}{llllll}
\lsptoprule
\label{bkm:Ref341895821}
\gll {\bluebold{tong}} {\bluebold{pemuda}} {\bluebold{ini}} {mati} {smua}\\ %
& \textsc{1pl} & youth & \textsc{d.prox} & die & all\\
\lspbottomrule
\end{tabular}
\ea
\glt 
‘\bluebold{we}, \bluebold{the young people here}, have all lost enthusiasm’ \textstyleExampleSource{[081006-017-Cv.0014]}
\z

\begin{tabular}{lllllll}
\lsptoprule
\label{bkm:Ref341895822}
\gll {tadi} {\bluebold{kam}} {\bluebold{IPA}} {\bluebold{satu}} {tra} {maing}\\ %
& earlier & \textsc{2pl} & natural.sciences & one & \textsc{neg} & play\\
\lspbottomrule
\end{tabular}
\ea
\glt 
‘earlier, \bluebold{you}, \bluebold{the Natural Science I (students)}, didn’t play’ \textstyleExampleSource{[081109-001-Cv.0162]}
\z

\begin{tabular}{lllllll}
\lsptoprule
\label{bkm:Ref341895818}
\gll {\bluebold{dong}} {\bluebold{tiga}} {\bluebold{orang}} {\bluebold{itu}} {datang} {duduk}\\ %
& \textsc{3pl} & three & person & \textsc{d.dist} & come & sit\\
\lspbottomrule
\end{tabular}
\ea
\glt 
‘\bluebold{they}, \bluebold{those three people}, came (and) sat (down)’ \textstyleExampleSource{[081006-023-CvEx.0074]}
\z

\begin{tabular}{lllllllll}
\lsptoprule
\label{bkm:Ref341895819}
\gll {…} {sa} {maki} {\bluebold{dorang}} {\bluebold{brapa}} {\bluebold{prempuang}} {di} {situ}\\ %
&  & \textsc{1sg} & abuse.verbally & \textsc{3pl} & several & woman & at & \textsc{l.med}\\
\lspbottomrule
\end{tabular}
\ea
\glt 
‘[last month,] I verbally abused \bluebold{them}, \bluebold{several women}, there’ \textstyleExampleSource{[080923-008-Cv.0001]}
\z


When the number of referents encoded by the apposited noun phrase is limited to two, Papuan Malay speakers also use dual constructions in which the pronoun is modified with the numeral \textitbf{dua} ‘two’, as in (0). In the corpus, however, such constructions are rare and the dual constructions are always formed with the third person plural pronoun.\footnote{\\
\\
\\
\\
\\
\\
\\
\\
\\
\\
\\
\\
\\
\\
\par The ‘\textsc{pro} \textsc{np}’ constructions presented in this section were analyzed as appositions. One question for further research is whether these constructions could instead be analyzed as noun phrases with pre-head pronouns. It is expected that such preposed pronouns would have an individuating function given that other pre-head determiners, namely numerals and quantifiers, also have an individuating function (see §8.3). One problem with such an analysis, however, would be ‘\textsc{pro} \textsc{np}’ constructions with singular pronouns, as in (0) to (0), given that singular pronouns would hardly have an individuating function. (For a discussion of the determiner function of post-head pronouns see §6.2.)\\
}


\begin{styleExampleTitle}
Appositions with dual constructions: ‘\textsc{pro-pl} \textitbf{dua} \textsc{np}’
\end{styleExampleTitle}

\begin{tabular}{lllllllllll}
\lsptoprule
\label{bkm:Ref341895827}
\gll {\bluebold{dorang}} {\bluebold{dua}} {\bluebold{ade}} {\bluebold{kaka}} {\bluebold{itu}} {Agus} {dengang} {Fredi} {tra} {baik}\\ %
& \textsc{3pl} & two & ySb & oSb & \textsc{d.dist} & Agus & with & Fredi & \textsc{neg} & be.good\\
\lspbottomrule
\end{tabular}
\ea
\glt
‘\bluebold{the two of them}, \bluebold{those siblings}, Agus and Fredi, are not good’ \textstyleExampleSource{[081014-003-Cv.0012]}
\end{styleFreeTranslEngxvpt}

\section{Adnominal uses}
\label{bkm:Ref352678481}
Papuan Malay pronouns are very often employed as determiners in post-head position, such that ‘\textsc{np} \textsc{pro}’. Cross-linguistically, this function of personal pronouns is rather common, as {\citet[141]{Lyons1999}} points out: they “combine with nouns to produce expressions whose reference is thereby determined in terms of the identity of the referent”; hence, they are “personal determiners”. In this function as “definite expressions”, adopting {Helmbrecht’s (2004: 26)} terminology, they indicate that the addressees are assumed to be able to identify the referent of an expression (see also {Bhat 2007: 11; Lyons 1999: 26–32; Lyons 1977: 454–455}).



It is argued here that – given the lack of inflectional person-number marking on nouns and verbs and further given the lack of definite articles –the adnominally used Papuan Malay pronouns also have this determiner function. That is, they allow the unambiguous identification of the referents as speakers or addressees, or as individuals or entities being talked about. Hence, Papuan Malay post-head pronouns are neither resumptive pronouns nor proclitic agreement markers on verbs.
\end{styleBodyvafter}


This is illustrated with the example in (0). In the ‘\textsc{np} \textsc{2sg}’ noun phrase \textitbf{Wili ko} ‘you Wili’, the second person pronoun marks the person spoken to as the intended addressee. In the ‘\textsc{np} \textsc{3sg}’ noun phrase \textitbf{tanta dia itu} ‘that aunt’ (literally ‘that she aunt’), the third person pronoun signals that the interlocutors are assumed to know the referent. The brackets indicate the constituent structure within the noun phrase. Details are discussed in §6.2.1 and §6.2.2.
\end{styleBodyvvafter}

\begin{styleExampleTitle}
‘\textsc{np} \textsc{2sg}’ and ‘\textsc{np} \textsc{3sg}’ noun phrases
\end{styleExampleTitle}

\begin{tabular}{llllllll}
\lsptoprule
\label{bkm:Ref368472118}
\gll {[\bluebold{Wili}} {\bluebold{ko}]} {jangang} {gara{\Tilde}gara} {[\bluebold{tanta}} {\bluebold{dia}} {\bluebold{itu}]}\\ %
& Wili & \textsc{2sg} & \textsc{neg.imp} & \textsc{rdp}{\Tilde}irritate & aunt & \textsc{3sg} & \textsc{d.dist}\\
\lspbottomrule
\end{tabular}
\ea
\glt 
[Addressing a young boy:] ‘\bluebold{you Wili} don’t irritate \bluebold{that aunt}!’ \textstyleExampleSource{[081023-001-Cv.0038]}
\z


Adnominal pronouns are available for all person-number values, with the exception of the first person singular. This unexpected restriction may have to do with the function of the adnominally used pronouns which is to disambiguate the participants in a speech act, as discussed in detail throughout this section. It seems that Papuan Malay presumes addressees to have difficulties in identifying first person plural, second person and third person participants. To disambiguate the referents, the respective nouns can be modified with the appropriate pronouns. With first person singular referents, however, no such difficulties are expected. Hence, such referents do not need to be disambiguated, as demonstrated with the example in (0).\footnote{\\
\\
\\
\\
\\
\\
\\
\\
\\
\\
\\
\\
\\
\\
\par See also {Bickel and Witzlack-Makarevich’s (2008: 15)} cross-linguistic study on “Referential scales and case alignment”, which shows that “first person singular is indeed often treated differently from other persons”.\\
}



The utterances in (0) are part of a conversation between a mother and her son. As the family wants to go on a trip, the son wants to obtain a leave of absence from school. He is afraid, though, that his mother will not remind him in time to ask for this leave. In trying to soothe him, his mother tells him that she will remind him in time and that she will not depart without him. In doing so, the speaker alternatively refers to herself with the noun \textitbf{mama} ‘mother’ and with first person singular \textitbf{sa} ‘\textsc{1sg}’. In this context, \textitbf{mama} ‘mother’ unambiguously refers to the speaker. Hence, there is no need to further disambiguate the referent by adding the first person singular pronoun.
\end{styleBodyvvafter}

\begin{styleExampleTitle}
Speech acts with first person singular referents
\end{styleExampleTitle}

\begin{tabular}{lllllllllllllllllllllll}
\lsptoprule
\label{bkm:Ref368471305}
\gll {hari} {\multicolumn{4}{l}{jumat}} {ko} {\multicolumn{3}{l}{mo}} {\multicolumn{2}{l}{jalang,}} {\multicolumn{4}{l}{baru}} {\multicolumn{3}{l}{\bluebold{mama}}} {kas} {\multicolumn{2}{l}{taw}} {…}\\ %
& day & \multicolumn{4}{l}{Friday} & \textsc{2sg} & \multicolumn{3}{l}{want} & \multicolumn{2}{l}{walk} & \multicolumn{4}{l}{and.then} & \multicolumn{3}{l}{mother} & give & \multicolumn{2}{l}{know} & \\
& \bluebold{sa} & \multicolumn{2}{l}{tida} & \multicolumn{4}{l}{bisa} & \multicolumn{2}{l}{kas} & \multicolumn{3}{l}{tinggal} & \multicolumn{2}{l}{ko} & \multicolumn{2}{l}{…} & hari & \multicolumn{3}{l}{jumat} & \multicolumn{2}{l}{ko}\\
& \textsc{1sg} & \multicolumn{2}{l}{\textsc{neg}} & \multicolumn{4}{l}{be.able} & \multicolumn{2}{l}{give} & \multicolumn{3}{l}{stay} & \multicolumn{2}{l}{\textsc{2sg}} & \multicolumn{2}{l}{} & day & \multicolumn{3}{l}{Friday} & \multicolumn{2}{l}{\textsc{2sg}}\\
& \multicolumn{2}{l}{mo} & \multicolumn{2}{l}{jalang,} & \multicolumn{4}{l}{baru} & \multicolumn{2}{l}{\bluebold{mama}} & \multicolumn{3}{l}{kasi} & \multicolumn{9}{l}{ingat}\\
& \multicolumn{2}{l}{want} & \multicolumn{2}{l}{walk} & \multicolumn{4}{l}{and.then} & \multicolumn{2}{l}{mother} & \multicolumn{3}{l}{give} & \multicolumn{9}{l}{remember}\\
\lspbottomrule
\end{tabular}
\ea
\glt 
‘on Friday (when) you want to go (and ask for the leave), \bluebold{I (‘mama’)} will remind (you) … \bluebold{I} cannot leave you (behind) … on Friday (when) you want to go, \bluebold{I (‘mama’)} will remind you’ \textstyleExampleSource{[080917-003b-CvEx.0011/0015/0020]}
\z


Table  ‎6 .8 gives an overview of the adnominal uses of pronouns as determiners.\footnote{\\
\\
\\
\\
\\
\\
\\
\\
\\
\\
\\
\\
\\
\\
\par The free translations in Table  ‎6 .8 are taken from the glossed texts. Therefore, the tenses may vary; likewise, the translations for \textitbf{dia}/\textitbf{de} ‘\textsc{3sg}’ vary.\\
}


\begin{stylecaption}
\label{bkm:Ref352917188}Table ‎6.\stepcounter{Table}{\theTable}:  Adnominal pronouns as determiners\footnote{\\
\\
\\
\\
\\
\\
\\
\\
\\
\\
\\
\\
\\
\\
\par Documentation: Long pronoun forms – 080923-009-Cv.0051, 081023-001-Cv.0038, 080924-001-Pr.0008, 081110-005-Pr.0107, 080923-012-CNP.0011, 080919-003-NP.0002; short pronoun forms – 081011-023-Cv.0167, 081115-001a-Cv.0001, 081006-009-Cv.0013, 081014-015-Cv.0006, 081006-024-CvEx.0043.\\
}
\end{stylecaption}

\tablehead{
 Example & \arraybslash Free translation\\
}
\begin{tabular}{ll}
\lsptoprule
\multicolumn{2}{l}{Long pronoun forms}\\
\textitbf{de bilang, a }\textitbfUndl{om ko ini} tra liat … & ‘he said, ‘a \textstyleChUnderl{you uncle here} didn’t see …’’\\
\textsc{3sg}\textitbf{\textmd{\textup{ say ah uncle }}}\textsc{2sg} \textsc{d.prox} \textsc{neg} see & \\
\textitbf{Wili ko jangang gara-gara }\textitbfUndl{tanta dia itu} & ‘you Wili don’t irritate \textstyleChUnderl{that aunt}’\\
Wili \textsc{2sg imp-neg} irritate aunt\textitbf{\textmd{\textup{ }}}\textsc{3sg}\textitbf{\textmd{\textup{ }}}\textsc{d.dist} & \\
\textitbf{jadi}\textitbfUndl{ nene kitorang ini}\textitbf{ masak} & ‘so \textstyleChUnderl{we grandmothers here} cook’\\
\textitbf{\textmd{\textup{so grandmother }}}\textsc{1pl}\textitbf{\textmd{\textup{ }}}\textsc{d.prox}\textitbf{\textmd{\textup{ cook}}} & \\
\textitbf{jadi }\textitbfUndl{laki{\Tilde}laki kitong}\textitbf{ harus bayar …} & ‘so \textstyleChUnderl{we men} have to pay …’\\
\textitbf{\textmd{\textup{so }}}\textsc{rdp}\textitbf{\textmd{\textup{{\Tilde}husband }}}\textsc{1pl}\textitbf{\textmd{\textup{ have.to pay}}} & \\
\textitbfUndl{bangsat kamu tu}\textitbf{ tinggal lari} & ‘\textstyleChUnderl{you rascals there} keep running’\\
\textitbf{\textmd{\textup{rascal }}}\textsc{2pl}\textitbf{\textmd{\textup{ }}}\textsc{d.dist}\textitbf{\textmd{\textup{ stay run}}} & \\
\textitbf{… biking malam untuk }\textitbfUndl{anjing dorang} & ‘[ the sagu porridge that my wife] had made at night for \textstyleChUnderl{the dogs}’\\
\textitbf{\textmd{\textup{make night for dog }}}\textsc{3pl} & \\
\multicolumn{2}{l}{Short pronoun forms}\\
\textitbf{… sampe }\textitbfUndl{bapa de}\textitbf{ pukul sa deng pisow} & ‘… until (my) \textstyleChUnderl{husband} hit me with a knife\\
\textitbf{\textmd{\textup{until father }}}\textsc{3sg}\textitbf{\textmd{\textup{ hit }}}\textsc{1sg}\textitbf{\textmd{\textup{ with knife}}} & \\
\textitbf{itu yang }\textitbfUndl{Lodia torang}\textitbf{ bilang …} & ‘that’s why \textstyleChUnderl{Lodia and her companions including me} said …’\\
\textsc{d.dist}\textitbf{\textmd{\textup{ }}}\textsc{rel}\textitbf{\textmd{\textup{ }}}Lodia\textitbf{\textmd{\textup{ }}}\textsc{1pl}\textitbf{\textmd{\textup{ say}}} & \\
\textitbf{… }\textitbfUndl{Pawlus tong}\textitbf{ bicara sama dia itu} & ‘\textstyleChUnderl{Pawlus and his companions including me} spoke to him (\textsc{emph})’\\
Pawlus\textsc{ 1pl}\textitbf{\textmd{\textup{ speak to }}}\textsc{3sg}\textitbf{\textmd{\textup{ }}}\textsc{d.dist} & \\
\textitbf{kamu }\textitbfUndl{ana prempuang kam}\textitbf{ latiang} & ‘you, \textstyleChUnderl{you girls} practice’\\
\textsc{2pl}\textitbf{\textmd{\textup{ child woman }}}\textsc{2pl}\textitbf{\textmd{\textup{ practice}}} & \\
\textitbf{tong biasa tanya sama }\textitbfUndl{kaka dong} & ‘we usually ask \textstyleChUnderl{the older siblings}’\\
\textsc{1pl} be.usual ask to oSb \textsc{3pl} & \\
\lspbottomrule
\end{tabular}

Some of the examples in Table  ‎6 .8 do not readily translate into English, given that “personal determiners” in English are subject to constraints concerning their person-number values {\citep[27]{Lyons1999}}. In English, only ‘we’ and ‘you (\textsc{pl})’ occur freely as determiners, while ‘you (\textsc{sg})’ occurs in exclamations only; the remaining pronouns do not have any determiner uses.\footnote{\\
\\
\\
\\
\\
\\
\\
\\
\\
\\
\\
\\
\\
\\
\par English examples are ‘we teachers’, ‘you students’, or ‘you idiot’ {(Lyons 1999: 451–442)}.\\
} Other languages, however, are less constrained. In German, for example, the first and second persons, both singular and plural, occur as determiners, while the third person does not ({Lyons 1999: 142}; see also Helmbrecht 2004: 189 for the determiner uses of personal pronouns). Along similar lines, in the Oslo dialect of Norwegian, the female third person singular pronoun functions as a determiner {\citep{Johannessen2006}}. In addition, pronouns can occur as determiners with proper names in some Germanic languages, such as German, Icelandic, and Norwegian: in German it is the first or second person singular pronouns {(Roehr 2005: 264ff)}, in Icelandic it is the third person singular and the first and second person plural pronouns {(Sigurðsson 2006: 218ff)}, and in Northern Norwegian it is the third person singular pronoun {\citep[581]{Matushansky2008}}. Still other languages are “completely unconstrained in this respect” {\citep[142]{Lyons1999}}, as for instance Warlpiri ({Hale 1973 }in {Lyons 1999: 142}).
\end{styleBodyaftervbefore}


{\citet[134]{Lyons1999}} suggests, “that personal pronouns are the pronominal counterpart of definite articles”. This is the case for Warlpiri which has “no definite article” but “a full paradigm of personal determiners” {(1999: 142, 144)}. And it is also the case for Papuan Malay which has no definite article either but an almost complete paradigm of personal determiners, the exception being the first singular person. Other Austronesian languages, by contrast, which do have a definite article also employ this article as a determiner with proper names. Examples, provided in {\citet{Campbell2000a}}, are Balinese {\citep{Kersten1948}}, Chamorro {(Topping and Ogo 1960)}, and Fijian {(Milner 1959; Schütz and Komaitai 1971)}, and, presented in {\citet{Campbell2000b}}, Malagasy {\citep{Arakin1963}}, Maori {\citep{Krupa1967}}, Minangkabau {\citep{Moussay1981}}, Tagalog ({Bowen and Philippine Center for Language Study 1965; Ramos 1971)}, Tahitian {\citep{Arakin1981}}, and Tongan {\citep{Churchward1953}}.
\end{styleBodyvafter}


As for noun phrases with adnominal pronouns in other regional Malay varieties, only limited information is available. Brief descriptions or examples are offered for Ambon Malay {(van Minde 1997)}, Balai Berkuak Malay {\citep{Tadmor2002}}, Dobo Malay {(R.J. Nivens, p.c. 2013)}, Kupang Malay {(Grimes and Jacob 2008)}, Manado Malay {\citep{Stoel2005}}, and Sri Lanka Malay {\citep{Slomanson2013}}. In each case, however, the descriptions are limited to the associative plural interpretation of ‘\textsc{np} \textsc{pro-pl}’ expressions (see §6.2.2.3). A determiner function of the pronouns is not mentioned in any of these descriptions.
\end{styleBodyvafter}


In addition, some descriptions of regional Malay varieties mention ‘\textsc{np} \textsc{pro}’ constructions, most of which are analyzed as topic-comment constructions.
\end{styleBodyvvafter}

\begin{itemize}
\item \begin{styleIIndented}
Ambon Malay: {van \citet[284]{Minde1997}} mentions constructions in which “a preposed NP is copied by a co-referential pronoun in the remainder of the clause”. In each case, the pronoun is the short third person singular \textitbf{de} ‘\textsc{3sg}’. In addition, {van \citet[285]{Minde1997}} presents examples in which a pronoun follows a noun phrase with an adnominal demonstrative at its right periphery. 
\end{styleIIndented}\item \begin{styleIIndented}
Banda Malay: {\citet[165]{Paauw2009}} gives examples of ‘\textsc{np} \textsc{pro}’ constructions which he also analyzes as topic-comment constructions. The pronoun is third person singular \textitbf{dia} ‘\textsc{3sg}’ and the preceding noun phrase is set off with an adnominal demonstrative.
\end{styleIIndented}\item \begin{styleIIndented}
Northern Moluccan Malay: {\citet[5]{Voorhoeve1983}} analyzes similar constructions as topic-comment constructions “in which the topic is cross-referenced by a pronoun subject in the comment”. Again, the pronoun is third person singular \textitbf{dia} ‘\textsc{3sg}’ and the preceding noun phrase is set off with an adnominal demonstrative.
\end{styleIIndented}\end{itemize}
\begin{itemize}
\item \begin{styleIvI}
Papuan Malay: {Paauw (2009: 166–168)} presents ‘\textsc{np} \textsc{pro}’ constructions in which the short third person forms \textitbf{de} ‘\textsc{3sg}’ and \textitbf{dong} ‘\textsc{3pl}’ occur between a subject and a verb. {\citet{Paauw2009}} analyzes these pronouns as “proclitics” that function as subject agreement markers on verbs.
\end{styleIvI}\end{itemize}

In the following sections, the adnominal uses of the pronouns are examined in detail. That is, these sections discuss the function of the pronouns to signal definiteness and person-number values, whereby they allow the unambiguous identification of the referents as speakers, addressees, or third-person participants.



The adnominal uses of the singular pronouns are discussed in §6.2.1 and those of the plural pronouns in §6.2.2. For the singular pronouns a major issue is the question whether ‘\textsc{np} \textsc{pro}’ expressions are indeed noun phrases with adnominal pronouns or whether these expressions should be analyzed as topic-comment constructions, as in other regional Malay varieties. For the plural pronouns, two interpretations of ‘\textsc{np} \textsc{pro}’ constructions are discussed, additive, and associative inclusory plurality. In giving examples for ‘\textsc{np} \textsc{pro}’ expressions, brackets are used to signal the constituent structure within the noun phrase, where deemed necessary.
\end{styleBodyvxvafter}

\subsection{Adnominal singular personal pronouns}
\label{bkm:Ref353216789}
In their determiner uses, the singular personal pronouns indicate the definiteness as well as the person and the number values, namely singularity, of their referents. ‘\textsc{np} \textsc{pro-sg}’ expressions with \textitbf{ko} ‘\textsc{2sg}’ are presented in §6.2.1.1, and those with \textitbf{dia}/\textitbf{de} ‘\textsc{3sg}’ in §6.2.1.2. In all examples given in §6.2.1.1 and §6.2.1.2, the ‘\textsc{np} \textsc{pro-sg}’ expressions constitute intonation units, unless mentioned otherwise; that is, the pronouns are not set off from their head nominals by a comma intonation. In addition, however, the corpus also contains ‘\textsc{np} \textsc{pro-sg}’ expressions in which the nouns are set off from the following pronouns by intonation; these noun phrases are briefly discussed in §6.2.1.3. Finally, §6.2.1.4 presents the reasons for analyzing ‘\textsc{np} \textsc{pro-sg}’ expressions as noun phrases with adnominal pronouns rather than as topic-comment constructions.
\end{styleBodyxvafter}

\paragraph[‘np 2sg’ noun phrases]{‘\textsc{np} \textsc{2sg}’ noun phrases}
\label{bkm:Ref357669243}\label{bkm:Ref353262282}
‘\textsc{np} \textsc{2sg}’ noun phrases have three different functions: (1) in direct speech they mark the person spoken to as the intended addressee, (2) in direct quotations they signal that the referent is the addressee of the reported speech, and (3) as rhetorical figures of speech they give an unexpected emotional impulse to a speaker’s discourse. These functions are explored one by one, followed by a summary of the syntactic and lexical properties of ‘\textsc{np} \textsc{2sg}’ noun phrases.
\end{styleBodyxvafter}

\subparagraph[‘np 2sg’ noun phrases in direct speech]{‘\textsc{np} \textsc{2sg}’ noun phrases in direct speech}

In direct speech, speakers employ ‘\textsc{np} \textsc{2sg}’ noun phrases when they want to send an unambiguous signal that the person spoken to is indeed the intended addressee. In such noun phrases, the second person \textitbf{ko} ‘\textsc{2sg}’ marks the referent encoded in the head nominal as the addressee of the utterance. The head nominal can be a common noun or a proper noun, as shown in (0) to (0).


\begin{styleExampleTitle}
‘\textsc{np} \textsc{2sg}’ noun phrases in direct speech
\end{styleExampleTitle}

\begin{tabular}{lllllllll}
\lsptoprule
\label{bkm:Ref354129601}
\gll {[\bluebold{mama-ade}} {\bluebold{ko}]} {masak} {daging} {sa} {biking} {papeda} {e?}\\ %
& aunt & \textsc{2sg} & cook & meat & \textsc{1sg} & make & sagu.porridge & eh\\
\lspbottomrule
\end{tabular}
\ea
\glt 
‘\bluebold{you aunt} cook the meat, I make the sagu porridge, eh?’ \textstyleExampleSource{[080921-001-CvNP.0073]}
\z

\begin{tabular}{llllll}
\lsptoprule
\label{bkm:Ref357521429}
\gll {[\bluebold{mace}} {\bluebold{ko}]} {rasa} {lucu} {jadi}\\ %
& woman & \textsc{2sg} & feel & be.funny & so\\
\lspbottomrule
\end{tabular}
\ea
\glt 
[Reaction to a narrative:] ‘because \bluebold{you Madam} would have felt funny’ \textstyleExampleSource{[081010-001-Cv.0206]}
\z

\begin{tabular}{llllllll}
\lsptoprule
\label{bkm:Ref440024383}
\gll {[\bluebold{Wili}} {\bluebold{ko}]} {jangang} {gara{\Tilde}gara} {tanta} {dia} {itu}\\ %
& Wili & \textsc{2sg} & \textsc{neg.imp} & \textsc{rdp}{\Tilde}irritate & aunt & \textsc{3sg} & \textsc{d.dist}\\
\lspbottomrule
\end{tabular}
\ea
\glt 
[Addressing a young boy:] ‘\bluebold{you Wili} don’t irritate that aunt!’ \textstyleExampleSource{[081023-001-Cv.0038]}
\z

\begin{tabular}{llllll}
\lsptoprule
\label{bkm:Ref353984245}
\gll {[\bluebold{Susana}} {\bluebold{ko}]} {pigi} {kaka} {cebo}\\ %
& Susana & \textsc{2sg} & go & oSb & wash.after.defecating\\
\lspbottomrule
\end{tabular}
\ea
\glt 
[Addressing her three-year old daughter:] ‘\bluebold{you Susana}, go, (your) older sister will wash (you)!’ \textstyleExampleSource{[081014-006-CvPr.0048]}
\z


When the head nominal is a common noun, second person \textitbf{ko} ‘\textsc{2sg}’ indicates which particular individual is being referred to. Thereby the pronoun allows the unambiguous identification of the addressee as the intended referent. Often speakers chose this strategy when they address an individual in a group of several interlocutors as in (0) and (0), or when they give an order to someone, as in (0) and (0).



When \textitbf{ko} ‘\textsc{2sg}’ co-occurs with a proper noun, as in (0) or (0), one might argue that such noun phrases are redundant with the pronoun as adnominal determiner being superfluous, since proper nouns are inherently definite. In Papuan Malay, however, ‘\textsc{np} \textsc{2sg}’ expressions constitute direct speech-act strategies which allow speakers to single out participants and to mark them unambiguously as the intended referents of the proper nouns. Being addressed with such a ‘\textsc{np} \textsc{2sg}’ noun phrase leaves the addressees little room for interpretation. Thereby, ‘\textsc{np} \textsc{2sg}’ nouns phrases are much more direct that then the indirect strategies presented in (0) and (0),
\end{styleBodyvafter}


Most often, speakers do not address their interlocutor with a direct ‘\textsc{np} \textsc{2sg}’ expression. Instead, they tend to use more indirect, face-preserving strategies by addressing their interlocutor with a kinship term or their proper name. This applies especially when issuing a request or an order, as shown in (0) and (0). In (0), a daughter asks her father for money by addressing him with the kinship term \textitbf{bapa} ‘father’. In (0), a father requests his daughter to talk to him by addressing her with her proper name \textitbf{Nofela}.
\end{styleBodyvvafter}

\begin{styleExampleTitle}
Indirect forms of address with bare proper noun or kinship term
\end{styleExampleTitle}

\begin{tabular}{llllll}
\lsptoprule
\label{bkm:Ref353969017}
\gll {\bluebold{bapa}} {ingat} {tong} {itu} {uang!}\\ %
& father & remember & \textsc{1pl} & \textsc{d.dist} & money\\
\lspbottomrule
\end{tabular}
\ea
\glt 
‘\bluebold{you (‘father’)} remember our, what’s-its-name, money!’ \textstyleExampleSource{[080922-001a-CvPh.0857]}
\z

\begin{tabular}{llll}
\lsptoprule
\label{bkm:Ref353969016}
\gll {\bluebold{Nofela}} {bicara} {suda!}\\ %
& Nofela & speak & already\\
\lspbottomrule
\end{tabular}
\ea
\glt
‘\bluebold{you (‘Nofela’)} speak (to me)!’ \textstyleExampleSource{[080922-001a-CvPh.0805]}
\end{styleFreeTranslEngxvpt}

\subparagraph[‘np 2sg’ noun phrases in reported speech]{‘\textsc{np} \textsc{2sg}’ noun phrases in reported speech}

Speakers also employ ‘\textsc{np} \textsc{2sg}’ noun phrases when they report direct speech. This reporting is usually done through quoting. Cross-linguistically, direct quotations serve “to dramatize and highlight important elements in a narrative”, while indirect speech “seems less vivid and colorful”, as {Bublitz and \citet[552]{Bednarek2006}} point out. The same applies to Papuan Malay, as speakers typically use quotes when reporting direct speech, as demonstrated in (0) and (0).



When relating what had been said to a particular individual, speakers usually begin the quote with an ‘\textsc{np} \textsc{2sg}’ noun phrase, as (0) and (0). This has two functions. First, it indicates the referent as the addressee of the reported speech. Second, ‘\textsc{np} \textsc{2sg}’ noun phrases mark the referent as familiar or given. Thereby they signal the hearers that they should be able to identify the referent. Subsequently, speakers continue the direct quote by referring to, or “addressing”, the referent with bare \textitbf{ko} ‘\textsc{2sg}’, as in (0). Note that the first occurrence of \textitbf{Iskia} in (0) is not part of the quote but the direct object of \textitbf{bilang} ‘say’.
\end{styleBodyvxafter}

\begin{tabular}{llllllllll}
\lsptoprule
\label{bkm:Ref353991839}
\gll {de} {bilang,} {\bluebold{Natalia}} {\bluebold{ko}} {bisa} {liat} {orang} {di} {luar?}\\ %
& \textsc{3sg} & say & Natalia & \textsc{2sg} & be.able & see & person & at & outside\\
\lspbottomrule
\end{tabular}
\ea
\glt 
[About hospitality:] ‘[(my father said to me,) ‘if you close the door, can you see the people outside?’,] he said, ‘can \bluebold{you Natalia} see the people outside?’’ \textstyleExampleSource{[081110-008-CvNP.0104]}
\z

\begin{tabular}{llllllllm{4.5984238E-4in}lllll}
\lsptoprule
\label{bkm:Ref353990718}
\gll {tong} {\multicolumn{2}{l}{dua}} {\multicolumn{2}{l}{bilang}} {Iskia,} {\multicolumn{2}{l}{\bluebold{Iskia}}} {\bluebold{ko}} {\multicolumn{2}{l}{temani,}} {\bluebold{ko}} {temani}\\ %
& \textsc{1pl} & \multicolumn{2}{l}{two} & \multicolumn{2}{l}{say} & Iskia & \multicolumn{2}{l}{Iskia} & \textsc{2sg} & \multicolumn{2}{l}{accompany} & \textsc{2sg} & accompany\\
& \multicolumn{2}{l}{karna} & \multicolumn{2}{l}{su} & \multicolumn{3}{l}{larut} & \multicolumn{3}{l}{malam} & \multicolumn{3}{l}{sedikit}\\
& \multicolumn{2}{l}{because} & \multicolumn{2}{l}{already} & \multicolumn{3}{l}{be.protracted} & \multicolumn{3}{l}{night} & \multicolumn{3}{l}{few}\\
\lspbottomrule
\end{tabular}
\ea
\glt
‘the two of us said to Iskia, ‘\bluebold{you Iskia} come with (us), \bluebold{you} come with (us) because it’s already a bit late in the evening’’ \textstyleExampleSource{[081025-006-Cv.0323/0325]}
\end{styleFreeTranslEngxvpt}

\subparagraph[‘np 2sg’ noun phrases as rhetorical figures of speech (“apostrophes”)]{‘\textsc{np} \textsc{2sg}’ noun phrases as rhetorical figures of speech (“apostrophes”)}
\label{bkm:Ref357669414}
‘\textsc{np} \textsc{2sg}’ noun phrases also serve as rhetorical figures of speech. Speakers suddenly interrupt the flow of their discourse and employ a noun phrase modified with second person \textitbf{ko} ‘\textsc{2sg}’, whereby they unexpectedly address a different audience of absent persons or nonhuman entities.



This “turning away from an audience and addressing a second audience” as a rhetorical figure of speech has been termed “apostrophe” {\citep[75]{Bussmann2000}}. Generally speaking, speakers employ apostrophes to give “a sudden emotional impetus” {(Abrams and Harpham 2009: 313)} to their discourse and thereby to create an emotional reaction in their audience. Following {\citet{Kacandes1994}}, this emotional reaction to apostrophe can be explained “by its power of calling another into being”; that is, “[t]he audience witnesses an invigoration of a being who previously was not ‘present’”. Moreover, the “[l]inguistic properties of the second-person pronoun invite the hypothesis that one also reacts strongly to apostrophe because one can so easily become the ‘you’ and thus feel oneself called into the relationship it creates” {\citep{Kacandes1994}}.
\end{styleBodyvafter}


This explanation also applies to ‘\textsc{np} \textsc{2sg}’ noun phrase apostrophes in Papuan Malay as illustrated in (0) to (0). Structurally, these utterances resemble direct quotations. Contrasting with the direct speech situations in (0) to (0), however, the addressed referents were not present when the utterances occurred. And in contrast to the reported speech situation in (0) and (0), the speakers in (0) to (0) do not relate direct quotes. Instead, they “turn away” from their audience to “address a second audience” of human or nonhuman referents.
\end{styleBodyvafter}


The example in (0) is part of a story about a fight between \textitbf{Martin} and \textitbf{Fitri}, with the speaker relating how \textitbf{Martin} attacked \textitbf{Fitri}. Notably, neither \textitbf{Martin} nor \textitbf{Fitri} were present when the speaker recounted the incident. When mentioned the first time, \textitbf{Martin} is marked as a third-person actor, as is typical of narratives with non-speech-act participants. Later in the discourse, however, \textitbf{Martin} is marked as the addressee. More specifically, the speaker first refers to \textitbf{Martin} with the ‘\textsc{np} \textsc{3sg}’ noun phrase \textitbf{Martin dia lewat} ‘Martin went past’ (literally ‘he Martin’) (see also §6.2.1.2). Next, the speaker refers to \textitbf{Martin} with the third person singular pronoun in \textitbf{de lompat} ‘he jumped’. Now \textitbf{Fitri} returns the attack and kicks \textitbf{Martin} badly. At this point, the speaker interrupts the flow of her narrative about the two non-speech-act participants and employs the ‘\textsc{np} \textsc{2sg}’ noun phrase \textitbf{Martin ko} ‘you Martin’ to relate that \textitbf{Martin} fell to the ground. In turning away from her audience and addressing absent \textitbf{Martin}, the speaker gives “emotional impetus” to the fact that \textitbf{Martin} went down after having been kicked by a woman, thereby creating an emotional reaction in her audience.
\end{styleBodyvvafter}

\begin{styleExampleTitle}
‘\textsc{np} \textsc{2sg}’ noun phrases in apostrophes: Human referents
\end{styleExampleTitle}

\begin{tabular}{llllllllllllllllllllll}
\lsptoprule
\label{bkm:Ref354054930}
\gll {\multicolumn{3}{l}{\bluebold{Martin}}} {\multicolumn{3}{l}{\bluebold{dia}}} {\multicolumn{2}{l}{lewat}} {\multicolumn{5}{l}{tete,}} {\multicolumn{2}{l}{de}} {\multicolumn{2}{l}{lompat}} {\multicolumn{2}{l}{mo}} {pukul} {Fitri}\\ %
& \multicolumn{3}{l}{Martin} & \multicolumn{3}{l}{\textsc{3sg}} & \multicolumn{2}{l}{pass.by} & \multicolumn{5}{l}{grandfather} & \multicolumn{2}{l}{\textsc{3sg}} & \multicolumn{2}{l}{jump} & \multicolumn{2}{l}{want} & hit & Fitri\\
& … & \multicolumn{3}{l}{Fitri} & \multicolumn{3}{l}{kas} & \multicolumn{3}{l}{naik} & \multicolumn{2}{l}{kaki} & di & \multicolumn{3}{l}{sini,} & \multicolumn{2}{l}{\bluebold{Martin}} & \bluebold{ko} & \multicolumn{2}{l}{jatu,}\\
&  & \multicolumn{3}{l}{Fitri} & \multicolumn{3}{l}{give} & \multicolumn{3}{l}{ascend} & \multicolumn{2}{l}{foot} & at & \multicolumn{3}{l}{\textsc{l.prox}} & \multicolumn{2}{l}{Martin} & \textsc{2sg} & \multicolumn{2}{l}{fall}\\
& \multicolumn{2}{l}{dia} & \multicolumn{3}{l}{lari} & \multicolumn{2}{l}{ke} & \multicolumn{2}{l}{mari,} & \multicolumn{2}{l}{dia} & \multicolumn{3}{l}{mo} & \multicolumn{2}{l}{pukul} & \multicolumn{5}{l}{Fitri}\\
& \multicolumn{2}{l}{\textsc{3sg}} & \multicolumn{3}{l}{run} & \multicolumn{2}{l}{to} & \multicolumn{2}{l}{hither} & \multicolumn{2}{l}{\textsc{3sg}} & \multicolumn{3}{l}{want} & \multicolumn{2}{l}{hit} & \multicolumn{5}{l}{Fitri}\\
\lspbottomrule
\end{tabular}
\ea
\glt 
[About a fight between Fitri and Martin:] ‘\bluebold{Martin} went past grandfather, he jumped (and) wanted to hit Fitri [and Fitri caught his foot and] Fitri kicked (Martin) here, \bluebold{you Martin} fell, (then) he ran (over) here, he wanted to hit Fitri’ \textstyleExampleSource{[081015-001-Cv.0018-0019]}
\z


‘\textsc{np} \textsc{2sg}’ apostrophes are also formed with nonhuman referents. They “imply a \textstyleChItalic{personification}\textit{ }of the nonhuman object that is addressed” {(Abrams and Harpham 2009: 314)}. In (0), for instance, the speaker recounts a stormy boat trip. Suddenly, she turns away from her audience to address the main protagonist \textitbf{anging} ‘wind’ with the ‘\textsc{np} \textsc{2sg}’ noun phrase \textitbf{anging ko} ‘you wind’. In the example in (0), repeated as (0), the speaker relates how one of his ancestors came down to the coast. Seeing the ocean for the first time, he mistakes it for a wide river. At this point the speaker turns away from his audience to address this \textitbf{sungay} ‘river’ with the ‘\textsc{np} \textsc{2sg}’ noun phrase \textitbf{sungay ko} ‘you river’. Note that the apostrophe is part of an appositional ‘\textsc{pro} \textsc{np}’ construction with preposed \textitbf{ko} ‘\textsc{2sg}’, such that \textitbf{ko sungay ko} ‘you, you river’ (see §6.1.6).


\begin{styleExampleTitle}
‘\textsc{np} \textsc{2sg}’ noun phrases in apostrophes: Nonhuman referents
\end{styleExampleTitle}

\begin{tabular}{llllllllll}
\lsptoprule
\label{bkm:Ref354054931}
\gll {…} {\bluebold{anging}} {\bluebold{ko}} {datang} {suda,} {hujang} {besar} {datang} {suda}\\ %
&  & wind & \textsc{2sg} & come & already & rain & be.big & come & already\\
\lspbottomrule
\end{tabular}
\ea
\glt 
[About a storm during a boat trip:] ‘\bluebold{you wind} already came up, a big rain already came up’ \textstyleExampleSource{[080917-008-NP.0137]}
\z

\begin{tabular}{lllllllllll}
\lsptoprule
\label{bkm:Ref354473261}
\gll {…} {tida} {perna} {dia} {liat,} {[\bluebold{ko}]} {[\bluebold{sungay}} {\bluebold{ko}]} {bisa} {terbuka}\\ %
&  & \textsc{neg} & once & \textsc{3sg} & see & \textsc{2sg} & river & \textsc{2sg} & be.able & be.opened\\
& \multicolumn{10}{l}{begini}\\
& \multicolumn{10}{l}{like.this}\\
\lspbottomrule
\end{tabular}
\ea
\glt
[Seeing the ocean for the first time:] ‘[never before has he seen, what, a river that is so very big like this ocean,] never before has he seen \bluebold{you, you river} can be wide like this?’ \textstyleExampleSource{[080922-010a-CvNF.0212-0213]}
\end{styleFreeTranslEngxvpt}

\subparagraph[‘np 2sg’ noun phrases and their head nominals]{‘\textsc{np} \textsc{2sg}’ noun phrases and their head nominals}

This section summarizes the syntactic and lexical properties of ‘\textsc{np} \textsc{2sg}’ noun phrases.



In the corpus, ‘\textsc{np} \textsc{2sg}’ noun phrases typically take the subject slot in clause-initial position, as in (0) to (0). There are a few exceptions, however: in (0) \textitbf{babi ko} ‘you pig’ occurs as an exclamation in clause-final position; in (0) \textitbf{kaka ko} ‘you older sibling’ denotes the possessor in an adnominal possessive construction which, in turn, takes the clausal object slot; and in (0) \textitbf{pace ko} ‘you man’ expresses the possessor in an adnominal possessive construction which, in turn, takes the complement slot in a prepositional phrase. The referent can be encoded with a common noun as in (0), a proper noun as in (0), or a noun phrase with an adnominal modifier as in (0). The referent is typically human; it can, however, also be inanimate such as \textitbf{anging} ‘wind’ in (0).
\end{styleBodyvafter}


The utterances in (0) to (0) and (0) to (0) also show that \textitbf{ko} ‘\textsc{2sg}’ is freely used as a determiner; that is, its uses are not limited to exclamations, such as \textitbf{babi … ko} ‘you … pig’ in (0) and (0).
\end{styleBodyvxafter}

\begin{tabular}{llllll}
\lsptoprule
\label{bkm:Ref353003521}
\gll {…} {dasar} {bodo} {\bluebold{babi}} {\bluebold{ko}}\\ %
&  & base & be.stupid & pig & \textsc{2sg}\\
\lspbottomrule
\end{tabular}
\ea
\glt 
‘[you (\textsc{sg}) here, do you (\textsc{sg}) have ears (or) not,] (you are of course) stupid, \bluebold{you pig}’ \textstyleExampleSource{[081014-016-Cv.0047]}
\z

\begin{tabular}{lllllll}
\lsptoprule
\label{bkm:Ref340775178}
\gll {\bluebold{babi}} {\bluebold{puti}} {\bluebold{ko}} {dari} {atas} {turung}\\ %
& pig & be.white & \textsc{2sg} & from & top & descend\\
\lspbottomrule
\end{tabular}
\ea
\glt 
[About an acquaintance:] ‘\bluebold{you white pig} came down from up (there)’ \textstyleExampleSource{[081025-006-Cv.0260]}
\z

\begin{tabular}{lllllll}
\lsptoprule
\label{bkm:Ref353003522}
\gll {sa} {taw} {\bluebold{kaka}} {\bluebold{ko}} {pu} {ruma}\\ %
& \textsc{1sg} & know & oSb & \textsc{2sg} & \textsc{poss} & house\\
\lspbottomrule
\end{tabular}
\ea
\glt 
‘I know \bluebold{you older brother}’s house’ \textstyleExampleSource{[080922-010a-CvNF.0238]}
\z

\begin{tabular}{llllllllll}
\lsptoprule
\label{bkm:Ref353192175}
\gll {nanti} {kitong} {lewat} {di} {\bluebold{pace}} {\bluebold{ko}} {pu} {kampung} {itu}\\ %
& very.soon & \textsc{1pl} & pass.by & at & man & \textsc{2sg} & \textsc{poss} & village & \textsc{d.dist}\\
\lspbottomrule
\end{tabular}
\ea
\glt 
‘later we’ll pass by \bluebold{you man}’s village there’ \textstyleExampleSource{[081012-001-Cv.0017]}
\z

\begin{tabular}{llllllllll}
\lsptoprule
\label{bkm:Ref353009429}
\gll {de} {blang,} {a,} {\bluebold{om}} {\bluebold{ko}} {\bluebold{ini}} {tra} {liat} {…}\\ %
& \textsc{3sg} & say & ah! & uncle & \textsc{2sg} & \textsc{d.prox} & \textsc{neg} & see & \\
\lspbottomrule
\end{tabular}
\ea
\glt 
‘he said, ‘ah, \bluebold{you uncle here} didn’t see …’’ \textstyleExampleSource{[080923-009-Cv.0051]}
\z

\begin{tabular}{llllll}
\lsptoprule
\label{bkm:Ref386817015}\label{bkm:Ref353011958}
\gll {\bluebold{Barce}} {\bluebold{ko}} {\bluebold{ini}} {ko} {takut}\\ %
& Barce & \textsc{2sg} & \textsc{d.prox} & \textsc{2sg} & feel.afraid(.of)\\
\lspbottomrule
\end{tabular}
\ea
\glt 
‘\bluebold{you Barce here}, you feel afraid’ \textstyleExampleSource{[081109-001-Cv.0131]}
\z

\begin{tabular}{llllllllll}
\lsptoprule
\label{bkm:Ref436750579}\label{bkm:Ref353011959}
\gll {\bluebold{Eferdina}} {\bluebold{ko}} {\bluebold{itu}} {ko} {taw} {kata} {pis} {ka} {tida}\\ %
& Eferdina & \textsc{2sg} & \textsc{d.dist} & \textsc{2sg} & know & word & please[E] & or & \textsc{neg}\\
\lspbottomrule
\end{tabular}
\ea
\glt
‘\bluebold{you Eferdina there}, do you know the word ‘please’ or not?’ \textstyleExampleSource{[081115-001a-Cv.0145]}
\end{styleFreeTranslEngxvpt}

\paragraph[‘np 3sg’ noun phrases]{‘\textsc{np} \textsc{3sg}’ noun phrases}
\label{bkm:Ref353262283}
In ‘\textsc{np} \textsc{3sg}’ noun phrases, the determiner pronoun indicates and accentuates that the speakers assume their interlocutors to know the referents, encoded by the head nominals. That is, marking referents as familiar or given, \textitbf{dia}/\textitbf{de} ‘\textsc{3sg}’ signals the hearers that they should be in a position to identify them. The determiner uses of \textitbf{dia}/\textitbf{de} ‘\textsc{3sg}’ can be situational or anaphoric. Both uses are discussed one by one, followed by a summary of the syntactic and lexical properties of ‘\textsc{np} \textsc{3sg}’ noun phrases.
\end{styleBodyxvafter}

\subparagraph[Situational uses of dia/de ‘3sg’ in ‘np 3sg’ noun phrases]{Situational uses of \textitbf{dia}/\textitbf{de} ‘\textsc{3sg}’ in ‘\textsc{np} \textsc{3sg}’ noun phrases}

In the situational uses of adnominal pronouns, “the physical situation in which the speaker and hearer are located contributes to the familiarity of the referent of the definite noun phrase” {\citep[4]{Lyons1999}}. This cross-linguistic observation also applies to the situational uses of adnominally used \textitbf{dia}/\textitbf{de} ‘\textsc{3sg}’, as illustrated in (0) to (0).



In (0), the situation is an obvious one: the hearer \textitbf{Wili} has been irritating his \textitbf{tanta} ‘aunt’ and is told to stop doing this. In (0), the speaker illustrates local bride-price customs with an example. The determiner \textitbf{de} ‘\textsc{3sg}’ marks the familiarity of the referent \textitbf{bapa} ‘father’. This in turn leads the interlocutor to interpret \textitbf{bapa} ‘father’ as the speaker’s husband. In (0), the interlocutors discuss motorbike problems. Suddenly, the speaker quotes what \textitbf{Dodo de} ‘Dodo’ (literally ‘he Dodo’) had said. \textitbf{Dodo} had not been mentioned earlier and was not present at this conversation. Determiner \textitbf{de} ‘\textsc{3sg}’, however, signals the hearers that they are familiar with the referent which, in turn, leads them to interpret the referent as the speaker’s older brother \textitbf{Dodo}.
\end{styleBodyvxafter}

\begin{tabular}{llllllll}
\lsptoprule
\label{bkm:Ref340775173}
\gll {Wili} {ko} {jangang} {gara{\Tilde}gara} {[\bluebold{tanta}} {\bluebold{dia}} {\bluebold{itu}]}\\ %
& Wili & \textsc{2sg} & \textsc{neg.imp} & \textsc{rdp}{\Tilde}irritate & aunt & \textsc{3sg} & \textsc{d.dist}\\
\lspbottomrule
\end{tabular}
\ea
\glt 
‘you Wili don’t irritate \bluebold{that aunt}’ \textstyleExampleSource{[081023-001-Cv.0038]}
\z

\begin{tabular}{lllllllllll}
\lsptoprule
\label{bkm:Ref353022698}
\gll {macang} {kalo} {[\bluebold{bapa}} {\bluebold{de}]} {kasi} {nona} {ini,} {a,} {nanti} {…}\\ %
& variety & if & father & \textsc{3sg} & give & girl & \textsc{d.prox} & ah! & very.soon & \\
\lspbottomrule
\end{tabular}
\ea
\glt 
[About bride-price children:] ‘for example, if (my) \bluebold{husband} gives this (our) girl (to our relatives), ah, later …’ \textstyleExampleSource{[081006-024-CvEx.0079]}
\z

\begin{tabular}{llllllllll}
\lsptoprule
\label{bkm:Ref353022700}
\gll {[\bluebold{Dodo}} {\bluebold{de}]} {bilang,} {adu} {coba} {ko} {kas} {taw} {sa}\\ %
& Dodo & \textsc{3sg} & say & oh.no! & if.only & \textsc{2sg} & give & know & \textsc{1sg}\\
\lspbottomrule
\end{tabular}
\ea
\glt
‘\bluebold{Dodo} said, ‘oh no, if only you had let me know’’ \textstyleExampleSource{[081014-003-Cv.0029]}
\end{styleFreeTranslEngxvpt}

\subparagraph[Anaphoric uses of dia/de ‘3sg’ in ‘np 3sg’ noun phrases]{Anaphoric uses of \textitbf{dia}/\textitbf{de} ‘\textsc{3sg}’ in ‘\textsc{np} \textsc{3sg}’ noun phrases}

In the anaphoric uses of adnominal pronouns, the referents of the definite noun phrase are “familiar not from the physical situation but from the linguistic context” {\citep[4]{Lyons1999}}, as they were mentioned earlier in the discourse. The same observation applies to the anaphoric uses of adnominally used \textitbf{dia}/\textitbf{de} ‘\textsc{3sg}’.



In Papuan Malay, when introducing new protagonists, speakers typically introduce these individuals or entities with bare common or proper nouns. At their next mention, these non-speech participants are encoded with ‘\textsc{np} \textsc{3sg}’ noun phrases, with the third person pronoun marking the referents as definite. This, in turn, signals the hearers that they are assumed to be familiar with the referents. This strategy is illustrated with the two narrative extracts in (0) and (0).\footnote{\\
\\
\\
\\
\\
\\
\\
\\
\\
\\
\\
\\
\\
\\
\par Introducing new characters with a bare noun and subsequently marking them as familiar with the adnominally used third person pronoun as a potential discourse strategy in Papuan Malay was brought to the author’s attention by {A.T. van Engelenhoven (p.c. 2013)}.\\
}
\end{styleBodyvafter}


The utterances in (0) are part of a narrative about some bad news that the speaker received from his grandmother. The speaker introduces his grandmother as a new protagonist with the bare kinship term \textitbf{nene} ‘grandmother’. This introduction involves two mentions of \textitbf{nene} ‘grandmother’; the repetition gives the speaker time to reflect who it was that had been accompanying his grandmother when they met. Following this introduction, the speaker employs the ‘\textsc{np} \textsc{3sg}’ noun phrase \textitbf{nene de} ‘grandmother’ (literally ‘she grandmother’), which marks the new character as given and familiar.
\end{styleBodyvvafter}

\begin{styleExampleTitle}
Anaphoric uses of \textitbf{dia}/\textitbf{de} ‘\textsc{3sg}’: Example \#1
\end{styleExampleTitle}

\begin{tabular}{lllllllm{-2.4015456E-4in}llllll}
\lsptoprule
\label{bkm:Ref354211131}
\gll {…} {\multicolumn{3}{l}{pas}} {ketemu} {\multicolumn{2}{l}{deng}} {sa} {\multicolumn{2}{l}{pu}} {\multicolumn{2}{l}{\bluebold{nene},}} {\bluebold{nene},}\\ %
&  & \multicolumn{3}{l}{precisely} & meet & \multicolumn{2}{l}{with} & \textsc{1sg} & \multicolumn{2}{l}{\textsc{poss}} & \multicolumn{2}{l}{grandmother} & grandmother\\
& \multicolumn{2}{l}{trus} & kaka & \multicolumn{3}{l}{laki{\Tilde}laki,} & \multicolumn{3}{l}{mama-tua} & \multicolumn{2}{l}{pu} & \multicolumn{2}{l}{ana}\\
& \multicolumn{2}{l}{next} & oSb & \multicolumn{3}{l}{\textsc{rdp}{\Tilde}husband} & \multicolumn{3}{l}{aunt} & \multicolumn{2}{l}{\textsc{poss}} & \multicolumn{2}{l}{child}\\
\lspbottomrule
\end{tabular}
\ea
\glt 
‘[I passed by (and) reached the village market there, I was sitting, standing there,] right then (I) met my \bluebold{grandmother}, \bluebold{grandmother} and then (my) older brother, aunt’s child’
\z

\begin{tabular}{llllllllll} & baru & \bluebold{nene} & \bluebold{de} & mulay & tanya & saya, & \bluebold{de} & blang & …\\
\lsptoprule
& and.then & grandmother & \textsc{3sg} & start & ask & \textsc{1sg} & \textsc{3sg} & say & \\
\lspbottomrule
\end{tabular}
\ea
\glt 
‘and then \bluebold{grandmother} started asking me, \bluebold{she} said, …’ \textstyleExampleSource{[080918-001-CvNP.0056-0057]}
\z


The utterance in (0) occurred during a narrative about a bad-mannered intruder and a young woman named \textitbf{Rahab} who observed this person’s behavior. Employing a bare proper noun, the speaker introduces \textitbf{Rahab} as a new character on the scene. At its next mention, this new protagonist is encoded by the ‘\textsc{np} \textsc{3sg}’ noun phrase \textitbf{Rahab de} ‘Rahab’ (literally ‘she Rahab’), which marks this non-speech participant as given and familiar. In the following, the speaker refers to \textitbf{Rahab} with the bare third person pronoun \textitbf{de} ‘\textsc{3sg}’.


\begin{styleExampleTitle}
Anaphoric uses of \textitbf{dia}/\textitbf{de} ‘\textsc{3sg}’: Example \#2
\end{styleExampleTitle}

\begin{tabular}{llllllllllllm{4.5984238E-4in}llll}
\lsptoprule
\label{bkm:Ref354211138}
\gll {baru} {\multicolumn{2}{l}{de}} {\multicolumn{2}{l}{luda{\Tilde}luda}} {keee,} {\multicolumn{2}{l}{…}} {\multicolumn{2}{l}{\bluebold{Rahab}}} {\multicolumn{2}{l}{yang}} {liat,} {\multicolumn{2}{l}{\bluebold{Rahab}}} {\bluebold{de}}\\ %
& and.then & \multicolumn{2}{l}{\textsc{3sg}} & \multicolumn{2}{l}{\textsc{rdp}{\Tilde}spit} & spoot! & \multicolumn{2}{l}{} & \multicolumn{2}{l}{Rahab} & \multicolumn{2}{l}{\textsc{rel}} & see & \multicolumn{2}{l}{Rahab} & \textsc{3sg}\\
& \multicolumn{2}{l}{jemur{\Tilde}jemur} & \multicolumn{2}{l}{pakeang} & \multicolumn{3}{l}{begini} & \multicolumn{2}{l}{baru} & \multicolumn{2}{l}{\bluebold{de}} & \multicolumn{3}{l}{perhatikang,} & \multicolumn{2}{l}{…}\\
& \multicolumn{2}{l}{\textsc{rdp}{\Tilde}be.dry} & \multicolumn{2}{l}{clothes} & \multicolumn{3}{l}{like.this} & \multicolumn{2}{l}{and.then} & \multicolumn{2}{l}{\textsc{3sg}} & \multicolumn{3}{l}{observe} & \multicolumn{2}{l}{}\\
\lspbottomrule
\end{tabular}
\ea
\glt
[About a bad-mannered intruder:] ‘and then he was spitting ‘spoot!’ … (it was) \bluebold{Rahab} who saw (it), \bluebold{Rahab} was drying clothes at that moment, then \bluebold{she} noticed …’ \textstyleExampleSource{[081006-035-CvEx.0042]}
\end{styleFreeTranslEngxvpt}

\subparagraph[‘np 3sg’ noun phrases and their head nominals]{‘\textsc{np} \textsc{3sg}’ noun phrases and their head nominals}

This section summarizes the syntactic and lexical properties of ‘\textsc{np} \textsc{3sg}’ noun phrases.



‘\textsc{np} \textsc{3sg}’ noun phrases in the corpus typically take the subject slot in clause-initial position, as in (0) to (0). Other slots, however, are also possible, such as the direct object slot in (0), or the possessor slot in (0). The referent can be expressed with common nouns as in (0), proper nouns as in (0), or noun phrases with adnominal modifier, as in (0). Further, determiner \textitbf{dia}/\textitbf{de} ‘\textsc{3sg}’ also occurs in complex noun phrases, as in \textitbf{bapa dari Jepang dia} ‘the man from Japan’ (literally ‘he man from Japan’) in (0), or \textitbf{kaka pendeta di Mambramo de tu} ‘that older pastor sibling in the Mambramo area’ (literally ‘that he older pastor in the Mambramo area’) in (0). The referents in ‘\textsc{np} \textsc{3sg}’ noun phrases are usually human, but they can also be animate nonhuman such as \textitbf{kaswari} ‘cassowary’ in (0), or inanimate such as \textitbf{bua mangga} ‘mango fruit’ in (0).
\end{styleBodyvvafter}

\begin{styleExampleTitle}
Determiner uses of \textitbf{dia}/\textitbf{de} ‘\textsc{3sg}’
\end{styleExampleTitle}

\begin{tabular}{llllllllllll}
\lsptoprule
\label{bkm:Ref353197377}
\gll {…} {di} {dano} {situ} {di} {[[\bluebold{kaka}} {\bluebold{laki{\Tilde}laki}} {\bluebold{de}]} {pu} {[tempat} {situ]]}\\ %
&  & at & lake & \textsc{l.med} & at & oSb & \textsc{rdp}{\Tilde}husband & \textsc{3sg} & \textsc{poss} & place & \textsc{l.med}\\
\lspbottomrule
\end{tabular}
\ea
\glt 
‘[we wanted to pray a whole night while picnicking, at what’s-its-name,] at the lake there, at \bluebold{the older brother}’s place there’ \textstyleExampleSource{[080922-002-Cv.0090]}
\z

\begin{tabular}{lllllllllll}
\lsptoprule
\label{bkm:Ref353265869}
\gll {…} {\multicolumn{2}{l}{karna}} {ini} {\bluebold{bapa}} {\bluebold{dari}} {\bluebold{Jepang}} {\bluebold{dia}} {suda} {kutuk}\\ %
&  & \multicolumn{2}{l}{because} & \textsc{d.prox} & father & from & Japan & \textsc{3sg} & already & curse\\
& \multicolumn{2}{l}{kota} & \multicolumn{8}{l}{ini}\\
& \multicolumn{2}{l}{city} & \multicolumn{8}{l}{\textsc{d.prox}}\\
\lspbottomrule
\end{tabular}
\ea
\glt 
‘… because, what’s-his-name, \bluebold{the gentleman from Japan} already cursed this city’ \textstyleExampleSource{[080917-008-NP.0021]}
\z

\begin{tabular}{lllllllll}
\lsptoprule
\label{bkm:Ref353197378}
\gll {\bluebold{kaka}} {\bluebold{pendeta}} {\bluebold{di}} {\bluebold{Mambramo}} {\bluebold{de}} {\bluebold{tu}} {jual} {RW}\\ %
& oSb & pastor & at & Mambramo & \textsc{3sg} & \textsc{d.dist} & sell & cooked.dog.meat\\
\lspbottomrule
\end{tabular}
\ea
\glt 
‘\bluebold{that older sibling pastor in (the) Mambramo (area)} sells cooked dog meat’ \textstyleExampleSource{[081011-022-Cv.0105]}
\z

\begin{tabular}{lllllllllll}
\lsptoprule
\label{bkm:Ref340775186}
\gll {…} {ato} {\bluebold{kaswari}} {\bluebold{dia}} {ada} {berdiri} {pas} {perhatikang} {begini} {…}\\ %
&  & or & cassowary & \textsc{3sg} & exist & stand & be.exact & watch & like.this & \\
\lspbottomrule
\end{tabular}
\ea
\glt 
‘[if you see a cassowary’s footprint] or \bluebold{the cassowary} is standing right there watching (you) like this, …’ \textstyleExampleSource{[080923-014-CvEx.0022]}
\z

\begin{tabular}{lllllllllllllllll}
\lsptoprule
\label{bkm:Ref354394292}
\gll {…} {\multicolumn{2}{l}{bawa}} {\multicolumn{2}{l}{anaang}} {\multicolumn{2}{l}{pinang,}} {\multicolumn{2}{l}{anaang}} {\multicolumn{2}{l}{sagu,}} {\multicolumn{2}{l}{bibit}} {klapa,} {\multicolumn{2}{l}{bibit}}\\ %
&  & \multicolumn{2}{l}{bring} & \multicolumn{2}{l}{offspring} & \multicolumn{2}{l}{betel.nut} & \multicolumn{2}{l}{offspring} & \multicolumn{2}{l}{sago} & \multicolumn{2}{l}{seedling} & coconut & \multicolumn{2}{l}{seedling}\\
& \multicolumn{2}{l}{pisang,} & \multicolumn{2}{l}{…} & \multicolumn{2}{l}{mungking} & \multicolumn{2}{l}{\bluebold{bua}} & \multicolumn{2}{l}{\bluebold{mangga}} & \multicolumn{2}{l}{\bluebold{de}} & punya & \multicolumn{2}{l}{bibit} & …\\
& \multicolumn{2}{l}{banana} & \multicolumn{2}{l}{} & \multicolumn{2}{l}{maybe} & \multicolumn{2}{l}{fruit} & \multicolumn{2}{l}{mango} & \multicolumn{2}{l}{\textsc{3sg}} & \textsc{poss} & \multicolumn{2}{l}{seedling} & \\
\lspbottomrule
\end{tabular}
\ea
\glt
[About wedding customs:] ‘[(when) we bring (our son,] (we) bring betel nut seedlings, sago seedlings, coconut seedlings, banana seedlings, … maybe \bluebold{seedlings of the mangga fruit}, …’ \textstyleExampleSource{[081110-005-CvPr.0056-0057]}
\end{styleFreeTranslEngxvpt}

\paragraph[‘np pro{}-sg’ expressions with comma intonation]{‘\textsc{np} \textsc{pro-sg}’ expressions with comma intonation}
\label{bkm:Ref354422229}
The corpus also contains ‘\textsc{np} \textsc{pro-sg}’ expressions in which the nouns are set off from the following pronouns by a comma intonation (“{\textbar}”), as in (0) to (0).



In ‘\textsc{np} \textsc{pro-sg}’ expressions with second person \textitbf{ko} ‘\textsc{2sg}’, the marked-off nouns function as vocatives, that is, as forms of direct address. Cross-linguistically, ‘\textsc{voc} \textsc{pro}’ expressions serve to specify “a person out of a group of persons while using a second person singular pronoun” with the vocative noun being “separated from the rest of the sentence by intonation” {\citep[46]{Bhat2007}}. This strategy of singling out and addressing particular individuals through a ‘\textsc{2sg}, \textsc{pro}’ or ‘\textsc{voc}, \textsc{pro}’ expression is shown in (0) and (0), respectively: \textitbf{mama} ‘mother’ and \textitbf{Ise} are vocatives which are set off from second person \textitbf{ko} ‘2\textsc{sg}’ with a distinct comma intonation. Hence, these expressions cannot be interpreted as ‘\textsc{np} \textsc{2sg}’ noun phrases.
\end{styleBodyvvafter}

\begin{styleExampleTitle}
Topic-comment constructions with comma intonation: ‘\textsc{np} {\textbar} \textsc{2sg}’
\end{styleExampleTitle}

\begin{tabular}{lllllllllll}
\lsptoprule
\label{bkm:Ref354406860}
\gll {trus} {Martina} {de} {tanya} {saya,} {\bluebold{mama}} {{\textbar}} {\bluebold{ko}} {rasa} {bagemana?}\\ %
& next & Martina & \textsc{3sg} & ask & \textsc{1sg} & mother &  & \textsc{2sg} & feel & how\\
\lspbottomrule
\end{tabular}
\ea
\glt 
‘and then Martina asked me, ‘\bluebold{mother}, how do \bluebold{you} feel?’’ \textstyleExampleSource{[081015-005-NP.0018]}
\z

\begin{tabular}{llllllllllllll}
\lsptoprule
\label{bkm:Ref354406861}
\gll {\multicolumn{2}{l}{jadi}} {Ise} {\multicolumn{2}{l}{ni}} {tong} {su} {bilang} {dia,} {\bluebold{Ise}} {{\textbar}} {\bluebold{ko}} {tinggal}\\ %
& \multicolumn{2}{l}{so} & Ise & \multicolumn{2}{l}{\textsc{d.prox}} & \textsc{1pl} & already & say & \textsc{3sg} & Ise &  & \textsc{2sg} & stay\\
& di & \multicolumn{3}{l}{sini} & \multicolumn{9}{l}{suda!}\\
& at & \multicolumn{3}{l}{\textsc{l.prox}} & \multicolumn{9}{l}{already}\\
\lspbottomrule
\end{tabular}
\ea
\glt 
‘so Ise here, we already told her, ‘\bluebold{Ise}, \bluebold{you }stay here!’’ \textstyleExampleSource{[080917-008-NP.0026]}
\z


‘\textsc{np} \textsc{3sg}’ expressions with a comma intonation are analyzed as topic-comment constructions. Cross-linguistically, in topic-comment constructions, the “topic is generally expected to continue” and therefore “third person pronouns […] are used in order to represent the continued occurrence of a topic” {\citep[209]{Bhat2007}}. This observation also applies to Papuan Malay ‘\textsc{np}, \textsc{3sg}’ expressions, that is, the preposed noun phrase signals the topic, while co-referential \textitbf{dia}/\textitbf{de} ‘\textsc{3sg}’ has comment function. This strategy of forming topic-comment constructions is shown in (0) and (0): \textitbf{orang Senggi} and \textitbf{Klara} designate the topics while \textitbf{dia} ‘\textsc{3sg}’ and \textitbf{de} ‘\textsc{3sg}’ function as comments, respectively.


\begin{styleExampleTitle}
Topic-comment constructions with comma intonation: ‘\textsc{np} {\textbar} \textsc{3sg}’
\end{styleExampleTitle}

\begin{tabular}{lllllllllllllll}
\lsptoprule
\label{bkm:Ref354406862}
\gll {\multicolumn{2}{l}{baru}} {dia} {\multicolumn{4}{l}{datang,}} {\multicolumn{2}{l}{orang}} {\multicolumn{2}{l}{Jayapura}} {sana,} {kawang} {itu,}\\ %
& \multicolumn{2}{l}{and.then} & \textsc{3sg} & \multicolumn{4}{l}{come} & \multicolumn{2}{l}{person} & \multicolumn{2}{l}{Jayapura} & \textsc{l.dist} & friend & \textsc{d.dist}\\
& \bluebold{orang} & \multicolumn{3}{l}{\bluebold{Senggi}} & {\textbar} & \bluebold{dia} & \multicolumn{2}{l}{datang} & \multicolumn{2}{l}{de} & \multicolumn{4}{l}{duduk}\\
& person & \multicolumn{3}{l}{Senggi} &  & \textsc{3sg} & \multicolumn{2}{l}{come} & \multicolumn{2}{l}{\textsc{3sg}} & \multicolumn{4}{l}{sit}\\
\lspbottomrule
\end{tabular}
\ea
\glt 
[Talking about a friend:] ‘and then she came, (the) person (from) Jayapura over there, that friend, \bluebold{(the) person (from) Senggi}, \bluebold{she} came (and) she sat (down)’ \textstyleExampleSource{[080917-008-NP.0107]}
\z

\begin{tabular}{llllllll}
\lsptoprule
\label{bkm:Ref354406863}
\gll {Klara} {{\textbar}} {\bluebold{de}} {lompat} {satu} {kali} {tu}\\ %
& Klara &  & \textsc{3sg} & jump & one & time & \textsc{d.dist}\\
\lspbottomrule
\end{tabular}
\ea
\glt 
‘\bluebold{Klara}, \bluebold{she} jumped once (\textsc{emph})’ \textstyleExampleSource{[081025-006-Cv.0216]}
\z


Topic-comment constructions with no comma intonation are also possible, however. In this type of constructions, the topic is expressed in a noun phrase with a pronoun determiner and demonstrative modifier such that ‘\textsc{n pro-sg dem}’. This is illustrated in (0), repeated as (0), and in (0) in §6.2.1.1 (p. \pageref{bkm:Ref436750579}). Very often, however, the preposed topical noun phrase does not contain a pronoun determiner, such that ‘\textsc{n dem}’. This is demonstrated with the topic-comment constructions \textitbf{ade ini de} ‘this younger sibling, he/she’ in (0), and \textitbf{Ise ni de} ‘Ise here, she’ in (0). In ‘[\textsc{n (pro-sg) dem]} \textsc{pro-sg}’ topic-comment constructions, the demonstrative sets aside the topic, and therefore no comma intonation is needed.


\begin{styleExampleTitle}
Topic-comment constructions with demonstrative: ‘\textsc{np dem} \textsc{pro-sg}’
\end{styleExampleTitle}

\begin{tabular}{llllll}
\lsptoprule
\label{bkm:Ref354418305}
\gll {[\bluebold{Barce}} {\bluebold{ko}} {\bluebold{ini}]} {[\bluebold{ko}]} {takut}\\ %
& Barce & \textsc{2sg} & \textsc{d.prox} & \textsc{2sg} & feel.afraid(.of)\\
\lspbottomrule
\end{tabular}
\ea
\glt 
‘\bluebold{you Barce here}, \bluebold{you} feel afraid’ \textstyleExampleSource{[081109-001-Cv.0131]}
\z

\begin{tabular}{llllll}
\lsptoprule
\label{bkm:Ref353034924}
\gll {baru} {[\bluebold{ade}} {\bluebold{ini}]} {[\bluebold{de}]} {sakit}\\ %
& and.then & ySb & \textsc{d.prox} & \textsc{3sg} & be.sick\\
\lspbottomrule
\end{tabular}
\ea
\glt 
‘and then \bluebold{this younger sibling, he/she} is sick’ \textstyleExampleSource{[080917-002-Cv.0020]}
\z

\begin{tabular}{lllllllll}
\lsptoprule
\label{bkm:Ref353034925}
\gll {…} {[\bluebold{Ise}} {\bluebold{ni}]} {[\bluebold{de}]} {su} {mulay} {takut} {ini}\\ %
&  & Ise & \textsc{d.prox} & \textsc{3sg} & already & start & feel.afraid(.of) & \textsc{d.prox}\\
\lspbottomrule
\end{tabular}
\ea
\glt 
‘[this tree began shaking, shaking like this, and] \bluebold{Ise here,} \bluebold{she} already started feeling afraid’ \textstyleExampleSource{[080917-008-NP.0028]}
\z


At this stage in the research on Papuan Malay, it is not possible to tell if there are rules governing the choice between ‘\textsc{np}, \textsc{pro-sg}’ and ‘\textsc{np dem} \textsc{pro-sg}’ topic-comment constructions. To answer this question more research is needed.
\end{styleBodyxvafter}

\paragraph[Analysis of ‘np pro{}-sg’ expressions as noun phrases and not as topic{}-comment constructions]{Analysis of ‘\textsc{np} \textsc{pro-sg}’ expressions as noun phrases and not as topic-comment constructions}
\label{bkm:Ref362253852}
There are four reasons for analyzing the ‘\textsc{np} \textsc{2sg}’ expressions in (0) to (0) and the ‘\textsc{np} \textsc{3sg}’ constructions in (0) to (0) as noun phrases with pronominal determiner and not as topic-comment constructions.



First, ‘\textsc{np} \textsc{pro-sg}’ expressions can occur in positions other than the clause-initial subject slot, as shown with the ‘\textsc{np} \textsc{2sg}’ noun phrases in (0), (0), and (0), and the ‘\textsc{np} \textsc{3sg}’ noun phrases in (0) and (0). In these positions, however, the respective common nouns cannot be interpreted as topics in topic-comment constructions. This is due to the fact that topicalized constituents do not remain in-situ but are fronted to the clause-initial position (see also example (0) in §1.6.1.4, p. \pageref{bkm:Ref436750646}).
\end{styleBodyvafter}


Second, an ‘\textsc{np} \textsc{pro-sg}’ expression can be modified with a demonstrative, as in the ‘\textsc{np} \textsc{2sg}’ noun phrases in (0) to (0), or the ‘\textsc{np} \textsc{3sg}’ noun phrases in (0) and (0). In these ‘\textsc{np} \textsc{pro-sg}’ expressions, the demonstratives have scope over the pronouns. The fact that the pronouns occur in noun phrases with adnominal demonstrative, in turn, supports the conclusion that in ‘\textsc{np} \textsc{pro-sg}’ expressions the pronoun functions as determiner. Moreover, in two of the examples, the ‘\textsc{np} \textsc{pro-sg} \textsc{dem}’ expressions have topic function in topic-comment constructions, namely the ‘\textsc{np} \textsc{2sg} \textsc{dem}’ noun phrases in (0) and (0). In both cases, the preposed noun phrases are copied by co-referential \textitbf{ko} ‘\textsc{2sg}’ which has comment function.\footnote{\\
\\
\\
\\
\\
\\
\\
\\
\\
\\
\\
\\
\\
\\
\par There is no comma intonation between the topical noun phrases and the pronominal comments in (0) and (0).\\
} Neither bare \textitbf{Barce} in (0), nor bare \textitbf{Eferdina} in (0) can be topics in topic-comment constructions. Instead it is the entire noun phrase, including determiner \textitbf{ko} ‘2\textsc{sg}’, which has topic function. This, in turn, also supports the conclusion that in ‘\textsc{np} \textsc{pro-sg}’ expressions, the pronoun functions as a pronominal determiner.
\end{styleBodyvafter}


Third, by indicating person, singularity, and definiteness of their referents, determiner pronouns have pertinent discourse functions. In direct speech, ‘\textsc{np} \textsc{pro-sg}’ expressions with second person \textitbf{ko} ‘2\textsc{sg}’ mark the referent of the head nominal unambiguously as the intended addressee. In reported speech, ‘\textsc{np} \textsc{2sg}’ noun phrases indicate that the referent is the addressee of the direct quotation. In addition, they signal the hearers that they are in a position to identify the referent. Finally, as apostrophes in rhetoric figures of direct speech they serve as “exclamatory addressees”. ‘\textsc{np} \textsc{pro-sg}’ expressions with third person \textitbf{dia}/\textitbf{de} ‘3\textsc{sg}’ signal and accentuate that the speakers expect their hearers to be familiar with the referents encoded by their head nominals. That is, the interlocutors are communicated that they should be able to identify the referents.
\end{styleBodyvafter}


Fourth, the corpus includes a number of utterances, in which speakers repeat an ‘\textsc{np} \textsc{pro-sg}’ expression as a form of hesitation or delay; in each case the pronoun is third person \textitbf{dia}/\textitbf{de} ‘3\textsc{sg}’. Two of these repetitions are presented in (0) and (0). It is noted that the speakers do not repeat the respective bare nouns \textitbf{pace} ‘man’ and \textitbf{Markus}, but the ‘\textsc{np} \textsc{3sg}’ expressions \textitbf{pace de} ‘the man’ (literally ‘he man’) and \textitbf{Markus de} ‘Markus’ (literally ‘he Markus’). This suggests that they perceive these expressions to be cohesive entities which, in turn, supports their analysis as single noun phrases.
\end{styleBodyvxafter}

\begin{tabular}{llllllllll}
\lsptoprule
\label{bkm:Ref340775174}
\gll {[\bluebold{pace}} {\bluebold{de}],} {[\bluebold{pace}} {\bluebold{de}]} {mandi} {rapi,} {de} {mandi} {rapi}\\ %
& man & \textsc{3sg} & man & \textsc{3sg} & bathe & be.neat & \textsc{3sg} & bathe & be.neat\\
\lspbottomrule
\end{tabular}
\ea
\glt 
‘\bluebold{the man}, \bluebold{the man} bathed neatly, he bathed neatly’ \textstyleExampleSource{[081109-007-JR.0002]}
\z

\begin{tabular}{llllllll}
\lsptoprule
\label{bkm:Ref353022699}
\gll {akirnya} {[\bluebold{Markus}} {\bluebold{de}],} {[\bluebold{Markus}} {\bluebold{dia}]} {turung} {begini}\\ %
& finally & Markus & \textsc{3sg} & Markus & \textsc{3sg} & descend & like.this\\
\lspbottomrule
\end{tabular}
\ea
\glt
‘finally \bluebold{Markus}, \bluebold{Markus} came down (to the coast) like this’ \textstyleExampleSource{[080922-010a-CvNF.0204]}
\end{styleFreeTranslEngxvpt}

\subsection{Adnominal plural personal pronouns}
\label{bkm:Ref353216791}
Plural pronouns also function as determiners in noun phrases, as illustrated in (0) and (0). They signal the definiteness and person-number values of their referents, and thereby allow their unambiguous identification.
\end{styleBodyxafter}

\begin{tabular}{lllll}
\lsptoprule
\label{bkm:Ref353035913}
\gll {[\bluebold{pemuda}} {\bluebold{dong}]} {snang} {skali}\\ %
& youth & \textsc{3pl} & feel.happy(.about) & very\\
\lspbottomrule
\end{tabular}
\ea
\glt 
‘\bluebold{the young people} feel very happy’ (Lit. ‘\bluebold{youth they}’) \textstyleExampleSource{[080925-003-Cv.0220]}
\z

\begin{tabular}{lllll}
\lsptoprule
\label{bkm:Ref353036361}
\gll {[\bluebold{Ise}} {\bluebold{dong}]} {su} {datang}\\ %
& Ise & \textsc{3pl} & already & come\\
\lspbottomrule
\end{tabular}
\ea
\glt 
‘\bluebold{Ise and her companions including herself} already came’ (Lit. ‘\bluebold{Ise they}’) \textstyleExampleSource{[080925-003-Cv.0169]}
\z


The examples in (0) and (0) also show that ‘\textsc{n} \textsc{pro-pl}’ noun phrases have two readings.



First with an indefinite referent, such as \textitbf{pemuda} ‘youth’ in (0), ‘\textsc{n} \textsc{pro-pl}’ noun phrases have an additive plural reading. Second with a definite referent such as \textitbf{Ise} in (0), ‘\textsc{n} \textsc{pro-pl}’ noun phrases receive an associative inclusory plural reading. This makes Papuan Malay belong to the large group of languages in Asia where the “associative plural marker […] is also used to express additive plurals” {(Daniel and Moravcsik 2011: 5–6)}.
\end{styleBodyvafter}


The additive plural interpretation of ‘\textsc{n} \textsc{pro-pl}’ noun phrases is discussed in §6.2.2.1 and the associative inclusory plural reading in §6.2.2.2. These descriptions are followed in §6.2.2.3 by a brief overview of the associative plural in other regional Malay varieties.
\end{styleBodyvxvafter}

\paragraph[Additive plural interpretation]{Additive plural interpretation}
\label{bkm:Ref345074606}
In ‘\textsc{n} \textsc{pro-pl}’ noun phrases with indefinite referents, adnominal plural pronouns have two functions. They signal the definiteness of their referents and an additive plural reading of the respective noun phrases with the basic meaning of ‘the Xs’.



Cross-linguistically, the additive interpretation implies referential homogeneity of the group. That is, “every referent of the plural form is also a referent of the stem” {(Daniel and Moravcsik 2011: 1)}. This additive reading of Papuan Malay ‘\textsc{n} \textsc{pro-pl}’ is illustrated in (0) to (0). In (0), \textitbf{kitorang} ‘\textsc{1pl}’ denotes the plurality of its bare head nominal \textitbf{nene} ‘grandmother’, while in (0) \textitbf{kamu} ‘\textsc{2pl}’ signals the plurality of \textitbf{bangsat} ‘rascal’, and in (0) \textitbf{dong} ‘\textsc{3pl}’ indicates the plurality of \textitbf{anjing} ‘dog’. These examples also show that the referent is always animate. It can be human as in (0) and (0), or nonhuman as in (0); inanimate referents are unattested.
\end{styleBodyvvafter}

\begin{styleExampleTitle}
Additive plural interpretation with bare head nominal
\end{styleExampleTitle}

\begin{tabular}{llllll}
\lsptoprule
\label{bkm:Ref340775189}
\gll {jadi} {\bluebold{nene}} {\bluebold{kitorang}} {\bluebold{ini}} {masak}\\ %
& so & grandmother & \textsc{1pl} & \textsc{d.prox} & cook\\
\lspbottomrule
\end{tabular}
\ea
\glt 
‘so \bluebold{we grandmothers here} cook’ \textstyleExampleSource{[080924-001-Pr.0008]}
\z

\begin{tabular}{llllllllll}
\lsptoprule
\label{bkm:Ref345065397}
\gll {\bluebold{bangsat}} {\bluebold{kamu}} {\bluebold{tu}} {tinggal} {lari} {ke} {sana} {ke} {mari}\\ %
& rascal & \textsc{2pl} & \textsc{d.dist} & stay & run & to & \textsc{l.dist} & to & hither\\
\lspbottomrule
\end{tabular}
\ea
\glt 
‘\bluebold{you rascals there} keep running back and forth’ \textstyleExampleSource{[080923-012-CvNP.0011]}
\z

\begin{tabular}{lllllll}
\lsptoprule
\label{bkm:Ref340775190}
\gll {…} {di} {mana} {\bluebold{anjing}} {\bluebold{dong}} {gong-gong}\\ %
&  & at & where & dog & \textsc{3pl} & bark(.at)\\
\lspbottomrule
\end{tabular}
\ea
\glt 
‘[I just ran closing in on the pig] where \bluebold{the dogs} were barking’ \textstyleExampleSource{[080919-003-NP.0007]}
\z


In (0) to (0) the number of referents is left unspecified. When this number is limited to two, speakers very often use a dual construction, such that ‘bare \textsc{n} \textsc{pro-pl} \textitbf{dua}’. In such a construction, the two referents are not explicitly mentioned but subsumed under the postposed adnominal numeral \textitbf{dua} ‘two’, as in (0) and (0).


\begin{styleExampleTitle}
Additive dual interpretation
\end{styleExampleTitle}

\begin{tabular}{lllll}
\lsptoprule
\label{bkm:Ref345071039}
\gll {\bluebold{laki{\Tilde}laki}} {\bluebold{kam}} {\bluebold{dua}} {sapu}\\ %
& \textsc{rdp}{\Tilde}husband & \textsc{2pl} & two & sweep\\
\lspbottomrule
\end{tabular}
\ea
\glt 
‘\bluebold{you two boys} sweep’ \textstyleExampleSource{[081115-001b-Cv.0010]}
\z

\begin{tabular}{lllllll}
\lsptoprule
\label{bkm:Ref345071040}
\gll {\bluebold{pace}} {\bluebold{dorang}} {\bluebold{dua}} {\bluebold{ini}} {ke} {atas}\\ %
& man & \textsc{3pl} & two & \textsc{d.prox} & to & top\\
\lspbottomrule
\end{tabular}
\ea
\glt
‘\bluebold{the two men here} (went) up (there)’ \textstyleExampleSource{[081006-034-CvEx.0010]}
\end{styleFreeTranslEngxvpt}

\paragraph[Associative inclusory plural interpretation]{Associative inclusory plural interpretation}
\label{bkm:Ref345150041}
‘\textsc{n} \textsc{pro-pl}’ noun phrases with a definite referent and an adnominal plural pronoun receive an associative inclusory plural reading. 



In her cross-linguistic semantic analysis of associative plurals, {Moravcsik (2003: 470–471)} defines “associative plurals as “constructions whose meaning is ‘X and X’s associate(s)’, where all members are individuals, X is the focal referent, and the associate(s) form a group centering around X”. In Papuan Malay, the focal referent is always encoded with a noun or noun phrase heading the phrasal construction, while the associates are encoded with a post-head plural pronoun. In (0) and (0), for instance, \textitbf{Lodia} and \textitbf{Pawlus} are the focal referents while the pronouns \textitbf{torang} ‘\textsc{1pl}’ and \textitbf{dorang} ‘\textsc{3pl}’ denote the associates, respectively.



The reading of Papuan Malay ‘\textsc{n} \textsc{pro-pl}’ noun phrases is not only associative, however. Adopting {Moravcsik’s (2003: 479) analysis, the reading of such noun phrases is }also “inclusory”, in that “all members of the plural set are summarily referred to by a pronoun” (see also {Haspelmath 2004: 25; Gil 2009}). That is, the reference of the plural pronoun in a Papuan Malay ‘\textsc{n} \textsc{pro-pl}’ noun phrase includes the reference of the focal referent, such that ‘\textsc{pro} including X’ In (0), for instance, the pronoun \textitbf{torang} ‘\textsc{1pl}’ includes not only the companions and the speaker, but all members of the plural set, “including \textitbf{Lodia}”. That is, the ‘\textsc{n} \textsc{pro-pl}’ noun phrase \textitbf{Lodia torang} does not signal an additive relation in the sense of ‘Lodia plus we companions’. Likewise in (0), the reference of \textitbf{dorang} ‘\textsc{3pl}’ includes not only the associates of the focal referent \textitbf{Pawlus}, but all members of the plural set, “including Pawlus”.
\end{styleBodyvvafter}

\begin{styleExampleTitle}
Associative inclusory plural interpretation
\end{styleExampleTitle}

\begin{tabular}{llllllll}
\lsptoprule
\label{bkm:Ref345071064}
\gll {itu} {yang} {\bluebold{Lodia}} {\bluebold{torang}} {bilang} {begini} {…}\\ %
& \textsc{d.dist} & \textsc{rel} & Lodia & \textsc{1pl} & say & like.this & \\
\lspbottomrule
\end{tabular}
\ea
\glt 
‘that’s why \bluebold{Lodia and we companions including her} said like this, …’ \textstyleExampleSource{[081115-001a-Cv.0001]}
\z

\begin{tabular}{llllll}
\lsptoprule
\label{bkm:Ref352587323}
\gll {tanta} {ada} {mara} {\bluebold{Pawlus}} {\bluebold{dorang}}\\ %
& aunt & exist & be.angry & Pawlus & \textsc{3pl}\\
\lspbottomrule
\end{tabular}
\ea
\glt 
‘aunt is being angry with \bluebold{Pawlus and his companions including Pawlus}’ \textstyleExampleSource{[081006-009-Cv.0002]}
\z


In the following, the semantic properties of associative inclusory expressions are examined. Also discussed are the lexical classes used in these expressions and the types of relationships expressed within the associated groups.



Cross-linguistically, associative inclusory expressions imply three distinct semantic properties, namely “referential heterogeneity”, “reference to groups”, and “asymmetry” ({Daniel and Moravcsik 2011; Moravcsik 2003)}. The notion of “referential heterogeneity” implies that “the associative plural designates a heterogeneous set” {(Daniel and Moravcsik 2011: 1)}. The semantic property of “reference to groups” refers to a high degree of internal cohesion within the plural construction; that is, the focal referent and the associates form “a spatially or conceptually coherent group” {\citep[471]{Moravcsik2003}}. The notion of “asymmetry” implies that the groups are “ranked”, in that the associative plural names its pragmatically most salient or highest ranking member, the focal referent {(2003: 471)}.
\end{styleBodyvafter}


Referential heterogeneity of Papuan Malay associative inclusory expressions is illustrated with the examples in (0) to (0). In (0), \textitbf{bapa Iskia dong} ‘father Iskia and them’ does not denote several people called \textitbf{Iskia}; neither does \textitbf{bapa desa dorang} ‘father mayor and them’ refer to more than one mayor. The same applies to the examples in (0) and (0) (in this context \textitbf{dokter} ‘doctor’ has a definite reading as the local hospital has only one doctor). In each case, the plural pronoun encodes a heterogeneous set of associates “centering around X”, the focal referent. Moreover, the pronouns include the focal referents in their reference.
\end{styleBodyvvafter}

\begin{styleExampleTitle}
Associative inclusory plural interpretation with the third person plural pronoun
\end{styleExampleTitle}

\begin{tabular}{lllllllllll}
\lsptoprule
\label{bkm:Ref345071078}
\gll {\multicolumn{2}{l}{\bluebold{bapa}}} {\multicolumn{2}{l}{\bluebold{Iskia}}} {\bluebold{dong}} {bunu} {babi,} {\bluebold{bapa}} {\bluebold{desa}} {\bluebold{dorang}}\\ %
& \multicolumn{2}{l}{father} & \multicolumn{2}{l}{Iskia} & \textsc{3pl} & kill & pig & father & village & \textsc{3pl}\\
& dong & \multicolumn{2}{l}{bunu} & \multicolumn{7}{l}{babi}\\
& \textsc{3pl} & \multicolumn{2}{l}{kill} & \multicolumn{7}{l}{pig}\\
\lspbottomrule
\end{tabular}
\ea
\glt 
‘\bluebold{father Iskia and his companions including Iskia} killed a pig, \bluebold{father mayor and his companions including the mayor}, they killed a pig’ \textstyleExampleSource{[080917-008-NP.0120]}
\z

\begin{tabular}{llllllllll}
\lsptoprule
\label{bkm:Ref354490074}
\gll {Ise} {ko} {tinggal} {di} {sini} {suda} {deng} {\bluebold{mama-tua}} {\bluebold{dorang}!}\\ %
& Ise & \textsc{2sg} & stay & at & \textsc{l.prox} & just & with & aunt & \textsc{3pl}\\
\lspbottomrule
\end{tabular}
\ea
\glt 
‘you Ise just stay here with \bluebold{aunt and her companions including aunt}!’ \textstyleExampleSource{[080917-008-NP.0026]}
\z

\begin{tabular}{llllll}
\lsptoprule
\label{bkm:Ref354490076}
\gll {\bluebold{dokter}} {\bluebold{dorang}} {bilang} {begini} {…}\\ %
& doctor & \textsc{3pl} & say & like.this & \\
\lspbottomrule
\end{tabular}
\ea
\glt 
‘\bluebold{the doctor and his companions including the doctor} said like this, …’ \textstyleExampleSource{[081015-005-NP.0047]}
\z


The semantic property “reference to groups” is shown in (0) and (0). In the two examples, the ‘\textsc{n} \textsc{pro}’ noun phrases denote coherent groups of inherently associated individuals, namely \textitbf{bapa Iskia dong} ‘father Iskia and them’, \textitbf{bapa desa dorang} ‘father mayor and them’, and \textitbf{mama-tua dorang} ‘aunt and them’, respectively. These examples also illustrate the notion of “ranking” in associative inclusory expressions. The pragmatically highest ranking members are the focal referents \textitbf{bapa Iskia} ‘father Iskia’ and \textitbf{bapa desa} ‘father mayor’ in (0), and \textitbf{mama-tua} ‘aunt’ in (0). The remaining members of the plural sets, by contrast, are not fully enumerated but subsumed under the plural pronoun \textitbf{dong} / \textitbf{dorang }‘3\textsc{pl}’.



Typically, the associates are encoded with the third person plural pronoun. Less frequently, the associates are encoded with the first person plural pronoun, as in (0), repeated as (0), or with the second person plural pronoun as in (0) and (0). In associative inclusory expressions formed with the second person plural pronoun, the focal referent is typically the addressee as in (0). Alternatively, although much less often, one of the associates can be the addressee as in (0) (the focal referent \textitbf{Lodia} was not present during this conversation).
\end{styleBodyvvafter}

\begin{styleExampleTitle}
Associative inclusory plural interpretation with the first and second person plural pronouns
\end{styleExampleTitle}

\begin{tabular}{llllllll}
\lsptoprule
\label{bkm:Ref354490068}
\gll {itu} {yang} {\bluebold{Lodia}} {\bluebold{torang}} {bilang} {begini} {…}\\ %
& \textsc{d.dist} & \textsc{rel} & Lodia & \textsc{1pl} & say & like.this & \\
\lspbottomrule
\end{tabular}
\ea
\glt 
‘that’s why \bluebold{Lodia and her companions including me} said like this, …’ \textstyleExampleSource{[081115-001a-Cv.0001]}
\z

\begin{tabular}{lllllll}
\lsptoprule
\label{bkm:Ref354490069}
\gll {\bluebold{tanta}} {\bluebold{Oktofina}} {\bluebold{kam}} {pulang} {jam} {brapa?}\\ %
& aunt & Oktofina & \textsc{2pl} & go.home & hour & several\\
\lspbottomrule
\end{tabular}
\ea
\glt 
‘what time did \bluebold{you aunt Oktofina and your companions including you (Oktofina)} come home?’ \textstyleExampleSource{[081006-010-Cv.0001]}
\z

\begin{tabular}{llllllllll}
\lsptoprule
\label{bkm:Ref354490070}
\gll {\bluebold{Lodia}} {\bluebold{kam}} {pake} {trek} {ke} {sana} {baru} {sa} {…}\\ %
& Lodia & \textsc{2pl} & use & truck & to & \textsc{l.dist} & and.then & \textsc{1sg} & \\
\lspbottomrule
\end{tabular}
\ea
\glt 
‘\bluebold{Lodia and her companions including you (addressee)} took the truck to (go) over there, and then I …’ \textstyleExampleSource{[081022-001-Cv.0001]}
\z


In (0) to (0), the number of referents is not specified. When only two participants are involved, however, that is the focal referent plus one associate, Papuan Malay speakers very often use a dual construction, such that ‘bare \textsc{n} \textsc{pro-pl} \textitbf{dua}’, as in (0). Like dual constructions with an additive reading (§6.2.2.1), the associate is not explicitly mentioned but subsumed under the post-head numeral \textitbf{dua} ‘two’.


\begin{styleExampleTitle}
Associative inclusory dual interpretation
\end{styleExampleTitle}

\begin{tabular}{lllllllll}
\lsptoprule
\label{bkm:Ref354490071}
\gll {\bluebold{om}} {\bluebold{kitong}} {\bluebold{dua}} {kluar} {mo} {pergi} {cari} {pinang}\\ %
& uncle & \textsc{1pl} & two & go.out & want & go & search & betel.nut\\
\lspbottomrule
\end{tabular}
\ea
\glt 
‘\bluebold{uncle and I} went out and wanted to look for betel nuts’ \textstyleExampleSource{[081006-009-Cv.0014]}
\z


In terms of the lexical classes employed in associative plural expressions, {Daniel and \citet[3]{Moravcsik2011}} observe “a clear preference for associative plurals formed from proper names over kin terms over non-kin human common nouns over non-human nouns”. This also applies to Papuan Malay, in that the focal referents in associative inclusory expressions are formed from human nouns while non-human animate focal referents are unattested. Among human nouns in the corpus, however, kin terms as in (0) are more common than proper names as in (0). This has to do with the fact that culturally people prefer not to use proper names, if they have another option, especially if the person is older and/or present. In addition, although not very often, associative plural expressions are formed from non-kin terms such as the title noun expression \textitbf{bapa desa} ‘father mayor’ in (0), or the common noun \textitbf{dokter} ‘doctor’ in (0). (See also {Moravcsik 2003: 471–473}.)



Concerning the relationship between the focal referent X and the associates, {Daniel and \citet[3]{Moravcsik2011}} note that “the group may be: (i) X’s family, (ii) X’s friends, or familiar associates, or (iii) an occasional group that X is a member of” with “kin forming the most commonly understood associates”. Papuan Malay also conforms to this cross-linguistic finding in that the associates are most often X’s family as in (0). Less commonly, X’s associates are friends or companions in a shared activity as in (0). Associative plurals denoting occasional groups or, following {\citet[473]{Moravcsik2003}}, “incidental association”, have not been identified in the corpus.
\end{styleBodyvxvafter}

\paragraph[Associative plural in other regional Malay varieties]{Associative plural in other regional Malay varieties}
\label{bkm:Ref345150044}
The associative plural interpretation for noun phrases with adnominal plural pronoun is also quite common for other regional Malay varieties, such as Ambon, Bali Berkuak, Dobo, Kupang, Manado, or Sri Lanka Malay. In Ternate Malay, however, pronouns do not have adnominal functions {\citep[141]{Litamahuputty2012}}. The associative plural reading of noun phrases with adnominal plural pronouns found in regional Malay varieties is illustrated in the examples in (0) to (0).



In Ambon, Dobo, Kupang, and Sri Lanka Malay, the adnominal pronoun is postposed as in Papuan Malay, as demonstrated in (0) to (0). In Balai Berkuak or Manado Malay, by contrast, the pronoun is in pre-head position, as shown in (0) and (0).
\end{styleBodyvafter}


In all examples, the pronoun is the third person plural pronoun. In most varieties only the short pronoun form is used as for instance in Ambon or Dobo Malay, as shown in (0) and (0). Only in Manado Malay, the short and long forms are used, as shown in (0). Contrasting with Papuan Malay, these regional Malay varieties do not use the first and second person plural pronouns to express associative plurality.
\end{styleBodyvxafter}

\begin{tabular}{lllll}
\lsptoprule
\multicolumn{5}{l}{Ambon Malay {(van Minde 1997: 169)}}\\
\label{bkm:Ref354490079}
\gll {\bluebold{mama}} {\bluebold{dong}} {} {‘mother and the others’}\\ %
& mother & \textsc{3pl} &  & \\
\multicolumn{5}{l}{Dobo Malay {(R.J. Nivens, p.c. 2013)}}\\
\label{bkm:Ref354490081}
\gll {\bluebold{pa}} {\bluebold{Kace}} {\bluebold{dong}} {‘Mr. Kace and his associates’}\\ %
& man & Kace & \textsc{3pl} & \\
\multicolumn{5}{l}{Kupang Malay {(Grimes and Jacob 2008)}}\\
(\stepcounter{}{\the}) & \bluebold{Yan} & \bluebold{dong} &  & ‘Yan and his family / mates’\\
& Yan & \textsc{3pl} &  & \\
\multicolumn{5}{l}{Sri Lanka Malay {\citep{Slomanson2013}}}\\
\label{bkm:Ref354490083}
\gll {\bluebold{Miflal}} {\bluebold{derang}} {} {‘Miflal and his friends’}\\ %
& Miflal & \textsc{3pl} &  & \\
\multicolumn{5}{l}{Balai Berkuak Malay {\citep[7]{Tadmor2002}}}\\
\label{bkm:Ref354490084}
\gll {\bluebold{sidaq}} {\bluebold{Katalq}} {} {‘Katalq and her gang’}\\ %
& \textsc{3pl} & Katalq &  & \\
\multicolumn{5}{l}{Manado Malay {\citep[30]{Stoel2005}}}\\
\label{bkm:Ref354490086}
\gll {\bluebold{dong/dorang}} {\bluebold{Yoram}} {} {‘Yoram and his family’}\\ %
& \textsc{3pl} & Yoram &  & \\
\lspbottomrule
\end{tabular}

In short, among the eastern Malay varieties Papuan Malay is unique given that associative plural expressions are formed with all three plural persons, including the long and the short pronoun forms. This different behavior of Papuan Malay ‘\textsc{n} \textsc{pro-pl}’ noun phrases supports the conclusion put forward in §1.8 that the history of Papuan Malay is different from that of the other eastern Malay varieties.\footnote{\\
\\
\\
\\
\\
\\
\\
\\
\\
\\
\\
\\
\\
\\
\par It is important to note, though, that the observed differences could also result from gaps in the descriptions of the other eastern Malay varieties.\\
}


\section{Summary}
\label{bkm:Ref345150045}
The Papuan Malay pronoun system distinguishes singular and plural numbers and three persons. In addition to signaling the person-number values of their referents they also signal their definiteness. Each pronoun has at least one long and one short form, with the exception of second person singular \textitbf{ko} ‘\textsc{2sg}’. The use of the long and short forms does not mark grammatical distinctions but represents speaker preferences. The pronouns have pronominal and adnominal uses.



In their pronominal uses, the pronouns substitute for noun phrases and designate speech roles. The long and short pronoun forms occur in all syntactic slots within the clause. For the direct and oblique object slots, however, speakers use the long forms much more often. These preferences interrelate with the preferred use of the ‘heavy’ long pronoun forms in clause-final position. This, in turn, reflects the cross-linguistic tendency for the clause-final position to be taken by ‘heavy’ constituents. In adnominal possessive constructions, the pronouns only take the possessor slot; most often it is the short pronouns that take this slot. Pronouns also occur in inclusory conjunction, summary conjunction, and appositional constructions.
\end{styleBodyvafter}


In their adnominal uses, the pronouns occur in post-head position and function as determiners. That is, signaling definiteness and person-number values, the pronouns allow the unambiguous identification of their referents. As determiners, the pronoun forms of all person-number values are employed, with the exception of the first person singular. \textsc{np} \textsc{pro}’ noun phrases with plural pronouns have two possible interpretations. With indefinite referents, they have an additive plural reading, while with definite referents they have an associative inclusory reading.
\end{styleBodyvafter}

%\setcounter{page}{1}\chapter[Demonstratives and locatives]{Demonstratives and locatives}
\label{bkm:Ref374460677}\label{bkm:Ref374456049}
This chapter discusses the Papuan Malay demonstratives and locatives, focusing on their different functions and domains of use. Demonstratives and locatives are deictic expressions that provide orientation to the hearer in the outside world and in the speech situation. Papuan Malay employs a two-term demonstrative system and a three-term locative system as shown in Table  ‎7 .1.


\begin{stylecaption}
\label{bkm:Ref322434439}Table ‎7.\stepcounter{Table}{\theTable}:  Papuan Malay demonstratives and locatives
\end{stylecaption}

\begin{tabular}{llll}
\lsptoprule

\multicolumn{2}{l}{ Papuan Malay \textsc{dem}} & \multicolumn{2}{l}{ Gloss}\\
Proximal & \textitbf{ini} / \textitbf{ni} & \textsc{\textup{‘}}\textsc{d.prox}\textsc{\textup{’}} & ‘this’\\
Distal & \textitbf{itu} /\textitbf{tu} & ‘\textsc{d.dist}’ & ‘that’\\
\multicolumn{2}{l}{ Papuan Malay \textsc{loc}} & \multicolumn{2}{l}{ Gloss}\\
Proximal & \textitbf{sini} & ‘\textsc{l.prox}’ & ‘here’\\
Medial & \textitbf{situ} & ‘\textsc{l.med}’ & ‘there’\\
Distal & \textitbf{sana} & ‘\textsc{l.dist}’ & ‘over there’\\
\lspbottomrule
\end{tabular}

Both systems are distance oriented. Cross-linguistically, such systems “indicate the relative distance of a referent in the speech situation vis-à-vis the deictic center” which “is defined by the speaker’s location at the time of the utterance” {\citep[430]{Diessel2006}}. The unmarked deictic center is defined by the speaker in the ‘here’ and ‘now’. In reported direct speech and narratives, however, the deictic center is readjusted to the reported speech situation and defined by the quoted speakers and the location and time of their speaking.
\end{styleBodyaftervbefore}


While their major domain of use is to provide spatial orientation, the demonstratives and locatives also signal distance in metaphorical terms. Their main domains of use are presented in Table  ‎7 .2.
\end{styleBodyvvafter}

\begin{stylecaption}
\label{bkm:Ref320270864}Table ‎7.\stepcounter{Table}{\theTable}:  Demonstratives (\textsc{dem}) and locatives (\textsc{loc}) and their of domains of use
\end{stylecaption}

\tablehead{
 Domains of use & Function & \textsc{dem} & \arraybslash \textsc{loc}\\
}
\begin{tabular}{llll}
\lsptoprule
Spatial & to provide spatial orientation to the hearer & X & \arraybslash X\\
Figurative locational & to signal a figurative locational endpoint &  & \arraybslash X\\
Temporal & to indicate the temporal setting of the situation or event talked about & X & \arraybslash X\\
Psychological & to signal the speaker’s psychological involvement with the situation or event talked about & X & \arraybslash X\\
Identificational & to aid in the identification of referents & X & \\
Textual anaphoric & to keep track of a discourse antecedent & X & \arraybslash X\\
Textual discourse deictic & to establish an overt link between two propositions & X & \\
Placeholder & to substitute for specific lexical items in the context of word-formulation trouble & X & \\
\lspbottomrule
\end{tabular}

In the following sections the demonstratives (§7.1) and locatives (§7.2) are investigated in more detail. The ways in which the demonstratives and locatives can be combined are discussed in §7.3. The different findings for the demonstratives and locatives are summarized and compared in §7.4.


\section{Demonstratives}
\label{bkm:Ref322435473}
In the following sections, the syntactic properties and forms of the Papuan Malay demonstratives are reviewed and discussed (§7.1.1), followed by an in-depth discussion of their different functions and domains of use (§7.1.2).
\end{styleBodyxvafter}

\subsection{Syntax and forms of demonstratives}
\label{bkm:Ref322436100}
The distributional properties of the demonstratives are briefly reviewed in §7.1.1.1. This review is followed in §7.1.1.2 by a discussion of the distribution and frequencies of the long and short demonstrative forms.
\end{styleBodyxvafter}

\paragraph[Distributional properties of demonstratives]{Distributional properties of demonstratives}
\label{bkm:Ref322449679}
The Papuan Malay demonstratives have the following distributional properties (for more details see §5.6):


%\setcounter{itemize}{0}
\begin{itemize}
\item \begin{styleIIndented}
Co-occurrence with noun phrases (adnominal uses): \textsc{n}/\textsc{np} \textsc{dem} (§5.6.1)
\end{styleIIndented}\item \begin{styleIIndented}
Substitution for noun phrases (pronominal uses) (§5.6.2)
\end{styleIIndented}\item \begin{styleIIndented}
Modification with relative clauses (pronominal uses): \textsc{dem rel} (§5.6.2).
\end{styleIIndented}\item \begin{styleIIndented}
Co-occurrence with verbs or adverbs (adverbial uses): \textsc{v dem} and \textsc{adv dem} (§5.6.3)
\end{styleIIndented}\item \begin{styleIiiI}
Stacking of demonstratives: \textsc{dem} \textsc{dem} and \textsc{n} \textsc{dem} \textsc{dem} (§5.6.4)
\end{styleIiiI}\end{itemize}
\paragraph[Distribution of the long versus short demonstrative forms]{Distribution of the long versus short demonstrative forms}
\label{bkm:Ref320894463}
This section investigates the distribution and frequencies of the long versus the short demonstrative forms and explores the factors that contribute to this distribution. The data show that the reduced demonstrative forms are fast-speech phenomena that fulfill the same syntactic functions as the long forms. With two exceptions they are also employed in the same domains of use.



The corpus contains a total of 2,304 \textitbf{ini} ‘\textsc{d.prox}’ tokens of which 2,046 (88.8\%) are the long form and 258 (11.2\%) the short form, as shown in Table  ‎7 .3. The number of \textitbf{itu} ‘\textsc{d.dist}’ token is considerably larger with a total of 4,159 token of which 3,491 (83.9\%) are the long form and 668 (16.1\%) the short form.
\end{styleBodyvvafter}

\begin{stylecaption}
\label{bkm:Ref320896268}Table ‎7.\stepcounter{Table}{\theTable}:  Demonstratives according to their phonological environment
\end{stylecaption}

\begin{tabular}{lllllllll} & \multicolumn{2}{l}{ \textitbf{ini}} & \multicolumn{2}{l}{ \textitbf{itu}} & \multicolumn{2}{l}{ \textitbf{ni}} & \multicolumn{2}{l}{ \textitbf{tu}}\\
\lsptoprule
Clause-initial & \raggedleft 86 & \raggedleft 4\% & \raggedleft 279 & \raggedleft 8\% & \raggedleft 7 & \raggedleft 3\% & \raggedleft 28 & \raggedleft\arraybslash 4\%\\
Post-vowel & \raggedleft 1,156 & \raggedleft 57\% & \raggedleft 1,671 & \raggedleft 48\% & \raggedleft 222 & \raggedleft 86\% & \raggedleft 531 & \raggedleft\arraybslash 80\%\\
Post-nasal & \raggedleft 432 & \raggedleft 21\% & \raggedleft 833 & \raggedleft 24\% & \raggedleft 18 & \raggedleft 7\% & \raggedleft 67 & \raggedleft\arraybslash 10\%\\
Post-consonant & \raggedleft 372 & \raggedleft 18\% & \raggedleft 708 & \raggedleft 20\% & \raggedleft 11 & \raggedleft 4\% & \raggedleft 42 & \raggedleft\arraybslash 6\%\\
Total & \raggedleft 2,046 & \raggedleft 100\% & \raggedleft 3,491 & \raggedleft 100\% & \raggedleft 258 & \raggedleft 100\% & \raggedleft 668 & \raggedleft\arraybslash 100\%\\
\lspbottomrule
\end{tabular}

The long forms occur in all phonological environments with Table  ‎7 .3 indicating some differences in distribution, however. First, the high number of long demonstratives (about 50\%) that follow lexical items with word-final vowels is due to the fact that in Papuan Malay more lexical items have a word-final vowel than a word-final nasal or consonant. Second, the low number of long demonstratives ({\textless}10\%) in clause-initial position is due to the fact that the number of pronominally used demonstratives is much lower than that of adnominally or adverbially used ones.
\end{styleBodyaftervbefore}


The short forms also occur in all phonological environments. Table  ‎7 .3 shows, however, that most of them (${\geq}$80\%) occur after lexical items with a word-final vowel. The percentage of short demonstratives following lexical items with word-final nasal or consonant is considerably lower compared to the long forms.
\end{styleBodyvafter}


Interestingly, the short demonstratives also occur in clause-initial position: of the seven short \textitbf{ini} ‘\textsc{d.prox}’ tokens occurring clause-initially, five occur at the beginning of an utterance. The remaining two tokens occur clause-initially in the middle of an utterance. In both cases the preceding clause-final lexical item has a word-final vowel which appears to condition these two short forms. Of the 28 short \textitbf{itu} ‘\textsc{d.dist}’ tokens occurring clause-initially, eleven occur at the beginning of an utterance. The remaining 17 tokens occur clause-initially in the middle of an utterance. Of these, 14 tokens are conditioned by the preceding word-final phoneme: in eleven cases the preceding clause-final lexical item has a word-final vowel while in the remaining three cases the preceding clause-final lexical item has coda /t/.
\end{styleBodyvafter}


These findings suggest that for the most part the short demonstrative forms are conditioned by the environment of their occurrence and constitute fast-speech phenomena. The listed exceptions require further investigation.
\end{styleBodyvafter}


The short demonstratives fulfill the same syntactic functions as the long ones. With two exceptions they are also employed in the same domains of use. The data does, however, suggest some preferences. Table  ‎7 .4 presents the short demonstratives according to their syntactic functions. The data show a clear preference for their adnominal uses. Of the 209 adnominally used short \textitbf{ini} ‘\textsc{d.prox}’ tokens, 170 (81\%) modify noun phrases with nominal heads, while 34 (16\%) modify noun phrases with pronominal heads; the remaining five tokens modify interrogatives. Likewise, of the 482 adnominally used short \textitbf{itu} ‘\textsc{d.dist}’ tokens, 345 (72\%) modify noun phrases with nominal heads, while 105 (22\%) modify noun phrases with pronominal heads, and 30 (6\%) modify locatives; the remaining two tokens modify interrogatives. Considerably less frequently, the short demonstratives have pronominal uses (${\leq}$8\%) or adverbial uses (${\leq}$13\%). (Given their large numbers, the long demonstratives have not been quantified according to their syntactic functions.)
\end{styleBodyvvafter}

\begin{stylecaption}
\label{bkm:Ref320896269}Table ‎7.\stepcounter{Table}{\theTable}:  Reduced demonstratives according to their syntactic functions
\end{stylecaption}

\begin{tabular}{lllll}
\lsptoprule

 Syntactic functions & \multicolumn{2}{l}{ \textitbf{ni}} & \multicolumn{2}{l}{ \textitbf{tu}}\\
Adnominal uses & \raggedleft 209 & \raggedleft 81\% & \raggedleft 482 & \raggedleft\arraybslash 72\%\\
Pronominal uses & \raggedleft 20 & \raggedleft 8\% & \raggedleft 44 & \raggedleft\arraybslash 7\%\\
Adverbial uses (verbal modifier) & \raggedleft 20 & \raggedleft 8\% & \raggedleft 86 & \raggedleft\arraybslash 13\%\\
Adverbial uses (adverbial modifier) & \raggedleft 9 & \raggedleft 3\% & \raggedleft 56 & \raggedleft\arraybslash 8\%\\
Total & \raggedleft 258 & \raggedleft 100\% & \raggedleft 668 & \raggedleft\arraybslash 100\%\\
\lspbottomrule
\end{tabular}

The short demonstratives are also employed in the same domains of use as the long ones, except for the identificational and placeholder uses. Given their rather frequent alternative readings (§7.1.2), the demonstratives have not been quantified according to their domains of use.
\end{styleBodyaftervbefore}


Table  ‎7 .5 and Table  ‎7 .6 present an overview of the short demonstratives according to their syntactic functions and domains of use. For \textitbf{ini} ‘\textsc{d.prox}’ in Table  ‎7 .5, its syntactic functions are exemplified as follows: adnominal in (1) to (4); pronominal in (5); and adverbial in (6) and (7). Its domains of use are given as follows: spatial in (1) and (5); temporal in (1); psychological in (1) (emotional involvement), (3), (4), (6), (7) (vividness), and (2) (contrast); and textual in (5) (anaphoric) and (5) (discourse deictic).
\end{styleBodyvvafter}

\begin{stylecaption}
\label{bkm:Ref320896270}Table ‎7.\stepcounter{Table}{\theTable}:  Short \textitbf{ini} ‘\textsc{d.prox}’: Syntactic functions and domains of use\footnote{\\
\\
\\
\\
\\
\\
\\
\\
\\
\\
\\
\\
\\
\\
\par Documentation: (1) 081025-003-Cv.0042, 080918-001-CvNP.0055, 081025-003-Cv.0135; (2) 081014-007-Pr.0053; (3) 080922-010a-NF.0101; (4) 080922-004-Cv.0017; (5) 080922-010a-NF.0081, 080917-006-CvHtEx.0005, 080917-010-CvEx.0116; (6) 080919-005-Cv.0015; (7) 080922-001a-CvPh.0735.\\
}
\end{stylecaption}

\tablehead{ & Papuan Malay with gloss & \arraybslash Free translation\\
}
\begin{tabular}{lll}
\lsptoprule
\multicolumn{3}{l}{%\setcounter{itemize}{0}
\begin{itemize}
\item \label{bkm:Ref321579119}Adnominal uses (modifies nouns) / Spatial uses in (), temporal uses in (), psychological uses (emotional involvement) in ()\end{itemize}
}\\
\label{bkm:Ref363302048} & \textitbf{bawa }\textitbfUndl{mace ni}\textitbf{ ke ruma-sakit} & ‘(I) brought \bluebold{(my) wife here} to the hospital’\\
& bring woman \textsc{d.prox} to hospital & \\
\label{bkm:Ref363302049} & \textitbf{sekertaria ni }\textitbfUndl{pagi ni}\textitbf{ sedi …} & ‘\bluebold{this morning} the secretary was (very) sad’\footnotemark{}\\
& secretariat \textsc{d.prox} morning\textsc{ d.prox} be.sad & \\
\label{bkm:Ref316473801} & \textitbf{kalo }\textitbfUndl{Ise ni}\textitbf{ selesay …} & ‘when \bluebold{this (my daughter) Ise} has finished (school) …’\footnotemark{}\\
& if Ise \textsc{d.prox} finish & \\
\multicolumn{3}{l}{\begin{itemize}
\item \label{bkm:Ref321579120}Adnominal uses (modifies personal pronouns) / Psychological uses (contrast)\end{itemize}
}\\
& \textitbfUndl{kitong ni}\textitbf{ tra bisa} & ‘\bluebold{we, by contrast}, can’t (work like that)’\\
& \textsc{1pl} \textsc{d.prox} \textsc{neg} be.able & \\
\multicolumn{3}{l}{\begin{itemize}
\item \label{bkm:Ref321579121}Adnominal uses (modifies demonstratives) / Psychological uses (vividness)\end{itemize}
}\\
& \textitbf{dia buang }\textitbfUndl{ini ni} & ‘he threw away \bluebold{these very (ones)}’\\
& \textsc{3sg} throw(.away) \textsc{d.prox} \textsc{d.prox} & \\
\multicolumn{3}{l}{\begin{itemize}
\item \label{bkm:Ref321579122}Adnominal uses (modifies interrogatives) / Psychological uses (vividness)\end{itemize}
}\\
& \textitbf{de mo ke }\textitbfUndl{mana ni} & ‘\bluebold{where (}\blueboldSmallCaps{emph}\bluebold{)} does he want to go?’\\
& \textsc{3sg} want to where \textsc{d.prox} & \\
\multicolumn{3}{l}{\begin{itemize}
\item \label{bkm:Ref321579123}Pronominal uses / Spatial uses in (), textual uses (anaphoric) in (), textual uses (discourse deictic) in ()\end{itemize}
}\\
\label{bkm:Ref321578071} & \textitbf{ada} \textitbfUndl{ni} & ‘(the fish) are \bluebold{here}’\\
& exist \textsc{d.prox} & \\
\label{bkm:Ref321578075} & \textitbf{de menyala }\textitbfUndl{ni} & ‘he puts fire to \bluebold{this}’\\
& \textsc{3sg} put.fire.to \textsc{d.prox} & \\
\label{bkm:Ref321578077} & \textitbfUndl{ni}\textitbf{ usul saja} & ‘\bluebold{this} is just a proposal’\\
& \textsc{d.prox} proposal just & \\
\multicolumn{3}{l}{\begin{itemize}
\item \label{bkm:Ref321579127}Adverbial uses (modifies verbs) / Psychological uses (vividness)\end{itemize}
}\\
& \textitbf{sa masi }\textitbfUndl{hidup ni} & ‘I’m still \bluebold{very much alive}’\\
& \textsc{1sg} still live \textsc{d.prox} & \\
\multicolumn{3}{l}{\begin{itemize}
\item \label{bkm:Ref321579129}Adverbial uses (modifies adverbs) / Psychological uses (vividness)\end{itemize}
}\\
& \textitbf{… tapi }\textitbfUndl{skarang ni}\textitbf{ ada libur} & ‘but \bluebold{right now} (we) are on vacation’\\
& … but now \textsc{d.prox}\textsc{\textup{ }}exist vacation & \\
\hhline{--~}
\lspbottomrule
\end{tabular}
\addtocounter{footnote}{-2}
\stepcounter{footnote}\footnotetext{\\
\\
\\
\\
\\
\\
\\
\\
\\
\\
\\
\\
\\
\\
The speaker made a mistake; instead of saying \textitbf{sekertaris} ‘secretary’, he produced \textitbf{sekertaria} ‘secretariat’.\\
}
\stepcounter{footnote}\footnotetext{\\
\\
\\
\\
\\
\\
\\
\\
\\
\\
\\
\\
\\
\\
The referent \textitbf{Ise} was not present when the speaker talked about his daughter; hence proximal \textitbf{ini} ‘\textsc{d.prox}’ does not have spatial uses in this context.\\
}

Table  ‎7 .6 lists the syntactic functions and domains of use of \textitbf{itu} ‘\textsc{d.dist}’. Its syntactic functions are listed as follows: adnominal in (1) to (5); pronominal in (6); and adverbial in (7) and (8). Its domains of use are given as follows: spatial in (6); temporal in (1); psychological in (1), (4) (emotional involvement), (3) (7) (vividness), and (2), (8) (contrast); textual in (1) (anaphoric), and (6), (6) (discourse deictic).


\begin{stylecaption}
\label{bkm:Ref320896273}Table ‎7.\stepcounter{Table}{\theTable}:  Short \textitbf{itu} ‘\textsc{d.dist}’: Syntactic functions and domains of use\footnote{\\
\\
\\
\\
\\
\\
\\
\\
\\
\\
\\
\\
\\
\\
\par Documentation: (1) 081011-005-Cv.0001, 081025-009b-Cv.0016, 081014-004-Cv.0019; (2) 080922-001a-CvPh.0455; (3) 081115-001a-Cv.0145; (4) 081109-001-Cv.0092; (5) 081006-022-CvEx.0150; (6) 081025-009b-Cv.0006, 081006-022-CvEx.0113, 081013-002-Cv.0003; (7) 081023-001-Cv.0020; (8) 081115-001a-Cv.0058.\\
}
\end{stylecaption}

\tablehead{ & Papuan Malay with gloss & \multicolumn{2}{l}{ Free translation}\\
}
\begin{tabular}{llll}
\lsptoprule
\multicolumn{4}{l}{%\setcounter{itemize}{0}
\begin{itemize}
\item \label{bkm:Ref321581105}Adnominal uses (modifies nouns) / Temporal uses in (), psychological uses (emotional involvement) in (), discourse uses (anaphoric) in ()\end{itemize}
}\\
\label{bkm:Ref316482647} & \textitbfUndl{rabu tu}\textitbf{ … ko datang} & \multicolumn{2}{l}{‘\bluebold{next Wednesday }you’ll come’}\\
& Wednesday \textsc{d.dist} \textsc{2sg} come &  & \\
\label{bkm:Ref321581108} & \textitbfUndl{ko pu swara tu}\textitbf{ bahaya} & \multicolumn{2}{l}{‘\bluebold{that voice of yours} is dangerous’}\\
& 2\textsc{sg} \textsc{poss} voice \textsc{d.dist} be.dangerous &  & \\
\label{bkm:Ref316482651} & \textitbfUndl{Herman tu}\textitbf{ biasa tida …} & \multicolumn{2}{l}{‘\bluebold{that Herman} usually (can)not …’}\\
& Herman \textsc{d.dist} be.usual \textsc{neg} … &  & \\
\multicolumn{4}{l}{\begin{itemize}
\item \label{bkm:Ref321581111}Adnominal uses (modifies personal pronouns) / Psychological uses (contrast)\end{itemize}
}\\
& \textitbfUndl{sa tu}\textitbf{ rajing skola} & \multicolumn{2}{l}{‘\bluebold{I, nonetheless}, go to school diligently’}\\
& \textsc{1sg} \textsc{d.dist} be.diligent go.to.school &  & \\
\multicolumn{4}{l}{\begin{itemize}
\item \label{bkm:Ref321581112}Adnominal uses (modifies demonstratives) / Psychological uses (vividness)\end{itemize}
}\\
& \textitbfUndl{itu tu}\textitbf{ kata}{\Tilde}\textitbf{kata dasar …} & \multicolumn{2}{l}{‘\bluebold{that very} (word belongs to) the basic words …’}\\
& \textsc{d.dist} \textsc{d.dist} \textsc{rdp}{\Tilde}word base &  & \\
\multicolumn{4}{l}{\begin{itemize}
\item \label{bkm:Ref321581113}Adnominal uses (modifies interrogatives) / Psychological uses (emotional involvement)\end{itemize}
}\\
& \textitbf{itu }\textitbfUndl{apa tu}\textitbf{?} & \multicolumn{2}{l}{‘\bluebold{what (}\blueboldSmallCaps{emph}\bluebold{)} was that?}\\
& \textsc{d.dist} what \textsc{d.dist} &  & \\
\multicolumn{4}{l}{\begin{itemize}
\item \label{bkm:Ref321581116}Adnominal uses (modifies locatives) / Psychological uses (vividness)\end{itemize}
}\\
& \textitbf{di }\textitbfUndl{sini tu}\textitbf{ ada orang swanggi satu} & \multicolumn{2}{l}{‘\bluebold{here (}\blueboldSmallCaps{emph}\bluebold{)} there is a certain evil sorcerer’}\\
& at \textsc{l.prox} \textsc{d.dist} exist person nocturnal.evil.spirit one &  & \\
\multicolumn{4}{l}{\begin{itemize}
\item \label{bkm:Ref321581117}Pronominal uses / Spatial uses in (),textual uses (discourse deictic) in () and ()\end{itemize}
}\\
\label{bkm:Ref316482637} & \multicolumn{2}{l}{\textitbf{de ada }\textitbfUndl{tu}\textitbf{, de ada }\textitbfUndl{tu}} & ‘she is \bluebold{over there}, she is \bluebold{over there}’\\
& \multicolumn{2}{l}{\textsc{3sg} exist \textsc{d.dist} \textsc{3sg} exist \textsc{d.dist}} & \\
\label{bkm:Ref316482638} & \multicolumn{2}{l}{\textitbf{dorang liat kitorang, }\textitbfUndl{tu}\textitbf{ herang}} & ‘they see us, \bluebold{that’s} surprising’\\
& \multicolumn{2}{l}{\textsc{3pl} see \textsc{1pl} \textsc{d.dist} feel surprised (about)} & \\
\label{bkm:Ref316482640} & \multicolumn{2}{l}{\textitbfUndl{tu yang}\textitbf{ sa tampeleng Aleks}} & ‘\bluebold{that’s why} I slapped Aleks in the face’\\
& \multicolumn{2}{l}{\textsc{d.dist} \textsc{rel} \textsc{1sg} slap Aleks} & \\
\multicolumn{4}{l}{\begin{itemize}
\item \label{bkm:Ref321581123}Adverbial uses (modifies verbs) / Psychological uses (vividness)\end{itemize}
}\\
& \textitbf{tong }\textitbfUndl{maing tu}\textitbf{ hancur} & \multicolumn{2}{l}{‘we did our \bluebold{very playing} poorly’}\\
& \textsc{1pl} play \textsc{d.dist} be.shattered &  & \\
\multicolumn{4}{l}{\begin{itemize}
\item \label{bkm:Ref321581125}Adverbial uses (modifies adverbs) / Psychological uses (contrast)\end{itemize}
}\\
& \textitbf{de}\bluebold{ }\textitbfUndl{skarang tu}\textitbf{ tida terlalu …} & \multicolumn{2}{l}{‘he’s \bluebold{now (as opposed to the past)} not too …’}\\
& \textsc{3sg} now \textsc{d.dist} \textsc{neg} too &  & \\
\hhline{--~~}
\lspbottomrule
\end{tabular}

In summary, the data in the corpus suggests that the short demonstrative forms are fast-speech phenomena that for the most part are conditioned by their phonological environment. The data also show that the long and short demonstrative forms fulfill the same syntactic functions. Moreover, they are employed in the same domains of use with two exceptions (the identificational and placeholder uses).


\subsection{Functions of demonstratives}
\label{bkm:Ref322436088}
The Papuan Malay demonstratives have a range of different functions and uses which are discussed in more detail in the following sections: spatial uses in §7.1.2.1, temporal uses in §7.1.2.2, psychological uses in §7.1.2.3, identificational uses in §7.1.2.4, textual uses in §7.1.2.5, and placeholder uses in §7.1.2.6. Unless the context of an utterance is clear and explicit, the specific domain of use of the demonstrative may have multiple possible readings
\end{styleBodyxvafter}

\paragraph[Spatial uses of demonstratives]{Spatial uses of demonstratives}
\label{bkm:Ref320540699}
The major domain of use for the demonstratives is to provide spatial orientation. This is achieved by drawing the hearer’s attention to specific entities in the discourse or surrounding situation. Proximal \textitbf{ini} ‘\textsc{d.prox}’ indicates that the referent is conceived as spatially close to the speaker, whereas \textitbf{itu} ‘\textsc{d.dist}’ signals that the referent is conceived as being located further away. This distinction is shown in three sets of contrastive examples.



In the first set of examples in (0) the contrast is illustrated for the adnominally used demonstratives, each of them modifying the common noun \textitbf{ruma} ‘house’. This example is part of a conversation that took place at the speaker’s house. Employing \textitbf{ini} ‘\textsc{d.prox}’, the speaker relates her plans to move from her current house, \textitbf{ruma ini} ‘this house’, to a different house in a neighboring village. Because the new house is smaller than the older one, the speaker’s husband is going to enlarge \textitbf{ruma itu} ‘that house’, with \textitbf{itu} ‘\textsc{d.dist}’ indicating that the new house is located at some distance.
\end{styleBodyvvafter}

\begin{styleExampleTitle}
Spatial uses: Examples set \#1
\end{styleExampleTitle}

\begin{tabular}{llllllllllllllllllll}
\lsptoprule
\label{bkm:Ref338955246}
\gll {\multicolumn{2}{l}{ini}} {\multicolumn{3}{l}{kasi}} {\multicolumn{3}{l}{tinggal,}} {\multicolumn{5}{l}{ana{\Tilde}ana}} {\multicolumn{2}{l}{dong}} {\multicolumn{2}{l}{tinggal,}} {tong} {pi}\\ %
& \multicolumn{2}{l}{\textsc{d.prox}} & \multicolumn{3}{l}{give} & \multicolumn{3}{l}{stay} & \multicolumn{5}{l}{\textsc{rdp}{\Tilde}child} & \multicolumn{2}{l}{\textsc{3pl}} & \multicolumn{2}{l}{stay} & \textsc{1pl} & go\\
& tinggal & \multicolumn{3}{l}{di} & \multicolumn{3}{l}{Sawar} & \multicolumn{3}{l}{sana} & … & \multicolumn{3}{l}{\bluebold{ruma}} & \multicolumn{2}{l}{\bluebold{ini}} & \multicolumn{2}{l}{tinggal} & …\\
& stay & \multicolumn{3}{l}{at} & \multicolumn{3}{l}{Sawar} & \multicolumn{3}{l}{\textsc{l.dist}} &  & \multicolumn{3}{l}{house} & \multicolumn{2}{l}{\textsc{d.prox}} & \multicolumn{2}{l}{stay} & \\
& \multicolumn{3}{l}{baru} & \multicolumn{3}{l}{\bluebold{ruma}} & \multicolumn{3}{l}{\bluebold{itu}} & \multicolumn{3}{l}{biking} & \multicolumn{7}{l}{besar}\\
& \multicolumn{3}{l}{and.then} & \multicolumn{3}{l}{house} & \multicolumn{3}{l}{\textsc{d.dist}} & \multicolumn{3}{l}{make} & \multicolumn{7}{l}{be.big}\\
\lspbottomrule
\end{tabular}
\ea
\glt 
‘(we’ll) leave this (house) behind, the children will stay (here), (and) we’ll move to Sawar over there …, (we’ll) leave \bluebold{this house} behind … and then (we’ll) make \bluebold{that house} (in Sawar) bigger’ \textstyleExampleSource{[081110-001-Cv.0012/0022/0025/0027]}
\z


The second set of examples in (0) and (0) illustrates how the pronominally used demonstratives signal spatial distance. The example in (0) occurred when the speaker and his brother were fishing. When asked where he had put the fish they had just caught, the speaker employs \textitbf{ini} ‘\textsc{d.prox}’ to convey that the fish \textitbf{ada ini} ‘are here’, in the bucket right next to him. In (0), the speaker replies to the question where a certain other person was. Employing \textitbf{itu} ‘\textsc{d.dist}’ the speaker states that \textitbf{de ada tu} ‘she is over there’.


\begin{styleExampleTitle}
Spatial uses: Examples set \#2
\end{styleExampleTitle}

\begin{tabular}{lll}
\lsptoprule
\label{bkm:Ref338955250}
\gll {ada} {\bluebold{ni}}\\ %
& exist & \textsc{d.prox}\\
\lspbottomrule
\end{tabular}
\ea
\glt 
[Reply to a question:] ‘(the fish) are \bluebold{here}’ \textstyleExampleSource{[080917-006-CvHt.0005]}
\z

\begin{tabular}{lllllll}
\lsptoprule
\label{bkm:Ref338955256}
\gll {de} {ada} {\bluebold{tu},} {de} {ada} {\bluebold{tu}}\\ %
& \textsc{3sg} & exist & \textsc{d.dist} & \textsc{3sg} & exist & \textsc{d.dist}\\
\lspbottomrule
\end{tabular}
\ea
\glt 
[Reply to a question:] ‘she’s \bluebold{over there}, she’s \bluebold{over there}’ \textstyleExampleSource{[081025-009b-Cv.0006]}
\z


In the third set of examples in (0) and (0), the demonstratives are used adverbially. The utterance in (0) occurred during a discussion about the teenagers living in the house. Noting that they are ill-behaved, the speaker uses \textitbf{ini} ‘\textsc{d.prox}’ to assert that they \textitbf{tinggal ini} ‘live here’ in this house. In (0) the speaker relates that she used to live in a different part of the regency, namely in Takar. Employing \textitbf{itu} ‘\textsc{d.dist}’, the speaker maintains that she used to \textitbf{tinggal itu} ‘live there’.


\begin{styleExampleTitle}
Spatial uses: Examples set \#3
\end{styleExampleTitle}

\begin{tabular}{llll}
\lsptoprule
\label{bkm:Ref338955257}
\gll {ko} {\bluebold{tinggal}} {\bluebold{ini}}\\ %
& \textsc{2sg} & stay & \textsc{d.prox}\\
\lspbottomrule
\end{tabular}
\ea
\glt 
‘you \bluebold{live here}’ \textstyleExampleSource{[081115-001b-Cv.0030]}
\z

\begin{tabular}{lllllllllllllll}
\lsptoprule
\label{bkm:Ref338955259}
\gll {\multicolumn{2}{l}{waktu}} {\multicolumn{2}{l}{kitong}} {dari} {\multicolumn{2}{l}{Jayapura}} {\multicolumn{2}{l}{baru}} {\multicolumn{3}{l}{pulang}} {ke} {kampung}\\ %
& \multicolumn{2}{l}{when} & \multicolumn{2}{l}{\textsc{1pl}} & from & \multicolumn{2}{l}{Jayapura} & \multicolumn{2}{l}{and.then} & \multicolumn{3}{l}{go.home} & to & village\\
& di & \multicolumn{2}{l}{Takar} & \multicolumn{3}{l}{Pante-Timur,} & \multicolumn{2}{l}{baru} & \multicolumn{2}{l}{kitong} & \bluebold{tinggal} & \multicolumn{3}{l}{\bluebold{itu}}\\
& at & \multicolumn{2}{l}{Takar} & \multicolumn{3}{l}{Pante-Timur} & \multicolumn{2}{l}{and.then} & \multicolumn{2}{l}{\textsc{1pl}} & stay & \multicolumn{3}{l}{\textsc{d.dist}}\\
\lspbottomrule
\end{tabular}
\ea
\glt
‘when we (were back) from Jayapura, then (we) returned home to the village at Takar at Pante-Timur, and then we \bluebold{lived there}’ \textstyleExampleSource{[081006-022-CvEx.0159]}
\end{styleFreeTranslEngxvpt}

\paragraph[Temporal uses of demonstratives]{Temporal uses of demonstratives}
\label{bkm:Ref320351329}
In their temporal uses, the demonstratives signal the temporal setting of the situation or event talked about in terms of some temporal reference point. This function is attested for the adnominally, pronominally, and adverbially used demonstratives.



The (near) contrastive examples in (0) to (0) demonstrate the temporal uses of the adnominally used demonstratives. Proximal \textitbf{ini} ‘\textsc{d.prox}’ signals that the event is temporally close to the current speech situation, as in \textitbf{hari ni} ‘today’ in (0). By contrast, \textitbf{itu} ‘\textsc{d.dist}’ indicates that the temporal reference point is located at some distance from the current speech situation, either in the past as in \textitbf{hari itu} ‘that day’ in (0) or in the future as in \textitbf{rabu tu, hari kamis itu} ‘next Wednesday, next Thursday’ in (0).
\end{styleBodyvvafter}

\begin{styleExampleTitle}
Temporal uses of the adnominally used demonstratives
\end{styleExampleTitle}

\begin{tabular}{llllllll}
\lsptoprule
\label{bkm:Ref338955260}
\gll {\bluebold{hari}} {\bluebold{ni}} {ko} {kasi} {makang,} {nanti} {…}\\ %
& day & \textsc{d.prox} & \textsc{2sg} & give & food & very.soon & \\
\lspbottomrule
\end{tabular}
\ea
\glt 
[About helping each other:] ‘\bluebold{today} you feed (others), at some point in the future [they’ll feed your children]’ \textstyleExampleSource{[081110-008-CvNP.0254]}
\z

\begin{tabular}{lllllllllllll}
\lsptoprule
\label{bkm:Ref338955261}
\gll {yo,} {\multicolumn{2}{l}{dong}} {\multicolumn{2}{l}{dua}} {pergi} {ke} {skola} {lagi,} {\bluebold{hari}} {\bluebold{itu}} {dong}\\ %
& yes & \multicolumn{2}{l}{\textsc{3pl}} & \multicolumn{2}{l}{two} & go & to & school & again & day & \textsc{d.dist} & \textsc{3pl}\\
& \multicolumn{2}{l}{ada} & \multicolumn{2}{l}{meter} & \multicolumn{8}{l}{sedikit}\\
& \multicolumn{2}{l}{exist} & \multicolumn{2}{l}{meter} & \multicolumn{8}{l}{few}\\
\lspbottomrule
\end{tabular}
\ea
\glt 
‘yes, they both went to school, \bluebold{that day} they were a little drunk’ \textstyleExampleSource{[081115-001a-Cv.0038]}
\z

\begin{tabular}{lllllllll}
\lsptoprule
\label{bkm:Ref338955262}
\gll {\bluebold{rabu}} {\bluebold{tu},} {\bluebold{hari}} {\bluebold{kamis}} {\bluebold{itu},} {ko} {datang} {…}\\ %
& Wednesday & \textsc{d.dist} & day & Thursday & \textsc{d.dist} & \textsc{2sg} & come & \\
\lspbottomrule
\end{tabular}
\ea
\glt 
‘\bluebold{next Wednesday, next Thursday}, you’ll come …’ \textstyleExampleSource{[081011-005-Cv.0001]}
\z


Pronominally used demonstratives also have temporal uses as shown in (0) and (0). Again, \textitbf{ini} ‘\textsc{d.prox}’ in (0) indicates that the event is temporally close to the current speech situation: \textitbf{ini} ‘right now’ (literally ‘this (is when)’). Distal \textitbf{itu} ‘\textsc{d.dist}’ in (0), by contrast, signals temporal distance: \textitbf{itu} ‘at that time’ (literally ‘that (is when)’).


\begin{styleExampleTitle}
Temporal uses of the pronominally used demonstratives
\end{styleExampleTitle}

\begin{tabular}{llllllll}
\lsptoprule
\label{bkm:Ref320351531}
\gll {mandi} {cepat{\Tilde}cepat,} {\bluebold{ini}} {tong} {mo} {lanjut} {lagi}\\ %
& bathe & \textsc{rdp}{\Tilde}be.fast & \textsc{d.prox} & \textsc{1pl} & want & continue & again\\
\lspbottomrule
\end{tabular}
\ea
\glt 
‘bathe very quickly, \bluebold{right now} we want to continue further’ \textstyleExampleSource{[080917-008-NP.0134]}
\z

\begin{tabular}{llllllllll}
\lsptoprule
\label{bkm:Ref320351532}
\gll {satu} {kali} {tong} {pergi} {berdoa} {…} {\bluebold{itu}} {de} {ikut}\\ %
& one & time & \textsc{1pl} & go & pray &  & \textsc{d.dist} & \textsc{3sg} & follow\\
\lspbottomrule
\end{tabular}
\ea
\glt 
‘one time we went to pray …, \bluebold{at that time} she (my daughter) also followed (us)’ \textstyleExampleSource{[080917-008-NP.0175]}
\z


The temporal uses of the adverbially used demonstratives are illustrated in (0) and (0). Again, \textitbf{ini} ‘\textsc{d.prox}’ signals temporal proximity as in \textitbf{ada datang ini} ‘is coming right now’ in (0), while \textitbf{itu} ‘\textsc{d.dist}’ indicates temporal distance as in \textitbf{bangung itu} ‘woke up at that time’ in (0).


\begin{styleExampleTitle}
Temporal uses of the adverbially used demonstratives
\end{styleExampleTitle}

\begin{tabular}{lllllllll}
\lsptoprule
\label{bkm:Ref338955266}
\gll {…} {o,} {betul,} {Papua-Satu} {ini} {\bluebold{ada}} {\bluebold{datang}} {\bluebold{ini}}\\ %
&  & oh! & true & Papua-Satu & \textsc{d.prox} & exist & come & \textsc{d.prox}\\
\lspbottomrule
\end{tabular}
\ea
\glt 
‘[and then we saw,] ‘oh!, (it’s) true, this Papua-Satu (ship) is \bluebold{coming right now}’’ (Lit. ‘\bluebold{this coming}’) \textstyleExampleSource{[080917-008-NP.0130]}
\z

\begin{tabular}{llllllllllllll}
\lsptoprule
\label{bkm:Ref338955267}
\gll {sa} {bawa} {\multicolumn{2}{l}{pulang}} {…} {\multicolumn{3}{l}{mace}} {\multicolumn{2}{l}{\bluebold{bangung}}} {\bluebold{itu}} {dia} {suda}\\ %
& \textsc{1sg} & bring & \multicolumn{2}{l}{go.home} &  & \multicolumn{3}{l}{woman} & \multicolumn{2}{l}{wake.up} & \textsc{d.dist} & \textsc{3sg} & already\\
& \multicolumn{3}{l}{snang} & \multicolumn{3}{l}{karna} & liat & \multicolumn{2}{l}{ada} & \multicolumn{4}{l}{makangang}\\
& \multicolumn{3}{l}{feel.happy(.about)} & \multicolumn{3}{l}{because} & see & \multicolumn{2}{l}{exist} & \multicolumn{4}{l}{food}\\
\lspbottomrule
\end{tabular}
\ea
\glt
‘I brought home (the game that I had shot) … (when my) wife \bluebold{got up at that time}, already she was glad because (she) saw there was food’ (Lit. ‘\bluebold{that waking up}’) \textstyleExampleSource{[080919-004-NP.0030/0032]}
\end{styleFreeTranslEngxvpt}

\paragraph[Psychological uses of demonstratives]{Psychological uses of demonstratives}
\label{bkm:Ref354834656}\label{bkm:Ref320352961}
In their psychological uses, the demonstratives signal the speakers’ psychological involvement with the situation or event talked about {\citep[347]{Lakoff1974}}. Three major domains of psychological use are attested: emotional involvement, vividness, and contrast.
\end{styleBodyxvafter}

\subparagraph[Demonstratives signaling emotional involvement]{Demonstratives signaling emotional involvement}
\label{bkm:Ref316543906}
Speakers employ the demonstratives to signal their emotional involvement, close association, and/or attitudes concerning the subject matter. Proximal \textitbf{ini} ‘\textsc{d.prox}’ indicates emotional proximity or positive attitudes, while \textitbf{itu} ‘\textsc{d.dist}’ signals emotional distance or negative attitudes, as illustrated in three sets of examples.



In the first set of examples in (0) and (0), the demonstratives modify the personal pronoun \textitbf{ko} ‘\textsc{2sg}’. In (0), a mother scolds her daughter for having ripped off the blossoms of the garden’s flowers. In shouting at her youngest, she uses \textitbf{ini} ‘\textsc{d.prox}’, thereby signaling her nevertheless close emotional involvement with her daughter. By contrast, in (0) a teacher is exasperated with one of his students who does not know the English word ‘please’. In voicing his frustration, the speaker uses \textitbf{itu} ‘\textsc{d.dist}’ and thereby signals his momentary emotional distance from the referent.
\end{styleBodyvvafter}

\begin{styleExampleTitle}
Demonstratives signaling emotional involvement: Examples set \#1
\end{styleExampleTitle}

\begin{tabular}{llllllllllll}
\lsptoprule
\label{bkm:Ref338955268}
\gll {ko} {liat} {\multicolumn{2}{l}{Luisa}} {\multicolumn{2}{l}{pu}} {\multicolumn{2}{l}{bagus,}} {suda} {kembang} {banyak,}\\ %
& \textsc{2sg} & see & \multicolumn{2}{l}{Luisa} & \multicolumn{2}{l}{\textsc{poss}} & \multicolumn{2}{l}{be.good} & already & flowering & many\\
& \bluebold{ko} & \multicolumn{2}{l}{\bluebold{ini},} & \multicolumn{2}{l}{bunga} & \multicolumn{2}{l}{tida} & \multicolumn{4}{l}{slamat}\\
& \textsc{2sg} & \multicolumn{2}{l}{\textsc{d.prox}} & \multicolumn{2}{l}{flower} & \multicolumn{2}{l}{\textsc{neg}} & \multicolumn{4}{l}{safe}\\
\lspbottomrule
\end{tabular}
\ea
\glt 
[After the speaker’s daughter had ripped off blossoms:] ‘you see Luisa’s (flowers) are good, (they are) already flowering a lot, \bluebold{you (}\blueboldSmallCaps{emph}\bluebold{)}, the flowers (you picked) can’t be saved’ \textstyleExampleSource{[081006-021-CvHt.0002]}
\z

\begin{tabular}{lllllllllllllllll}
\lsptoprule
\label{bkm:Ref338955269}
\gll {Dodo} {\multicolumn{2}{l}{kipas}} {\multicolumn{2}{l}{de}} {\multicolumn{2}{l}{suda}} {mo,} {\multicolumn{2}{l}{\bluebold{ko}}} {\multicolumn{3}{l}{\bluebold{tu},}} {\multicolumn{2}{l}{ora}} {orang}\\ %
& Dodo & \multicolumn{2}{l}{beat} & \multicolumn{2}{l}{\textsc{3sg}} & \multicolumn{2}{l}{already} & want & \multicolumn{2}{l}{\textsc{2sg}} & \multicolumn{3}{l}{\textsc{d.dist}} & \multicolumn{2}{l}{\textsc{tru}{}-person} & person\\
& \multicolumn{2}{l}{bilang} & \multicolumn{2}{l}{please,} & \multicolumn{2}{l}{kata} & \multicolumn{3}{l}{pis} & \multicolumn{2}{l}{saja} & tida & \multicolumn{2}{l}{taw,} & \multicolumn{2}{l}{goblok}\\
& \multicolumn{2}{l}{say} & \multicolumn{2}{l}{please[E]} & \multicolumn{2}{l}{word} & \multicolumn{3}{l}{please[E]} & \multicolumn{2}{l}{just} & \textsc{neg} & \multicolumn{2}{l}{know} & \multicolumn{2}{l}{be.stupid}\\
\lspbottomrule
\end{tabular}
\ea
\glt 
‘Dodo reprimanded her immediately, ‘\bluebold{you there}, people[\textsc{tru}], people say ‘please’, don’t (you) know the word ‘please’?!, (you’re) stupid!’’ \textstyleExampleSource{[081115-001a-Cv.0140]}
\z


In the second set of examples in (0) and (0), each demonstrative modifies the common noun \textitbf{ana} ‘child’. The utterance in (0) is part of a story about a motorbike accident that the speaker had in a remote area of the regency. The speaker relates how her nephew came and picked her up and took her all the way to the next hospital in the regency capital. In choosing \textitbf{ini} ‘\textsc{d.prox}’, the speaker signals her emotional closeness to the referent, who was not present when the speaker related her story; in fact, at that time the nephew was living about 300 km away. The utterance (0) occurred during a conversation about the speaker’s youngest brother. Her interlocutors relate several complaints about the referent, who was present during the conversation. Finally, the speaker joins her interlocutors and comments that \textitbf{ana kecil itu} ‘that small child’ constantly changes his opinion. By employing \textitbf{itu} ‘\textsc{d.dist}’, the speaker, who often criticizes her brother publicly, signals that she wishes to dissociate herself from her brother.


\begin{styleExampleTitle}
Demonstratives signaling emotional involvement: Examples set \#2
\end{styleExampleTitle}

\begin{tabular}{lllllllllll}
\lsptoprule
\label{bkm:Ref338955270}
\gll {baru} {\bluebold{sa}} {\bluebold{punya}} {\bluebold{ana}} {\bluebold{ini}} {mantri} {de} {pi} {ambil} {saya}\\ %
& and.then & \textsc{1sg} & \textsc{poss} & child & \textsc{d.prox} & male.nurse & \textsc{3sg} & go & fetch & \textsc{1sg}\\
\lspbottomrule
\end{tabular}
\ea
\glt 
‘and then \bluebold{this child of mine}, the male nurse came to get me’ \textstyleExampleSource{[081015-005-NP.0044]}
\z

\begin{tabular}{llllll}
\lsptoprule
\label{bkm:Ref338955271}
\gll {putar} {putar} {\bluebold{ana}} {\bluebold{kecil}} {\bluebold{itu}}\\ %
& turn.around & turn.around & child & be.small & \textsc{d.dist}\\
\lspbottomrule
\end{tabular}
\ea
\glt 
‘(he’s) constantly changing (his opinion), \bluebold{that small child}’ \textstyleExampleSource{[081011-003-Cv.0016]}
\z


In the third set of examples in (0) and (0), each demonstrative modifies an adnominal possessive construction with the possessum \textitbf{swara} ‘voice’. The utterance in (0) is part of a conversation about the young people living in the house, none of whom is present at this conversation. The speaker relates that the teenagers enjoy singing. Using direct speech, the speaker conveys her positive attitudes about the teenagers’ singing: they should sing more in public because \textitbf{kamu pu swara ini} ‘these voices of yours’ are good. The utterance in (0) occurred during a conversation outside at night. When one of the teenagers laughs out loudly, the others reprimand her. Employing \textitbf{itu} ‘\textsc{d.dist}’ in \textitbf{ko pu swara tu} ‘that voice of yours’, the speaker conveys her negative attitudes about this behavior.


\begin{styleExampleTitle}
Demonstratives signaling emotional involvement: Examples set \#3
\end{styleExampleTitle}

\begin{tabular}{llllllllll}
\lsptoprule
\label{bkm:Ref338955272}
\gll {…} {dang} {menyanyi} {…} {\bluebold{kamu}} {\bluebold{pu}} {\bluebold{swara}} {\bluebold{ini}} {bagus}\\ %
&  & and & sing &  & \textsc{2pl} & \textsc{poss} & voice & \textsc{d.prox} & be.good\\
\lspbottomrule
\end{tabular}
\ea
\glt 
‘[come in front] and sing … \bluebold{these voices of yours} are good’ \textstyleExampleSource{[081014-015-Cv.0026/0028]}
\z

\begin{tabular}{llllllll}
\lsptoprule
\label{bkm:Ref338955275}
\gll {\bluebold{ko}} {\bluebold{pu}} {\bluebold{swara}} {\bluebold{tu}} {bahaya,} {ko} {stop}\\ %
& \textsc{2sg} & \textsc{poss} & voice & \textsc{d.dist} & be.dangerous & \textsc{2sg} & stop\\
\lspbottomrule
\end{tabular}
\ea
\glt
‘\bluebold{that voice of yours} is dangerous, stop (it)!’ \textstyleExampleSource{[081025-009b-Cv.0016]}
\end{styleFreeTranslEngxvpt}

\subparagraph[Demonstratives signaling vividness]{Demonstratives signaling vividness}

The emotional involvement does not need to be as substantial as described in ‘Demonstratives signaling emotional involvement’ below. The demonstratives are also used in more general terms to indicate that the subject matter is vivid “to the mind of the speaker”, adopting {Anderson and Keenan’s (1985: 278)} {terminology}. To signal that an event or situation is of special interest to them, the speakers use the demonstratives adnominally or adverbially, or employ demonstrative stacking. This section discusses both of these strategies.



The first strategy to signal vividness is to employ the demonstratives adnominally or adverbially to modify and thereby intensify nominal and pronominal constituents as in (0) and (0), or verbs as in (0) and (0).
\end{styleBodyvafter}


In the first set of examples in (0) and (0), the short demonstrative forms modify nominal and pronominal constituents.
\end{styleBodyvafter}


The utterance in (0) occurred after the speaker had been provoked verbally by an older relative. In her reaction, the speaker modifies the constituents \textitbf{bapa-tua} ‘uncle’ and \textitbf{emosi} ‘feel angry (about)’ with \textitbf{ini} ‘\textsc{d.prox}’, thereby emphasizing them. In choosing the proximal rather than the distal demonstrative to modify the constituent \textitbf{bapa-tua} ‘uncle’, the speaker also signals that the referent is still nearby.
\end{styleBodyvafter}


The utterance in (0) is part of a conversation about the work stamina of a wife from the Pante-Barat area. When she and her husband lived in a different area, the women from that area were surprised how hard the Pante-Barat woman worked. The utterance in (0) relates the husband’s response to these women. Having referred to his wife twice with the personal pronoun \textitbf{de} ‘\textsc{3sg}’, the speaker refers to her again. This time he modifies the personal pronoun with short \textitbf{itu} ‘\textsc{d.dist}’, thereby emphasizing it. This example again illustrates the at times overlapping functions of the demonstratives. In addition to signaling vividness, the distal demonstrative also signals that the referent was not present in the speech situation.
\end{styleBodyvvafter}

\begin{styleExampleTitle}
Adnominal uses to signal vividness
\end{styleExampleTitle}

\begin{tabular}{llllllll}
\lsptoprule
\label{bkm:Ref338955276}
\gll {sa} {bilang,} {\multicolumn{2}{l}{adu,}} {\bluebold{bapa-tua}} {\bluebold{ni}} {mancing}\\ %
& \textsc{1sg} & say & \multicolumn{2}{l}{oh.no!} & uncle & \textsc{d.prox} & fish.with.rod\\
& \multicolumn{3}{l}{\bluebold{emosi}} & \multicolumn{4}{l}{\bluebold{ni}}\\
& \multicolumn{3}{l}{feel.angry(.about)} & \multicolumn{4}{l}{\textsc{d.prox}}\\
\lspbottomrule
\end{tabular}
\ea
\glt 
[After having been provoked:] ‘I said, ‘oh no, \bluebold{uncle (}\blueboldSmallCaps{emph}\bluebold{)} is provoking \bluebold{(our) emotions (}\blueboldSmallCaps{emph}\bluebold{)}’’ \textstyleExampleSource{[081025-008-Cv.0124]}
\z

\begin{tabular}{llllllllllllll}
\lsptoprule
\label{bkm:Ref338955277}
\gll {…} {\multicolumn{2}{l}{de}} {\multicolumn{2}{l}{bilang,}} {…} {de} {\multicolumn{2}{l}{suda}} {biasa} {de} {bisa} {kerja,}\\ %
&  & \multicolumn{2}{l}{\textsc{3sg}} & \multicolumn{2}{l}{say} &  & \textsc{3sg} & \multicolumn{2}{l}{already} & be.usual & \textsc{3sg} & be.able & work\\
& \multicolumn{2}{l}{\bluebold{de}} & \multicolumn{2}{l}{\bluebold{tu}} & \multicolumn{2}{l}{kerja} & \multicolumn{2}{l}{kaya} & \multicolumn{5}{l}{laki{\Tilde}laki}\\
& \multicolumn{2}{l}{\textsc{3sg}} & \multicolumn{2}{l}{\textsc{d.dist}} & \multicolumn{2}{l}{work} & \multicolumn{2}{l}{like} & \multicolumn{5}{l}{\textsc{rdp}{\Tilde}husband}\\
\lspbottomrule
\end{tabular}
\ea
\glt 
‘[and then my husband told (them),] he said, ‘… she’s already used (to working like this), she can work (hard), \bluebold{she (}\blueboldSmallCaps{emph}\bluebold{)} works like a man’’ \textstyleExampleSource{[081014-007-CvEx.0049-0050]}
\z


In the second set of examples in (0) and (0), the demonstratives are used adverbially to signal vividness. In (0), \textitbf{ini} ‘\textsc{d.prox}’ modifies the verb \textitbf{hidup} ‘live’, resulting in the emphatic reading \textitbf{hidup ini} ‘to be very much alive’. Along similar lines, in (0), \textitbf{itu} ‘\textsc{d.dist}’ modifies the verb \textitbf{lompat} ‘jump’, giving the emphatic reading \textitbf{lompat itu} ‘really jumped’. Again, these examples illustrate the overlapping functions of the demonstratives: while indicating vividness, they also have temporal uses. They signal temporal proximity indicating present tense in (0), and temporal distance indicating past tense in (0).


\begin{styleExampleTitle}
Adverbial uses to signal vividness
\end{styleExampleTitle}

\begin{tabular}{llllllllllll}
\lsptoprule
\label{bkm:Ref338955278}\label{bkm:Ref320535088}
\gll {\multicolumn{2}{l}{wa,}} {sa} {\multicolumn{2}{l}{masi}} {\bluebold{hidup}} {\bluebold{ni},} {kam} {suda} {hinggap} {di}\\ %
& \multicolumn{2}{l}{wow!} & \textsc{1sg} & \multicolumn{2}{l}{still} & live & \textsc{d.prox} & \textsc{2pl} & already & perch & at\\
& sa & \multicolumn{3}{l}{punya} & \multicolumn{7}{l}{badang}\\
& \textsc{1sg} & \multicolumn{3}{l}{\textsc{poss}} & \multicolumn{7}{l}{body}\\
\lspbottomrule
\end{tabular}
\ea
\glt 
[After having been pestered by flies:] ‘wow!, I’m still \bluebold{very much alive}, you (blue flies) had already perched upon my body’ \textstyleExampleSource{[080919-005-Cv.0015]}
\z

\begin{tabular}{lllllll}
\lsptoprule
\label{bkm:Ref338955279}\label{bkm:Ref320535089}
\gll {sunggu} {sa} {\bluebold{lompat}} {\bluebold{itu}} {dengang} {tenaga}\\ %
& be.true & \textsc{1sg} & jump & \textsc{d.dist} & with & energy\\
\lspbottomrule
\end{tabular}
\ea
\glt 
‘truly, I \bluebold{really jumped} with energy’ \textstyleExampleSource{[081025-006-Cv.0218]}
\z


The second, although less common, strategy to signal vividness is the stacking of demonstratives. In the corpus the first demonstrative is always a long one, while the second is always the corresponding short one. In these constructions, the first demonstrative may be used adnominally as in (0) and (0), or pronominally as in (0) and (0). In each case, the result of the stacking is an emphatic reading of the entire noun phrase.



In (0) and (0) the second demonstrative modifies a nested noun phrase with an adnominal demonstrative such that ‘[[\textsc{n} \textsc{dem}] \textsc{dem}]’. The result of the stacking is an emphatic reading in the sense of ‘this/that very \textsc{n}’: \textitbf{orang ini ni} ‘this very person’ in (0) and \textitbf{ruma itu tu} ‘that very house’ in (0).
\end{styleBodyvvafter}

\begin{styleExampleTitle}
Adnominal uses of stacked demonstrative to signal vividness
\end{styleExampleTitle}

\begin{tabular}{llllllll}
\lsptoprule
\label{bkm:Ref439954271}\label{bkm:Ref338955281}
\gll {[[\bluebold{orang}} {\bluebold{ini}]} {\bluebold{ni}]} {percaya} {sama} {Tuhang} {Yesus}\\ %
& person & \textsc{d.prox} & \textsc{d.prox} & trust & to & God & Jesus\\
\lspbottomrule
\end{tabular}
\ea
\glt 
‘\bluebold{this very person }believes in God Jesus’ \textstyleExampleSource{[081006-022-CvEx.0177]}
\z

\begin{tabular}{lllllllll}
\lsptoprule
\label{bkm:Ref338955282}
\gll {waktu} {kitorang} {masuk} {di} {[[\bluebold{ruma}} {\bluebold{itu}]} {\bluebold{tu}]} {…}\\ %
& when & \textsc{1pl} & enter & at & house & \textsc{d.dist} & \textsc{d.dist} & \\
\lspbottomrule
\end{tabular}
\ea
\glt 
‘when we moved into \bluebold{that very house}, …’ \textstyleExampleSource{[081006-022-CvEx.0167]}
\z


In (0) and (0) the second demonstrative modifies a pronominally used first one. The result is an emphatic reading in the sense of ‘this/that very (one)’: \textitbf{ini ni} ‘these very (ones)’ in (0), and \textitbf{itu tu} ‘those very (ones)’ in (0).


\begin{styleExampleTitle}
Pronominal uses of stacked demonstrative to signal vividness
\end{styleExampleTitle}

\begin{tabular}{lllllllll}
\lsptoprule
\label{bkm:Ref338955283}
\gll {ada} {segala} {macang} {tulang,} {dia} {buang} {[\bluebold{ini}} {\bluebold{ni}]}\\ %
& exist & all & variety & bone & \textsc{3sg} & throw(.away) & \textsc{d.prox} & \textsc{d.prox}\\
\lspbottomrule
\end{tabular}
\ea
\glt 
‘there were all kinds of bones, he threw away \bluebold{these very (ones)}’ \textstyleExampleSource{[080922-010a-CvNF.0101]}
\z

\begin{tabular}{llllllllllllll}
\lsptoprule
\label{bkm:Ref338955286}
\gll {ko} {\multicolumn{2}{l}{taw}} {\multicolumn{2}{l}{kata}} {\multicolumn{2}{l}{pis}} {ka} {\multicolumn{2}{l}{tida,}} {[\bluebold{itu}} {\bluebold{tu}]} {kata{\Tilde}kata}\\ %
& \textsc{2sg} & \multicolumn{2}{l}{know} & \multicolumn{2}{l}{word} & \multicolumn{2}{l}{please[E]} & or & \multicolumn{2}{l}{\textsc{neg}} & \textsc{d.dist} & \textsc{d.dist} & \textsc{rdp}{\Tilde}word\\
& \multicolumn{2}{l}{dasar} & \multicolumn{2}{l}{yang} & \multicolumn{2}{l}{harusnya} & \multicolumn{3}{l}{kamu} & \multicolumn{4}{l}{taw}\\
& \multicolumn{2}{l}{base} & \multicolumn{2}{l}{\textsc{rel}} & \multicolumn{2}{l}{appropriately} & \multicolumn{3}{l}{\textsc{2pl}} & \multicolumn{4}{l}{know}\\
\lspbottomrule
\end{tabular}
\ea
\glt
[Addressing a school student:] ‘do you know the (English) word ‘please’ or not?, \bluebold{that very} (word belongs to) the basic words that you should know’ \textstyleExampleSource{[081115-001a-Cv.0145]}
\end{styleFreeTranslEngxvpt}

\subparagraph[Demonstratives signaling contrast between two entities]{Demonstratives signaling contrast between two entities}

In their contrastive uses, the demonstratives signal contrast between a discourse referent and another entity, thereby conveying the speakers’ attitudes about the subject matter. This contrastive use is illustrated with three sets of examples.



In the first set of examples in (0) and (0), the demonstratives modify the personal pronoun \textitbf{saya} ‘\textsc{1sg}’, each time indicating an explicit contrast.
\end{styleBodyvafter}


In (0), the speaker compares the ill-behaved young people living in the house to himself. While they have the privilege of staying with relatives in the regional city to complete their secondary schooling, he had to stay with strangers when he was young. This contrast is indicated with \textitbf{ini} ‘\textsc{d.prox}’.
\end{styleBodyvvafter}

\begin{styleExampleTitle}
Demonstrative signaling contrast: \textitbf{saya ini} ‘\textsc{1sg} \textsc{d.prox}’
\end{styleExampleTitle}

\begin{tabular}{llllllllll}
\lsptoprule
\label{bkm:Ref338955291}
\gll {kamu} {\multicolumn{2}{l}{ana{\Tilde}ana}} {skarang} {ini} {susa} {…} {\bluebold{saya}} {\bluebold{ini}}\\ %
& \textsc{2pl} & \multicolumn{2}{l}{\textsc{rdp}{\Tilde}child} & now & \textsc{d.prox} & be.difficult &  & \textsc{1sg} & \textsc{d.prox}\\
& \multicolumn{2}{l}{tinggal} & dengang & \multicolumn{6}{l}{orang}\\
& \multicolumn{2}{l}{stay} & with & \multicolumn{6}{l}{person}\\
\lspbottomrule
\end{tabular}
\ea
\glt 
‘you, the young people, nowadays are difficult … \bluebold{I, by contrast}, stayed with (other) people’ (Lit. ‘\bluebold{this I}’) \textstyleExampleSource{[081115-001b-Cv.0038/0040]}
\z


The exchange in (0) occurred during a phone conversation when a daughter asked her father to buy her a cell-phone. In (0) her father suggests that a cell-phone would distract her from her studies. The daughter responds with the contrastive statement in (0) in which \textitbf{itu} ‘\textsc{d.dist}’ modifies \textitbf{saya} ‘\textsc{1sg}’, resulting in the contrastive reading \textitbf{sa tu} ‘I, nevertheless’. The exact semantic distinctions between \textitbf{ini} ‘\textsc{d.prox}’ and \textitbf{itu} ‘\textsc{d.dist}’ need further investigation, though. The use of \textitbf{itu} ‘\textsc{d.dist}’ with \textitbf{saya} ‘\textsc{1sg}’ is especially surprising given that a first person singular pronoun is inherently proximal. A temporal non-contemporaneous interpretation is not likely since the speaker talks about her behavior in general.


\begin{styleExampleTitle}
Demonstrative signaling contrast: \textitbf{saya itu} ‘\textsc{1sg} \textsc{d.dist}’
\end{styleExampleTitle}

\begin{tabular}{llllllllllllll}
\lsptoprule
\label{bkm:Ref338955293}\label{bkm:Ref320364631}
\gll {\label{bkm:Ref363302183}} {Father:} {kalo} {bli} {\multicolumn{2}{l}{HP}} {di} {\multicolumn{2}{l}{situ}} {\multicolumn{2}{l}{nanti}} {\multicolumn{2}{l}{su}}\\ %
&  &  & if & buy & \multicolumn{2}{l}{cell.phone} & at & \multicolumn{2}{l}{\textsc{l.med}} & \multicolumn{2}{l}{very.soon} & \multicolumn{2}{l}{already}\\
&  &  & tra & \multicolumn{2}{l}{bisa} & \multicolumn{3}{l}{skola,} & \multicolumn{2}{l}{maing} & \multicolumn{2}{l}{HP} & saja\\
&  &  & \textsc{neg} & \multicolumn{2}{l}{be.able} & \multicolumn{3}{l}{go.to.school} & \multicolumn{2}{l}{play} & \multicolumn{2}{l}{cell.phone} & just\\
\lspbottomrule
\end{tabular}
\begin{styleFreeTranslIndentiicmEng}
Father: ‘if (you) buy a cell-phone there then (you) won’t be able to do (any) schooling, (you’ll) just play (with your) cell-phone’
\end{styleFreeTranslIndentiicmEng}

\begin{tabular}{lllllll} & \label{bkm:Ref363302186} & Daughter: & \bluebold{sa} & \bluebold{tu} & rajing & skola\\
\lsptoprule
&  &  & \textsc{1sg} & \textsc{d.dist} & be.diligent & go.to.school\\
\lspbottomrule
\end{tabular}
\begin{styleFreeTranslIndentiicmEng}
Daughter: ‘\bluebold{I, nonetheless}, go to school diligently’ (Lit. ‘\bluebold{that I}’) \textstyleExampleSource{[080922-001a-CvPh.0448/0455]}
\end{styleFreeTranslIndentiicmEng}


In the second set of examples in (0) and (0), the demonstratives modify the personal pronoun \textitbf{ko} ‘\textsc{2sg}’: the contrast is implicit in (0), while it is explicit in (0).



The example in (0) is part of joke about a boy who chooses to attend a choir rather than a karate club together with his friends. The father is upset about his son’s choice. Finally, he vents his anger with a contrastive statement in which \textitbf{ini} ‘\textsc{d.prox}’ modifies \textitbf{ko} ‘\textsc{2sg}’. Thereby, the father contrasts his son implicitly with his friends: \textitbf{ko ni} ‘and what about you’.
\end{styleBodyvvafter}

\begin{styleExampleTitle}
Demonstrative signaling contrast: \textitbf{ko ini} ‘\textsc{2sg} \textsc{d.prox}’
\end{styleExampleTitle}

\begin{tabular}{llllllllllllllllllllm{-9.4015896E-4in}l}
\lsptoprule
\label{bkm:Ref338955294}
\gll {…} {\multicolumn{3}{l}{sampe}} {\multicolumn{4}{l}{dep}} {\multicolumn{3}{l}{bapa}} {\multicolumn{3}{l}{su}} {\multicolumn{4}{l}{mara,}} {\bluebold{ko}} {\multicolumn{2}{l}{\bluebold{ni}}}\\ %
&  & \multicolumn{3}{l}{until} & \multicolumn{4}{l}{\textsc{3sg}:\textsc{poss}} & \multicolumn{3}{l}{father} & \multicolumn{3}{l}{already} & \multicolumn{4}{l}{feel.angry(.about)} & \textsc{2sg} & \multicolumn{2}{l}{\textsc{d.prox}}\\
& \multicolumn{3}{l}{setiap} & hari & \multicolumn{2}{l}{ko} & \multicolumn{4}{l}{ikut} & \multicolumn{3}{l}{latiang} & \multicolumn{4}{l}{paduang-swara} & \multicolumn{3}{l}{trus,} & kalo\\
& \multicolumn{3}{l}{every} & day & \multicolumn{2}{l}{\textsc{2sg}} & \multicolumn{4}{l}{follow} & \multicolumn{3}{l}{practice} & \multicolumn{4}{l}{choir} & \multicolumn{3}{l}{be.continuous} & if\\
& \multicolumn{2}{l}{dong} & \multicolumn{3}{l}{pukul} & \multicolumn{2}{l}{ko} & \multicolumn{2}{l}{ko} & \multicolumn{3}{l}{bisa} & \multicolumn{3}{l}{tangkis} & ka & \multicolumn{5}{l}{tida}\\
& \multicolumn{2}{l}{\textsc{3pl}} & \multicolumn{3}{l}{hit} & \multicolumn{2}{l}{\textsc{2sg}} & \multicolumn{2}{l}{\textsc{2sg}} & \multicolumn{3}{l}{be.able} & \multicolumn{3}{l}{ward.off} & or & \multicolumn{5}{l}{\textsc{neg}}\\
\lspbottomrule
\end{tabular}
\ea
\glt 
‘[his father sees him (practicing in a choir) while his other friends practice self-defense] until his father gets angry (with his son), ‘\bluebold{and what about you}, every day you attend the choir practice, (but) if someone hits you, can you defend (yourself) or not?’’ (Lit. ‘\bluebold{this you}’) \textstyleExampleSource{[081109-006-JR.0001-0003]}
\z


In (0), an aunt gives advice to her niece who had been insulted by her younger cousin. Agreeing that the younger cousin has lighter skin and longer hair than the referent, the speaker continues her advice with a contrastive statement in which \textitbf{itu} ‘\textsc{d.dist}’ modifies \textitbf{ko} ‘\textsc{2sg}’: \textitbf{ko itu} ‘you, however’.


\begin{styleExampleTitle}
Demonstrative signaling contrast: \textitbf{ko itu} ‘\textsc{2sg} \textsc{d.dist}’
\end{styleExampleTitle}

\begin{tabular}{llllllllllll}
\lsptoprule
\label{bkm:Ref338955297}
\gll {ade} {\multicolumn{2}{l}{tu}} {\multicolumn{2}{l}{biar}} {\multicolumn{2}{l}{puti,}} {rambut} {mayang} {tinggal} {rambut}\\ %
& ySb & \multicolumn{2}{l}{\textsc{d.dist}} & \multicolumn{2}{l}{let} & \multicolumn{2}{l}{be.white} & hair & palm.blossom & stay & hair\\
& \multicolumn{2}{l}{panjang,} & \multicolumn{2}{l}{\bluebold{ko}} & \multicolumn{2}{l}{\bluebold{itu}} & \multicolumn{5}{l}{jalang}\\
& \multicolumn{2}{l}{be.long} & \multicolumn{2}{l}{\textsc{2sg}} & \multicolumn{2}{l}{\textsc{d.dist}} & \multicolumn{5}{l}{walk}\\
\lspbottomrule
\end{tabular}
\ea
\glt 
‘let that younger sister have light skin, (let her have) hair that’s long down to her bottom, \bluebold{you, however}, go (your own way)’ (Lit. ‘\bluebold{that you}’) \textstyleExampleSource{[081115-001a-Cv.0244]}
\z


In the third set in (0), the demonstratives modify temporal adverbs, thereby signaling temporal contrasts. In (0), a wife and her husband recount how a young man damaged his leg during a motorbike accident. In (0) the wife relates that \textitbf{skarang} ‘now’ the referent walks crookedly. Her husband continues the narrative in (0) with a contrastive statement in which \textitbf{itu} ‘\textsc{d.dist}’ modifies the temporal adverb \textitbf{dulu} ‘first, in the past’, thereby signaling a temporal contrast: \textitbf{dulu itu} ‘in the past, however’. Subsequently, the wife further elaborates on the referent’s condition. She concludes the exchange with yet another contrastive statement in (0) in which \textitbf{ini} ‘\textsc{d.prox}’ modifies the temporal adverb \textitbf{skarang} ‘now’, again signaling a temporal contrast: \textitbf{skarang ini} ‘(it’s) just now’.


\begin{styleExampleTitle}
Demonstrative signaling contrast: Modifying temporal adverbs
\end{styleExampleTitle}

\begin{tabular}{llllllll}
\lsptoprule
\label{bkm:Ref338955298}\label{bkm:Ref320364632}
\gll {\label{bkm:Ref363302268}} {Wife:} {skarang} {ada} {jalang} {bengkok} {sedikit}\\ %
&  &  & now & exist & walk & be.crooked & few\\
\lspbottomrule
\end{tabular}
\begin{styleFreeTranslIndentiicmEng}
Wife: ‘now he’s walking a little crookedly (because of his motorbike accident)’
\end{styleFreeTranslIndentiicmEng}

\begin{tabular}{llllllll} & \label{bkm:Ref363302270} & Husband: & \bluebold{dulu} & \bluebold{itu} & de & jalang & lurus\\
\lsptoprule
&  &  & first & \textsc{d.dist} & \textsc{3sg} & walk & be.straight\\
\lspbottomrule
\end{tabular}
\begin{styleFreeTranslIndentiicmEng}
Husband: ‘\bluebold{in the past, however}, he walked straight’
\end{styleFreeTranslIndentiicmEng}

\begin{tabular}{lllllllllllll} & \label{bkm:Ref320373858} & Wife: & … & \multicolumn{2}{l}{ini} & \multicolumn{2}{l}{bengkok} & \multicolumn{2}{l}{ini,} & \multicolumn{2}{l}{kaki} & ini,\\
\lsptoprule
&  &  &  & \multicolumn{2}{l}{\textsc{d.prox}} & \multicolumn{2}{l}{be.crooked} & \multicolumn{2}{l}{\textsc{d.prox}} & \multicolumn{2}{l}{foot} & \textsc{d.prox}\\
&  &  & \multicolumn{2}{l}{\bluebold{skarang}} & \multicolumn{2}{l}{\bluebold{ini}} & \multicolumn{2}{l}{baru} & \multicolumn{2}{l}{ada} & \multicolumn{2}{l}{baik{\Tilde}baik}\\
&  &  & \multicolumn{2}{l}{now} & \multicolumn{2}{l}{\textsc{d.prox}} & \multicolumn{2}{l}{recently} & \multicolumn{2}{l}{exist} & \multicolumn{2}{l}{\textsc{rdp}{\Tilde}be.good}\\
\lspbottomrule
\end{tabular}
\begin{styleFreeTranslIndentiicmEngxxpt}
Wife: ‘this (foot) was crooked here, this foot, (\bluebold{it’s) just now} that (it got) well’ \textstyleExampleSource{[081006-020-Cv.0006-0007/0013]}
\end{styleFreeTranslIndentiicmEngxxpt}

\paragraph[Identificational uses of demonstratives]{Identificational uses of demonstratives}
\label{bkm:Ref320275768}
The demonstratives have identificational uses when they appear in the subject slot of a nominal predicate clause (§12.2). In this context, the demonstratives aid in the identification of a definite or identifiable referent encoded by the predicate. For instance, \textitbf{ini} ‘\textsc{d.prox}’ takes the subject slot in (0) and \textitbf{itu} ‘\textsc{d.dist}’ in (0). In this domain of use only the long demonstratives are attested.
\end{styleBodyxafter}

\begin{tabular}{llllllllll}
\lsptoprule
\label{bkm:Ref338955301}
\gll {\bluebold{ini}} {daging} {yang} {saya} {bawa} {antar} {buat} {sodara} {dorang}\\ %
& \textsc{d.prox} & meat & \textsc{rel} & \textsc{1sg} & bring & deliver & for & sibling & \textsc{3pl}\\
\lspbottomrule
\end{tabular}
\ea
\glt 
‘\bluebold{this} is the (wild pig) meat that I brought (and) delivered for (my) relatives’ \textstyleExampleSource{[080919-003-NP.0021]}
\z

\begin{tabular}{llll}
\lsptoprule
\label{bkm:Ref338955302}
\gll {\bluebold{itu}} {kali} {Biri}\\ %
& \textsc{d.dist} & river & Biri\\
\lspbottomrule
\end{tabular}
\ea
\glt
‘\bluebold{that} is the Biri river’ \textstyleExampleSource{[081025-008-Cv.0006]}
\end{styleFreeTranslEngxvpt}

\paragraph[Textual uses of demonstratives]{Textual uses of demonstratives}
\label{bkm:Ref320548390}
In their textual uses, the demonstratives provide discourse orientation. Across languages, two major discourse uses of demonstratives can be distinguished, according to {Diessel (1999: 95–105)}: “anaphoric” and “discourse deictic” uses. In their anaphoric uses, the demonstratives “are coreferential with a prior NP” and thereby “keep track of discourse participants” {(1999: 93)}. In their discourse deictic uses, the demonstratives are not coreferent with the referent of a previously established noun phrase. Instead, they are coreferential with a preceding or following proposition. That is, they “establish an overt link between two propositions: the one in which they are embedded and the one to which they refer” {(1999: 101)}. (See also {Himmelmann 1996.})



Both discourse functions also apply to the Papuan Malay demonstratives.
\end{styleBodyvxvafter}

\subparagraph[Anaphoric uses]{Anaphoric uses}

The anaphoric uses of the Papuan Malay demonstratives are demonstrated in (0) to (0). Being coreferential with a preceding NP, they keep track of different discourse participants. In this use the demonstratives may be employed adnominally as in (0) and (0), pronominally as in (0) and (0), or adverbially as in (0) and (0). The exact semantic distinctions between \textitbf{ini} ‘\textsc{d.prox}’ and \textitbf{itu} ‘\textsc{d.dist}’ as participant tracking devices, however, are yet to be investigated in more detail.



The examples in (0) and (0) demonstrate the adnominal anaphoric uses of the demonstratives. The utterance in (0) is part of a joke about a school student who does not know to draw. The teacher orders the students to \textitbf{gambar monyet} ‘draw a monkey’. When the \textitbf{monyet} ‘monkey’ is mentioned the next time, it is marked with \textitbf{ini} ‘\textsc{d.prox}’ thereby indicating co-reference with this specific monkey. The example in (0) is part of a narrative that describes how the speaker’s ancestor first came down to the coast where he finds a \textitbf{bua mera} ‘red fruit’. At its next mention, the noun phrase \textitbf{bua mera} ‘red fruit’ is marked with \textitbf{itu} ‘\textsc{d.dist}’ to signal co-reference with that specific fruit.
\end{styleBodyvvafter}

\begin{styleExampleTitle}
Adnominal anaphoric uses
\end{styleExampleTitle}

\begin{tabular}{lllllllllllllllllllll}
\lsptoprule
\label{bkm:Ref338956198}
\gll {\multicolumn{3}{l}{ibu}} {\multicolumn{3}{l}{mulay}} {\multicolumn{3}{l}{suru}} {\multicolumn{3}{l}{ana{\Tilde}ana}} {\multicolumn{2}{l}{murit}} {\multicolumn{2}{l}{mulay}} {\multicolumn{2}{l}{gambar}} {\bluebold{monyet}} {di}\\ %
& \multicolumn{3}{l}{woman} & \multicolumn{3}{l}{start} & \multicolumn{3}{l}{order} & \multicolumn{3}{l}{\textsc{rdp}{\Tilde}child} & \multicolumn{2}{l}{pupil} & \multicolumn{2}{l}{start} & \multicolumn{2}{l}{draw} & monkey & at\\
& \multicolumn{2}{l}{atas} & \multicolumn{3}{l}{pohong} & \multicolumn{3}{l}{pisang} & \multicolumn{2}{l}{…} & trus & \multicolumn{2}{l}{de} & \multicolumn{2}{l}{gambar} & \multicolumn{2}{l}{\bluebold{monyet}} & \multicolumn{3}{l}{\bluebold{ini}}\\
& \multicolumn{2}{l}{top} & \multicolumn{3}{l}{tree} & \multicolumn{3}{l}{banana} & \multicolumn{2}{l}{} & next & \multicolumn{2}{l}{\textsc{3sg}} & \multicolumn{2}{l}{draw} & \multicolumn{2}{l}{monkey} & \multicolumn{3}{l}{\textsc{d.prox}}\\
& di & \multicolumn{3}{l}{bawa} & \multicolumn{3}{l}{pohong} & \multicolumn{13}{l}{pisang}\\
& at & \multicolumn{3}{l}{bottom} & \multicolumn{3}{l}{tree} & \multicolumn{13}{l}{banana}\\
\lspbottomrule
\end{tabular}
\ea
\glt 
‘Ms. (Teacher) starts ordering the school kids to start drawing \bluebold{a monkey} on a banana tree … and then he draws \bluebold{this monkey} under the banana tree’ \textstyleExampleSource{[081109-002-JR.0001-0002]}
\z

\begin{tabular}{lllllllllllll}
\lsptoprule
\label{bkm:Ref338956300}
\gll {trus} {di} {\multicolumn{2}{l}{situ}} {…} {ada} {\multicolumn{2}{l}{\bluebold{bua}}} {\bluebold{mera}} {…} {de} {pegang}\\ %
& next & at & \multicolumn{2}{l}{\textsc{l.med}} &  & exist & \multicolumn{2}{l}{fruit} & be.red &  & \textsc{3sg} & hold\\
& \bluebold{bua} & \multicolumn{2}{l}{\bluebold{mera}} & \multicolumn{2}{l}{\bluebold{itu}} & dang & de & \multicolumn{5}{l}{jalang}\\
& fruit & \multicolumn{2}{l}{be.red} & \multicolumn{2}{l}{\textsc{d.dist}} & and & \textsc{3sg} & \multicolumn{5}{l}{walk}\\
\lspbottomrule
\end{tabular}
\ea
\glt 
‘and then there … was \bluebold{a red fruit} … he took \bluebold{that red fruit} and he walked (further)’ \textstyleExampleSource{[080922-010a-CvNF.0218-219]}
\z


The examples in (0) and (0) illustrate the pronominal anaphoric uses of the demonstratives. The remark in (0) is part of a description of sagu production. After having introduced the main tool, \textitbf{penokok kayu} ‘wooden pounder’, the speaker replaces it at its next mention with \textitbf{ini} ‘\textsc{d.prox}’. In (0) the speaker talks about a female weight lifter. Noting that she is talking about weights in kilogram, she employs short \textitbf{itu} ‘\textsc{d.dist}’ which is coreferent with \textitbf{de pu brat} ‘her weights’.


\begin{styleExampleTitle}
Pronominal anaphoric uses
\end{styleExampleTitle}

\begin{tabular}{llllllllll}
\lsptoprule
\label{bkm:Ref338956121}\label{bkm:Ref314215992}
\gll {ada} {\bluebold{penokok}} {\bluebold{kayu}} {…} {smua} {orang} {tokok} {dengang} {\bluebold{ini}}\\ %
& exist & pounder & wood &  & all & person & tap & with & \textsc{d.prox}\\
\lspbottomrule
\end{tabular}
\ea
\glt 
‘there is \bluebold{a wooden pounder} … all people pound (sagu) with \bluebold{this}’ \textstyleExampleSource{[081014-006-CvPr.0011/0057]}
\z

\begin{tabular}{lllllllllllllll}
\lsptoprule
\label{bkm:Ref338956122}\label{bkm:Ref320523362}
\gll {\multicolumn{2}{l}{prempuang}} {\multicolumn{2}{l}{Bandung}} {\multicolumn{2}{l}{itu}} {\multicolumn{2}{l}{\bluebold{de}}} {\multicolumn{2}{l}{\bluebold{pu}}} {\multicolumn{2}{l}{\bluebold{brat}}} {yang} {itu}\\ %
& \multicolumn{2}{l}{woman} & \multicolumn{2}{l}{Bandung} & \multicolumn{2}{l}{\textsc{d.dist}} & \multicolumn{2}{l}{\textsc{3sg}} & \multicolumn{2}{l}{\textsc{poss}} & \multicolumn{2}{l}{be.heavy} & \textsc{rel} & \textsc{d.dist}\\
& sa & \multicolumn{2}{l}{angkat,} & \multicolumn{2}{l}{\bluebold{tu}} & \multicolumn{2}{l}{kilo} & \multicolumn{2}{l}{…} & \multicolumn{2}{l}{dlapang} & pulu & \multicolumn{2}{l}{tiga}\\
& \textsc{1sg} & \multicolumn{2}{l}{pick-up} & \multicolumn{2}{l}{\textsc{d.dist}} & \multicolumn{2}{l}{kilogram} & \multicolumn{2}{l}{} & \multicolumn{2}{l}{eight} & tens & \multicolumn{2}{l}{three}\\
\lspbottomrule
\end{tabular}
\ea
\glt 
‘that woman from Bandung, \bluebold{her weights} which I lifted, \bluebold{that} (is in) kilogram … eighty three (kilogram)’ \textstyleExampleSource{[081023-003-Cv.0003]}
\z


The examples in (0) and (0) illustrate the adverbial anaphoric uses of both demonstratives. The utterance in (0) is part of a narrative about a youth retreat. During their journey to the retreat, the teenagers meet an old woman who gives them advice for the retreat. The woman mentions the verb \textitbf{jalang} ‘walk’ three times while advising the teenagers where to walk and how to behave. When she mentions \textitbf{jalang} ‘walk’ again, she marks it with \textitbf{ini} ‘\textsc{d.prox}’.


\begin{styleExampleTitle}
Adverbial anaphoric uses of \textitbf{ini} ‘\textsc{d.prox}’
\end{styleExampleTitle}

\begin{tabular}{lllllllllllllm{4.5984238E-4in}llll}
\lsptoprule
\label{bkm:Ref338956123}\label{bkm:Ref315712037}
\gll {\multicolumn{2}{l}{kamu}} {\multicolumn{2}{l}{\bluebold{jalang},}} {\multicolumn{2}{l}{\bluebold{jalang}}} {\multicolumn{3}{l}{baik{\Tilde}baik}} {saja,} {\multicolumn{2}{l}{kamu}} {\multicolumn{2}{l}{tinggal,}} {\multicolumn{2}{l}{kamu}} {\bluebold{jalang},}\\ %
& \multicolumn{2}{l}{\textsc{2pl}} & \multicolumn{2}{l}{walk} & \multicolumn{2}{l}{walk} & \multicolumn{3}{l}{\textsc{rdp}{\Tilde}be.good} & just & \multicolumn{2}{l}{\textsc{2pl}} & \multicolumn{2}{l}{stay} & \multicolumn{2}{l}{\textsc{2pl}} & walk\\
& tida & \multicolumn{2}{l}{bole} & \multicolumn{3}{l}{ini} & ini & … & \multicolumn{3}{l}{kamu} & \multicolumn{2}{l}{\bluebold{jalang}} & \multicolumn{2}{l}{\bluebold{ini}} & \multicolumn{2}{l}{untuk}\\
& \textsc{neg} & \multicolumn{2}{l}{may} & \multicolumn{3}{l}{\textsc{d.prox}} & \textsc{d.prox} &  & \multicolumn{3}{l}{\textsc{2pl}} & \multicolumn{2}{l}{walk} & \multicolumn{2}{l}{\textsc{d.prox}} & \multicolumn{2}{l}{for}\\
& apa & \multicolumn{4}{l}{pekerjaang} & \multicolumn{12}{l}{Tuhang}\\
& what & \multicolumn{4}{l}{work} & \multicolumn{12}{l}{God}\\
\lspbottomrule
\end{tabular}
\ea
\glt 
‘you \bluebold{travel}, (just) \bluebold{travel} well, (when) you stay (at Takar and when you) \bluebold{walk around} (in Takar), (you) shouldn’t (do) this (and) this, … you (do) \bluebold{this traveling} for, what-is-it, God’s work’ \textstyleExampleSource{[081025-008-Cv.0142/0144]}
\z


The exchange (0) occurred between two sisters just before the youth retreat. In (0) the younger one states that she wants to \textitbf{jalang} ‘travel’ to the youth retreat without, however, attending the services; instead she plans to stay at the guesthouse. Her older sister responds in (0) that in that case it were better if she stayed home. Being upset about this reaction, the younger one asks her older sister in (0) why she said so. In her reply in (0), the older sister mentions \textitbf{jalang} ‘walk’ again, this time modifying it with \textitbf{itu} ‘\textsc{d.dist}’.


\begin{styleExampleTitle}
Adverbial anaphoric uses of \textitbf{itu} ‘\textsc{d.dist}’
\end{styleExampleTitle}

\begin{tabular}{llllllllll}
\lsptoprule
\label{bkm:Ref338956124}\label{bkm:Ref320523367}
\gll {\label{bkm:Ref320524760}} {Younger sister:} {sa} {\bluebold{jalang}} {tra} {sembayang} {tinggal} {di} {ruma}\\ %
&  &  & \textsc{1sg} & walk & \textsc{neg} & worship & stay & at & house\\
\lspbottomrule
\end{tabular}
\begin{styleFreeTranslIndentiicmEng}
Young sister: ‘I’ll \bluebold{go} (to the youth retreat, but) I won’t worship, (I’ll) stay at the house’
\end{styleFreeTranslIndentiicmEng}

\begin{tabular}{llllllllllll} & \label{bkm:Ref320524761} & Older sister: & kalo & \multicolumn{2}{l}{mo} & \multicolumn{2}{l}{tinggal} & di & ruma & tinggal & di\\
\lsptoprule
&  &  & if & \multicolumn{2}{l}{want} & \multicolumn{2}{l}{stay} & at & house & stay & at\\
&  &  & \multicolumn{2}{l}{ruma} & \multicolumn{2}{l}{sini} & \multicolumn{5}{l}{…}\\
&  &  & \multicolumn{2}{l}{house} & \multicolumn{2}{l}{\textsc{l.prox}} & \multicolumn{5}{l}{}\\
\lspbottomrule
\end{tabular}
\begin{styleFreeTranslIndentiicmEng}
Older sister: ‘if (you) want to stay at the house, stay home …’
\end{styleFreeTranslIndentiicmEng}

\begin{tabular}{llll} & \label{bkm:Ref320524762} & Younger sister: & knapa?\\
\lsptoprule
&  &  & why\\
\lspbottomrule
\end{tabular}
\begin{styleFreeTranslIndentiicmEng}
Young sister: ‘why?’
\end{styleFreeTranslIndentiicmEng}

\begin{tabular}{lllllllll} & \label{bkm:Ref320524763} & Older sister: & orang & \bluebold{jalang} & \bluebold{itu} & mo & pergi & sembayang\\
\lsptoprule
&  &  & person & walk & \textsc{d.dist} & want & go & worship\\
\lspbottomrule
\end{tabular}
\begin{styleFreeTranslIndentiicmEng}
Older sister: ‘people (doing) \bluebold{that traveling} want to go worship’ \textstyleExampleSource{[081006-016-Cv.0012-0015]}
\end{styleFreeTranslIndentiicmEng}


Alternatively, however, one might argue that in (0) and (0) the demonstratives do not function as participant tracking devices, but rather signal emotional involvement.
\end{styleBodyxvafter}

\subparagraph[Discourse deictic uses]{Discourse deictic uses}

The discourse deictic uses of the Papuan Malay demonstratives are demonstrated in (0) to (0). In this use, they are coreferential with a preceding or following proposition. As shown in (0) to (0), though, only the pronominally used demonstratives have discourse deictic uses.



Proximal \textitbf{ini} ‘\textsc{d.prox}’ may refer to a preceding statement as in (0) or to a following statement as in (0). The example in (0) is part of a conversation about difficult children. Maintaining that children should be disciplined, the speaker makes a number of suggestions how to do so. Employing short \textitbf{ini} ‘\textsc{d.prox}’, the speaker summarizes her previous statements. Thereby she creates a link to her closing statement that her interlocutor should decide for herself what to make of these suggestions. In (0), \textitbf{ini} ‘\textsc{d.prox}’ creates a link to the following direct quote.
\end{styleBodyvvafter}

\begin{styleExampleTitle}
Discourse deictic uses of \textitbf{ini} ‘\textsc{d.prox}’
\end{styleExampleTitle}

\begin{tabular}{lllllllll}
\lsptoprule
\label{bkm:Ref338956125}
\gll {…} {\bluebold{ni}} {usul} {saja} {jadi} {kaka} {sendiri} {…}\\ %
&  & \textsc{d.prox} & proposal & just & so & oSb & be.alone & \\
\lspbottomrule
\end{tabular}
\ea
\glt 
‘\bluebold{this} is just a proposal, so you (‘older sister’) (have to decide for) yourself …’ \textstyleExampleSource{[080917-010-CvEx.0116]}
\z

\begin{tabular}{lllllllllll}
\lsptoprule
\label{bkm:Ref338956126}
\gll {pace} {de} {bilang} {\bluebold{ini},} {mace} {ko} {sendiri} {yang} {ikut} {…}\\ %
& man & \textsc{3sg} & say & \textsc{d.prox} & wife & \textsc{2sg} & be.alone & \textsc{rel} & follow & \\
\lspbottomrule
\end{tabular}
\ea
\glt 
‘(my) husband said \bluebold{this}, ‘you wife yourself (should) go (with them) …’’ (Lit. ‘(it’s) you wife yourself who …’) \textstyleExampleSource{[081025-009a-Cv.0032]}
\z


Distal \textitbf{itu} ‘\textsc{d.dist}’ is used only to create a link to a preceding statement, as in (0). This example is part of joke about an uneducated person who notes that \textitbf{di kalender dua blas} ‘in the calendar are twelve (moons)’ while \textitbf{di langit ini cuma satu} ‘in this sky is only one’. Distal \textitbf{itu} ‘\textsc{d.dist}’ summarizes these remarks, creating an overt link to the speaker’s conclusion that this state of affairs is \textitbf{tipu skali} ‘very deceptive’.


\begin{styleExampleTitle}
Discourse deictic uses of \textitbf{itu} ‘\textsc{d.dist}’
\end{styleExampleTitle}

\begin{tabular}{lllllllllm{4.5984238E-4in}lllll}
\lsptoprule
\label{bkm:Ref338956127}
\gll {\multicolumn{2}{l}{masa}} {\multicolumn{2}{l}{di}} {\multicolumn{3}{l}{kalender}} {\multicolumn{2}{l}{dua}} {\multicolumn{2}{l}{blas,}} {baru} {di} {langit}\\ %
& \multicolumn{2}{l}{be.impossible} & \multicolumn{2}{l}{at} & \multicolumn{3}{l}{calendar} & \multicolumn{2}{l}{two} & \multicolumn{2}{l}{teens} & and.then & at & sky\\
& ini & \multicolumn{2}{l}{cuma} & \multicolumn{2}{l}{satu} & …, & \multicolumn{2}{l}{\bluebold{itu}} & \multicolumn{2}{l}{tipu} & \multicolumn{4}{l}{skali}\\
& \textsc{d.prox} & \multicolumn{2}{l}{just} & \multicolumn{2}{l}{one} &  & \multicolumn{2}{l}{\textsc{d.dist}} & \multicolumn{2}{l}{cheat} & \multicolumn{4}{l}{very}\\
\lspbottomrule
\end{tabular}
\ea
\glt 
[Joke:] ‘(it’s) impossible, in a calendar are twelve (moons), but in the sky here is only one (moon) … \bluebold{that’s} very deceptive’’ \textstyleExampleSource{[081109-007-JR.0003]}
\z


The discourse deictic uses of \textitbf{itu} ‘\textsc{d.dist}’ are very commonly extended to that of a “sentence connective” that signals “a causal link between two propositions” {\citep[125]{Diessel1999}}, as illustrated in (0) and (0). Standing alone, \textitbf{itu} ‘\textsc{d.dist}’ introduces a reason relation as in (0). When co-occurring with the relativizer \textitbf{yang} ‘\textsc{rel}’, \textitbf{itu} ‘\textsc{d.dist}’ marks a result relation as in (0).



In (0), the speaker recounts a conversation with a local doctor after a motorbike accident. In using \textitbf{itu} ‘\textsc{d.dist}’ the doctor summarizes the speaker’s comments concerning her health and creates an overt link to his explanation why she is in pain. In this context \textitbf{itu} ‘\textsc{d.dist}’ functions as a causal link that marks a reason relation.
\end{styleBodyvvafter}

\begin{styleExampleTitle}
Discourse deictic uses of \textitbf{itu} ‘\textsc{d.dist}’: Marker of a reason relation
\end{styleExampleTitle}

\begin{tabular}{llllllllllllllll}
\lsptoprule
\label{bkm:Ref338956128}
\gll {sa} {\multicolumn{2}{l}{bilang,}} {\multicolumn{2}{l}{tulang}} {\multicolumn{3}{l}{bahu}} {\multicolumn{2}{l}{yang}} {pata,} {tulang} {rusuk,} {o,} {a,}\\ %
& \textsc{1sg} & \multicolumn{2}{l}{say} & \multicolumn{2}{l}{bone} & \multicolumn{3}{l}{shoulder} & \multicolumn{2}{l}{\textsc{rel}} & break & bone & rib & oh! & ah!\\
& \multicolumn{2}{l}{mama} & \multicolumn{2}{l}{\bluebold{itu}} & \multicolumn{2}{l}{hanya} & ko & \multicolumn{2}{l}{jatu} & \multicolumn{6}{l}{kaget}\\
& \multicolumn{2}{l}{mother} & \multicolumn{2}{l}{\textsc{d.dist}} & \multicolumn{2}{l}{only} & \textsc{2sg} & \multicolumn{2}{l}{fall} & \multicolumn{6}{l}{feel.startled(.by)}\\
\lspbottomrule
\end{tabular}
\ea
\glt 
‘I said, ‘(it’s my) shoulder bone that is broken, (my) ribs’, (the doctor said,) ‘oh! ah, Mother \bluebold{that is} just \bluebold{because} you’re in shock’’ \textstyleExampleSource{[081015-005-NP.0048]}
\z


The utterance in (0) is part of a conversation about the speaker’s husband who had fallen sick after a straining journey. Recounting some details about the journey, the speaker relates that her husband had returned home hungry. At the beginning of the next clause \textitbf{itu} ‘\textsc{d.dist}’ summarizes this account and, combined with the relativizer \textitbf{yang} ‘\textsc{rel}’, signals a result relation: \textitbf{itu yang de sakit} ‘that’s why he’s sick’.


\begin{styleExampleTitle}
Discourse deictic uses of \textitbf{itu} ‘\textsc{d.dist}’: Marker of a result relation
\end{styleExampleTitle}

\begin{tabular}{lllllllllll}
\lsptoprule
\label{bkm:Ref338956129}
\gll {pace} {de} {tida} {makang} {…} {lapar,} {\bluebold{itu}} {\bluebold{yang}} {de} {sakit}\\ %
& man & \textsc{3sg} & \textsc{neg} & eat &  & be.hungry & \textsc{d.dist} & \textsc{rel} & \textsc{3sg} & be.sick\\
\lspbottomrule
\end{tabular}
\ea
\glt
‘he (my) husband hadn’t eaten … (he was) hungry, \bluebold{that’s why} he’s sick’ \textstyleExampleSource{[080921-004b-CvNP.0003/0007]}
\end{styleFreeTranslEngxvpt}

\paragraph[Placeholder uses of demonstratives]{Placeholder uses of demonstratives}
\label{bkm:Ref350604711}\label{bkm:Ref320548826}
Demonstratives are also rather commonly employed pronominally as “placeholders” in the context of “word-formulation trouble”, as Hayashi and \citet{Yoon2006} show in their cross-linguistic study. In this function, they serve “as temporary substitutes for specific lexical items that have eluded the speaker” {(2006: 499)}. Besides, demonstratives are also used as “interjective hesitators” that signal “the speaker’s hesitation in utterance production” {(2006: 512–513)}. (See also {Dooley and Levinsohn 2001: 36.})



The placeholder and hesitator uses also apply to the Papuan Malay demonstratives, as shown in (0) to (0) and in (0), respectively. In this function, however, the long demonstrative forms are attested, while the short forms are unattested.
\end{styleBodyvafter}


As placeholders, the demonstratives can substitute for any lexical item, such as nouns as in (0), personal pronouns as in (0), or verbs as in (0). More investigation is needed, though, to account for the alternation of \textitbf{ini} ‘\textsc{d.prox}’ and \textitbf{itu} ‘\textsc{d.dist}’ in this context. In most cases, as in (0) and (0), the demonstrative is set off from the subsequently produced target word by a comma intonation (“{\textbar}”). Often, however, there is no audible pause between the placeholder and the target word as in (0).
\end{styleBodyvvafter}

\begin{styleExampleTitle}
Placeholder for a proper noun
\end{styleExampleTitle}

\begin{tabular}{llllll}
\lsptoprule
\label{bkm:Ref338956130}
\gll {…} {saya} {ingat} {\bluebold{ini}} {\bluebold{Ise}}\\ %
&  & \textsc{1sg} & remember & \textsc{d.prox} & Ise\\
\lspbottomrule
\end{tabular}
\ea
\glt 
‘(at that particular time) I remembered, \bluebold{what’s-her-name, Ise}’ \textstyleExampleSource{[080917-008-NP.0102]}
\z

\begin{styleExampleTitle}
Placeholder for a personal pronoun
\end{styleExampleTitle}

\begin{tabular}{lllllllll}
\lsptoprule
\label{bkm:Ref338956131}
\gll {wa,} {\bluebold{ini}} {{\textbar}} {\bluebold{kitong}} {lari{\Tilde}lari} {kemaring} {sampe} {…}\\ %
& wow & \textsc{d.prox} &  & \textsc{1pl} & \textsc{rdp}{\Tilde}run & yesterday & reach & \\
\lspbottomrule
\end{tabular}
\ea
\glt 
‘wow, \bluebold{what’s-their-name}, \bluebold{we} drove yesterday all the way to …’ \textstyleExampleSource{[081006-033-Cv.0007]}
\z

\begin{styleExampleTitle}
Placeholder for a verb
\end{styleExampleTitle}

\begin{tabular}{llllllll}
\lsptoprule
\label{bkm:Ref338956132}
\gll {skarang} {sa} {\bluebold{itu}} {{\textbar}} {\bluebold{simpang}} {sratus} {ribu}\\ %
& now & \textsc{1sg} & \textsc{d.dist} &  & store & one.hundred & thousand\\
\lspbottomrule
\end{tabular}
\ea
\glt 
‘now I (already), \bluebold{what’s-its-name}, \bluebold{set aside} one hundred thousand (rupiah)’ \textstyleExampleSource{[081110-002-Cv.0039]}
\z


While in (0) to (0), the demonstratives are used referentially to substitute for a lexical item, this is not the case in (0). In this example, \textitbf{itu} ‘\textsc{d.dist}’ is used as a non-referential interjective hesitator. This is evidenced by the fact that \textitbf{itu} ‘\textsc{d.dist}’ does not agree with adnominally used \textitbf{ini} ‘\textsc{d.prox}’, which modifies the head nominal \textitbf{pace} ‘man’. Overall, however, the hesitator uses of the demonstratives are rare; most commonly, Papuan Malay speakers use the hesitator \textitbf{e(m)} ‘uh’ (see §5.13.2).


\begin{styleExampleTitle}
Interjective hesitator
\end{styleExampleTitle}

\begin{tabular}{llllllllll}
\lsptoprule
\label{bkm:Ref338956133}
\gll {yo} {\bluebold{itu}} {\bluebold{itu}} {\bluebold{pace}} {\bluebold{ini}} {de} {baru} {ambil} {…}\\ %
& oh! & \textsc{d.dist} & \textsc{d.dist} & man & \textsc{d.prox} & \textsc{3sg} & recently & fetch & \\
\lspbottomrule
\end{tabular}
\ea
\glt 
‘oh, \bluebold{umh}, \bluebold{umh}, \bluebold{this man}, he recently took …’ \textstyleExampleSource{[081011-009-Cv.0007]}
\z


Further investigation is required, to explore whether and in which ways Papuan Malay makes a distinction between the placeholder and non-referential hesitator uses of its demonstratives and whether it may in fact be using the right-displacement attested in (0) as a deliberate construction for emphasis in some contexts.
\end{styleBodyxvafter}

\section{Locatives}
\label{bkm:Ref322435478}
In the following sections, the syntactic properties and forms of the Papuan Malay locatives are reviewed and discussed (§7.2.1), followed by an in-depth discussion of their different functions and domains of use (§7.2.2).
\end{styleBodyxvafter}

\subsection{Syntax and forms of locatives}
\label{bkm:Ref322436096}
The distributional properties of the locatives are briefly reviewed in §7.2.1.1. This review is followed in §7.2.1.2 by a discussion of the distribution and frequencies of the pronominally versus the adnominally used locatives.
\end{styleBodyxvafter}

\paragraph[Distributional properties of locatives]{Distributional properties of locatives}
\label{bkm:Ref322427936}
The Papuan Malay locatives have the following distributional properties (for more details see §5.7):


%\setcounter{itemize}{0}
\begin{itemize}
\item \begin{styleIIndented}
Substitution for noun phrases that occur in prepositional phrases (pronominal uses) (§5.7.1)
\end{styleIIndented}\item \begin{styleIIndented}
Modification with demonstratives or relative clauses (pronominal uses) (§5.7.1)
\end{styleIIndented}\item \begin{styleIiiI}
Co-occurrence with noun phrases(adnominal uses): \textsc{n}/\textsc{np} \textsc{loc} (§5.7.2)
\end{styleIiiI}\end{itemize}
\paragraph[Distribution of the pronominally versus the adnominally used locatives]{Distribution of the pronominally versus the adnominally used locatives}
\label{bkm:Ref322427937}
This section describes the distribution and frequencies of the pronominally versus the adnominally used locatives (their semantic distinctions are discussed in §7.2.2).



The corpus includes a total of 1,366 locative tokens: 494 \textitbf{sini} ‘\textsc{l.prox}’ (36\%), 411 \textitbf{situ} ‘\textsc{l.med}’ (30\%), and 461 \textitbf{sana} ‘\textsc{l.dist}’ (34\%) tokens. Most commonly the locatives are employed pronominally (1,106/1,366 tokens – 91\%), while their adnominal uses are considerably less common (260/1,366 tokens – 19\%), as shown in Table  ‎7 .7.
\end{styleBodyvvafter}

\begin{stylecaption}
\label{bkm:Ref317082098}Table ‎7.\stepcounter{Table}{\theTable}:  Locatives according to their syntactic functions
\end{stylecaption}

\begin{tabular}{lllllll} & \multicolumn{2}{l}{ \textitbf{sini} ‘\textsc{l.prox}’} & \multicolumn{2}{l}{ \textitbf{situ} ‘\textsc{l.med}’} & \multicolumn{2}{l}{ \textitbf{sana} ‘\textsc{l.dist}}\\
\lsptoprule
Pronominal uses & \raggedleft 416 & \raggedleft 84\% & \raggedleft 345 & \raggedleft 84\% & \raggedleft 345 & \raggedleft\arraybslash 75\%\\
Adnominal uses & \raggedleft 78 & \raggedleft 16\% & \raggedleft 66 & \raggedleft 16\% & \raggedleft 116 & \raggedleft\arraybslash 25\%\\
Total & \raggedleft 494 & \raggedleft 100\% & \raggedleft 411 & \raggedleft 100\% & \raggedleft 461 & \raggedleft\arraybslash 100\%\\
\lspbottomrule
\end{tabular}

The distribution of the pronominally used locatives is presented in Table  ‎7 .8. In their pronominal uses, as mentioned in §5.7.1, the locatives always occur in prepositional phrases. Most often they are introduced with an overt preposition: 384/416 \textitbf{sini} ‘\textsc{l.prox}’ (92\%), 302/345 \textitbf{situ} ‘\textsc{l.med}’ (87\%), and 311/345 \textitbf{sana} ‘\textsc{l.dist}’ (90\%) tokens. When the context allows the disambiguation of the semantic role or relationship of the locative, however, the preposition can also be deleted (the elision of prepositions is discussed in §10.1.5): 32/416 \textitbf{sini} ‘\textsc{l.prox}’ (8\%), 45/345 \textitbf{situ} ‘\textsc{l.med}’ (13\%), and 34/345 \textitbf{sana} ‘\textsc{l.dist}’ (10\%) tokens. Overall, however, pronominally used locatives with zero-preposition are rather rare (109/1,106 tokens – 10\%), as shown in Table  ‎7 .8.


\begin{stylecaption}
\label{bkm:Ref320640149}Table ‎7.\stepcounter{Table}{\theTable}:  Pronominally used locatives in prepositional phrases (PP) with or without preposition (\textsc{prep})
\end{stylecaption}

\begin{tabular}{lllllll} & \multicolumn{2}{l}{ \textitbf{sini} ‘\textsc{l.prox}’} & \multicolumn{2}{l}{ \textitbf{situ} ‘\textsc{l.med}’} & \multicolumn{2}{l}{ \textitbf{sana} ‘\textsc{l.dist}}\\
\lsptoprule
PP with \textsc{prep} & \raggedleft 384 & \raggedleft 92\% & \raggedleft 302 & \raggedleft 88\% & \raggedleft 311 & \raggedleft\arraybslash 90\%\\
PP with zero \textsc{prep} & \raggedleft 32 & \raggedleft 8\% & \raggedleft 43 & \raggedleft 12\% & \raggedleft 34 & \raggedleft\arraybslash 10\%\\
Total & \raggedleft 416 & \raggedleft 100\% & \raggedleft 345 & \raggedleft 100\% & \raggedleft 345 & \raggedleft\arraybslash 100\%\\
\lspbottomrule
\end{tabular}

The distribution of the adnominally used locatives is presented in Table  ‎7 .9. In their adnominal uses, as mentioned in §5.7.2, the locatives most commonly co-occur with noun phrases that occur in prepositional phrases (225/260 tokens – 87\%). Like the pronominally used locatives, the vast majority of adnominally used locatives occur in prepositional phrases with an overt preposition: 46/78 \textitbf{sini} ‘\textsc{l.prox}’ (59\%), 55/66 \textitbf{situ} ‘\textsc{l.med}’ (84\%), and 86/116 \textitbf{sana} ‘\textsc{l.dist}’ (74\%) tokens. Far fewer locative tokens occur in prepositional phrases with zero preposition: 16/78 \textitbf{sini} ‘\textsc{l.prox}’ (21\%), 6/66 \textitbf{situ} ‘\textsc{l.med}’ (9\%), and 16/116 \textitbf{sana} ‘\textsc{l.dist}’ (14\%) tokens. The number of locative tokens occurring in unembedded noun phrases is equally low or still lower: 16/78 \textitbf{sini} ‘\textsc{l.prox}’ (21\%), 5/66 \textitbf{situ} ‘\textsc{l.med}’ (7\%), and 14/116 \textitbf{sana} ‘\textsc{l.dist}’ (12\%) tokens.


\begin{stylecaption}
\label{bkm:Ref317070623}Table ‎7.\stepcounter{Table}{\theTable}:  Adnominally used locatives in prepositional phrases and unembedded noun phrases
\end{stylecaption}

\begin{tabular}{lllllll} & \multicolumn{2}{l}{ \textitbf{sini} ‘\textsc{l.prox}’} & \multicolumn{2}{l}{ \textitbf{situ} ‘\textsc{l.med}’} & \multicolumn{2}{l}{ \textitbf{sana} ‘\textsc{l.dist}}\\
\lsptoprule
PP with \textsc{prep} & \raggedleft 46 & \raggedleft 59\% & \raggedleft 55 & \raggedleft 84\% & \raggedleft 86 & \raggedleft\arraybslash 74\%\\
PP with zero \textsc{prep} & \raggedleft 16 & \raggedleft 21\% & \raggedleft 6 & \raggedleft 9\% & \raggedleft 16 & \raggedleft\arraybslash 14\%\\
Unembedded NP & \raggedleft 16 & \raggedleft 21\% & \raggedleft 5 & \raggedleft 7\% & \raggedleft 14 & \raggedleft\arraybslash 12\%\\
Total & \raggedleft 78 & \raggedleft 100\% & \raggedleft 66 & \raggedleft 100\% & \raggedleft 116 & \raggedleft\arraybslash 100\%\\
\lspbottomrule
\end{tabular}
\subsection{Functions of locatives}
\label{bkm:Ref438308386}\label{bkm:Ref322436092}
The locatives have a number of different functions and uses which are discussed in more detail in the following sections: spatial uses in §7.2.2.1, figurative locational uses in §7.2.2.2, temporal uses in §7.2.2.3, psychological uses in §7.2.2.4, and textual uses in §7.2.2.5.
\end{styleBodyxvafter}

\paragraph[Spatial uses of locatives]{Spatial uses of locatives}
\label{bkm:Ref322418216}
In their spatial uses, the Papuan Malay locatives designate the location of an entity relative to that of the speaker and focus the hearer’s attention to the specific location of these entities. In the following, two issues are explored in more detail: the semantic distinctions between the three locatives, and the semantic distinctions between the pronominally and adnominally used locatives.
\end{styleBodyxvafter}

\subparagraph[Semantic distinctions between the three locatives]{Semantic distinctions between the three locatives}

Generally speaking, proximal \textitbf{sini} ‘\textsc{l.prox}’ signals proximity to a deictic center, while distal \textitbf{sana} ‘\textsc{l.dist}’ expresses distance from this center. Medial \textitbf{situ} ‘\textsc{l.med}’ indicates mid distance; that is, the referent is more remote from the speaker than the referent of \textitbf{sini} ‘\textsc{l.prox}’ but not as far as the referent of \textitbf{sana} ‘\textsc{l.dist}’. The actual distances signaled with the locatives are relative, however, and depend on the speakers’ perceptions. The data also shows that the locatives are very commonly used independently of the parameter of visibility. Although \textitbf{sini} ‘\textsc{l.prox}’ most commonly denotes visible locations, it can also refer to invisible ones; likewise, the non-proximal locatives can refer to visible or invisible locations.



The spatial uses of \textitbf{sini} ‘\textsc{l.prox}’ are illustrated in (0) and (0). The semantic distinctions between \textitbf{sini} ‘\textsc{l.prox}’ and \textitbf{sana} ‘\textsc{l.dist}’ are shown in (0). The spatial uses of \textitbf{situ} ‘\textsc{l.med}’ and its semantic distinctions from \textitbf{sini} ‘\textsc{l.prox}’ and \textitbf{sana} ‘\textsc{l.dist}’ are illustrated in (0) and (0).
\end{styleBodyvafter}


The examples in (0) and (0) illustrate the spatial uses of \textitbf{sini} ‘\textsc{l.prox}’. In both cases, adnominally used \textitbf{sini} ‘\textsc{l.prox}’ indicates the location of an entity close to the speaker: \textitbf{ember sini} ‘the bucket here’ in (0) and \textitbf{Sawar sini} ‘Sawar here’ in (0). The actual distances signaled with \textitbf{sini} ‘\textsc{l.prox}’ differ, however, depending on the speakers’ perceptions. In (0), \textitbf{ember sini} ‘the bucket here’ is standing right next to the speaker. By contrast in (0), \textitbf{Sawar sini} ‘Sawar here’ denotes a location that is situated about ten kilometers away from the speaker’s location. The speaker, however, perceives \textitbf{Sawar} as being close to his own location given that \textitbf{Apawer} is situated still further away. The examples in (0) and (0) also illustrate that \textitbf{sini} ‘\textsc{l.prox}’ is used independently of the parameter of visibility: the locative is used for a visible location in (0) and for an invisible one in (0).
\end{styleBodyvvafter}

\begin{styleExampleTitle}
Spatial uses of \textitbf{sini} ‘\textsc{l.prox}’
\end{styleExampleTitle}

\begin{tabular}{lllllll}
\lsptoprule
\label{bkm:Ref338956134}
\gll {sa} {su} {taru} {di} {\bluebold{ember}} {\bluebold{sini}}\\ %
& \textsc{1sg} & already & put & at & bucket & \textsc{l.prox}\\
\lspbottomrule
\end{tabular}
\ea
\glt 
‘I already put (the fish) in \bluebold{the bucket here}’ \textstyleExampleSource{[080917-006-CvHt.0003]}
\z

\begin{tabular}{lllllllllll}
\lsptoprule
\label{bkm:Ref338956135}
\gll {de} {mulay} {turung} {dari} {Apawer} {…} {sampe} {di} {\bluebold{Sawar}} {\bluebold{sini}}\\ %
& \textsc{3sg} & start & descend & from & Apawer &  & reach & at & Sawar & \textsc{l.prox}\\
\lspbottomrule
\end{tabular}
\ea
\glt 
‘he (the ancestor) started coming down from Apawer … (and) reached \bluebold{Sawar here}’ \textstyleExampleSource{[080922-010a-CvNF.0149]}
\z


The example in (0) illustrates the semantic distinctions between \textitbf{sini} ‘\textsc{l.prox}’ and \textitbf{sana} ‘\textsc{l.dist}’. The utterance occurred during a conversation outside at night. Noting that their neighbors are also sitting outside, the speaker employs the distal locative to refer to the neighbors’ location \textitbf{sana} ‘over there’ and the proximal locative to refer to their own location \textitbf{sini} ‘here’.


\begin{styleExampleTitle}
Spatial uses of \textitbf{sini} ‘\textsc{l.prox}’ and \textitbf{sana} ‘\textsc{l.dist}’
\end{styleExampleTitle}

\begin{tabular}{lllllllllll}
\lsptoprule
\label{bkm:Ref338956136}
\gll {dong} {juga} {duduk} {di} {\bluebold{sana}} {tong} {juga} {duduk} {di} {\bluebold{sini}}\\ %
& \textsc{3pl} & also & sit & at & \textsc{l.dist} & \textsc{1pl} & also & sit & at & \textsc{l.prox}\\
\lspbottomrule
\end{tabular}
\ea
\glt 
‘they also sit (outside) \bluebold{over there}, we also sit (outside) \bluebold{here}’ \textstyleExampleSource{[081025-009b-Cv.0075]}
\z


The examples in (0) and (0) show the spatial uses of \textitbf{situ} ‘\textsc{l.med}’ and its semantic distinctions from \textitbf{sini} ‘\textsc{l.prox}’ and \textitbf{sana} ‘\textsc{l.dist}’.



The exchange in (0) took place at night in front of the house while a meeting took place inside in the living room where the teenagers usually sleep. The young people were waiting for the guests to leave so that they could go to sleep. Employing \textitbf{situ} ‘\textsc{l.med}’, the first teenager wonders what the adults are doing \textitbf{situ} ‘there’ in the living room. Finally, the second teenager suggests they do not wait any longer: using \textitbf{sini} ‘\textsc{l.prox}’ she proposes that they sleep \textitbf{luar sini} ‘outside here’. The utterance in (0) occurred during a conversation about a street-building project. The speaker informs his interlocutor that the construction work has already reached the village of \textitbf{Warmer}, located to the east of the interlocutors’ location. Employing \textitbf{situ} ‘\textsc{l.med}’ and \textitbf{sana} ‘\textsc{l.dist}’, the speaker maintains that the construction work would continue \textitbf{dari situ} ‘from there (Warmer)’ further eastwards \textitbf{ke sana} ‘to over there’.
\end{styleBodyvvafter}

\begin{styleExampleTitle}
Spatial uses of \textitbf{situ} ‘\textsc{l.med}’
\end{styleExampleTitle}

\begin{tabular}{lllllllll}
\lsptoprule
\label{bkm:Ref338956137}
\gll { & Teenager-1: & dong & dong & biking & apa & \bluebold{situ} & …}\\ %
&  &  & \textsc{3pl} & \textsc{3pl} & make & what & \textsc{l.med} & \\
\lspbottomrule
\end{tabular}
\begin{styleFreeTranslIndentiicmEng}
Teenager-1: ‘what are they they doing \bluebold{there}? …’
\end{styleFreeTranslIndentiicmEng}

\begin{tabular}{llllllll} &  & Teenager-2: & yo, & kitong & tidor & \bluebold{luar} & \bluebold{sini}\\
\lsptoprule
&  &  & yes & 1\textsc{pl} & sleep & outside & \textsc{l.prox}\\
\lspbottomrule
\end{tabular}
\begin{styleFreeTranslIndentiicmEng}
Teenager-2: ‘yes, we sleep \bluebold{outside here}’ \textstyleExampleSource{[080921-009-Cv.0001/0013]}
\end{styleFreeTranslIndentiicmEng}

\begin{tabular}{lllllllllllllll}
\lsptoprule
\label{bkm:Ref338956139}
\gll {\multicolumn{2}{l}{yo,}} {\multicolumn{2}{l}{mulay}} {\multicolumn{2}{l}{menuju}} {\multicolumn{3}{l}{jembatang}} {Warmer} {…} {kalo} {dari} {\bluebold{situ}}\\ %
& \multicolumn{2}{l}{yes} & \multicolumn{2}{l}{start} & \multicolumn{2}{l}{aim.at} & \multicolumn{3}{l}{bridge} & Warmer &  & if & from & \textsc{l.med}\\
& ke & \multicolumn{2}{l}{\bluebold{sana},} & \multicolumn{2}{l}{o} & \multicolumn{2}{l}{itu} & dia & \multicolumn{6}{l}{…}\\
& to & \multicolumn{2}{l}{\textsc{l.dist}} & \multicolumn{2}{l}{oh!} & \multicolumn{2}{l}{\textsc{d.dist}} & \textsc{3sg} & \multicolumn{6}{l}{}\\
\lspbottomrule
\end{tabular}
\ea
\glt 
‘yes, (they) started working toward the Warmer bridge … when (they’ll work the stretch of the street) from \bluebold{there} to \bluebold{over there}, oh, what’s-its-name, it …’ \textstyleExampleSource{[081006-033-Cv.0013/0015/0017]}
\z


The examples in (0) and (0) again show that distances signaled with the non-proximal locatives are relative. In (0) \textitbf{sana} ‘\textsc{l.dist}’ refers to the neighbor’s house, situated about fifty meters away from where the speakers are sitting. By contrast, \textitbf{situ} ‘\textsc{l.med}’ in (0) denotes the village of Warmer which is located several kilometers away from the speaker’s location, while \textitbf{sana} ‘\textsc{l.dist}’ refers to the area beyond Warmer. Besides, these examples show that the non-proximal locatives are also used independently of the parameter of visibility. Distal \textitbf{sana} ‘\textsc{l.dist}’ is used for a visible location in (0) and an invisible one in (0). Medial \textitbf{situ} ‘\textsc{l.med}’ refers to a visible location in (0) and an invisible one in (0).
\end{styleBodyxvafter}

\subparagraph[Semantic distinctions between the pronominally and adnominally used locatives]{Semantic distinctions between the pronominally and adnominally used locatives}

In designating the location of a referent relative to that of the speaker, Papuan Malay makes a distinction between the pronominally and the adnominally used locatives.



Pronominally used locatives provide additional information about the location of an entity or referent without restricting its referential scope. Adnominally used locatives, by contrast, have a restrictive function, thereby assisting the hearer in the identification of the referent. That is, by directing the hearer’s attention to the referent’s location, adnominal locatives indicate that the referent is precisely the one situated in the location designated by the locative. This distinction is illustrated with the (near) contrastive examples in (0) and (0).
\end{styleBodyvafter}


The prepositional phrases with pronominally used \textitbf{sini} ‘\textsc{l.prox}’ in (0) and (0) provide additional information about the location of the referents, information that is non-essential for their identification: \textitbf{orang di sini} ‘the people here’ in (0) and \textitbf{dorang di sini} ‘them here’ in (0). By contrast, in (0) and (0) the respective head nominals \textitbf{orang} ‘person’ and \textitbf{dorang} ‘\textsc{3pl}’ are modified with \textitbf{sini} ‘\textsc{l.prox}’. In both cases, the locative indicates that the referents of \textitbf{orang} ‘person’ and \textitbf{dorang} ‘\textsc{3pl}’ are precisely the ones located \textitbf{sini} ‘here’ as opposed to other locations: \textitbf{orang sini} ‘the people that are here’ in (0) and \textitbf{dorang sini} ‘they that are here’ in (0).
\end{styleBodyvvafter}

\begin{styleExampleTitle}
Adnominally versus pronominally used locatives
\end{styleExampleTitle}

\begin{tabular}{lllllll}
\lsptoprule
\label{bkm:Ref338956140}\label{bkm:Ref319582711}
\gll {\label{bkm:Ref319582727}} {\bluebold{orang}} {\bluebold{di}} {\bluebold{sini}} {bilang} {pake{\Tilde}pake}\\ %
&  & person & at & \textsc{l.prox} & say & practice.black.magic\\
\lspbottomrule
\end{tabular}
\begin{styleFreeTranslIndentiicmEng}
‘\bluebold{the people here} say ‘black magic’’ \textstyleExampleSource{[081006-022-CvEx.0028]}
\end{styleFreeTranslIndentiicmEng}

\begin{tabular}{llllllll} & \label{bkm:Ref319582729} & jadi & \bluebold{orang} & \bluebold{sini} & bilang, & kemaring & dulu\\
\lsptoprule
&  & so & person & \textsc{l.prox} & say & yesterday & be.prior\\
\lspbottomrule
\end{tabular}
\begin{styleFreeTranslIndentiicmEng}
‘so \bluebold{the people (that are) here} say ‘the day before yesterday’’ \textstyleExampleSource{[081006-019-Cv.0015]}
\end{styleFreeTranslIndentiicmEng}

\begin{tabular}{llllllllllll}
\lsptoprule
\label{bkm:Ref338956141}\label{bkm:Ref320882456}
\gll {\label{bkm:Ref320882549}} {Lodia} {datang} {ke} {mari,} {de} {kas} {bodo} {\bluebold{dorang}} {\bluebold{di}} {\bluebold{sini}}\\ %
&  & Lodia & come & to & hither & \textsc{3sg} & give & be.stupid & \textsc{3pl} & at & \textsc{l.prox}\\
\lspbottomrule
\end{tabular}
\begin{styleFreeTranslIndentiicmEng}
‘(when) Lodia came here, she told \bluebold{them here} how stupid they were’ (Lit. ‘made \bluebold{them here} stupid’) \textstyleExampleSource{[081115-001a-Cv.0136]}
\end{styleFreeTranslIndentiicmEng}

\begin{tabular}{llllllllll} & \label{bkm:Ref319582731} & baru & sa & liat & \bluebold{dorang} & \bluebold{sini} & su & terlalu & enak\\
\lsptoprule
&  & and.then & \textsc{1sg} & see & \textsc{3pl} & \textsc{l.prox} & already & too & be.pleasant\\
\lspbottomrule
\end{tabular}
\begin{styleFreeTranslIndentiicmEngxxpt}
[Comment about ill-behaved teenagers:] ‘and then I see \bluebold{(that) they (that are) here} already (have) too pleasant (lives)’ \textstyleExampleSource{[081115-001a-Cv.0311]}
\end{styleFreeTranslIndentiicmEngxxpt}

\paragraph[Figurative locational uses of locatives]{Figurative locational uses of locatives}
\label{bkm:Ref320548328}
The spatial uses of the locatives can be expanded to figurative locational uses in narratives. Employing a locative preceded by \textitbf{sampe di} ‘reach at’, the narrators bring their stories to a figurative locational endpoint. Such uses are attested for \textitbf{sini} ‘\textsc{l.prox}’ as in (0) and \textitbf{situ} ‘\textsc{l.med}’ as in (0), but not for \textitbf{sana} ‘\textsc{l.dist}’.


\begin{tabular}{lllllllll}
\lsptoprule
\label{bkm:Ref338956142}
\gll {…} {sa} {su} {sembu,} {trima-kasi} {sampe} {di} {\bluebold{sini}}\\ %
&  & \textsc{1sg} & already & be.healed & thank.you & reach & at & \textsc{l.prox}\\
\lspbottomrule
\end{tabular}
\ea
\glt 
‘[after this accident] I already recovered, thank you! \bluebold{this is all}’ (Lit. ‘reach \bluebold{here}’) \textstyleExampleSource{[081015-005-NP.0051]}
\z

\begin{tabular}{llllllll}
\lsptoprule
\label{bkm:Ref338956143}
\gll {sa} {pikir} {mungking} {sampe} {di} {\bluebold{situ}} {dulu}\\ %
& \textsc{1sg} & think & maybe & reach & at & \textsc{l.med} & first\\
\lspbottomrule
\end{tabular}
\ea
\glt
‘I think maybe \bluebold{that’s all} for now’ (Lit. ‘reach \bluebold{there}’) \textstyleExampleSource{[080919-004-NP.0083]}
\end{styleFreeTranslEngxvpt}

\paragraph[Temporal uses of locatives]{Temporal uses of locatives}
\label{bkm:Ref322418223}
The locative \textitbf{situ} ‘\textsc{l.med}’ also has temporal uses. Preceded by the preposition \textitbf{dari} ‘from’, \textitbf{situ} ‘\textsc{l.med}’ signals the temporal setting of the event talked about with respect to some temporal reference point in the past, as illustrated in (0). Overall, however, this domain of use is not very common, with the corpus containing only two such occurrences. The proximal and distal locatives are not attested to have temporal uses.
\end{styleBodyxafter}

\begin{tabular}{llllllllll}
\lsptoprule
\label{bkm:Ref338956144}
\gll {\bluebold{dari}} {\multicolumn{2}{l}{\bluebold{situ}}} {sa} {punya} {mama} {tida} {maw} {jualang}\\ %
& from & \multicolumn{2}{l}{\textsc{l.med}} & \textsc{1sg} & \textsc{poss} & mother & \textsc{neg} & want & merchandise/sell\\
& \multicolumn{2}{l}{pagi} & \multicolumn{7}{l}{lagi}\\
& \multicolumn{2}{l}{morning} & \multicolumn{7}{l}{again}\\
\lspbottomrule
\end{tabular}
\ea
\glt
‘\bluebold{from that moment on} my mother didn’t want to do any more vending in the morning’ (Lit. ‘\bluebold{from there}’) \textstyleExampleSource{[081014-014-NP.0006]}
\end{styleFreeTranslEngxvpt}

\paragraph[Psychological uses of locatives]{Psychological uses of locatives}
\label{bkm:Ref320352963}
The locatives also have limited psychological uses to signal the speakers’ emotional involvement and attitudes. The corpus contains a fair number of utterances in which the speakers switch from \textitbf{sana} ‘\textsc{l.dist}’ to \textitbf{situ} ‘\textsc{l.med}’ to refer to the same location. With this switch the speakers indicate that the location has become vivid to their minds and psychologically closer than \textitbf{sana} ‘\textsc{l.dist}’ would signal, as illustrated in (0) and (0).



In (0) a father relates that he will bring his two oldest children to the provincial capital Jayapura for further schooling once the younger one has finished high school. Anaphorically used \textitbf{sana} ‘\textsc{l.dist}’ signals that Jayapura is at considerable distance from the speaker’s current location (ca. 300 km). The subsequent use of \textitbf{situ} ‘\textsc{l.med}’ indicates that with his two children going to live there, distant Jayapura has become psychologically much closer.
\end{styleBodyvvafter}

\begin{styleExampleTitle}
Psychological uses: Example \#1
\end{styleExampleTitle}

\begin{tabular}{llllm{4.5984238E-4in}lllllllll}
\lsptoprule
\label{bkm:Ref338956145}
\gll {\multicolumn{2}{l}{kalo}} {Ise} {\multicolumn{2}{l}{ni}} {\multicolumn{3}{l}{selesay}} {saya} {mo} {bawa} {dong} {dua}\\ %
& \multicolumn{2}{l}{if} & Ise & \multicolumn{2}{l}{\textsc{d.prox}} & \multicolumn{3}{l}{finish} & \textsc{1sg} & want & bring & \textsc{3pl} & two\\
& ke & \multicolumn{3}{l}{\bluebold{sana}} & \multicolumn{2}{l}{tinggal} & di & \multicolumn{6}{l}{\bluebold{situ}}\\
& to & \multicolumn{3}{l}{\textsc{l.dist}} & \multicolumn{2}{l}{stay} & at & \multicolumn{6}{l}{\textsc{l.med}}\\
\lspbottomrule
\end{tabular}
\ea
\glt 
‘when Ise here has finished (her schooling) I want to bring the two of them to (Jayapura) \bluebold{over there} to live \bluebold{there}’ \textstyleExampleSource{[081025-003-Cv.0135]}
\z


Likewise in (0) the speaker switches from \textitbf{sana} ‘\textsc{l.dist}’ to \textitbf{situ} ‘\textsc{l.med}’ to refer to the Mambramo area \textitbf{sana} ‘over there’, situated about 100 km to the west. The switch occurs at the moment when the speaker considers his own involvement with the Mambramo area, namely that he has never been \textitbf{situ} ‘there’. Again, this switch indicates that the location talked about has become more vivid and psychologically closer to the speaker’s mind.


\begin{styleExampleTitle}
Psychological uses: Example \#2
\end{styleExampleTitle}

\begin{tabular}{llllllllllllllll}
\lsptoprule
\label{bkm:Ref338956146}
\gll {kaka} {\multicolumn{2}{l}{dong}} {\multicolumn{2}{l}{di}} {\multicolumn{2}{l}{\bluebold{sana}}} {sodara} {\multicolumn{3}{l}{banyak}} {skali} {…} {sa} {juga}\\ %
& oSb & \multicolumn{2}{l}{\textsc{3pl}} & \multicolumn{2}{l}{at} & \multicolumn{2}{l}{\textsc{l.dist}} & sibling & \multicolumn{3}{l}{many} & very &  & \textsc{1sg} & also\\
& \multicolumn{2}{l}{blum} & \multicolumn{2}{l}{perna} & \multicolumn{2}{l}{sa} & \multicolumn{3}{l}{kunjungang} & ke & \multicolumn{5}{l}{\bluebold{situ}}\\
& \multicolumn{2}{l}{not.yet} & \multicolumn{2}{l}{once} & \multicolumn{2}{l}{\textsc{1sg}} & \multicolumn{3}{l}{visit} & to & \multicolumn{5}{l}{\textsc{l.med}}\\
\lspbottomrule
\end{tabular}
\ea
\glt
‘the older relatives \bluebold{over there}, (the) relatives are very many in (the Mambramo area), … me too, I have never been \bluebold{there}’ \textstyleExampleSource{[080922-010a-CvNF.0158]}
\end{styleFreeTranslEngxvpt}

\paragraph[Textual anaphoric uses of locatives]{Textual anaphoric uses of locatives}
\label{bkm:Ref322418229}
In their textual uses, locatives are used anaphorically; that is, they are co-referential with a discourse antecedent that denotes a location. In (0) \textitbf{sini} ‘\textsc{l.prox}’ corefers with the place where the speaker was standing, namely where there were \textitbf{daung klapa … itu} ‘those coconut leaves’. Medial \textitbf{situ} ‘\textsc{l.med}’ in (0) corefers with \textitbf{laut} ‘sea’, and \textitbf{sana} ‘\textsc{l.dist}’ in (0) with \textitbf{sa pu temang} ‘my friend’. The three examples also show that in their anaphoric uses the locatives may be employed pronominally as in (0) and (0), or adnominally as in (0).
\end{styleBodyxafter}

\begin{tabular}{llllllllllllllllllll}
\lsptoprule
\label{bkm:Ref338956154}
\gll {\multicolumn{2}{l}{baru}} {\multicolumn{2}{l}{daung}} {\multicolumn{3}{l}{klapa}} {itu} {\multicolumn{3}{l}{\bluebold{daung}}} {\multicolumn{2}{l}{\bluebold{klapa}}} {\multicolumn{2}{l}{\bluebold{yang}}} {\multicolumn{2}{l}{\bluebold{saya}}} {\multicolumn{2}{l}{\bluebold{ada}}}\\ %
& \multicolumn{2}{l}{and.then} & \multicolumn{2}{l}{leaf} & \multicolumn{3}{l}{coconut} & \textsc{d.dist} & \multicolumn{3}{l}{leaf} & \multicolumn{2}{l}{coconut} & \multicolumn{2}{l}{\textsc{rel}} & \multicolumn{2}{l}{\textsc{1sg}} & \multicolumn{2}{l}{exist}\\
& \bluebold{berdiri} & \multicolumn{2}{l}{\bluebold{itu}} & \multicolumn{2}{l}{…} & sa & \multicolumn{3}{l}{bilang} & … & \multicolumn{2}{l}{dari} & \multicolumn{2}{l}{\bluebold{sini}} & \multicolumn{2}{l}{sa} & \multicolumn{2}{l}{kutuk} & dia\\
& stand & \multicolumn{2}{l}{\textsc{d.dist}} & \multicolumn{2}{l}{} & \textsc{1sg} & \multicolumn{3}{l}{say} &  & \multicolumn{2}{l}{from} & \multicolumn{2}{l}{\textsc{l.prox}} & \multicolumn{2}{l}{\textsc{1sg}} & \multicolumn{2}{l}{curse} & \textsc{3sg}\\
\lspbottomrule
\end{tabular}
\ea
\glt 
‘and then those coconut leaves, \bluebold{those coconut leaves where I was standing} … I said, ‘… from \bluebold{here} I curse him (the evil spirit)’’ \textstyleExampleSource{[080917-008-NP.0101/0103]}
\z

\begin{tabular}{llllllllllllll}
\lsptoprule
\label{bkm:Ref338956158}
\gll {\multicolumn{2}{l}{ey,}} {kam} {dua} {\multicolumn{2}{l}{pi}} {\multicolumn{2}{l}{mandi}} {di} {\bluebold{laut}} {suda!,} {trus} {kam}\\ %
& \multicolumn{2}{l}{hey!} & \textsc{2pl} & two & \multicolumn{2}{l}{go} & \multicolumn{2}{l}{bathe} & at & sea & already & next & \textsc{2pl}\\
& dua & \multicolumn{2}{l}{cuci} & \multicolumn{2}{l}{celana} & \multicolumn{2}{l}{di} & \multicolumn{6}{l}{\bluebold{situ}}\\
& two & \multicolumn{2}{l}{wash} & \multicolumn{2}{l}{trouser} & \multicolumn{2}{l}{at} & \multicolumn{6}{l}{\textsc{l.med}}\\
\lspbottomrule
\end{tabular}
\ea
\glt 
[A mother addressing her young sons:] ‘hey, you two go bathe in the \bluebold{sea} already!, then you two wash (your) trousers \bluebold{there}!’ \textstyleExampleSource{[080917-006-CvHt.0007]}
\z

\begin{tabular}{lllllllllllllllll}
\lsptoprule
\label{bkm:Ref338956159}
\gll {\multicolumn{2}{l}{tong}} {\multicolumn{2}{l}{dari}} {\bluebold{sa}} {\multicolumn{2}{l}{\bluebold{pu}}} {\multicolumn{3}{l}{\bluebold{temang}}} {\multicolumn{2}{l}{pinjam}} {\multicolumn{2}{l}{trening}} {untuk} {besok}\\ %
& \multicolumn{2}{l}{\textsc{1pl}} & \multicolumn{2}{l}{from} & \textsc{1sg} & \multicolumn{2}{l}{\textsc{poss}} & \multicolumn{3}{l}{friend} & \multicolumn{2}{l}{borrow} & \multicolumn{2}{l}{tracksuit} & for & tomorrow\\
& … & \multicolumn{2}{l}{tu} & \multicolumn{3}{l}{yang} & \multicolumn{2}{l}{tadi} & sa & \multicolumn{2}{l}{ke} & \multicolumn{2}{l}{\bluebold{temang}} & \multicolumn{3}{l}{\bluebold{sana}}\\
&  & \multicolumn{2}{l}{\textsc{d.dist}} & \multicolumn{3}{l}{\textsc{rel}} & \multicolumn{2}{l}{earlier} & \textsc{1sg} & \multicolumn{2}{l}{to} & \multicolumn{2}{l}{friend} & \multicolumn{3}{l}{\textsc{l.dist}}\\
\lspbottomrule
\end{tabular}
\ea
\glt
‘we (are back) from \bluebold{my friend} (from whom we) borrowed a tracksuit for tomorrow … that’s why a short while ago I (went) to (my) \bluebold{friend (who is) over there}’ \textstyleExampleSource{[081011-020-Cv.0052/0056]}
\end{styleFreeTranslEngxvpt}

\section{Combining demonstratives and locatives}
\label{bkm:Ref322435482}
Demonstratives and locatives can be combined with an adnominally used demonstrative modifying a pronominally used locative as in (0) and (0), or an adnominally used locative as in (0). In these constructions, the demonstrative serves to intensify the locative, resulting in an emphatic reading that conveys vividness.



Short distal \textitbf{itu} ‘\textsc{d.dist}’ modifies proximal \textitbf{sini} ‘\textsc{l.prox}’ in (0) and medial \textitbf{situ} ‘\textsc{l.med}’ in (0). In (0) long distal \textitbf{itu} ‘\textsc{d.dist}’ modifies distal \textitbf{sana} ‘\textsc{l.dist}’.
\end{styleBodyvxafter}

\begin{tabular}{llllllllllllllll}
\lsptoprule
\label{bkm:Ref338956160}
\gll {\multicolumn{2}{l}{dorang}} {tida} {\multicolumn{2}{l}{bisa}} {dekat} {\multicolumn{2}{l}{sama}} {\multicolumn{2}{l}{dorang}} {…} {di} {\multicolumn{2}{l}{\bluebold{sini}}} {\bluebold{tu}}\\ %
& \multicolumn{2}{l}{\textsc{3pl}} & \textsc{neg} & \multicolumn{2}{l}{be.able} & near & \multicolumn{2}{l}{with} & \multicolumn{2}{l}{\textsc{3pl}} &  & at & \multicolumn{2}{l}{\textsc{l.prox}} & \textsc{d.dist}\\
& ada & \multicolumn{3}{l}{orang} & \multicolumn{3}{l}{swanggi} & \multicolumn{2}{l}{satu} & \multicolumn{2}{l}{de} & \multicolumn{2}{l}{bertobat} & \multicolumn{2}{l}{…}\\
& exist & \multicolumn{3}{l}{person} & \multicolumn{3}{l}{nocturnal.evil.spirit} & \multicolumn{2}{l}{one} & \multicolumn{2}{l}{\textsc{3sg}} & \multicolumn{2}{l}{repent} & \multicolumn{2}{l}{}\\
\lspbottomrule
\end{tabular}
\ea
\glt 
‘they (the evil spirits) can’t be close to them (God’s children)… \bluebold{here (}\blueboldSmallCaps{emph}\bluebold{)} is one evil sorcerer, he has become a Christian’ \textstyleExampleSource{[081006-022-CvEx.0146/0150]}
\z

\begin{tabular}{llllllllllllll}
\lsptoprule
\label{bkm:Ref338956161}
\gll {tida} {\multicolumn{2}{l}{bisa}} {\multicolumn{2}{l}{kamu}} {\multicolumn{2}{l}{tinggal}} {\multicolumn{2}{l}{di}} {situ,} {di} {\bluebold{situ}} {\bluebold{tu}}\\ %
& \textsc{neg} & \multicolumn{2}{l}{be.able} & \multicolumn{2}{l}{\textsc{2pl}} & \multicolumn{2}{l}{stay} & \multicolumn{2}{l}{at} & \textsc{l.med} & at & \textsc{l.med} & \textsc{d.dist}\\
& \multicolumn{2}{l}{ruma} & \multicolumn{2}{l}{tu} & \multicolumn{2}{l}{ada} & \multicolumn{2}{l}{setang} & \multicolumn{5}{l}{banyak}\\
& \multicolumn{2}{l}{house} & \multicolumn{2}{l}{\textsc{d.dist}} & \multicolumn{2}{l}{exist} & \multicolumn{2}{l}{evil.spirit} & \multicolumn{5}{l}{many}\\
\lspbottomrule
\end{tabular}
\ea
\glt 
‘you can’t live there, \bluebold{there (}\blueboldSmallCaps{emph}\bluebold{)}, (in) that house are many evil spirits’ \textstyleExampleSource{[081006-022-CvEx.0164]}
\z

\begin{tabular}{lllllll}
\lsptoprule
\label{bkm:Ref338956162}
\gll {sana,} {te} {ada} {di} {\bluebold{sana}} {\bluebold{itu}}\\ %
& \textsc{l.dist} & tea & exist & at & \textsc{l.dist} & \textsc{d.dist}\\
\lspbottomrule
\end{tabular}
\ea
\glt 
‘there, the tea (is) \bluebold{over there (}\blueboldSmallCaps{emph}\bluebold{)}’ \textstyleExampleSource{[081014-011-CvEx.0010]}
\z


In all attested combinations, it is the distal demonstrative that modifies a locative. Modification of a locative with proximal \textitbf{ini} ‘\textsc{d.prox}’ is also possible although unattested, as discussed in §5.7.1).
\end{styleBodyxvafter}

\section{Summary}
\label{bkm:Ref322435484}
The Papuan Malay demonstratives and locatives are deictic expressions. They provide orientation to the hearer in the outside world and in the speech situation, in spatial as well as in non-spatial domains. Both deictic systems are distance oriented, in that they signal the relative distance of an entity vis-à-vis a deictic center. At the same time, the two systems differ in a number of respects. They are distinct both in terms of their syntactic characteristics and forms and in terms of their functions.



The differences between the demonstratives and the locatives with respect to their syntactic characteristics and forms are summarized in Table  ‎7 .10.
\end{styleBodyvvafter}

\begin{stylecaption}
\label{bkm:Ref322446153}Table ‎7.\stepcounter{Table}{\theTable}:  Syntax and forms of the demonstratives (\textsc{dem}) and locatives (\textsc{loc})
\end{stylecaption}

\tablehead{
 Syntax and forms & \textsc{dem} & \arraybslash \textsc{loc}\\
}
\begin{tabular}{lll}
\lsptoprule
Deictic forms & Two term system:

\begin{itemize}
\item proximal \textitbf{ini} ‘\textsc{d.prox}’

\begin{itemize}
\item distal \textitbf{itu} ‘\textsc{d.dist}’\end{itemize}
\end{itemize} & Three-term system:

\begin{itemize}
\item proximal \textitbf{sini} ‘\textsc{l.prox}’\item medial \textitbf{situ} ‘\textsc{l.med}’

\begin{itemize}
\item distal \textitbf{sana} ‘\textsc{l.dist}’\end{itemize}
\\
%%\end{itemize}
Distributional properties & 
%%\begin{itemize}\item  
adnominal uses\item pronominal uses

\begin{itemize}
\item adverbial uses\end{itemize}
\end{itemize} & 
%%\begin{itemize}\item  
adnominal uses

\begin{itemize}
\item pronominal uses\end{itemize}
\\
%%\end{itemize}
Pronominal uses & 
%%\begin{itemize}\item  
in unembedded NPs\item in PPs

\begin{itemize}
\item in adnominal possessive constructions\end{itemize}
\end{itemize} & 
%%\begin{itemize}\item  
in PPs\\
%%\end{itemize}
Adnominal uses & 
%%\begin{itemize}\item  
can be stacked\end{itemize} & 
%%\begin{itemize}\item  
unattested\\
%%\end{itemize}
\lspbottomrule
\end{tabular}

The main distinctions between the demonstratives and the locatives in terms of their various functions are summarized in Table  ‎7 .11.


\begin{stylecaption}
\label{bkm:Ref322446307}Table ‎7.\stepcounter{Table}{\theTable}:  Functions of the demonstratives (\textsc{dem}) and locatives (\textsc{loc})
\end{stylecaption}

\tablehead{
 Domains of use & \textsc{dem} & \arraybslash \textsc{loc}\\
}
\begin{tabular}{lll}
\lsptoprule
Spatial & provide spatial orientation by drawing the hearer’s at\-tention to specific entities in the discourse or surround\-ing situation & provide spatial orientation by designating the location of an entity and focusing the hearer’s attention to its specific location\\
Figurative locational & {}-{}-{}- & signal a figurative locational endpoint\\
Temporal & indicate the temporal setting of an event/situation & indicate the temporal setting of an event/situation (medial locative only)\\
Psychological & indicate the speaker’s emo\-tional involvement with an event/situation

signal vividness

indicate contrast & indicate the speakers’ emo\-tional involvement with an event/situation

signal vividness\\
Identificational & aid in the identification of referents (long forms) & \arraybslash {}-{}-{}-\\
Textual anaphoric & keep track of discourse participants & keep track of the location of an entity\\
Textual discourse deictic & establish an overt link be\-tween two propositions & \arraybslash {}-{}-{}-\\
Placeholder & substitute for specific lexi\-cal items in the context of word-formulation trouble (long forms) & \arraybslash {}-{}-{}-\\
\lspbottomrule
\end{tabular}

In summary, with respect to their syntactic properties, the demonstratives have a wider range of uses (adnominal, pronominal, and adverbial uses) than the locatives. Likewise, in terms of their functions, the demonstratives have a wider range of uses than the locatives. The locative system, by contrast, allows finer semantic distinctions to be made than the demonstrative system, given that the former expresses a three-way deictic contrast, whereas the latter expresses a two-way deictic contrast.
\end{styleBodyaftervbefore}

%\setcounter{page}{1}\chapter[Noun phrases]{Noun phrases}
\label{bkm:Ref361998469}\section{Introduction}
\label{bkm:Ref439954707}
This chapter describes the Papuan Malay noun phrase with its different types of structures. Also included is a description of noun phrase apposition; noun phrase coordination is not discussed here but in Chapter 14.



An overview of the possible constituents of the Papuan Malay noun phrase is given in Table  ‎8 .1 (the parenthesis in the table header signal that the modifiers are optional). Modifying elements listed in the same column represent choices; constituents in the same row do not necessarily co-occur.
\end{styleBodyvvafter}

\begin{stylecaption}
\label{bkm:Ref294348610}Table ‎8.\stepcounter{Table}{\theTable}:  Possible constituents of the Papuan Malay noun phrase
\end{stylecaption}

\tablehead{
 (\textsc{mod}) & \textsc{head} & \multicolumn{4}{l}{ (\textsc{mod})}\\
&  & Post-1 & Post-2 & Post-3 & \arraybslash Post-4\\
}
\begin{tabular}{llllll}
\lsptoprule
\textsc{num} & \textsc{n} & \textsc{v} & \textsc{pro} & \textsc{dem} & \textsc{dem}\\
\textsc{qt} &  & \textsc{n} &  & \textsc{loc} & \\
\textsc{possr-np} &  & \textsc{pp} &  & \textsc{int} & \\
&  & \textsc{rc} &  & \textsc{num} & \\
&  &  &  & \textsc{qt} & \\
& \textsc{pro} & \textsc{pp} &  & \textsc{loc} & \textsc{dem}\\
&  & \textsc{rc} &  & \textsc{num} & \\
&  &  &  & \textsc{qt} & \\
\textsc{possr-np} & \textsc{dem} & \textsc{rc} &  &  & \textsc{dem}\\
& \textsc{loc} & \textsc{rc} &  &  & \textsc{dem}\\
\textsc{possr-np} & \textsc{int} & \textsc{rc} &  &  & \textsc{dem}\\
\lspbottomrule
\end{tabular}

In the following, examples are presented for the different types of constituents that can function as the head of a noun phrase. In giving examples, brackets are used to indicate the constituent structure within the noun phrase, where deemed necessary. In (0) the head is a noun, in (0) it is a personal pronoun, in (0) a demonstrative, in (0) a locative, and in (0) an interrogative. Head nouns allow the widest range of modifiers, while personal pronouns, demonstrative, locatives, and interrogatives allow only a subset of modifiers, as shown throughout this chapter.


\begin{styleExampleTitle}
Types of constituents functioning as heads in noun phrases
\end{styleExampleTitle}

\begin{tabular}{llllll}
\lsptoprule
\label{bkm:Ref361842519}
\gll {kitong} {cari} {\bluebold{ana}} {kecil} {itu}\\ %
& \textsc{1pl} & search & child & be.small & \textsc{d.dist}\\
\lspbottomrule
\end{tabular}
\ea
\glt 
‘we were looking for that small \bluebold{kid}’ \textstyleExampleSource{[080921-004a-CvNP.0070]}
\z

\begin{tabular}{lllll}
\lsptoprule
\label{bkm:Ref361842520}
\gll {\bluebold{dong}} {dua} {tu} {ikut}\\ %
& \textsc{3pl} & two & \textsc{d.dist} & follow\\
\lspbottomrule
\end{tabular}
\ea
\glt 
[About an upcoming event:] ‘both of \bluebold{them (}\blueboldSmallCaps{emph}\bluebold{)} are going to participate’ \textstyleExampleSource{[081115-001a-Cv.0115]}
\z

\begin{tabular}{lllllllllll}
\lsptoprule
\label{bkm:Ref361842521}
\gll {\bluebold{itu}} {tu} {rahasia} {mo} {mo} {biking} {apa} {ka,} {mo} {…}\\ %
& \textsc{d.dist} & \textsc{d.dist} & secret & want & want & make & what & or & want & \\
\lspbottomrule
\end{tabular}
\ea
\glt 
[About raising children:] ‘\bluebold{that} (\textsc{emph}) is the secret (when we) want want to do something or want to …’ \textstyleExampleSource{[080917-010-CvEx.0160]}
\z

\begin{tabular}{lllllll}
\lsptoprule
\label{bkm:Ref361842522}
\gll {e,} {sa} {tinggal} {di} {\bluebold{situ}} {tu}\\ %
& uh & \textsc{1sg} & stay & at & \textsc{l.med} & \textsc{d.dist}\\
\lspbottomrule
\end{tabular}
\ea
\glt 
‘uh, I lived \bluebold{there (}\blueboldSmallCaps{emph}\bluebold{)}’ \textstyleExampleSource{[080922-002-Cv.0112]}
\z

\begin{tabular}{lllllllll}
\lsptoprule
\label{bkm:Ref361842523}
\gll {ana} {laki{\Tilde}laki} {ini} {de} {mo} {ke} {\bluebold{mana}} {ni}\\ %
& child & \textsc{rdp}{\Tilde}husband & \textsc{d.prox} & \textsc{3sg} & want & to & where & \textsc{d.prox}\\
\lspbottomrule
\end{tabular}
\ea
\glt 
‘this boy, \bluebold{where} (\textsc{emph}) does he want to (go)?’ \textstyleExampleSource{[080922-004-Cv.0017]}
\z


The minimal noun phrase consists of a bare head nominal. Modifiers are optional and occur in pre- and/or post-head position. Attested in the corpus is the co-occurrence of up to three post-head constituents. Modifiers listed in the same pre- or post-head slots in Table  ‎8 .1 are unattested.\footnote{\\
\\
\\
\\
\\
\\
\\
\\
\\
\\
\\
\\
\\
\\

‘the duties of all these children are already taken care of’ [Elicited ME151120.002]\par \\
} There is one exception, however, namely the quantifier \textitbf{brapa} ‘several’, as shown in in (0).



Pre-head modifiers can be numerals such as \textitbf{empat} ‘four’ in (0), quantifiers such as \textitbf{smua} ‘all’ in (0), or possessor noun phrases in adnominal possessive constructions such \textitbf{orang{\Tilde}orang besar} ‘big people’ in (0). The co-occurrence of pre-head modifiers is unattested, with one exception. The mid-range quantifier \textitbf{brapa} ‘several’ co-occurs with certain numerals, such as \textitbf{ratus} ‘hundred’, or \textitbf{ribu} ‘thousand’, as in \textitbf{brapa ratus orang} ‘several hundred people’ in (0). 
\end{styleBodyvvafter}

\begin{styleExampleTitle}
\textsc{mod} – \textsc{head}
\end{styleExampleTitle}

\begin{tabular}{lllll}
\lsptoprule
\label{bkm:Ref340513719}
\gll {jadi} {saya} {\bluebold{empat}} {\bluebold{ana}}\\ %
& so & \textsc{1sg} & four & child\\
\lspbottomrule
\end{tabular}
\ea
\glt 
\textsc{num} – \textsc{head}: ‘so, I (have) \bluebold{four children}’ \textstyleExampleSource{[081006-024-CvEx.0002]}
\z

\begin{tabular}{lllll}
\lsptoprule
\label{bkm:Ref340513721}
\gll {\bluebold{smua}} {\bluebold{buku}} {bisa} {basa}\\ %
& all & book & be.able & be.wet\\
\lspbottomrule
\end{tabular}
\ea
\glt 
\textsc{qt} – \textsc{head}: ‘\bluebold{all books} could get wet’ \textstyleExampleSource{[080917-008-NP.0189]}
\z

\begin{tabular}{llllllll}
\lsptoprule
\label{bkm:Ref361905570}
\gll {…} {bukang} {\bluebold{orang{\Tilde}orang}} {\bluebold{besar}} {\bluebold{pu}} {\bluebold{ana}} {…}\\ %
&  & \textsc{neg} & \textsc{[rdp}{\Tilde}person & big] & \textsc{poss} & child & \\
\lspbottomrule
\end{tabular}
\ea
\glt 
‘(she’s a child of farmers) not the \bluebold{child of big people} …’ \textstyleExampleSource{[081110-005-Pr.0094]}
\z

\begin{tabular}{llllllll}
\lsptoprule
\label{bkm:Ref361906278}
\gll {…} {tentara} {itu} {ada} {\bluebold{brapa}} {\bluebold{ratus}} {\bluebold{orang}}\\ %
&  & soldier & \textsc{d.dist} & exist & several & hundred & person\\
\lspbottomrule
\end{tabular}
\ea
\glt 
\textsc{qt} – \textsc{num} – \textsc{head}: ‘[one time, I brought the military (into the forest),] those soldiers were \bluebold{several hundred people}’ \textstyleExampleSource{[081029-005-Cv.0131]}
\z


The post-head modifier slots attract a wider range of constituents: verbs, nouns, prepositional phrases, and relative clauses occur in slot Post-1, personal pronouns in slot Post-2, and demonstratives, locatives, interrogatives, numerals, and quantifiers in slot Post-3. In addition, the demonstratives also occur in slot Post-4. The modifiers occurring in slot Post-1 have attributive function, while those in slot Post-2 to Post-4 have determining function.



Co-occurrences of modifiers listed in the same slot are unattested, whereas those listed in different slots are attested to co-occur, as demonstrated in (0) to (0). In \textitbf{tangang pendek satu tu} ‘that one short-handed (one)’ in (0), an adnominally used stative verb co-occurs with a numeral and a demonstrative. In \textitbf{babi puti ko} ‘you white pig’ in (0), an adnominally used verb co-occurs with a personal pronoun. In \textitbf{pisang Sorong sana tu} ‘those bananas (from) Sorong over there’ in (0), an adnominally used noun co-occurs with a locative and a demonstrative. In \textitbf{pace dorang dua ini} ‘the two men here’ in (0), an adnominally used personal pronoun co-occurs with a numeral and a demonstrative. In \textitbf{kaka dari Mambramo satu} ‘a certain older brother from (the) Mambramo (area)’ in (0), an adnominally used prepositional phrase co-occurs with a numeral. In \textitbf{dong di Papua tu} ‘they in Papua there’ in (0), an adnominally used prepositional phrase co-occurs with a demonstrative. Finally, in \textitbf{kata itu tu} ‘those very words’ in (0) two adnominally used demonstratives co-occur.
\end{styleBodyvvafter}

\begin{styleExampleTitle}
\textsc{head} – \textsc{mod}
\end{styleExampleTitle}

\begin{tabular}{llllll}
\lsptoprule
\label{bkm:Ref340513723}
\gll {\bluebold{tangang}} {\bluebold{pendek}} {\bluebold{satu}} {\bluebold{tu}} {((laughter))}\\ %
& hand & be.short & one & \textsc{d.dist} & \\
\lspbottomrule
\end{tabular}
\ea
\glt 
\textsc{head} – \textsc{v} – \textsc{num} – \textsc{dem}: [About an acquaintance:] ‘\bluebold{that one short-armed (one)} ((laughter))’ \textstyleExampleSource{[081006-016-Cv.0036]}
\z

\begin{tabular}{lllllll}
\lsptoprule
\label{bkm:Ref340513724}
\gll {\bluebold{babi}} {\bluebold{puti}} {\bluebold{ko}} {dari} {atas} {turung}\\ %
& pig & be.white & \textsc{2sg} & from & top & descend\\
\lspbottomrule
\end{tabular}
\ea
\glt 
\textsc{head} – \textsc{v} – \textsc{pro}: [About an acquaintance:] ‘\bluebold{you white pig} came down from up (there)’ \textstyleExampleSource{[081025-006-Cv.0260]}
\z

\begin{tabular}{llllllll}
\lsptoprule
\label{bkm:Ref340513722}
\gll {\bluebold{pisang}} {\bluebold{Sorong}} {\bluebold{sana}} {\bluebold{tu},} {iii,} {besar{\Tilde}besar} {manis}\\ %
& banana & Sorong & \textsc{l.dist} & \textsc{d.prox} & oh! & \textsc{rdp}{\Tilde}be.big & be.sweet\\
\lspbottomrule
\end{tabular}
\ea
\glt 
\textsc{head} – \textsc{n} – \textsc{loc} – \textsc{dem}: ‘\bluebold{those bananas (from) Sorong over there}, oooh, (they) are all big (and) sweet’ \textstyleExampleSource{[081011-003-Cv.0017]}
\z

\begin{tabular}{lllllll}
\lsptoprule
\label{bkm:Ref340513725}
\gll {\bluebold{pace}} {\bluebold{dorang}} {\bluebold{dua}} {\bluebold{ini}} {ke} {atas}\\ %
& man & \textsc{3pl} & two & \textsc{d.prox} & to & top\\
\lspbottomrule
\end{tabular}
\ea
\glt 
\textsc{head} – \textsc{pro} – \textsc{num} – \textsc{dem}: ‘\bluebold{both of the two men here} (went) up (there)’ \textstyleExampleSource{[081006-034-CvEx.0010]}
\z

\begin{tabular}{lllllll}
\lsptoprule
\label{bkm:Ref439577754}\label{bkm:Ref340513726}
\gll {trus} {tamba} {\bluebold{kaka}} {\bluebold{dari}} {\bluebold{Mambramo}} {\bluebold{satu}}\\ %
& next & add & oSb & from & Mambramo & one\\
\lspbottomrule
\end{tabular}
\ea
\glt 
\textsc{head} – \textsc{pp} – \textsc{num}: [About forming a sports team:] ‘then add \bluebold{a certain older brother from (the) Mambramo (area)}’ \textstyleExampleSource{[081023-001-Cv.0002]}
\z

\begin{tabular}{llllllll}
\lsptoprule
\label{bkm:Ref340513727}
\gll {\bluebold{dong}} {\bluebold{di}} {\bluebold{Papua}} {\bluebold{tu}} {dong} {makang} {papeda}\\ %
& \textsc{3pl} & at & Papua & \textsc{d.dist} & \textsc{3pl} & eat & sagu.porridge\\
\lspbottomrule
\end{tabular}
\ea
\glt 
\textsc{head} – \textsc{pp} – \textsc{dem}: ‘\bluebold{they in Papua there}, they eat sagu porridge’ \textstyleExampleSource{[081109-009-JR.0001]}
\z

\begin{tabular}{llllllll}
\lsptoprule
\label{bkm:Ref340513728}
\gll {\bluebold{kata}} {\bluebold{itu}} {\bluebold{tu}} {yang} {biking} {sa} {bertahang}\\ %
& word & \textsc{d.dist} & \textsc{d.dist} & \textsc{rel} & make & \textsc{1sg} & hold.(out/back)\\
\lspbottomrule
\end{tabular}
\ea
\glt 
\textsc{head} – \textsc{dem} – \textsc{dem} ‘(it was) \bluebold{those very words} that made me hold out’ \textstyleExampleSource{[081115-001a-Cv.0235]}
\z


This brief overview shows that Papuan Malay employs two distinct types of noun phrase structures: (1) a head – modifier or ‘\textsc{n-mod}’ structure, and (2) a modifier – head or ‘\textsc{mod-n}’ structure. The particular structure of a noun phrase depends on the syntactic properties of its adnominal constituents:


\begin{itemize}
\item \begin{styleIIndented}
\textsc{n-mod} structure with adnominally used verbs, nouns, personal pronouns, demonstratives, locatives, interrogatives, prepositional phrases, and relative clauses.
\end{styleIIndented}\item \begin{styleIIndented}
\textsc{n-mod} or \textsc{mod-n} structure with adnominally used numerals and quantifiers (the constituent order depends on the semantics of the phrasal structure).
\end{styleIIndented}\end{itemize}
\begin{itemize}
\item \begin{styleIvI}
\textsc{mod-n} structure in adnominal possessive constructions.
\end{styleIvI}\end{itemize}

Noun phrases with an \textsc{n-mod} structure are examined in §8.2 and those with an \textsc{n-mod} or \textsc{mod-n} structure in §8.3. Adnominal possessive constructions with a \textsc{mod-n} structure are briefly mentioned in §8.4, and fully discussed in Chapter 9. In addition, apposition is discussed in §8.5. The main points of this chapter are summarized in §8.6.
\end{styleBodyxvafter}

\section{\textsc{n-mod} structure}
\label{bkm:Ref290140347}
In noun phrases with an \textsc{n-mod} structure, the head occurs in initial position followed by the modifying elements. The following modifiers are discussed: verbs (§8.2.1), nouns (§8.2.2), personal pronouns (§8.2.3), demonstratives (§8.2.4), locatives (§8.2.5), interrogatives (§8.2.6), prepositional phrases (§8.2.7), and relative clauses (§8.2.8).
\end{styleBodyxvafter}

\subsection}]{Verbs [\textsc{n} \textsc{v}]}
\label{bkm:Ref288639680}
Adnominally used verbs always follow their head nominals such that ‘\textsc{n} \textsc{v}’, as shown in (0) to (0). Most often, the adnominal modifier is a stative verb, as in (0) to (0), although noun phrases with adnominally used dynamic verbs also occur, as in (0) to (0). (The distributional preferences of attributively used stative and dynamic verbs are discussed in §5.3.2.)



In noun phrases with adnominally used stative verbs, as in (0) to (0), the head nominal is typically a bare noun as in (0), or a reduplicated noun as in (0). The adnominal modifier is usually a bare stative verb, such as \textitbf{besar} ‘be big’ in (0) or \textitbf{panjang} ‘be long’ in (0). However, the modifier can also be a multi-word phrase with an overt coordinator as in \textitbf{puti dengang hitam} ‘white and black’ in (0), or with juxtaposed constituents as in the elicited near contrastive example in (0). Overall, though, multi-word modifier phrases are rare and limited to phrases with two adnominally used verbs.
\end{styleBodyvvafter}

\begin{styleExampleTitle}
Noun phrases with adnominal stative verbs
\end{styleExampleTitle}

\begin{tabular}{lllllllll}
\lsptoprule
\label{bkm:Ref340774340}
\gll {sa} {su} {liat} {ada} {\bluebold{pohong}} {\bluebold{besar}} {di} {depang}\\ %
& \textsc{1sg} & already & see & exist & tree & be.big & at & front\\
\lspbottomrule
\end{tabular}
\ea
\glt 
‘I already saw there was a \bluebold{big tree} in front’ \textstyleExampleSource{[081025-008-Cv.0019]}
\z

\begin{tabular}{lllll}
\lsptoprule
\label{bkm:Ref340774343}
\gll {langsung} {\bluebold{kuku{\Tilde}kuku}} {\bluebold{panjang}} {kluar}\\ %
& immediately & \textsc{rdp}{\Tilde}digit.nail & be.long & go.out\\
\lspbottomrule
\end{tabular}
\ea
\glt 
‘immediately (his) \bluebold{long claws} came out’ \textstyleExampleSource{[081115-001a-Cv.0077]}
\z

\begin{tabular}{llllllllllll}
\lsptoprule
\label{bkm:Ref340774345}
\gll {sa} {\multicolumn{2}{l}{pu}} {\multicolumn{2}{l}{bapa}} {kubur} {\multicolumn{2}{l}{sa}} {pu} {tete} {pu}\\ %
& \textsc{1sg} & \multicolumn{2}{l}{\textsc{poss}} & \multicolumn{2}{l}{father} & bury & \multicolumn{2}{l}{\textsc{1sg}} & \textsc{poss} & grandfather & \textsc{poss}\\
& \multicolumn{2}{l}{[[\bluebold{kaing}]} & \multicolumn{2}{l}{[\bluebold{puti}} & \multicolumn{3}{l}{\bluebold{dengang}} & \multicolumn{4}{l}{\bluebold{hitam}]]}\\
& \multicolumn{2}{l}{cloth} & \multicolumn{2}{l}{be.white} & \multicolumn{3}{l}{with} & \multicolumn{4}{l}{be.black}\\
\lspbottomrule
\end{tabular}
\ea
\glt 
‘my father buried my grandfather’s \bluebold{white and black cloth}’ \textstyleExampleSource{[081014-014-NP.0047]}
\z

\begin{tabular}{lllllm{4.5984238E-4in}lllll}
\lsptoprule
\label{bkm:Ref340774346}
\gll {sa} {\multicolumn{2}{l}{pu}} {\multicolumn{2}{l}{bapa}} {kubur} {sa} {pu} {tete} {pu}\\ %
& \textsc{1sg} & \multicolumn{2}{l}{\textsc{poss}} & \multicolumn{2}{l}{father} & bury & \textsc{1sg} & \textsc{poss} & grandfather & \textsc{poss}\\
& \multicolumn{2}{l}{[[\bluebold{kaing}]} & \multicolumn{2}{l}{[\bluebold{hitam}} & \multicolumn{6}{l}{\bluebold{puti}]]}\\
& \multicolumn{2}{l}{cloth} & \multicolumn{2}{l}{be.black} & \multicolumn{6}{l}{be.white}\\
\lspbottomrule
\end{tabular}
\ea
\glt 
‘my father buried my grandfather’s \bluebold{white (and) black cloth}’ \textstyleExampleSource{[Elicited BR130221.036]}\footnote{\\
\\
\\
\\
\\
\\
\\
\\
\\
\\
\\
\\
\\
\\
\par According to one consultant, Papuan Malay speakers prefer \textitbf{hitam puti} ‘black (and) white over \textitbf{puti hitam} ‘white (and) black’, although both constructions are acceptable.\\
}
\z


Adnominally used dynamic verbs denote activities, associated with the head nominal, as in (0) to (0). The head nominal can denote an agent who carries out the activity encoded by the verb, as with monovalent \textitbf{jalang} ‘walk’ in (0), or a patient who undergoes this activity, as with bivalent \textitbf{bakar} ‘burn’ in (0). The head can also express a spatial or temporal location where the activity occurs as with monovalent \textitbf{mandi} ‘bathe’ in (0) and \textitbf{bangung} ‘get up’ in (0), respectively.


\begin{styleExampleTitle}
Noun phrases with adnominal dynamic verbs
\end{styleExampleTitle}

\begin{tabular}{lllll}
\lsptoprule
\label{bkm:Ref340862464}
\gll {ana} {itu} {\bluebold{tukang}} {\bluebold{jalang}}\\ %
& child & \textsc{d.dist} & craftsman & walk\\
\lspbottomrule
\end{tabular}
\ea
\glt 
‘that kid \bluebold{doesn’t like staying at home}’ (Lit. ‘\bluebold{specialist (in) walk(ing)}’) \textstyleExampleSource{[080927-001-Cv.0007]}
\z

\begin{tabular}{lllllllll}
\lsptoprule
\label{bkm:Ref340862465}
\gll {pi} {ambil} {\bluebold{kayu}} {\bluebold{bakar},} {\bluebold{kayu}} {\bluebold{bakar}} {buat} {Natal}\\ %
& go & fetch & wood & burn & wood & burn & for & Christmas\\
\lspbottomrule
\end{tabular}
\ea
\glt 
‘(we) went to get \bluebold{firewood}, \bluebold{firewood} for Christmas’ (Lit. ‘\bluebold{wood to burn}’) \textstyleExampleSource{[081006-017-Cv.0014]}
\z

\begin{tabular}{lllll}
\lsptoprule
\label{bkm:Ref340862466}
\gll {tra} {ada} {\bluebold{kamar}} {\bluebold{mandi}}\\ %
& \textsc{neg} & exist & room & bathe\\
\lspbottomrule
\end{tabular}
\ea
\glt 
‘there weren’t (any) \bluebold{bathrooms}’ (Lit. ‘\bluebold{room (where) to bathe}’) \textstyleExampleSource{[081025-009a-Cv.0059]}
\z

\begin{tabular}{llllllll}
\lsptoprule
\label{bkm:Ref340862467}
\gll {sa} {pu} {\bluebold{jam{\Tilde}jam}} {\bluebold{bangung}} {bukang} {jam} {empat}\\ %
& \textsc{1sg} & \textsc{poss} & \textsc{rdp}{\Tilde}hour & get.up & \textsc{neg} & hour & four\\
\lspbottomrule
\end{tabular}
\ea
\glt 
‘my \bluebold{time to get up} is not four o’clock’ (Lit ‘\bluebold{hours (when) to wake-up}’) \textstyleExampleSource{[081025-006-Cv.0061]}
\z


Noun phrases with adnominally used verbs can further be modified with numerals. In the corpus, the adnominally used numeral is always the numeral \textitbf{satu} ‘one’, as in (0) and (0) (the non-enumerating function of \textitbf{satu} ‘one’ as a marker of ‘specific indefiniteness’, as in (0), is discussed in §5.9.4).


\begin{styleExampleTitle}
Noun phrases with adnominal verbs and numerals
\end{styleExampleTitle}

\begin{tabular}{llllll}
\lsptoprule
\label{bkm:Ref340774350}
\gll {[[[\bluebold{tangang}} {\bluebold{pendek}]} {\bluebold{satu}]} {\bluebold{tu}]} {((laughter))}\\ %
& hand & be.short & one & \textsc{d.dist} & \\
\lspbottomrule
\end{tabular}
\ea
\glt 
[About an acquaintance:] ‘\bluebold{that one short-handed (one)} ((laughter))’ \textstyleExampleSource{[081006-016-Cv.0036]}
\z

\begin{tabular}{lllllllll}
\lsptoprule
\label{bkm:Ref340774351}
\gll {[[\bluebold{kampung}} {\bluebold{tua}]} {\bluebold{satu}]} {yang} {perna} {om} {Wili} {…}\\ %
& village & be.old & one & \textsc{rel} & once & uncle & Wili & \\
\lspbottomrule
\end{tabular}
\ea
\glt
‘\bluebold{a certain old village} where uncle Wili once …’ \textstyleExampleSource{[080922-010a-CvNF.0290]}
\end{styleFreeTranslEngxvpt}

\subsection}]{Nouns [\textsc{n} \textsc{n}]}
\label{bkm:Ref293498535}\label{bkm:Ref387857912}\label{bkm:Ref374450598}\label{bkm:Ref374450559}\label{bkm:Ref374448591}\label{bkm:Ref374434079}
In noun phrases with adnominally used nouns, a post-head noun \textsc{n2} modifies the head nominal \textsc{n1}, such that ‘\textsc{n1n2}’. Such constructions are characterized by the semantic subordination of the \textsc{n2} modifier under the head nominal \textsc{n1}.



In Papuan Malay, the distinction between a noun phrase with an adnominally used noun, hereafter \textsc{n1n2-np}, and a compound with juxtaposed nominal constituents is not clear-cut, however. Word combinations or collocations range from two word expressions with compositional transparent semantics such as \textitbf{air sagu} ‘liquid of the sago palm tree’, to less compositional two-word expressions, such as \textitbf{kampung-tana} ‘home village’ (literally ‘village-ground’). This section focuses on \textsc{n1n2-np}s; the demarcation of such phrasal expression from compounds, and compounding in general, are discussed in §3.2.1.
\end{styleBodyvafter}


\textsc{n1n2-np}s denote important features for subclassification of the superordinate head nominal. Typically, the head of an \textsc{n1n2-np} is a noun, as shown in (0) to (0). Less often, the head is a deverbal constituent as in (0) and (0). Semantically, \textsc{n1n2-np}s denote a wide range of associative relations between the \textsc{n1} and the \textsc{n2}, as shown in (0) to (0): part-whole, property-of, affiliated-with, name-of, subtype-of, composed-of, and purpose-for relations, as well as locational, temporal, and event relations. \textsc{n1n2-np}s encode inalienable and alienable concepts.
\end{styleBodyvafter}


Inalienable ‘part-whole’ relations of body parts and plants are given in (0) and (0), respectively, while (0) illustrates an alienable ‘part-whole’ relation. (More types of ‘part-whole’ relations are found in Table  ‎8 .2.)
\end{styleBodyvvafter}

\begin{styleExampleTitle}
‘Part-whole’ relations
\end{styleExampleTitle}

\begin{tabular}{lllllll}
\lsptoprule
\label{bkm:Ref340774352}\label{bkm:Ref289766948}
\gll {sa} {bilang,} {\bluebold{tulang}} {\bluebold{bahu}} {yang} {pata}\\ %
& \textsc{1sg} & say & bone & shoulder & \textsc{rel} & break\\
\lspbottomrule
\end{tabular}
\ea
\glt 
‘I said, ‘(it’s my) \bluebold{shoulder bone} that is broken’’ \textstyleExampleSource{[081015-005-NP.0048]}
\z

\begin{tabular}{lllllll}
\lsptoprule
\label{bkm:Ref340774353}
\gll {adu} {sa} {pu} {\bluebold{daung}} {\bluebold{bawang}} {itu}\\ %
& oh.no! & \textsc{1sg} & \textsc{poss} & leaf & onion & \textsc{d.dist}\\
\lspbottomrule
\end{tabular}
\ea
\glt 
[After someone had plucked some onion leaves:] ‘oh no, my \bluebold{onion leaves} there!’ \textstyleExampleSource{[081006-024-CvEx.0043]}
\z

\begin{tabular}{lllllllll}
\lsptoprule
\label{bkm:Ref340774354}
\gll {…} {pukul} {…} {dengang} {\bluebold{blakang}} {\bluebold{kapak}} {juga} {bisa}\\ %
&  & hit &  & with & backside & axe & also & be.able\\
\lspbottomrule
\end{tabular}
\ea
\glt 
[About killing dogs] ‘[(it’s) also possible to bow shoot him,] to beat (him to death) … with the \bluebold{backside of an axe} is also possible’ \textstyleExampleSource{[081106-001-CvPr.0002]}
\z


\textsc{n1n2-np}s expressing ‘property-of’ and ‘affiliated-with’ relations are given in (0) and (0), respectively.


\begin{styleExampleTitle}
‘Property-of” and ‘affiliated-with’ relations
\end{styleExampleTitle}

\begin{tabular}{lllllll}
\lsptoprule
\label{bkm:Ref340774355}
\gll {dari} {situ} {kembali} {ambil} {\bluebold{seng}} {\bluebold{greja}}\\ %
& from & \textsc{l.med} & return & fetch & corrugated.iron & church\\
\lspbottomrule
\end{tabular}
\ea
\glt 
‘from there (I) returned (and) took \bluebold{the corrugated iron (sheets) of the church}’ \textstyleExampleSource{[080927-004-CvNP.0005]}
\z

\begin{tabular}{llllllllll}
\lsptoprule
\label{bkm:Ref340774356}
\gll {…} {sa} {su} {bakar} {ruma} {itu,} {\bluebold{ruma}} {\bluebold{setang}} {itu}\\ %
&  & \textsc{1sg} & already & burn & house & \textsc{d.dist} & house & evil.spirit & \textsc{d.dist}\\
\lspbottomrule
\end{tabular}
\ea
\glt 
‘[(if) I, umh, for example, were in Aruswar or Niwerawar,] I would already have burnt that house, that \bluebold{evil spirit’s house}’ \textstyleExampleSource{[081025-009a-Cv.0198]}
\z


‘Name-of’ relations are presented in (0) and (0). (Other types of ‘name-of’ relations are found in Table  ‎8 .2.)


\begin{styleExampleTitle}
‘Name-of’ relations
\end{styleExampleTitle}

\begin{tabular}{lllllll}
\lsptoprule
\label{bkm:Ref340774359}
\gll {yo} {bapa,} {\bluebold{hari}} {\bluebold{minggu}} {sa} {datang}\\ %
& yes & father & day & Sunday & \textsc{1sg} & come\\
\lspbottomrule
\end{tabular}
\ea
\glt 
‘yes father, on \bluebold{Sunday} I’ll come’ (Lit. ‘\bluebold{Sunday day}’) \textstyleExampleSource{[080922-001a-CvPh.0344]}
\z

\begin{tabular}{lllllllll}
\lsptoprule
\label{bkm:Ref340774360}
\gll {knapa} {ko} {gambar} {monyet} {di} {bawa} {\bluebold{pohong}} {\bluebold{pisang}}\\ %
& why & \textsc{2sg} & draw & monkey & at & under & tree & banana\\
\lspbottomrule
\end{tabular}
\ea
\glt 
‘why did you draw the monkey under \bluebold{the banana tree}?’ \textstyleExampleSource{[081109-002-JR.0004]}
\z


‘Subtype-of’ relations are presented in (0) to (0).


\begin{styleExampleTitle}
‘Subtype-of’ relations
\end{styleExampleTitle}

\begin{tabular}{lllllll}
\lsptoprule
\label{bkm:Ref340774361}
\gll {…} {maka} {pake} {[[\bluebold{bahasa}]} {\bluebold{orang}} {\bluebold{bisu}]}\\ %
&  & therefore & use & language & person & be.mute\\
\lspbottomrule
\end{tabular}
\ea
\glt 
‘[she couldn’t speak the Indonesian language,] therefore (she) used \bluebold{sign language}’ (Lit. ‘\bluebold{language of mute people}’) \textstyleExampleSource{[081006-023-CvEx.0073]}
\z

\begin{tabular}{llllllll}
\lsptoprule
\label{bkm:Ref439580439}
\gll {…} {supaya} {Sarmi} {ada} {[[\bluebold{petinju}} {\bluebold{prempuang}]} {satu]}\\ %
&  & so.that & Sarmi & exist & boxer & woman & one\\
\lspbottomrule
\end{tabular}
\ea
\glt 
‘… so that Sarmi has a certain \bluebold{woman boxer}’ \textstyleExampleSource{[081023-003-Cv.0005]}
\z


\textsc{n1n2-np}s expressing ‘composed-of’ and ‘purpose-for’ relations are illustrated in (0) and (0), respectively.


\begin{styleExampleTitle}
‘Composed-of’ and ‘purpose-for’ relations
\end{styleExampleTitle}

\begin{tabular}{llllll}
\lsptoprule
\label{bkm:Ref340774364}
\gll {smua} {jalang} {kaya} {\bluebold{kapal}} {\bluebold{kayu}}\\ %
& all & walk & like & ship & wood\\
\lspbottomrule
\end{tabular}
\ea
\glt 
‘(they) all were strolling around like \bluebold{wooden boats}’ \textstyleExampleSource{[081025-009a-Cv.0188]}
\z

\begin{tabular}{lllllll}
\lsptoprule
\label{bkm:Ref340774366}
\gll {yo,} {\bluebold{net}} {\bluebold{laki{\Tilde}laki}} {tong} {yang} {bli}\\ %
& yes & (sport.)net & \textsc{rdp}{\Tilde}husband & \textsc{1pl} & \textsc{rel} & buy\\
\lspbottomrule
\end{tabular}
\ea
\glt 
‘yes, the (volley-ball) \bluebold{net for men}, (it was) us who (bought it)’ \textstyleExampleSource{[081023-001-Cv.0012]}
\z


Locational and temporal relations between the \textsc{n1} and \textsc{n2} are illustrated in (0) to (0). The \textsc{n2} denotes a locational relation in (0), and a temporal relation in (0). In (0) the first two nominals express a locational relation between the head \textitbf{ampas} ‘waste’ and its modifier noun, the source \textitbf{pinang} ‘betel nut’. This \textsc{n1n2} construction is modified with the third nominal \textitbf{malam} ‘night’ which denotes a temporal relation (\textsc{n1n2-np}s with more than three nominal constituents are unattested in the corpus). (Other types of locational relations are found in Table  ‎8 .2.)


\begin{styleExampleTitle}
Locational and temporal relations
\end{styleExampleTitle}

\begin{tabular}{llllll}
\lsptoprule
\label{bkm:Ref340774370}
\gll {orang} {Papua} {bilang} {\bluebold{jing}} {\bluebold{kayu}}\\ %
& person & Papua & say & genie & wood\\
\lspbottomrule
\end{tabular}
\ea
\glt 
‘Papuans call (them) \bluebold{tree genies}’ \textstyleExampleSource{[081006-022-CvEx.0054]}
\z

\begin{tabular}{llll}
\lsptoprule
\label{bkm:Ref340774371}
\gll {[[\bluebold{jam}} {\bluebold{tiga}]} {\bluebold{pagi}]?}\\ %
& hour & three & morning\\
\lspbottomrule
\end{tabular}
\ea
\glt 
‘(was it) \bluebold{three o’clock in the morning}?’ \textstyleExampleSource{[080918-001-CvNP.0042]}
\z

\begin{tabular}{lllllll}
\lsptoprule
\label{bkm:Ref340774373}
\gll {[[[\bluebold{ampas}} {\bluebold{pinang}]} {\bluebold{malam}]} {tu]} {sa} {taru}\\ %
& waste & betel.nut & night & \textsc{d.dist} & \textsc{1sg} & put\\
\lspbottomrule
\end{tabular}
\ea
\glt 
‘that \bluebold{evening’s betel nut waste}, I put (it aside)’ \textstyleExampleSource{[081025-006-Cv.0294]}
\z


An \textsc{n1n2-np} can also be formed with a deverbal nominal head as in (0) and (0). Semantically, the \textsc{n1n2-np} in (0) expresses an event relation in which adnominal \textitbf{tugu} ‘monument’ is affected by the event expressed by the deverbal head \textsc{n1}. The \textsc{n1n2-np} in (0) denotes a locational relation with the deverbal head \textsc{n1} originating from the nominal spatial source \textsc{n2}.


\begin{styleExampleTitle}
Subordinate \textsc{n1n2-np}s with deverbal constituent
\end{styleExampleTitle}

\begin{tabular}{lllll}
\lsptoprule
\label{bkm:Ref340774374}
\gll {ada} {[[\bluebold{pasang}} {\bluebold{tugu}]} {\bluebold{itu}]}\\ %
& exist & install & monument & \textsc{d.dist}\\
\lspbottomrule
\end{tabular}
\ea
\glt 
[Giving directions:] ‘there is \bluebold{that statue installation}’ \textstyleExampleSource{[080917-008-NP.0017]}
\z

\begin{tabular}{llllllll}
\lsptoprule
\label{bkm:Ref340774378}\label{bkm:Ref289940286}
\gll {kalo} {angkat} {air} {jemur} {di} {\bluebold{panas}} {\bluebold{mata-hari}}\\ %
& if & lift & water & dry & at & be.hot & sun\\
\lspbottomrule
\end{tabular}
\ea
\glt 
‘when (you) fetch water, warm (it) up in \bluebold{the heat of the sun}’ \textstyleExampleSource{[081006-013-Cv.0005]}
\z


Table  ‎8 .2 gives an overview of the different associative meaning relations expressed with \textsc{n1n2-np}s.


\begin{stylecaption}
\label{bkm:Ref360720115}Table ‎8.\stepcounter{Table}{\theTable}:  Associative meaning relations encoded by \textsc{n1n2-np}s
\end{stylecaption}

\tablehead{ & Papuan Malay \textsc{n1n2} & \multicolumn{2}{l}{ Glosses} & \arraybslash Free translation\\
}
\begin{tabular}{lllll}
\lsptoprule
\multicolumn{5}{l}{%\setcounter{itemize}{0}
\begin{itemize}
\item Part-whole relation – \textsc{n1} is a part of \textsc{n2}: (a) human body part, (b) nonhuman body part, (c) plant part, (d) spatial location of a concrete object, (e) temporal location of an abstract object, (f) time segment within a time period, (g) member of an institution\end{itemize}
}\\
(a) & \textitbf{urat kaki} & tendon foot & \multicolumn{2}{l}{‘foot tendon’}\\
(b) & \textitbf{duri ikang} & thorn fish & \multicolumn{2}{l}{‘fish bone’}\\
(c) & \textitbf{pelepa sagu} & stem sago & \multicolumn{2}{l}{‘sago stem’}\\
(d) & \textitbf{blakang kapak} & backside axe & \multicolumn{2}{l}{‘backside of an axe’}\\
(e) & \textitbf{tenga sembayang} & middle worship & \multicolumn{2}{l}{‘middle of the worship’}\\
(f) & \textitbf{malam hari} & night day & \multicolumn{2}{l}{‘evening (of the day)’}\\
(g) & \textitbf{petugas polisi} & official police & \multicolumn{2}{l}{‘police official’}\\
\multicolumn{5}{l}{\begin{itemize}
\item Property relation – \textsc{n1} is a property of \textsc{n2}\end{itemize}
}\\
& \textitbf{ruma orang} & house person & \multicolumn{2}{l}{‘(other) people’s house’}\\
& \textitbf{cara orang Papua} & way person Papua & \multicolumn{2}{l}{‘Papuan traditions’}\\
\multicolumn{5}{l}{\begin{itemize}
\item ‘Affiliated-with’ relation: \textsc{n1} is affiliated with \textsc{n2}\end{itemize}
}\\
& \textitbf{ruma setang} & house evil.spirit & \multicolumn{2}{l}{‘house of an evil spirit’}\\
& \textitbf{ana{\Tilde}ana iblis} & \textsc{rdp}{\Tilde}child devil & \multicolumn{2}{l}{‘children of the devil’}\\
\multicolumn{5}{l}{\begin{itemize}
\item Name-of relation – \textsc{n2} designates the name of \textsc{n1}: (a) animal, (b) plant, (c) personal name, (d) clan/ethnic group, (e) disease, (f) building/institution, (g) language, (h) religion, (i) spatial location, (j) temporal location\end{itemize}
}\\
(a) & \textitbf{ikang gurango} & fish shark & \multicolumn{2}{l}{‘shark fish’}\\
(b) & \textitbf{sayur bayam} & vegetable amaranth & \multicolumn{2}{l}{‘amaranth vegetable’}\\
(c) & \textitbf{nama Nofela} & name Nofela & \multicolumn{2}{l}{‘(of the) name Nofela}\\
(d) & \textitbf{marga Sope} & clan Sope & \multicolumn{2}{l}{‘Sope clan’}\\
(e) & \textitbf{penyakit malaria} & disease malaria & \multicolumn{2}{l}{‘malaria disease’}\\
(f) & \textitbf{greja Kema-Injil} & church Kema-Injil & \multicolumn{2}{l}{‘Kema-Injil church’}\\
(g) & \textitbf{bahasa Inggris} & language England & \multicolumn{2}{l}{‘English language’}\\
(h) & \textitbf{agama Kristen} & religion Christian & \multicolumn{2}{l}{‘Christian religion’}\\
(i) & \textitbf{kota Sarmi} & city Sarmi & \multicolumn{2}{l}{‘Sarmi city’}\\
(j) & \textitbf{hari kamis} & day Thursday & \multicolumn{2}{l}{‘Thursday’}\\
\multicolumn{5}{l}{\begin{itemize}
\item ‘Subtype-of’ relation – \textsc{n2} designates a specific type of \textsc{n1}\end{itemize}
}\\
& \textitbf{ana murit} & child school & \multicolumn{2}{l}{‘school kid’}\\
& \textitbf{kaing sprey} & cloth bed sheet & \multicolumn{2}{l}{‘bed sheets’}\\
\multicolumn{5}{l}{\begin{itemize}
\item ‘Composed-of’ relation – \textsc{n1} is composed of / made from \textsc{n2}\end{itemize}
}\\
& \textitbf{ruma batu} & house stone & \multicolumn{2}{l}{‘stone house’}\\
& \textitbf{kantong plastik} & bag plastic & \multicolumn{2}{l}{‘plastic bag’}\\
\multicolumn{5}{l}{\begin{itemize}
\item ‘Purpose-for’ relation: \textsc{n1} is intended for / at the disposal of \textsc{n2}\end{itemize}
}\\
& \textitbf{net laki{\Tilde}laki} & net \textsc{rdp}{\Tilde}husband & \multicolumn{2}{l}{‘(volleyball) net for men’}\\
& \textitbf{sikat gigi} & brush tooth & \multicolumn{2}{l}{‘toothbrush’}\\
\multicolumn{5}{l}{\begin{itemize}
\item Locational relation: (a) \textsc{n1} contains \textsc{n2}; (b) \textsc{n1} is located at/in/on \textsc{n2}; (c) \textsc{n1} originates from spatial source \textsc{n2}; (d) \textsc{n1} originates from nonspatial source \textsc{n2}\end{itemize}
}\\
(a) & \textitbf{lampu gas} & lamp gas & \multicolumn{2}{l}{‘gas lamp’}\\
(b) & \textitbf{jing{\Tilde}jing kayu} & genies wood & \multicolumn{2}{l}{‘tree genies’}\\
(c) & \textitbf{pisang Sorong} & banana Sorong & \multicolumn{2}{l}{‘bananas from Sorong’}\\
(d) & \textitbf{mop orang Sarmi} & joke people Sarmi & \multicolumn{2}{l}{‘joke by the Sarmi people’}\\
\multicolumn{5}{l}{\begin{itemize}
\item Temporal relation – \textsc{n2} gives temporal specifications for \textsc{n1}\end{itemize}
}\\
& \textitbf{jam dua pagi} & hour two morning & \multicolumn{2}{l}{‘two o’clock in the morning’}\\
& \textitbf{hari sening depang} & day Monday front & \multicolumn{2}{l}{‘next Monday’}\\
\multicolumn{5}{l}{\begin{itemize}
\item Event relation: \textsc{n2} is affected by event \textsc{n1}\end{itemize}
}\\
& \textitbf{pasang tugu} & install monument & \multicolumn{2}{l}{‘statue installation’}\\
\lspbottomrule
\end{tabular}
\subsection}]{Personal pronouns [\textsc{n} \textsc{pro}]}
\label{bkm:Ref361668509}\label{bkm:Ref309724486}
Papuan Malay noun phrases are often modified with personal pronouns in post-head position, such that ‘\textsc{n} \textsc{pro}’, as illustrated in (0) to (0). Signaling the definiteness, person, and number of their referents, the adnominally used personal pronouns allow the unambiguous identification of their referents. They are available for all person-number values, except for first person singular \textitbf{saya}/\textitbf{sa} ‘\textsc{1sg}’ which is unattested; the long and short pronoun forms are used interchangeably. (Personal pronouns and their adnominal uses are discussed in detail in Chapter 6; see also §5.5.)



Adnominal singular personal pronouns indicate the singularity of their referents, as shown with \textitbf{ko} ‘\textsc{2sg}’ and \textitbf{dia} ‘\textsc{3sg}’ in (0). In addition, they have pertinent discourse functions, discussed in detail in §6.2.1. Noun phrases with adnominally used plural personal pronouns have two readings. With an indefinite referent, such as \textitbf{laki{\Tilde}laki} ‘man’ in (0), the noun phrase receives an additive plural reading. With a definite referent such as \textitbf{Roni} in (0), the noun phrase receives an associative inclusory plural reading. Both readings are discussed in detail in §6.2.2.
\end{styleBodyvvafter}

\begin{styleExampleTitle}
Noun phrases with adnominal personal pronouns
\end{styleExampleTitle}

\begin{tabular}{llllllll}
\lsptoprule
\label{bkm:Ref365365430}
\gll {\bluebold{Wili}} {\bluebold{ko}} {jangang} {gara{\Tilde}gara} {\bluebold{tanta}} {\bluebold{dia}} {\bluebold{itu}}\\ %
& Wili & \textsc{2sg} & \textsc{neg.imp} & \textsc{rdp}{\Tilde}irritate & aunt & \textsc{3sg} & \textsc{d.dist}\\
\lspbottomrule
\end{tabular}
\ea
\glt 
‘\bluebold{you Wili} don’t irritate \bluebold{that aunt}’ \textstyleExampleSource{[081023-001-Cv.0038]}
\z

\begin{tabular}{llllllllll}
\lsptoprule
\label{bkm:Ref340775188}
\gll {jadi} {\bluebold{laki{\Tilde}laki}} {\bluebold{kitong}} {harus} {bayar} {spulu} {juta} {sama} {…}\\ %
& so & \textsc{rdp}{\Tilde}husband & \textsc{1pl} & have.to & pay & ten & million & with & \\
\lspbottomrule
\end{tabular}
\ea
\glt 
‘so \bluebold{we men} have to pay ten million to …’ \textstyleExampleSource{[081110-005-CvPr.0107]}
\z

\begin{tabular}{lllllllll}
\lsptoprule
\label{bkm:Ref340775171}
\gll {\bluebold{Roni}} {\bluebold{dong}} {kas} {tinggal} {itu} {babi} {di} {sini}\\ %
& Roni & \textsc{3pl} & give & stay & \textsc{d.dist} & pig & at & \textsc{l.prox}\\
\lspbottomrule
\end{tabular}
\ea
\glt
‘\bluebold{Roni and the others including Roni} left, what’s-its-name, the pig here’ \textstyleExampleSource{[080917-008-NP.0135]}
\end{styleFreeTranslEngxvpt}

\subsection}]{Demonstratives [\textsc{n} \textsc{dem}]}
\label{bkm:Ref293500756}
Within the noun phrase, adnominally used demonstratives are placed at the right periphery, where they have scope over the entire noun phrase, such that ‘\textsc{n} \textsc{dem}’: proximal \textitbf{ini} ‘\textsc{d.prox}’ or distal \textitbf{itu} ‘\textsc{d.dist}’, or their respective reduced forms \textitbf{ni} ‘\textsc{d.prox}’ and \textitbf{tu} ‘\textsc{d.dist}’. Like adnominally used personal pronouns (§8.2.3), the adnominal demonstratives function as determiners. Unlike the personal pronouns, however, they signal specificity rather than definiteness.\footnote{\\
\\
\\
\\
\\
\\
\\
\\
\\
\\
\\
\\
\\
\\
\par Concerning the semantic distinctions between the notion of definiteness and the notion of specificity, see Footnote 164 in §5.5 (p. \pageref{bkm:Ref438387261}).\\
}



The head nominal can be a noun such as \textitbf{ana} ‘child’ in (0), a personal pronoun such as \textitbf{dia} ‘\textsc{3sg}’ in (0), a locative such as \textitbf{sana} ‘\textsc{l.dist}’ in (0), or another demonstrative such as \textitbf{itu} ‘\textsc{d.dist}’ in (0). (Demonstratives and their adnominal uses are discussed in detail in §7.1; see also §5.6.)
\end{styleBodyvxafter}

\begin{tabular}{llllllll}
\lsptoprule
\label{bkm:Ref340861548}
\gll {\bluebold{ana}} {\bluebold{itu}} {sa} {paling} {sayang} {\bluebold{dia}} {\bluebold{tu}}\\ %
& child & \textsc{d.dist} & \textsc{1sg} & most & love & \textsc{3sg} & \textsc{d.dist}\\
\lspbottomrule
\end{tabular}
\ea
\glt 
‘\bluebold{that child}, I love \bluebold{her (}\blueboldSmallCaps{emph}\bluebold{)} most’ \textstyleExampleSource{[081011-023-Cv.0097]}
\z

\begin{tabular}{lllllll}
\lsptoprule
\label{bkm:Ref361850124}
\gll {sana,} {te} {ada} {di} {\bluebold{sana}} {\bluebold{itu}}\\ %
& \textsc{l.dist} & tea & exist & at & \textsc{l.dist} & \textsc{d.dist}\\
\lspbottomrule
\end{tabular}
\ea
\glt 
‘there, the tea is \bluebold{over there (}\blueboldSmallCaps{emph}\bluebold{)}’ \textstyleExampleSource{[081014-011-CvEx.0010]}
\z

\begin{tabular}{llllllllll}
\lsptoprule
\label{bkm:Ref365365566}
\gll {…} {\bluebold{itu}} {\bluebold{tu}} {kata{\Tilde}kata} {dasar} {yang} {harusnya} {kamu} {taw}\\ %
&  & \textsc{d.dist} & \textsc{d.dist} & \textsc{rdp}{\Tilde}word & base & \textsc{rel} & appropriately & \textsc{2pl} & know\\
\lspbottomrule
\end{tabular}
\ea
\glt
[Addressing a school student:] ‘[do you know the (English) word ‘please’ or not?,] \bluebold{that very} (word belongs to) the basic words that you should know’ \textstyleExampleSource{[081115-001a-Cv.0145]}
\end{styleFreeTranslEngxvpt}

\subsection}]{Locatives [\textsc{n} \textsc{loc}]}
\label{bkm:Ref288639682}
Adnominally used locatives occur in post-head position, such that ‘\textsc{n} \textsc{loc}’. This is illustrated with proximal \textitbf{sini} ‘\textsc{l.prox}’ in (0), and distal \textitbf{sana} ‘\textsc{l.dist}’ as in (0). The head nominal may be a noun such as \textitbf{ana} ‘child’ in (0), or a personal pronoun such as \textitbf{dong} ‘\textsc{3pl}’ in (0). (A detailed discussion on locatives and their adnominal uses is found in §7.2; see also §5.7.)
\end{styleBodyxafter}

\begin{tabular}{lllllll}
\lsptoprule
\label{bkm:Ref340861568}
\gll {kamu} {\bluebold{ana{\Tilde}ana}} {\bluebold{sini}} {\bluebold{tu}} {enak} {skali}\\ %
& \textsc{2pl} & \textsc{rdp}{\Tilde}child & \textsc{l.prox} & \textsc{d.dist} & be.pleasant & very\\
\lspbottomrule
\end{tabular}
\ea
\glt 
‘you, \bluebold{the young people here (}\blueboldSmallCaps{emph}\bluebold{)}, (live) very pleasant (lives)’ \textstyleExampleSource{[081115-001b-Cv.0060]}
\z

\begin{tabular}{lllll}
\lsptoprule
\label{bkm:Ref365365568}
\gll {\bluebold{dong}} {\bluebold{sana}} {cari} {anging}\\ %
& \textsc{3pl} & \textsc{l.dist} & search & wind\\
\lspbottomrule
\end{tabular}
\ea
\glt
‘\bluebold{they over there} are looking for a breeze’ \textstyleExampleSource{[081025-009b-Cv.0076]}
\end{styleFreeTranslEngxvpt}

\subsection}]{Interrogatives [\textsc{n} \textsc{int}]}
\label{bkm:Ref288639683}
In their adnominal uses, the interrogatives occur in post-head position, such that ‘\textsc{n} \textsc{int}’. Syntactically, the interrogatives remain in-situ; that is, noun phrases with adnominally used interrogatives correspond to their non-interrogative expressions. This is illustrated with \textitbf{siapa} ‘who’ in (0), \textitbf{apa} ‘what’ in (0), and \textitbf{mana} ‘where, which’ in (0).



The interrogatives always occur in noun phrases with a nominal head such as the common noun \textitbf{kaka} ‘older sibling’ in (0). Modification of personal pronouns or other constituents is unattested. (More details on the interrogatives and their adnominal uses are found in §5.8.)
\end{styleBodyvxafter}

\begin{tabular}{llllllll}
\lsptoprule
\label{bkm:Ref340862229}
\gll {skarang} {sa} {tanya,} {[\bluebold{orang}} {\bluebold{siapa}]} {yang} {benar?}\\ %
& now & \textsc{1sg} & ask & person & who & \textsc{rel} & be.true\\
\lspbottomrule
\end{tabular}
\ea
\glt 
‘now I asked, ‘\bluebold{which person} (is the one) who is right?’ \textstyleExampleSource{[080917-010-CvEx.0197]}
\z

\begin{tabular}{llllllll}
\lsptoprule
\label{bkm:Ref340862234}
\gll {[\bluebold{hari}} {\bluebold{apa}]} {yang} {sa} {ketemu} {dia} {e?}\\ %
& day & what & \textsc{rel} & \textsc{1sg} & meet & \textsc{3sg} & eh\\
\lspbottomrule
\end{tabular}
\ea
\glt 
‘\bluebold{which day} (is the one) that I met her, eh?’ \textstyleExampleSource{[080922-004-Cv.0013]}
\z

\begin{tabular}{llllllll}
\lsptoprule
\label{bkm:Ref340862232}
\gll {sa} {tanya} {dia,} {di} {[\bluebold{posisi}} {\bluebold{mana}]} {skarang?}\\ %
& 1sg & ask & 3sg & at & position & where & now\\
\lspbottomrule
\end{tabular}
\ea
\glt
‘I asked him, ‘\bluebold{which position} (is the one that are you) at now?’ \textstyleExampleSource{[081011-008-Cv.0023]}
\end{styleFreeTranslEngxvpt}

\subsection}]{Prepositional phrases [\textsc{n} \textsc{pp}]}
\label{bkm:Ref294015134}
Noun phrases can be modified with prepositional phrases, such that ‘\textsc{n} \textsc{pp}’. Overall, however, such noun phrases are uncommon. In the corpus, four prepositions occur in adnominally used prepositional phrases, namely locative \textitbf{di} ‘at, in’ as in (0), repeated as (0), elative \textitbf{dari} ‘from’ as in (0), benefactive \textitbf{untuk} ‘for’ as in (0), and similative \textitbf{sperti} ‘like’ as in (0). (For a detailed discussion on prepositions and prepositional phrases see Chapter 10; see also §5.11.)


\begin{styleExampleTitle}
Noun phrases with adnominal prepositional phrases
\end{styleExampleTitle}

\begin{tabular}{llllllll}
\lsptoprule
\label{bkm:Ref340905235}
\gll {\bluebold{dong}} {\bluebold{di}} {\bluebold{Papua}} {\bluebold{tu}} {dong} {makang} {papeda}\\ %
& \textsc{3pl} & at & Papua & \textsc{d.dist} & \textsc{3pl} & eat & sagu.porridge\\
\lspbottomrule
\end{tabular}
\ea
\glt 
‘\bluebold{they in Papua there}, they eat sagu porridge’ \textstyleExampleSource{[081109-009-JR.0001]}
\z

\begin{tabular}{llllllll}
\lsptoprule
\label{bkm:Ref340905236}
\gll {itu} {\bluebold{iblis{\Tilde}iblis}} {\bluebold{dari}} {\bluebold{ruangang}} {\bluebold{ini}} {yang} {ganggu}\\ %
& \textsc{d.dist} & \textsc{rdp}{\Tilde}devil & from & room & \textsc{d.prox} & \textsc{rel} & disturb\\
\lspbottomrule
\end{tabular}
\ea
\glt 
‘it’s \bluebold{the devils from this room} who are disturbing (you)’ \textstyleExampleSource{[081011-008-CvPh.0018]}
\z

\begin{tabular}{llllllll}
\lsptoprule
\label{bkm:Ref340905237}
\gll {di} {sana} {kang} {masi} {\bluebold{tempat}} {\bluebold{untuk}} {\bluebold{kafir}}\\ %
& at & \textsc{l.dist} & you.know & still & place & for & unbeliever\\
\lspbottomrule
\end{tabular}
\ea
\glt 
‘(the area) over there, you know, is still \bluebold{a location for unbelievers}’ \textstyleExampleSource{[081011-022-Cv.0238]}
\z

\begin{tabular}{llllllll}
\lsptoprule
\label{bkm:Ref340905239}
\gll {\bluebold{orang{\Tilde}orang}} {\bluebold{sperti}} {\bluebold{begitu}} {yang} {tida} {mengenal} {Kristus}\\ %
& \textsc{rdp}{\Tilde}person & like & like.that & \textsc{rel} & \textsc{neg} & know & Kristus\\
\lspbottomrule
\end{tabular}
\ea
\glt
‘(it’s) \bluebold{people like those} who don’t know Christ …’ \textstyleExampleSource{[081006-023-CvEx.0034]}
\end{styleFreeTranslEngxvpt}

\subsection}]{Relative clauses [\textsc{n} \textsc{rc}]}
\label{bkm:Ref362506247}
Relative clauses are introduced with the relativizer \textitbf{yang} ‘\textsc{rel}’. They always follow their head nominal, such that ‘\textsc{n} \textsc{rc}’. The head nominal can be a noun as in (0), a personal pronoun as in (0), a demonstrative as in (0), a locative as in (0), or an interrogative as in (0). The syntax of relatives clauses is discussed in detail in §14.3.2 (see also the respective sections in Chapter 5 on ‘Word classes‘, as well as Chapter 6, and Chapter 7).
\end{styleBodyxafter}

\begin{tabular}{llllllll}
\lsptoprule
\label{bkm:Ref362507993}
\gll {…} {tapi} {di} {sini} {\bluebold{prempuang}} {\bluebold{yang}} {tokok}\\ %
&  & but & at & \textsc{l.prox} & woman & \textsc{rel} & tap\\
\lspbottomrule
\end{tabular}
\ea
\glt 
‘[at Pante-Timur all the men pound (sago),] but here (it’s) \bluebold{the women who} pound (sago)’ \textstyleExampleSource{[081014-007-CvEx.0073]}
\z

\begin{tabular}{llllll}
\lsptoprule
\label{bkm:Ref362507994}
\gll {a,} {\bluebold{ko}} {\bluebold{yang}} {tanya} {to?}\\ %
& ah! & \textsc{2sg} & \textsc{rel} & ask & right?\\
\lspbottomrule
\end{tabular}
\ea
\glt 
‘ah, (it was) \bluebold{you who} asked, right?’ \textstyleExampleSource{[080923-014-CvEx.0010]}
\z

\begin{tabular}{lllllll}
\lsptoprule
\label{bkm:Ref362507996}
\gll {\bluebold{itu}} {\bluebold{yang}} {orang} {Papua} {skarang} {maw}\\ %
& \textsc{d.dist} & \textsc{rel} & person & Papua & now & want\\
\lspbottomrule
\end{tabular}
\ea
\glt 
‘(it’s) \bluebold{that what} Papuans want nowadays’ \textstyleExampleSource{[081025-004-Cv.0077]}
\z

\begin{tabular}{llllll}
\lsptoprule
\label{bkm:Ref362507997}
\gll {di} {\bluebold{sini}} {\bluebold{yang}} {tra} {banyak}\\ %
& at & \textsc{l.prox} & \textsc{rel} & \textsc{neg} & many\\
\lspbottomrule
\end{tabular}
\ea
\glt 
[About logistic problems:] ‘(it’s) \bluebold{here where} there weren’t many (passengers)’ \textstyleExampleSource{[081025-008-Cv.0140]}
\z

\begin{tabular}{lllllllll}
\lsptoprule
\label{bkm:Ref362507995}
\gll {kamu} {tida} {perna} {dengar} {\bluebold{apa}} {\bluebold{yang}} {orang-tua} {bicara}\\ %
& \textsc{2pl} & \textsc{neg} & ever & listen & what & \textsc{rel} & parent & speak\\
\lspbottomrule
\end{tabular}
\ea
\glt
‘because you never listened to \bluebold{what} the elders said’ \textstyleExampleSource{[081115-001a-Cv.0338]}
\end{styleFreeTranslEngxvpt}

\section{\textsc{n-mod} / \textsc{mod-n} structure}
\label{bkm:Ref294348207}
Noun phrases with adnominally used numerals or quantifiers can have an \textsc{n-mod} or a \textsc{mod-n} structure, depending on the semantics of the phrasal structure. When preposed, adnominal numerals and quantifiers signal individuality, while postposed numerals and quantifiers express exhaustivity or positions within series. Post-head numerals and quantifiers have scope over their head nominal including its verbal and/or nominal modifiers, while they, in turn, are within the scope of the demonstratives. Adnominally used numerals are discussed in §8.3.1, and adnominal quantifiers in §8.3.2.
\end{styleBodyxvafter}

\subsection}]{Numerals [\textsc{n} \textsc{n}um / \textsc{num} \textsc{n}]}
\label{bkm:Ref374454416}\label{bkm:Ref374450333}\label{bkm:Ref361674989}\label{bkm:Ref275783875}
Two types of noun phrases with adnominally used numerals can be distinguished: (1) noun phrases with pre-head numerals, such that ‘\textsc{n}um \textsc{n}’, are presented in (0) to (0), and (2) noun phrases with post-head numerals\textstyleBodyafterChar{,} such that ‘\textsc{n} \textsc{n}um’, are illustrated in (0) to (0). (For a discussion of numerals as a word class see §5.9.)



Noun phrases with preposed numerals (‘\textsc{n}um\textsc{n-np}’) express a sense of individuality by signaling the composite nature of their referents. This is achieved in that \textsc{n}um\textsc{n-np}s denote absolute numbers of items expressed by their head nominals, including quantities as in (0) or periods of time as in (0).
\end{styleBodyvvafter}

\begin{styleExampleTitle}
\textsc{n}um\textsc{n-np}s denoting definite quantities of countable referents: Individuality
\end{styleExampleTitle}

\begin{tabular}{llllllll}
\lsptoprule
\label{bkm:Ref340905241}
\gll {…} {brarti} {suda} {\bluebold{empat}} {\bluebold{orang}} {bisa} {masuk}\\ %
&  & mean & already & four & person & be.able & enter\\
\lspbottomrule
\end{tabular}
\ea
\glt 
[About local elections:] ‘… that means that already \bluebold{four people} can be included (in the list of nominees)’ \textstyleExampleSource{[080919-001-Cv.0149]}
\z

\begin{tabular}{lllllllllllllll}
\lsptoprule
\label{bkm:Ref340905245}
\gll {\multicolumn{2}{l}{ini}} {\multicolumn{2}{l}{untuk}} {\multicolumn{3}{l}{balita}} {\multicolumn{2}{l}{dang}} {\multicolumn{2}{l}{bayi}} {yang} {usia} {dari}\\ %
& \multicolumn{2}{l}{\textsc{d.prox}} & \multicolumn{2}{l}{for} & \multicolumn{3}{l}{children.under.five} & \multicolumn{2}{l}{and} & \multicolumn{2}{l}{baby} & \textsc{rel} & age & from\\
& \bluebold{lima} & \multicolumn{2}{l}{\bluebold{taung}} & \multicolumn{2}{l}{ke} & bawa & \multicolumn{2}{l}{sampe} & \multicolumn{2}{l}{\bluebold{dua}} & \multicolumn{4}{l}{\bluebold{bulang}}\\
& five & \multicolumn{2}{l}{year} & \multicolumn{2}{l}{to} & bottom & \multicolumn{2}{l}{until} & \multicolumn{2}{l}{two} & \multicolumn{4}{l}{month}\\
\lspbottomrule
\end{tabular}
\ea
\glt 
‘this is for children and babies who are \bluebold{five years} down to \bluebold{two months}’ \textstyleExampleSource{[081010-001-Cv.0197]}
\z


If the exact absolute number of items is unknown, two numerals can be juxtaposed to indicate approximate quantities, as in (0) and (0). The approximated quantities are usually rather small such as \textitbf{satu dua} ‘one or two’ in (0) or \textitbf{tiga empat} ‘three or four’ as in (0).


\begin{styleExampleTitle}
\textsc{n}um\textsc{n-n}um\textsc{n-np}s denoting approximate quantities
\end{styleExampleTitle}

\begin{tabular}{lllllllll}
\lsptoprule
\label{bkm:Ref340905247}
\gll {jangang} {ko} {lama} {ko} {\bluebold{satu}} {\bluebold{dua}} {\bluebold{hari}} {saja}\\ %
& \textsc{neg.imp} & \textsc{2sg} & be.long & \textsc{2sg} & one & two & day & just\\
\lspbottomrule
\end{tabular}
\ea
\glt 
‘don’t (stay) long, just \bluebold{one} or \bluebold{two days}’ \textstyleExampleSource{[080922-001a-CvPh.0736]}
\z

\begin{tabular}{llllllll}
\lsptoprule
\label{bkm:Ref340905248}
\gll {\bluebold{tiga}} {\bluebold{empat}} {\bluebold{kluarga}} {harus} {ada} {di} {situ}\\ %
& three & four & family & have.to & exist & at & \textsc{l.med}\\
\lspbottomrule
\end{tabular}
\ea
\glt 
‘\bluebold{three} or \bluebold{four families} have to be there’ \textstyleExampleSource{[080923-007-Cv.0018]}
\z


Noun phrases with post-head numerals (‘\textsc{nn}um\textsc{{}-np}’) signal exhaustivity of definite referents, as in (0) and (0), or mark unique positions within series or sequences as in (0) and (0).



With head nominals undifferentiated in terms of their ranking, \textsc{nn}um\textsc{{}-np}s indicate exhaustivity of definite referents. The head can be a noun, as in the elicited example in (0), or a personal pronoun as in (0).\footnote{\\
\\
\\
\\
\\
\\
\\
\\
\\
\\
\\
\\
\\
\\
\par The elicited example in (0) is based on the example in (0) in §8.1 (p. \pageref{bkm:Ref439577754}).\\
}
\end{styleBodyvvafter}

\begin{styleExampleTitle}
\textsc{nn}um\textsc{{}-np}s denoting definite quantities of countable referents: Exhaustivity
\end{styleExampleTitle}

\begin{tabular}{llllllll}
\lsptoprule
\label{bkm:Ref340905254}
\gll {trus} {tamba} {[[[\bluebold{kaka}} {\bluebold{dari}} {\bluebold{Mambramo}]} {\bluebold{tiga}]} {\bluebold{ni}]}\\ %
& next & add & oSb & from & Mambramo & one & \textsc{d.prox}\\
\lspbottomrule
\end{tabular}
\ea
\glt 
[About forming a volleyball team:] ‘and then add \bluebold{these three older brothers from Mambramo}’ \textstyleExampleSource{[Elicited BR111018.004]}
\z

\begin{tabular}{lllllllll}
\lsptoprule
\label{bkm:Ref340905255}
\gll {nanti} {\bluebold{kitong}} {\bluebold{empat}} {su} {tidor} {di} {luar} {…}\\ %
& very.soon & \textsc{1pl} & four & already & sleep & at & outside & \\
\lspbottomrule
\end{tabular}
\ea
\glt 
‘after \bluebold{the four of us} had already been sleeping outside …’ \textstyleExampleSource{[081025-009a-Cv.0004]}
\z


With head nominals differentiated in terms of their ranking within a series, \textsc{nn}um\textsc{{}-np}s signal the unique position of a referent within such a ranking as in (0), or they specify unique points in time as in (0).


\begin{styleExampleTitle}
\textsc{nn}um\textsc{{}-np}s denoting definite quantities of countable referents: Unique positions or points in time
\end{styleExampleTitle}

\begin{tabular}{lllllll}
\lsptoprule
\label{bkm:Ref340905249}
\gll {kitong} {lari{\Tilde}lari} {sampe} {di} {\bluebold{SP}} {\bluebold{tuju}}\\ %
& \textsc{1pl} & \textsc{rdp}{\Tilde}run & reach & at & transmigration.settlement & seven\\
\lspbottomrule
\end{tabular}
\ea
\glt 
‘we drove all the way to \bluebold{transmigration settlement number seven}’ (Lit. ‘\bluebold{the seventh transmigration settlement}’) \textstyleExampleSource{[081006-033-Cv.0007]}
\z

\begin{tabular}{lllllllll}
\lsptoprule
\label{bkm:Ref340905251}
\gll {\bluebold{jam}} {\bluebold{dua},} {tong} {kluar} {dari} {sini} {\bluebold{jam}} {\bluebold{satu}}\\ %
& hour & two & \textsc{1pl} & go.out & from & \textsc{l.prox} & hour & one\\
\lspbottomrule
\end{tabular}
\ea
\glt 
‘(we arrived at) \bluebold{two o’clock}, we left from here at \bluebold{one o’clock}’ \textstyleExampleSource{[081025-008-Cv.0099]}
\z


In (0) to (0), the opposition between the pre- and post-head positions is illustrated with (near) contrastive examples. In (0) pre-head \textitbf{dua} ‘two’ designates the absolute number of items expressed by its head. In (0) post-head \textitbf{dua} ‘two’ modifies a head nominal undifferentiated in terms of its ranking, whereby it signals the exhaustivity of its referent. In (0) post-head \textitbf{dua} ‘two’ signals a unique position within a series.


\begin{styleExampleTitle}
Opposition between \textsc{n}um\textsc{n-np}s and \textsc{nn}um\textsc{{}-np}s
\end{styleExampleTitle}

\begin{tabular}{lllllllllll}
\lsptoprule
\label{bkm:Ref340905259}
\gll {saya} {jaga} {\bluebold{dua}} {\bluebold{jam},} {yo} {kurang} {lebi} {\bluebold{dua}} {\bluebold{jam}} {…}\\ %
& \textsc{1sg} & guard & two & hour & yes & lack & more & two & hour & \\
\lspbottomrule
\end{tabular}
\ea
\glt 
‘I kept watch for \bluebold{two hours}, yes, more or less for \bluebold{two hours} …’ \textstyleExampleSource{[080919-004-NP.0016]}
\z

\begin{tabular}{lllll}
\lsptoprule
\label{bkm:Ref340905261}
\gll {\bluebold{sidi}} {\bluebold{dua}} {dia} {potong}\\ %
& CD.player & two & \textsc{3sg} & cut\\
\lspbottomrule
\end{tabular}
\ea
\glt 
‘\bluebold{both CD players}, he destroyed (them)’ \textstyleExampleSource{[081011-009-Cv.0006]}
\z

\begin{tabular}{llllll}
\lsptoprule
\label{bkm:Ref340905260}
\gll {ini} {suda} {\bluebold{jam}} {\bluebold{dua}} {malam}\\ %
& \textsc{d.prox} & already & hour & two & night\\
\lspbottomrule
\end{tabular}
\ea
\glt 
‘this is already \bluebold{two o’clock} at night’ \textstyleExampleSource{[080916-001-CvNP.0001]}
\z


The data in (0) to (0) suggests that the \textsc{nn}um order is favored in more specific and definite constructions, namely to signal exhaustivity of definite referents or unique positions within series or sequences. The \textsc{n}um\textsc{n} order, by contrast, is associated with less specific or less definite constructions which express the absolute number of items denoted by the head nominal. These patterns contrast with {Greenberg’s (1978: 284)} cross-linguistic findings concerning the word order in noun phrases with adnominal numerals:


\begin{styleIvI}
44. The order noun-numeral is favored in indefinite and approximative constructions.
\end{styleIvI}


{\citet[284]{Greenberg1978}} does note, however, that this statement is a generalization rather than a universal, given cross-linguistic variations in quantifier-noun [Q-N] order. Noting that “in some languages either QN or NQ may occur with any numeral” and that this “contrast of order may then have semantic or syntactic function”, {\citet[284]{Greenberg1978}} presents a number of languages that, like Papuan Malay, employ \textsc{nn}um order in definite constructions rather than in indefinite ones.



Following {\citet[284]{Greenberg1978}}, the Papuan Malay \textsc{nn}um order in definite constructions is a variation of a much more common \textsc{n}um\textsc{n} order for these constructions. In his critique of {Greenberg’s (1978: 284)} generalization \#44, {\citet{Donohue2005a}} demonstrates, however, that the \textsc{nn}um order in definite constructions is not a mere “variation” found in “some languages”. Rather, “there is a strong tendency for postnominal numerals to be interpreted in highly specific, highly definite ways” {(2005a: 34)}. The data presented here suggests that the Papuan Malay word order in noun phrases with adnominally used numerals follows this same “strong tendency”.
\end{styleBodyvxvafter}

\subsection}]{Quantifiers [\textsc{n} \textsc{qt} / \textsc{qt} \textsc{n}]}
\label{bkm:Ref293904993}\label{bkm:Ref275783878}
Noun phrases with adnominally used quantifiers have syntactic properties similar to those with adnominally used numerals. Noun phrases with pre-head quantifiers (‘\textsc{q}t\textsc{n-np}’) express non-numeric amounts or quantities of the items indicated by their head nominals; they only modify countable referents. Noun phrases with post-head quantifier (‘\textsc{nq}t\textsc{{}-np}’), by contrast, either denote exhaustivity of indefinite referents or signal unknown positions within series or sequences; they modify countable as well as uncountable referents. (For a discussion of quantifiers as a word class see §5.10.)



The following adnominal quantifiers are attested: universal \textitbf{masing-masing} ‘each’, \textitbf{segala} ‘all’, \textitbf{sembarang} ‘any (kind of)’, \textitbf{(se)tiap} ‘every’, and \textitbf{smua} ‘all’, and mid-range \textitbf{banyak} ‘many’, \textitbf{brapa} ‘several’, \textitbf{sedikit} ‘few’, and \textitbf{stenga} ‘half’.
\end{styleBodyvafter}


Five quantifiers can occur in pre- or post-head position, namely \textitbf{banyak} ‘many’, \textitbf{brapa} ‘several’, \textitbf{masing-masing} ‘each’, \textitbf{sedikit} ‘few’, and \textitbf{smua} ‘all’, as shown in (0) to (0). The other four quantifiers, that is, \textitbf{segala} ‘all’, \textitbf{sembarang} ‘any (kind of)’, \textitbf{(se)tiap} ‘every’, and \textitbf{stenga} ‘half’, only occur in pre-head position where they signal non-numeric quantities of countable referents, as illustrated in (0) to (0). While \textitbf{sembarang} ‘any (kind of)’ is only used with animate referents as in (0), \textitbf{(se)tiap} ‘every’ and \textitbf{stenga} ‘half’ are only used with inanimate referents as in (0) and (0), respectively.\footnote{\\
\\
\\
\\
\\
\\
\\
\\
\\
\\
\\
\\
\\
\\
\par To express the notion of ‘every person’, speakers prefer quantification with \textitbf{masing-masing} ‘each’.\\
} Quantifier \textitbf{segala} ‘all’ is always combined with the noun \textitbf{macang} ‘variety’ with \textitbf{segala macang} expressing the notion of ‘all kinds, whatever kind’ as in (0).
\end{styleBodyvvafter}

\begin{styleExampleTitle}
\textsc{q}t\textsc{n-np}s denoting indefinite quantities of countable referents: Individuality
\end{styleExampleTitle}

\begin{tabular}{lllll}
\lsptoprule
\label{bkm:Ref340905803}
\gll {\bluebold{segala}} {\bluebold{macang}} {dia} {biking}\\ %
& all & variety & \textsc{3sg} & make\\
\lspbottomrule
\end{tabular}
\ea
\glt 
[About an ancestor’s achievements:] ‘\bluebold{all kinds (of things)}, he made (them)’ \textstyleExampleSource{[080922-010a-CvNF.0297]}
\z

\begin{tabular}{lllllll}
\lsptoprule
\label{bkm:Ref340905269}
\gll {sa} {tra} {bisa} {kasi} {\bluebold{sembarang}} {\bluebold{orang}}\\ %
& \textsc{1sg} & \textsc{neg} & be.able & give & any(.kind.of) & person\\
\lspbottomrule
\end{tabular}
\ea
\glt 
‘I can’t give (the gasoline to just) \bluebold{any person}’ \textstyleExampleSource{[081110-002-Cv.0080]}
\z

\begin{tabular}{lllllllll}
\lsptoprule
\label{bkm:Ref340905270}
\gll {\bluebold{setiap}} {\bluebold{renungang}} {\bluebold{pagi}} {sa} {su} {kasi} {nasihat} {itu}\\ %
& every & meditation & morning & \textsc{1sg} & already & give & advice & \textsc{d.dist}\\
\lspbottomrule
\end{tabular}
\ea
\glt 
‘(during) \bluebold{each morning devotions}, I already give (them) that (same) advice’ \textstyleExampleSource{[081115-001b-Cv.0008]}
\z

\begin{tabular}{lllllll}
\lsptoprule
\label{bkm:Ref340905802}
\gll {mungking} {\bluebold{stenga}} {\bluebold{jam}} {saja} {sa} {tidor}\\ %
& maybe & half & hour & just & \textsc{1sg} & sleep\\
\lspbottomrule
\end{tabular}
\ea
\glt 
‘I slept for maybe just \bluebold{half an hour}’ \textstyleExampleSource{[081115-001b-Cv.0056]}
\z


The quantifiers \textitbf{banyak} ‘many’, \textitbf{brapa} ‘several’, \textitbf{masing-masing} ‘each’, \textitbf{sedikit} ‘few’, and \textitbf{smua} ‘all’ can precede or follow their head nominals, as demonstrated in (0) to (0). Both phrasal structures serve distinct semantic functions similar to those of adnominal numerals, discussed in §8.3.1, although the contrast is more subtle. The examples presented in this section also illustrate that the quantifiers can be used with animate or inanimate referents.



\textsc{q}t\textsc{n-np}s with pre-head \textitbf{banyak} ‘many’, \textitbf{brapa} ‘several’, \textitbf{masing-masing} ‘each’, \textitbf{sedikit} ‘few’, and \textitbf{smua} ‘all’ denote the non-numeric quantities of countable referents. Thereby, \textsc{q}t\textsc{n-np}s express the composite nature of their referents which conveys a sense of individuality, such that ‘\textsc{q}t amount of \textsc{n}’ as in (0) to (0). The corpus includes only few noun phrases with adnominally used \textitbf{sedikit} ‘few’ all of which have \textitbf{sedikit} ‘few’ in post-head position. According to one of the consultants, however, adnominal modification with pre-head \textitbf{sedikit} ‘few’ is natural and common, as illustrated with the elicited example in (0).
\end{styleBodyvvafter}

\begin{styleExampleTitle}
\textsc{q}t\textsc{n-np}s denoting indefinite quantities of countable referents: Individuality
\end{styleExampleTitle}

\begin{tabular}{lllllllll}
\lsptoprule
\label{bkm:Ref406519586}
\gll {de} {itu} {kalo} {\bluebold{banyak}} {\bluebold{orang}} {de} {biasa} {begitu}\\ %
& \textsc{3sg} & \textsc{d.dist} & when & many & person & \textsc{3sg} & be.usual & like.that\\
\lspbottomrule
\end{tabular}
\ea
\glt 
‘if there’re \bluebold{many people}, he’s usually like that’ \textstyleExampleSource{[081025-006-Cv.0272]}
\z

\begin{tabular}{llllllllll}
\lsptoprule
(\stepcounter{}{\the}) & tentara & \multicolumn{2}{l}{itu} & ada & brapa & ratus & orang, & ada & sekitar\\
& soldier & \multicolumn{2}{l}{\textsc{d.dist}} & exist & several & hundred & person & exist & vicinity\\
& \multicolumn{2}{l}{\bluebold{brapa}} & \multicolumn{7}{l}{\bluebold{pleton}}\\
& \multicolumn{2}{l}{several} & \multicolumn{7}{l}{platoon}\\
\lspbottomrule
\end{tabular}
\ea
\glt 
‘those soldiers were several hundred people, (they) were approximately \bluebold{several platoons}’ \textstyleExampleSource{[081029-005-Cv.0131]}
\z

\begin{tabular}{lllllll}
\lsptoprule
(\stepcounter{}{\the}) & bayar & mas-kawing & ini & laing & \bluebold{masing-masing} & \bluebold{budaya}\\
& pay & bride.price & \textsc{d.prox} & be.different & each & culture\\
\lspbottomrule
\end{tabular}
\ea
\glt 
‘paying this bride price is different (for) \bluebold{each culture}’ \textstyleExampleSource{[081006-029-CvEx.0014]}
\z

\begin{tabular}{lllllllll}
\lsptoprule
\label{bkm:Ref340905810}
\gll {de} {itu} {kalo} {\bluebold{sedikit}} {\bluebold{orang}} {de} {biasa} {begitu}\\ %
& \textsc{3sg} & \textsc{d.dist} & when & few & person & \textsc{3sg} & be.usual & like.that\\
\lspbottomrule
\end{tabular}
\ea
\glt 
‘if there’re \bluebold{few people}, he’s usually like that’ \textstyleExampleSource{[Elicited BR111021.004]}
\z

\begin{tabular}{lllll}
\lsptoprule
\label{bkm:Ref340905805}
\gll {\bluebold{smua}} {\bluebold{buku}} {bisa} {basa}\\ %
& all & book & be.able & be.wet\\
\lspbottomrule
\end{tabular}
\ea
\glt 
‘\bluebold{all books} could get wet’ \textstyleExampleSource{[080917-008-NP.0189]}
\z


\textsc{nq}t\textsc{{}-np}s with post-head \textitbf{banyak} ‘many’, \textitbf{brapa} ‘several’, \textitbf{masing-masing} ‘each’, \textitbf{sedikit} ‘few’, and \textitbf{smua} ‘all’ typically signal exhaustivity of indefinite countable referents, as shown in (0) to (0). Besides, \textsc{nq}t\textsc{{}-np}s with post-head \textitbf{brapa} ‘how many’ can denote unknown positions within series of countable referents, as in (0). While the head in \textsc{nq}t\textsc{{}-np}s is typically noun, as in (0), it can also be a personal pronoun as in (0).



Most often,\textsc{ nq}t\textsc{{}-np}s signal a contrastive sense of exhaustivity: \textsc{n} \textitbf{banyak} translates with ‘many (and not just a few) \textsc{n}’ as in (0), \textsc{n} \textitbf{masing-masing} with ‘several (and not just a few) \textsc{n}’ as in (0), \textsc{n} \textitbf{masing-masing} with ‘each \textsc{n }(with nobody missing)’ as in the elicited example in (0), \textsc{n} \textitbf{sedikit} with ‘few (and not many) \textsc{n}’ as in (0), and \textsc{n} \textitbf{smua} with ‘the entire collection of \textsc{n} (with nobody/nothing missing)’ as in (0). As mentioned above, the corpus includes only few noun phrases with adnominally used \textitbf{sedikit} ‘few’ one of which is presented in (0): \textitbf{ikang sedikit} ‘few fish’. Alternatively, however, \textitbf{ikang sedikit} could receive the predicative reading ‘the fish are few’. Therefore, an additional elicited example is given in (0).
\end{styleBodyvvafter}

\begin{styleExampleTitle}
\textsc{nq}t\textsc{{}-np}s denoting indefinite quantities of countable referents: Exhaustivity
\end{styleExampleTitle}

\begin{tabular}{llllll}
\lsptoprule
\label{bkm:Ref340905814}
\gll {…} {baca} {\bluebold{buku}} {\bluebold{banyak}} {\bluebold{skali}}\\ %
&  & read & book & many & very\\
\lspbottomrule
\end{tabular}
\ea
\glt 
‘… (I’ve) read \bluebold{very many books}’ \textstyleExampleSource{[080917-010-CvEx.0172]}
\z

\begin{tabular}{llllll}
\lsptoprule
\label{bkm:Ref362343918}
\gll {sa} {maki} {\bluebold{dorang}} {\bluebold{brapa}} {itu}\\ %
& \textsc{1sg} & abuse.verbally & \textsc{3pl} & several & \textsc{d.dist}\\
\lspbottomrule
\end{tabular}
\ea
\glt 
‘I verbally abused \bluebold{several of them} there’ \textstyleExampleSource{[080923-008-Cv.0012]}
\z

\begin{tabular}{llllllllllll}
\lsptoprule
\label{bkm:Ref340905819}
\gll {\multicolumn{2}{l}{dong}} {antar} {\multicolumn{2}{l}{petatas}} {\multicolumn{2}{l}{dengang}} {sayur} {dulu} {taru} {tumpukang}\\ %
& \multicolumn{2}{l}{\textsc{3pl}} & bring & \multicolumn{2}{l}{sweet.potato} & \multicolumn{2}{l}{with} & vegetable & first & put & pile\\
& di & \multicolumn{3}{l}{\bluebold{klompok}} & \multicolumn{2}{l}{\bluebold{masing-masing}} & \multicolumn{5}{l}{begitu}\\
& at & \multicolumn{3}{l}{group} & \multicolumn{2}{l}{each} & \multicolumn{5}{l}{like.that}\\
\lspbottomrule
\end{tabular}
\ea
\glt 
‘first they bring the sweet potatoes and vegetables (and) place the piles (of food) in (front of) \bluebold{each group} like that’ \textstyleExampleSource{[Elicited BR111021.001]}\footnote{\\
\\
\\
\\
\\
\\
\\
\\
\\
\\
\\
\\
\\
\\
\par The elicited example in (0) is the corrected version of the original recording \textitbf{tumpukang masing klompok masing-masing} ‘pile each[\textsc{tru}] group each’ [081014-017-CvPr.0043]. That is, the speaker started off by saying \textitbf{tumpukang masing-masing} but she corrected herself, resulting in the truncated quantifier \textitbf{masing} ‘each[\textsc{tru}]’ and the missing locative preposition \textitbf{di} ‘at’.\\
}
\z

\begin{tabular}{lllllll}
\lsptoprule
\label{bkm:Ref340905815}
\gll {kalo} {\bluebold{ikang}} {\bluebold{sedikit},} {itu} {untuk} {tamu}\\ %
& if & fish & few & \textsc{d.dist} & for & guest\\
\lspbottomrule
\end{tabular}
\ea
\glt 
‘as for the \bluebold{few fish}, those are for the guests’ \textstyleExampleSource{[081014-011-CvEx.0008]}
\z

\begin{tabular}{lllllllll}
\lsptoprule
\label{bkm:Ref340905818}
\gll {sa} {ada} {bawa} {\bluebold{kladi}} {\bluebold{sedikit}} {buat} {mama} {dong}\\ %
& \textsc{1sg} & exist & bring & taro.root & few & for & mother & \textsc{3pl}\\
\lspbottomrule
\end{tabular}
\ea
\glt 
‘I’m bringing \bluebold{a few taro roots} for mother and the others’ \textstyleExampleSource{[Elicited }\textstyleExampleSource{BR111021.006}\textstyleExampleSource{]}
\z

\begin{tabular}{lllll}
\lsptoprule
\label{bkm:Ref340905813}
\gll {\bluebold{tong}} {\bluebold{smua}} {dari} {kampung}\\ %
& \textsc{1pl} & all & from & village\\
\lspbottomrule
\end{tabular}
\ea
\glt 
‘\bluebold{we all} are from the village’ \textstyleExampleSource{[081010-001-Cv.0084]}
\z


Depending on the semantics of the head nominal, \textsc{nq}t\textsc{{}-np}s with post-head \textitbf{brapa} ‘several’ can also mark unknown positions within series expressed by their referents, as in (0).


\begin{styleExampleTitle}
\textsc{nq}t\textsc{{}-np}s denoting indefinite quantities of countable referents: Exhaustivity or unknown positions within series
\end{styleExampleTitle}

\begin{tabular}{lllllllll}
\lsptoprule
\label{bkm:Ref340905820}
\gll {kalo} {di} {situ} {kang,} {\bluebold{jam}} {\bluebold{brapa}} {saja} {bisa}\\ %
& if & at & \textsc{l.med} & you.know & hour & several & just & be.able\\
\lspbottomrule
\end{tabular}
\ea
\glt 
‘as for (the office) there, you know, (you) can (go there) \bluebold{any time}’ (Lit. ‘\bluebold{several hours}’) \textstyleExampleSource{[081005-001-Cv.0001]}
\z


Noun phrases with uncountable referents are modified with post-head quantifiers only, as shown in (0) to (0). This restriction is due to the semantics of mass nouns which, per se, do not convey the sense of individuality encoded by the pre-head position of the quantifiers, presented in (0) to (0). Adnominal quantifiers for mass nouns are \textitbf{banyak} ‘many’ as in (0), \textitbf{sedikit} ‘few’ as in (0), or \textitbf{smua} ‘all’ as in (0).


\begin{styleExampleTitle}
\textsc{nq}t\textsc{{}-np}s denoting indefinite quantities of uncountable referents: Exhaustivity
\end{styleExampleTitle}

\begin{tabular}{lllllll}
\lsptoprule
\label{bkm:Ref340905821}
\gll {minum} {\bluebold{te}} {\bluebold{banyak},} {minum} {te} {dulu}\\ %
& drink & tea & many & drink & tea & first\\
\lspbottomrule
\end{tabular}
\ea
\glt 
‘drink \bluebold{lots of tea}, drink tea for now!’ \textstyleExampleSource{[081011-001-Cv.0240]}
\z

\begin{tabular}{llllllllll}
\lsptoprule
\label{bkm:Ref340905822}
\gll {tida} {bisa} {\bluebold{air}} {\bluebold{sedikit}} {pung} {sentu} {sa} {pu} {mulut}\\ %
& \textsc{neg} & be.able & water & few & even & touch & \textsc{1sg} & \textsc{poss} & mouth\\
\lspbottomrule
\end{tabular}
\ea
\glt 
[About a sickness:] ‘not even \bluebold{the least bit of water} could touch my mouth’ \textstyleExampleSource{[081006-035-CvEx.0050]}
\z

\begin{tabular}{lllllll}
\lsptoprule
\label{bkm:Ref340905823}
\gll {…} {buka} {\bluebold{de}} {\bluebold{pu}} {\bluebold{kulit}} {\bluebold{smua}}\\ %
&  & open & \textsc{3sg} & \textsc{poss} & skin & all\\
\lspbottomrule
\end{tabular}
\ea
\glt 
‘(they) peel off \bluebold{his entire skin}’ \textstyleExampleSource{[081029-004-Cv.0047]}
\z


Typically, post-head \textitbf{smua} ‘all’ forms a constituent with the quantified nominal. Alternatively, however, it can float to a clause-final position, as shown in (0) and (0).


\begin{styleExampleTitle}
Floating adnominal quantifier \textitbf{smua} ‘all’
\end{styleExampleTitle}

\begin{tabular}{lllll}
\lsptoprule
\label{bkm:Ref340905831}
\gll {\bluebold{makangang}} {kas} {tinggal} {\bluebold{smua}}\\ %
& food & give & stay & all\\
\lspbottomrule
\end{tabular}
\ea
\glt 
‘(he was made) to leave \bluebold{all (his) food} (untouched)’ \textstyleExampleSource{[081025-008-Cv.0048]}
\z

\begin{tabular}{llll}
\lsptoprule
\label{bkm:Ref341895804}
\gll {\bluebold{dong}} {diam} {\bluebold{smua}}\\ %
& \textsc{3pl} & be.quiet & all\\
\lspbottomrule
\end{tabular}
\ea
\glt
‘\bluebold{they} were \bluebold{all} quiet’ \textstyleExampleSource{[080922-003-Cv.0095]}
\end{styleFreeTranslEngxvpt}

\section{\textsc{mod-n} structure: Adnominal possession}
\label{bkm:Ref343337600}
In Papuan Malay, adnominal possessive relations between two noun phrases are marked with the possessive ligature \textitbf{punya}; alternative realization of the ligature are reduced \textitbf{pu}, clitic \textitbf{=p}, or a zero morpheme.



Such possessive constructions have a \textsc{mod-n} constituent order which is opposite to the canonical \textsc{n-mod} structure. That is, the head nominal encoding the possessum (\textsc{possm}) takes the \textsc{n2} slot, following the possessive ligature (\textsc{lig}), whereas the modifier expressing the possessor (\textsc{possr}) takes the \textsc{n1} slot, such that ‘\textsc{possr}{}-\textsc{np} – \textsc{lig} – \textsc{possm}{}-\textsc{np}’. This is shown with the adnominal possessive construction in (0).
\end{styleBodyvxafter}

\begin{tabular}{llllllll}
\lsptoprule
\label{bkm:Ref341895805}
\gll { & \textsc{possr} & \textsc{lig} & \multicolumn{2}{l}{ \textsc{possm}} &  & }\\ %
& nanti & \bluebold{Hendro} & \bluebold{punya} & \bluebold{ade} & \bluebold{prempuang} & kawing & …\\
& very.soon & Hendro & \textsc{poss} & ySb & woman & marry.inofficially & \\
\lspbottomrule
\end{tabular}
\ea
\glt 
‘eventually \bluebold{Hendro’s younger sister} would marry …’ \textstyleExampleSource{[081006-028-CvEx.0007]}
\z


Syntactically, a variety of constituents can encode the possessor and the possessum, as shown in (0). The possessor slot can be taken by a lexical noun as in (0,) a personal pronoun as in (0,), a demonstrative as in (0), the interrogative \textitbf{siapa} ‘who’ as in (0), or a noun phrase as in (0). The possessum can be encoded by a lexical noun as in (0,,), a demonstrative as in (0,), the interrogative \textitbf{siapa} ‘who’ as in (0), or a noun phrase as in (0). Possessive noun phrases with a personal pronoun possessum are unattested.


\begin{styleExampleTitle}
Syntactic constituents of adnominal possessive constructions\footnote{\\
\\
\\
\\
\\
\\
\\
\\
\\
\\
\\
\\
\\
\\
\par Documentation: 080919-006-CvNP.0028, \textstyleExampleSource{080921-009-Cv.0020}, 080922-001a-CvPh.1123, 080925-004-Cv.0006, 081006-019-Cv.0002, 081025-006-Cv.0058, 081106-001-Ex.0007.\\
}
\end{styleExampleTitle}

\tablehead{ &  & \textsc{possr} & \textsc{lig} & \textsc{possm} & Adnominal possessive construction\\
}
\begin{tabular}{llllll}
\lsptoprule
\label{bkm:Ref336693926}
\gll {\label{bkm:Ref336695191}} {\textsc{n}} {\textitbf{pu}} {\textsc{dem}} {\textitbf{ade pu itu}}\\ %
&  &  &  &  & ‘younger sister’s (fish)’\\
& \label{bkm:Ref336695193} & \textsc{n} & \textitbf{pu} & \textsc{dem} & \textitbf{Fitri pu ini}\\
&  &  &  &  & ‘Fitri’s (belongings)’\\
& \label{bkm:Ref336695194} & \textsc{pro} & \textitbf{punya} & \textsc{n} & \textitbf{de punya bulu{\Tilde}bulu}\\
&  &  &  &  & ‘its (the dog’s) body hair’\\
& \label{bkm:Ref336695195} & \textsc{pro} & \textitbf{pu} & \textsc{int} & \textitbf{sa pu siapa}\\
&  &  &  &  & ‘who of my (relatives)’\\
& \label{bkm:Ref336695196} & \textsc{dem} & \textitbf{pu} & \textsc{n} & \textitbf{ini pu muka}\\
&  &  &  &  & ‘this (one’s) face’\\
& \label{bkm:Ref336695197} & \textsc{int} & \textitbf{pu} & \textsc{n} & \textitbf{siapa pu sandal}\\
&  &  &  &  & ‘whose sandals’\\
& \label{bkm:Ref336695198} & \textsc{np} & \textitbf{pu} & \textsc{np} & \textitbf{mama Klara pu ana prempuang}\\
&  &  &  &  & ‘mother Klara’s daughter’\\
\lspbottomrule
\end{tabular}

The examples in (0) and (0) show that adnominal possessive constructions designate possession of a definite possessum. Adnominal possession, including the non-canonical functions of the possessive marker, is discussed in detail in Chapter 9 (see also §11.4.1 for the uses of adnominal possessive constructions in two-argument existential clauses). Possession of an indefinite possessum is expressed with a two-argument existential clause or with a nominal clause; details are presented in §11.4.2 and §12.2, respectively.


\section{Apposition}
\label{bkm:Ref361845032}
In an apposition two “or more noun phrases” have “the same referent” and stand “in the same syntactical relation to the rest of the sentence” {\citep[5093]{Asher1994}}. Papuan Malay employs two types of appositional constructions, namely apposition of a noun with another noun or noun phrase, such that ‘\textsc{n} \textsc{np}’, and apposition of a personal pronoun with a noun or noun phrase, such that ‘\textsc{pro} \textsc{np}’. This section describes ‘\textsc{n} \textsc{np}’ appositions, while ‘\textsc{pro} \textsc{np}’ appositions are discussed in §6.1.6.



Papuan Malay ‘\textsc{n} \textsc{np}’ appositions are restrictive. That is, the apposited or juxtaposed noun phrase is needed for the appropriate identification of the referent encoded by the initial noun. There are no formal distinctions, though, between the ‘\textsc{n} \textsc{np}’ appositions discussed here and noun phrases with adnominally used nouns (\textsc{n1n2-np}), discussed in §8.2.2; the distinction is based on semantics.
\end{styleBodyvafter}


In the corpus, ‘\textsc{n} \textsc{np}’ appositions are rare, and in each case the initial noun encodes a kinship term, as in (0) and (0). The juxtaposed noun phrase \textitbf{ibu pendeta} ‘Ms. Pastor’ in (0) is appositional to the first noun \textitbf{kaka} ‘older sibling’. It provides information necessary for the identification of the referent. In (0), the appositional noun phrase \textitbf{ketua klasis} ‘church district chairperson’ serves as an identifying explanation for the reference of the initial noun \textitbf{bapa} ‘father’.
\end{styleBodyvxafter}

\begin{tabular}{lllllllllllll}
\lsptoprule
\label{bkm:Ref341895808}
\gll {bapa-ade} {\multicolumn{2}{l}{ini,}} {\multicolumn{2}{l}{kaka,}} {[\bluebold{kaka}]} {\multicolumn{3}{l}{[\bluebold{ibu}}} {\bluebold{pendeta}]} {dengang} {ini}\\ %
& uncle & \multicolumn{2}{l}{\textsc{d.prox}} & \multicolumn{2}{l}{oSb} & oSb & \multicolumn{3}{l}{woman} & pastor & with & \textsc{d.prox}\\
& \multicolumn{2}{l}{mama-tua,} & \multicolumn{2}{l}{nene} & \multicolumn{3}{l}{ini} & dong & \multicolumn{4}{l}{tertawa}\\
& \multicolumn{2}{l}{aunt} & \multicolumn{2}{l}{grandmother} & \multicolumn{3}{l}{\textsc{d.prox}} & \textsc{3pl} & \multicolumn{4}{l}{laugh}\\
\lspbottomrule
\end{tabular}
\ea
\glt 
‘uncle here (and) older sibling, \bluebold{older sibling, Ms. Pastor}, and, what’s-her-name, aunt, grandmother here, they were laughing’ \textstyleExampleSource{[080922-001a-CvPh.0824]}
\z

\begin{tabular}{llllllll}
\lsptoprule
\label{bkm:Ref341895809}
\gll {…} {bapa} {di} {dalam,} {[\bluebold{bapa}]} {[\bluebold{ketua}} {\bluebold{klasis}]}\\ %
&  & father & at & inside & father & chairperson & church.district\\
\lspbottomrule
\end{tabular}
\ea
\glt
‘[that’s what I’ve never told older sibling, what’s-his-name,] father (who’s) inside, \bluebold{father, the church district chairperson}’ \textstyleExampleSource{[080922-010a-CvNF.0104]}
\end{styleFreeTranslEngxvpt}

\section{Summary}
\label{bkm:Ref288639721}
The head of a noun phrase is typically a noun or personal pronoun. Further, although less common, demonstratives, locatives, or interrogatives can also function as heads. The canonical word order within the noun phrase is \textsc{head-modifier}. Depending on the syntactic properties of the adnominal constituents, though, a \textsc{modifier-head} order is also common. Attested in the corpus is the co-occurrence of up to three post-head modifiers. The possible constituents of the maximally extended noun phrase and the order of these constituents is summarized in the template in Table  ‎8 .3 (the items in parenthesis are optional).


\begin{stylecaption}
\label{bkm:Ref294359251}Table ‎8.\stepcounter{Table}{\theTable}:  Template of the maximally extended noun phrase
\end{stylecaption}

\begin{tabular}{llllll}
\lsptoprule
(\textsc{num}) & \textsc{head} & (\textsc{v}) & (\textsc{pro}) & (\textsc{dem}) & \arraybslash (\textsc{dem})\\
(\textsc{qt}) &  & (\textsc{n}) &  & (\textsc{loc}) & \\
(\textsc{possr-np}) &  & (\textsc{pp}) &  & (\textsc{int}) & \\
&  & (\textsc{rc}) &  & (\textsc{num}) & \\
&  &  &  & (\textsc{qt}) & \\
\lspbottomrule
\end{tabular}

The template in Table  ‎8 .3 shows that noun phrases with adnominally used verbs, nouns, personal pronouns, demonstratives, locatives, interrogatives, prepositional phrases, and relative clauses have an \textsc{n-mod} structure. Adnominal possessive constructions, by contrast, have a \textsc{mod-n} structure with the modifying possessor phrase occurring in pre-head position. Noun phrases with adnominally used numerals and quantifiers have an \textsc{n-mod} or \textsc{mod-n} structure depending on the semantics of the phrasal structure. Adnominally used demonstratives can occur in two slots. They can take the same slot as adnominally used locatives, interrogatives, numerals, or quantifiers, and in addition they can occur at the right periphery of the noun phrase where they have scope over the entire noun phrase.
\end{styleBodyaftervbefore}


Papuan Malay uses two types of appositional constructions: those consisting of a noun followed by another noun or noun phrase, and those consisting of a personal pronoun followed by a noun or noun phrase, the latter being discussed in §6.1.6. Appositions with juxtaposed nouns or noun phrases are restrictive.
\end{styleBodyvafter}

%\setcounter{page}{1}\chapter[Adnominal possessive relations]{Adnominal possessive relations}
\label{bkm:Ref358725848}
In Papuan Malay, adnominal possessive relations between two noun phrases are encoded with the possessive ligature \textitbf{punya} ‘\textsc{poss}’. The noun phrase preceding the ligature (\textsc{lig}) designates the possessor (\textsc{possr}), while the noun phrase following it expresses the possessum (\textsc{possm}), such that ‘\textsc{possr}{}-\textsc{np} – \textitbf{punya} ‘\textsc{poss}’ – \textsc{possm}{}-\textsc{np}’.



The main function of adnominal possessive constructions is to denote possession of a definite possessum. In addition, \textitbf{punya} ‘\textsc{poss}’ serves other functions in ‘\textsc{possr}{}-\textsc{np} – \textitbf{punya} ‘\textsc{poss}’ – \textsc{possm}{}-\textsc{np}’ constructions. It is employed to mark and emphasize locational, temporal, or associative relations, to indicate beneficiary relations, or to signal speaker attitudes and evaluations. Besides, the ligature is also used in reflexive expressions. (Possession of an indefinite possessum is not expressed with an adnominal possessive construction, but with a two-argument existential clause or a nominal clause; details are presented in §11.4.2 and §12.2, respectively.)
\end{styleBodyvafter}


The three constituents of an adnominal possessive construction have different realizations, as illustrated in Table  ‎9 .1. The possessive marker can be realized with long \textitbf{punya}, reduced \textitbf{pu}, clitic \textitbf{=p}, or a zero morpheme. The noun phrases expressing the possessor and possessum can belong to different syntactic categories. The most common constituents are lexical nouns and noun phrases. Demonstratives can also take either slot. Also very common are personal pronoun possessors. In non-canonical possessive constructions, the possessor and possessum slots can also be filled by verbs. In addition, mid-range quantifiers, temporal adverbs and prepositional phrases can take the possessum slot. In both canonical and non-canonical possessive constructions, the possessum can be omitted.
\end{styleBodyvvafter}

\begin{stylecaption}
\label{bkm:Ref340663803}Table ‎9.\stepcounter{Table}{\theTable}:  Adnominal possessive constructions
\end{stylecaption}

\begin{tabular}{lll}
\lsptoprule

 \textsc{possr} & \textsc{lig} & \arraybslash \textsc{possm}\\
Lexical nouns & \textitbf{punya} & Lexical nouns\\
Noun phrases & \textitbf{pu} & Noun phrases\\
Demonstratives & \textitbf{pu} & Demonstratives\\
Personal pronouns & \textitbf{=p} & Verbs\\
Verbs & ${\varnothing}$ & Quantifiers\\
&  & Adverbs\\
&  & Prepositional phrases\\
&  & ${\varnothing}$\\
\lspbottomrule
\end{tabular}

Semantically, the possessor and the possessum can designate human, nonhuman animate, or inanimate referents. Overall, adnominal possessive constructions do not make a distinction between alienable and inalienable possession, with one exception. Possessive constructions with the omitted possessive marker signal inalienable possession of body parts or kinship relations.
\end{styleBodyaftervbefore}


In the following sections, adnominal possessive constructions are discussed in more detail. The possessive marker \textitbf{punya} ‘\textsc{poss}’ with its different realizations is examined in §9.1. The different realizations of the possessor and possessum noun phrases are described in §9.2. Non-canonical possessive constructions are discussed in §9.3. The main points of this chapter are summarized in §9.4. (As for the uses of adnominal possessive constructions in two-argument existential clauses, see §11.4.1.)
\end{styleBodyvxvafter}

\section{Possessive marker \textitbf{punya} ‘\textsc{poss}’}
\label{bkm:Ref340732509}
The possessive marker \textitbf{punya} ‘\textsc{poss}’ is related to the full bivalent verb \textitbf{punya} ‘have’ which is still used synchronically in two-argument clauses to predicate possession of an indefinite possessum. In such clauses, the possessor is encoded by the grammatical subject (S) while the indefinite possessum is the direct object (O) of the verb (V) \textitbf{punya} ‘have’. This is illustrated in (0): the possessor \textitbf{sa} ‘1\textsc{sg}’ is the grammatical subject while the possessum \textitbf{rencana} ‘thought’ is the direct object of \textitbf{punya} ‘have’. Overall, however, verbal clauses with \textitbf{punya} ‘have’ are rather rare. Instead, speakers typically express possession of an indefinite possessum with a two-argument existential clause with \textitbf{ada} ‘exist’. This is demonstrated in (0): the possessor \textitbf{sa} ‘\textsc{1sg}’ is the subject while the indefinite possessum \textitbf{ana} ‘child’ is the direct object of existential \textitbf{ada} ‘exist’. (This type of two-argument existential clause is discussed in detail in §11.4.2.)


\begin{styleExampleTitle}
Predicative reading of \textitbf{punya} ‘have’ constructions
\end{styleExampleTitle}

\begin{tabular}{llllll}
\lsptoprule
\label{bkm:Ref439954837}\label{bkm:Ref339635919}
\gll { & \textsc{s} &  & \textsc{v} & \arraybslash \textsc{o}}\\ %
& malam & \bluebold{saya} & suda & \bluebold{punya} & \bluebold{rencana}\\
& night & \textsc{1sg} & already & have & plan\\
\lspbottomrule
\end{tabular}
\ea
\glt 
‘the night (before I go hunting) \bluebold{I} already \bluebold{have a plan}’ \textstyleExampleSource{[080919-004-NP.0002]}
\z

\begin{styleExampleTitle}
Two-argument existential clause denoting possession
\end{styleExampleTitle}

\begin{tabular}{lllllllllll}
\lsptoprule
\label{bkm:Ref364856031}
\gll {\textsc{s}} {\textsc{v}} {\textsc{o}} {} {} {} {} {} {} {}\\ %
& \bluebold{sa} & \bluebold{ada} & \bluebold{ana}, & jadi & sa & kasi & untuk & sa & pu & sodara\\
& 1\textsc{sg} & exist & child & so & \textsc{1sg} & give & for & \textsc{1sg} & \textsc{poss} & sibling\\
\lspbottomrule
\end{tabular}
\ea
\glt 
‘\bluebold{I have children}, so I gave (one) to my relative’ \textstyleExampleSource{[081006-024-CvEx.0010]}
\z


The most common function of Papuan Malay \textitbf{punya} ‘\textsc{poss}’ is that of a ligature in adnominal referential possessive constructions, that is, possessive constructions with definite referents. Such constructions have the syntactic structure ‘\textsc{possr-np} \textitbf{punya} \textsc{possm-np}’. As shown in (0), this type of possessive construction contrasts with the verbal constructions in (0) and (0): the possessive relation is not encoded by a two-argument clause but in a single construction which consists of two noun phrases, which in turn functions as an argument in a clause. Hence, \textitbf{Yosina} in (0) is not a grammatical subject but the possessor. Likewise, \textitbf{swara} ‘voice’ is not the direct object of a verbal clause, but a definite possessum. The entire possessive construction in (0) functions as the direct object of the bivalent verb \textitbf{dengar} ‘hear’. The contrastive examples in (0) and (0) also illustrate the distinctions between possession of an indefinite and a definite possessum, respectively.


\begin{styleExampleTitle}
Adnominal reading of \textitbf{punya} ‘\textsc{poss}’ constructions
\end{styleExampleTitle}

\begin{tabular}{llllllll}
\lsptoprule
\label{bkm:Ref339635920}
\gll { &  &  &  & \textsc{possr-np} & \textsc{lig} & \arraybslash \textsc{possm-np}}\\ %
& bapa & kwatir & tertarik & dengar & \bluebold{Yosina} & \bluebold{punya} & \bluebold{swara}\\
& father & afraid & be.pulled & hear & Yosina & poss & voice\\
\lspbottomrule
\end{tabular}
\ea
\glt 
‘I (‘father’) was worried (and) longed to hear \bluebold{your (‘Yosina’s’) voice}’ \textstyleExampleSource{[080922-001a-CvPh.0205]}
\z


Sometimes, however, it is ambiguous whether the \textitbf{punya} construction should receive a predicative reading as in (0) or adnominal interpretation as in (0), as there is no difference in intonation or stress between both utterances.


\begin{styleExampleTitle}
Predicative and adnominal readings of \textitbf{punya} ‘have/\textsc{poss}’ constructions
\end{styleExampleTitle}

\begin{tabular}{llllll}
\lsptoprule
\label{bkm:Ref289701422}
\gll {\label{bkm:Ref320374399}} {[de]} {[\bluebold{punya}]} {[piring} {kusus]}\\ %
&  & \textsc{3sg} & have/\textsc{poss} & plate & be.special\\
\lspbottomrule
\end{tabular}
\begin{styleFreeTranslIndentiicmEng}
Predicative reading: ‘he/she \bluebold{has} special plates’ \textstyleExampleSource{[081006-029-CvEx.0016]}
\end{styleFreeTranslIndentiicmEng}

\begin{tabular}{llllll} & \label{bkm:Ref320374401} & [\bluebold{de} & \bluebold{punya} & \bluebold{piring}] & [kusus]\\
\lsptoprule
&  & \textsc{3sg} & have/\textsc{poss} & plate & be.special\\
\lspbottomrule
\end{tabular}
\begin{styleFreeTranslIndentiicmEng}
Adnominal reading: ‘\bluebold{his/her plates} are special’ \textstyleExampleSource{[081006-029-CvEx.0016]}
\end{styleFreeTranslIndentiicmEng}


In adnominal possessive constructions, the ligature \textitbf{punya} ‘\textsc{poss}’ has four different realizations which are discussed in the following sections: long \textitbf{punya} ‘\textsc{poss}’ and its reduced form \textitbf{pu} in §9.1.1, the clitic \textitbf{=p} ‘\textsc{poss}’ in §9.1.2, and elision in §9.1.3. In §9.1.4, a possible grammaticalization of the possessive marker is examined.
\end{styleBodyxvafter}

\subsection{\textsc{possr-np} \textitbf{punya}/\textitbf{pu} \textsc{possm-np}}
\label{bkm:Ref340578108}
In adnominal possessive constructions, the possessive marker is most commonly realized with the long form \textitbf{punya} ‘\textsc{poss}’ or the reduced monosyllabic form \textitbf{pu} ‘\textsc{poss}’. This reduction is independent of the syntactic or semantic properties of the possessor or possessum, as illustrated in (0).



Both ligature forms occur with possessors encoded by lexical nouns as in (0{}-), by personal pronouns as in (0,), or by noun phrases as in (0{}-). With either ligature form, the possessor can denote a human referent as in (0{}-, {}-), a nonhuman animate referent as in (0,, ), or an inanimate referent as in (0,). Likewise, the reduction is independent of the possessum’s properties. Both markers occur with possessa encoded by nouns as in (0,,,,), by demonstratives as in (0,), or by noun phrases as in (0,,{}-). With either marker, the possessum can express an inalienably possessed referent as in (0,,,) or an alienably possessed referent as in (0{}-,,,).
\end{styleBodyvvafter}

\begin{styleExampleNumCard}
\label{bkm:Ref282076033}(\stepcounter{}{\the})  Adnominal possessive constructions with the long possessive marker \textitbf{punya} ‘\textsc{poss}’ and short \textitbf{pu} ‘\textsc{poss}’\footnote{\\
\\
\\
\\
\\
\\
\\
\\
\\
\\
\\
\\
\\
\\
\par Documentation: 080919-004-NP.0013, 080919-006-CvNP.0028, 080922-001a-CvPh.0141, 081006-019-Cv.0002, 081006-022-CvEx.0029, 081006-022-CvEx.0084, 081110-002-Cv.0075, 081006-024-CvEx.0016, 081011-007-Cv.0003, 081025-006-Cv.0021, 081025-006-Cv.0058, 081106-001-Ex.0007.\\
}
\end{styleExampleNumCard}

\tablehead{ & \textsc{possr} & \textsc{lig} & \textsc{possm} & \arraybslash Possessive construction\\
}
\begin{tabular}{lllll}
\lsptoprule
\label{bkm:Ref282082236} & \textsc{n} (\textsc{hum}) & \textitbf{punya} & \textsc{np} (\textsc{inal}) & \textitbf{mama punya ade laki{\Tilde}laki}\\
&  &  &  & mother \textsc{poss} ySb \textsc{rdp}{\Tilde}husband\\
&  &  &  & ‘mother’s younger brother’\\
\label{bkm:Ref282082237} & \textsc{n} (\textsc{hum}) & \textitbf{pu} & \textsc{n} (\textsc{inal}) & \textitbf{bapa pu mata}\\
&  &  &  & father \textsc{poss} eye\\
&  &  &  & ‘father’s eyes’\\
\label{bkm:Ref282082252} & \textsc{n} (\textsc{hum}) & \textitbf{punya} & \textsc{dem} (\textsc{al}) & \textitbf{Fitri pu ini}\\
&  &  &  & Fitri \textsc{poss} \textsc{d.prox}\\
&  &  &  & ‘Fitri’s (belongings)’\\
\label{bkm:Ref282082253} & \textsc{n} (\textsc{hum}) & \textitbf{pu} & \textsc{dem} (\textsc{al}) & \textitbf{ade pu itu}\\
&  &  &  & ySb \textsc{poss} \textsc{d.dist}\\
&  &  &  & ‘younger sister’s (fish)’\\
\label{bkm:Ref282082239} & \textsc{n} (\textsc{an}) & \textitbf{punya} & \textsc{n} (\textsc{al}) & \textitbf{setang punya kwasa}\\
&  &  &  & evil.spirit \textsc{poss} power\\
&  &  &  & ‘force of an evil spirit’\\
\label{bkm:Ref282082241} & \textsc{n} (\textsc{an}) & \textitbf{pu} & \textsc{n} (\textsc{al}) & \textitbf{setang pu pake{\Tilde}pake}\\
&  &  &  & evil.spirit \textsc{poss} black.magic\\
&  &  &  & ‘an evil spirit’s black magic’\\
\label{bkm:Ref282082243} & \textsc{n} (\textsc{inan}) & \textitbf{pu} & \textsc{np} (\textsc{al}) & \textitbf{LNG pu terpol itu}\\
&  &  &  & LNG \textsc{poss} container \textsc{d.dist}\\
&  &  &  & ‘metal jerry can’\footnotemark{}\\
\label{bkm:Ref282082244} & \textsc{pro} (\textsc{an}) & \textitbf{punya} & \textsc{n} (\textsc{inal}) & \textitbf{de punya bulu{\Tilde}bulu}\\
&  &  &  & \textsc{3sg} \textsc{poss} body.hair\\
&  &  &  & ‘its (the dog’s) body hair’\\
\label{bkm:Ref282082249} & \textsc{pro} (\textsc{hum}) & \textitbf{pu} & \textsc{np} (\textsc{al}) & \textitbf{de pu sikat gigi deng odol}\\
&  &  &  & \textsc{3sg} \textsc{poss} toothbrush with toothpaste\\
&  &  &  & ‘her toothbrush and toothpaste’\\
\label{bkm:Ref282082250} & \textsc{np} (\textsc{hum}) & \textitbf{punya} & \textsc{np} (\textsc{al}) & \textitbf{orang Isirawa punya, apa, cara kawing}\\
&  &  &  & person Isirawa \textsc{poss} what manner marry\\
&  &  &  & ‘the Isirawa’s, what-is-it, way of marrying’\\
\label{bkm:Ref282082251} & \textsc{np} (\textsc{hum}) & \textitbf{pu} & \textsc{np} (\textsc{inal}) & \textitbf{mama Klara pu ana prempuang}\\
&  &  &  & mother Klara \textsc{poss} child woman\\
&  &  &  & ‘mother Klara’s daughter’\\
\label{bkm:Ref282082242} & \textsc{np} (\textsc{inan}) & \textitbf{punya} & \textsc{n} (\textsc{al}) & \textitbf{kebung ini punya hasil}\\
&  &  &  & garden \textsc{d.prox} \textsc{poss} product\\
&  &  &  & ‘this garden’s products’\\
\lspbottomrule
\end{tabular}
\footnotetext{\\
\\
\\
\\
\\
\\
\\
\\
\\
\\
\\
\\
\\
\\
The proper noun \textitbf{LNG} has developed from the noun phrase ‘Liquefied Natural Gas’.\\
}

With respect to the possessive marking of personal pronouns, there are no prosodic restrictions on the use of the two possessive marker forms: either can occur with the long and the short pronoun forms, as illustrated in Table  ‎9 .2.\footnote{\\
\\
\\
\\
\\
\\
\\
\\
\\
\\
\\
\\
\\
\\

‘\bluebold{our/their house} is over there’ [Elicited BR111020-001.002-003]\par \\
} (The pronoun \textitbf{ko} ‘2\textsc{sg}’ does not have a short form.) (Pronouns are discussed in details in Chapter 6.)


\begin{stylecaption}
\label{bkm:Ref281646465}Table ‎9.\stepcounter{Table}{\theTable}:  Possessive marking of personal pronouns\footnote{\\
\\
\\
\\
\\
\\
\\
\\
\\
\\
\\
\\
\\
\\
\par Documentation: 080916-001-CvNP.0006, 080917-008-NP.0166, 080919-004-NP.0018, 080919-004-NP.0053, 080919-004-NP.0071, 080919-004-NP.0079, 080922-001a-CvPh.0834, 080922-002-Cv.0006, 080922-005-CvEx.0004, 080922-010a-NF.0002, 080922-010a-NF.0288, 081006-022-CvEx.0043, 081006-022-CvEx.0047, 081006-029-CvEx.0015, 081011-011-Cv.0055, 081011-011-Cv.0057, 081015-005-NP.0011, 081015-005-NP.0023, 081110-001-Cv.0026, 081110-002-Cv.0015, 081110-002-Cv.0018, 081110-003-Cv.0023, 081110-008-CvHt.0058, 081110-008-CvHt.0101, 081115-001a-Cv.0275, 081115-001b-Cv.0026, 081115-001b-Cv.0026, 081115-001b-Cv.0057.\\
}
\end{stylecaption}

\tablehead{
 Possessive construction & \multicolumn{2}{l}{ Glosses} & \arraybslash Free translation\\
}
\begin{tabular}{llll}
\lsptoprule
\multicolumn{4}{l}{Possessive marking with \textitbf{punya} ‘\textsc{poss}’}\\
\multicolumn{4}{l}{Long personal pronoun form – \textitbf{punya} ‘\textsc{poss}’}\\
\multicolumn{2}{l}{\textitbf{saya punya sabit}} & 1\textsc{sg} \textsc{poss} sickle & ‘my sickle’\\
\multicolumn{2}{l}{\textitbf{ko punya barang}} & 2\textsc{sg} \textsc{poss} stuff & ‘your belongings’\\
\multicolumn{2}{l}{\textitbf{dia punya nama}} & 3\textsc{sg} \textsc{poss} name & ‘his name’\\
\multicolumn{2}{l}{\textitbf{kitorang punya kekurangang}} & 1\textsc{pl} \textsc{poss} shortcoming & ‘our shortcomings’\\
\multicolumn{2}{l}{\textitbf{kitong punya muka}} & 1\textsc{pl} \textsc{poss} face & ‘our faces’\\
\multicolumn{2}{l}{\textitbf{kita punya bapa}} & 1\textsc{pl} \textsc{poss} father & ‘our father’\\
\multicolumn{2}{l}{\textitbf{kamu punya otak}} & 2\textsc{pl} \textsc{poss} brain & ‘your brains’\\
\multicolumn{2}{l}{\textitbf{dorang punya kampung}} & 3\textsc{pl} \textsc{poss} village & ‘their village’\\
\multicolumn{4}{l}{Short personal pronoun form – \textitbf{punya} ‘\textsc{poss}’}\\
\textitbf{sa punya nokeng} & \multicolumn{2}{l}{1\textsc{sg} \textsc{poss} stringbag} & ‘my stringbag’\\
\textitbf{de punya swami} & \multicolumn{2}{l}{3\textsc{sg} \textsc{poss} husband} & ‘her husband’\\
\textitbf{torang punya orang-tua} & \multicolumn{2}{l}{1\textsc{pl} \textsc{poss} parent} & ‘our parents’\\
\textitbf{tong punya ipar} & \multicolumn{2}{l}{1\textsc{pl} \textsc{poss} sibling-in-law} & ‘our sister in-law’\\
\textitbf{ta punya kampung} & \multicolumn{2}{l}{1\textsc{pl} \textsc{poss} village} & ‘our village’\\
\textitbf{kam punya nasip} & \multicolumn{2}{l}{2\textsc{pl} \textsc{poss} destiny} & ‘your destinies’\\
\textitbf{dong punya ruma} & \multicolumn{2}{l}{3\textsc{pl} \textsc{poss} house} & ‘their house’\\
\multicolumn{4}{l}{Possessive marking with \textitbf{pu} ‘\textsc{poss}’}\\
\multicolumn{4}{l}{Long personal pronoun form – \textitbf{pu} ‘\textsc{poss}’}\\
\textitbf{saya pu hasil kebung} & \multicolumn{2}{l}{1\textsc{sg} \textsc{poss} product garden} & ‘my garden products’\\
\textitbf{ko pu kampung} & \multicolumn{2}{l}{2\textsc{sg} \textsc{poss} village} & ‘your village’\\
\textitbf{dia pu maytua} & \multicolumn{2}{l}{3\textsc{sg} \textsc{poss} wife} & ‘his wife’\\
\textitbf{kitorang pu keadaang} & \multicolumn{2}{l}{1\textsc{pl} \textsc{poss} condition} & ‘our condition’\\
\textitbf{kitong pu kawang} & \multicolumn{2}{l}{1\textsc{pl} \textsc{poss} friend} & ‘our friend’\\
\textitbf{kita pu adat} & \multicolumn{2}{l}{1\textsc{pl} \textsc{poss} customs} & ‘our customs’\\
\textitbf{kamu pu cara hidup} & \multicolumn{2}{l}{2\textsc{pl} \textsc{poss} manner live} & ‘your ways of life’\\
\multicolumn{4}{l}{Short personal pronoun form – \textitbf{pu} ‘\textsc{poss}’}\\
\textitbf{sa pu motor} & \multicolumn{2}{l}{1\textsc{sg} \textsc{poss} motorbike} & ‘my motorbike’\\
\textitbf{de pu bahu} & \multicolumn{2}{l}{3\textsc{sg} \textsc{poss} shoulder} & ‘her shoulder’\\
\textitbf{tong pu pakeang} & \multicolumn{2}{l}{1\textsc{pl} \textsc{poss} clothing} & ‘our clothing’\\
\textitbf{ta pu orang-tua} & \multicolumn{2}{l}{1\textsc{pl} \textsc{poss} parent} & ‘our parents’\\
\textitbf{kam pu sabung} & \multicolumn{2}{l}{2\textsc{pl} \textsc{poss} soap} & ‘their soap’\\
\textitbf{dong pu jaring} & \multicolumn{2}{l}{3\textsc{pl} \textsc{poss} net} & ‘their net’\\
\lspbottomrule
\end{tabular}

These examples show that the reduction of the disyllabic form \textitbf{punya} ‘\textsc{poss}’ to monosyllabic \textitbf{pu} ‘\textsc{poss}’ does not interact with the long versus reduced shape of the personal pronouns. These findings contrast with those of {\citet{Donohue2003}} who found that the long pronoun forms may not co-occur with the reduced possessive marker \textitbf{pu} ‘\textsc{poss}’ (for more details see {Donohue 2003: 24–25}).
\end{styleBodyaftervbefore}


Very occasionally, the reduced ligature takes on the form /\textstyleChCharisSIL{pum}/, /\textstyleChCharisSIL{pun}/, or /\textstyleChCharisSIL{puŋ}/ ‘\textsc{poss}’. This variation is usually due to assimilation to the word-initial segment of the following possessum, as illustrated in Table  ‎9 .3. That is, speakers realize short \textitbf{pu} ‘\textsc{poss}’ with a word-final nasal which receives its place features from the onset segment of the following prosodic word; when the following word has a vowel as onset, the nasal is typically realized as velar [\textstyleChCharisSIL{ŋ}]. (For more details on nasal place assimilation see §2.2.1.)


\begin{stylecaption}
\label{bkm:Ref350591650}Table ‎9.\stepcounter{Table}{\theTable}:  Assimilation of short \textitbf{pu} ‘\textsc{poss}’
\end{stylecaption}

\tablehead{
 Item & Orthogr. & \arraybslash Gloss\\
}
\begin{tabular}{lll}
\lsptoprule
/\textstyleChCharisSIL{dɛ pu}\textstyleChCharisSILBlueBold{m}\textstyleChCharisSIL{ }\textstyleChCharisSILBlueBold{b}\textstyleChCharisSIL{apa}/ & \textitbf{de pu bapa} & ‘his/her father’\\
/\textstyleChCharisSIL{dɛ pu}\textstyleChCharisSILBlueBold{n}\textstyleChCharisSIL{ }\textstyleChCharisSILBlueBold{t}\textstyleChCharisSIL{ɛman{\Tilde}tɛmaŋ}/ & \textitbf{de pu temang{\Tilde}temang} & ‘his/her friends’\\
/\textstyleChCharisSIL{sa pu}\textstyleChCharisSILBlueBold{ŋ}\textstyleChCharisSIL{ }\textstyleChCharisSILBlueBold{k}\textstyleChCharisSIL{aka}/ & \textitbf{sa pu kaka} & ‘my older sibling’\\
/\textstyleChCharisSIL{d}ɔ\textstyleChCharisSIL{m pu}\textstyleChCharisSILBlueBold{ŋ} \textstyleChCharisSILBlueBold{a}srama/ & \textitbf{dong pu asrama} & ‘their dormitory’\\
\lspbottomrule
\end{tabular}

In a few cases, however, the reduced ligature takes on the form /\textstyleChCharisSIL{puŋ}/ regardless of the form of the following segment, as illustrated in (0) to (0).


\begin{tabular}{lllll}
\lsptoprule
\label{bkm:Ref339635953}
\gll {ada} {sa} {/\textstyleChCharisSILBlueBold{puŋ}/} {\bluebold{d}usung}\\ %
& exist & 1\textsc{sg} & \textsc{poss} & garden\\
\lspbottomrule
\end{tabular}
\ea
\glt 
‘(over there) is \bluebold{my} garden’ \textstyleExampleSource{[081110-008-CvNP.0009]}
\z

\begin{tabular}{llllllll}
\lsptoprule
(\stepcounter{}{\the}) & dɔng & /\textstyleChCharisSILBlueBold{puŋ}/ & \bluebold{p}\textstyleChCharisSIL{ɛ}s\textstyleChCharisSIL{ɛ}rta & juga & macang & tra & …\\
& \textsc{3pl} & \textsc{poss} & participant & also & variety & \textsc{neg} & \\
\lspbottomrule
\end{tabular}
\ea
\glt 
‘\bluebold{their} participants also, like (they) didn’t …’ \textstyleExampleSource{[081025-009a-Cv.0132]}
\z

\begin{tabular}{llllllll}
\lsptoprule
\label{bkm:Ref339635955}
\gll {…} {tɔng} {/\textstyleChCharisSILBlueBold{puŋ}/} {\bluebold{c}ara} {makang} {babi} {juga}\\ %
&  & \textsc{1pl} & \textsc{poss} & manner & eat & pig & also\\
\lspbottomrule
\end{tabular}
\ea
\glt
‘[our way of eating is just like the Toraja one,] \bluebold{our} way of eating pigs also’ \textstyleExampleSource{[081014-017-CvPr.0053]}
\end{styleFreeTranslEngxvpt}

\subsection{\textsc{possr-np} \textitbf{=p} \textsc{possm-np}}
\label{bkm:Ref340578109}
The possessive marker can be reduced further to \textitbf{=p} ‘\textsc{poss}’, if the possessor noun phrase ends in a vowel, as in (0) to (0). In this case, the marker is cliticized to the possessor. In this type of reduced possessive construction, the possessor is almost always a singular personal pronoun, such as short first person \textitbf{sa} ‘\textsc{1sg}’, second person \textitbf{ko} ‘\textsc{2sg}’ in (0), or short third person \textitbf{de} ‘\textsc{3sg}’ as in (0). The possessor may, however, also be expressed by a noun as in (0), although in the corpus this example is the only one attested. Again, the same construction is used for alienable and inalienable possession.
\end{styleBodyxafter}

\begin{tabular}{llllllllll}
\lsptoprule
\label{bkm:Ref339635956}
\gll {sa} {bilang,} {i,} {\bluebold{sa=p}} {\bluebold{kaka},} {de} {bilang} {\bluebold{ko=p}} {\bluebold{kaka}}\\ %
& \textsc{1sg} & say & ugh! & \textsc{1sg=poss} & oSb & \textsc{3sg} & say & \textsc{2sg=poss} & oSb\\
\lspbottomrule
\end{tabular}
\ea
\glt 
‘I said, ‘ugh!, (that’s) \bluebold{my older sister}’, she said, ‘\bluebold{your older sister}?’’ \textstyleExampleSource{[080919-006-CvNP.0026]}
\z

\begin{tabular}{lllllllll}
\lsptoprule
\label{bkm:Ref340576039}
\gll {de} {timbul} {\bluebold{de=p}} {\bluebold{cucu}} {tanya} {dia,} {tete} {knapa}\\ %
& \textsc{3sg} & emerge & \textsc{3sg=poss} & grandchild & ask & \textsc{3sg} & grandfather & why\\
\lspbottomrule
\end{tabular}
\ea
\glt 
‘(when) he (grandfather) emerged, his grandchild asked him, ‘grandfather, what happened?’’ \textstyleExampleSource{[081109-005-JR.0009]}
\z

\begin{tabular}{lllllll}
\lsptoprule
\label{bkm:Ref339635958}
\gll {Fredi} {de} {pu} {\bluebold{ade=p}} {\bluebold{motor}} {…}\\ %
& Fredi & \textsc{3sg} & \textsc{poss} & ySb\textsc{=poss} & motorbike & \\
\lspbottomrule
\end{tabular}
\ea
\glt
‘Fredi’s \bluebold{younger brother’s motorbike} …’ \textstyleExampleSource{[081002-001-CvNP.0058]}
\end{styleFreeTranslEngxvpt}

\subsection{\textsc{possr-np} ${\varnothing}$ \textsc{possm-np}}
\label{bkm:Ref340578110}
The possessive marker can also be elided, as illustrated in (0) to (0). The elision is limited, however, to certain semantic kinds of possession. Attested are inalienable possession of body parts, as in (0) and (0), and kinship relations, as in (0) and (0). Most commonly, the possessor is human as in (0) to (0), but it may also be animate nonhuman as in (0).



In \textsc{possr-possm} constructions, the possessor is usually encoded by a short personal pronoun form, as in (0) to (0). Much less often, the possessor is expressed with a lexical noun such as \textitbf{bapa} ‘father’ in (0). Also rather infrequently, the possessor is expressed by a noun phrase such as \textitbf{pace de} ‘the man’ in (0), where adnominally used \textitbf{de} ‘\textsc{3sg}’ modifies \textitbf{pace} ‘man’ (for details on the adnominal uses of the personal pronouns, see §6.2).
\end{styleBodyvxafter}

\begin{tabular}{lllllll}
\lsptoprule
\label{bkm:Ref339635962}
\gll {adu,} {\bluebold{bapa}} {\textstyleChBold{${\varnothing}$}} {\bluebold{mulut}} {jahat} {skali}\\ %
& oh.no! & father &  & mouth & be.bad & very\\
\lspbottomrule
\end{tabular}
\ea
\glt 
‘oh no, \bluebold{father’s language }is very bad’ (Lit. ‘\bluebold{father’s mouth}’) \textstyleExampleSource{[080923-008-Cv.0019]}
\z

\begin{tabular}{llllllll}
\lsptoprule
\label{bkm:Ref339635959}
\gll {\bluebold{pace}} {\bluebold{de}} {\textstyleChBold{${\varnothing}$}} {\bluebold{tangang}} {kluar} {ke} {samping}\\ %
& man & \textsc{3sg} &  & arm & go.out & to & side\\
\lspbottomrule
\end{tabular}
\ea
\glt 
[About an accident:] ‘\bluebold{the man’s hand} stuck out sideways’ \textstyleExampleSource{[081108-001-JR.0003]}
\z

\begin{tabular}{lllllll}
\lsptoprule
\label{bkm:Ref339635960}
\gll {\bluebold{de}} {\bluebold{${\varnothing}$}} {\bluebold{mama}} {\bluebold{ini}} {ke} {atas}\\ %
& \textsc{3sg} &  & see & \textsc{3sg} & \textsc{poss} & wife\\
\lspbottomrule
\end{tabular}
\ea
\glt 
‘\bluebold{his mother here} (went) up (there)’ \textstyleExampleSource{[080923-001-CvNP.0019]}
\z

\begin{tabular}{llllllllllll}
\lsptoprule
\label{bkm:Ref339635961}
\gll {dia} {liat} {dia} {pu} {maytua} {…} {ah,} {\bluebold{sa}} {\bluebold{${\varnothing}$}} {\bluebold{maytua}} {cantik}\\ %
& \textsc{3sg} & see & \textsc{3sg} & \textsc{poss} & wife &  & ah! & \textsc{1sg} &  & wife & be.beautiful\\
\lspbottomrule
\end{tabular}
\ea
\glt 
‘he saw his wife … ‘ah, my wife is beautiful’’ \textstyleExampleSource{[080922-010a-CvNF.0020]}
\z

\begin{tabular}{llllllllllll}
\lsptoprule
\label{bkm:Ref340575013}
\gll {\multicolumn{3}{l}{langsung}} {\multicolumn{2}{l}{potong}} {dia} {\multicolumn{2}{l}{buang}} {\multicolumn{3}{l}{tali-prutnya}}\\ %
& \multicolumn{3}{l}{immediately} & \multicolumn{2}{l}{cut} & \textsc{3sg} & \multicolumn{2}{l}{throw(.away)} & \multicolumn{3}{l}{intestines:\textsc{3possr}}\\
& \bluebold{de} & \bluebold{${\varnothing}$} & \multicolumn{2}{l}{\bluebold{tali-prut}} & \multicolumn{3}{l}{buang,} & \multicolumn{2}{l}{tinggal} & isi & saja\\
& \textsc{3sg} &  & \multicolumn{2}{l}{intestines} & \multicolumn{3}{l}{throw(.away)} & \multicolumn{2}{l}{stay} & contents & just\\
\lspbottomrule
\end{tabular}
\ea
\glt 
[About killing dogs:] ‘cut him up at once (and) throw away the intestines, (after having) thrown away \bluebold{his intestines}, just the meat remains’ \textstyleExampleSource{[081106-001-CvPr.0005]}
\z


Contrary to the possessive constructions presented in §9.1.1 and §9.1.2, the data presented in (0) to (0) shows that Papuan Malay also has the option to signal inalienable possession by omitting the possessive marker.



This alienable versus inalienable distinction is also found in other Austronesian languages of the Papuan contact zone, whereas it is not found in Western Malayo-Polynesian languages. As in other Austronesian and Papuan languages of this contact zone {(Klamer et al. 2008: 116)}, it is body parts and kinship terms that can be inalienably possessed.\footnote{\\
\\
\\
\\
\\
\\
\\
\\
\\
\\
\\
\\
\\
\\
\par {\citet[116]{KlamerEtAl2008}} note that this “innovation must have occurred prior to the population of Oceania”, a conclusion that is based on {Ross’s (2001)} hypothesis that it “is also probable that the formal distinction between alienable and inalienable possession entered Proto-Oceanic or an immediate precursor through Papuan contact”.\\
}
\end{styleBodyvxvafter}

\subsection{Grammaticalization of \textitbf{punya} ‘\textsc{poss}’}
\label{bkm:Ref340578111}
In §9.1.1 to §9.1.3, the reduction of possessive marker \textitbf{punya} ‘\textsc{poss}’ to its monosyllabic variants \textitbf{pu} or \textitbf{=p} ‘\textsc{poss}’ and its omission in \textsc{possr-possm} constructions was described.



One explanation for this reduction would be to consider it as the result of a grammaticalization process. As {\citet[719]{Bybee2006}} observes, the phonetic reduction of high-frequency words “can lead to the establishment of a new construction with its own categories” and “the grammaticization of the new construction”. One could argue that {Bybee’s (2006: 719)} observation also applies to the high-frequency morpheme \textitbf{punya} with its variable status between a full verb ‘have’, a clitic possessive marker, and a zero morpheme. That is, the variable status could be taken as an as-yet incomplete grammaticalization from the independent lexical item \textitbf{punya} ‘have’ via the possessive marker \textitbf{punya} ‘\textsc{poss}’ into a clitic \textitbf{=p} ‘\textsc{poss}’ or a new possessive construction without overt marker.\footnote{\\
\\
\\
\\
\\
\\
\\
\\
\\
\\
\\
\\
\\
\\
\par One reason why \textitbf{punya} constructions are so frequent in Papuan Malay and other eastern Malay varieties, is that unlike the western Malay varieties, the eastern Malay varieties do not use suffix \textitbf{\-nya} ‘\textsc{3poss}’ as a marker of possessive relations, as for instance in western Malay \textitbf{tangang\-nya} ‘his/her hand’ {(H. Hammarström, p.c. 2013)}.\\
}
\end{styleBodyvafter}


In the corpus, the reductions of the possessive marker to the clitic \textitbf{=p} ‘\textsc{poss}’ or a zero morpheme occur with about the same frequency. Typically, the two constructions occur when the possessor is expressed with a short singular personal pronoun. It remains to be seen whether and to what extent over time (1) one of the constructions is going to become dominant, and (2) one or both constructions are going to occur with possessors encoded by the plural personal pronouns or common nouns. Such developments could be taken as an indication of a grammaticalization process of the possessive marker.
\end{styleBodyvxvafter}

\section{Realizations of \textsc{possr-np} and \textsc{possm-np}}
\label{bkm:Ref340732512}
This section discusses the different realizations of the possessor and possessum noun phrases in adnominal possessive constructions. The syntactic categories that can take the possessor or possessum slots, together with their semantic properties are discussed in §9.2.1.1. Elision of the possessum noun phrase is described in §9.2.1.2, followed by a brief discussion of recursive possessive constructions in §9.2.1.3.
\end{styleBodyxvafter}

\paragraph[Syntactic and semantic properties]{Syntactic and semantic properties}
\label{bkm:Ref340649443}
In adnominal possessive constructions, the possessor and/or possessum can be expressed by lexical nouns as in (0) and (0), by demonstratives as in (0) and (0), or by noun phrases as in (0) to (0). Further, the possessor can be encoded by a personal pronoun as in (0) to (0). Semantically, the possessor and the possessum can be human as in (0), nonhuman animate as in (0), or inanimate as in (0), respectively.



In (0) and (0), the possessor and the possessum are expressed by lexical nouns.
\end{styleBodyvvafter}

\begin{styleExampleTitle}
Lexical nouns expressing the possessor / possessum
\end{styleExampleTitle}

\begin{tabular}{lllllll}
\lsptoprule
\label{bkm:Ref339635922}\label{bkm:Ref282076024}
\gll {sa} {masi} {ingat} {\bluebold{bapa}} {\bluebold{pu}} {\bluebold{muka}}\\ %
& \textsc{1sg} & still & remember & father & \textsc{poss} & front\\
\lspbottomrule
\end{tabular}
\ea
\glt 
‘I still remember \bluebold{father’s face}’ \textstyleExampleSource{[080922-001a-CvPh.1307]}
\z

\begin{tabular}{llllll}
\lsptoprule
\label{bkm:Ref339635926}\label{bkm:Ref336676246}
\gll {…} {pake} {\bluebold{setang}} {\bluebold{punya}} {\bluebold{kwasa}}\\ %
&  & use & evil.spirit & \textsc{poss} & power\\
\lspbottomrule
\end{tabular}
\ea
\glt 
[About the power of evil spirits:] ‘[the sleeping person can’t wake up because the sorcerers are] using \bluebold{the evil spirit’s power}’ \textstyleExampleSource{[081006-022-CvEx.0084]}
\z


In (0) the proximal demonstrative \textitbf{ini} ‘\textsc{d.prox}’ takes the possessor slot and in (0) distal \textitbf{itu} ‘\textsc{d.dist}’ takes the possessum slot.


\begin{styleExampleTitle}
Demonstratives expressing the possessor / possessum
\end{styleExampleTitle}

\begin{tabular}{lllllllllll}
\lsptoprule
\label{bkm:Ref339635931}
\gll {bapa} {\multicolumn{2}{l}{masi}} {\multicolumn{2}{l}{kenal}} {kaka} {Siduas} {pu,} {masi} {kenal}\\ %
& father & \multicolumn{2}{l}{still} & \multicolumn{2}{l}{know} & oSb & Siduas & \textsc{poss} & still & know\\
& \multicolumn{2}{l}{\bluebold{ini}} & \multicolumn{2}{l}{\bluebold{pu}} & \multicolumn{6}{l}{\bluebold{muka}}\\
& \multicolumn{2}{l}{\textsc{d.prox}} & \multicolumn{2}{l}{\textsc{poss}} & \multicolumn{6}{l}{front}\\
\lspbottomrule
\end{tabular}
\ea
\glt 
‘do you (‘father’) still know Siduas’, still know \bluebold{this (one)’s face}?’ \textstyleExampleSource{[080922-001a-CvPh.1123]}
\z

\begin{tabular}{lllllll}
\lsptoprule
\label{bkm:Ref339635943}
\gll {ko} {ambil} {dulu} {\bluebold{ade}} {\bluebold{pu}} {\bluebold{itu}}\\ %
& \textsc{2sg} & fetch & first & ySb & \textsc{poss} & \textsc{d.dist}\\
\lspbottomrule
\end{tabular}
\ea
\glt 
‘you pick (it) up first, \bluebold{that (fish) of (your) younger sister}’ (Lit. ‘\bluebold{younger sibling’s that}’) \textstyleExampleSource{[081006-019-Cv.0002]}
\z


In (0) to (0), noun phrases take the possessor or the possessum slot (the scope of the noun phrases is indicated with brackets). In (0) the possessor is encoded by a noun phrase with a verbal modifier plus an adnominal demonstrative, while in (0) the possessor is expressed by a coordinate noun phrase.


\begin{styleExampleTitle}
Noun phrases expressing the possessor
\end{styleExampleTitle}

\begin{tabular}{lllllllll}
\lsptoprule
\label{bkm:Ref339635937}
\gll {sebut} {[[[\bluebold{orang}} {\bluebold{mati}]} {\bluebold{tu}]} {\bluebold{pu}} {[\bluebold{nama}]]} {karna} {…}\\ %
& name & person & die & \textsc{d.dist} & \textsc{poss} & name & because & \\
\lspbottomrule
\end{tabular}
\ea
\glt 
‘(he has) to mention \bluebold{that dead person’s name} because …’ \textstyleExampleSource{[080923-013-CvEx.0019]}
\z

\begin{tabular}{lllllllll}
\lsptoprule
\label{bkm:Ref339635934}
\gll {itu} {ko} {pu} {[[\bluebold{ko}} {\bluebold{deng}} {\bluebold{Mateus}]} {\bluebold{pu}} {[\bluebold{tugas}]]}\\ %
& \textsc{d.dist} & \textsc{2sg} & \textsc{poss} & \textsc{2sg} & with & Mateus & \textsc{poss} & duty\\
\lspbottomrule
\end{tabular}
\ea
\glt 
‘that is your, \bluebold{your and Mateus’ duty}’ \textstyleExampleSource{[081005-001-Cv.0035]}
\z


In (0), the possessum is encoded by a noun phrase with an adnominally used stative verb plus an adnominal demonstrative. In (0), a noun phrase with nominal modifier plus an adnominal demonstrative takes the possessum slot. In (0) the possessum is expressed by a coordinate noun phrase. In (0), repeated as (0), a noun phrase with a modifying nonfinite clause takes the possessum slot. The examples in (0) to (0) also illustrate that a personal pronoun can take the possessor slot; personal pronouns do not take the possessum slot.


\begin{styleExampleTitle}
Noun phrases expressing the possessum
\end{styleExampleTitle}

\begin{tabular}{llllllllll}
\lsptoprule
\label{bkm:Ref339635945}
\gll {[[\bluebold{de}]} {\multicolumn{2}{l}{\bluebold{pu}}} {[[\bluebold{cucu}} {\bluebold{kecil}]} {\bluebold{itu}]]} {tiap} {hari} {de}\\ %
& \textsc{3sg} & \multicolumn{2}{l}{\textsc{poss}} & grandchild & be.small & \textsc{d.dist} & every & day & \textsc{3sg}\\
& \multicolumn{2}{l}{menangis} & \multicolumn{7}{l}{trus}\\
& \multicolumn{2}{l}{cry} & \multicolumn{7}{l}{be.continuous}\\
\lspbottomrule
\end{tabular}
\ea
\glt 
‘\bluebold{that small grandchild of his}, every day he/she cries continuously’ \textstyleExampleSource{[081011-009-Cv.0055]}
\z

\begin{tabular}{llllllllll}
\lsptoprule
\label{bkm:Ref364861256}
\gll {sa} {tida} {maw} {[[\bluebold{sa}]} {\bluebold{punya}} {[[\bluebold{sodara}} {\bluebold{prempuang}]} {\bluebold{itu}]]} {mendrita}\\ %
& \textsc{1sg} & \textsc{neg} & want & \textsc{1sg} & \textsc{poss} & sibling & woman & \textsc{d.dist} & suffer\\
\lspbottomrule
\end{tabular}
\ea
\glt 
‘I don’t want \bluebold{that sister of mine} to suffer’ \textstyleExampleSource{[081006-024-CvEx.0108]}
\z

\begin{tabular}{llllllllll}
\lsptoprule
\label{bkm:Ref339635946}
\gll {nanti} {[[\bluebold{de}]} {\bluebold{punya}} {[\bluebold{bapa}} {\bluebold{dengang}} {\bluebold{mama}]]} {langsung} {pergi} {…}\\ %
& very.soon & \textsc{3sg} & \textsc{poss} & father & with & mother & immediately & go & \\
\lspbottomrule
\end{tabular}
\ea
\glt 
‘very soon \bluebold{her father and mother} will go …’ \textstyleExampleSource{[081110-005-CvPr.0079]}
\z

\begin{tabular}{llllllll}
\lsptoprule
\label{bkm:Ref339635947}
\gll {…} {[[\bluebold{tong}]} {\bluebold{pu}} {[\bluebold{cara}} {[\bluebold{makang}} {\bluebold{babi}]]]} {juga}\\ %
&  & \textsc{1pl} & \textsc{poss} & manner & eat & pig & also\\
\lspbottomrule
\end{tabular}
\ea
\glt 
‘[our way of eating is just like the Toraja one,] \bluebold{our way of eating pigs} also’ \textstyleExampleSource{[081014-017-CvPr.0053]}
\z


In (0) to (0) the possessor is always animate and/or human. It can, however, also be inanimate as shown in (0).


\begin{styleExampleTitle}
Inanimate possessor
\end{styleExampleTitle}

\begin{tabular}{lllllll}
\lsptoprule
\label{bkm:Ref339635927}\label{bkm:Ref336676247}
\gll {…} {\bluebold{LNG}} {\bluebold{pu}} {\bluebold{terpol}} {\bluebold{itu}} {tinggal}\\ %
&  & liquefied.natural.gas & \textsc{poss} & container & \textsc{d.dist} & stay\\
\lspbottomrule
\end{tabular}
\ea
\glt
[About the need to buy gasoline:] ‘[those jerry cans] \bluebold{that LNG jerry can} stays behind’ (Lit. ‘\bluebold{the LNG’s container}’) \textstyleExampleSource{[081110-002-Cv.0075]}
\end{styleFreeTranslEngxvpt}

\paragraph[Elision of the possessum noun phrase]{Elision of the possessum noun phrase}
\label{bkm:Ref340649444}
It is also possible to omit the possessum when its identity was established earlier; this applies to inalienably as well as alienably possessed referents, as illustrated in (0) to (0). Such ‘\textsc{possessor} \textitbf{punya}’ constructions are typically used in contexts where the possessor identity is under discussion.



In\textsc{ possessor} \textitbf{punya}’ constructions, speakers most commonly employ long \textitbf{punya} ‘\textsc{poss}’, but as shown in (0) and (0), constructions with reduced \textitbf{pu} ‘\textsc{poss}’ are also possible.\footnote{\\
\\
\\
\\
\\
\\
\\
\\
\\
\\
\\
\\
\\
\\
\par In his analysis of similar possessive constructions in Classical Malay, {\citet[157]{YapEtAl2004}} conclude that Classical Malay \textitbf{(em)punya} constructions with omitted possessum denote “pronominal possessive constructions”. More specifically, {\citet[7]{Yap2007}} maintains that “in such constructions \textit{(em)punya }identifies a possessee in relation to its possessor (the genitive function), while at the same time alluding to the morphologically unrealized possessee as well (the pronominal function). Consequently, possessive pronominal \textit{(em)punya }allows us to focus on the possessor, while still referring to the possessee”. It seems that this analysis is also applicable to Papuan Malay.\\
} Elision of the possessor is unattested. Instead speakers employ a demonstrative, as in (0), when the identity of the possessor has already been established.
\end{styleBodyvxafter}

\begin{tabular}{llllllllllll}
\lsptoprule
\label{bkm:Ref339635986}
\gll {Nofi} {\multicolumn{2}{l}{tu}} {\multicolumn{2}{l}{itu}} {\multicolumn{2}{l}{bukang}} {\bluebold{bapa}} {\bluebold{Lukas}} {\bluebold{punya}} {\bluebold{${\varnothing}$}}\\ %
& Nofi & \multicolumn{2}{l}{\textsc{d.dist}} & \multicolumn{2}{l}{\textsc{d.dist}} & \multicolumn{2}{l}{\textsc{neg}} & father & Lukas & \textsc{poss} & \\
& \multicolumn{2}{l}{\bluebold{mama}} & \multicolumn{2}{l}{\bluebold{Nofita}} & \multicolumn{2}{l}{\bluebold{punya}} & \multicolumn{5}{l}{\bluebold{${\varnothing}$}}\\
& \multicolumn{2}{l}{mother} & \multicolumn{2}{l}{Nofita} & \multicolumn{2}{l}{\textsc{poss}} & \multicolumn{5}{l}{}\\
\lspbottomrule
\end{tabular}
\ea
\glt 
‘Nofi here, that’s not \bluebold{father Lukas’} (son nor) \bluebold{mother Nofita’s} (son)’ \textstyleExampleSource{[081006-024-CvEx.0011]}
\z

\begin{tabular}{lllll}
\lsptoprule
(\stepcounter{}{\the}) & itu & \bluebold{de} & \bluebold{punya} & \bluebold{${\varnothing}$}\\
& \textsc{d.dist} & \textsc{3sg} & \textsc{poss} & \\
\lspbottomrule
\end{tabular}
\ea
\glt 
‘those are \bluebold{his} (banana plants)’ \textstyleExampleSource{[081110-008-CvNP.0121]}
\z

\begin{tabular}{llllllll}
\lsptoprule
\label{bkm:Ref339635988}
\gll {sedangkang} {Pawlus} {ini} {itu} {\bluebold{jing}} {\bluebold{pu}} {\bluebold{${\varnothing}$}}\\ %
& whereas & Pawlus & \textsc{d.prox} & \textsc{d.dist} & genie & \textsc{poss} & \\
\lspbottomrule
\end{tabular}
\ea
\glt 
‘whereas Pawlus here, that’s \bluebold{the genie’s} (child)’ \textstyleExampleSource{[081025-006-Cv.0276]}
\z

\begin{tabular}{llllllllll}
\lsptoprule
\label{bkm:Ref339635989}
\gll {ko} {liat} {\bluebold{Luisa}} {\bluebold{pu}} {\bluebold{${\varnothing}$}} {bagus,} {suda} {kembang} {banyak}\\ %
& \textsc{2sg} & see & Luisa & \textsc{poss} &  & be.good & already & flowering & many\\
\lspbottomrule
\end{tabular}
\ea
\glt
‘you see \bluebold{Luisa’s} (flowers) are good, (they are) already flowering a lot’ \textstyleExampleSource{[081006-021-CvHt.0002]}
\end{styleFreeTranslEngxvpt}

\paragraph[Recursive adnominal possessive constructions]{Recursive adnominal possessive constructions}
\label{bkm:Ref340649448}
Adnominal possessive constructions can be stacked to form recursive possessive constructions as illustrated in (0) to (0). Double possessive constructions as in (0) and (0) are quite common, especially to express kinship and social relations as in (0). Triple possessive constructions are also possible but extremely rare: the corpus contains only one such construction, which is presented in (0) in slightly modified form.
\end{styleBodyxafter}

\begin{tabular}{llllllllllll}
\lsptoprule
\label{bkm:Ref339635964}
\gll {kalo} {memang} {\bluebold{ko}} {\bluebold{punya}} {\bluebold{maytua}} {\bluebold{punya}} {\bluebold{waktu}} {pas} {di} {kapal} {…}\\ %
& if & indeed & 2\textsc{sg} & \textsc{poss} & wife & \textsc{poss} & time & precisely & at & ship & \\
\lspbottomrule
\end{tabular}
\ea
\glt 
‘if indeed \bluebold{your wife’s time} (to give birth) is right then (when you’re) on the ship …’ \textstyleExampleSource{[080922-001a-CvPh.0010]}
\z

\begin{tabular}{llllllll}
\lsptoprule
\label{bkm:Ref339635965}
\gll {ini} {\bluebold{kaka}} {\bluebold{Natanael}} {\bluebold{pu}} {\bluebold{laki}} {\bluebold{pu}} {\bluebold{mobil}}\\ %
& \textsc{d.prox} & oSb & Natanael & \textsc{poss} & husband & \textsc{poss} & car\\
\lspbottomrule
\end{tabular}
\ea
\glt 
‘this is \bluebold{sister Natanael’s husband’s car}’ \textstyleExampleSource{[081006-015-Cv.0001]}
\z

\begin{tabular}{llllllllllll}
\lsptoprule
\label{bkm:Ref339635966}
\gll {de\textsuperscript{i}} {\multicolumn{2}{l}{pu}} {ana} {\multicolumn{3}{l}{kawing}} {\bluebold{de}\textsuperscript{i}} {\bluebold{pu}} {\bluebold{laki}} {\bluebold{punya}}\\ %
& \textsc{3sg} & \multicolumn{2}{l}{\textsc{poss}} & child & \multicolumn{3}{l}{marry.inofficially} & \textsc{3sg} & \textsc{poss} & husband & \textsc{poss}\\
& \multicolumn{2}{l}{\bluebold{kaka}} & \multicolumn{3}{l}{\bluebold{prempuang}} & \bluebold{pu} & \multicolumn{5}{l}{\bluebold{ana}}\\
& \multicolumn{2}{l}{oSb} & \multicolumn{3}{l}{woman} & \textsc{poss} & \multicolumn{5}{l}{child}\\
\lspbottomrule
\end{tabular}
\ea
\glt 
‘her\textsuperscript{i} child (wants to) marry \bluebold{the son of her}\textsuperscript{i}\bluebold{ husband’s older sister}’ \textstyleExampleSource{[Elicited BR111020.026]}\footnote{\\
\\
\\
\\
\\
\\
\\
\\
\\
\\
\\
\\
\\
\\
\par The elicited utterance in (0) is based on an original triple possessive construction which contains the demonstrative \textitbf{ini} ‘\textsc{d.prox}’: \textitbf{… de pu laki, ini, punya kaka prempuang pu ana}. In this context, \textitbf{ini} ‘\textsc{d.prox}’ functions as a placeholder and therefore does not mark off \textitbf{de pu laki} ‘her husband’ as one noun phrase (see §7.1.2.6 for a discussion of the placeholder uses of demonstratives).\\
The subscript letters keep track of what each term refers to.}
\z


As discussed in §9.1.1, the long ligature form \textitbf{punya} ‘\textsc{poss}’ and short \textitbf{pu} ‘\textsc{poss}’ are freely used in adnominal possessive constructions without any syntactic or semantic restrictions. This also applies to recursive possessive constructions, as illustrated in (0) to (0). In terms of attested frequencies in such constructions, however, short \textitbf{pu} ‘\textsc{poss}’ is employed more often than long \textitbf{punya} ‘\textsc{poss}’.
\end{styleBodyxvafter}

\section{Non-canonical adnominal possessive constructions}
\label{bkm:Ref374440224}\label{bkm:Ref288567195}
In addition to encoding adnominal possession, \textitbf{punya} ‘\textsc{poss}’ (including its reduced forms) also serves other functions in possessive constructions, namely as (1) an emphatic marker that signals locational relations or association (§9.3.1), (2) a marker of beneficiary relations (§9.3.2), or (3) an attitudinal intensifier or stance (§9.3.3). And (4), the possessive ligature is used in reflexive construction (§9.3.4).



Syntactically, not only nouns, personal pronouns, demonstratives, or noun phrases can take the possessor or possessum slots. In addition, these slots can be filled by verbs. Further, the possessum slot can be taken by mid-range quantifiers, temporal adverbs, or prepositional phrases. Finally, the possessum can be omitted.
\end{styleBodyvxvafter}

\subsection{Locational relations and association}
\label{bkm:Ref352849773}
Cross-linguistically, one non-canonical function of possessive constructions, is to signal that the possessum is “perceived to be closely related” to the possessor {\citep[278]{Dixon2010a}}. In Papuan Malay, this includes locational relations, both spatial and temporal, and relations that express an association, as illustrated in (0) to (0). With this function of \textitbf{punya} ‘\textsc{poss}’, the possessive construction receives an emphatic reading; in the following examples the English translation attempts to convey this emphatic reading with the additional italicized information.



The possessive marker can signal locational relations, or, employing {Dixon’s (2010a: 263)} terminology, relations of “orientation or location”. The locational relations can be spatial, as in (0) and (0), or temporal, as in (0) to (0).\footnote{\\
\\
\\
\\
\\
\\
\\
\\
\\
\\
\\
\\
\\
\\
\\
\par {Dixon’s (2010b: 263)} term “orientation/location” refers to spatial relations; temporal relations are not mentioned.}
\end{styleBodyvafter}


In (0) and (0), \textitbf{pu} ‘\textsc{poss}’ marks spatial relations between the possessor and the possessum, with the possessive construction receiving an emphatic reading. In (0), a spatial referent, encoded with the proper noun \textitbf{Jayapura}, takes the possessor slot. It denotes the location or source for the referent expressed by the possessum, \textitbf{dua blas orang} ‘twelve people’. In (0), the spatial referent, encoded in the prepositional phrase \textitbf{di dalam itu} ‘in that inside’, takes the possessum slot.\footnote{\\
\\
\\
\\
\\
\\
\\
\\
\\
\\
\\
\\
\\
\\
\\
\par The locative preposition \textitbf{di} ‘at, in’ can also be deleted (see §10.1.5) resulting in \textitbf{de pu dalam itu} ‘that inside (part) of it’.} It designates the location for the referent expressed by the pronominal possessor \textitbf{de} ‘\textsc{3sg}’.
\end{styleBodyvvafter}

\begin{styleExampleTitle}
Spatial locational relations
\end{styleExampleTitle}

\begin{tabular}{lllllllll}
\lsptoprule
\label{bkm:Ref339635967}
\gll {\bluebold{Jayapura}} {\bluebold{pu}} {\bluebold{dua}} {\bluebold{blas}} {\bluebold{orang}} {yang} {lulus} {ka?}\\ %
& Jayapura & \textsc{poss} & two & tens & person & \textsc{rel} & pass(.a.test) & or\\
\lspbottomrule
\end{tabular}
\ea
\glt 
‘aren’t there \bluebold{twelve people from Jayapura who graduated} (\textstyleChItalic{as opposed to other cities with fewer graduates})?’ (Lit. ‘\bluebold{Jayapura’s twelve people}’) \textstyleExampleSource{[081025-003-Cv.0311]}
\z

\begin{tabular}{lllllllllm{-9.4015896E-4in}lllll}
\lsptoprule
\label{bkm:Ref339635971}
\gll {\multicolumn{2}{l}{baru}} {\multicolumn{2}{l}{ambil}} {\multicolumn{2}{l}{bayi}} {\multicolumn{2}{l}{tu,}} {\multicolumn{3}{l}{bayi}} {yang} {\bluebold{de}} {\bluebold{pu}}\\ %
& \multicolumn{2}{l}{and.then} & \multicolumn{2}{l}{fetch} & \multicolumn{2}{l}{palm.stem} & \multicolumn{2}{l}{\textsc{d.dist}} & \multicolumn{3}{l}{palm.stem} & \textsc{rel} & \textsc{3sg} & \textsc{poss}\\
& \bluebold{di} & \multicolumn{2}{l}{\bluebold{dalam}} & \multicolumn{2}{l}{\bluebold{itu}} & \multicolumn{2}{l}{kang} & \multicolumn{2}{l}{kaya} & kapas & \multicolumn{4}{l}{to?}\\
& at & \multicolumn{2}{l}{inside} & \multicolumn{2}{l}{\textsc{d.dist}} & \multicolumn{2}{l}{you.know} & \multicolumn{2}{l}{like} & cotton & \multicolumn{4}{l}{right?}\\
\lspbottomrule
\end{tabular}
\ea
\glt 
‘and then (he) took that palm stem, \bluebold{that inside (part) of it} (\textstyleChItalic{as opposed to other parts}), you know, is like cotton, right?’ \textstyleExampleSource{[080922-010a-CvNF.0073]}
\z


In the elicited examples in (0) to (0), the possessive marker signals temporal locational relations. In these examples, the third person singular pronoun \textitbf{de} ‘\textsc{3sg}’ takes the possessor slot. It designates the temporal reference point for the event under discussion. The possessum slot is taken by a temporal expression such as \textitbf{besok} ‘tomorrow’ in (0), \textitbf{pagi} ‘morning’ in (0), and \textitbf{malam} ‘night’ in (0). This temporal expression denotes a specific point in time relative to the temporal reference point expressed by the possessor.


\begin{styleExampleTitle}
Temporal locational relations
\end{styleExampleTitle}

\begin{tabular}{llllllllllllll}
\lsptoprule
\label{bkm:Ref340850327}
\gll {…} {trus} {\multicolumn{2}{l}{sa}} {\multicolumn{2}{l}{tinggal}} {\multicolumn{2}{l}{di}} {sana,} {trus} {\bluebold{de}} {\bluebold{pu}} {\bluebold{besok}}\\ %
&  & next & \multicolumn{2}{l}{\textsc{1sg}} & \multicolumn{2}{l}{stay} & \multicolumn{2}{l}{at} & \textsc{l.dist} & next & \textsc{3sg} & \textsc{poss} & tomorrow\\
& \multicolumn{3}{l}{baru} & \multicolumn{2}{l}{sa} & \multicolumn{2}{l}{kembali} & \multicolumn{6}{l}{…}\\
& \multicolumn{3}{l}{and.then} & \multicolumn{2}{l}{\textsc{1sg}} & \multicolumn{2}{l}{return} & \multicolumn{6}{l}{}\\
\lspbottomrule
\end{tabular}
\ea
\glt 
‘[two days ago I went to Abepura,] and then I stayed there, and then \bluebold{the} (\textstyleChItalic{very}) \bluebold{next day} only then did I return …’ (Lit. ‘\bluebold{its tomorrow}’) \textstyleExampleSource{[Elicited BR111020.008]}
\z

\begin{tabular}{lllllllllllllll}
\lsptoprule
\label{bkm:Ref340850328}
\gll {dong} {\multicolumn{2}{l}{kerja}} {\multicolumn{2}{l}{ruma}} {\multicolumn{2}{l}{dari}} {\multicolumn{2}{l}{pagi}} {sampe} {malam} {\bluebold{de}} {\bluebold{pu}} {\bluebold{pagi},}\\ %
& \textsc{3pl} & \multicolumn{2}{l}{work} & \multicolumn{2}{l}{house} & \multicolumn{2}{l}{from} & \multicolumn{2}{l}{morning} & until & night & \textsc{3sg} & \textsc{poss} & morning\\
& \multicolumn{2}{l}{baru} & \multicolumn{2}{l}{dong} & \multicolumn{2}{l}{kasi} & \multicolumn{2}{l}{selesay} & \multicolumn{6}{l}{smua}\\
& \multicolumn{2}{l}{and.then} & \multicolumn{2}{l}{\textsc{3pl}} & \multicolumn{2}{l}{\textsc{give}} & \multicolumn{2}{l}{finish} & \multicolumn{6}{l}{all}\\
\lspbottomrule
\end{tabular}
\ea
\glt 
‘they worked on the house from morning until evening, \bluebold{the} (\textstyleChItalic{very}) \bluebold{next morning} only then did they finish everything’ (Lit. ‘\bluebold{its morning}’) \textstyleExampleSource{[Elicited BR111020.009]}
\z

\begin{tabular}{lllllllllllllllll}
\lsptoprule
\label{bkm:Ref340850329}
\gll {\multicolumn{2}{l}{Petrus}} {\multicolumn{2}{l}{deng}} {Tinus} {\multicolumn{2}{l}{dong}} {\multicolumn{2}{l}{pi}} {\multicolumn{3}{l}{mandi}} {di} {pante} {tadi} {pagi,}\\ %
& \multicolumn{2}{l}{Petrus} & \multicolumn{2}{l}{with} & Tinus & \multicolumn{2}{l}{\textsc{3pl}} & \multicolumn{2}{l}{go} & \multicolumn{3}{l}{bathe} & at & coast & earlier & morning\\
& \bluebold{de} & \multicolumn{2}{l}{\bluebold{pu}} & \multicolumn{3}{l}{\bluebold{malam}} & \multicolumn{2}{l}{dong} & \multicolumn{2}{l}{pi} & ke & \multicolumn{5}{l}{Jayapura}\\
& \textsc{3sg} & \multicolumn{2}{l}{\textsc{poss}} & \multicolumn{3}{l}{night} & \multicolumn{2}{l}{\textsc{3pl}} & \multicolumn{2}{l}{go} & to & \multicolumn{5}{l}{Jayapura}\\
\lspbottomrule
\end{tabular}
\ea
\glt 
‘Petrus and Tinus went bathing at the beach this morning (and) \bluebold{this} (\textstyleChItalic{very}) \bluebold{evening} they went to Jayapura’ (Lit. ‘\bluebold{its night}’) \textstyleExampleSource{[Elicited BR111020.009]}
\z


Another cross-linguistically rather common function of the possessive marker it to indicate an “association” between the possessum and the possessor {\citep[285]{Dixon2010a}. This also applies to Papuan Malay, as shown }in (0) and (0). In (0), \textitbf{punya} ‘\textsc{poss}’ signals that the possessum \textitbf{tu} ‘\textsc{d.dist}’ is associated with the possessor \textitbf{lima juta} ‘five million’, giving the emphatic reading ‘a minimum of five-million (\textstyleChItalic{as opposed to lower prices})’.\footnote{\\
\\
\\
\\
\\
\\
\\
\\
\\
\\
\\
\\
\\
\\
\\
\par Alternatively, one might classify the possessive construction in (0) as an “‘appositive genitive’, where the two noun phrases are equated denotatively”, adopting {Quirk et al.’s (1972: 193)} terminology.} Along similar lines, in (0), the ligature indicates an association between the possessum \textitbf{tu} ‘\textsc{d.dist}’ and the possessor \textitbf{tingkat propinsi} ‘provincial level’, resulting in the emphatic reading ‘(a meeting at) the provincial level (\textstyleChItalic{and not at the regency level})’.


\begin{styleExampleTitle}
Association
\end{styleExampleTitle}

\begin{tabular}{llllllllll}
\lsptoprule
\label{bkm:Ref339635975}
\gll {yang} {\multicolumn{2}{l}{mahal}} {\multicolumn{2}{l}{yang}} {di} {atas} {satu} {jut}\\ %
& \textsc{rel} & \multicolumn{2}{l}{be.expensive} & \multicolumn{2}{l}{\textsc{rel}} & at & top & one & \textsc{tru}{}-million\\
& \bluebold{lima} & \bluebold{juta} & \multicolumn{2}{l}{\bluebold{punya}} & \multicolumn{5}{l}{\bluebold{tu}}\\
& five & million & \multicolumn{2}{l}{\textsc{poss}} & \multicolumn{5}{l}{\textsc{d.dist}}\\
\lspbottomrule
\end{tabular}
\ea
\glt 
‘(traditional cloths from Sorong) which are expensive, which (cost) more than one million[\textsc{tru}], \bluebold{a minimum of five million} (\textstyleChItalic{as opposed to lower prices})’ (Lit. ‘\bluebold{that} (price) \bluebold{of five million}’) \textstyleExampleSource{[081006-029-CvEx.0009]}
\z

\begin{tabular}{llllllll}
\lsptoprule
\label{bkm:Ref339635976}
\gll {kitong} {ikut} {ini} {\bluebold{tingkat}} {\bluebold{propinsi}} {\bluebold{punya}} {\bluebold{tu}}\\ %
& \textsc{1pl} & follow & \textsc{d.prox} & floor & province & \textsc{poss} & \textsc{d.dist}\\
\lspbottomrule
\end{tabular}
\ea
\glt
‘we attended (a meeting at), what’s-its-name, \bluebold{the provincial level} (\textstyleChItalic{and not at the regency level})’ (Lit. ‘\bluebold{that} (meeting) \bluebold{of the provincial level}’) \textstyleExampleSource{[081010-001-Cv.0043]}
\end{styleFreeTranslEngxvpt}

\subsection{Beneficiary relations}
\label{bkm:Ref336853953}
The possessive marker \textitbf{punya} ‘\textsc{poss}’ is also used to signal beneficiary relations. Speakers employ this construction when they want to signal that the recipient is the beneficiary of a definite theme, as discussed in §11.1.3.3. This is illustrated in (0) and (0). In the respective examples, the possessors \textitbf{mama} ‘mother’ and \textitbf{de} ‘\textsc{3sg}’ express the recipients/beneficiaries of the events expressed by the verbs \textitbf{simpang} ‘store’ and \textitbf{bli} ‘buy’, while the possessa \textitbf{makang} ‘food’ and \textitbf{alat{\Tilde}alat} ‘utensils’ denote the anticipated objects of possession or themes.
\end{styleBodyxafter}

\begin{tabular}{llllllll}
\lsptoprule
\label{bkm:Ref339635978}
\gll {mama,} {kitong} {suda} {simpang} {\bluebold{mama}} {\bluebold{punya}} {\bluebold{makang}}\\ %
& mother & \textsc{1pl} & already & store & mother & \textsc{poss} & food\\
\lspbottomrule
\end{tabular}
\ea
\glt 
‘mother, we already put \bluebold{food for you} aside’ (Lit. ‘\bluebold{mama’s food}’) \textstyleExampleSource{[080924-002-Pr.0005]}
\z

\begin{tabular}{llllllll}
\lsptoprule
\label{bkm:Ref339635977}
\gll {dong} {su} {bli} {\bluebold{de}} {\bluebold{punya}} {\bluebold{alat{\Tilde}alat}} {\bluebold{ini}}\\ %
& \textsc{3pl} & already & buy & \textsc{3sg} & \textsc{poss} & \textsc{rdp}{\Tilde}equipment & \textsc{d.prox}\\
\lspbottomrule
\end{tabular}
\ea
\glt
‘they already bought \bluebold{these utensils for him}’ (Lit. ‘\bluebold{his utensils}’) \textstyleExampleSource{[080922-001a-CvPh.0558]}
\end{styleFreeTranslEngxvpt}

\subsection{Intensifying function of \textitbf{punya} ‘\textsc{poss}’}
\label{bkm:Ref336853956}
Another non-canonical function of possessive \textitbf{punya} ‘\textsc{poss}’ is that of an intensifier or stance that signals speaker attitudes or evaluations. The attested data suggest three different constructions in which Papuan Malay speakers use \textitbf{punya} ‘\textsc{poss}’ in such a way: constructions with (1) a nominal possessor and a quantifier possessum (§9.3.3.1), (2) a nominal possessor and a verbal possessum (§9.3.3.2), and (3) a verbal possessor and a verbal possessum (§9.3.3.3).
\end{styleBodyxvafter}

\paragraph[n{}-possr – punya – qt{}-possm constructions]{\textsc{n-possr} – \textitbf{punya} – \textsc{qt-possm} constructions}
\label{bkm:Ref341196712}
In the possessive constructions in (0) to (0), a nominal constituent takes the possessor slot while a quantifier takes the possessum slot.



Attested in the corpus is only the one example in (0) in which the mid-range quantifier \textitbf{banyak} ‘many’ takes the possessum slot. A second, elicited example is presented in (0). Possessive constructions with the mid-range quantifier \textitbf{sedikit} ‘few’ are also possible as illustrated with the elicited examples in (0) and (0). In these examples \textitbf{punya} ‘\textsc{poss}’ functions as an attitudinal intensifier, expressing speaker evaluations, such as feelings of annoyance in (0), of surprise in (0) and in (0), or of alarm in (0).
\end{styleBodyvxafter}

\begin{tabular}{llllllll}
\lsptoprule
\label{bkm:Ref339636008}
\gll {baru,} {mama,} {setang} {\bluebold{pu}} {\bluebold{banyak}} {di} {situ}\\ %
& and.then & mother & evil.spirit & \textsc{poss} & many & at & \textsc{l.med}\\
\lspbottomrule
\end{tabular}
\ea
\glt 
‘and then, mother, (there) are \bluebold{really many }evil spirits over there’ (Lit. ‘\bluebold{many of}’) \textstyleExampleSource{[081025-006-Cv.0062]}
\z

\begin{tabular}{lllllllll}
\lsptoprule
\label{bkm:Ref340850356}
\gll {natal} {tu} {ana{\Tilde}ana} {dong} {maing} {kembang-api} {\bluebold{pu}} {\bluebold{banyak}}\\ %
& Christmas & \textsc{d.dist} & \textsc{rdp}{\Tilde}child & \textsc{3pl} & play & fire-cracker & \textsc{poss} & many\\
\lspbottomrule
\end{tabular}
\ea
\glt 
‘(during) Christmas (time) the children play with \bluebold{really many} fire-crackers’ (Lit. ‘\bluebold{many of}’) \textstyleExampleSource{[Elicited BR111020.005]}
\z

\begin{tabular}{lllllll}
\lsptoprule
\label{bkm:Ref340850357}
\gll {di} {gunung} {itu} {pohong} {\bluebold{pu}} {\bluebold{sedikit}}\\ %
& at & mountain & \textsc{d.dist} & tree & \textsc{poss} & few\\
\lspbottomrule
\end{tabular}
\ea
\glt 
‘on that mountain, there are \bluebold{very few} trees’ (Lit. ‘\bluebold{few of}’) \textstyleExampleSource{[Elicited BR111020.006]}
\z

\begin{tabular}{lllllll}
\lsptoprule
\label{bkm:Ref340850358}
\gll {tete} {de} {minum} {air} {\bluebold{pu}} {\bluebold{sedikit}}\\ %
& grandfather & \textsc{3sg} & drink & water & \textsc{poss} & few\\
\lspbottomrule
\end{tabular}
\ea
\glt 
‘grandfather drinks \bluebold{very little} water’ (Lit. ‘\bluebold{few of}’) \textstyleExampleSource{[Elicited BR111020.007]}
\z


Possessive constructions with other quantifiers or with numerals taking the possessum slot are ungrammatical.
\end{styleBodyxvafter}

\paragraph[n{}-possr – punya – v{}-possm constructions]{\textsc{n-possr} – \textitbf{punya} – \textsc{v-possm} constructions}
\label{bkm:Ref341196713}
In the possessive constructions in (0) to (0), a nominal constituent takes the possessor slot while a mono- or bivalent verb takes the possessum slot. In these constructions, speakers typically use short \textitbf{pu} ‘\textsc{poss}’ rather than long \textitbf{punya} ‘\textsc{poss}’; more investigation is needed however, to further explore these speaker preferences.



In the examples in (0) to (0), a monovalent verb takes the possessum slot. Again, the ligature functions as an attitudinal intensifier. In (0) \textitbf{pu} ‘\textsc{poss}’ adds emphasis to stative \textitbf{malas} ‘be listless’. In (0), \textitbf{pu} ‘\textsc{poss}’ precedes stative \textitbf{brat} ‘be heavy’, and thereby signals feelings of annoyance. Finally, in (0), the possessive marker precedes dynamic \textitbf{mendrita} ‘suffer’, thereby indicating negative feelings of disbelief.
\end{styleBodyvvafter}

\begin{styleExampleTitle}
Intensifying function of \textitbf{punya} ‘\textsc{poss}’: Preceding monovalent verbs
\end{styleExampleTitle}

\begin{tabular}{llllllllllll}
\lsptoprule
\label{bkm:Ref339635998}
\gll {dong} {\multicolumn{2}{l}{tida}} {\multicolumn{3}{l}{taw}} {umpang,} {smua} {tra} {taw} {toser,}\\ %
& \textsc{3pl} & \multicolumn{2}{l}{\textsc{neg}} & \multicolumn{3}{l}{know} & pass.ball & all & \textsc{neg} & know & pass.ball\\
& \multicolumn{2}{l}{adu} & \multicolumn{2}{l}{sa} & \bluebold{pu} & \multicolumn{6}{l}{\bluebold{malas}}\\
& \multicolumn{2}{l}{oh.no!} & \multicolumn{2}{l}{\textsc{1sg}} & \textsc{poss} & \multicolumn{6}{l}{be.listless}\\
\lspbottomrule
\end{tabular}
\ea
\glt 
[About playing volleyball:] ‘none of them knows (how) to pass a ball, none of them knows (how) to pass a ball, oh no, I’m \bluebold{so very listless} (to play with them)’ (Lit. ‘\bluebold{the being listless of}’) \textstyleExampleSource{[081109-001-Cv.0127]}
\z

\begin{tabular}{lllll}
\lsptoprule
\label{bkm:Ref339635994}
\gll {damay,} {de} {\bluebold{pu}} {\bluebold{brat}}\\ %
& peace & \textsc{3sg} & \textsc{poss} & be.heavy\\
\lspbottomrule
\end{tabular}
\ea
\glt 
‘my goodness, he was \bluebold{so heavy}’ (Lit. ‘\bluebold{the being} \bluebold{heavy of}’) \textstyleExampleSource{[081025-009b-Cv.0041]}
\z

\begin{tabular}{llllll}
\lsptoprule
\label{bkm:Ref339635993}
\gll {adu,} {dong} {dua} {\bluebold{pu}} {\bluebold{mendrita}}\\ %
& oh.no! & \textsc{3pl} & two & \textsc{poss} & suffer\\
\lspbottomrule
\end{tabular}
\ea
\glt 
‘oh no, the two of them were \bluebold{suffering so much}’ (Lit. ‘\bluebold{the suffering of}’) \textstyleExampleSource{[081025-006-Cv.0059]}
\z


In (0) to (0), the possessum slot is taken by bivalent verbs. Again, the possessive marker has intensifying, asserting and/or evaluative function.


\begin{styleExampleTitle}
Intensifying function of \textitbf{punya} ‘\textsc{poss}’: Preceding bivalent verbs
\end{styleExampleTitle}

\begin{tabular}{llllllll}
\lsptoprule
\label{bkm:Ref339636001}
\gll {ka} {Sarles} {juga} {de} {\bluebold{pu}} {\bluebold{maing}} {\bluebold{pisow}}\\ %
& oSb & Sarles & also & \textsc{3sg} & \textsc{poss} & play & knife\\
\lspbottomrule
\end{tabular}
\ea
\glt 
‘older brother Sarles also, he \bluebold{has a fast and smart way of playing}’ (Lit. ‘\bluebold{the knife playing of}’) \textstyleExampleSource{[081023-001-Cv.0009]}
\z

\begin{tabular}{llllllll}
\lsptoprule
\label{bkm:Ref339636003}
\gll {baru} {nanti} {tong} {\bluebold{pu}} {\bluebold{lawang}} {deng} {siapa}\\ %
& and.then & very.soon & \textsc{1pl} & \textsc{poss} & oppose & with & who\\
\lspbottomrule
\end{tabular}
\ea
\glt
‘and then later who will be \bluebold{our opponent}?’ (Lit. ‘\bluebold{the opposing of}’) \textstyleExampleSource{[081109-001-Cv.0136]}
\end{styleFreeTranslEngxvpt}

\paragraph[v{}-possr – punya – v{}-possm constructions]{\textsc{v-possr} – \textitbf{punya} – \textsc{v-possm} constructions}
\label{bkm:Ref341196714}
In non-canonical possessive constructions, both the possessor and the possessum slot can be taken by verbs, as illustrated in (0) to (0). More specifically, a dynamic verb takes the possessor slot, while a stative verb takes the possessum slot. Attested in the corpus is only the one example in (0), while the examples in (0) to (0) are elicited.



With its intensifying function, \textitbf{punya} ‘\textsc{poss}’ signals an emphatic reading of both the verbal possessor and the verbal possessum, as illustrated in (0): \textitbf{mandi punya} ‘really bathing’ and \textitbf{punya jaw} ‘very far away (of)’.
\end{styleBodyvxafter}

\begin{tabular}{lllllllllll}
\lsptoprule
\label{bkm:Ref339636006}
\gll {dong} {mandi} {di} {kali} {Biri,} {mm-mm,} {\bluebold{mandi}} {\bluebold{punya}} {\bluebold{jaw}} {itu}\\ %
& \textsc{3pl} & bathe & at & river & Biri & mhm & bathe & \textsc{poss} & be.far & \textsc{d.dist}\\
\lspbottomrule
\end{tabular}
\ea
\glt 
[About a run-away boy:] ‘they were bathing in the Biri river, mhm, (they were) \bluebold{really bathing very far away}’ (Lit. ‘\bluebold{the being far away of the bathing}’) \textstyleExampleSource{[081025-008-Cv.0032-0033]}
\z

\begin{tabular}{lllll}
\lsptoprule
\label{bkm:Ref341196773}
\gll {de} {\bluebold{kerja}} {\bluebold{punya}} {\bluebold{cepat}}\\ %
& \textsc{3sg} & work & \textsc{poss} & be.fast\\
\lspbottomrule
\end{tabular}
\ea
\glt 
‘he \bluebold{really worked very fast}’ (Lit. ‘\bluebold{the being fast of the working}’) \textstyleExampleSource{[Elicited BR111020.022]}
\z

\begin{tabular}{llllll}
\lsptoprule
(\stepcounter{}{\the}) & mama & de & \bluebold{masak} & \bluebold{punya} & \bluebold{enak}\\
& mother & \textsc{3sg} & cook & \textsc{poss} & be.pleasant\\
\lspbottomrule
\end{tabular}
\ea
\glt 
‘mother \bluebold{really cooks very tastily}’ (Lit. ‘\bluebold{the being tasty of the cooking}’) \textstyleExampleSource{[Elicited BR111020.023]}
\z

\begin{tabular}{lllllllll}
\lsptoprule
\label{bkm:Ref341196775}
\gll {Marice} {deng} {Matius} {dong} {dua} {\bluebold{bicara}} {\bluebold{punya}} {\bluebold{kras}}\\ %
& Marice & with & Matius & \textsc{3pl} & two & speak & \textsc{poss} & be.harsh\\
\lspbottomrule
\end{tabular}
\ea
\glt
‘the two of them Marice and Matius \bluebold{really spoke very loudly} (with each other)’ (Lit. ‘\bluebold{the being loud of the speaking}’) \textstyleExampleSource{[Elicited BR111020.024]}
\end{styleFreeTranslEngxvpt}

\subsection{\textitbf{punya} ‘\textsc{poss}’ in reflexive expressions}
\label{bkm:Ref336853954}
The possessive marker \textitbf{punya} ‘\textsc{poss}’ is also used to create reflexive expressions. Generally speaking, reflexives designate constructions “where subject and object refer to the same entity, explicitly […] or implicitly” {\citep[5164]{Asher1994}}. Typically, explicit reflexive expressions are formed with a reflexive pronoun “which refers to the same person or thing as the subject of the verb” {(1994: 5165)}. As Papuan Malay does not have reflexive pronouns, an alternative strategy is used. Reflexive relations are expressed with an adnominal possessive construction where a personal pronoun in the possessor slot and the reflexive noun \textitbf{diri} ‘self’ in the possessum slot express the reflexive relationship between both, as illustrated with \textitbf{sa pu diri} ‘myself’ in (0) and \textitbf{kita punya diri} ‘ourselves’ in (0).
\end{styleBodyxafter}

\begin{tabular}{lllllllllllllll}
\lsptoprule
\label{bkm:Ref339635980}
\gll {\multicolumn{2}{l}{bukang}} {\multicolumn{2}{l}{sa}} {\multicolumn{2}{l}{rasa}} {\multicolumn{2}{l}{bahwa}} {\multicolumn{2}{l}{sa}} {\multicolumn{2}{l}{ini}} {sa} {banggakang}\\ %
& \multicolumn{2}{l}{\textsc{neg}} & \multicolumn{2}{l}{\textsc{1sg}} & \multicolumn{2}{l}{feel} & \multicolumn{2}{l}{that} & \multicolumn{2}{l}{\textsc{1sg}} & \multicolumn{2}{l}{\textsc{d.prox}} & \textsc{1sg} & praise\\
& \bluebold{sa} & \multicolumn{2}{l}{\bluebold{pu}} & \multicolumn{2}{l}{\bluebold{diri}} & \multicolumn{2}{l}{tapi} & \multicolumn{2}{l}{itu} & \multicolumn{2}{l}{yang} & \multicolumn{3}{l}{terjadi}\\
& \textsc{1sg} & \multicolumn{2}{l}{\textsc{poss}} & \multicolumn{2}{l}{self} & \multicolumn{2}{l}{but} & \multicolumn{2}{l}{\textsc{d.dist}} & \multicolumn{2}{l}{\textsc{rel}} & \multicolumn{3}{l}{happen}\\
\lspbottomrule
\end{tabular}
\ea
\glt 
‘it’s not that I feel that I here, (that) I praise \bluebold{myself}, but that’s what happened’ (Lit. ‘\bluebold{the self of me}’) \textstyleExampleSource{[081110-008-CvNP.0152]}
\z

\begin{tabular}{lllllllll}
\lsptoprule
\label{bkm:Ref339635985}
\gll {kita} {rencana,} {manusia} {yang} {mengatur} {\bluebold{kita}} {\bluebold{punya}} {\bluebold{diri}}\\ %
& \textsc{1pl} & plan & human.being & \textsc{rel} & arrange & \textsc{1pl} & \textsc{poss} & self\\
\lspbottomrule
\end{tabular}
\ea
\glt
‘we make plans, (it’s us) human beings who manage \bluebold{our own lives}’ (Lit. ‘\bluebold{the self of us}’) \textstyleExampleSource{[080918-001-CvNP.0032]}
\end{styleFreeTranslEngxvpt}

\section{Summary and discussion}
\label{bkm:Ref288650133}
In Papuan Malay, adnominal possessive constructions consist of two noun phrases linked with the possessive marker \textitbf{punya} ‘\textsc{poss}’, such that ‘\textsc{possessor} \textitbf{punya} \textsc{possessum}’. In addition to signaling adnominal possessive relations between two noun phrases, \textitbf{punya} ‘\textsc{poss}’ has a number of derived, non-canonical functions, namely as (1) an emphatic marker of locational relations or relations of association, (2) a marker of beneficiary relations, (3) an attitudinal intensifier or stance, and (4) a ligature in reflexive constructions.



Such non-canonical functions of the possessive ligature have also been noted in other eastern Malay varieties. Examples are its functions as a marker of beneficiary relations in Ambon Malay {(van Minde 1997: 164),} as a marker of locational or temporal relations in Ternate Malay {(Litamahuputty 1994: 52–53, 96–97)}, and as an attitudinal intensifier in Manado Malay {\citep[45]{Stoel2005}}.
\end{styleBodyvafter}


Two explanations have been suggested for these extended uses of the possessive marker in Malay speech varieties.
\end{styleBodyvafter}


One is to propose a substratum influence of Chinese languages. Some of the non-canonical functions of the possessive marker have long been noted for Bazaar Malay and have been linked to the substratum influence of Chinese speech varieties, namely the function of \textitbf{punya} ‘\textsc{poss}’ to link a locative or temporal modifier or a modifying adjective in the possessor slot preceding the ligature with its head in the possessum slot{ (see Shellabear 1904: 6–7; Winstedt 1913: 115; Winstedt 1938: 41; Lim 1988; Bao 2009}). {Yap (2007: 1, 8ff)} argues that under the influence of southern Chinese speech varieties, the colloquial Malay possessive marker developed into an “attitudinal intensifier” or “stance” that transforms statements into evaluative “assertions that are often laced with strong feelings, including feelings of awe, […] or feelings of incredulity or even annoyance”. For the different synchronic functions of \textitbf{(em)punya} in classical and colloquial Malay, {\citet[159]{YapEtAl2004}} propose the following development or grammaticalization path: “lexical verb {\textgreater} genitive {\textgreater} pronominal {\textgreater} stance development”.
\end{styleBodyvafter}


A second explanation proposes a grammaticalization process of the possessive marker without any substratum influence from Chinese varieties. {\citet[2]{Gil1999}} argues that the influence of Chinese languages does not “account for the presence of the \textitbf{punya}\textit{ }construction” in Malay varieties which have “little obvious contact with Chinese languages” such as Riau Indonesian or Papuan Malay; neither does this influence “account for the choice of the specific marker \textitbf{punya}”. Instead, Gil submits that the interpretation of the \textitbf{punya} construction underwent a semantic change from predicative possessive to adnominal possessive to non-canonical possessive, such that: “thing associated with X’s having” {\textgreater} “thing associated with X” {\textgreater} “property associated with X” {(1999: 6, 8)}.
\end{styleBodyvafter}


Possessive constructions with \textitbf{punya} ‘\textsc{poss}’ have a number of different realizations. The possessive marker can be represented with long \textitbf{punya}, reduced \textitbf{pu}, clitic \textitbf{=p}, or a zero morpheme. There are no syntactic or semantic restrictions on the uses of the long and reduced possessive marker forms. By contrast, omission of \textitbf{punya} only occurs when the possessive construction expresses inalienable possession of body parts or kinship relations. The possessor and the possessum can be expressed with different kinds of syntactic constituents, such as lexical nouns, noun phrases, or demonstratives. In addition, personal pronouns can also express the possessor. In non-canonical possessive constructions, verbs can also take the possessor and/or possessum slots. Further, mid-range quantifiers, temporal adverbs, and prepositional phrases can take the possessum slot. In canonical possessive constructions, the possessum can also be omitted. Semantically, the possessor and the possessum can denote human, nonhuman animate, or inanimate referents.


%\setcounter{page}{1}\chapter[Prepositions and the prepositional phrase]{Prepositions and the prepositional phrase}
\label{bkm:Ref361998646}
This chapter describes prepositional phrases in Papuan Malay, that is, constructions which consist of a preposition followed by a noun phrase, such that ‘\textsc{prep} \textsc{np}’.



Papuan Malay employs eleven prepositions that can be grouped semantically into (1) prepositions encoding location in space and time, (2) prepositions encoding accompaniment/instruments, goals, and benefaction, and (3) prepositions encoding comparisons. The defining characteristics of prepositions are discussed in §5.11.
\end{styleBodyvafter}


Prepositional phrases have the following defining characteristics:
\end{styleBodyvvafter}

%\setcounter{itemize}{0}
\begin{itemize}
\item \begin{styleIIndented}
All prepositional phrases function as peripheral adjuncts; as such they do not have a grammatically restricted position within the clause but can be moved to different positions.
\end{styleIIndented}\item \begin{styleIIndented}
Most prepositional phrases also function as nonverbal predicates and/or oblique arguments (see§12.4, and §11.1.3.2, respectively).
\end{styleIIndented}\item \begin{styleIIndented}
Some prepositional phrases also function as modifiers within noun phrases (§8.2.7)
\end{styleIIndented}\item \begin{styleIvI}
Prepositional phrases that function as nonverbal predicates can be modified by aspectual adverbs (§5.4.1), while such modification is unattested for prepositional phrases having other functions.
\end{styleIvI}\end{itemize}

In the following, Papuan Malay prepositional phrases are discussed according to the semantics of their prepositional head: location in space and time in §10.1, accompaniment/instruments, goals, and benefaction in §10.2, and comparisons in §10.3. The main points of this chapter are summarized in §10.4.
\end{styleBodyxvafter}

\section{Prepositions encoding location in space and time}
\label{bkm:Ref374462993}\label{bkm:Ref374455052}\label{bkm:Ref374450631}\label{bkm:Ref320188446}
Papuan Malay employs four prepositions that express location in space and time: locative \textitbf{di} ‘at, in’ designates static location (§10.1.1), allative \textitbf{ke} ‘to’ denotes direction toward a location (§10.1.2), elative \textitbf{dari} ‘from’ expresses direction away from or out of a location (§10.1.3), and lative \textitbf{sampe} ‘until’ designates direction up to a non-spatial temporal location (§10.1.4).
\end{styleBodyxvafter}

\subsection{\textitbf{di} ‘at, in’}
\label{bkm:Ref319761129}
Prepositional phrases introduced with locative \textitbf{di} ‘at, in’ indicate static location in spatial and non-spatial figurative terms. Most often the preposition denotes location ‘at’ or ‘in’ a referent; depending on its context, though, it is also translatable as ‘on’.



Very commonly, \textitbf{di} ‘at, in’ introduces a peripheral location as in \textitbf{di kampung} ‘in the village’ in (0) or \textitbf{di dia} ‘at hers’ in (0). When following placement verbs such as \textitbf{taru} ‘put’ in (0), \textitbf{di} ‘at, in’ introduces oblique locative arguments that indicate the location of the referent as in \textitbf{di sini} ‘here’. Frequently \textitbf{di} ‘at, in’ also introduces nonverbal predicates as in (0) (see §12.4). Only rarely, \textitbf{di} ‘at, in’ introduces locations encoded by adnominal prepositional phrases as in \textitbf{pasar di bawa tu} ‘the market down there’ in (0). The examples in (0) to (0) also show that \textitbf{di} ‘at, in’ introduces animate and inanimate, as well as nominal and pronominal referents.\footnote{\\
\\
\\
\\
\\
\\
\\
\\
\\
\\
\\
\\
\\
\\
\\
\par In the corpus only the following pronominal complements of \textitbf{di} ‘at, in’ are attested: \textsc{2sg}, \textsc{1pl}, and \textsc{3pl}.}
\end{styleBodyvxafter}

\begin{tabular}{lllllllllll}
\lsptoprule
\label{bkm:Ref339634455}
\gll {waktu} {\multicolumn{2}{l}{saya}} {dengang} {bapa} {tinggal} {\bluebold{di}} {\bluebold{kampung}} {saya} {kerja}\\ %
& time & \multicolumn{2}{l}{\textsc{1sg}} & with & father & stay & at & village & \textsc{1sg} & work\\
& \multicolumn{2}{l}{sperti} & \multicolumn{8}{l}{laki{\Tilde}laki}\\
& \multicolumn{2}{l}{similar.to} & \multicolumn{8}{l}{\textsc{rdp}{\Tilde}husband}\\
\lspbottomrule
\end{tabular}
\ea
\glt 
‘when I and (my) husband (‘father’) were living \bluebold{in the village}, I worked like a man’ \textstyleExampleSource{[081014-007-CvEx.0048]}
\z

\begin{tabular}{llllllllllllll}
\lsptoprule
\label{bkm:Ref348705488}
\gll {\multicolumn{2}{l}{jadi}} {\multicolumn{2}{l}{saya}} {\multicolumn{2}{l}{besar}} {di} {Ida} {dengang} {de} {punya} {laki} {tu}\\ %
& \multicolumn{2}{l}{so} & \multicolumn{2}{l}{\textsc{1sg}} & \multicolumn{2}{l}{be.big} & at & Ida & with & \textsc{3sg} & \textsc{poss} & husband & \textsc{d.dist}\\
& … & \multicolumn{2}{l}{besar} & \multicolumn{2}{l}{\bluebold{di}} & \multicolumn{8}{l}{\bluebold{dia}}\\
&  & \multicolumn{2}{l}{be.big} & \multicolumn{2}{l}{at} & \multicolumn{8}{l}{\textsc{3sg}}\\
\lspbottomrule
\end{tabular}
\ea
\glt 
‘so I grew up with Ida and that husband of hers …, (I) grew up \bluebold{at hers}’ \textstyleExampleSource{[080927-007-CvNP.0017/0019]}
\z

\begin{tabular}{llllllll}
\lsptoprule
\label{bkm:Ref339634457}
\gll {skarang} {kamu} {kasi} {terpol{\Tilde}terpol,} {taru} {\bluebold{di}} {\bluebold{sini}}\\ %
& now & \textsc{2pl} & give & \textsc{rdp}{\Tilde}jerry.can & put & at & \textsc{l.prox}\\
\lspbottomrule
\end{tabular}
\ea
\glt 
‘now you give (me) the jerry cans, put (them) \bluebold{here}’ \textstyleExampleSource{[081110-002-Cv.0065]}
\z

\begin{tabular}{lllll}
\lsptoprule
\label{bkm:Ref339634458}
\gll {sa} {\bluebold{di}} {\bluebold{IPS}} {\bluebold{satu}}\\ %
& \textsc{1sg} & at & social.sciences & one\\
\lspbottomrule
\end{tabular}
\ea
\glt 
[About course tracks in high school:] ‘I (am) \bluebold{in Social Sciences I}’ \textstyleExampleSource{[081023-004-Cv.0020]}
\z

\begin{tabular}{llllll}
\lsptoprule
\label{bkm:Ref348779840}
\gll {\bluebold{pasar}} {\bluebold{di}} {\bluebold{bawa}} {\bluebold{tu}} {raaame}\\ %
& market & at & bottom & \textsc{d.dist} & be.bustling\\
\lspbottomrule
\end{tabular}
\ea
\glt
‘\bluebold{the market down there} is very bustling’ \textstyleExampleSource{[081109-005-JR.0008]}
\end{styleFreeTranslEngxvpt}

\subsection{\textitbf{ke} ‘to’}
\label{bkm:Ref319761130}
Prepositional phrases introduced with allative \textitbf{ke} ‘to’ denote direction toward a referent. Following motion verbs such as \textitbf{lari} ‘run’ in (0) or \textitbf{datang} ‘come’ in (0), \textitbf{ke} ‘to’ introduces oblique locative arguments which indicate the goal of the motion as in \textitbf{ke pante} ‘to the beach’ or \textitbf{ke kitong} ‘to us’, respectively. Allative \textitbf{ke} ‘to’ also very often introduces nonverbal predicates as in (0). The three examples also show that \textitbf{ke} ‘to’ introduces animate and inanimate, as well as nominal and pronominal referents.\footnote{\\
\\
\\
\\
\\
\\
\\
\\
\\
\\
\\
\\
\\
\\
\\
\par In the corpus only the following pronominal complements of \textitbf{ke} ‘to’ are attested: \textsc{1sg}, \textsc{3sg}, \textsc{2pl}.}
\end{styleBodyxafter}

\begin{tabular}{lllll}
\lsptoprule
\label{bkm:Ref366496017}
\gll {dong} {lari} {\bluebold{ke}} {\bluebold{pante}}\\ %
& \textsc{3pl} & run & to & coast\\
\lspbottomrule
\end{tabular}
\ea
\glt 
‘they ran \bluebold{to the beach}’ \textstyleExampleSource{[081115-001a-Cv.0008]}
\z

\begin{tabular}{llllllllll}
\lsptoprule
\label{bkm:Ref339634459}
\gll {…} {dia} {punya} {aroa} {datang} {\bluebold{ke}} {\bluebold{kitong}} {kasi} {tanda}\\ %
&  & \textsc{3sg} & \textsc{poss} & departed.spirit & come & to & \textsc{1pl} & give & sign\\
\lspbottomrule
\end{tabular}
\ea
\glt 
‘[so when there is another person (who) dies in a different village,] (then) his/her departed spirit comes \bluebold{to us} (and) gives (us) a sign’ \textstyleExampleSource{[081014-014-NP.0048]}
\z

\begin{tabular}{llll}
\lsptoprule
\label{bkm:Ref339634460}
\gll {sa} {\bluebold{ke}} {\bluebold{ruma-sakit}}\\ %
& \textsc{1sg} & to & hospital\\
\lspbottomrule
\end{tabular}
\ea
\glt
‘I (went) \bluebold{to the hospital}’ \textstyleExampleSource{[081015-005-NP.0047]}
\end{styleFreeTranslEngxvpt}

\subsection{\textitbf{dari} ‘from’}
\label{bkm:Ref319761131}
Prepositional phrases introduced with elative \textitbf{dari} ‘from’ designate direction away from or out of a source location; depending on its context, though, \textitbf{dari} also translates with ‘of’. Most commonly, the source location is spatial. In addition, \textitbf{dari} ‘from’ expresses non-spatial figurative sources, temporal starting points, and the notions of superiority and dissimilarity in comparison constructions.



Elative \textitbf{dari} ‘from’ forms peripheral adjuncts as in \textitbf{dari blakang} ‘from the back’ in (0). When following motion verbs such as \textitbf{kluar} ‘go out’, it expresses the source of the motion in an oblique argument as in (0). Besides, elative \textitbf{dari} ‘from’ expresses spatial source locations in nonverbal predicates as in (0). Much less often, \textitbf{dari} ‘from’ introduces sources encoded by adnominal prepositional phrases as in (0).
\end{styleBodyvvafter}

\begin{styleExampleTitle}
Introducing spatial source locations
\end{styleExampleTitle}

\begin{tabular}{lllllllll}
\lsptoprule
\label{bkm:Ref339634461}
\gll {de} {tutup} {itu} {spit} {itu} {\bluebold{dari}} {\bluebold{blakang}} {…}\\ %
& \textsc{3sg} & close & \textsc{d.dist} & speedboat & \textsc{d.dist} & from & backside & \\
\lspbottomrule
\end{tabular}
\ea
\glt 
‘(this wave,) it totally covered, what’s-its-name, that speedboat \bluebold{from the back} [to the front]’ \textstyleExampleSource{[080923-015-CvEx.0021]}
\z

\begin{tabular}{lllllllll}
\lsptoprule
\label{bkm:Ref339634462}
\gll {…} {sa} {harus} {kluar} {\bluebold{dari}} {\bluebold{kam}} {\bluebold{pu}} {\bluebold{kluarga}}\\ %
&  & \textsc{1sg} & have.to & go.out & from & \textsc{2pl} & \textsc{poss} & family\\
\lspbottomrule
\end{tabular}
\ea
\glt 
‘[I hadn’t thought that] I would have to depart \bluebold{from your family}’ \textstyleExampleSource{[080919-006-CvNP.0012]}
\z

\begin{tabular}{lllll}
\lsptoprule
\label{bkm:Ref339634463}
\gll {tong} {smua} {\bluebold{dari}} {\bluebold{kampung}}\\ %
& \textsc{1pl} & all & from & village\\
\lspbottomrule
\end{tabular}
\ea
\glt 
‘we all are \bluebold{from the village}’ \textstyleExampleSource{[081010-001-Cv.0084]}
\z

\begin{tabular}{llllllllllllll}
\lsptoprule
\label{bkm:Ref348779841}
\gll {satu} {kali} {ini} {\multicolumn{2}{l}{\bluebold{de}}} {\multicolumn{2}{l}{\bluebold{pu}}} {\multicolumn{2}{l}{\bluebold{bapa}}} {\bluebold{pu}} {\bluebold{temang}} {\bluebold{dari}} {\bluebold{skola},}\\ %
& one & time & \textsc{d.prox} & \multicolumn{2}{l}{\textsc{3sg}} & \multicolumn{2}{l}{\textsc{poss}} & \multicolumn{2}{l}{father} & \textsc{poss} & friend & from & school\\
& \multicolumn{4}{l}{STT} & \multicolumn{2}{l}{dorang} & \multicolumn{2}{l}{pergi} & \multicolumn{5}{l}{…}\\
& \multicolumn{4}{l}{theological.seminary} & \multicolumn{2}{l}{\textsc{3pl}} & \multicolumn{2}{l}{go} & \multicolumn{5}{l}{}\\
\lspbottomrule
\end{tabular}
\ea
\glt 
‘this one time \bluebold{her father’s friends from school}, theological seminary, they went …’ \textstyleExampleSource{[081006-023-CvEx.0062]}
\z


The source location indicated with \textitbf{dari} ‘from’ can also be non-spatial figurative as in the prepositional predicate clauses \textitbf{dari uang} ‘up to the money’ in (0) or \textitbf{dari ko} ‘up to you’ in (0).


\begin{styleExampleTitle}
Introducing non-spatial figurative source locations
\end{styleExampleTitle}

\begin{tabular}{llllllllll}
\lsptoprule
\label{bkm:Ref339634464}
\gll {yo,} {tong} {mo} {biking} {cepat,} {smua} {itu} {\bluebold{dari}} {\bluebold{uang}}\\ %
& yes & \textsc{1pl} & want & make & be.fast & all & \textsc{d.dist} & from & money\\
\lspbottomrule
\end{tabular}
\ea
\glt 
‘yes, we want to do (it) quickly, all that (is) \bluebold{up to the money}’ (Lit. ‘\bluebold{from money}’) \textstyleExampleSource{[080927-006-CvNP.0034]}
\z

\begin{tabular}{llllllll}
\lsptoprule
\label{bkm:Ref339634468}
\gll {pinda} {ke} {IPA} {itu} {\bluebold{dari}} {\bluebold{ko}} {saja}\\ %
& move & to & natural.sciences & \textsc{d.dist} & from & \textsc{2sg} & just\\
\lspbottomrule
\end{tabular}
\ea
\glt 
‘switching (from Social Sciences) to Natural Sciences, that (is) \bluebold{up to you} alone’ (Lit. ‘\bluebold{from you}’) \textstyleExampleSource{[081023-004-Cv.0023]}
\z


The examples in (0) to (0) also illustrate that \textitbf{dari} ‘from’ introduces animate and inanimate, as well as nominal and pronominal referents.\footnote{\\
\\
\\
\\
\\
\\
\\
\\
\\
\\
\\
\\
\\
\\
\\
\par In the corpus one pronominal complement of \textitbf{dari} ‘from’ is unattested, namely \textsc{2pl}.}



Derived from its spatial semantics, \textitbf{dari} ‘from’ also very commonly introduces non-spatial temporal source locations, which are always encoded by peripheral adjuncts. The temporal starting point can be encoded by a noun that indicates time as in \textitbf{dari pagi} ‘from the morning’ in (0) or by a temporal adverb as in \textitbf{dari dulu} ‘from the past’ in (0).
\end{styleBodyvvafter}

\begin{styleExampleTitle}
Introducing temporal starting points
\end{styleExampleTitle}

\begin{tabular}{llllllllll}
\lsptoprule
\label{bkm:Ref339634469}
\gll {tra} {\multicolumn{2}{l}{bole}} {tutup} {pintu,} {\bluebold{dari}} {\bluebold{pagi}} {buka} {pintu}\\ %
& \textsc{neg} & \multicolumn{2}{l}{permitted} & close & door & from & morning & open & door\\
& \multicolumn{2}{l}{sampe} & \multicolumn{7}{l}{malam}\\
& \multicolumn{2}{l}{until} & \multicolumn{7}{l}{night}\\
\lspbottomrule
\end{tabular}
\ea
\glt 
‘you shouldn’t close the door, (you should keep it) open \bluebold{from morning} until night’ \textstyleExampleSource{[081110-008-CvNP.0108]}
\z

\begin{tabular}{lllllll}
\lsptoprule
\label{bkm:Ref339634470}
\gll {jadi} {itu} {suda} {kebiasaang} {\bluebold{dari}} {\bluebold{dulu}}\\ %
& so & \textsc{d.dist} & already & habit & from & first\\
\lspbottomrule
\end{tabular}
\ea
\glt 
‘so that (tradition) has already become a custom \bluebold{from the past}’ \textstyleExampleSource{[081014-007-CvEx.0063]}
\z


Finally, elative \textitbf{dari} ‘from’ is also used in comparative constructions marking degree or identity. In such constructions, \textitbf{dari} ‘from’ functions as the mark of comparison which introduces the standard. In (0), for instance, \textitbf{dari} ‘from’ serves as the mark in a in a comparative construction marking degree, namely superiority, while in (0) it serves as the mark in a comparative construction marking identity, namely dissimilarity (for details on comparative constructions, see §11.5).


\begin{styleExampleTitle}
Introducing standards of comparison
\end{styleExampleTitle}

\begin{tabular}{lllllllll}
\lsptoprule
\label{bkm:Ref339634471}
\gll {…} {dia} {lebi} {besar} {\bluebold{dari}} {\bluebold{smua}} {\bluebold{ana{\Tilde}ana}} {…}\\ %
&  & \textsc{3sg} & more & be.big & from & all & \textsc{rdp}{\Tilde}child & \\
\lspbottomrule
\end{tabular}
\ea
\glt 
‘[in that class] he’s bigger \bluebold{than all the kids} [in it]’ \textstyleExampleSource{[081109-003-JR.0001]}\footnote{\\
\\
\\
\\
\\
\\
\\
\\
\\
\\
\\
\\
\\
\\
\\
\par The original recording says \textitbf{dari smuat} rather than \textitbf{dari smua} ‘than all’. Most likely the speaker wanted to say \textitbf{dari smua temang} ‘than all friends’ but cut himself off to replace \textitbf{temang} ‘friend’ with \textitbf{ana{\Tilde}ana} ‘children’.}
\z

\begin{tabular}{llllll}
\lsptoprule
\label{bkm:Ref339634472}
\gll {sifat} {ini} {\bluebold{laing}} {\bluebold{dari}} {\bluebold{ko}}\\ %
& nature & \textsc{d.prox} & be.different & from & \textsc{2sg}\\
\lspbottomrule
\end{tabular}
\ea
\glt
‘this disposition is different \bluebold{from you}’ \textstyleExampleSource{[081110-008-CvNP.0089]}
\end{styleFreeTranslEngxvpt}

\subsection{\textitbf{sampe} ‘until’}
\label{bkm:Ref319761133}
The preposition \textitbf{sampe} ‘until’ introduces non-spatial temporal endpoints which are always encoded by peripheral adjuncts. Given these semantics, \textitbf{sampe} ‘until’ typically introduces nouns that indicate time, as in \textitbf{sampe sore} ‘until the afternoon’ in (0); that is, animate or pronominal referents of \textitbf{sampe} ‘until’ are unattested.


\begin{styleExampleTitle}
Introducing time-denoting nouns
\end{styleExampleTitle}

\begin{tabular}{lllll}
\lsptoprule
\label{bkm:Ref339634473}
\gll {saya} {tidor} {\bluebold{sampe}} {\bluebold{sore}}\\ %
& \textsc{1sg} & sleep & until & afternoon\\
\lspbottomrule
\end{tabular}
\ea
\glt 
‘I slept \bluebold{until the afternoon}’ \textstyleExampleSource{[081015-005-NP.0033]}
\z


Typically peripheral prepositional phrases can be moved to other positions within the clause with no change in meaning. This does not, however, apply to the example in (0). When the prepositional phrase is moved to the front it denotes the temporal starting rather than the temporal endpoint of \textitbf{tidor} ‘sleep’, as in (0). Hence, the meaning changes to ‘come afternoon’ (literally ‘reaches the afternoon’). One initial explanation for this change in meaning is that the utterance in (0) expresses a sequence of two events, namely the \textitbf{sampe} ‘reaching’ of the afternoon and subsequently the \textitbf{tidor} ‘sleeping’. In that case, \textitbf{sampe sore} does not express the prepositional phrase ‘until afternoon’ but the verbal clause ‘reached the afternoon’ or ‘come afternoon’. This explanation, however, requires further investigation.


\begin{styleExampleTitle}
Clause-initial position
\end{styleExampleTitle}

\begin{tabular}{lllll}
\lsptoprule
\label{bkm:Ref348710061}
\gll {\bluebold{sampe}} {\bluebold{sore}} {saya} {tidor}\\ %
& reach & afternoon & \textsc{1sg} & sleep\\
\lspbottomrule
\end{tabular}
\ea
\glt 
‘\bluebold{come afternoon} I slept’ (Lit. ‘\bluebold{reach the afternoon}’) \textstyleExampleSource{[Elicited BR120817.008]}
\z


Temporal \textitbf{sampe} ‘until’ also introduces temporal adverbs that denote a temporal endpoint as in \textitbf{sampe skarang} ‘until now’ in (0). Overall, however, these constructions are very rare in the corpus.


\begin{styleExampleTitle}
Introducing temporal adverbs
\end{styleExampleTitle}

\begin{tabular}{lllllll}
\lsptoprule
\label{bkm:Ref339634474}
\gll {…} {tapi} {\bluebold{sampe}} {\bluebold{skarang}} {blum} {brangkat}\\ %
&  & but & until & now & not.yet & leave\\
\lspbottomrule
\end{tabular}
\ea
\glt 
‘…but \bluebold{until now} (the team) hasn’t yet left’ \textstyleExampleSource{[081023-002-Cv.0001]}
\z


The preposition \textitbf{sampe} ‘until’ has trial word class membership. Besides introducing prepositional phrases, \textitbf{sampe} is also used as the bivalent verb ‘reach’, or as an anteriority-marking conjunction that introduces temporal clauses (see §14.2.3.3 for its uses as a conjunction; see also §5.14).
\end{styleBodyxvafter}

\subsection{Elision of prepositions encoding location}
\label{bkm:Ref388289764}\label{bkm:Ref374463022}\label{bkm:Ref374463005}\label{bkm:Ref374462679}\label{bkm:Ref374453761}\label{bkm:Ref357843349}\label{bkm:Ref354835855}
Two of the prepositions of location may be omitted if the semantic relationship between the complement and the predicate can be deduced from the context. The prepositions are locative \textitbf{di} ‘at, in’, as illustrated with the contrastive examples in (0) and (0), and allative \textitbf{ke} ‘to’, as shown in (0) and (0).



When locative \textitbf{di} ‘at, in’ introduces a spatial location and combines with a position verb such as \textitbf{tidor} ‘sleep’ as in (0) and (0), the preposition can be elided. Both the preceding verb and the complement of \textitbf{di} ‘at, in’ are already deictic and therefore allow the elision of \textitbf{di} ‘at, in’: the position verb \textitbf{tidor} ‘sleep’ implies the notion of static location, while the complement \textitbf{sana} ‘over there’ signals the position location.
\end{styleBodyvvafter}

\begin{styleExampleTitle}
Prepositional phrases with elided locative \textitbf{di} ‘at, in’
\end{styleExampleTitle}

\begin{tabular}{llllllll}
\lsptoprule
\label{bkm:Ref339634475}
\gll {ko} {punya} {mama} {ada} {tidor} {\bluebold{di}} {\bluebold{sana}}\\ %
& \textsc{2sg} & \textsc{poss} & mother & exist & sleep & at & \textsc{l.dist}\\
\lspbottomrule
\end{tabular}
\ea
\glt 
‘your mother is sleeping \bluebold{over there}’ \textstyleExampleSource{[081006-025-CvEx.0007]}
\z

\begin{tabular}{llllllllll}
\lsptoprule
\label{bkm:Ref339634478}
\gll {a,} {omong} {kosong,} {ko} {masuk} {tidor} {\bluebold{${\varnothing}$}} {\bluebold{sana}} {suda}\\ %
& ah! & way.of.talking & be.empty & \textsc{2sg} & enter & sleep &  & \textsc{l.dist} & already\\
\lspbottomrule
\end{tabular}
\ea
\glt 
‘ah, nonsense, you just go inside (and) \bluebold{sleep over there}’ \textstyleExampleSource{[081023-001-Cv.0057]}
\z


Along similar lines allative \textitbf{ke} ‘to’ can be omitted, when the preposition introduces a location and combines with a motion verb that also expresses direction such as \textitbf{masuk} ‘enter’ in (0) and (0). Again, both the verb and the complement of \textitbf{ke} ‘to’ are deictic, thereby allowing the elision of \textitbf{ke} ‘to’: the verb \textitbf{masuk} ‘enter’ implies the notion of motion and direction, while the complement \textitbf{hutang} ‘forest’ denotes the location toward which the motion is directed.


\begin{styleExampleTitle}
Prepositional phrases with elided allative \textitbf{ke} ‘to’
\end{styleExampleTitle}

\begin{tabular}{llllll}
\lsptoprule
\label{bkm:Ref339634479}
\gll {smua} {masarakat} {masuk} {\bluebold{ke}} {\bluebold{hutang}}\\ %
& all & community & enter & to & forest\\
\lspbottomrule
\end{tabular}
\ea
\glt 
‘the entire community went \bluebold{into the forest}’ \textstyleExampleSource{[081029-005-Cv.0012]}
\z

\begin{tabular}{lllll}
\lsptoprule
\label{bkm:Ref339634480}
\gll {smua} {masuk} {\bluebold{${\varnothing}$}} {\bluebold{hutang}}\\ %
& all & enter &  & forest\\
\lspbottomrule
\end{tabular}
\ea
\glt 
‘all went \bluebold{(into) the forest}’ \textstyleExampleSource{[081029-005-Cv.0111]}
\z


The elision typically affects prepositional phrases with common nouns denoting locations as in (0) or locatives as in (0). In addition, the elision can also affect prepositional phrases with location nouns as in (0) and (0): in (0) the omitted preposition is locative \textitbf{di} ‘at, in’ whereas in (0) it is allative \textitbf{ke} ‘to’.


\begin{styleExampleTitle}
Prepositional phrases with elided preposition and location noun complement
\end{styleExampleTitle}

\begin{tabular}{lllllll}
\lsptoprule
\label{bkm:Ref339634481}
\gll {baru} {kitong} {taru} {\bluebold{${\varnothing}$}} {\bluebold{depang}} {to?}\\ %
& and.then & \textsc{1pl} & put &  & front & right?\\
\lspbottomrule
\end{tabular}
\ea
\glt 
‘and then we put (the cake down) \bluebold{in front}, right?’ \textstyleExampleSource{[081011-005-Cv.0031]}
\z

\begin{tabular}{lllllllllll}
\lsptoprule
\label{bkm:Ref339634482}
\gll {itu} {yang} {sa} {bilang,} {kalo} {dong} {pinda} {\bluebold{${\varnothing}$}} {\bluebold{sebla}} {bole}\\ %
& \textsc{d.dist} & \textsc{rel} & \textsc{1sg} & say & if & \textsc{3pl} & move &  & side & may\\
\lspbottomrule
\end{tabular}
\ea
\glt 
‘that’s why I said, ‘if they move \bluebold{to the (other) side} (that’s) alright’’ \textstyleExampleSource{[081011-001-Cv.0144]}
\z


Elision of \textitbf{di} ‘at, in’ and \textitbf{ke} ‘to’ is not possible, though, in nonverbal prepositional predicate clauses as this would create nominal clauses with unacceptable semantics. This is illustrated with elided \textitbf{di} ‘at, in’ in (0), which is based on the example in (0), and with elided \textitbf{ke} ‘to’ in (0), which is based on the example in (0).


\begin{styleExampleTitle}
Nonverbal prepositional predicate clauses with elided locative \textitbf{di} ‘at, in’ and allative \textitbf{ke} ‘to’
\end{styleExampleTitle}

\begin{tabular}{llllll}
\lsptoprule
\label{bkm:Ref363468808}
\gll {*} {sa} {\bluebold{${\varnothing}$}} {\bluebold{IPS}} {\bluebold{satu}}\\ %
&  & \textsc{1sg} &  & social.sciences & one\\
\lspbottomrule
\end{tabular}
\ea
\glt 
[About course tracks in high school:] (‘I (am) \bluebold{Social Sciences I}’) \textstyleExampleSource{[based on 081023-004-Cv.0020]}
\z

\begin{tabular}{lllll}
\lsptoprule
\label{bkm:Ref363468809}
\gll {*} {sa} {\bluebold{${\varnothing}$}} {\bluebold{ruma-sakit}}\\ %
&  & \textsc{1sg} &  & hospital\\
\lspbottomrule
\end{tabular}
\ea
\glt 
(‘I (am) \bluebold{the hospital}’) \textstyleExampleSource{[based on 081015-005-NP.0047]}
\z


Elision of elative \textitbf{dari} ‘from’ and temporal \textitbf{sampe} ‘until’ is also not possible, as illustrated in (0) and (0). In the example in (0), which is based on (0), elative \textitbf{dari} ‘from’ is omitted, resulting in an ungrammatical utterance. In the example in (0), which is based on (0), temporal \textitbf{sampe} ‘until’ is elided. The result is a change in meaning of the entire utterance: ‘I slept (the entire) afternoon’.


\begin{styleExampleTitle}
Prepositional phrases with elided elative \textitbf{dari} ‘from’ and temporal \textitbf{sampe} ‘until’
\end{styleExampleTitle}

\begin{tabular}{llllllllll}
\lsptoprule
\label{bkm:Ref348712318}
\gll {*} {…} {sa} {harus} {kluar} {\bluebold{${\varnothing}$}} {\bluebold{kam}} {\bluebold{pu}} {\bluebold{kluarga}}\\ %
&  &  & \textsc{1sg} & have.to & go.out &  & \textsc{2pl} & \textsc{poss} & family\\
\lspbottomrule
\end{tabular}
\ea
\glt 
(‘[I hadn’t thought that] I would have to depart \bluebold{your family}’) \textstyleExampleSource{[Elicited BR120817.009]}
\z

\begin{tabular}{lllll}
\lsptoprule
\label{bkm:Ref348712319}
\gll {saya} {tidor} {\bluebold{${\varnothing}$}} {\bluebold{sore}}\\ %
& \textsc{1sg} & sleep &  & afternoon\\
\lspbottomrule
\end{tabular}
\ea
\glt
‘I slept (the entire) \bluebold{afternoon}’ \textstyleExampleSource{[Elicited BR120817.010]}
\end{styleFreeTranslEngxvpt}

\section{Prepositions encoding accompaniment/instruments, goals, and benefaction}
\label{bkm:Ref320188450}
Papuan Malay employs four prepositions that encode accompaniment/instruments, goals, and benefaction: comitative \textitbf{dengang} ‘with’ (§10.2.1), goal-oriented \textitbf{sama} ‘to’ (§10.2.2), benefactive \textitbf{untuk} ‘for’ (§10.2.3) and \textitbf{buat} ‘for’ (§10.2.4).
\end{styleBodyxvafter}

\subsection{\textitbf{dengang} ‘with’}
\label{bkm:Ref319761134}
Prepositional phrases introduced with comitative \textitbf{dengang} ‘with’, with its short form \textitbf{deng}, typically express accompaniment with animate or inanimate associates. Also very often, \textitbf{dengang} ‘with’ introduces instruments. In addition, \textitbf{dengang} ‘with’ introduces objects of mental verbs and the notion of identity in comparison constructions.



The associates introduced with \textitbf{dengang} ‘with’ are most commonly animate human as in \textitbf{deng mama-tua} ‘with aunt’ in (0), \textitbf{deng de pu temang{\Tilde}temang} ‘with his friends’ in (0) or in \textitbf{deng kamu} ‘with you’ in (0). These examples also show that the complements of \textitbf{dengang} ‘with’ can be nouns or personal pronouns. Besides animate associates, \textitbf{dengang} ‘with’ also introduces inanimate associates, as in \textitbf{deng motor} ‘with (his) motorbike’ in (0) or in \textitbf{deng} \textitbf{itu} ‘with those (spices)’ in (0). The associates introduced with \textitbf{dengang} ‘with’ are either encoded in peripheral adjuncts as in (0), or (0) to (0) or in nonverbal predicates as in (0). The example in (0) also illustrates that prepositional phrases functioning as nonverbal predicates can be modified by adverbs such as prospective \textitbf{masi} ‘still’; such modification is unattested for prepositional phrases having other functions.
\end{styleBodyvvafter}

\begin{styleExampleTitle}
Introducing associates
\end{styleExampleTitle}

\begin{tabular}{llllllll}
\lsptoprule
\label{bkm:Ref366497834}
\gll {sebentar} {Hurki} {datang} {ko} {pulang} {\bluebold{deng}} {\bluebold{mama-tua}}\\ %
& in.a.moment & Hurki & come & \textsc{2sg} & go.home & with & aunt\\
\lspbottomrule
\end{tabular}
\ea
\glt 
‘in a moment (when) Hurki comes, you’ll go home \bluebold{with me} (‘\bluebold{aunt}’)’ \textstyleExampleSource{[081011-006-Cv.0003]}
\z

\begin{tabular}{lllllll}
\lsptoprule
\label{bkm:Ref339634483}
\gll {Roni} {masi} {\bluebold{deng}} {\bluebold{de}} {\bluebold{pu}} {\bluebold{temang{\Tilde}temang}}\\ %
& Roni & still & with & \textsc{3sg} & \textsc{poss} & \textsc{rdp}{\Tilde}friend\\
\lspbottomrule
\end{tabular}
\ea
\glt 
‘Roni is still \bluebold{with his friends}’ \textstyleExampleSource{[081006-031-Cv.0011]}
\z

\begin{tabular}{llllllll}
\lsptoprule
\label{bkm:Ref348713251}
\gll {slama} {sa} {tinggal} {\bluebold{deng}} {\bluebold{kamu}} {sa} {kerja}\\ %
& as.long.as & \textsc{1sg} & stay & with & \textsc{2pl} & \textsc{1sg} & work\\
\lspbottomrule
\end{tabular}
\ea
\glt 
‘as long as I stayed \bluebold{with you} I worked’ \textstyleExampleSource{[080919-006-CvNP.0014]}
\z

\begin{tabular}{lllll}
\lsptoprule
\label{bkm:Ref339634484}
\gll {de} {jatu} {\bluebold{deng}} {\bluebold{motor}}\\ %
& \textsc{3sg} & fall & with & motorbike\\
\lspbottomrule
\end{tabular}
\ea
\glt 
‘he fell \bluebold{with (his) motorbike}’ \textstyleExampleSource{[081006-020-Cv.0008]}
\z

\begin{tabular}{lllllllll}
\lsptoprule
\label{bkm:Ref339634485}
\gll {itu} {nanti} {kitong} {tumbuk} {baru} {masak} {\bluebold{deng}} {\bluebold{itu}}\\ %
& \textsc{d.dist} & very.soon & \textsc{1pl} & pound & and.then & cook & with & \textsc{d.dist}\\
\lspbottomrule
\end{tabular}
\ea
\glt 
‘later we’ll pound those (spices and) and then cook \bluebold{with them}’ \textstyleExampleSource{[081010-001-Cv.0196]}
\z


Instruments introduced with comitative \textitbf{dengang} ‘with’ are expressed in peripheral adjuncts as in \textitbf{deng pisow} ‘with a knife’ in (0).


\begin{styleExampleTitle}
Introducing instruments
\end{styleExampleTitle}

\begin{tabular}{lllllll}
\lsptoprule
\label{bkm:Ref339634490}
\gll {bapa} {de} {pukul} {sa} {\bluebold{deng}} {\bluebold{pisow}}\\ %
& father & \textsc{3sg} & hit & \textsc{1sg} & with & knife\\
\lspbottomrule
\end{tabular}
\ea
\glt 
‘(my) husband stabbed me \bluebold{with a knife}’ \textstyleExampleSource{[081011-023-Cv.0167]}
\z


In addition, comitative \textitbf{dengang} ‘with’ introduces oblique arguments for mental verbs such as \textitbf{mara} ‘feel angry (about)’ in (0), \textitbf{takut} ‘feel afraid (of)’ in (0), or \textitbf{perlu} ‘need’ in (0).\footnote{\\
\\
\\
\\
\\
\\
\\
\\
\\
\\
\\
\\
\\
\\
\\
\par Bivalent verbs such as \textitbf{mara} ‘feel angry (about)’ or \textitbf{takut} ‘feel afraid (of)’ do not require but allow two syntactic arguments (see §5.3.1 and §11.1). That is, speakers quite commonly encode patients such as \textitbf{orang} in (0) or \textitbf{setang} ‘evil spirit’ in (0) as oblique arguments rather than as direct objects.}


\begin{styleExampleTitle}
Introducing objects of mental verbs
\end{styleExampleTitle}

\begin{tabular}{lllllllll}
\lsptoprule
\label{bkm:Ref339634491}
\gll {kalo} {saya} {mara} {\bluebold{dengang}} {\bluebold{orang}} {begitu} {sa} {takut}\\ %
& if & \textsc{1sg} & feel.angry(.about) & with & person & like.that & \textsc{1sg} & feel.afraid(.of)\\
\lspbottomrule
\end{tabular}
\ea
\glt 
‘if I was angry \bluebold{with someone} like that I’d feel afraid’ \textstyleExampleSource{[081110-008-CvNP.0067]}
\z

\begin{tabular}{llllllll}
\lsptoprule
\label{bkm:Ref339634492}
\gll {adu,} {kang} {dong} {terlalu} {takut} {\bluebold{dengang}} {\bluebold{setang}}\\ %
& oh.no! & you.know & \textsc{3pl} & too & feel.afraid(.of) & with & evil.spirit\\
\lspbottomrule
\end{tabular}
\ea
\glt 
‘oh no, you know, they feel too afraid \bluebold{of evil spirits}’ \textstyleExampleSource{[081025-006-Cv.0198]}
\z

\begin{tabular}{llllll}
\lsptoprule
\label{bkm:Ref339634493}
\gll {mama-ade} {sa} {perlu} {\bluebold{deng}} {\bluebold{mama-ade}}\\ %
& aunt & \textsc{1sg} & need & with & aunt\\
\lspbottomrule
\end{tabular}
\ea
\glt 
‘aunt, I need \bluebold{you} (‘\bluebold{aunt}’)’ (Lit. ‘need \bluebold{with aunt}’) \textstyleExampleSource{[081014-004-Cv.0004]}
\z


Comitative \textitbf{dengang} ‘with’ is also used in comparative constructions. As the mark of comparison, \textitbf{dengang} ‘with’ introduces the standard of comparison in identity-marking constructions. In (0), for example, \textitbf{dengang} ‘with’ serves as the mark in a similarity construction, while in (0) it is the mark in a dissimilarity construction (for more details on comparative constructions, see §11.5).


\begin{styleExampleTitle}
Introducing standards of comparison
\end{styleExampleTitle}

\begin{tabular}{llllll}
\lsptoprule
\label{bkm:Ref339634494}
\gll {de} {sombong} {sama} {\bluebold{deng}} {\bluebold{ko}}\\ %
& \textsc{3sg} & be.arrogant & same & with & \textsc{2sg}\\
\lspbottomrule
\end{tabular}
\ea
\glt 
‘she’ll be as arrogant \bluebold{as you} (are)’ \textstyleExampleSource{[081006-005-Cv.0002]}
\z

\begin{tabular}{lllllll}
\lsptoprule
\label{bkm:Ref339634495}
\gll {orang} {Papua} {beda} {\bluebold{dengang}} {\bluebold{orang}} {\bluebold{Indonesia}}\\ %
& person & Papua & be.different & with & person & Indonesia\\
\lspbottomrule
\end{tabular}
\ea
\glt 
‘Papuans are different \bluebold{from Indonesians}’ \textstyleExampleSource{[081029-002-Cv.0009]}
\z


The preposition \textitbf{dengang} ‘with’ has dual word class membership; it is also used as an addition-marking conjunction (§14.2.1.1; see also §5.14).
\end{styleBodyxvafter}

\subsection{\textitbf{sama} ‘to’}
\label{bkm:Ref319761138}
The goal preposition \textitbf{sama} ‘to’ is rather general in its meaning. Typically it translates with ‘to’ but depending on its context it also translates with ‘of, from, with’. The complement always denotes an animate referent which can be encoded in a noun or in a personal pronoun.



As the exchange in (0) shows, \textitbf{sama} ‘to’ usually introduces oblique goal or recipient arguments of transfer verbs such as \textitbf{bawa} ‘bring’ in (0) or \textitbf{kasi} ‘give’ in (0).
\end{styleBodyvvafter}

\begin{styleExampleTitle}
Introducing goals or recipients
\end{styleExampleTitle}

\begin{tabular}{lllllllllll}
\lsptoprule
\label{bkm:Ref339634496}\label{bkm:Ref318979891}
\gll {\label{bkm:Ref320373855}} {Speaker-1:} {ko} {bawa} {ke} {sana} {ko} {\bluebold{bawa}} {\bluebold{sama}} {\bluebold{ade}}\\ %
&  &  & \textsc{2sg} & take & to & \textsc{l.dist} & \textsc{2sg} & take & to & ySb\\
\lspbottomrule
\end{tabular}
\begin{styleFreeTranslIndentiicmEng}
Speaker-1: ‘bring (the ball) over there, \bluebold{bring} (it) \bluebold{to (your) younger cousin}’
\end{styleFreeTranslIndentiicmEng}

\begin{tabular}{llllllll} & \label{bkm:Ref320373857} & Speaker-2: & e, & \bluebold{kasi} & bola & \bluebold{sama} & \bluebold{ade}\\
\lsptoprule
&  &  & hey! & give & ball & to & ySb\\
\lspbottomrule
\end{tabular}
\begin{styleFreeTranslIndentiicmEng}
Speaker-2: ‘hey, \bluebold{give} the ball \bluebold{to (your) younger cousin}’ \textstyleExampleSource{[081011-009-Cv.0015-0016]}
\end{styleFreeTranslIndentiicmEng}


Also very commonly, \textitbf{sama} ‘to’ introduces oblique addressee arguments for communication verbs such as \textitbf{bicara} ‘speak’ in (0) or \textitbf{minta} ‘request’ in (0).


\begin{styleExampleTitle}
Introducing addressees
\end{styleExampleTitle}

\begin{tabular}{lllllllllll}
\lsptoprule
\label{bkm:Ref339634497}
\gll {sa} {minta} {maaf,} {e} {tadi} {sa} {\bluebold{bicara}} {kasar} {\bluebold{sama}} {\bluebold{ko}}\\ %
& \textsc{1sg} & ask & pardon & uh & earlier & \textsc{1sg} & speak & be.coarse & to & \textsc{2sg}\\
\lspbottomrule
\end{tabular}
\ea
\glt 
‘I apologize, uh, a short while ago I \bluebold{spoke to you} harshly’ \textstyleExampleSource{[081115-001a-Cv.0277]}
\z

\begin{tabular}{llllllll}
\lsptoprule
\label{bkm:Ref339634499}
\gll {de} {\bluebold{minta}} {apa} {\bluebold{sama}} {\bluebold{kitorang}} {kitorang} {kasi}\\ %
& \textsc{3sg} & ask & what & to & \textsc{1pl} & \textsc{1pl} & give\\
\lspbottomrule
\end{tabular}
\ea
\glt 
‘(whenever) she (our daughter) \bluebold{asks us} (for) something, we give (it to her)’ \textstyleExampleSource{[081006-025-CvEx.0022]}
\z


Goal preposition \textitbf{sama} ‘to’ denotes the goal of a transfer or communication without concurrently marking this goal as the beneficiary of the event talked about. In this it contrasts with benefactive \textitbf{untuk} ‘for’ and \textitbf{buat} ‘for’; compare the examples in (0) and (0) with \textitbf{kasi}/\textitbf{bicara untuk} ‘give/speak to and for’ in (0) and (0) in §10.2.3 (p. \pageref{bkm:Ref436750739}) and with \textitbf{kasi}/\textitbf{bicara buat} ‘give/speak to and for’ in (0) and (0) in §10.2.4 (p. \pageref{bkm:Ref436750800}).



In addition, \textitbf{sama} ‘to’ introduces oblique arguments of mental verbs such as \textitbf{ingat} ‘remember’ in (0), \textitbf{mara} ‘feel angry (about)’ in (0), or \textitbf{takut} ‘feel afraid (of)’ in (0). Most of the objects of mental verbs introduced with \textitbf{sama} ‘to’ can also occur with comitative \textitbf{dengang} ‘with’ (§10.2.1): compare \textitbf{mara sama} ‘feel angry about’ in (0) with \textitbf{mara dengang} ‘feel angry with’ in (0), or \textitbf{takut sama} ‘feel afraid of’ in (0) with \textitbf{takut dengang} ‘feel afraid of’ in (0).{223} Overall, however, the range of verbs is smaller for \textitbf{sama} ‘to’ than for comitative \textitbf{dengang} ‘with’.
\end{styleBodyvafter}


The semantic distinctions between \textitbf{sama} ‘to’ and \textitbf{dengang} ‘with’ are subtle. When speakers want to emphasize the agent of the mental verb they employ \textitbf{sama} ‘to’. If they want to signal that the object of the mental verb is also involved in the mental process talked about, they use comitative \textitbf{dengang} ‘with’. The contrastive examples in (0) and (0) illustrate this distinction. In (0) \textitbf{sama} ‘to’ emphasizes the fact that the agent \textitbf{de} ‘3\textsc{sg}’ \textitbf{mara} ‘feels angry’ about the patient \textitbf{pak Bolikarfus} ‘Mr. Bolikarfus’ whereas the patient himself is not involved in this mental process. By contrast in (0) \textitbf{deng(ang)} ‘with’ signals that in some ways the patient \textitbf{pak Bolikarfus} ‘Mr. Bolikarfus’ has contributed to the agent’s anger. Likewise, in (0) \textitbf{sama} ‘to’ focuses on the fact that the agent \textitbf{dia} ‘3\textsc{sg}’ \textitbf{takut} ‘feels afraid (of)’; again, the patient \textitbf{ana{\Tilde}ana Tuhang} ‘God’s children’ is not involved in this mental process. In (0), by contrast, \textitbf{deng(ang)} ‘with’ signals that the patient \textitbf{ana{\Tilde}ana Tuhang} ‘God’s children’ has contributed in some ways to the agent’s fear.
\end{styleBodyvvafter}

\begin{styleExampleTitle}
Introducing objects of mental verbs\footnote{\\
\\
\\
\\
\\
\\
\\
\\
\\
\\
\\
\\
\\
\\
\\
\par The examples in (0) and (0) are taken from the corpus while the examples in (0) and (0) are elicited.}
\end{styleExampleTitle}

\begin{tabular}{llllllllll}
\lsptoprule
\label{bkm:Ref339635081}
\gll {biar} {dia} {masi} {muda} {tapi} {Fitri} {ingat} {\bluebold{sama}} {\bluebold{Roni}}\\ %
& although & \textsc{3sg} & still & be.young & but & Fitri & remember & to & Roni\\
\lspbottomrule
\end{tabular}
\ea
\glt 
‘even though she was still young, Fitri was thinking \bluebold{of Roni}’ \textstyleExampleSource{[081006-024-CvEx.0067]}
\z

\begin{tabular}{lllllll}
\lsptoprule
\label{bkm:Ref339634500}\label{bkm:Ref318981330}
\gll {\label{bkm:Ref316473796}} {de} {mara} {\bluebold{sama}} {\bluebold{pak}} {\bluebold{Bolikarfus}}\\ %
&  & \textsc{3sg} & feel.angry(.about) & to & father & Bolikarfus\\
\lspbottomrule
\end{tabular}
\begin{styleFreeTranslIndentiicmEng}
‘he was angry \bluebold{about Mr. Bolikarfus}’ \textstyleExampleSource{[081014-016-Cv.0042]}
\end{styleFreeTranslIndentiicmEng}

\begin{tabular}{lllllll} & \label{bkm:Ref316473799} & de & mara & \bluebold{deng} & \bluebold{pak} & \bluebold{Bolikarfus}\\
\lsptoprule
&  & \textsc{3sg} & feel.angry(.about) & with & father & Bolikarfus\\
\lspbottomrule
\end{tabular}
\begin{styleFreeTranslIndentiicmEng}
‘he was angry \bluebold{with Mr. Bolikarfus}’ \textstyleExampleSource{[Elicited BR120817.001]}
\end{styleFreeTranslIndentiicmEng}

\begin{tabular}{llllllll}
\lsptoprule
\label{bkm:Ref339635080}\label{bkm:Ref318989523}
\gll {\label{bkm:Ref323024251}} {memang} {dia} {takut} {\bluebold{sama}} {\bluebold{ana{\Tilde}ana}} {\bluebold{Tuhang}}\\ %
&  & indeed & \textsc{3sg} & feel.afraid(.of) & to & \textsc{rdp}{\Tilde}child & God\\
\lspbottomrule
\end{tabular}
\begin{styleFreeTranslIndentiicmEng}
‘(that evil spirit) indeed he/she feels afraid \bluebold{of God’s children}’ \textstyleExampleSource{[081006-022-CvEx.0175]}
\end{styleFreeTranslIndentiicmEng}

\begin{tabular}{llllllll} & \label{bkm:Ref323024252} & memang & dia & takut & \bluebold{deng} & \bluebold{ana{\Tilde}ana} & \bluebold{Tuhang}\\
\lsptoprule
&  & indeed & \textsc{3sg} & feel.afraid(.of) & with & \textsc{rdp}{\Tilde}child & God\\
\lspbottomrule
\end{tabular}
\begin{styleFreeTranslIndentiicmEng}
‘(that evil spirit) indeed he/she feels afraid \bluebold{of God’s children}’ \textstyleExampleSource{[Elicited BR120817.001]}
\end{styleFreeTranslIndentiicmEng}


Furthermore, although not very frequently, \textitbf{sama} ‘to’ introduces animate associates. As with comitative \textitbf{dengang} ‘with’ (§10.2.1), associates are expressed in peripheral adjuncts as in \textitbf{sama dorang} ‘with them’ in (0) or in nonverbal predicates as in \textitbf{sama saya} ‘with me’ in (0).


\begin{styleExampleTitle}
Introducing animate associates
\end{styleExampleTitle}

\begin{tabular}{lllll}
\lsptoprule
\label{bkm:Ref339635082}
\gll {Papeas} {maing{\Tilde}maing} {\bluebold{sama}} {\bluebold{dorang}}\\ %
& Papeas & \textsc{rdp}{\Tilde}play & to & \textsc{3pl}\\
\lspbottomrule
\end{tabular}
\ea
\glt 
‘Papeas is going to play \bluebold{with them}’ \textstyleExampleSource{[080918-001-CvNP.0040]}
\z

\begin{tabular}{llllll}
\lsptoprule
\label{bkm:Ref339635083}
\gll {hanya} {tiga} {saja} {\bluebold{sama}} {\bluebold{saya}}\\ %
& only & three & just & to & \textsc{1sg}\\
\lspbottomrule
\end{tabular}
\ea
\glt 
‘just only three (of my children) are \bluebold{with me}’ \textstyleExampleSource{[081006-024-CvEx.0001]}
\z


The goal preposition \textitbf{sama} ‘to’ has trial word class membership. That is, besides being used as a preposition, it is also used as the stative verb \textitbf{sama} ‘be same’ and, although not very frequently, as an addition-marking conjunction (see §5.14; see also §14.2.1.3 for its uses as a conjunction).\footnote{\\
\\
\\
\\
\\
\\
\\
\\
\\
\\
\\
\\
\\
\\
\\
\label{bkm:Ref438368001}\par In terms of its etymology, {U. Tadmor (p.c. 2013)} notes that “\textitbf{sama}\textit{ }was borrowed from Sanskrit into Malay in ancient times with the meaning ‘same’. Much later it also came to mean ‘with’ in Bazaar Malay”.}
\end{styleBodyxvafter}

\subsection{\textitbf{untuk} ‘for’}
\label{bkm:Ref319602128}
The benefactive preposition \textitbf{untuk} usually translates with ‘for’; depending on its context, however, it also translates with ‘to, about’. The preposition introduces animate and inanimate, as well as nominal and pronominal referents. In most cases, the referents are beneficiaries or recipients (148 tokens). In this regard, \textitbf{untuk} ‘for’ is similar to benefactive \textitbf{buat} ‘for’ (§10.2.4). Contrasting with \textitbf{buat} ‘for’, however, \textitbf{untuk} ‘for’ has a wider distribution and more functions in that it (1) combines with demonstratives, (2) introduces inanimate referents, and (3) introduces circumstance.



Beneficiaries introduced with \textitbf{untuk} ‘for’ are typically animate human as in (0), (0) or (0). The beneficiary can, however, also be animate nonhuman as in \textitbf{untuk anjing dorang} ‘for the dogs’ in (0).
\end{styleBodyvafter}


Usually, \textitbf{untuk} ‘for’ follows bivalent verbs such as \textitbf{buat} ‘make, do’ or \textitbf{biking} ‘make’ and introduces beneficiaries encoded by peripheral adjuncts, as in (0) or (0), respectively. Only rarely is the beneficiary encoded by a nonverbal prepositional predicate (2 tokens), as in \textitbf{untuk tamu} ‘for the guests’ in (0) or an adnominal prepositional phrase (2 tokens), as in \textitbf{untuk kafir} ‘for unbelievers’ in (0). As for the low token frequencies of two each, one consultant suggested that these constructions are not native Papuan Malay but represent instances of code-switching with Indonesian. The low frequencies support this statement.
\end{styleBodyvvafter}

\begin{styleExampleTitle}
Introducing animate beneficiaries
\end{styleExampleTitle}

\begin{tabular}{llllll}
\lsptoprule
\label{bkm:Ref366499720}
\gll {Tuhang} {buat} {mujisat} {\bluebold{untuk}} {\bluebold{kita}}\\ %
& God & make & miracle & for & \textsc{1pl}\\
\lspbottomrule
\end{tabular}
\ea
\glt 
‘God made a miracle \bluebold{for us}’ \textstyleExampleSource{[080917-008-NP.0163]}
\z

\begin{tabular}{lllllllllll}
\lsptoprule
\label{bkm:Ref339635084}
\gll {…} {yang} {sa} {pu} {bini} {biking} {malam} {\bluebold{untuk}} {\bluebold{anjing}} {\bluebold{dorang}}\\ %
&  & \textsc{rel} & \textsc{1sg} & \textsc{poss} & wife & make & night & for & dog & \textsc{3pl}\\
\lspbottomrule
\end{tabular}
\ea
\glt 
‘[I fed the dogs with papeda] which my wife had made in the evening \bluebold{for the dogs}’ \textstyleExampleSource{[080919-003-NP.0002]}
\z

\begin{tabular}{llllll}
\lsptoprule
\label{bkm:Ref339635085}
\gll {ikang} {sedikit,} {itu} {\bluebold{untuk}} {\bluebold{tamu}}\\ %
& fish & few & \textsc{d.dist} & for & guest\\
\lspbottomrule
\end{tabular}
\ea
\glt 
‘(as for) the few fish, those are \bluebold{for the guests}’ \textstyleExampleSource{[081014-011-CvEx.0008]}
\z

\begin{tabular}{llllllll}
\lsptoprule
\label{bkm:Ref339635086}
\gll {di} {sana} {kang} {masi} {\bluebold{tempat}} {\bluebold{untuk}} {\bluebold{kafir}}\\ %
& at & \textsc{l.dist} & you.know & still & place & for & unbeliever\\
\lspbottomrule
\end{tabular}
\ea
\glt 
‘(the area) over there, you know, is still a location \bluebold{for unbelievers}’ \textstyleExampleSource{[081011-022-Cv.0238]}
\z


With transfer verbs, \textitbf{untuk} ‘for’ introduces benefactive recipients, and with communication verbs it introduces benefactive addressees. That is, the referent is not merely a recipient or addressee. Benefactive \textitbf{untuk} ‘for’ indicates that the referent is also the beneficiary of the transfer or communication, hence ‘benefactive recipient’ and ‘benefactive addressee’. This is illustrated with \textitbf{kasi untuk} ‘give to and for’ in (0), and \textitbf{bicara untuk} ‘speak to and for’ in (0).


\begin{styleExampleTitle}
Introducing benefactive recipients and addressees
\end{styleExampleTitle}

\begin{tabular}{llllll}
\lsptoprule
\label{bkm:Ref436750739}\label{bkm:Ref339635087}
\gll {sa} {kasi} {hadia} {\bluebold{untuk}} {\bluebold{kamu}}\\ %
& \textsc{1sg} & give & gift & for & \textsc{2pl}\\
\lspbottomrule
\end{tabular}
\ea
\glt 
‘I’ll give gifts \bluebold{to you for your benefit}’ \textstyleExampleSource{[080922-001a-CvPh.1332]}
\z

\begin{tabular}{llllllllll}
\lsptoprule
\label{bkm:Ref339635088}
\gll {jadi} {\multicolumn{2}{l}{sperti}} {itu,} {harus} {bicara} {\bluebold{untuk}} {\bluebold{dorang},} {ceritra}\\ %
& so & \multicolumn{2}{l}{similar.to} & \textsc{d.dist} & have.to & speak & for & \textsc{3pl} & tell\\
& \multicolumn{2}{l}{\bluebold{untuk}} & \multicolumn{7}{l}{\bluebold{dorang}}\\
& \multicolumn{2}{l}{for} & \multicolumn{7}{l}{\textsc{3pl}}\\
\lspbottomrule
\end{tabular}
\ea
\glt 
‘so it’s like that, (we) have to speak \bluebold{to them} (our children), talk \bluebold{to them for their benefit}’ \textstyleExampleSource{[081014-007-CvEx.0136]}
\z


Besides introducing animate referents, \textitbf{untuk} ‘for’ also introduces inanimate beneficiaries that are concrete, abstract, or temporal. In (0), the beneficiary is inanimate concrete: \textitbf{kamar mandi} ‘the bathroom’. In (0), the beneficiary is inanimate abstract: distal demonstrative \textitbf{itu} ‘\textsc{d.dist}’ summarizes the speaker’s previous statements about balanced birth rates across families related by marriage. In (0) and (0), the beneficiary is temporal: \textitbf{taung ini} ‘this year’ in (0) and \textitbf{besok} ‘tomorrow’ in (0). Overall, however, these uses of benefactive \textitbf{untuk} ‘for’ are quite rare, with the corpus including only very few examples.


\begin{styleExampleTitle}
Introducing inanimate beneficiaries
\end{styleExampleTitle}

\begin{tabular}{lllllll}
\lsptoprule
\label{bkm:Ref339635091}
\gll {tong} {mo} {pake} {\bluebold{untuk}} {\bluebold{kamar}} {\bluebold{mandi}}\\ %
& \textsc{1pl} & want & use & for & room & bathe\\
\lspbottomrule
\end{tabular}
\ea
\glt 
‘we want to use (the corrugated iron sheets) \bluebold{for the bathroom} (roof)’ \textstyleExampleSource{[080925-003-Cv.0005]}
\z

\begin{tabular}{lllllllllllllll}
\lsptoprule
\label{bkm:Ref339635092}
\gll {…} {\multicolumn{2}{l}{lahir}} {\multicolumn{2}{l}{ana}} {\multicolumn{2}{l}{suku}} {\multicolumn{2}{l}{A.,}} {a,} {\multicolumn{2}{l}{saya}} {\multicolumn{2}{l}{lahir}}\\ %
&  & \multicolumn{2}{l}{give.birth} & \multicolumn{2}{l}{child} & \multicolumn{2}{l}{ethnic.group} & \multicolumn{2}{l}{A.} & ah! & \multicolumn{2}{l}{\textsc{1sg}} & \multicolumn{2}{l}{give.birth}\\
& \multicolumn{2}{l}{suku} & \multicolumn{2}{l}{Y.} & \multicolumn{2}{l}{…} & \multicolumn{2}{l}{tujuangnya} & \multicolumn{3}{l}{hanya} & \multicolumn{2}{l}{\bluebold{untuk}} & \bluebold{itu}\\
& \multicolumn{2}{l}{ethnic.group} & \multicolumn{2}{l}{Y.} & \multicolumn{2}{l}{} & \multicolumn{2}{l}{purpose:\textsc{3possr}} & \multicolumn{3}{l}{only} & \multicolumn{2}{l}{for} & \textsc{d.dist}\\
\lspbottomrule
\end{tabular}
\ea
\glt 
[About the exchange of bride-price children:] ‘(our daughter) will give birth to a child (for) the A. family, well, I give birth for the Y. family … its purpose is only \bluebold{for that} (namely,\bluebold{ }a balanced birth rate across families)’ \textstyleExampleSource{[081006-024-CvEx.0079]}
\z

\begin{tabular}{lllllllll}
\lsptoprule
\label{bkm:Ref339635093}
\gll {\bluebold{untuk}} {\bluebold{taung}} {\bluebold{ini}} {kam} {kas} {los} {sa} {dulu}\\ %
& for & year & \textsc{d.prox} & \textsc{2pl} & give & loosen & \textsc{1sg} & first\\
\lspbottomrule
\end{tabular}
\ea
\glt 
‘\bluebold{for (the rest of) this year} you release me (from my duties) for now’ \textstyleExampleSource{[080922-002-Cv.0084]}
\z

\begin{tabular}{llllllllll}
\lsptoprule
\label{bkm:Ref339635094}
\gll {tong} {dari} {sa} {pu} {temang} {pinjam} {trening} {\bluebold{untuk}} {\bluebold{besok}}\\ %
& \textsc{1pl} & from & \textsc{1sg} & \textsc{poss} & friend & borrow & tracksuit & for & tomorrow\\
\lspbottomrule
\end{tabular}
\ea
\glt 
‘we (are back) from my friend (from whom we) borrowed a tracksuit \bluebold{for tomorrow}’ \textstyleExampleSource{[081011-020-Cv.0052]}
\z


In addition, \textitbf{untuk} ‘for’ introduces peripheral adjuncts that express the notion of circumstance as in \textitbf{untuk seng itu} ‘about those corrugated iron sheets’ in (0) or \textitbf{untuk masala tahang lapar} ‘about the problem of enduring to be hungry’ in (0).


\begin{styleExampleTitle}
Introducing circumstance
\end{styleExampleTitle}

\begin{tabular}{lllllll}
\lsptoprule
\label{bkm:Ref339635095}
\gll {tanya} {Sarles,} {bapa,} {\bluebold{untuk}} {\bluebold{seng}} {\bluebold{itu}}\\ %
& ask & Sarles & father & for & corrugated.iron & \textsc{d.dist}\\
\lspbottomrule
\end{tabular}
\ea
\glt 
‘father, ask Sarles \bluebold{about/for those corrugated iron} (sheets)’ \textstyleExampleSource{[080925-003-Cv.0003]}
\z

\begin{tabular}{llllllllllll}
\lsptoprule
\label{bkm:Ref339635096}
\gll {sa} {\multicolumn{2}{l}{bilang,}} {\bluebold{untuk}} {\multicolumn{2}{l}{\bluebold{masala}}} {\multicolumn{3}{l}{\bluebold{tahang}}} {\bluebold{lapar}} {kitong}\\ %
& \textsc{1sg} & \multicolumn{2}{l}{say} & for & \multicolumn{2}{l}{problem} & \multicolumn{3}{l}{hold.(out/back)} & be.hungry & \textsc{1pl}\\
& \multicolumn{2}{l}{bisa} & \multicolumn{3}{l}{tahang} & \multicolumn{2}{l}{lapar} & juga & \multicolumn{3}{l}{e?}\\
& \multicolumn{2}{l}{be.able} & \multicolumn{3}{l}{hold (out/back)} & \multicolumn{2}{l}{be.hungry} & also & \multicolumn{3}{l}{eh}\\
\lspbottomrule
\end{tabular}
\ea
\glt 
‘I say \bluebold{about the problem of enduring to be hungry}, we can also endure being hungry, eh?’ \textstyleExampleSource{[081025-009a-Cv.0118]}
\z


The preposition \textitbf{untuk} ‘for’ has dual word class membership; it is also used as a conjunction that introduces purpose clauses (§14.2.4.3; see also §5.14).
\end{styleBodyxvafter}

\subsection{\textitbf{buat} ‘for’}
\label{bkm:Ref319761140}
The core semantics of the preposition \textitbf{buat} ‘for’ are benefactive; that is, it introduces beneficiaries and benefactive recipients. In this, it is similar to benefactive \textitbf{untuk} ‘for’. Otherwise, as already mentioned in §10.2.3, \textitbf{buat} ‘for’ is more restricted in its distribution and functions: (1) it is not attested to combine with demonstratives, (2) it only rarely introduces inanimate referents, and (3) it is not attested to introduce other complements such as circumstance.



Most commonly, \textitbf{buat} ‘for’ follows bivalent action verbs such as \textitbf{putar} ‘stir’ and introduces peripheral adjuncts denoting human beneficiaries as in \textitbf{buat de bapa} ‘for her father’ in (0). Considerably less frequently, \textitbf{buat} ‘for’ introduces beneficiaries encoded by adnominal prepositional phrases as in \textitbf{buat torang} ‘for us’ in the exchange in (0).
\end{styleBodyvvafter}

\begin{styleExampleTitle}
Introducing animate beneficiaries
\end{styleExampleTitle}

\begin{tabular}{llllllll}
\lsptoprule
\label{bkm:Ref339635097}
\gll {Ika} {biking} {papeda} {putar} {\bluebold{buat}} {\bluebold{de}} {\bluebold{bapa}}\\ %
& Ika & make & sagu.porridge & stir & for & \textsc{3sg} & father\\
\lspbottomrule
\end{tabular}
\ea
\glt 
‘Ika made sagu porridge, she stirred (it) \bluebold{for her father}’ \textstyleExampleSource{[081006-032-Cv.0071]}
\z

\begin{tabular}{llllllllll}
\lsptoprule
\label{bkm:Ref339635098}
\gll { & Speaker-1: & sa & juga & dengang & ini & kaka & siapa & tu}\\ %
&  &  & \textsc{1sg} & also & with & \textsc{d.prox} & oSb & who & \textsc{d.dist}\\
\lspbottomrule
\end{tabular}
\begin{styleFreeTranslIndentiicmEng}
Speaker-1: ‘I was also with, what’s-his-name, that older brother, who-is-it?’
\end{styleFreeTranslIndentiicmEng}

\begin{tabular}{llllll} &  & Speaker-2: & satpam & \bluebold{buat} & \bluebold{torang}\\
\lsptoprule
&  &  & security.guard & for & \textsc{1pl}\\
\lspbottomrule
\end{tabular}
\begin{styleFreeTranslIndentiicmEng}
Speaker-2: ‘\bluebold{our} security guard’ (Lit. ‘the security guard \bluebold{for us}’ \textstyleExampleSource{[081025-006-Cv.0109]}
\end{styleFreeTranslIndentiicmEng}


Benefactive \textitbf{buat} ‘for’ also introduces benefactive recipients and addressees encoded by oblique arguments, as shown in (0) and (0), respectively. Hence, like \textitbf{untuk} ‘for’ (§10.2.3), benefactive \textitbf{buat} ‘for’ contrasts with goal-oriented \textitbf{sama} ‘to’ (§10.2.2), which expresses recipients and addressees, as in (0) to (0), without, however, signaling the concurrent notion of beneficiary.


\begin{styleExampleTitle}
Introducing benefactive recipients and addressees
\end{styleExampleTitle}

\begin{tabular}{lllllllll}
\lsptoprule
\label{bkm:Ref436750800}\label{bkm:Ref339635099}\label{bkm:Ref319077147}
\gll {slama} {ini} {de} {tida} {kasi} {uang} {\bluebold{buat}} {\bluebold{saya}}\\ %
& as.long.as & \textsc{d.prox} & \textsc{3sg} & \textsc{neg} & give & money & for & \textsc{1sg}\\
\lspbottomrule
\end{tabular}
\ea
\glt 
‘so far he hasn’t given (any) money \bluebold{to me for my benefit}’ \textstyleExampleSource{[081014-003-Cv.0034]}
\z

\begin{tabular}{llllllll}
\lsptoprule
\label{bkm:Ref339635102}\label{bkm:Ref319920589}
\gll {sa} {perna} {bicara} {\bluebold{buat}} {\bluebold{satu}} {\bluebold{ibu}} {…}\\ %
& \textsc{1sg} & once & speak & for & one & woman & \\
\lspbottomrule
\end{tabular}
\ea
\glt 
‘once I talked \bluebold{to a woman for her benefit} …’ \textstyleExampleSource{[081011-024-Cv.0073]}
\z


Benefactive \textitbf{buat} ‘for’ also introduces inanimate beneficiaries as in the adnominal prepositional phrase \textitbf{buat natal} ‘for Christmas’ in (0). This use, however, is very rare with the corpus including only this one example.


\begin{styleExampleTitle}
Introducing inanimate beneficiaries
\end{styleExampleTitle}

\begin{tabular}{lllllllll}
\lsptoprule
\label{bkm:Ref339635103}
\gll {pi} {ambil} {kayu} {bakar,} {kayu} {bakar} {\bluebold{buat}} {\bluebold{Natal}}\\ %
& go & fetch & wood & burn & wood & burn & for & Christmas\\
\lspbottomrule
\end{tabular}
\ea
\glt
‘(we) went to get firewood, firewood \bluebold{for Christmas}’ \textstyleExampleSource{[081006-017-Cv.0014]}
\end{styleFreeTranslEngxvpt}


The preposition \textitbf{buat} ‘for’ has dual word class membership; it is also used as the bivalent verb \textitbf{buat} ‘make’ (see §5.14).
\end{styleBodyxvafter}

\section{Prepositions encoding comparisons}
\label{bkm:Ref320188457}
Papuan Malay employs three prepositions of comparison: similative \textitbf{sperti} ‘similar to’ (§10.3.1) and \textitbf{kaya} ‘like’ (§10.3.2), and equative \textitbf{sebagey} ‘as’ (§10.3.3). All three introduce similes that express explicit resemblance or equatability between two bases of comparison.
\end{styleBodyxvafter}

\subsection{\textitbf{sperti} ‘similar to’}
\label{bkm:Ref319667233}
The preposition \textitbf{sperti} ‘similar to’ introduces similes that highlight resemblance or likeness in some respect between the two bases of comparison. Hence, \textitbf{sperti} ‘like’ is similar to \textitbf{kaya} ‘like’; for the distinctions between both similative prepositions see the discussion in §10.3.2.



Very commonly, \textitbf{sperti} ‘similar to’ forms peripheral adjuncts as in \textitbf{sperti klawar} ‘similar to a cave bat’ in (0). Also quite frequently, \textitbf{sperti} ‘similar to’ expresses resemblance in oblique arguments of some bivalent verbs as in (0): \textitbf{sperti manusia} ‘similar to a human’ is the oblique object of the change verb \textitbf{jadi} ‘become’. In addition, \textitbf{sperti} ‘similar to’ introduces the simile in nonverbal predicates with the complement being a common noun, a personal pronoun as in \textitbf{sperti ko} ‘similar to you’ in (0), or a demonstrative as in \textitbf{sperti itu}’ like that’ in (0). Finally, although rather infrequently, \textitbf{sperti} ‘similar to’ expresses resemblance in adnominal prepositional phrases as in \textitbf{baju sperti ini} ‘clothes like these’ (0). The examples in (0) to (0) also illustrate that \textitbf{sperti} ‘similar to’ introduces animate and inanimate, as well as nominal and pronominal referents.\footnote{\\
\\
\\
\\
\\
\\
\\
\\
\\
\\
\\
\\
\\
\\
\\
\par In the corpus only singular pronominal complements of \textitbf{sperti} ‘similar to’ are attested.}
\end{styleBodyvxafter}

\begin{tabular}{llllll}
\lsptoprule
\label{bkm:Ref339635104}
\gll {de} {bisa} {terbang} {\bluebold{sperti}} {\bluebold{klawar}}\\ %
& \textsc{3sg} & be.able & fly & similar.to & cave.bat\\
\lspbottomrule
\end{tabular}
\ea
\glt 
‘he/she (the evil spirit) can fly \bluebold{similar to a cave bat}’ \textstyleExampleSource{[081006-022-CvEx.0137]}
\z

\begin{tabular}{llllllll}
\lsptoprule
\label{bkm:Ref339635105}
\gll {setang} {itu} {de} {bisa} {jadi} {\bluebold{sperti}} {\bluebold{manusia}}\\ %
& evil.spirit & \textsc{d.dist} & \textsc{3sg} & be.able & become & similar.to & human.being\\
\lspbottomrule
\end{tabular}
\ea
\glt 
‘that evil spirit, he/she can become \bluebold{similar to a human}’ \textstyleExampleSource{[081006-022-CvEx.0010]}
\z

\begin{tabular}{lllllll}
\lsptoprule
\label{bkm:Ref339635106}
\gll {kalo} {kaka} {\bluebold{sperti}} {\bluebold{ko}} {kaka} {malu}\\ %
& if & oSb & similar.to & \textsc{2sg} & oSb & feel.embarrassed(.about)\\
\lspbottomrule
\end{tabular}
\ea
\glt 
‘if I (‘older sibling’) were \bluebold{similar to you}, I (‘older sibling’) would feel ashamed’ \textstyleExampleSource{[081115-001a-Cv.0040]}
\z

\begin{tabular}{llllll}
\lsptoprule
\label{bkm:Ref339635107}
\gll {mama} {pu} {hidup} {\bluebold{sperti}} {\bluebold{itu}}\\ %
& mother & \textsc{poss} & life & similar.to & \textsc{d.dist}\\
\lspbottomrule
\end{tabular}
\ea
\glt 
‘my (‘mother’s’) life is \bluebold{like that}’ \textstyleExampleSource{[080922-001a-CvPh.0932/0938]}
\z

\begin{tabular}{lllllllll}
\lsptoprule
\label{bkm:Ref339635108}
\gll {dorang} {tida} {pake} {\bluebold{baju}} {\bluebold{sperti}} {\bluebold{ini},} {pake} {daung{\Tilde}daung}\\ %
& \textsc{3pl} & \textsc{neg} & use & shirt & similar.to & \textsc{d.prox} & use & \textsc{rdp}{\Tilde}leaf\\
\lspbottomrule
\end{tabular}
\ea
\glt 
‘they don’t wear \bluebold{clothes like these}, (they) wear leaves’ \textstyleExampleSource{[081006-023-CvEx.0007]}
\z


The preposition \textitbf{sperti} ‘similar to’ has dual word class membership; it is also used as a conjunction that introduces similarity clauses (§14.2.6; see also §5.14).
\end{styleBodyxvafter}

\subsection{\textitbf{kaya} ‘like’}
\label{bkm:Ref319667235}
The core semantics of the preposition \textitbf{kaya} ‘like’ are similative: it indicates likeness between the two bases of comparison similar to \textitbf{sperti} ‘similar to’.\footnote{\\
\\
\\
\\
\\
\\
\\
\\
\\
\\
\\
\\
\\
\\
\\
\par In terms of its etymology, {U. Tadmor (p.c. 2013)} notes that the preposition \textitbf{kaya} ‘like’ is distinct from the stative verb \textitbf{kaya} ‘be rich’: stative “\textitbf{kaya} ‘be rich’ “was borrowed from Persian into Classical Malay” while similative “\textitbf{kaya} ‘like’ was borrowed from Javanese into colloquial varieties of Indonesian many centuries later. There is no etymological connection between the two”.} Unlike \textitbf{sperti} ‘like’, however, \textitbf{kaya} ‘like’ is not attested to combine with demonstratives. Moreover, \textitbf{kaya} ‘like’ is semantically distinct from \textitbf{sperti} ‘similar to’, as discussed below.



Most commonly, \textitbf{kaya} ‘like’ forms peripheral adjuncts as in \textitbf{kaya burung} ‘like a bird’ in (0). This example also illustrates that \textitbf{kaya} ‘like’ co-occurs with some of the same verbs as \textitbf{sperti} ‘similar to’, such as \textitbf{terbang} ‘fly’ in (0) (§10.3.1). Less frequently, \textitbf{kaya} ‘like’ introduces the simile in nonverbal predicates as in \textitbf{kaya buaya} ‘like a crocodile’ in (0). These examples also illustrate that typically the referent is animate and nominal; for an inanimate referent see the example in (0) and for a pronominal referent see (0).
\end{styleBodyvvafter}

\begin{styleExampleTitle}
Signaling overall likeness or resemblance
\end{styleExampleTitle}

\begin{tabular}{lllllllll}
\lsptoprule
\label{bkm:Ref339635109}
\gll {bisa} {terbang} {\bluebold{kaya}} {\bluebold{burung},} {bisa} {merayap} {\bluebold{kaya}} {\bluebold{ular}}\\ %
& be.able & fly & like & bird & be.able & creep & like & snake\\
\lspbottomrule
\end{tabular}
\ea
\glt 
‘(the evil spirit) can fly \bluebold{like a bird}, can creep \bluebold{like a snake}’ \textstyleExampleSource{[081006-022-CvEx.0031]}
\z

\begin{tabular}{lllllll}
\lsptoprule
\label{bkm:Ref339635114}
\gll {dong} {bilang} {soa-soa} {kang,} {\bluebold{kaya}} {\bluebold{buaya}}\\ %
& \textsc{3pl} & say & monitor.lizard & you.know & like & crocodile\\
\lspbottomrule
\end{tabular}
\ea
\glt 
‘they call (it) a monitor lizard, you know, (it’s) \bluebold{like a crocodile}’ \textstyleExampleSource{[080922-009-CvNP.0053]}
\z


The semantic distinctions between \textitbf{kaya} ‘like’ and \textitbf{sperti} ‘similar to’ are subtle. While both signal likeness in terms of appearance or behavior, they differ in terms of their semantic effect. Similative \textitbf{kaya} ‘like’ signals overall resemblance between the two bases of comparison. By contrast, the semantic effect of \textitbf{sperti} ‘similar to’ is more limited: it signals likeness or resemblance in some, most often implied, respect. This distinction is illustrated in the contrastive examples in (0) and (0).



In (0) \textitbf{kaya} ‘like’ signals overall physical resemblance: the speaker’s brother has the same facial features as their father. By contrast, in the elicited example in (0) \textitbf{sperti} ‘similar to’ signals limited or partial resemblance: that is, father and son share specific facial features. In (0), a teacher relates a conversation she had with a socially maladjusted student. Employing \textitbf{sperti} ‘similar to’, the teacher signals that she refers to some specific aspects of the student’s behavior: \textitbf{kalo kaka sperti ko} ‘if I (‘older sibling’) were similar to you (with respect to the behavior you’re displaying at school)’. If, by contrast, the teacher had used \textitbf{kaya} ‘like’, as in the elicited example in (0), the semantic effect of the comparison would have been much wider, not only referring to the student’s behavior at school but signaling overall resemblance between the speaker and her student.
\end{styleBodyvvafter}

\begin{styleExampleTitle}
Semantic distinctions between \textitbf{kaya} ‘like’ and \textitbf{sperti} ‘similar to’
\end{styleExampleTitle}

\begin{tabular}{lllllllll}
\lsptoprule
\label{bkm:Ref339635117}\label{bkm:Ref334180350}
\gll {\label{bkm:Ref334180370}} {de} {pu} {muka} {\bluebold{kaya}} {\bluebold{de}} {\bluebold{pu}} {\bluebold{bapa}}\\ %
&  & \textsc{3sg} & \textsc{poss} & face & like & \textsc{3sg} & \textsc{poss} & father\\
\lspbottomrule
\end{tabular}
\begin{styleFreeTranslIndentiicmEng}
‘his (my brother’s) face is \bluebold{like his father’s (face)}’ \textstyleExampleSource{[080922-001a-CvPh.1445]}
\end{styleFreeTranslIndentiicmEng}

\begin{tabular}{lllllllll} & \label{bkm:Ref334180371} & de & pu & muka & \bluebold{sperti} & \bluebold{de} & \bluebold{pu} & \bluebold{bapa}\\
\lsptoprule
&  & \textsc{3sg} & \textsc{poss} & face & similar.to & \textsc{3sg} & \textsc{poss} & father\\
\lspbottomrule
\end{tabular}
\begin{styleFreeTranslIndentiicmEng}
‘his (my brother’s) face is \bluebold{similar to his father’s (face)}’ \textstyleExampleSource{[Elicited BR120817.007]}
\end{styleFreeTranslIndentiicmEng}

\begin{tabular}{llllllll}
\lsptoprule
\label{bkm:Ref339635118}\label{bkm:Ref334180351}
\gll {\label{bkm:Ref334180372}} {kalo} {kaka} {\bluebold{sperti}} {\bluebold{ko}} {kaka} {malu}\\ %
&  & if & oSb & similar.to & \textsc{2sg} & oSb & feel.embarrassed(.about)\\
\lspbottomrule
\end{tabular}
\begin{styleFreeTranslIndentiicmEng}
‘if I (‘older sibling’) were \bluebold{similar to you}, I (‘older sibling’) would feel ashamed’ \textstyleExampleSource{[081115-001a-Cv.0040]}
\end{styleFreeTranslIndentiicmEng}

\begin{tabular}{llllllll} & \label{bkm:Ref334180373} & kalo & kaka & \bluebold{kaya} & \bluebold{ko} & kaka & malu\\
\lsptoprule
&  & if & oSb & like & \textsc{2sg} & oSb & feel.embarrassed(.about)\\
\lspbottomrule
\end{tabular}
\begin{styleFreeTranslIndentiicmEng}
‘if I (‘older sibling’) were \bluebold{like you}, I (‘older sibling’) would feel ashamed’ \textstyleExampleSource{[Elicited BR120817.006]}
\end{styleFreeTranslIndentiicmEng}


Signaling overall resemblance, similative \textitbf{kaya} ‘like’ is also employed when the speaker wants to make a more expressive, metaphorical comparison as in (0). This example also illustrates that the referent can be inanimate.


\begin{styleExampleTitle}
Introducing expressive similes
\end{styleExampleTitle}

\begin{tabular}{llllll}
\lsptoprule
\label{bkm:Ref339635119}
\gll {smua} {jalang} {\bluebold{kaya}} {\bluebold{kapal}} {\bluebold{kayu}}\\ %
& all & walk & like & ship & wood\\
\lspbottomrule
\end{tabular}
\ea
\glt 
‘[because they were so hungry] (they) all were strolling around \bluebold{like wooden boats}’ \textstyleExampleSource{[081025-009a-Cv.0188]}
\z


The preposition \textitbf{kaya} ‘like’ has dual word class membership; it is also used as a conjunction that introduces similarity clauses (§14.2.6; see also §5.14).
\end{styleBodyxvafter}

\subsection{\textitbf{sebagey} ‘as’}
\label{bkm:Ref319761143}
The equative preposition \textitbf{sebagey} ‘as’ introduces similes that express equatability between the two bases of comparison in terms of specific roles or capacities. Hence, \textitbf{sebagey} ‘as’ contrasts with the similarity prepositions \textitbf{sperti} ‘similar to’ (§10.3.1) and \textitbf{kaya} ‘like’ (§10.3.2) which express resemblance and likeness.



Most commonly, the complement is expressed in an adnominal prepositional phrase. In (0), for example, \textitbf{sebagey} ‘as’ links the head nominal \textitbf{torang} ‘\textsc{1pl}’ to the role-encoding adnominal constituent \textitbf{kepala kampung} ‘village heads’. Following mono- or bivalent action verbs, \textitbf{sebagey} ‘as’ expresses equatability in peripheral adjuncts. In (0), for example, \textitbf{sebagey} ‘as’ follows the communication verb \textitbf{bicara} ‘speak’ and relates the role-encoding complement \textitbf{ibu camat} ‘Ms. Subdistrict-Head’ to the clausal subject \textitbf{ko} ‘\textsc{2sg}’. The corpus also includes two examples in which \textitbf{sebagey} ‘as’ introduces nonverbal predicates to express equatability, as for example in (0) between the predicate \textitbf{kepala acara} ‘the head of the festivity’ and the clausal subject \textitbf{sa} ‘\textsc{1sg}’.
\end{styleBodyvxafter}

\begin{tabular}{lllllll}
\lsptoprule
\label{bkm:Ref339635120}
\gll {torang} {\bluebold{sebagey}} {\bluebold{kepala}} {\bluebold{kampung}} {juga} {penanggung-jawap}\\ %
& \textsc{1pl} & as & head & village & also & responsibility\\
\lspbottomrule
\end{tabular}
\ea
\glt 
‘we as \bluebold{village heads} are also bearers of responsibility’ \textstyleExampleSource{[081008-001-Cv.0035]}
\z

\begin{tabular}{llllllll}
\lsptoprule
\label{bkm:Ref339635121}
\gll {\multicolumn{2}{l}{sebentar}} {di} {\multicolumn{2}{l}{Diklat}} {ko} {bicara}\\ %
& \multicolumn{2}{l}{a.moment} & at & \multicolumn{2}{l}{government.education.program} & \textsc{2sg} & speak\\
& \bluebold{sebagey} & \multicolumn{3}{l}{\bluebold{ibu}} & \multicolumn{3}{l}{\bluebold{camat}}\\
& as & \multicolumn{3}{l}{woman} & \multicolumn{3}{l}{subdistrict.head}\\
\lspbottomrule
\end{tabular}
\ea
\glt 
‘a bit later at the government education and training (office) you’ll speak \bluebold{as Ms. Subdistrict-Head}’ \textstyleExampleSource{[081010-001-Cv.0099]}
\z

\begin{tabular}{lllllllll}
\lsptoprule
\label{bkm:Ref339635122}
\gll {paling} {sa} {tra} {kerja,} {sa} {\bluebold{sebagey}} {\bluebold{kepala}} {\bluebold{acara}}\\ %
& most & \textsc{1sg} & \textsc{neg} & work & \textsc{1sg} & as & head & festivity\\
\lspbottomrule
\end{tabular}
\ea
\glt 
[About organizing a festivity:] ‘most likely I won’t (have to) work, I’ll be \bluebold{the head of the festivity}’ (Lit. ‘\bluebold{as the head …}’) \textstyleExampleSource{[080919-004-NP.0068]}
\z


As for the syntactic properties of its complements, the examples in (0) to (0) show that equative \textitbf{sebagey} ‘as’ introduces common nouns, as similative \textitbf{sperti} ‘similar to’ (§10.3.1) and \textitbf{kaya} ‘like’ (§10.3.2) do. Unlike the findings for both similative prepositions, however, the corpus does not include prepositional phrases in which \textitbf{sebagey} ‘as’ introduces personal pronouns. Neither are examples attested in which \textitbf{sebagey} ‘as’ combines with demonstratives as \textitbf{sperti} ‘similar to’ does.
\end{styleBodyxvafter}

\section{Summary}
\label{bkm:Ref308522005}
Prepositional phrases consist of a preposition and a noun phrase complement which is obligatory and may not be fronted. The preposition indicates the grammatical and semantic relationship of the complement to the predicate. Prepositional phrases in Papuan Malay are formed with eleven different prepositions:


%\setcounter{itemize}{0}
\begin{itemize}
\item \begin{styleIIndented}
Prepositions encoding location in space or time: \textitbf{di} ‘at, in’, \textitbf{ke} ‘to’, \textitbf{dari} ‘from’, and \textitbf{sampe} ‘until’
\end{styleIIndented}\item \begin{styleIIndented}
Prepositions encoding accompaniment/instruments, goals, or benefaction: \textitbf{dengang} ‘with’, \textitbf{sama} ‘to’, \textitbf{untuk} ‘for’, and \textitbf{buat} ‘for’
\end{styleIIndented}\item \begin{styleIvI}
Prepositions encoding comparisons: \textitbf{sperti} ‘similar to’, \textitbf{kaya} ‘like’, and \textitbf{sebagey} ‘as’
\end{styleIvI}\end{itemize}

A substantial number of the prepositions have dual word class membership, two have trial class membership. That is, three prepositions are also used as verbs, namely \textitbf{buat} ‘for’, \textitbf{sama} ‘to’, and \textitbf{sampe} ‘until’ (see §5.3). Six prepositions are also used as conjunctions, namely \textitbf{dengang} ‘with’, \textitbf{kaya} ‘like’, \textitbf{sama} ‘to’, \textitbf{sampe} ‘until’, \textitbf{sperti} ‘similar to’, and \textitbf{untuk} ‘for’ (see §5.12 and Chapter 14). (Variation in word class membership is discussed in §5.14.)



Prepositional phrases take on different functions within the clause and combine with different types of syntactic constituents. The complements of the prepositions take different semantic roles within the clause, depending on the prepositions they are introduced with. These findings are summarized in Table  ‎10 .1 to Table  ‎10 .3; in these tables, the prepositions are listed according to the order in which they are discussed in this chapter, starting with \textitbf{di} ‘at, in’. Empty cells signal unattested constituent combinations.
\end{styleBodyvafter}


Table  ‎10 .1 lists the three syntactic functions that prepositional phrases can take within the clause according to the prepositions they are introduced with, that is, their functions as peripheral adjuncts, nonverbal predicates, and arguments. In addition, Table  ‎10 .1 lists those prepositions that introduce modifying, adnominal prepositional phrase and those that are also used as conjunctions.
\end{styleBodyvvafter}

\begin{stylecaption}
\label{bkm:Ref305151258}Table ‎10.\stepcounter{Table}{\theTable}:  Syntactic functions of prepositional phrases
\end{stylecaption}

\tablehead{ & \multicolumn{3}{l}{ Clausal functions} & \multicolumn{2}{l}{ Additional functions}\\
& \textsc{adjct} & \textsc{pred} & \textsc{argt} & \textsc{mod} & \arraybslash \textsc{cnj}\\
}
\begin{tabular}{llllll}
\lsptoprule
\textitbf{di} ‘at, in’ & X & X & X & X & \\
\textitbf{ke} ‘to’ & X & X & X &  & \\
\textitbf{dari} ‘from’ & X & X & X & X & \\
\textitbf{sampe} ‘until’ & X &  &  &  & \arraybslash X\\
\textitbf{dengang} ‘with’ & X & X & X &  & \arraybslash X\\
\textitbf{sama} ‘to’ & X & X & X &  & \arraybslash X\\
\textitbf{untuk} ‘for’ & X & X & X & X & \arraybslash X\\
\textitbf{buat} ‘for’ & X &  & X & X & \\
\textitbf{sperti} ‘similar to’ & X & X & X & X & \arraybslash X\\
\textitbf{kaya} ‘like’ & X & X &  &  & \arraybslash X\\
\textitbf{sebagey} ‘as’ & X & X &  & X & \\
\lspbottomrule
\end{tabular}

With respect to their complements, the data in Table  ‎10 .2 shows that the prepositions combine with different constituents from different word classes, namely nouns, personal pronouns, demonstratives, locatives, and temporal adverbs.


\begin{stylecaption}
\label{bkm:Ref307935215}Table ‎10.\stepcounter{Table}{\theTable}:  Word classes of complements
\end{stylecaption}

\tablehead{ & \textsc{n.com} & \textsc{n.loc} & \textsc{n.time} & \textsc{pro} & \textsc{dem} & \textsc{loc} & \arraybslash \textsc{adv.t}\\
}
\begin{tabular}{llllllll}
\lsptoprule
\textitbf{di} ‘at, in’ & X & X &  & X &  & X & \\
\textitbf{ke} ‘to’ & X & X &  & X &  & X & \\
\textitbf{dari} ‘from’ & X & X & X & X &  & X & \arraybslash X\\
\textitbf{sampe} ‘until’ &  &  & X &  &  &  & \arraybslash X\\
\textitbf{dengang} ‘with’ & X &  &  & X & X &  & \\
\textitbf{sama} ‘to’ & X &  &  & X & X &  & \\
\textitbf{untuk} ‘for’ & X &  & X & X & X &  & \arraybslash X\\
\textitbf{buat} ‘for’ & X &  &  & X &  &  & \\
\textitbf{sperti} ‘similar to’ & X &  &  & X & X &  & \\
\textitbf{kaya} ‘like’ & X &  &  & X &  &  & \\
\textitbf{sebagey} ‘as’ & X &  &  &  &  &  & \\
\lspbottomrule
\end{tabular}

Finally, the complements of prepositions take different semantic roles within the clause, depending on the prepositions they are introduced with. These different semantic roles are summarized in Table  ‎10 .3 with the primary role underlined.


\begin{stylecaption}
\label{bkm:Ref304993090}Table ‎10.\stepcounter{Table}{\theTable}:  Semantic roles of complements
\end{stylecaption}

\tablehead{ & \textsc{loct} & \textsc{assct} & \textsc{omv} & \textsc{ins} & \textsc{rec} & \textsc{ben} & \textsc{circ} & \arraybslash \textsc{std}\\
}
\begin{tabular}{lllllllll}
\lsptoprule
\textitbf{di} ‘at, in’ & \textstyleChUnderl{X} &  &  &  &  &  &  & \\
\textitbf{ke} ‘to’ & \textstyleChUnderl{X} &  &  &  &  &  &  & \\
\textitbf{dari} ‘from’ & \textstyleChUnderl{X} &  &  &  &  &  &  & \\
\textitbf{sampe} ‘until’ & \textstyleChUnderl{X} &  &  &  &  &  &  & \\
\textitbf{dengang} ‘with’ &  & \textstyleChUnderl{X} & X & X &  &  &  & \\
\textitbf{sama} ‘to’ &  & X & X &  & \textstyleChUnderl{X} &  &  & \\
\textitbf{untuk} ‘for’ &  &  &  &  & X & \textstyleChUnderl{X} & X & \\
\textitbf{buat} ‘for’ &  &  &  &  & X & \textstyleChUnderl{X} &  & \\
\textitbf{sperti} ‘similar to’ &  &  &  &  &  &  &  & \arraybslash \textstyleChUnderl{X}\\
\textitbf{kaya} ‘like’ &  &  &  &  &  &  &  & \arraybslash \textstyleChUnderl{X}\\
\textitbf{sebagey} ‘as’ &  &  &  &  &  &  &  & \arraybslash \textstyleChUnderl{X}\\
\lspbottomrule
\end{tabular}

If the context allows the disambiguation of the semantic relationship of the complement to the predicate, two of the prepositions of location can be omitted: locative \textitbf{di} ‘at, in’ and allative \textitbf{ke} ‘to’.
\end{styleBodyaftervbefore}

%\setcounter{page}{1}\chapter[Verbal clauses]{Verbal clauses}
\label{bkm:Ref289539095}
This chapter discusses different types of verbal predicate clauses in Papuan Malay, in which a verb occupies the syntactic and semantic core of the clause. In Papuan Malay verbal clauses, the predicate typically follows the subject and, in transitive clauses, precedes the direct object. In negated verbal clauses, the negator precedes the predicate.



Papuan Malay verbal clauses can be distinguished into intransitive and transitive clauses; this distinction is discussed in §11.1. The subsequent sections describe special types of (in)transitive clauses: causative clauses in §11.2, reciprocal clauses in §11.3, existential clauses in §11.4, and comparative clauses in §11.5. The main points of this chapter are summarized in §11.6. Negation is described in §13.1.
\end{styleBodyvxvafter}

\section{Intransitive and transitive clauses}
\label{bkm:Ref367459475}
Papuan Malay verbal clauses can be intransitive, monotransitive, or ditransitive. Typically, intransitive clauses are formed with monovalent verbs which take one core argument; as discussed below, though, bi- and trivalent verbs also occur in intransitive or monotransitive clauses. Monotransitive clauses are usually formed with bivalent verbs which take two core arguments, the subject and a direct object. These two types of verbs and verbal clauses are the most common ones in Papuan Malay. In addition, Papuan Malay has ditransitive clauses formed with a small number of trivalent verbs which take three core arguments, a subject and two objects.



It is important to note, though, that in Papuan Malay the trivalent verbs allow but do not require three syntactic arguments. Likewise, bivalent verbs allow but do not require two arguments. That is, in clauses with tri- or bivalent verbs, core arguments are often elided when they are understood from the context. (See also {Margetts and Austin’s 2007: 401 }cross-linguistic typology for the rather common elision of syntactic arguments.)
\end{styleBodyvafter}


Given this syntactic mismatch between valency and transitivity, this section on transitivity is not organized in terms of intransitive, monotransitive, and ditransitive clauses. Instead, it is organized in terms of the valency of the verbs, and describes how the three verb classes are used in transitive and/or intransitive clauses. Verbal clauses with monovalent verbs are discussed in §11.1.1, with bivalent verbs in §11.1.2, and with trivalent verbs in §11.1.3. (The properties of verbs are described in §5.3. For details on optional linguistic expressions providing additional information about the setting of the events or states depicted by the verbs, see Chapter 10; see also §5.2.5.)
\end{styleBodyvxvafter}

\subsection{Verbal clauses with monovalent verbs}
\label{bkm:Ref367529000}
Papuan Malay has a large open class of monovalent verbs. Involving only one participant, they always occur in intransitive clauses (490 monovalent verbs are attested in the corpus; for a list of examples see Table  ‎5 .14 in §5.3.1).



Semantically, the attested 490 verbs can be divided into dynamic ones (139 verbs) and stative ones (351 verbs), as is typical of languages lacking a class of adjectives. The former denote actions, while the latter designate states or more time-stable properties. Syntactically, however, there are no distinctions between dynamic and stative verbs.
\end{styleBodyvafter}


Typically, monovalent verbs follow their clausal subjects, as shown with dynamic \textitbf{lari} ‘run’ in (0), and with stative \textitbf{bagus} ‘be good’ in (0).
\end{styleBodyvvafter}

\begin{styleExampleTitle}
Monovalent verbs with canonical subject-verb word order
\end{styleExampleTitle}

\begin{tabular}{llll}
\lsptoprule
\label{bkm:Ref346026840}
\gll {o,} {babi} {\bluebold{lari}}\\ %
& oh & pig & run\\
\lspbottomrule
\end{tabular}
\ea
\glt 
‘o, the pig \bluebold{ran}’ \textstyleExampleSource{[080919-004-NP.0021]}
\z

\begin{tabular}{llll}
\lsptoprule
\label{bkm:Ref340309692}
\gll {itu} {\bluebold{bagus}} {skali}\\ %
& \textsc{d.dist} & be.good & very\\
\lspbottomrule
\end{tabular}
\ea
\glt 
‘that is very \bluebold{good}’ \textstyleExampleSource{[081025-003-Cv.0267]}
\z


If speakers want to emphasize the predicate with a monovalent stative verb, they can front it, such as stative \textitbf{bagus} ‘be good’ in (0). In this case, the predicate is set-off by a boundary intonation, which is achieved by marking the stressed syllable of the verb with a slight increase in pitch (“~\'{~}~”). Consultants disagree, however, whether monovalent dynamic verbs can be fronted. While two consultants stated that dynamic \textitbf{jatu} ‘fall’ in the elicited example in (0) can be fronted, a third one rejected the example as ungrammatical. Furthermore, one of the consultants who accepted the verbal clause in (0) suggested that the fronting of monovalent dynamic verbs is a recent development and that older Papuan Malay speakers would not use such a construction.


\begin{styleExampleTitle}
Preposed monovalent verbs
\end{styleExampleTitle}

\begin{tabular}{llll}
\lsptoprule
\label{bkm:Ref340309704}
\gll {\bluebold{bágus}} {skali} {itu}\\ %
& be.good & very & \textsc{d.dist}\\
\lspbottomrule
\end{tabular}
\ea
\glt 
‘very \bluebold{good} is that’ \textstyleExampleSource{[081025-003-Cv.0270]}
\z

\begin{tabular}{llll}
\lsptoprule
\label{bkm:Ref367266728}
\gll {o,} {\bluebold{játu}} {dia!}\\ %
& oh & fall & \textsc{3sg}\\
\lspbottomrule
\end{tabular}
\ea
\glt 
‘oh, he \bluebold{fell}’ \textstyleExampleSource{[Elicited BR131227.001]}
\z


The subject can also be omitted if it can be inferred from the context. In (0) the elided subject is \textitbf{sa} ‘\textsc{1sg}’, and in (0) it is \textitbf{dia}/\textitbf{de} ‘\textsc{3sg}’.


\begin{styleExampleTitle}
Elision of the subject argument
\end{styleExampleTitle}

\begin{tabular}{llllllllllll}
\lsptoprule
\label{bkm:Ref367274023}
\gll {siang} {Ø} {\bluebold{jalang},} {trus} {malam} {Ø} {\bluebold{duduk}} {\bluebold{menyanyi}} {sampe} {jam} {dua}\\ %
& day &  & walk & next & night &  & sit & sing & until & hour & two\\
\lspbottomrule
\end{tabular}
\ea
\glt 
‘(during) the day (I) \bluebold{went} (over there), then in the evening (I) \bluebold{sat about} (and) \bluebold{sang} (songs) until two o’clock (in the morning)’ \textstyleExampleSource{[080923-003-CvNP.0002]}
\z

\begin{tabular}{lllll}
\lsptoprule
\label{bkm:Ref340309695}
\gll {Speaker-2:} {adu,} {Ø} {\bluebold{nakal}}\\ %
&  & oh.no! &  & be.mischievous\\
\lspbottomrule
\end{tabular}
\ea
\glt
[Speaker 1: ah, that Petrus!]\\
Speaker-2: oh no, (he’s) \bluebold{mischievous}’ \textstyleExampleSource{[081115-001a-Cv.0033]}
\end{styleFreeTranslEngxvpt}

\subsection{Verbal clauses with bivalent verbs}
\label{bkm:Ref367529001}
Papuan Malay has a large open class of bivalent verbs (535 are attested in the corpus; for a set of examples see Table  ‎5 .14 in §5.3.1). Bivalent verbs have two core arguments, a subject and an object. In terms of their semantic roles, “two-place predicates take an agent-like argument A, and a non-agent-like argument P”, adopting {Margetts and Austin’s (2007: 396)} terminology. As mentioned, though, bivalent verbs in Papuan Malay allow but do not require two syntactic arguments. Examples of bivalent verbs are \textitbf{bunu} ‘kill’ in (0) and \textitbf{potong} ‘cut’ in (0).


\begin{styleExampleTitle}
Bivalent verbs with two arguments and canonical subject-verb-object order
\end{styleExampleTitle}

\begin{tabular}{lllllllll}
\lsptoprule
\label{bkm:Ref348008196}
\gll {kalo} {ko} {masi} {mo} {berjuang} {kitorang} {\bluebold{bunu}} {ko}\\ %
& if & \textsc{2sg} & still & want & struggle & \textsc{1pl} & kill & \textsc{2sg}\\
\lspbottomrule
\end{tabular}
\ea
\glt 
‘if you still want to fight, we’ll \bluebold{kill} you’ \textstyleExampleSource{[081029-004-Cv.0072]}
\z

\begin{tabular}{lllll}
\lsptoprule
\label{bkm:Ref348008199}
\gll {jadi} {kamu} {\bluebold{potong}} {sapi}\\ %
& so & \textsc{2pl} & cut & cow\\
\lspbottomrule
\end{tabular}
\ea
\glt 
‘so you \bluebold{cut up} the cow’ \textstyleExampleSource{[080925-005-CvPh.0007]}
\z


The monotransitive clauses in (0) and (0) illustrate the canonical subject-verb-object order for bivalent verbs. If speakers want to emphasize the object, they can also front it. Unlike clauses with preposed monovalent verbs, though, there is no clear boundary intonation to set-off the preposed object arguments from the rest of the clause. In (0), the preposed object \textitbf{paylot} ‘pilot’ is marked with a slight increase in pitch of its stressed penultimate syllable (“~\'{~}~”) and it is separated from the rest of the clause with a comma intonation (“{\textbar}”). Besides, the ultimate syllable of \textitbf{bunu} ‘kill’ receives final lengthening, signaled with the vowel tripling. In (0), the preposed object remains unmarked but the clause-final verb \textitbf{potong} ‘cut’ is marked with a slight increase in pitch of its stressed penultimate syllables.


\begin{styleExampleTitle}
Bivalent verbs with preposed object arguments
\end{styleExampleTitle}

\begin{tabular}{lllll}
\lsptoprule
\label{bkm:Ref367295788}
\gll {páylot} {{\textbar}} {dorang} {\bluebold{bunuuu}}\\ %
& pilot &  & \textsc{3pl} & kill\\
\lspbottomrule
\end{tabular}
\ea
\glt 
‘the pilot they \bluebold{killed}’ \textstyleExampleSource{[081025-004-Cv.0040]}
\z

\begin{tabular}{lllllllll}
\lsptoprule
\label{bkm:Ref367295790}
\gll {dong} {dua} {pu} {telefisi} {sidi} {dua} {dia} {\bluebold{pótong}}\\ %
& \textsc{3pl} & two & \textsc{poss} & television & CD.player & two & \textsc{3sg} & cut\\
\lspbottomrule
\end{tabular}
\ea
\glt 
‘the television (and) both CDs of the two of them he \bluebold{destroyed}’ \textstyleExampleSource{[081011-009-Cv.0006]}
\z


When one or both of the core arguments are understood from the context, they can be omitted, as shown in (0) to (0).\footnote{\\
\\
\\
\\
\\
\\
\\
\\
\\
\\
\\
\\
\\
\\
\\
\par At this point in the research, the number of clauses with overt and elided core arguments has not been quantified to examine which strategy is the preferred one.} Elision of the object argument is illustrated for \textitbf{bunu} ‘kill’ in (0), and \textitbf{potong} ‘cut (up) in (0).


\begin{styleExampleTitle}
Elision of the object argument and retention of the subject argument
\end{styleExampleTitle}

\begin{tabular}{lllllllll}
\lsptoprule
\label{bkm:Ref367281460}
\gll {…} {kalo} {prempuang} {melahirkang} {laki{\Tilde}laki} {dong} {\bluebold{bunu}} {Ø}\\ %
&  & if & woman & give.birth & \textsc{rdp}{\Tilde}husband & \textsc{3pl} & kill & \\
\lspbottomrule
\end{tabular}
\ea
\glt 
‘[indeed, these women can’t live with men,] when a woman gives birth to a boy, they \bluebold{kill} (him)’ \textstyleExampleSource{[081006-023-CvEx.0058]}
\z

\begin{tabular}{lllllll}
\lsptoprule
\label{bkm:Ref346965802}
\gll {…} {tong} {\bluebold{potong}} {Ø} {hari} {itu}\\ %
&  & \textsc{1pl} & cut &  & day & \textsc{d.dist}\\
\lspbottomrule
\end{tabular}
\ea
\glt 
‘[we shouldered it, the pig, (and) carried (it) to the garden shelter,] we \bluebold{cut} (it) \bluebold{up} that day’ \textstyleExampleSource{[080919-003-NP.0013-0014]}
\z


Elision of the subject argument is demonstrated for \textitbf{bunu} ‘kill’ in (0), and \textitbf{potong} ‘cut’ in (0).


\begin{styleExampleTitle}
Elision of the subject argument and retention of the object argument
\end{styleExampleTitle}

\begin{tabular}{lllllll}
\lsptoprule
\label{bkm:Ref367289488}
\gll {Ø} {\bluebold{bunu}} {dia,} {Ø} {\bluebold{bunu}} {dia}\\ %
&  & kill & \textsc{3sg} &  & kill & \textsc{3sg}\\
\lspbottomrule
\end{tabular}
\ea
\glt 
‘(they) \bluebold{kill} him, (they) \bluebold{kill} him’ \textstyleExampleSource{[081006-022-CvEx.0088]}
\z

\begin{tabular}{llllllll}
\lsptoprule
\label{bkm:Ref367289489}
\gll {baru} {Ø} {\bluebold{potong}} {pisang} {di} {tenga{\Tilde}tenga} {to?}\\ %
& and.then &  & cut & banana & at & \textsc{rdp}{\Tilde}middle & right?\\
\lspbottomrule
\end{tabular}
\ea
\glt 
‘and then (we) \bluebold{cut} the bananas in the middle, right?’ \textstyleExampleSource{[080922-009-CvNP.0041]}
\z


Finally, speakers can also omit both core arguments at the same time, as shown for \textitbf{bunu} ‘kill’ in (0), and \textitbf{potong} ‘cut’ in (0).


\begin{styleExampleTitle}
Elision of the subject and object arguments
\end{styleExampleTitle}

\begin{tabular}{llllllll}
\lsptoprule
\label{bkm:Ref367291415}
\gll {Ø} {\bluebold{bunu}} {Ø} {tapi} {kasi} {hidup} {lagi}\\ %
&  & kill &  & but & give & live & again\\
\lspbottomrule
\end{tabular}
\ea
\glt 
[About sorcerers who can resurrect the dead:] ‘(they) \bluebold{kill} (him) but (they) make (him) live again’ \textstyleExampleSource{[081006-022-CvEx.0087]}
\z

\begin{tabular}{lllll}
\lsptoprule
\label{bkm:Ref367291417}
\gll {Ø} {\bluebold{potong}} {Ø} {kecil{\Tilde}kecil}\\ %
&  & cut &  & \textsc{rdp}{\Tilde}be.small\\
\lspbottomrule
\end{tabular}
\ea
\glt
‘(I) \bluebold{cut} (the meat) very small’ \textstyleExampleSource{[080919-003-NP.0016]}
\end{styleFreeTranslEngxvpt}

\subsection{Verbal clauses with trivalent verbs}
\label{bkm:Ref367529002}
Papuan Malay has a small number of trivalent verbs with three core arguments, that is, a subject, and two objects. In the corpus seven trivalent verbs are attested: \textitbf{ambil} ‘fetch’, \textitbf{bawa} ‘bring’, \textitbf{bli} ‘buy’, \textitbf{ceritra} ‘tell’, \textitbf{kasi} ‘give’, \textitbf{kirim} ‘send’, and \textitbf{minta} ‘request’.



In terms of their semantic roles, three-place predicates “take an agent-like A, a participant that will label R on the basis of its most common role as recipient (but that may also be a beneficiary, goal, addressee, location, or source), and a T (typically some thing or information conveyed by A to R)”, applying {Margetts and Austin’s (2007: 396)} terminology. As mentioned, though, trivalent verbs in Papuan Malay allow but do not require three syntactic arguments.
\end{styleBodyvafter}


Trivalent verbs exhibit dative alternation in that they appear in ditransitive clauses with double-object constructions (§11.1.3.1), or in monotransitive clauses with oblique constructions (§11.1.3.2). Alternatively, the R and T arguments can be combined into one noun phrase with an adnominal possessor (§11.1.3.3). Another option is to omit the R and/or T arguments (§11.1.3.4). The distributional frequencies for these strategies are discussed in §11.1.3.5.
\end{styleBodyvxvafter}

\paragraph[Double{}-object constructions]{Double-object constructions}
\label{bkm:Ref366767542}
In Papuan Malay ditransitive clauses with double-object constructions, the R and T arguments are unflagged and occur in the order R-T. In this construction type, as {\citet[173]{Payne1997}} puts it, the semantically peripheral R is brought “center-stage” while the T has “status as the ‘second object’”. Cross-linguistically, the R typically precedes the T which, as {\citet[16]{MalchukovEtAl2010}} suggest, “probably derives from the fact that the R is generally human (and often definite) and thus tends to be more topical than the T, which is typically inanimate (and often indefinite)”.



Papuan Malay double object constructions with R-T word order are presented in (0) to (0). Overall, however, double-object constructions are not very common in Papuan Malay. The corpus contains only 30 constructions among a total of 1,160 verbal clauses formed with trivalent verbs (2.6\%).\footnote{\\
\\
\\
\\
\\
\\
\\
\\
\\
\\
\\
\\
\\
\\
\\
\par This total excludes serial verb constructions formed with \textitbf{kasi} ‘give’ (see §11.2.1.2).}
\end{styleBodyvvafter}

\begin{styleExampleTitle}
Double-object constructions: R-T word order
\end{styleExampleTitle}

\begin{tabular}{lllllllll}
\lsptoprule
\label{bkm:Ref365624061}
\gll {mungking} {de} {suru} {dia,} {ko} {\bluebold{ambil}} {sa} {air!}\\ %
& maybe & \textsc{3sg} & order & \textsc{3sg} & \textsc{2sg} & fetch & \textsc{1sg} & water\\
\lspbottomrule
\end{tabular}
\ea
\glt 
‘maybe he/she’ll order him/her, ‘you \bluebold{fetch} me water!’’ \textstyleExampleSource{[081006-024-CvEx.0092]}
\z

\begin{tabular}{lllllllll}
\lsptoprule
\label{bkm:Ref365624646}
\gll {tiga} {orang} {itu} {datang} {…} {\bluebold{bawa}} {dong} {pakeang}\\ %
& three & person & \textsc{d.dist} & come &  & bring & \textsc{3pl} & clothes\\
\lspbottomrule
\end{tabular}
\ea
\glt 
‘those three people came … (and) \bluebold{brought} them clothes’ \textstyleExampleSource{[081006-023-CvEx.0074]}
\z

\begin{tabular}{lllllll}
\lsptoprule
\label{bkm:Ref349311269}
\gll {paytua} {dia} {\bluebold{bli}} {Andi} {satu} {set}\\ %
& husband & \textsc{3sg} & buy & Andi & one & set\\
\lspbottomrule
\end{tabular}
\ea
\glt 
‘the gentleman \bluebold{bought} Andi one (TV/CD) set’ \textstyleExampleSource{[081011-009-Cv.0055]}
\z

\begin{tabular}{lllllllllll}
\lsptoprule
(\stepcounter{}{\the}) & nanti & \multicolumn{2}{l}{waktu} & tidor & de & bilang, & a, & bapa & \bluebold{ceritra} & ko\\
& very.soon & \multicolumn{2}{l}{time} & sleep & \textsc{3sg} & say & ah! & father & tell & \textsc{2sg}\\
& \multicolumn{2}{l}{dongeng{\Tilde}dongeng} & \multicolumn{8}{l}{dulu}\\
& \multicolumn{2}{l}{\textsc{rdp}{\Tilde}legend} & \multicolumn{8}{l}{first}\\
\lspbottomrule
\end{tabular}
\ea
\glt 
‘later at bed-time he’ll say, ‘ah, I (‘father’) \bluebold{tell} you some stories first’’ \textstyleExampleSource{[081110-008-CvNP.0140]}
\z

\begin{tabular}{llllll}
\lsptoprule
(\stepcounter{}{\the}) & skarang & dong & \bluebold{kasi} & dia & senter\\
& now & \textsc{3pl} & give & \textsc{3sg} & flashlight\\
\lspbottomrule
\end{tabular}
\ea
\glt 
‘now they \bluebold{give} him a flashlight’ \textstyleExampleSource{[081108-003-JR.0002]}
\z

\begin{tabular}{lllllllll}
\lsptoprule
(\stepcounter{}{\the}) & sa & baru{\Tilde}baru & bilang, & … & kaka & \bluebold{kirim} & dong & uang!\\
& \textsc{1sg} & just.now & say &  & oSb & send & \textsc{3pl} & money\\
\lspbottomrule
\end{tabular}
\ea
\glt 
‘just now I said, ‘older sibling \bluebold{send} them money!’’ \textstyleExampleSource{[080922-001a-CvPh.0860]}
\z

\begin{tabular}{lllllllll}
\lsptoprule
\label{bkm:Ref349292209}
\gll {trus} {sa} {bukang} {orang} {miskin} {\bluebold{minta{\Tilde}minta}} {kamu} {uang}\\ %
& next & \textsc{1sg} & \textsc{neg} & person & be.poor & \textsc{rdp}{\Tilde}request & \textsc{2pl} & money\\
\lspbottomrule
\end{tabular}
\ea
\glt 
‘and I’m not a poor person (who) \bluebold{keeps begging} you (for) money’ \textstyleExampleSource{[081011-020-Cv.0043/0045]}
\z


The T can also precede the R in double-object constructions, as shown in (0) and (0). This T-R order “is relatively widespread in South-East Asia”, as {\citet[17]{MalchukovEtAl2010}} point out. Building on {Dik and Hengeveld’s (1997: 435–436)} notion of “iconic sequencing”, {\citet[17]{MalchukovEtAl2010}} suggest that “the order T-R is more iconic than the order R-T, because in the unfolding of the event the T is first involved in the action, which reaches the R only in a second step”.



In Papuan Malay, however, T-R constructions are even less common than R-T constructions; the corpus contains 17 constructions among the total of 1,160 verbal clauses formed with trivalent verbs (1.5\%). All of them are formed with \textitbf{kasi} ‘give’, as in (0) and (0). In 12 of them, the T is \textitbf{nasihat} ‘advice’ as in (0), in two it is \textitbf{ijing} or \textitbf{ijing{\Tilde}ijing} ‘permission’ as in (0), and in the remaining three the Ts are \textitbf{ana} ‘child’, \textitbf{kemerdekaang} ‘independence’ and \textitbf{swara} ‘voice’.
\end{styleBodyvvafter}

\begin{styleExampleTitle}
Double-object constructions: T-R word order
\end{styleExampleTitle}

\begin{tabular}{llllllll}
\lsptoprule
\label{bkm:Ref348543243}
\gll {sa} {bilang} {begini,} {sa} {\bluebold{kasi}} {nasihat} {kamu}\\ %
& \textsc{1sg} & say & like.this & \textsc{1sg} & give & advice & \textsc{2pl}\\
\lspbottomrule
\end{tabular}
\ea
\glt 
‘I said like this, ‘I \bluebold{give} you advice’ \textstyleExampleSource{[081115-001a-Cv.0332]}
\z

\begin{tabular}{llllllll}
\lsptoprule
\label{bkm:Ref348543242}
\gll {adu,} {nene} {knapa} {\bluebold{kasi}} {ijing{\Tilde}ijing} {dia} {begitu}\\ %
& oh.no! & grandmother & why & give & \textsc{rdp}{\Tilde}permission & \textsc{3sg} & like.that\\
\lspbottomrule
\end{tabular}
\ea
\glt 
‘oh no!, why did you (‘grandmother’) \bluebold{give} him permission like that?’ \textstyleExampleSource{[081014-008-CvNP.0026]}
\z


In double-object constructions the R is most often encoded by a personal pronoun, namely in 42/47 attested constructions (89\%), as in (0) and (0). In the remaining five constructions, the R is encoded by a nominal. Three nominals occur in R-T constructions, namely in \textitbf{bli Andi} ‘buy Andi’ in (0), and in \textitbf{kirim bapa} ‘send father’, and \textitbf{minta Noferus} ‘request Noferus’. The remaining two occur in T-R constructions, namely in ‘\textitbf{kasi nasihat} R’ constructions. The respective Rs are \textitbf{pendeta} ‘pastor’ and \textitbf{ana{\Tilde}ana} ‘children’. These distributional frequencies are discussed in §11.1.3.5.
\end{styleBodyxvafter}

\paragraph[R{}-type oblique constructions]{R-type oblique constructions}
\label{bkm:Ref366767543}
One common alternative to double-object constructions is the “oblique strategy” {(Margetts and Austin 2007: 411)} in which “the verb takes only two direct arguments and the third participant is expressed as an oblique argument or an adjunct”. Very commonly, it is the R that is expressed with a prepositional phrase; hence “R-type oblique”. Alternatively, the T is encoded in this manner; hence, “T-type oblique”. ({Margetts and Austin 2007: 413; see also Malchukov et al. 2010: 17.)}\footnote{\\
\\
\\
\\
\\
\\
\\
\\
\\
\\
\\
\\
\\
\\
\\
\par Alternatively, the oblique strategy is also called “‘dative alternation’, earlier ‘dative shift’ or ‘dative movement’” {(Malchukov et al. 2010: 18)}; an alternative term for “R-type obliques” is “indirective alignment” {(2010: 3)}.}



In Papuan Malay oblique constructions, it is always the R that is expressed as an oblique, with the R following the T, as shown in (0) to (0). Overall, however, R-type oblique constructions are not very common. The corpus contains only 41 R-type obliques among the total of 1,160 verbal clauses formed with trivalent verbs (3.5\%). Moreover, in the corpus, R-type obliques are not attested for all seven verbs (the examples for \textitbf{bawa} ‘bring’ in (0), \textitbf{bli} ‘buy’ in (0), and \textitbf{kirim} ‘send’ in (0) are elicited). Most R-type obliques are introduced with the benefactive prepositions \textitbf{buat} ‘for’ or \textitbf{untuk} ‘for’ (26/41 tokens – 63\%), while the remaining 15 R-type obliques are formed with goal-oriented \textitbf{sama} ‘to’. (The semantics of the three prepositions are discussed in §10.2.)
\end{styleBodyvvafter}

\begin{styleExampleTitle}
R-type oblique constructions
\end{styleExampleTitle}

\begin{tabular}{llllll}
\lsptoprule
\label{bkm:Ref348362995}
\gll {pi} {\bluebold{ambil}} {bola} {sama} {ade}\\ %
& go & fetch & ball & to & ySb\\
\lspbottomrule
\end{tabular}
\ea
\glt 
[Talking to a young boy:] ‘go (and) \bluebold{fetch} the ball for the younger sibling!’ \textstyleExampleSource{[081011-009-Cv.0022]}
\z

\begin{tabular}{llllllllll}
\lsptoprule
\label{bkm:Ref348362996}
\gll {kemaring} {Lukas} {de} {\bluebold{bawa}} {kayu} {bakar} {buat} {Dodo} {dorang}\\ %
& yesterday & Lukas & \textsc{3sg} & bring & wood & burn & for & Dodo & \textsc{3pl}\\
\lspbottomrule
\end{tabular}
\ea
\glt 
‘yesterday Lukas \bluebold{brought} fire wood to Dodo and his associates for their benefit’ \textstyleExampleSource{[Elicited BR130221.035]}
\z

\begin{tabular}{lllllllllll}
\lsptoprule
\label{bkm:Ref348362997}
\gll {bapa} {de} {su} {\bluebold{bli}} {baju} {natal} {buat} {sa} {pu} {ade}\\ %
& father & \textsc{3sg} & already & buy & shirt & Christmas & for & \textsc{1sg} & \textsc{poss} & ySb\\
\lspbottomrule
\end{tabular}
\ea
\glt 
‘father already \bluebold{bought} a Christmas shirt for my younger sibling’ \textstyleExampleSource{[Elicited BR130221.002]}
\z

\begin{tabular}{llllllll}
\lsptoprule
\label{bkm:Ref349227831}
\gll {…} {nanti} {sa} {\bluebold{ceritra}} {ini} {sama} {dia}\\ %
&  & very.soon & \textsc{1sg} & tell & \textsc{d.prox} & to & \textsc{3sg}\\
\lspbottomrule
\end{tabular}
\ea
\glt 
‘[when he has returned home,] then I’ll \bluebold{tell} this to him’ \textstyleExampleSource{[080921-010-Cv.0004]}
\z

\begin{tabular}{llllllllll}
\lsptoprule
\label{bkm:Ref348362998}
\gll {sa} {\bluebold{kasi}} {hadia} {untuk} {kamu} {kalo} {kam} {kenal} {bapa}\\ %
& \textsc{1sg} & give & gift & for & \textsc{2pl} & if & \textsc{2pl} & know & father\\
\lspbottomrule
\end{tabular}
\ea
\glt 
‘I’ll \bluebold{give} a gift to you for your benefit if you recognize me (‘father’)’ \textstyleExampleSource{[080922-001a-CvPh.1334]}
\z

\begin{tabular}{lllllllllll}
\lsptoprule
\label{bkm:Ref349292545}
\gll {kaka} {dorang} {su} {\bluebold{kirim}} {uang} {banyak} {sama} {dong} {pu} {mama}\\ %
& oSb & \textsc{3pl} & already & send & money & many & to & \textsc{3pl} & \textsc{poss} & mother\\
\lspbottomrule
\end{tabular}
\ea
\glt 
‘older sibling and his/her associates already \bluebold{sent} lots of money to their mother’ \textstyleExampleSource{[Elicited BR130221.003]}
\z

\begin{tabular}{lllllllll}
\lsptoprule
\label{bkm:Ref348362999}
\gll {de} {bilang,} {yo,} {sa} {\bluebold{minta}} {maaf} {sama} {paytua}\\ %
& \textsc{3sg} & say & yes & \textsc{1sg} & request & pardon & to & husband\\
\lspbottomrule
\end{tabular}
\ea
\glt 
‘he said, ‘yes, I \bluebold{beg} pardon of (your) husband’’ \textstyleExampleSource{[081011-024-Cv.0140]}
\z


In the R-type oblique constructions in the corpus, the R is most often encoded by a noun or a noun phrase, namely in 28/41 attested constructions (68\%), as for instance in (0) and (0). In the remaining 13 constructions (32\%), the R is encoded by a personal pronoun, as in (0) or (0). The distributional frequencies and possible explanations for them are further discussed in §11.1.3.5.
\end{styleBodyxvafter}

\paragraph[Adnominal possessive constructions]{Adnominal possessive constructions}
\label{bkm:Ref366767544}
Another, cross-linguistic alternative to encode the R and T arguments is to express them in an adnominal possessive construction, in which “the agent and the theme are expressed as syntactic arguments of the verb, while the R-type participant, which will be the beneficiary with transfer verbs […], is expressed as a grammatical dependent of the theme, namely as its possessor” {(Margetts and Austin 2007: 426)}.



In Papuan Malay, speakers use adnominal possessive constructions when the T is definite. The corpus includes 14 such constructions among the 1,160 clauses formed with trivalent verbs (1.2\%). Examples are given for \textitbf{ambil} ‘fetch’ in (0), \textitbf{bli} ‘buy’ in (0), and \textitbf{kasi} ‘give’ in (0). In each case, the possessor denotes the benefiting R of the event expressed by the verb; the possessum denotes the T as the anticipated object of possession. In the corpus, the possessor is typically encoded by a personal pronoun (13/14 tokens – 93\%), as in (0) and (0). Only in one construction, presented in (0), the possessor is expressed with a noun, namely the proper noun \textitbf{Sofia}. (Adnominal possession is described in detail in Chapter 9.)
\end{styleBodyvvafter}

\begin{styleExampleTitle}
Adnominal possessive constructions
\end{styleExampleTitle}

\begin{tabular}{llllllll}
\lsptoprule
\label{bkm:Ref349294969}
\gll {mama} {nanti} {\bluebold{ambil}} {[Sofia} {pu} {ijasa} {SD]}\\ %
& mother & very.soon & fetch & Sofia & \textsc{poss} & diploma & primary.school\\
\lspbottomrule
\end{tabular}
\ea
\glt 
‘later you (‘mother’) \bluebold{fetch} the primary school diploma for Sofia’ (Lit. ‘Sofia’s primary school diploma’) \textstyleExampleSource{[081011-023-Cv.0065]}
\z

\begin{tabular}{llllllllll}
\lsptoprule
\label{bkm:Ref348372758}
\gll {dia} {punya} {ulang-taung} {kita} {\bluebold{bli}} {[de} {punya} {pakeang} {ulang-taung]}\\ %
& \textsc{3sg} & \textsc{poss} & birthday & \textsc{1pl} & buy & \textsc{3sg} & \textsc{poss} & clothes & birthday\\
\lspbottomrule
\end{tabular}
\ea
\glt 
‘(for) her birthday we \bluebold{buy} birthday clothes for her’ (Lit. ‘her birthday clothes’) \textstyleExampleSource{[081006-025-CvEx.0022]}
\z

\begin{tabular}{llllllllll}
\lsptoprule
\label{bkm:Ref348607542}
\gll {ibu} {distrik} {de} {\bluebold{kasi}} {[kitong} {dua} {pu} {uang} {ojek]}\\ %
& woman & district & \textsc{3sg} & give & \textsc{1pl} & two & \textsc{poss} & money & motorbike.taxi\\
\lspbottomrule
\end{tabular}
\ea
\glt
‘Ms. District \bluebold{gave} us two money for the motorbike taxis’ (Lit. ‘our two motorbike taxi money’) \textstyleExampleSource{[081110-002-Cv.0036]}
\end{styleFreeTranslEngxvpt}

\paragraph[Elision]{Elision}
\label{bkm:Ref366767545}
Elision is a third alternative to double-object constructions and used when the T and/or R are understood from the context. In this case, one or both of them can be omitted. In the corpus, this strategy is used in 1,058 of 1,160 verbal clauses formed with trivalent verbs (91\%).



Most often the R is elided and the T retained (601/1,058 tokens – 57\%); these distributional frequencies are further discussed in §11.1.3.5. Examples are given for \textitbf{bli} ‘buy’ in (0), \textitbf{ceritra} ‘tell’ in (0), and \textitbf{kirim} ‘send’ in (0).
\end{styleBodyvvafter}

\begin{styleExampleTitle}
Elision of R and retention of T
\end{styleExampleTitle}

\begin{tabular}{llllllllll}
\lsptoprule
\label{bkm:Ref348371046}
\gll {kalo} {besok} {ada} {berkat} {sa} {\bluebold{bli}} {Ø} {komputer} {baru}\\ %
& if & tomorrow & exist & blessing & \textsc{1sg} & buy &  & computer & be.new\\
\lspbottomrule
\end{tabular}
\ea
\glt 
‘if there is a (financial) blessing in the near future, I’ll \bluebold{buy} (us) a new computer’ \textstyleExampleSource{[081025-003-Cv.0086]}
\z

\begin{tabular}{llllllllll}
\lsptoprule
\label{bkm:Ref349230366}
\gll {malam} {nanti} {Matias} {bilang,} {mama} {\bluebold{ceritra}} {Ø} {dongeng} {ka?}\\ %
& night & very.soon & Matias & say & mother & tell &  & legend & or\\
\lspbottomrule
\end{tabular}
\ea
\glt 
‘later tonight Matias will say, ‘are you (‘mother’) going to \bluebold{tell} (me) a story?’’ \textstyleExampleSource{[081110-008-CvNP.0142]}
\z

\begin{tabular}{llllll}
\lsptoprule
\label{bkm:Ref348012696}
\gll {bapa} {\bluebold{kirim}} {Ø} {uang} {banyak{\Tilde}banyak!}\\ %
& father & send &  & money & \textsc{rdp}{\Tilde}many\\
\lspbottomrule
\end{tabular}
\ea
\glt 
‘[father I want to buy a cell-phone for myself,] father \bluebold{send} (me) lots of money!’ \textstyleExampleSource{[080922-001a-CvPh.0440]}
\z


Constructions with elided T and retained R\textsc{ }occur much less often in the corpus (75/1,058 tokens – 7\%). In most cases, the retained R is encoded as an oblique (49/75 tokens – 65\%). This is demonstrated for \textitbf{bawa} ‘bring’ in (0), \textitbf{ceritra} ‘tell’ in (0), and \textitbf{kasi} ‘give’ in (0).


\begin{styleExampleTitle}
Elision of T and retention of oblique R
\end{styleExampleTitle}

\begin{tabular}{llllllllllll}
\lsptoprule
\label{bkm:Ref348373445}
\gll {e,} {ko} {bawa} {Ø} {ke} {sana,} {ko} {\bluebold{bawa}} {Ø} {sama} {ade}\\ %
& hey! & \textsc{2sg} & bring &  & to & \textsc{l.dist} & \textsc{2sg} & bring &  & to & ySb\\
\lspbottomrule
\end{tabular}
\ea
\glt 
[Talking to a young boy:] ‘hey, \bluebold{bring} (the ball) over there, \bluebold{bring} (the ball) to the younger sibling’ \textstyleExampleSource{[081011-009-Cv.0015]}
\z

\begin{tabular}{lllllllll}
\lsptoprule
\label{bkm:Ref348373446}
\gll {…} {baru} {dia} {yang} {\bluebold{ceritra}} {Ø} {sama} {saya}\\ %
&  & and.then & \textsc{3sg} & \textsc{rel} & tell &  & to & \textsc{1sg}\\
\lspbottomrule
\end{tabular}
\ea
\glt 
‘[I’d already forgotten who this gentleman was,] and then (it was) him (who) \bluebold{told} (this story) to me’ \textstyleExampleSource{[080917-008-NP.0005]}
\z

\begin{tabular}{llllllll}
\lsptoprule
\label{bkm:Ref349326228}
\gll {ko} {\bluebold{kasi}} {Ø} {sama} {kaka} {mantri,} {e?}\\ %
& \textsc{2sg} & give &  & with & oSb & male.nurse & eh\\
\lspbottomrule
\end{tabular}
\ea
\glt 
‘\bluebold{give} (the keys) to the older brother nurse, eh?’ \textstyleExampleSource{[080922-010a-CvNF.0167]}
\z


Less often (26/75 tokens – 35\%), the retained R is encoded as a direct object. This is illustrated for \textitbf{kasi} ‘give’ in (0), and \textitbf{minta} ‘request’ in (0).


\begin{styleExampleTitle}
Elision of T and retention of direct-object R
\end{styleExampleTitle}

\begin{tabular}{lllllllll}
\lsptoprule
\label{bkm:Ref349229927}
\gll {…} {hari} {ini} {dorang} {bisa} {\bluebold{kasi}} {ko} {Ø}\\ %
&  & day & \textsc{d.prox} & \textsc{3pl} & be.able & give & \textsc{2sg} & \\
\lspbottomrule
\end{tabular}
\ea
\glt 
‘[if (you) say (you also want) a trillion (rupiah),] today they can \bluebold{give} you (the money)’ \textstyleExampleSource{[081029-004-Cv.0023]}
\z

\begin{tabular}{llllllllll}
\lsptoprule
\label{bkm:Ref348371041}
\gll {piring{\Tilde}piring} {kosong,} {sa} {\bluebold{minta}} {Ise} {Ø,} {sa} {bilang} {…}\\ %
& \textsc{rdp}{\Tilde}plate & be.empty & \textsc{1sg} & request & Ise &  & \textsc{1sg} & say & \\
\lspbottomrule
\end{tabular}
\ea
\glt 
‘the (cake) plates were empty, I \bluebold{asked} Ise (for a piece of cake), I said …’ \textstyleExampleSource{[081011-005-Cv.0034]}
\z


In constructions with elided T and retained R, the R is most often encoded by a nominal (56/75 tokens – 75\%). This applies to oblique Rs (39/49 – 80\%), as in (0), as well as to direct-object Rs (17/26 – 65\%), as in (0). Retained pronominal Rs, by contrast, occur much less often (19/75 tokens – 25\%), be they oblique Rs as in (0), or direct-object Rs as in (0). These distributional frequencies are discussed in §11.1.3.5.



Finally, elision can also affect the R and the T at the same time. That is, both can be omitted at once if they are understood from the context. In the corpus, this applies to a substantial number of verbal clauses formed with trivalent verbs (382/1,160 tokens – 36\%). This type of elision is illustrated for \textitbf{ambil} ‘fetch’ in (0), \textitbf{bli} ‘buy’ in (0), and \textitbf{kirim} ‘send’ in (0).
\end{styleBodyvvafter}

\begin{styleExampleTitle}
Elision of R and T
\end{styleExampleTitle}

\begin{tabular}{lllllllllll}
\lsptoprule
\label{bkm:Ref349234368}
\gll {…} {Matias} {nanti} {anjing,} {cepat,} {ko} {\bluebold{ambil}} {Ø} {Ø} {dulu!}\\ %
&  & Matias & very.soon & dog & be.fast & \textsc{2sg} & fetch &  &  & first\\
\lspbottomrule
\end{tabular}
\ea
\glt 
‘[Matias, younger sister’s fish fell down,] Matias, very soon the dogs (will get it), quick, you \bluebold{fetch} (your sister the fish)!’ \textstyleExampleSource{[081006-019-Cv.0002]}
\z

\begin{tabular}{lllllllllllll}
\lsptoprule
\label{bkm:Ref349295464}
\gll {…} {de} {pu} {tete} {tanya} {dia,} {ko} {\bluebold{bli}} {Ø} {Ø} {di} {mana?}\\ %
&  & \textsc{3sg} & \textsc{poss} & grandfather & ask & \textsc{3sg} & \textsc{2sg} & buy &  &  & at & where\\
\lspbottomrule
\end{tabular}
\ea
\glt 
‘[when the grandchild emerged, he was holding a fried banana,] then his grandfather asked him, ‘where did you \bluebold{buy} (yourself the fried banana)?’’ \textstyleExampleSource{[081109-005-JR.0007]}
\z

\begin{tabular}{llllllllll}
\lsptoprule
\label{bkm:Ref349234371}
\gll {…} {mama} {dong} {di} {kampung} {tra} {\bluebold{kirim}} {Ø} {Ø}\\ %
&  & mother & \textsc{3pl} & at & village & \textsc{neg} & send &  & \\
\lspbottomrule
\end{tabular}
\ea
\glt
‘[it’s difficult, there is no money,] mother and the others in the village don’t \bluebold{send} (us money)’ \textstyleExampleSource{[080922-001a-CvPh.0943/0945]}
\end{styleFreeTranslEngxvpt}

\paragraph[Distributional frequencies]{Distributional frequencies}
\label{bkm:Ref367258538}
The above description of how Papuan Malay trivalent verbs are used in verbal clauses shows three types of variation, namely in word order, in encoding the R and T arguments, and in eliding one or both of these arguments. The data also indicate distributional preferences for these three variation types. Summarizing this variation, this section provides an explanation for the distributional frequencies and preferences in terms of salience.



Cross-linguistically, ditransitive alignment variation is related to distinctions between the R and T arguments in terms of three “salience scales (animacy, definiteness, person)”, with {\citet[84]{Haspelmath2007b}} presenting the following scale for “differential R marking”:\footnote{\\
\\
\\
\\
\\
\\
\\
\\
\\
\\
\\
\\
\\
\\
\\
\par See also {Comrie’s (1989)} animacy hierarchy, {Dixon’s (1979: 85)} agency scale, and {Silverstein’s (1976)} hierarchy of features.}
\end{styleBodyvvafter}

\begin{styleIvI}
1st/2nd {\textgreater} 3rd {\textgreater} proper noun {\textgreater} human {\textgreater} non-human
\end{styleIvI}


When the R is more salient than the T, speakers favor a double-object construction. This preference applies especially to pronominal Rs, which are the most salient ones. Otherwise, as {\citet[83]{Haspelmath2007b}} states, the oblique construction is the favored one:


\begin{styleIvI}
Special (“indirective” or “dative”) R-marking is the more likely, the lower the R is on the animacy, definiteness, and person scales.
\end{styleIvI}


The same distributional preferences apply to Papuan Malay, as shown in Table  ‎11 .2. Before discussing the distribution of nominal and pronominal Rs, however, Table  ‎11 .1 gives an overview of the distributional frequencies for trivalent verbs in the different constructions types discussed in the preceding sections.



Table  ‎11 .1 shows that Papuan Malay disfavors clauses in which both the R and T arguments are overtly mentioned. Double-object constructions are rare (4.1\%); the 47 clauses include 30 clauses with R-T order and 17 with T-R order. Likewise, R-type oblique constructions are rare (3.5\%). Adnominal possessive constructions with an R possessor are even rarer (1.2\%). Instead, trivalent verbs usually occur in clauses with elided R and/or T arguments (91\%). Details on elision are presented in Table  ‎11 .3.
\end{styleBodyvvafter}

\begin{stylecaption}
\label{bkm:Ref366829879}Table ‎11.\stepcounter{Table}{\theTable}:  Distributional preferences for trivalent verbs
\end{stylecaption}

\tablehead{ & Token \# & \arraybslash \%\\
}
\begin{tabular}{lll}
\lsptoprule
DO & \raggedleft 47 & \raggedleft\arraybslash 4.1\%\\
Obl. & \raggedleft 41 & \raggedleft\arraybslash 3.5\%\\
AdPoss. & \raggedleft 14 & \raggedleft\arraybslash 1.2\%\\
Elision & \raggedleft 1,058 & \raggedleft\arraybslash 91.2\%\\
Total & \raggedleft 1,160 & \raggedleft\arraybslash 100\%\\
\lspbottomrule
\end{tabular}

As for the distribution of nominal and pronominal Rs, Table  ‎11 .2 indicates clear preferences. Only five nominal Rs occur in double-object constructions (6\%), and about one third in R-type oblique constructions (28/90 tokens – 31\%). Besides, one nominal R is used in an adnominal possessive construction (1\%). Instead, most nominal Rs occur in clauses with elided T arguments (56/90 tokens – 62\%; Table  ‎11 .3 gives details on elision). By contrast, about half of the pronominal Rs occur in double-object constructions (42/87 tokens – 48\%), while 13 Rs are used in R-type oblique constructions (15\%). Another 13 Rs occur in adnominal possessive constructions (15\%; compare with one token for nominal Rs). Yet another 19 Rs occur in clauses with elided T (22\%; compare with 56 nominal Rs).


\begin{stylecaption}
\label{bkm:Ref366829880}Table ‎11.\stepcounter{Table}{\theTable}:  Distribution of nominal and pronominal Rs\footnote{\\
\\
\\
\\
\\
\\
\\
\\
\\
\\
\\
\\
\\
\\
\\
\par As percentages are rounded to one decimal place, they do not always add up to 100\%.}
\end{stylecaption}

\tablehead{ & DO & Obl. & AdPoss. & T Elision & \arraybslash Total\\
}
\begin{tabular}{llllll}
\lsptoprule
\textsc{nom}{}-R & \raggedleft 5 & \raggedleft 28 & \raggedleft 1 & \raggedleft 56 & \raggedleft\arraybslash 90\\
& \raggedleft 5.6\% & \raggedleft 31.1\% & \raggedleft 1.1\% & \raggedleft 62.2\% & \raggedleft\arraybslash 100\%\\
\textsc{pro}{}-R & \raggedleft 42 & \raggedleft 13 & \raggedleft 13 & \raggedleft 19 & \raggedleft\arraybslash 87\\
& \raggedleft 48.3\% & \raggedleft 14.9\% & \raggedleft 14.9\% & \raggedleft 21.8\% & \raggedleft\arraybslash 100\%\\
Total & \raggedleft 47 & \raggedleft 41 & \raggedleft 14 & \raggedleft 75 & \raggedleft\arraybslash 177\\
& \raggedleft 26.6\% & \raggedleft 23.2\% & \raggedleft 7.9\% & \raggedleft 42.4\% & \raggedleft\arraybslash 100\%\\
\lspbottomrule
\end{tabular}

This tendency for pronominal Rs to occur in double-object constructions, while nominal Rs are more often used in R-type oblique constructions is in line with {Haspelmath’s (2007b: 84)} scale for differential R marking, presented above. As mentioned, this scale suggests that speakers favor a double-object construction when the R is more salient than the T, a preference that applies especially to pronominal Rs. Otherwise, speakers favor an oblique construction.
\end{styleBodyaftervbefore}


There is one exception, though. When speakers want to signal that a pronominal R is also the beneficiary of the transfer, they encode this R as an R-type oblique, which is introduced with benefactive \textitbf{buat} ‘for’ or \textitbf{untuk} ‘for’ (both prepositions and their semantics are discussed in §10.2). This benefactive marking of the R is not possible in double-object constructions. Hence, speakers have to use an R-type oblique construction; this applies to 13 pronominal Rs in the corpus occurring in R-type oblique constructions. In nine of them (70\%), the oblique is introduced with a benefactive preposition.
\end{styleBodyvafter}


As already discussed, however, Papuan Malay disfavors constructions in which the R and T arguments are both overtly mentioned. Instead, trivalent verbs usually occur in clauses in which the R and/or T arguments are elided (1,058/1,160 tokens – 91\%; see Table  ‎11 .1). Most often, the more salient R is omitted while the less salient T is retained (601/1,058 tokens – 57\%), as shown in Table  ‎11 .3. Clauses in which the R and the T are both elided at the same time are also rather common (382/1,058 tokens – 36\%). Only rarely, the T is omitted while the R is retained (75/1,058 tokens – 7\%).
\end{styleBodyvafter}


Retention of the R most often affects nominal Rs (\textsc{nom}{}-R) (56/75 tokens – 75\%); most of them are encoded as R-type obliques (39/56 tokens – 70\%). Retention of pronominal Rs (\textsc{pro}{}-R), which are more salient than nominal ones, is much less frequent (19/75 tokens – 25\%). In light of the data given in Table  ‎11 .2, one would expect the 19 pronominal Rs to be encoded as direct objects rather than as R-type obliques. As shown in Table  ‎11 .3, however, ten of the 19 pronominal Rs are encoded as R-type obliques (53\%). Again, this has to do with their marking as benefactive Rs: seven of the ten pronominal Rs are introduced with a benefactive preposition, similar to the 13 pronominal R-type obliques listed in Table  ‎11 .2.
\end{styleBodyvvafter}

\begin{stylecaption}
\label{bkm:Ref366829881}Table ‎11.\stepcounter{Table}{\theTable}:  Distributional preferences for argument elision and retention
\end{stylecaption}

\tablehead{ & R \textsc{els}\\
T \textsc{ret} & T \textsc{els}\\
DO-R \textsc{ret} & T \textsc{els}\\
Obl.-R \textsc{ret} & T \textsc{els}\\
R \textsc{els} & \arraybslash Total\\
}
\begin{tabular}{llllll}
\lsptoprule
\multicolumn{6}{l}{Distribution of elided and retained arguments}\\
Total & \raggedleft 601 & \raggedleft 26 & \raggedleft 49 & \raggedleft 382 & \raggedleft\arraybslash 1,058\\
& \raggedleft 57\% & \raggedleft 2\% & \raggedleft 5\% & \raggedleft 36\% & \raggedleft\arraybslash 100\%\\
\multicolumn{6}{l}{Encoding of retained Rs}\\
\textsc{nom}{}-R & \raggedleft {}-{}-{}- & \raggedleft 17 & \raggedleft 39 & \raggedleft {}-{}-{}- & \raggedleft\arraybslash 56\\
\textsc{pro}{}-R & \raggedleft {}-{}-{}- & \raggedleft 9 & \raggedleft 10 & \raggedleft {}-{}-{}- & \raggedleft\arraybslash 19\\
Total & \raggedleft {}-{}-{}- & \raggedleft 26 & \raggedleft 49 & \raggedleft {}-{}-{}- & \raggedleft\arraybslash 75\\
&  & \raggedleft 25\% & \raggedleft 75\% &  & \raggedleft\arraybslash 100\%\\
\lspbottomrule
\end{tabular}

An explanation for this preference to delete the R argument and to retain the T argument is given by {\citet{Polinsky1998}} in her study on asymmetries in double-object constructions (DOC) in English. The author explains the optional deletion of the R arguments “as sensitive to topic”, in that it applies “to those elements of [Information Structure …] that have already been activated and are accessible to speaker and hearer. More topical information is easily backgrounded, which explains why the recipient is more easily deleted” {(1998: 416)}. Hence, {\citet[407]{Polinsky1998}} presents the following implication: “If the patient of DOC can undergo optional deletion, the recipient of DOC can undergo optional deletion, too”.
\end{styleBodyaftervbefore}


This observation that the more accessible argument can be deleted also provides an explanation for the preference of Papuan Malay to elide the more salient R argument and to retain the less salient T argument.
\end{styleBodyvafter}


The observed tendency to omit the R and/or T arguments has also been noted for western Austronesian languages in general. In these languages, as {\citet[171]{Himmelmann2005}} points out, “there are few (if any) morphosyntactic constraints on the omission of coreferential arguments in clause sequences. That is, the possibility to omit a coreferential argument is not restricted to subject arguments”. This also applies to other eastern Malay varieties, such as Ambon Malay {(van Minde 1997: 209)}, and Manado Malay {(Stoel 2005: 133–154)}. Along similar lines, {\citet{Mosel2010}} notes for the Oceanic language Teop that “[all] three arguments of ditransitive constructions can be elided in both topical and non-topical positions”. These studies, however, do not discuss whether the languages under investigation have a preference for omitting the R or the T arguments in ditransitive constructions, and what the reasons for such a preference might be. An exception is {Klamer and Moro’s (2013)} study on ‘give’-constructions in heritage and baseline Ambon Malay. Noting that elision affects the R but not the T, the authors suggest that these distributional preferences are due to “a difference in the prominence of T and R” {(2013: 9)}.
\end{styleBodyvxvafter}

\section{Causative clauses}
\label{bkm:Ref367459479}
Papuan Malay employs three types of causative constructions: syntactic, lexical, and periphrastic causatives.



Generally speaking, causative clauses are constructions which involve two events: “(1) the causing event in which the causer does something, and (2) the caused event in which the causee carries out an action or undergoes a change of condition or state as a result of the causer’s action” {\citep[265]{Song2006}}. Hence, causative constructions are the result of a valency-increasing operation: in addition to the arguments of the cause event, or “non-causative predicate”, there is also the “causer” {\citep[175]{Comrie1989}}. This valency-increasing operation is possible with intransitive and transitive events.
\end{styleBodyvafter}


Cross-linguistically, four major strategies of encoding the notion of causation can be distinguished: lexical, morphological, syntactic, and periphrastic causatives. These constructions differ with respect to the degree of “structural integration” between the causing event, or the “predicate of cause”, and the caused event, or the “predicate of effect” {(Payne 1997: 159–160)}. Lexical causatives show a maximal degree of structural integration in that the cause and effect are encoded in a single lexical item. Periphrastic causative constructions, by contrast, show the least degree of structural integration in that the cause and effect are encoded in two separate clauses. According to {Kulikov (2001: 888–889)}, however, lexical causatives do not “qualify as \textstyleChItalic{causatives sensu stricto}” as they do not involve a morphological or syntactic change; neither do periphrastic constructions qualify as \textstyleChItalic{causatives sensu stricto} given their biclausal structure.
\end{styleBodyvafter}


Morphological and syntactic causatives differ from lexical and periphrastic causatives in that they integrate the cause with the caused event into a single predication. Hence, a causativized intransitive event yields a transitive causative construction, while a causativized transitive caused event yields a ditransitive construction. The integration of the causer is achieved by demoting the agent of the caused event, the causee. Cross-linguistically, {\citet[176]{Comrie1989}} notes the following grammatical relation hierarchy for this process: “subject {\textgreater} direct object {\textgreater} indirect object {\textgreater} oblique object”; that is, “the causee occupies the highest (leftmost) position on this hierarchy that is not already filled”.
\end{styleBodyvafter}


As mentioned, Papuan Malay uses three of the four types of causative constructions: lexical, syntactic, and periphrastic causatives; morphological causatives are unattested. The main topic of this section is syntactic causatives (§11.2.1), since only they qualify as \textstyleChItalic{causatives sensu stricto} {(Kulikov 2001: 888–889)}. Lexical and periphrastic causatives are mentioned only briefly in §11.2.2 and §11.2.3, respectively. The main points of this section are summarized in §11.2.4.
\end{styleBodyvxvafter}

\subsection{Syntactic causatives}
\label{bkm:Ref290751396}
In syntactic causatives, or “compound” causatives {\citep[450]{Song2011}}, the notion of causation is encoded in a monoclausal construction which consists of two constituents, namely a causative verb, which expresses the notion of cause, and a second constituent that denotes the effect {\citep[887]{Kulikov2001}}.



In Papuan Malay syntactic causatives, a serial verb construction V\textsubscript{1}V\textsubscript{2} encodes the causation: the causative verb V\textsubscript{1} expresses the cause event and the V\textsubscript{2} the caused event. Two free verb forms are used as causative verbs: trivalent \textitbf{kasi} ‘give’ and bivalent \textitbf{biking} ‘make’. In \textitbf{kasi}{}-causatives the V\textsubscript{2} can be monovalent or bivalent while in \textitbf{biking}{}-causatives the V\textsubscript{2} is always monovalent.
\end{styleBodyvafter}


Semantically, causatives with \textitbf{kasi} ‘give’ focus on the outcome of the causation or manipulation. Causatives with \textitbf{biking} ‘make’, by contrast, focus on the manipulation of circumstances that ultimately leads to the caused event or effect. This is shown with the contrastive examples in (0) and (0) both of which are formed with monovalent stative \textitbf{bersi} ‘be clean’. In (0), \textitbf{kasi bersi} ‘cause to be clean’ stresses the outcome of the washing process, namely that the clothes are clean. In the elicited example in (0), by contrast, \textitbf{biking bersi} ‘make clean’ focuses on the manipulation itself, which leads to the effect that the clothes are clean.
\end{styleBodyvvafter}

\begin{styleExampleTitle}
\textitbf{kasi} ‘give’ versus \textitbf{biking} ‘make’ causatives
\end{styleExampleTitle}

\begin{tabular}{lllllll}
\lsptoprule
\label{bkm:Ref371154580}
\gll {malam} {cuci} {pakeang} {\bluebold{kasi}} {\bluebold{bersi}} {jemur}\\ %
& night & wash & clothes & give & be.clean & dry\\
\lspbottomrule
\end{tabular}
\ea
\glt 
‘(if you have to do laundry at night time) wash (your clothes), \bluebold{clean} them, (and hang them up) to dry’ \textstyleExampleSource{[081011-019-Cv.0009]}
\z

\begin{tabular}{lllllll}
\lsptoprule
\label{bkm:Ref371154582}
\gll {malam} {cuci} {pakeang} {\bluebold{biking}} {\bluebold{bersi}} {jemur}\\ %
& night & wash & clothes & make & be.clean & dry\\
\lspbottomrule
\end{tabular}
\ea
\glt 
‘(if you have to do laundry at night time) wash (your clothes), \bluebold{clean} them, (and hang them up) to dry’ \textstyleExampleSource{[Elicited BR131103.001]}
\z


The following sections discuss the syntax and semantics of Papuan Malay syntactic causatives in more detail. The two verbs that qualify as causative verbs are presented in §11.2.1.1, followed by a description of syntactic causatives with the causative verb \textitbf{kasi} ‘give’ in §11.2.1.2, and with \textitbf{biking} ‘make’ in §11.2.1.3.
\end{styleBodyxvafter}

\paragraph[Causative verbs]{Causative verbs}
\label{bkm:Ref290751580}
The Papuan Malay verbs which express the notion of cause in syntactic causatives, \textitbf{kasi} ‘give’ and \textitbf{biking} ‘make’, are used synchronically as full transitive verbs, as shown in (0) to (0). Trivalent \textitbf{kasi} ‘give’ exhibits dative alternation, as illustrated with the double-object constructions in (0) and the R-type oblique construction in (0) (see §11.1.3 for more details on dative alternation). The transitive uses of \textitbf{biking} ‘make’ are illustrated in (0).
\end{styleBodyxafter}

\begin{tabular}{lllllll}
\lsptoprule
\label{bkm:Ref365011709}
\gll {a,} {kam} {\bluebold{kasi}} {sa} {air} {ka}\\ %
& ah & \textsc{2pl} & give & \textsc{1sg} & water & or\\
\lspbottomrule
\end{tabular}
\ea
\glt 
‘ah, you \bluebold{give} me water, please’ \textstyleExampleSource{[080919-008-CvNP.0005]}
\z

\begin{tabular}{lllllll}
\lsptoprule
\label{bkm:Ref365011710}
\gll {de} {\bluebold{kasi}} {sratus} {ribu} {sama} {Madga}\\ %
& \textsc{3sg} & give & one.hundred & thousand & to & Madga\\
\lspbottomrule
\end{tabular}
\ea
\glt 
‘he \bluebold{gave} one hundred thousand (rupiah) to Madga’ \textstyleExampleSource{[081014-003-Cv.0008]}
\z

\begin{tabular}{llll}
\lsptoprule
\label{bkm:Ref365011708}
\gll {Ika} {\bluebold{biking}} {papeda}\\ %
& Ika & make & sagu.porridge\\
\lspbottomrule
\end{tabular}
\ea
\glt
‘Ika \bluebold{made} sagu porridge’ \textstyleExampleSource{[081006-032-Cv.0071]}
\end{styleFreeTranslEngxvpt}

\paragraph[Syntactic causatives with kasi ‘give’]{Syntactic causatives with \textitbf{kasi} ‘give’}
\label{bkm:Ref290797493}
As a causative, trivalent \textitbf{kasi} ‘give’, with its short form \textitbf{kas}, is used with two types of verbal bases: monovalent ones, as in (0) to (0), or bivalent ones as in (0) and (0). Semantically, causative \textitbf{kasi} ‘give’ highlights the outcome of a causation.
\end{styleBodyxvafter}

\subparagraph[Monovalent bases]{Monovalent bases}

Cross-linguistically, in causatives with monovalent bases, the agent of the caused event is demoted from its intransitive subject function (S) to the transitive object or \textsc{patient} (P) function, while the incoming causer takes the transitive subject or \textsc{agent} (A) function {(Comrie 1989: 110–111)}. This strategy, which corresponds to {Comrie’s (1989: 176)} mentioned causative hierarchy, is also used in Papuan Malay causatives with monovalent bases. This is illustrated with the monoclausal causative constructions in (0) to (0): causatives with monovalent non-agentive bases are presented in (0) to (0) and causatives with monovalent agentive bases in (0) to (0). (Compare also with the biclausal causative constructions in §11.2.3.)



In causatives with monovalent non-agentive bases, the effect expression can be a stative verb such as \textitbf{panjang} ‘be long’ in (0), or a non-agentive dynamic verb such as \textitbf{gugur} ‘fall (prematurely)’ in (0). The resulting V\textsubscript{1}V\textsubscript{2} expressions function as transitive predicates.
\end{styleBodyvvafter}

\begin{styleExampleTitle}
Causatives with monovalent non-agentive bases
\end{styleExampleTitle}

\begin{tabular}{lllllll}
\lsptoprule
\label{bkm:Ref365021741}
\gll {…} {mama} {harus} {\bluebold{kas}} {\bluebold{panjang}} {kaki}\\ %
&  & mother & have.to & give & long & foot\\
\lspbottomrule
\end{tabular}
\ea
\glt 
[Addressing someone with a bad knee:] ‘[you shouldn’t fold (your legs) under,] you (‘mother’) have to \bluebold{stretch out} (your) legs’ \textstyleExampleSource{[080921-004a-CvNP.0069]}
\z

\begin{tabular}{llllllll}
\lsptoprule
(\stepcounter{}{\the}) & ko & \bluebold{kasi} & \bluebold{sembu} & sa & punya & ana & ini!\\
& \textsc{2sg} & give & be.healed & \textsc{1sg} & \textsc{poss} & child & \textsc{d.prox}\\
\lspbottomrule
\end{tabular}
\ea
\glt 
[Addressing an evil spirit:] ‘you \bluebold{heal} this child of mine!’ \textstyleExampleSource{[081006-023-CvEx.0031]}
\z

\begin{tabular}{llllllllll}
\lsptoprule
\label{bkm:Ref365021743}
\gll {perna} {dia} {\multicolumn{2}{l}{punya}} {pikirang} {untuk} {de} {mo} {\bluebold{kasi}}\\ %
& ever & \textsc{3sg} & \multicolumn{2}{l}{have} & thought & for & \textsc{3sg} & want & give\\
& \multicolumn{3}{l}{\bluebold{gugur}} & \multicolumn{6}{l}{Ø}\\
& \multicolumn{3}{l}{fall(.prematurely)} & \multicolumn{6}{l}{}\\
\lspbottomrule
\end{tabular}
\ea
\glt 
‘once she had the thought that she wanted to \bluebold{abort} (the child)’ \textstyleExampleSource{[080917-010-CvEx.0097]}
\z

\begin{tabular}{llllllll}
\lsptoprule
\label{bkm:Ref436750473}\label{bkm:Ref365021742}
\gll {banyak} {mati} {di} {lautang,} {\bluebold{kas}} {\bluebold{tenggelam}} {Ø}\\ %
& many & die & at & ocean & give & sink & \\
\lspbottomrule
\end{tabular}
\ea
\glt 
[About people in a container who died in the ocean:] ‘many died in the (open) ocean, (the murderers) \bluebold{sank} (the containers)’ \textstyleExampleSource{[081029-002-Cv.0025]}
\z


In causatives with monovalent agentive bases, the effect expression is encoded by a monovalent dynamic verb, as shown in (0) to (0).


\begin{styleExampleTitle}
Causatives with monovalent agentive bases
\end{styleExampleTitle}

\begin{tabular}{llllllllll}
\lsptoprule
\label{bkm:Ref371179284}
\gll {sa} {di} {bawa,} {Roni} {\bluebold{kas}} {\bluebold{duduk}} {sa} {di} {atas}\\ %
& \textsc{1sg} & at & bottom & Roni & give & sit & \textsc{1sg} & at & top\\
\lspbottomrule
\end{tabular}
\ea
\glt 
[A ten-year old boy on a truck-trip:] ‘I was down (in the cargo area, but) Roni \bluebold{enabled} me \bluebold{to sit} on top (of the cab)’ \textstyleExampleSource{[081022-002-CvNP.0012]}
\z

\begin{tabular}{lllllllllll}
\lsptoprule
\label{bkm:Ref365097670}
\gll {…} {tapi} {dong} {kasi} {bangkit} {dia} {lagi,} {\bluebold{kasi}} {\bluebold{hidup}} {dia}\\ %
&  & but & \textsc{3pl} & give & be.resurrected & \textsc{3sg} & again & give & live & \textsc{3sg}\\
\lspbottomrule
\end{tabular}
\ea
\glt 
[About sorcerers who can resurrect the dead:] ‘[he’s already (dead),] but they resurrect him again, \bluebold{make} him \bluebold{live}’ \textstyleExampleSource{[081006-022-CvEx.0095]}
\z

\begin{tabular}{llllll}
\lsptoprule
\label{bkm:Ref365036185}
\gll {kam} {\bluebold{kas}} {\bluebold{kluar}} {pasir} {dulu!}\\ %
& \textsc{2pl} & give & go.out & sand & first\\
\lspbottomrule
\end{tabular}
\ea
\glt 
‘you \bluebold{remove} the sand first!’ \textstyleExampleSource{[080925-002-CvHt.0005]}
\z

\begin{tabular}{lllllllll}
\lsptoprule
\label{bkm:Ref365036186}
\gll {kam} {\bluebold{kas}} {\bluebold{kluar}} {Dodo} {dari} {dalam} {meja} {situ!}\\ %
& \textsc{2pl} & give & go.out & Dodo & from & inside & table & \textsc{l.med}\\
\lspbottomrule
\end{tabular}
\ea
\glt 
[About a fearful person hiding under the table:] ‘you \bluebold{remove} Dodo / \bluebold{enable} Dodo \bluebold{to get out} from under the table there!’ \textstyleExampleSource{[081025-009b-Cv.0028]}
\z


Cross-linguistically, causative constructions receive different readings, depending on the causee’s level of agentivity {(Kulikov 2001: 891–893)}. This also applies to Papuan Malay. When the causee has no control, the causative receives a “manipulative or directive” reading, while it receives an “assistive or cooperative” reading, when the causee has some level of agentivity {(2001: 892)}.



In causatives with monovalent non-agentive bases, as in (0) to (0), the causer controls the event while the causee has no control. Hence, these causatives always receive a directive reading. Likewise, causatives with monovalent agentive bases receive a directive reading when the causee is inanimate, or animate but helpless. This is the case in (0) and (0). When, by contrast, the causee has some level of control, as in (0), the causation is less direct; hence, the causative receives an assistive reading. Sometimes, however, the reading of a causative is ambiguous, as in (0). If the causee \textitbf{Dodo} unconscious out of fear and thereby helpless, the causative receives the directive reading ‘remove’. But if \textitbf{Dodo} is conscious and can move, the causative receives the assistive reading ‘enable to come out’.
\end{styleBodyvxvafter}

\subparagraph[Bivalent bases]{Bivalent bases}

In causatives with bivalent bases, the cross-linguistically expected operation is for the \textsc{patient} (P) of the caused event to retain its P function and for the \textsc{agent} (A) of the caused event to be demoted to the indirect object function {\citep[176]{Comrie1989}}.



Papuan Malay, however, uses a different strategy, in that all the arguments involved shift their functions. That is, the A of the caused event, or causee, is demoted to the P function. In turn, the P of the caused event is moved out of the core into an oblique slot; as an oblique, P is encoded in a prepositional phrase introduced with comitative \textitbf{dengang} ‘with’, with its short form \textitbf{deng} (see also §10.2.1). This is shown with the examples in (0) and (0).
\end{styleBodyvafter}


In (0), for instance, the original A, or causee, \textitbf{anjing} ‘dog’, is demoted to the P function and juxtaposed to the V\textsubscript{1}V\textsubscript{2} construction. Semantically, the causee becomes the theme argument of the causative expression \textitbf{kas makang} ‘give to eat’. With the P slot being taken, the original P \textitbf{papeda} ‘sagu porridge’ is moved out of the core into an oblique slot.
\end{styleBodyvvafter}

\begin{styleExampleTitle}
Causatives with bivalent bases: Demoting the A and P functions
\end{styleExampleTitle}

\begin{tabular}{lllllll}
\lsptoprule
\label{bkm:Ref365021749}
\gll {saya} {\bluebold{kas}} {\bluebold{makang}} {anjing} {deng} {papeda}\\ %
& \textsc{1sg} & give & eat & dog & with & sagu.porridge\\
\lspbottomrule
\end{tabular}
\ea
\glt 
‘I \bluebold{fed} the dogs with papeda’ \textstyleExampleSource{[080919-003-NP.0002]}
\z

\begin{tabular}{llllllll}
\lsptoprule
\label{bkm:Ref439601467}\label{bkm:Ref365021748}
\gll {dia} {\bluebold{kasi}} {\bluebold{minum}} {kitong} {dengang} {kopi} {air}\\ %
& \textsc{3sg} & give & drink & \textsc{1pl} & with & coffee & water\\
\lspbottomrule
\end{tabular}
\ea
\glt 
‘he’ll \bluebold{give} us coffee and water \bluebold{to drink}’ \textstyleExampleSource{[080919-004-NP.0069]}
\z


In the attested causatives with bivalent bases, the causees are able to control their own actions. Therefore, \textitbf{kasi} ‘give’ receives an assistive or cooperative reading, as in (0) and (0). Causative with bivalent bases and inanimate, or animate but helpless referents are unattested.
\end{styleBodyxvafter}

\paragraph[Syntactic causatives with biking ‘make’]{Syntactic causatives with \textitbf{biking} ‘make’}
\label{bkm:Ref290798166}
As a causative, bivalent \textitbf{biking} ‘make’ is used with monovalent bases. Semantically, this causative type stresses the causer’s manipulation of circumstances, which leads to the caused event or effect. That is, \textitbf{biking}{}-causatives are causer-controlled, with the causee having no control. Therefore, causatives with \textitbf{biking} ‘make’ are formed with monovalent non-agentive bases, or with monovalent agentive bases with inanimate or with animate but helpless causees. This is shown in (0) to (0). Overall, though, \textitbf{biking}{}-causatives are rare in the corpus.



The causative in (0), for example, is formed with non-agentive stative \textitbf{pusing} ‘be dizzy, be confused’. The use of \textitbf{biking} ‘make’ stresses the manipulating behavior of the causer \textitbf{ana{\Tilde}ana} ‘children’ which leads to the effect \textitbf{pusing} ‘be worried’; the causee \textitbf{mama} ‘mother’ has no control. The elicited examples in (0) and (0) contrast with the corresponding \textitbf{kasi}{}-causatives in (0) and (0). They show that \textitbf{biking}{}-causatives are also formed with monovalent non-agentive dynamic bases, such as \textitbf{gugur} ‘abort’ or \textitbf{tenggelam} ‘sink’, respectively. Again, the manipulation itself is stressed. The base can also be agentive dynamic if the causee is animate but helpless. This is illustrated with the elicited example in (0), which contrasts with the corresponding \textitbf{kasi}{}-causative in (0). The base is agentive dynamic \textitbf{hidup} ‘live’ but the animate causee is helpless and therefore has no control.
\end{styleBodyvvafter}

\begin{styleExampleTitle}
Causatives with monovalent non-agentive bases
\end{styleExampleTitle}

\begin{tabular}{lllll}
\lsptoprule
\label{bkm:Ref365021737}
\gll {ana{\Tilde}ana} {\bluebold{biking}} {\bluebold{pusing}} {mama}\\ %
& \textsc{rdp}{\Tilde}child & make & be.dizzy & mother\\
\lspbottomrule
\end{tabular}
\ea
\glt 
‘the kids \bluebold{worry} (their mother)’ (Lit. ‘\bluebold{make to be dizzy/confused}’) \textstyleExampleSource{[081014-007-CvEx.0047]}
\z

\begin{tabular}{llllllllll}
\lsptoprule
\label{bkm:Ref371178806}
\gll {perna} {dia} {punya} {pikirang} {untuk} {de} {mo} {\bluebold{biking}} {\bluebold{gugur}}\\ %
& ever & \textsc{3sg} & have & thought & for & \textsc{3sg} & want & make & fall(.prematurely)\\
\lspbottomrule
\end{tabular}
\ea
\glt 
‘once she had the thought that she wanted to \bluebold{abort} (the child)’ \textstyleExampleSource{[Elicited BR131103.002]}
\z

\begin{tabular}{lllllll}
\lsptoprule
\label{bkm:Ref371407708}
\gll {banyak} {mati} {di} {lautang,} {\bluebold{biking}} {\bluebold{tenggelam}}\\ %
& many & die & at & ocean & make & sink\\
\lspbottomrule
\end{tabular}
\ea
\glt 
[About people in a container who died in the ocean:] ‘many died in the (open) ocean, (the murderers) \bluebold{sank} (the containers)’ \textstyleExampleSource{[Elicited BR131103.003]}
\z

\begin{tabular}{lllllllllll}
\lsptoprule
\label{bkm:Ref371407218}
\gll {…} {tapi} {dong} {\bluebold{biking}} {bangkit} {dia} {lagi,} {\bluebold{biking}} {\bluebold{hidup}} {dia}\\ %
&  & but & \textsc{3pl} & make & be.resurrected & \textsc{3sg} & again & make & live & \textsc{3sg}\\
\lspbottomrule
\end{tabular}
\ea
\glt 
[About sorcerers who can resurrect the dead:] ‘[he’s already (dead),] but they \bluebold{resurrect} him again, \bluebold{make} him \bluebold{live}’ \textstyleExampleSource{[Elicited BR131103.005]}
\z


Causatives with agentive bases are unacceptable. This is due to the fact that \textitbf{biking}{}-causatives focus on the causer’s manipulation of circumstances itself while the causee has no control. This is illustrated with the unacceptable \textitbf{biking}{}-causatives in (0) and (0), which are formed with monovalent dynamic \textitbf{duduk} ‘sit’ and bivalent \textitbf{makang} ‘eat’ respectively. The two elicited examples contrast with the corresponding \textitbf{kasi}{}-causatives in (0) and (0).


\begin{styleExampleTitle}
Causatives with monovalent and bivalent agentive bases
\end{styleExampleTitle}

\begin{tabular}{lllllllllll}
\lsptoprule
\label{bkm:Ref371408574}
\gll {*} {sa} {di} {bawa,} {Roni} {\bluebold{biking}} {\bluebold{duduk}} {sa} {di} {atas}\\ %
&  & \textsc{1sg} & at & bottom & Roni & make & sit & \textsc{1sg} & at & top\\
\lspbottomrule
\end{tabular}
\ea
\glt 
Intended reading: ‘I was down (in the cargo area, but) Roni \bluebold{made} me \bluebold{sit} on top (of the cab)’ \textstyleExampleSource{[Elicited BR131103.006]}
\z

\begin{tabular}{llllllll}
\lsptoprule
\label{bkm:Ref371408575}
\gll {*} {saya} {\bluebold{biking}} {\bluebold{makang}} {anjing} {deng} {papeda}\\ %
&  & \textsc{1sg} & make & eat & dog & with & sagu.porridge\\
\lspbottomrule
\end{tabular}
\ea
\glt
Intended reading: ‘I \bluebold{made} the dogs \bluebold{eat} papeda’ \textstyleExampleSource{[Elicited BR131103.009]}
\end{styleFreeTranslEngxvpt}

\subsection{Lexical causatives}
\label{bkm:Ref290751393}
Generally speaking, lexical causatives “are in a suppletive relation with their non-causative counterparts” {\citep[887]{Kulikov2001}}. That is, the notion of causation is encoded in the semantics of the causative verb itself and not in an additional morpheme as in syntactic causative constructions.



For Papuan Malay, this suppletive relation is illustrated with the lexical causatives \textitbf{bunu} ‘kill’ and \textitbf{tebang} ‘fell’ in (0) and (0), and their respective non-causative counterparts \textitbf{mati} ‘die’ and \textitbf{jatu} ‘fall’ in (0) and (0), respectively.
\end{styleBodyvxafter}

\begin{tabular}{lllllllllllll}
\lsptoprule
\label{bkm:Ref365011704}
\gll {\multicolumn{2}{l}{de}} {\multicolumn{2}{l}{bisa}} {\multicolumn{2}{l}{jalang}} {gigit,} {\multicolumn{2}{l}{\bluebold{bunu}}} {manusia,} {sperti} {ular,}\\ %
& \multicolumn{2}{l}{\textsc{3sg}} & \multicolumn{2}{l}{be.able} & \multicolumn{2}{l}{walk} & bite & \multicolumn{2}{l}{kill} & human.being & similar.to & snake\\
& de & \multicolumn{2}{l}{bisa} & \multicolumn{2}{l}{gigit,} & \multicolumn{3}{l}{orang} & \multicolumn{4}{l}{\bluebold{mati}}\\
& \textsc{3sg} & \multicolumn{2}{l}{be.able} & \multicolumn{2}{l}{bite} & \multicolumn{3}{l}{person} & \multicolumn{4}{l}{die}\\
\lspbottomrule
\end{tabular}
\ea
\glt 
[About an evil spirit:] ‘he/she can go (and) bite (and) \bluebold{kill} humans like a snake, he/she can bite (and) someone \bluebold{dies}’ \textstyleExampleSource{[081006-022-CvEx.0133]}
\z

\begin{tabular}{llllllll}
\lsptoprule
\label{bkm:Ref365011706}
\gll {…} {itu} {yang} {monyet} {\bluebold{jatu}} {dari} {atas}\\ %
&  & \textsc{d.dist} & \textsc{rel} & monkey & fall & from & top\\
\lspbottomrule
\end{tabular}
\ea
\glt 
‘…that’s why the monkey \bluebold{fell} off from the top (of the banana plant)’ \textstyleExampleSource{[081109-002-JR.0005]}
\z

\begin{tabular}{llll}
\lsptoprule
\label{bkm:Ref365011705}
\gll {mo} {\bluebold{tebang}} {sagu}\\ %
& want & fell & sago\\
\lspbottomrule
\end{tabular}
\ea
\glt
‘(I) want to \bluebold{fell} a sago tree’ \textstyleExampleSource{[081014-006-CvPr.0069]}
\end{styleFreeTranslEngxvpt}

\subsection{Periphrastic causative constructions}
\label{bkm:Ref365275569}
Cross-linguistically, periphrastic causative constructions are defined as constructions which involve two predicates: (1) a “matrix predicate” which “contains the notion of causation”, the “predicate of cause”, and (2) an embedded predicate which “expresses the effect of the causative situation”, the “predicate of effect” {(Payne 1997: 159–160)}.



Papuan Malay periphrastic causative constructions are presented in (0) to (0). The matrix verb is \textitbf{kasi} ‘give’ in (0) and (0), and \textitbf{biking} ‘make’ in (0) and (0). Besides, Papuan Malay forms periphrastic causatives with a wide range of speech verbs; they are not further discussed here.
\end{styleBodyvxafter}

\begin{tabular}{lllllllll}
\lsptoprule
\label{bkm:Ref365013633}
\gll {kalo} {de} {minta} {kesembuang,} {setang} {\bluebold{kasi}} {\bluebold{de}} {\bluebold{sembu}}\\ %
& if & \textsc{3sg} & ask & recovery & evil.spirit & give & \textsc{3sg} & be.healed\\
\lspbottomrule
\end{tabular}
\ea
\glt 
‘when she asks for recovery, the evil spirit \bluebold{has her healed}’ \textstyleExampleSource{[081006-023-CvEx.0082]}
\z

\begin{tabular}{lllllllll}
\lsptoprule
\label{bkm:Ref365013632}
\gll {…} {baru} {mo} {biking} {papeda} {\bluebold{kasi}} {\bluebold{ana{\Tilde}ana}} {\bluebold{makang}}\\ %
&  & and.then & want & make & sagu.porridge & give & \textsc{rdp}{\Tilde}child & food\\
\lspbottomrule
\end{tabular}
\ea
\glt 
‘[they said (they) wanted to catch chickens,] and then (they) wanted to make sagu porridge (and) \bluebold{have the children eat}’ \textstyleExampleSource{[081010-001-Cv.0191]}
\z

\begin{tabular}{lllllllll}
\lsptoprule
\label{bkm:Ref365013628}
\gll {de} {pu} {swami} {\bluebold{biking}} {\bluebold{de}} {\bluebold{sakit}} {\bluebold{hati}} {to?}\\ %
& \textsc{3sg} & \textsc{poss} & husband & make & \textsc{3sg} & be.sick & liver & right?\\
\lspbottomrule
\end{tabular}
\ea
\glt 
‘her husband \bluebold{made her feel miserable}, right?’ \textstyleExampleSource{[081025-006-Cv.0161]}
\z

\begin{tabular}{llllllll}
\lsptoprule
\label{bkm:Ref365013629}
\gll {kata} {itu} {tu} {yang} {\bluebold{biking}} {\bluebold{sa}} {\bluebold{bertahang}}\\ %
& word & \textsc{d.dist} & \textsc{d.dist} & \textsc{rel} & make & \textsc{1sg} & hold(.out/back)\\
\lspbottomrule
\end{tabular}
\ea
\glt
‘(it was) those very words that \bluebold{made me hold out}’ \textstyleExampleSource{[081115-001a-Cv.0234]}
\end{styleFreeTranslEngxvpt}

\subsection{Summary}
\label{bkm:Ref365537073}
Papuan Malay employs three different strategies to express the notion of causation: syntactic, periphrastic, and lexical causatives. The description of causation focused on the syntax and semantics of syntactic causatives. Lexical and periphrastic causatives were discussed only briefly.



Papuan Malay syntactic causatives are monoclausal V\textsubscript{1}V\textsubscript{2} constructions in which a causative verb V\textsubscript{1}, namely trivalent \textitbf{kasi} ‘give’ or bivalent \textitbf{biking} ‘make’, encodes the notion of cause while the V\textsubscript{2} denotes the notion of effect. Syntactic causatives have monovalent or bivalent bases. In causatives with monovalent bases, the original A is demoted from its intransitive S function to the transitive P function, while the incoming causer takes the transitive A function. In causatives with bivalent bases, the original A is demoted to the P function while the original P is moved out of the core into an oblique slot. Hence, in causatives with monovalent bases the grammatical relations correspond to those established by {\citet[176]{Comrie1989}}, whereas in causatives with bivalent bases they do not correspond.
\end{styleBodyvafter}


Semantically, causatives with \textitbf{kasi} ‘give’ focus on the outcome of the manipulation, whereas causatives with \textitbf{biking} ‘make’ focus on the manipulation of the circumstances itself, which results in the effect. Both causative verbs typically generate causer-controlled causatives, in which the causer controls the event while the causee has no agentivity. This applies especially to \textitbf{biking}{}-causatives which stress the manipulation itself. Causatives with \textitbf{kasi} ‘give’ however, can also receive an assistive, rather than the typical directive, reading. This applies to agentive monovalent or bivalent bases when the causee has some level of agentivity.
\end{styleBodyvafter}


Most causative constructions in the corpus are formed with \textitbf{kasi} ‘give’ while causatives with \textitbf{biking} ‘make’ are much fewer. Table  ‎11 .4 lists the type and token frequencies for both causative verbs in the corpus.
\end{styleBodyvvafter}

\begin{stylecaption}
\label{bkm:Ref365123879}Table ‎11.\stepcounter{Table}{\theTable}:  Frequencies of causative constructions
\end{stylecaption}

\tablehead{ & \multicolumn{4}{l}{ \textitbf{kasi} ‘give’} & \multicolumn{4}{l}{ \textitbf{biking} ‘make’}\\
Base & \multicolumn{2}{l}{ Type \# / \%} & \multicolumn{2}{l}{ Token \# / \%} & \multicolumn{2}{l}{ Type \# / \%} & \multicolumn{2}{l}{ Token \# / \%}\\
}
\begin{tabular}{lllllllll}
\lsptoprule
\textsc{v.mo(st)} & \raggedleft 24 & \raggedleft 30\% & \raggedleft 36 & \raggedleft 8\% & \raggedleft 16 & \raggedleft 100\% & \raggedleft 25 & \raggedleft\arraybslash 100\%\\
\textsc{v.mo}(\textsc{dy}) & \raggedleft 18 & \raggedleft 22\% & \raggedleft 115 & \raggedleft 24\% & \raggedleft {}-{}-{}- & \raggedleft {}-{}-{}- & \raggedleft {}-{}-{}- & \raggedleft\arraybslash {}-{}-{}-\\
\textsc{v.bi} & \raggedleft 39 & \raggedleft 48\% & \raggedleft 327 & \raggedleft 68\% & \raggedleft {}-{}-{}- & \raggedleft {}-{}-{}- & \raggedleft {}-{}-{}- & \raggedleft\arraybslash {}-{}-{}-\\
Total & \raggedleft 81 & \raggedleft 100\% & \raggedleft 478 & \raggedleft 100\% & \raggedleft 16 & \raggedleft 100\% & \raggedleft 25 & \raggedleft\arraybslash 100\%\\
\lspbottomrule
\end{tabular}

In the corpus, \textitbf{kasi} ‘give’ is used most often with bivalent bases. Less often, \textitbf{kasi} ‘give’ occurs with monovalent bases, which can be agentive or non-agentive. Most monovalent bases are dynamic, whereas stative bases, which are mostly non-agentive, are much rarer. Most monovalent dynamic bases, in turn, are agentive, while non-agentive dynamic bases are rare. By contrast, \textitbf{biking} ‘make’ always takes monovalent bases which are typically stative and non-agentive. Causatives with monovalent non-agentive dynamic bases are also possible, although they are unattested in the corpus. Causatives with monovalent agentive bases are only possible if the causee is inanimate or animate but helpless. Table  ‎11 .5 shows these distributional patterns.


\begin{stylecaption}
\label{bkm:Ref350326318}Table ‎11.\stepcounter{Table}{\theTable}:  Properties of causative constructions
\end{stylecaption}

\begin{tabular}{llll}
\lsptoprule

 Base & Agentivity & \textitbf{kasi} ‘give’ & \arraybslash \textitbf{biking} ‘make’\\
\textsc{v.mo}(\textsc{st}) & \textsc{non-agt} & Less often & Most often\\
\textsc{v.mo}(\textsc{dy}) & \textsc{non-agt} & Rarely & Possible although unattested\\
\textsc{v.mo}(\textsc{dy}) & \textsc{agt} & Less often & Possible with inanimate or with animate but helpless causees although unattested\\
\textsc{v.bi} & \textsc{agt} & Most often & Unacceptable\\
\lspbottomrule
\end{tabular}
\section{Reciprocal clauses}
\label{bkm:Ref367459483}
Papuan Malay employs two different strategies to express reciprocal relations: a syntactic strategy with the dedicated reciprocity marker \textitbf{baku} ‘\textsc{recp}’, and a lexical strategy.



Generally speaking, reciprocal clauses describe situations “in which two participants equally act upon each other” {\citep[181]{Payne1997}}, with the two participants performing “two identical semantic roles” {\citep[6]{Nedjalkov2007}}. That is, in reciprocal clauses “two subevents are shown as one event or situation” by presenting two predications as one {(2007: 7)}.
\end{styleBodyvafter}


Cross-linguistically, four major strategies of encoding the notion of reciprocity structurally are distinguished, according to {Nedjalkov (2007: 9–16)}: syntactic, morphological, clitic, and lexical constructions.\footnote{\\
\\
\\
\\
\\
\\
\\
\\
\\
\\
\\
\\
\\
\\
\\
\par {\citet[10]{Nedjalkov2007}} groups syntactic, morphological, and clitic reciprocal constructions together as grammatical or derived reciprocals.} Syntactic reciprocals are formed with reciprocal pronouns or reciprocal adverbs. Morphological reciprocals are formed by means of affixation, reduplication, compounding, or periphrastic constructions involving an auxiliary.
\end{styleBodyvafter}


Papuan Malay syntactic reciprocals are discussed in §11.3.1, and lexical reciprocals are briefly mentioned in §11.3.2. Morphological and clitic reciprocal constructions are unattested.
\end{styleBodyvxvafter}

\subsection{Syntactic reciprocals}
\label{bkm:Ref366595040}
Papuan Malay forms syntactic reciprocals with the dedicated reciprocity marker \textitbf{baku} ‘\textsc{recp}’. A typical example is given in (0).


\begin{styleExampleTitle}
Papuan Malay reciprocity marker \textitbf{baku} ‘\textsc{recp}’
\end{styleExampleTitle}

\begin{tabular}{lllllll}
\lsptoprule
\label{bkm:Ref365463736}
\gll {kitong} {dua} {\bluebold{baku}} {\bluebold{melawang}} {gara-gara} {ikang}\\ %
& \textsc{1pl} & two & \textsc{recp} & oppose & because & fish\\
\lspbottomrule
\end{tabular}
\ea
\glt 
‘the two of us are \bluebold{fighting each other} because of the fish’ \textstyleExampleSource{[081109-011-JR.0008]}
\z


The corpus contains 101 reciprocal clauses formed with 42 different verbs. The vast majority are bivalent: 37 verbs (88\%) accounting for 95 tokens (94\%). One reciprocal clause is formed with trivalent \textitbf{ceritra} ‘tell’. The remaining four verbs are monovalent dynamic (accounting for five tokens) (for details see §11.3.1.1).



Structurally, Papuan Malay uses two different types of syntactic reciprocals: (1) a “simple reciprocal construction” (§11.3.1.1), and (2) a “discontinuous construction” (§11.3.1.2), using {Nedjalkov’s (2007: 27–30)} terminology. In simple reciprocals \textitbf{baku} ‘\textsc{recp}’ can receive a reciprocal or a sociative reading, while in discontinuous reciprocals the marker always receives a reciprocal reading.
\end{styleBodyvafter}


Cross-linguistically, the reciprocity marker is classified in different ways; in some languages it is classified as a pronoun or an adverb, in others as an affix or an auxiliary (see {Nedjalkov’s 2007: 9–16} above-mentioned distinction of syntactic and morphological reciprocals). As for the Papuan Malay reciprocity marker, this grammar analyzes \textitbf{baku} ‘\textsc{recp}’ as an independent word and not as an affix, without, however, further specifying its morphosyntactic status at this point. This analysis as a separate word is based on the fact that \textitbf{baku} ‘\textsc{recp}’ can be reduplicated, as shown in (0). Affixes, by contrast, are not reduplicated in Papuan Malay, as discussed in §4.1.
\end{styleBodyvvafter}

\begin{styleExampleTitle}
Reduplication of \textitbf{baku} ‘\textsc{recp}’
\end{styleExampleTitle}

\begin{tabular}{lllllllll}
\lsptoprule
\label{bkm:Ref376434631}
\gll {itu} {sampe} {tong} {\bluebold{baku{\Tilde}baku}} {tawar} {ini} {deng} {doseng}\\ %
& \textsc{d.dist} & until & \textsc{1pl} & \textsc{rdp}{\Tilde}\textsc{recp} & bargain & \textsc{d.prox} & with & lecturer\\
\lspbottomrule
\end{tabular}
\ea
\glt 
‘it got to the point that we and the lecturer were arguing \bluebold{constantly with each other}’ \textstyleExampleSource{[080917-010-CvEx.0177]}
\z


This analysis of \textitbf{baku} ‘\textsc{recp}’ as an independent word is also applied by {\citet[24]{Donohue2003}}, while other researchers such as {van \citet[324]{Velzen1995}} treat the reciprocity marker as a prefix. For most of the other eastern Malay varieties, the reciprocity marker is also treated as a prefix, namely for Ambon Malay {(van Minde 1997: 101–105)}, Banda Malay {\citep[250]{Paauw2009}}, Kupang Malay {\citep[46]{Steinhauer1983}}, Manado Malay {\citep[23]{Stoel2005}}, and North Moluccan / Ternate Malay ({Taylor 1983: 19; Voorhoeve 1983: 4}{;} {Litamahuputty 2012: 130–133}).
\end{styleBodyxvafter}

\paragraph[Simple reciprocal constructions]{Simple reciprocal constructions}
\label{bkm:Ref365529708}
Most reciprocal constructions in the corpus (86/101 – 85\%) are “simple reciprocals”. In such a construction, both participants are encoded as the clausal subject, which is called the “reciprocator”, following {Haspelmath’s (2007c: 2092)} terminology.\footnote{\\
\\
\\
\\
\\
\\
\\
\\
\\
\\
\\
\\
\\
\\
\\
\par {\citet[6]{Nedjalkov2007}} uses the term “reciprocant” rather than “reciprocator”.} Hence, the typical structure for simple reciprocals is ‘\textsc{reciprocator} \textitbf{baku} \textsc{v}’, as shown in (0) to (0). The reciprocator can be a coordinate noun phrase such as \textitbf{nona{\Tilde}nona ana laki{\Tilde}laki} ‘the girls (and) boys’ in (0), or a plural personal pronoun such as \textitbf{kamu} ‘\textsc{2pl}’ in (0).



In ‘\textsc{reciprocator} \textitbf{baku} \textsc{v}’ constructions, \textitbf{baku} ‘\textsc{recp}’ can receive a reciprocal reading in the sense of ‘\textsc{reciprocator} \textsc{v} each other’, or a sociative reading in the sense of ‘\textsc{reciprocator} \textsc{v} together’.
\end{styleBodyvafter}


‘\textsc{reciprocator} \textitbf{baku} \textsc{v}’ constructions with a reciprocal reading are characterized by a reduction in syntactic valency, which corresponds to the reduction in semantic valency: with both participants being encoded by the clausal subject, the object that typically encodes a second participant is deleted. This is shown in (0) to (0); reciprocals with a sociative reading are given in (0) to (0).
\end{styleBodyvafter}


Typically, the verbal base in a ‘\textsc{reciprocator} \textitbf{baku} \textsc{v}’ construction is bivalent (80/86 reciprocals – 93\%); the corpus also contains one reciprocal construction formed with trivalent \textitbf{ceritra} ‘tell’. Examples are given in (0) to (0). These examples show that the bases can have reciprocal/bidirectional semantics such as \textitbf{cium} ‘kiss’ in (0), or non-reciprocal/unidirectional semantics such as \textitbf{benci} ‘hate’ in (0). (Reciprocals with monovalent bases are presented in (0) and (0).)
\end{styleBodyvvafter}

\begin{styleExampleTitle}
‘\textsc{reciprocator} \textitbf{baku} \textsc{v}’ constructions with bivalent verbs: Reciprocal reading
\end{styleExampleTitle}

\begin{tabular}{lllllllllllll}
\lsptoprule
\label{bkm:Ref365463735}
\gll {\multicolumn{3}{l}{nona{\Tilde}nona,}} {\multicolumn{2}{l}{ana}} {\multicolumn{3}{l}{laki{\Tilde}laki}} {\bluebold{baku}} {\bluebold{pacar}} {di} {pinggir}\\ %
& \multicolumn{3}{l}{\textsc{rdp}{\Tilde}girl} & \multicolumn{2}{l}{child} & \multicolumn{3}{l}{\textsc{rdp}{\Tilde}husband} & \textsc{recp} & date/lover & at & edge\\
& skola & … & \multicolumn{2}{l}{\bluebold{baku}} & \multicolumn{2}{l}{\bluebold{cium}} & di & \multicolumn{5}{l}{pinggir{\Tilde}pinggir}\\
& school &  & \multicolumn{2}{l}{\textsc{recp}} & \multicolumn{2}{l}{kiss} & at & \multicolumn{5}{l}{\textsc{rdp}{\Tilde}edge}\\
\lspbottomrule
\end{tabular}
\ea
\glt 
‘the girls (and) boys are \bluebold{courting each other} at the edge of the school (grounds), … (they) are \bluebold{kissing each other} at the edges (of the school grounds)’ \textstyleExampleSource{[081115-001a-Cv.0017]}
\z

\begin{tabular}{llllllllll}
\lsptoprule
\label{bkm:Ref365463734}
\gll {kamu} {tida} {bole} {\bluebold{baku}} {\bluebold{benci},} {tida} {bole} {\bluebold{baku}} {\bluebold{mara}}\\ %
& \textsc{2pl} & \textsc{neg} & may & \textsc{recp} & hate & \textsc{neg} & may & \textsc{recp} & feel.angry(.about)\\
\lspbottomrule
\end{tabular}
\ea
\glt 
‘you must not \bluebold{hate each other}, (you) must not \bluebold{feel angry with each other}’ \textstyleExampleSource{[081115-001a-Cv.0271]}
\z

\begin{tabular}{lllllll}
\lsptoprule
\label{bkm:Ref350270691}
\gll {Markus} {deng} {Yan} {dong} {\bluebold{baku}} {\bluebold{ceritra}}\\ %
& Markus & with & Yan & \textsc{3sg} & \textsc{recp} & tell\\
\lspbottomrule
\end{tabular}
\ea
\glt 
‘they Markus and Yan were \bluebold{talking to each other}’ \textstyleExampleSource{[Elicited BR130601.001]}\footnote{\\
\\
\\
\\
\\
\\
\\
\\
\\
\\
\\
\\
\\
\\
\\
\par The corpus contains one reciprocal construction formed with trivalent \textitbf{ceritra} ‘tell’, similar to the elicited one in (0). For the most part, however, the original utterance it unclear, as the speaker mumbles.}
\z


‘\textsc{reciprocator} \textitbf{baku} \textsc{v}’ constructions with monovalent dynamic bases are also possible, but rare. Of the attested 86 simple reciprocals, only five are formed with monovalent verbs (6\%), namely with \textitbf{bertengkar} ‘quarrel’ (1 token), \textitbf{saing} ‘compete’ (1 token), \textitbf{tampil} ‘perform’ (2 tokens), and \textitbf{tanding} ‘compete’ (1 token) (none of the four verbs occur in discontinuous reciprocal constructions). Examples are given for \textitbf{saing} ‘compete’ in (0) and for \textitbf{tanding} ‘compete’ in (0).


\begin{styleExampleTitle}
‘\textsc{reciprocator} \textitbf{baku} \textsc{v}’ constructions with monovalent dynamic verbs: Reciprocal reading
\end{styleExampleTitle}

\begin{tabular}{llll}
\lsptoprule
\label{bkm:Ref365463738}
\gll {ade-kaka} {\bluebold{baku}} {\bluebold{saing}}\\ %
& ySb-oSb &  & \\
& siblings & \textsc{recp} & compete\\
\lspbottomrule
\end{tabular}
\ea
\glt 
‘the siblings were \bluebold{competing with each other}’ \textstyleExampleSource{[080919-006-CvNP.0001]}
\z

\begin{tabular}{lllllllll}
\lsptoprule
\label{bkm:Ref365463737}
\gll {dong} {ada} {brapa} {orang} {itu} {\bluebold{baku}} {\bluebold{tanding}} {rekam}\\ %
& \textsc{3pl} & exist & several & person & \textsc{d.dist} & \textsc{recp} & compete & record\\
\lspbottomrule
\end{tabular}
\ea
\glt 
‘they were (indeed) several people (who) were \bluebold{competing with each other} to record (their songs)’ \textstyleExampleSource{[080923-016-CvNP.0006]}
\z


Most of the verbs used in reciprocal clauses in the corpus also occur in non-reciprocal transitive clauses (38/42 verbs). This is illustrated with \textitbf{gendong} ‘hold’ in (0). The remaining four verbs are only used in reciprocal constructions: bivalent \textitbf{ancam} ‘threaten’ (1 token) and \textitbf{cium} ‘kiss’ (2 tokens), and monovalent \textitbf{bertengkar} ‘quarrel’ (1 token) and \textitbf{tanding} ‘compete’ (1 token). Whether these verbs can also occur in non-reciprocal transitive clauses requires further investigation.


\begin{styleExampleTitle}
Reciprocal and non-reciprocal uses of verbs
\end{styleExampleTitle}

\begin{tabular}{llllllllllllllll}
\lsptoprule
\label{bkm:Ref366589257}
\gll {\multicolumn{2}{l}{Nofela}} {\multicolumn{2}{l}{\bluebold{gendong}}} {\multicolumn{3}{l}{\bluebold{bapa}}} {ato} {\multicolumn{2}{l}{bapa}} {\multicolumn{2}{l}{yang}} {\multicolumn{2}{l}{\bluebold{gendong}}} {\bluebold{Nofela}}\\ %
& \multicolumn{2}{l}{Nofela} & \multicolumn{2}{l}{hold} & \multicolumn{3}{l}{father} & or & \multicolumn{2}{l}{father} & \multicolumn{2}{l}{\textsc{rel}} & \multicolumn{2}{l}{hold} & Nofela\\
& \bluebold{deng} & \multicolumn{2}{l}{\bluebold{Siduas}} & \multicolumn{2}{l}{ka} & … & \multicolumn{3}{l}{kitong} & \multicolumn{2}{l}{\bluebold{baku}} & \multicolumn{2}{l}{\bluebold{gendong}} & \multicolumn{2}{l}{to?}\\
& with & \multicolumn{2}{l}{Siduas} & \multicolumn{2}{l}{or} &  & \multicolumn{3}{l}{\textsc{1pl}} & \multicolumn{2}{l}{\textsc{recp}} & \multicolumn{2}{l}{hold} & \multicolumn{2}{l}{right?}\\
\lspbottomrule
\end{tabular}
\ea
\glt 
[During a phone conversation between a father and his children:] ‘you (‘Nofela’) will \bluebold{hold me (‘father’)} or I (‘father’) will \bluebold{hold you (‘Nofela’) and Siduas} … we’ll \bluebold{hold each other}, right?’ \textstyleExampleSource{[080922-001a-CvPh.0687/0695]}
\z


In the simple reciprocals presented so far, \textitbf{baku} ‘\textsc{recp}’ denotes reciprocal relations. Alternatively, though, ‘\textsc{reciprocator} \textitbf{baku} \textsc{v}’ clauses can signal sociative relations in the sense of ‘\textsc{reciprocator} \textsc{v} together’.



Generally speaking, the “sociative meaning (also called associative, collective, cooperative, etc.) suggests that an action is performed jointly and simultaneously by a group of people (at least two) named by the subject […] and engaged in the same activity”, as {\citet[33]{Nedjalkov2007}} notes in his typology of reciprocal constructions. Reciprocals with a sociative reading are characterized by valency retention, in that “the number of the participants increases without changing the syntactic structure” {(2007: 22)}.
\end{styleBodyvafter}


This observation also applies to ‘\textsc{reciprocator} \textitbf{baku} \textsc{v}’ constructions, as shown in (0) and (0). That is, reciprocal clauses with a sociative reading are characterized by valency retention, although the number of participants increases.
\end{styleBodyvvafter}

\begin{styleExampleTitle}
‘\textsc{reciprocator} \textitbf{baku} \textsc{v}’ constructions: Sociative reading
\end{styleExampleTitle}

\begin{tabular}{llllll}
\lsptoprule
\label{bkm:Ref365475550}
\gll {baru} {kitong} {mulay} {\bluebold{baku}} {\bluebold{ojek}}\\ %
& and.then & \textsc{1pl} & start & \textsc{recp} & take.motorbike.taxi\\
\lspbottomrule
\end{tabular}
\ea
\glt 
‘and then we\bluebold{ }started \bluebold{taking motorbike taxis together}’ \textstyleExampleSource{[081002-001-CvNP.0004]}
\z

\begin{tabular}{lllllll}
\lsptoprule
(\stepcounter{}{\the}) & kitong & mo & \bluebold{baku} & \bluebold{bagi} & \bluebold{swara} & bagemana\\
& \textsc{1pl} & want & \textsc{recp} & divide & voice & how\\
\lspbottomrule
\end{tabular}
\ea
\glt 
[About upcoming local elections:] ‘how do we want to \bluebold{share the votes together}?’ \textstyleExampleSource{[080919-001-Cv.0165]}
\z

\begin{tabular}{lllllllll}
\lsptoprule
\label{bkm:Ref365464036}
\gll {Aksamina} {deng} {Klara} {dong} {dua} {\bluebold{baku}} {\bluebold{rampas}} {\bluebold{bola}}\\ %
& Aksamina & with & Klara & \textsc{3pl} & two & \textsc{recp} & seize & ball\\
\lspbottomrule
\end{tabular}
\ea
\glt 
‘both Aksamina and Klara \bluebold{tackled the ball together}’ \textstyleExampleSource{[081006-014-Cv.0007]}
\z


Overall, the corpus contains only few ‘\textsc{reciprocator} \textitbf{baku} \textsc{v}’ constructions with a sociative reading. Further research is needed to determine whether there are any formal criteria that allow ‘\textsc{reciprocator} \textitbf{baku} \textsc{v}’ constructions with a reciprocal reading to be distinguished from those with a sociative reading.
\end{styleBodyxvafter}

\paragraph[Discontinuous reciprocal constructions]{Discontinuous reciprocal constructions}
\label{bkm:Ref365529710}
In discontinuous reciprocal constructions, cross-linguistically, only one of the participants is expressed as the subject, while the second participant “is a comitative phrase”, as {\citet[29]{Nedjalkov2007}} points out. Given that the second participant is not encoded as the direct object but as a prepositional phrase, discontinuous reciprocals result in a reduction in syntactic valency. Hence, pragmatically and syntactically, the second, non-subject participant is “a constituent of lower […] status” {(2007: 28)}; semantically, however, it is of the same status as the subject reciprocator.



These observations also apply to Papuan Malay. In discontinuous reciprocal constructions, the second participant, or “reciprocee”, adopting {Haspelmath’s (2007c: 2092)} {terminology, }is encoded by a prepositional phrase. This prepositional phrase is introduced with the comitative preposition \textitbf{dengang} ‘with’, with its short form \textitbf{deng} (see also §10.2.1).\footnote{\\
\\
\\
\\
\\
\\
\\
\\
\\
\\
\\
\\
\\
\\
\\
\par {\citet[8]{Nedjalkov2007}} refers to non-subject reciprocants as “co-participants”.} Hence, the structure for discontinuous reciprocals is ‘\textsc{reciprocator} \textitbf{baku} \textsc{v} \textitbf{dengang} \textsc{reciprocee}’.
\end{styleBodyvafter}


In the corpus, however, discontinuous constructions occur much less often than simple ones; only 15 of the 101 reciprocals are discontinuous (15\%). All of them designate reciprocal relations in the sense of ‘\textsc{reciprocator} \textsc{v} with \textsc{reciprocee}’, literally ‘\textsc{reciprocator} \textsc{v} each other with \textsc{reciprocee}’. Unlike the simple reciprocals in §11.3.1.1, discontinuous constructions do not express sociative relations.
\end{styleBodyvafter}


In most of the discontinuous reciprocals (10/15 – 67\%), the reciprocee is mentioned overtly, as in (0) to (0). (For discontinuous constructions with omitted reciprocee see the examples in (0) and (0).)
\end{styleBodyvvafter}

\begin{styleExampleTitle}
‘\textsc{reciprocator} \textitbf{baku} \textsc{v} \textitbf{deng(ang)} \textsc{reciprocee}’ constructions
\end{styleExampleTitle}

\begin{tabular}{lllllllll}
\lsptoprule
\label{bkm:Ref365464756}
\gll {…} {ko} {laki{\Tilde}laki} {bisa} {\bluebold{baku}} {\bluebold{dapat}} {\bluebold{deng}} {bapa}\\ %
&  & \textsc{2sg} & \textsc{rdp}{\Tilde}husband & be.able & \textsc{recp} & get & with & father\\
\lspbottomrule
\end{tabular}
\ea
\glt 
‘[I thought,] you, a man, can \bluebold{meet with} me (‘father’)’ (Lit. ‘can \bluebold{meet each other with} father’) \textstyleExampleSource{[080922-001a-CvPh.0234]}
\z

\begin{tabular}{lllllllll}
\lsptoprule
(\stepcounter{}{\the}) & sa & tida & perna & \bluebold{baku} & \bluebold{mara} & \bluebold{deng} & orang & laing\\
& \textsc{1sg} & \textsc{neg} & ever & \textsc{recp} & feel.angry(.about) & with & person & be.different\\
\lspbottomrule
\end{tabular}
\ea
\glt 
‘I never \bluebold{get angry with} other people’ (Lit. ‘\bluebold{feel angry about each other with} another person’) \textstyleExampleSource{[081110-008-CvNP.0067]}
\z

\begin{tabular}{lllllll}
\lsptoprule
\label{bkm:Ref365464758}
\gll {…} {de} {\bluebold{baku}} {\bluebold{tabrak}} {\bluebold{deng}} {Sarles}\\ %
&  & \textsc{3sg} & \textsc{recp} & hit.against & with & Sarles\\
\lspbottomrule
\end{tabular}
\ea
\glt 
‘[right then Sarles was standing by door,] he/she (the evil spirit) \bluebold{collided with} Sarles’ (Lit. ‘\bluebold{hit against each other with} Sarles’) \textstyleExampleSource{[081025-009b-Cv.0026]}
\z


Given the lower pragmatic status of the reciprocee, it can also remain “unspecified” {\citep[42]{Nedjalkov2007}}, as in (0) and (0). This applies to five of the 15 discontinuous constructions in the corpus (33\%). That is, if the second participant is understood from the context, or considered irrelevant, it can be omitted together with its preposition. In (0), the omitted reciprocee \textitbf{orang} ‘person’ was mentioned earlier. In (0), the omitted reciprocee ‘community’ is understood from the context, as the topic of the narrative is communal life in the village.


\begin{styleExampleTitle}
‘\textsc{reciprocator} \textitbf{baku} \textsc{v} Ø’ constructions
\end{styleExampleTitle}

\begin{tabular}{lllllllll}
\lsptoprule
\label{bkm:Ref365464761}
\gll {saya} {kalo} {macang} {\bluebold{baku}} {\bluebold{pukul}} {\bluebold{Ø}} {rasa} {takut}\\ %
& \textsc{1sg} & if & variety & \textsc{recp} & hit &  & feel & feel.afraid(.of)\\
\lspbottomrule
\end{tabular}
\ea
\glt 
‘(as for) me, when (I) kind of \bluebold{fight (with another person)}, I feel afraid’ (Lit. ‘\bluebold{hit each other}’) \textstyleExampleSource{[081110-008-CvNP.0066]}
\z

\begin{tabular}{llllllllll}
\lsptoprule
\label{bkm:Ref365464759}
\gll {…} {dia} {dapat} {babi,} {de} {biasa} {\bluebold{baku}} {\bluebold{bagi}} {\bluebold{Ø}}\\ %
&  & \textsc{3sg} & get & pig & \textsc{3sg} & be.usual & \textsc{recp} & divide & \\
\lspbottomrule
\end{tabular}
\ea
\glt
[How to be a good villager:] ‘[when he catches fish,] (when) he catches a pig, he usually \bluebold{shares} (it \bluebold{with the community)}’ (Lit. ‘\bluebold{divide each other}’) \textstyleExampleSource{[080919-004-NP.0063]}
\end{styleFreeTranslEngxvpt}

\subsection{Lexical reciprocals}
\label{bkm:Ref366595042}
Lexical reciprocals are, generally speaking, “words with an inherent reciprocal meaning” {\citep[14]{Nedjalkov2007}}. Therefore, they do not need to be marked with a reciprocity marker.



Papuan Malay lexical reciprocal are presented in (0) to (0). All three examples denote, what \citet[102]{Kemmer1993}{ calls}, “naturally reciprocal events”, such as \textitbf{ketemu} ‘meet’ in (0), \textitbf{nika} ‘marry’ in (0), or \textitbf{cocok} ‘be suitable’ in (0).
\end{styleBodyvxafter}

\begin{tabular}{llllll}
\lsptoprule
\label{bkm:Ref366516206}
\gll {sa} {\bluebold{ketemu}} {de} {di} {kampus}\\ %
& \textsc{1sg} & meet & \textsc{3sg} & at & campus\\
\lspbottomrule
\end{tabular}
\ea
\glt 
‘I \bluebold{met} him on the (university) campus’ \textstyleExampleSource{[080922-003-Cv.0102]}
\z

\begin{tabular}{llll}
\lsptoprule
\label{bkm:Ref366516207}
\gll {dorang} {dua} {\bluebold{nika}}\\ %
& \textsc{3pl} & two & marry.officially\\
\lspbottomrule
\end{tabular}
\ea
\glt 
‘the two of them \bluebold{married}’ \textstyleExampleSource{[081110-005-CvPr.0095]}
\z

\begin{tabular}{llll}
\lsptoprule
\label{bkm:Ref366516209}
\gll {kam} {dua} {\bluebold{cocok}}\\ %
& \textsc{2pl} & two & be.suitable\\
\lspbottomrule
\end{tabular}
\ea
\glt
‘the two of you \bluebold{match}’ \textstyleExampleSource{[080922-004-Cv.0033]}
\end{styleFreeTranslEngxvpt}

\subsection{Summary}

In Papuan Malay, the dedicated reciprocity marker \textitbf{baku} ‘\textsc{recp}’ signals reciprocity. In reciprocity clauses two predications are presented with the two subjects of each predication equally acting upon each other. The main focus of this description is syntactic reciprocal constructions; lexical reciprocals were mentioned only briefly. Two types of reciprocal constructions are attested, simple and discontinuous ones.



In simple reciprocals, both participants are encoded by the clausal subject. The base is most often a bivalent verb, although reciprocals with monovalent verbs are also attested. Usually, these clauses are the result of a valency-reducing operation and receive the reciprocal reading ‘\textsc{reciprocator} \textsc{v} each other’. Alternatively, these constructions can receive a sociative reading in which case the reciprocal clause is characterized by valency retention. Further investigation is needed to determine whether there are formal criteria to distinguish the reciprocal from the sociative readings. The basic scheme for simple reciprocals is given in (0).
\end{styleBodyvvafter}

\begin{styleExampleTitle}
Scheme for simple reciprocals
\end{styleExampleTitle}

\begin{tabular}{llll}
\lsptoprule
\label{bkm:Ref371510530}
\gll {\textsc{reciprocator}} {\textitbf{baku}} {\textsc{V}}\\ %
\lspbottomrule
\end{tabular}

In discontinuous reciprocals, one participant is encoded by the clausal subject while the second one, the \textsc{reciprocee}, is expressed in a prepositional phrase introduced with comitative \textitbf{dengang} ‘with’. This type of reciprocal also results from a valency-reducing operation and receives the reading ‘\textsc{reciprocator} \textsc{v} with \textsc{reciprocee}’. The second participant can also be omitted if it is understood from the context. The basic scheme for discontinuous reciprocals is given in (0).


\begin{styleExampleTitle}
Scheme for discontinuous reciprocals
\end{styleExampleTitle}

\begin{tabular}{llllll}
\lsptoprule
\label{bkm:Ref371510532}
\gll {\textsc{reciprocator}} {\textitbf{baku}} {\textsc{V}} {(\textitbf{dengang}} {\textsc{reciprocee})}\\ %
\lspbottomrule
\end{tabular}
\section{Existential clauses}
\label{bkm:Ref272582710}
In Papuan Malay, existential clauses are formed with the existential verb \textitbf{ada} ‘exist’. Structurally, two types of existential clauses can be distinguished: (1) intransitive clauses with one core argument and (2) transitive clauses with two core arguments.



In one-argument clauses, \textitbf{ada} ‘exist’ precedes or follows the theme expression depending on the theme’s definiteness. This clause type asserts the existence of an entity, expresses its availability, or, with definite themes, denotes possession. In two-argument clauses, \textitbf{ada} ‘links’ the subject with the direct object. This clause type signals possession of an indefinite possessum. One-argument clauses are described in §11.4.1 and two-argument clauses in §11.4.2; §11.4.3 summarizes the main points of this section. (Negation of existential clauses is discussed in §13.1.1.2.)
\end{styleBodyvxvafter}

\subsection{One-argument existential clauses}
\label{bkm:Ref263510864}
In one-argument existential clauses, \textitbf{ada} ‘exist’ precedes or follows the subject, or theme expression, such that ‘S \textitbf{ada}’ or ‘\textitbf{ada} S’. These differences in word order serve to distinguish nonidentifiable themes from identifiable ones, as shown with the near contrastive examples in (0) and (0). When the theme is pragmatically indefinite or nonidentifiable, \textitbf{ada} ‘exist’ precedes it, such that ‘\textitbf{ada} S’, as in (0). When the theme is definite or identifiable, \textitbf{ada} ‘exist’ follows it, such that ‘S \textitbf{ada}’, as in (0).


\begin{styleExampleTitle}
One-argument existential clauses: ‘\textitbf{ada} S’ versus ‘S \textitbf{ada}’ word order
\end{styleExampleTitle}

\begin{tabular}{lllll}
\lsptoprule
\label{bkm:Ref339731725}
\gll {ke} {mari,} {ada} {\bluebold{nasi}}\\ %
& to & hither & exist & cooked.rice\\
\lspbottomrule
\end{tabular}
\ea
\glt 
‘(come) here, there’s \bluebold{cooked rice}’ \textstyleExampleSource{[081006-035-CvEx.0052]}
\z

\begin{tabular}{llllll}
\lsptoprule
\label{bkm:Ref339731726}
\gll {\bluebold{nasi}} {ada} {itu,} {timba} {suda!}\\ %
& cooked.rice & exist & \textsc{d.dist} & spoon & already\\
\lspbottomrule
\end{tabular}
\ea
\glt 
‘\bluebold{the cooked rice} is there, just spoon (it)!’ \textstyleExampleSource{[081110-002-Cv.0051]}
\z


In existential clauses with indefinite or nonidentifiable themes, fronted \textitbf{ada} ‘exist’ has two functions, as shown in (0) and (0). One is to convey the existence of an entity, such that ‘a \textsc{theme} exists’, as in (0), where \textitbf{ada} ‘exist’ signals the existence of \textitbf{babi} ‘pig’. A second function is to signal availability in the sense of ‘a \textsc{theme} is available’, as in (0), where \textitbf{ada} ‘exist’ asserts the availability of \textitbf{kuskus} ‘cuscus’ and other game; see also the example in (0).


\begin{styleExampleTitle}
‘\textitbf{ada} S’ word order: Existence or availability of an indefinite/nonidentifiable theme
\end{styleExampleTitle}

\begin{tabular}{lllll}
\lsptoprule
\label{bkm:Ref339731720}
\gll {ada} {\bluebold{babi}} {di} {situ}\\ %
& exist & pig & at & \textsc{l.med}\\
\lspbottomrule
\end{tabular}
\ea
\glt 
‘there is \bluebold{a pig} there’ \textstyleExampleSource{[081006-023-CvEx.0004]}
\z

\begin{tabular}{llllllllllllll}
\lsptoprule
\label{bkm:Ref342316742}
\gll {maytua} {liat,} {\multicolumn{2}{l}{wa,}} {\multicolumn{2}{l}{kantong}} {\multicolumn{3}{l}{itu}} {fol,} {ada} {\bluebold{kuskus},} {ada}\\ %
& wife & see & \multicolumn{2}{l}{wow!} & \multicolumn{2}{l}{bag} & \multicolumn{3}{l}{\textsc{d.dist}} & be.full & exist & cuscus & exist\\
& \multicolumn{3}{l}{\bluebold{tikus-tana},} & \multicolumn{2}{l}{ada} & \multicolumn{2}{l}{\bluebold{kepiting}} & e, & \multicolumn{2}{l}{\bluebold{ketang},} & ada & \multicolumn{2}{l}{\bluebold{ikang}}\\
& \multicolumn{3}{l}{spiny.bandicoot} & \multicolumn{2}{l}{exist} & \multicolumn{2}{l}{crab} & uh & \multicolumn{2}{l}{crab} & exist & \multicolumn{2}{l}{fish}\\
\lspbottomrule
\end{tabular}
\ea
\glt 
[After a successful hunt:] ‘(my) wife saw, ‘wow!, that bag is full’, there was \bluebold{cuscus}, there were \bluebold{bandicoots}, there were \bluebold{crabs}, uh, \bluebold{crabs}, there were \bluebold{fish}’ \textstyleExampleSource{[080919-004-NP.0031]}
\z


In existential clauses with definite or identifiable themes, post-posed \textitbf{ada} ‘exist’ also has two functions, as demonstrated in (0) and (0). One function is to assert the existence of an already established theme, such that ‘the \textsc{theme} exists’. This is the case in the elicited example in (0), which contrasts with the existential clause in (0). This reading also applies to the examples in (0) and in (0).


\begin{styleExampleTitle}
‘S \textitbf{ada}’ word order: Existence of a definite/identifiable theme
\end{styleExampleTitle}

\begin{tabular}{lllll}
\lsptoprule
\label{bkm:Ref371429994}
\gll {\bluebold{babi}} {ada} {di} {situ}\\ %
& pig & exist & at & \textsc{l.med}\\
\lspbottomrule
\end{tabular}
\ea
\glt 
‘\bluebold{the pig} is there’ \textstyleExampleSource{[Elicited MY131105.004]}
\z

\begin{tabular}{lll}
\lsptoprule
\label{bkm:Ref339731741}
\gll {\bluebold{saya}} {ada}\\ %
& 1\textsc{sg} & exist\\
\lspbottomrule
\end{tabular}
\ea
\glt 
[About a motorbike accident:] ‘\bluebold{I} am alive’ \textstyleExampleSource{[081015-005-NP.0024]}
\z


A second function of post-posed \textitbf{ada} ‘exist’ is to designate possession of a definite or identifiable possessum, as shown in (0) and (0). To convey the notion of possession the theme is expressed in an adnominal possessive construction, such that ‘\textsc{possessive} \textsc{np} \textsc{exist}s’ or ‘\textsc{possessor} has the \textsc{possessum}’. The clause in (0) asserts the known existence of \textitbf{bapa pu motor} ‘father’s motorbike’. In this adnominal possessive construction, the possessor noun phrase \textitbf{bapa} ‘father’ modifies the identifiable possessum noun phrase \textitbf{motor} ‘motorbike’; both constituents are linked with the possessive marker \textitbf{pu} ‘\textsc{poss}’. The same applies to the clause in (0) which signals possession of the definite possessum noun phrase \textitbf{dana} ‘funds’. (Adnominal possessive relations are discussed in detail in Chapter 9. Possession of an indefinite possessum is expressed with a two-argument existential clause or with a nominal clause, as described in §11.4.2 and §12.2, respectively.)


\begin{styleExampleTitle}
‘S \textitbf{ada}’ word order: Possession of a definite/identifiable theme
\end{styleExampleTitle}

\begin{tabular}{lllll}
\lsptoprule
\label{bkm:Ref339731742}
\gll {\bluebold{bapa}} {\bluebold{pu}} {\bluebold{motor}} {ada}\\ %
& father & \textsc{poss} & motorbike & exist\\
\lspbottomrule
\end{tabular}
\ea
\glt 
[Reply to a question:] ‘\bluebold{father} had \bluebold{a motorbike}’ (Lit. ‘\bluebold{father’s motorbike} exists’) \textstyleExampleSource{[080919-002-Cv.0012]}
\z

\begin{tabular}{lllllllllll}
\lsptoprule
\label{bkm:Ref339731743}
\gll {kalo} {\bluebold{sa}} {\bluebold{pu}} {\bluebold{dana}} {suda} {ada} {brarti} {sa} {undang} {…}\\ %
& if & 1\textsc{sg} & \textsc{poss} & fund & already & exist & mean & \textsc{1sg} & invite & \\
\lspbottomrule
\end{tabular}
\ea
\glt 
[About a planned meeting:] ‘if \bluebold{I} already had \bluebold{the funds}, that means, I would invite …’ (Lit. ‘\bluebold{my funds} already exist’) \textstyleExampleSource{[081010-001-Cv.0131]}
\z


If the theme can be inferred from the context it can also be omitted as in (0). In this example, the omitted theme is \textitbf{bagiang dana} ‘funding department’. Having been presented in the previous clause, it is now omitted, which leaves \textitbf{ada} ‘exist’ as the sole constituent of the existential clause.


\begin{styleExampleTitle}
Omitted theme expression
\end{styleExampleTitle}

\begin{tabular}{lllllll}
\lsptoprule
\label{bkm:Ref439954950}\label{bkm:Ref339731747}
\gll {\bluebold{Ø}} {\bluebold{ada},} {de} {punya} {dana} {sendiri}\\ %
&  & exist & 3\textsc{sg} & have & fund & be.alone\\
\lspbottomrule
\end{tabular}
\ea
\glt 
‘\bluebold{(the funding department) exists}, it has its own funding’ \textstyleExampleSource{[081010-001-Cv.0174]}
\z


Definite or identifiable existential clauses also co-occur with prepositional phrases, such as the locational phrase \textitbf{di situ} ‘there’ in (0). This clause can be analyzed in two ways. One analysis is that of an existential clause with a locational adjunct which gives additional information about the theme’s current location. This analysis is substantiated by the contrastive example in (0), in which \textitbf{situ} ‘\textsc{l.med}’ is fronted to the clause-initial position. This possibility of fronting the prepositional phrase is typical for adjuncts. In (0) the fronting serves to emphasize the location (concerning the rather common elision of locative \textitbf{di} ‘at’, see §10.1.5). An alternative analysis of (0) is that of a prepositional predicate clause with progressive reading. This analysis is substantiated with the (near) contrastive examples in (0) to (0). The example in (0) presents a nonverbal clause in which \textitbf{di situ} ‘there’ serves as the predicate. The example in (0) shows how a prepositional predicate clause can undergo aspectual modification, as for instance with the prospective adverb \textitbf{masi} ‘still’. The example in (0) shows the progressive-marking function of existential \textitbf{ada} ‘exist’ in verbal clauses (see also §5.4.1). When presented with both analyses, however, one of the consultants rejected the first analysis. Instead this consultant maintained that \textitbf{ada} ‘exist’ in (0) has the same function as \textitbf{masi} ‘still’ in (0), namely to modify the prepositional predicate \textitbf{di situ} ‘there’. The two analyses and the reading chosen by one of the consultants for the clauses in (0) require further investigation.


\begin{styleExampleTitle}
Alternative readings of clauses with definite/identifiable themes and post-posed prepositional phrases
\end{styleExampleTitle}

\begin{tabular}{lllllllll}
\lsptoprule
\label{bkm:Ref339731759}
\gll {de} {\bluebold{ada}} {di} {situ,} {Martina} {\bluebold{ada}} {di} {situ}\\ %
& \textsc{3sg} & exist & at & \textsc{l.med} & Martina & exist & at & \textsc{l.med}\\
\lspbottomrule
\end{tabular}
\ea
\glt 
‘she \bluebold{was (being)} there, Martina \bluebold{was (being)} there’ \textstyleExampleSource{[081109-001-Cv.0087]}
\z

\begin{tabular}{llllllllllllll}
\lsptoprule
\label{bkm:Ref371438793}
\gll {…} {\multicolumn{2}{l}{pace}} {\multicolumn{2}{l}{de}} {tulis} {\multicolumn{2}{l}{di}} {kertas,} {suda,} {situ} {de} {\bluebold{ada},}\\ %
&  & \multicolumn{2}{l}{man} & \multicolumn{2}{l}{\textsc{3sg}} & write & \multicolumn{2}{l}{at} & paper & already & \textsc{l.med} & \textsc{3sg} & exist\\
& \multicolumn{2}{l}{de} & \multicolumn{2}{l}{su} & \multicolumn{3}{l}{biking} & \multicolumn{6}{l}{daftar}\\
& \multicolumn{2}{l}{\textsc{3sg}} & \multicolumn{2}{l}{already} & \multicolumn{3}{l}{make} & \multicolumn{6}{l}{list}\\
\lspbottomrule
\end{tabular}
\ea
\glt 
[Enrolling for a sports team:] ‘[Herman gave his name,] the man wrote (it) on a paper, that’s it, there it \bluebold{was}!, he (the man) had already made a list’ \textstyleExampleSource{[081023-001-Cv.0001]}
\z

\begin{tabular}{llll}
\lsptoprule
\label{bkm:Ref339731757}
\gll {de} {\bluebold{di}} {\bluebold{situ}}\\ %
& \textsc{3sg} & at & \textsc{l.med}\\
\lspbottomrule
\end{tabular}
\ea
\glt 
‘he \bluebold{(was) there}’ \textstyleExampleSource{[080922-010a-CvNF.0256]}
\z

\begin{tabular}{lllll}
\lsptoprule
\label{bkm:Ref371441825}
\gll {de} {\bluebold{masi}} {di} {situ}\\ %
& \textsc{3sg} & still & at & \textsc{l.med}\\
\lspbottomrule
\end{tabular}
\ea
\glt 
‘he (was) \bluebold{still} there’ \textstyleExampleSource{[Elicited MY131105.002]}
\z

\begin{tabular}{llllll}
\lsptoprule
\label{bkm:Ref371438794}
\gll {de} {\bluebold{ada}} {tidor} {di} {situ}\\ %
& \textsc{3sg} & exist & sleep & at & \textsc{l.med}\\
\lspbottomrule
\end{tabular}
\ea
\glt
‘he \bluebold{is} sleep\bluebold{ing} there’ \textstyleExampleSource{[Elicited MY131105.003]}
\end{styleFreeTranslEngxvpt}

\subsection{Two-argument existential clauses}
\label{bkm:Ref263510867}
In two-argument existential clauses, \textitbf{ada} ‘exist’ links both core arguments. This type of existential clause expresses possession of an indefinite possessum. As shown in (0) and (0), the possessor noun phrase takes the subject slot and the possessum noun phrase takes the direct object slot, such that ‘\textsc{possessor} \textsc{exist}s \textsc{possessum}’ or ‘\textsc{possessor} has a \textsc{possessum}’. In (0) \textitbf{ada} ‘exist’ links the possessor \textitbf{sa} ‘\textsc{1sg}’ with the possessum \textitbf{ana} ‘child’ which gives the possessive reading ‘I have children’. The possessum can be encoded by a bare noun as in (0), or by a noun phrase such as \textitbf{dia punya jing} ‘her genies’ in (0). (Alternatively, possession of an indefinite possessum can be expressed with a nominal predicate; for details see §12.2. Possession of a definite possessum is encoded by an adnominal possessive construction; for details see Chapter 9, and also §11.4.1.)
\end{styleBodyxafter}

\begin{tabular}{lllllllllll}
\lsptoprule
\label{bkm:Ref375582683}
\gll {sa} {\bluebold{ada}} {\bluebold{ana},} {jadi} {sa} {kasi} {untuk} {sa} {pu} {sodara}\\ %
& 1\textsc{sg} & exist & child & so & \textsc{1sg} & give & for & \textsc{1sg} & \textsc{poss} & sibling\\
\lspbottomrule
\end{tabular}
\ea
\glt 
‘I \bluebold{have children}, so I gave (one) to my relative’ \textstyleExampleSource{[081006-024-CvEx.0010]}
\z

\begin{tabular}{llllllll}
\lsptoprule
\label{bkm:Ref339731767}
\gll {prempuang} {iblis} {itu} {\bluebold{ada}} {\bluebold{dia}} {\bluebold{punya}} {\bluebold{jing}}\\ %
& woman & devil & \textsc{d.dist} & exist & 3\textsc{sg} & \textsc{poss} & genie\\
\lspbottomrule
\end{tabular}
\ea
\glt 
[About evil spirits taking on the form of women:] ‘that woman spirit \bluebold{has her (own) genies}’ \textstyleExampleSource{[081006-022-CvEx.0053]}
\z


Cross-linguistically, {\citet{Stassen2011b}} identifies five major types of predicate possession: Have-Possessive, Oblique Possessive, Genitive Possessive, Topic Possessive, and Conjunctional Possessive. In terms of this classification, the existential possessive constructions in (0) and (0) are best explained as Topic Possessives.\footnote{\\
\\
\\
\\
\\
\\
\\
\\
\\
\\
\\
\\
\\
\\
\\
\par As for the remaining four types of possessive constructions, the data in the corpus indicate the following: (1) the Have-Possessive is formed with the ditransitive verb \textitbf{punya} ‘have’, as in (0) in §9.1 (p. \pageref{bkm:Ref439954837}), and the Genitive Possessive is used to encode possessive relations in which the possessum has a definite reading, as in (0) and (0) in §11.4.1 (p. \pageref{bkm:Ref439954950}) (see also Chapter 9). The Oblique and Conjunctional Possessives are unattested.} According to {\citet[219]{Stassen2009}},


\begin{styleIvI}
[in] a standard Topic Possessive, the possessee is the subject of the be-verb. […] The possessor is constructed as a sentential topic and may or may not be marked as such, for example by sentence-initial position …
\end{styleIvI}


Following this analysis, an alternative translation for the possessive construction \textitbf{sa ada ana} ‘I have children’ in (0) would be: ‘(as for) me, children exist’.
\end{styleBodyxvafter}

\subsection{Summary}
\label{bkm:Ref263531077}
In Papuan Malay, existential clauses are formed with the existential verb \textitbf{ada} ‘exist’. Syntactically, two clause types can be distinguished: intransitive clauses with one core argument, and transitive clauses with two core arguments. Table  ‎11 .6 gives an overview of the different constructions and their functions, with one-argument clauses given in (1) and two-argument clauses in (2).



In one-argument clauses, \textitbf{ada} ‘exist’ precedes the theme expression when this is pragmatically indefinite or nonidentifiable, as in (1a). This construction conveys the existence or availability of an entity. When the theme is definite or identifiable, \textitbf{ada} ‘exist’ follows it, as in (1b). This construction asserts the existence of an already established theme or denotes possession of a definite/identifiable possessum. In two-argument clauses, \textitbf{ada} ‘exist’ links the subject and direct object arguments. This type of existential clause indicates possession of an indefinite possessum, as in (2).
\end{styleBodyvvafter}

\begin{stylecaption}
\label{bkm:Ref269475327}Table ‎11.\stepcounter{Table}{\theTable}:  Overview of existential clause constructions
\end{stylecaption}

\begin{tabular}{llllll}
\lsptoprule

\multicolumn{6}{l}{%\setcounter{itemize}{0}
\begin{itemize}
\item \label{bkm:Ref371512899}One-argument existential clauses\end{itemize}
}\\
& \multicolumn{5}{l}{%\setcounter{itemize}{0}
\begin{itemize}
\item \label{bkm:Ref371512903}\textitbf{ada} ‘exist’ precedes an indefinite/nonidentifiable theme\end{itemize}
}\\
&  & \textitbf{ada} \textsc{theme} & \multicolumn{2}{l}{‘a \textsc{theme} exists’} & Existence\\
&  &  & \multicolumn{2}{l}{‘a \textsc{theme} is available’} & Availability\\
& \multicolumn{5}{l}{\begin{itemize}
\item \label{bkm:Ref371512904}\textitbf{ada} ‘exist’ follows a definite/identifiable theme\end{itemize}
}\\
&  & \textsc{theme} \textitbf{ada} & \multicolumn{2}{l}{‘the \textsc{theme} exists’} & \textitbf{\textmd{\textup{Existence}}}\\
&  &  & \multicolumn{2}{l}{‘\textsc{possessor} has the \textsc{possessum}’} & Possession\\
\multicolumn{6}{l}{%\setcounter{itemize}{0}
\begin{itemize}
\item \label{bkm:Ref371512905}Two-argument existential clauses\end{itemize}
}\\
& \multicolumn{5}{l}{Possession of an indefinite possessum}\\
&  & \multicolumn{2}{l}{\textsc{subject} \textitbf{ada} \textsc{object}} & \multicolumn{2}{l}{‘\textsc{possessor} has a \textsc{possessum}’}\\
\lspbottomrule
\end{tabular}
\section{Comparative clauses}
\label{bkm:Ref367459492}
Papuan Malay employs two structurally distinct types of comparative constructions: degree-marking clauses, as shown in (0) and identity-marking clauses, as illustrated in (0).



Generally speaking, comparative clauses with gradable predicates involve “two participants being compared, and the property in terms of which they are compared” {\citep[788]{Dixon2008}}. The two participants being compared are the \textsc{comparee}, that is, the object of comparison, and the \textsc{standard} of comparison, in {Dixon’s (2008)} terminology. When the standard is expressed in a prepositional phrase, the preposition serves as the \textsc{mark} of the comparison. The property attributed to the comparee and standard is the \textsc{parameter} of comparison. The parameter is marked with an \textsc{index} of comparison which signals the “ordering relation” between the comparee and the standard “to the degree or amount to which they possess some property” {(Kennedy 2006: 690–691)}.
\end{styleBodyvvafter}

\begin{styleExampleTitle}
Degree-marking and identity-marking comparative clauses
\end{styleExampleTitle}

\begin{tabular}{llllll}
\lsptoprule
\label{bkm:Ref370914428}
\gll {\textsc{comparee}} {\textsc{index}} {\textsc{parameter}} {\textsc{mark}} {\textsc{standard}}\\ %
& dia & \bluebold{lebi} & \bluebold{tinggi} & dari & saya\\
& \textsc{3sg} & more & be.high & from & \textsc{1sg}\\
\lspbottomrule
\end{tabular}
\ea
\glt 
‘he/she is \bluebold{taller} than me’ (Lit. ‘be \bluebold{more tall} from me’) \textstyleExampleSource{[Elicited BR111011.002]}
\z

\begin{tabular}{llllll}
\lsptoprule
\label{bkm:Ref371079076}
\gll {\textsc{comparee}} {\textsc{parameter}} {\textsc{index}} {\textsc{mark}} {\textsc{standard}}\\ %
& de & \bluebold{sombong} & \bluebold{sama} & deng & ko\\
& \textsc{3sg} & be.arrogant & be.same & with & \textsc{2sg}\\
\lspbottomrule
\end{tabular}
\ea
\glt 
‘she’ll be \bluebold{as arrogant as} you (are)’ (Lit. ‘be \bluebold{arrogant same} with you’) \textstyleExampleSource{[081006-005-Cv.0002]}
\z


Papuan Malay degree-marking clauses, expressing the notions of superiority, as in the elicited example in (0), inferiority, or superlative, are discussed in §11.5.1. Identity-marking clauses, signaling similarity, as in (0), or dissimilarity, are described in §11.5.2. Both clause types differ in terms of their word order. In degree-marking clauses the parameter follows the index, while in identity-marking clauses the parameter precedes the index or is omitted.
\end{styleBodyxvafter}

\subsection{Degree-marking comparative clauses}
\label{bkm:Ref272565904}
Degree-marking comparative clauses convey the notions of superiority, inferiority, and superlative in the sense of ‘less than’, ‘more than’ and ‘most’, respectively, such that ‘\textsc{comparee} is more/less/most \textsc{parameter} (than \textsc{standard})’. In this type of comparative clause, the parameter follows the index, as illustrated in the elicited superiority clause in (0). The following constituents serve as index: the grading adverb \textitbf{lebi} ‘more’ signals superiority while \textitbf{paling} ‘most’ marks superlative; the bivalent verb \textitbf{kurang} ‘lack’ marks inferiority. The standard can be stated overtly, as in (0) and (0), or be omitted as in (0) to (0).



In clauses with an overt standard, the standard is expressed in a prepositional phrase which is introduced with the elative preposition \textitbf{dari} ‘from’, as illustrated in (0) and in the elicited example in (0). This preposition serves as the mark of the comparison. In the corpus, however, degree-marking clauses with an overt standard are rare. The corpus contains only two superiority clauses, one of which is given in (0). Inferiority clauses with an overt standard are also possible, as shown with the elicited example in (0). Superlative clauses with an overt standard are unattested.
\end{styleBodyvvafter}

\begin{styleExampleTitle}
Superiority and inferiority clauses with overt standard
\end{styleExampleTitle}

\begin{tabular}{llllllllllll}
\lsptoprule
\label{bkm:Ref340309710}
\gll {di} {klas} {itu} {dia} {\bluebold{lebi}} {\bluebold{besar}} {dari} {smua} {ana{\Tilde}ana} {di} {dalam}\\ %
& at & class & \textsc{d.dist} & \textsc{3sg} & more & be.big & from & all & \textsc{rdp}{\Tilde}child & at & inside\\
\lspbottomrule
\end{tabular}
\ea
\glt 
‘in that class he’s \bluebold{bigger} than all the (other) kids in it’ \textstyleExampleSource{[081109-003-JR.0001]}\footnote{\\
\\
\\
\\
\\
\\
\\
\\
\\
\\
\\
\\
\\
\\
\\
\par The original recording says \textitbf{dari smuat} rather than \textitbf{dari smua} ‘than all’. Most likely the speaker wanted to say \textitbf{dari smua temang} ‘than all friends’ but cut himself off to replace \textitbf{temang} ‘friend’ with \textitbf{ana{\Tilde}ana} ‘children’.}
\z

\begin{tabular}{llllll}
\lsptoprule
\label{bkm:Ref364513751}
\gll {saya} {\bluebold{kurang}} {\bluebold{tinggi}} {dari} {dia}\\ %
& \textsc{1sg} & lack & be.high & from & \textsc{3sg}\\
\lspbottomrule
\end{tabular}
\ea
\glt 
‘I am \bluebold{shorter} than him/her’ (Lit. ‘\bluebold{lack being tall}’) \textstyleExampleSource{[Elicited BR111011.001]}
\z


Most often, the standard is elided in degree-marking clauses, as it is usually known from the discourse, as in the examples in (0) to (0). The superiority clause in (0) is part of a conversation about a village mayors’ meeting which had been delayed several times. The speaker criticizes the fact that the mayors accepted this delay in spite of the fact that they had more authority than the elided standard ‘those who caused the delay’. Likewise, in (0) to (0) the standard of comparison is known from the preceding discourse. Besides, the example in (0) shows that a superlative comparison can be reinforced with the degree adverb \textitbf{skali} ‘very’.


\begin{styleExampleTitle}
Degree-marking clauses with omitted standard
\end{styleExampleTitle}

\begin{tabular}{llllllll}
\lsptoprule
\label{bkm:Ref340309712}
\gll {kam} {punya} {fungsi} {wewenang} {\bluebold{lebi}} {\bluebold{besar}} {\arraybslash Ø}\\ %
& \textsc{2pl} & \textsc{poss} & function & authority & more & be.big & \\
\lspbottomrule
\end{tabular}
\ea
\glt 
[About a mayors’ meeting:] ‘your function (and) authority is \bluebold{bigger} (than that of those who caused the delay)’ \textstyleExampleSource{[081008-003-Cv.0056]}
\z

\begin{tabular}{lllllll}
\lsptoprule
\label{bkm:Ref340309713}
\gll {…} {karna} {itu} {\bluebold{kurang}} {\bluebold{bagus}} {\arraybslash Ø}\\ %
&  & because & \textsc{d.dist} & lack & be.good & \\
\lspbottomrule
\end{tabular}
\ea
\glt 
‘… because those (old ways) are \bluebold{less good} (than our new ways)’ (Lit. ‘\bluebold{lack being good}’) \textstyleExampleSource{[080923-013-CvEx.0010]}
\z

\begin{tabular}{llllll}
\lsptoprule
\label{bkm:Ref340309715}
\gll {puri} {tu} {\bluebold{paling}} {\bluebold{besar}} {\arraybslash Ø}\\ %
& anchovy-like.fish & \textsc{d.dist} & most & be.big & \\
\lspbottomrule
\end{tabular}
\ea
\glt 
‘that anchovy-like fish is \bluebold{the biggest} (among the larger pile of fish)’ \textstyleExampleSource{[080927-003-Cv.0002]}
\z

\begin{tabular}{llllll}
\lsptoprule
\label{bkm:Ref340309714}
\gll {Aris} {\bluebold{paling}} {\bluebold{tinggi}} {skali} {\arraybslash Ø}\\ %
& Aris & most & be.high & very & \\
\lspbottomrule
\end{tabular}
\ea
\glt 
‘Aris is \bluebold{the very tallest} (among the two of you)’ \textstyleExampleSource{[080922-001b-CvPh.0026]}
\z


In the corpus, inferiority clauses formed with \textitbf{kurang} ‘lack’ occur much less often than superiority clauses with \textitbf{lebi} ‘more’. Instead of stating that the comparee is inferior to the standard in terms of a specific quality, as in the elicited example in (0), repeated as (0), speakers prefer to use a superiority clause which asserts that the comparee is superior to the standard, as in the elicited example in (0), repeated as (0).


\begin{styleExampleTitle}
Inferiority versus superiority clauses
\end{styleExampleTitle}

\begin{tabular}{llllll}
\lsptoprule
\label{bkm:Ref340309716}
\gll {saya} {\bluebold{kurang}} {\bluebold{tinggi}} {dari} {dia}\\ %
& \textsc{1sg} & lack & be.high & from & \textsc{3sg}\\
\lspbottomrule
\end{tabular}
\ea
\glt 
‘I am \bluebold{shorter} than him/her’ (Lit. ‘\bluebold{lack being tall}’) \textstyleExampleSource{[Elicited BR111011.001]}
\z

\begin{tabular}{llllll}
\lsptoprule
\label{bkm:Ref340309717}
\gll {dia} {\bluebold{lebi}} {\bluebold{tinggi}} {dari} {saya}\\ %
& \textsc{3sg} & more & be.high & from & \textsc{1sg}\\
\lspbottomrule
\end{tabular}
\ea
\glt 
‘he/she is \bluebold{taller} than I am’ \textstyleExampleSource{[Elicited BR111011.002]}
\z


Alternatively, the attested inferiority clauses could be interpreted as instances of mitigation used for politeness. This mitigating function is also illustrated with the inferiority clauses in (0) and (0): the speakers assert that the respective referents possess less of the positive qualities of being \textitbf{ajar} ‘taught, educated’ or \textitbf{hati{\Tilde}hati} ‘careful’, instead of stating that they are ‘impolite’ or ‘careless’.


\begin{styleExampleTitle}
Inferiority clauses: Mitigation function
\end{styleExampleTitle}

\begin{tabular}{llll}
\lsptoprule
\label{bkm:Ref271358837}
\gll {Klara} {\bluebold{kurang}} {\bluebold{ajar}}\\ %
& Klara & lack & teach\\
\lspbottomrule
\end{tabular}
\ea
\glt 
‘Klara was \bluebold{impolite}’ (Lit. ‘\bluebold{lack being educated}’) \textstyleExampleSource{[081025-009a-Cv.0045]}
\z

\begin{tabular}{lllll}
\lsptoprule
\label{bkm:Ref340309721}
\gll {itu} {karna} {\bluebold{kurang}} {\bluebold{hati{\Tilde}hati}}\\ %
& \textsc{d.dist} & because & lack & \textsc{rdp}{\Tilde}liver\\
\lspbottomrule
\end{tabular}
\ea
\glt 
‘that (happened) because (I) was \bluebold{careless}’ (Lit. ‘\bluebold{lack being careful}’) \textstyleExampleSource{[081011-017-Cv.0009]}
\z


For the most part, mitigating inferiority constructions are fixed expressions, such as the \textitbf{kurang} ‘lack’ constructions presented in (0), (0) and (0).



Superlative constructions have the additional function of expressing ‘high degrees of parameter’, as illustrated in (0) and (0). In (0), the superlative construction \textitbf{paling emosi} ‘feel most angry (about)’ conveys that the speaker was ‘very very angry’. Likewise in (0), the superlative construction signals ‘high degrees of parameter’. The superlative clauses in (0) and (0) do not involve a comparison, unlike the superlative constructions in (0) and (0).
\end{styleBodyvvafter}

\begin{styleExampleTitle}
Superlative clauses: ‘High degrees of parameter’
\end{styleExampleTitle}

\begin{tabular}{lll}
\lsptoprule
\label{bkm:Ref340309722}
\gll {\bluebold{paling}} {\bluebold{emosi}}\\ %
& most & feel.angry(.about)\\
\lspbottomrule
\end{tabular}
\ea
\glt 
‘(I) \bluebold{felt very very angry}’ (Lit. ‘\bluebold{most angry}’) \textstyleExampleSource{[081025-009a-Cv.0154]}
\z

\begin{tabular}{llll}
\lsptoprule
\label{bkm:Ref340309723}
\gll {de} {\bluebold{paling}} {\bluebold{takut}}\\ %
& \textsc{3sg} & most & feel.afraid(.of)\\
\lspbottomrule
\end{tabular}
\ea
\glt 
‘he \bluebold{felt very very afraid}’ (Lit. ‘\bluebold{feel most afraid}’) \textstyleExampleSource{[081115-001a-Cv.0060]}
\z


In summary, the scheme for degree-marking comparative constructions in Papuan Malay is ‘\textsc{comparee} – \textsc{index} – \textsc{parameter} (– \textsc{mark} – \textsc{standard})’.
\end{styleBodyxvafter}

\subsection{Identity-marking comparative clauses}
\label{bkm:Ref271634980}
Identity-marking comparative clauses express similarity or dissimilarity between a comparee and a standard, in the sense of ‘same as’ or ‘different from’, respectively. In this type of comparative clause, the index follows the parameter, as illustrated with the similarity clause in (0), repeated as (0).


\begin{styleExampleTitle}
Identity-marking comparative clauses
\end{styleExampleTitle}

\begin{tabular}{llllll}
\lsptoprule
\label{bkm:Ref340309724}
\gll {\textsc{comparee}} {\textsc{parameter}} {\textsc{index}} {\textsc{mark}} {\textsc{standard}}\\ %
& de & \bluebold{sombong} & \bluebold{sama} & deng & ko\\
& \textsc{3sg} & be.arrogant & be.same & with & \textsc{2sg}\\
\lspbottomrule
\end{tabular}
\ea
\glt 
‘she’ll be \bluebold{as arrogant as} you (are)’ \textstyleExampleSource{[081006-005-Cv.0002]}
\z


Similarity comparisons are presented in (0) to (0) and dissimilarity comparisons in (0) to (0).



In similarity clauses, the index is the stative verb \textitbf{sama} ‘be same’, and the mark is the comitative preposition \textitbf{dengang} ‘with’, with its short form \textitbf{deng}. The standard can be encoded in two ways. One option is to express it in a prepositional phrase, as in (0) to (0); the second possibility is illustrated in (0) to (0). In the similarity comparison in (0), the comparee and standard are considered to be similar in terms of a specific property, such that ‘\textsc{comparee} is as \textsc{parameter} as \textsc{standard}’. If, however, the parameter is known from the context, it can be omitted, such that ‘\textsc{comparee} is the same as \textsc{standard} (in terms of an understood \textsc{parameter})’, as in (0) where \textitbf{de} ‘\textsc{3sg}’ is the \textsc{comparee} and \textitbf{kitong} ‘\textsc{2pl}’ is the \textsc{standard}.
\end{styleBodyvvafter}

\begin{styleExampleTitle}
Similarity clauses: Standard is expressed in a prepositional phrase
\end{styleExampleTitle}

\begin{tabular}{llllllll}
\lsptoprule
\label{bkm:Ref340309725}
\gll {\multicolumn{2}{l}{orang}} {\multicolumn{2}{l}{itu}} {\bluebold{ganas}} {\bluebold{sama}} {deng}\\ %
& \multicolumn{2}{l}{person} & \multicolumn{2}{l}{\textsc{d.dist}} & feel.furious(.about) & be.same & with\\
& dong & \multicolumn{2}{l}{pu} & \multicolumn{4}{l}{penunggu}\\
& \textsc{3pl} & \multicolumn{2}{l}{\textsc{poss}} & \multicolumn{4}{l}{tutelary.spirit}\\
\lspbottomrule
\end{tabular}
\ea
\glt 
‘those people were \bluebold{as ferocious as} their tutelary spirits’ \textstyleExampleSource{[081025-006-Cv.0286]}
\z

\begin{tabular}{lllllll}
\lsptoprule
\label{bkm:Ref364850091}
\gll {de} {\bluebold{Ø}} {\bluebold{sama}} {dengang} {kitong} {juga}\\ %
& \textsc{3sg} &  & be.same & with & \textsc{1pl} & also\\
\lspbottomrule
\end{tabular}
\ea
\glt 
‘she is also \bluebold{the same as} we are (in terms of \bluebold{being foreign})’ \textstyleExampleSource{[081010-001-Cv.0061]}
\z


Alternatively, the standard can be encoded as the clausal subject together with the comparee, such that ‘\textsc{comparee} \& \textsc{standard} are equally \textsc{parameter}’, as in (0) to (0). The standard and comparee can be encoded by a coordinate noun phrase, as in (0), or a plural personal pronoun, as in (0). Again, the parameter can be omitted if it is understood from the context, such that ‘\textsc{comparee} \& \textsc{standard} are the same (in terms of an understood \textsc{parameter}), as in (0).


\begin{styleExampleTitle}
Similarity clauses: Standard is encoded as the clausal subject together with the comparee
\end{styleExampleTitle}

\begin{tabular}{lllllll}
\lsptoprule
\label{bkm:Ref371062195}
\gll {sa} {deng} {mace} {tu} {\bluebold{cocok}} {\bluebold{sama}}\\ %
& \textsc{1sg} & with & woman & \textsc{d.dist} & be.suitable & be.same\\
\lspbottomrule
\end{tabular}
\ea
\glt 
‘I and that woman are \bluebold{equally well-matched}’ \textstyleExampleSource{[081011-022-Cv.0016]}
\z

\begin{tabular}{lllllll}
\lsptoprule
\label{bkm:Ref340309726}
\gll {kam} {dua} {pu} {mulut} {\bluebold{besar}} {\bluebold{sama}}\\ %
& \textsc{2pl} & two & \textsc{poss} & mouth & be.big & be.same\\
\lspbottomrule
\end{tabular}
\ea
\glt 
‘the two of yours mouth is \bluebold{equally big}’ \textstyleExampleSource{[080922-004-Cv.0033]}
\z

\begin{tabular}{lllll}
\lsptoprule
\label{bkm:Ref340309729}
\gll {prempuang} {laki{\Tilde}laki} {\bluebold{Ø}} {\bluebold{sama}}\\ %
& woman & \textsc{rdp}{\Tilde}husband &  & be.same\\
\lspbottomrule
\end{tabular}
\ea
\glt 
‘women (and) men are \bluebold{the same} (in terms of \bluebold{having leadership qualities})’ \textstyleExampleSource{[081011-023-Cv.0244]}
\z


Not only the parameter, but also the standard can be omitted if it is understood from the context. In (0), for instance, the omitted standard is ‘the Yali children’, while the omitted parameter has to do with the fact that both the comparee and standard are adventurous and would rather roam the forest than study.


\begin{styleExampleTitle}
Similarity clauses with omitted standard and parameter
\end{styleExampleTitle}

\begin{tabular}{lllllllll}
\lsptoprule
\label{bkm:Ref340309734}
\gll {\multicolumn{4}{l}{misionaris{\Tilde}misionaris}} {dong} {punya} {ana{\Tilde}ana} {juga}\\ %
& \multicolumn{4}{l}{\textsc{rdp}{\Tilde}missionary} & \textsc{3pl} & \textsc{poss} & \textsc{rdp}{\Tilde}child & also\\
& \bluebold{Ø} & \bluebold{sama} & saja & \multicolumn{5}{l}{\bluebold{Ø}}\\
&  & be.same & just & \multicolumn{5}{l}{}\\
\lspbottomrule
\end{tabular}
\ea
\glt 
‘the missionaries’ children are just \bluebold{the same} (as the Yali children in terms of \bluebold{being adventurous})’ \textstyleExampleSource{[081011-022-Cv.0280]}
\z


Dissimilarity clauses are formed without an overt parameter. Instead, the comparee and standard are compared in terms of an understood attribute or quality, such that ‘\textsc{comparee} is different from \textsc{standard} (in terms of an understood \textsc{parameter})’, as illustrated in (0) to (0).



The index is the stative verb \textitbf{laing} ‘be different’ or \textitbf{beda} ‘be different’, and the mark is elative \textitbf{dari} ‘from’ or comitative \textitbf{dengang} ‘with’. Dissimilarity comparisons are typically formed with \textitbf{laing dari} ‘be different from’ as in (0). They signal that the two participants are dissimilar in terms of their overall nature. If speakers want to indicate that the two participants diverge from each other in terms of specific attributes or features rather than their overall nature, they use a dissimilarity clause formed with \textitbf{beda dengang} ‘be different with’. This is demonstrated with the elicited example in (0), which contrasts with the clause in (0). Another example is the dissimilarity clause in (0). Clauses formed with \textitbf{beda dari} ‘be different from’ are also acceptable but considered to be Indonesian-like rather than typical Papuan Malay. Clauses formed with \textitbf{laing dengang} ‘be different from’ are unacceptable.
\end{styleBodyvvafter}

\begin{styleExampleTitle}
Dissimilarity clauses: ‘\textsc{comparee} is different from \textsc{standard}’
\end{styleExampleTitle}

\begin{tabular}{llllll}
\lsptoprule
\label{bkm:Ref340309735}
\gll {sifat} {ini} {\bluebold{laing}} {\bluebold{dari}} {ko}\\ %
& nature & \textsc{d.prox} & be.different & from & \textsc{2sg}\\
\lspbottomrule
\end{tabular}
\ea
\glt 
‘this disposition (of mine) is \bluebold{different from} you (in every aspect)’ \textstyleExampleSource{[081110-008-CvNP.0089]}
\z

\begin{tabular}{llllll}
\lsptoprule
\label{bkm:Ref371329422}
\gll {sifat} {ini} {\bluebold{beda}} {\bluebold{dengang}} {ko}\\ %
& nature & \textsc{d.prox} & be.different & with & \textsc{2sg}\\
\lspbottomrule
\end{tabular}
\ea
\glt 
‘this disposition (of mine) is \bluebold{different from} you (in terms of some specific aspect)’ \textstyleExampleSource{[Elicited BR111011.008]}
\z

\begin{tabular}{lllllll}
\lsptoprule
\label{bkm:Ref340309737}
\gll {orang} {Papua} {\bluebold{beda}} {\bluebold{dengang}} {orang} {Indonesia}\\ %
& person & Papua & be.different & with & person & Indonesia\\
\lspbottomrule
\end{tabular}
\ea
\glt 
‘Papuans are \bluebold{different from} Indonesians (in terms of their physical features)’ \textstyleExampleSource{[081029-002-Cv.0009]}
\z


If the comparee is understood from the context, it can be omitted, as shown in (0).


\begin{styleExampleTitle}
Dissimilarity clauses with omitted comparee
\end{styleExampleTitle}

\begin{tabular}{llllllll}
\lsptoprule
\label{bkm:Ref340310721}
\gll {banyak,} {tapi} {Ø} {\bluebold{beda}} {\bluebold{dengang}} {Jayapura} {punya}\\ %
& many & but &  & be.different & with & Jayapura & \textsc{poss}\\
\lspbottomrule
\end{tabular}
\ea
\glt 
[Comparing different melinjo varieties:] ‘(there’re) lots (of melinjo), but (they’re) \bluebold{different from} Jayapura’s (melinjos in terms of \bluebold{being bitter})’ \textstyleExampleSource{[080923-004-Cv.0010]}
\z


In summary, the typical scheme for identity-marking comparative constructions in Papuan Malay is ‘(\textsc{comparee} – \textsc{parameter}) – \textsc{index} – \textsc{mark} – \textsc{standard}’. Alternatively, the standard can be encoded as the clausal subject together with the comparee, such that ‘\textsc{comparee} \& \textsc{standard} are equally \textsc{parameter}’.
\end{styleBodyxvafter}

\subsection{Summary}

Papuan Malay employs two structurally distinct types of comparative constructions: (1) degree-marking clauses, and (2) identity-marking clauses.



Degree-marking clauses signal superiority, inferiority, or superlative. The following constituents serve as index: \textitbf{lebi} ‘more’ (superiority), \textitbf{kurang} ‘lack’ (inferiority), and \textitbf{paling} ‘most’ (superlative). The\textstyleBodyvvafterChar{ }mark is elative \textitbf{dari} ‘from’. The index precedes the parameter. The standard together with its mark can be omitted. The basic scheme for this type of comparative clauses is given in (0).
\end{styleBodyvvafter}

\begin{styleExampleTitle}
Scheme for degree-marking clauses
\end{styleExampleTitle}

\begin{tabular}{llllll}
\lsptoprule
\label{bkm:Ref371174960}
\gll {\textsc{comparee}} {\textsc{index}} {\textsc{parameter}} {(\textsc{mark}} {\textsc{standard})}\\ %
\lspbottomrule
\end{tabular}

Identity-marking clauses express similarity or dissimilarity. In similarity clauses the index is \textitbf{sama} ‘be same’ and the mark is comitative \textitbf{dengang} ‘with’. In dissimilarity clauses, the index is \textitbf{laing} ‘be different’ in combination with the mark \textitbf{dari} ‘from’, or \textitbf{beda} ‘be different’ in combination with the mark \textitbf{dengang} ‘with’. Clauses formed with \textitbf{laing dari} ‘be different from’ indicate overall dissimilarity, whereas clauses with \textitbf{beda dengang} ‘be different from’ signal dissimilarity in terms of some specific features. In identity-marking clauses the index follows the parameter, which is optional. The standard is typically encoded in a prepositional phrase, with the preposition serving as the mark of comparison. This scheme for identity-marking clauses is illustrated in (0). In similarity clauses, the standard can also be encoded as the clausal subject together with the comparee, as shown in (0).


\begin{styleExampleTitle}
Schemes for identity-marking clauses
\end{styleExampleTitle}

\begin{tabular}{llllllll}
\lsptoprule
\label{bkm:Ref371174961}
\gll {(\textsc{comparee}} {\multicolumn{2}{l}{\textsc{parameter})}} {\textsc{index}} {\multicolumn{2}{l}{\textsc{mark}}} {\textsc{standard}}\\ %
&  & \multicolumn{2}{l}{} &  & \multicolumn{2}{l}{} & \\
\label{bkm:Ref371174962}
\gll {\multicolumn{2}{l}{\textsc{comparee} \& \textsc{standard}} & \multicolumn{3}{l}{(\textsc{parameter})} & \multicolumn{2}{l}{\textsc{index}}}\\ %
\lspbottomrule
\end{tabular}
\section{Summary}
\label{bkm:Ref367459571}
This chapter has described different types of verbal clauses. The most pertinent distinction is that between intransitive and transitive clauses. It is important to note, though, that Papuan Malay verbs allow but do not require core arguments. Trivalent verbs most often occur in monotransitive or intransitive clauses rather than in ditransitive clauses. Along similar lines, bivalent verbs are very commonly used in intransitive clauses.



Also discussed are causative clauses. They are the result of a valency-increasing operation. Papuan Malay causatives are monoclausal V\textsubscript{1}V\textsubscript{2} constructions in which causative V\textsubscript{1} encodes the notion of cause while V\textsubscript{2} expresses the notion of effect. Papuan Malay has two causative verbs which usually produce causer-controlled causatives: trivalent \textitbf{kasi} ‘give’, and bivalent \textitbf{biking} ‘make’. While \textitbf{kasi}{}-causatives stress the outcome of the manipulation, \textitbf{biking}{}-causatives focus on the manipulation of circumstances, which leads to the effect. Causatives with \textitbf{kasi} ‘give’ can have mono- or bivalent bases, while \textitbf{biking}{}-causatives always have monovalent bases.
\end{styleBodyvafter}


Reciprocal clauses are a third type of clauses described in this chapter. They are formed with the reciprocity marker \textitbf{baku} ‘\textsc{recp}’. In these clauses, two predications are presented as one, with two participants equivalently acting upon each other. In simple reciprocals, both participants are encoded as the clausal subject. In discontinuous reciprocals, the reciprocee is expressed with a comitative phrase. Both clause types typically result in a reduction in syntactic valency. The exception is simple constructions with a sociative reading which are characterized by valency retention.
\end{styleBodyvafter}


Also discussed are existential clauses formed with the existential verb \textitbf{ada} ‘exist’. Two clause types can be distinguished: intransitive clauses with one core argument, and transitive clauses with two core arguments. In one-argument clauses, \textitbf{ada} ‘exist’ precedes or follows the subject, or theme, depending on its definiteness. Existential clauses express existence, availability, or possession.
\end{styleBodyvafter}


A final type of verbal clauses discussed in this chapter are degree-marking and identity-marking comparative clauses. Degree-marking clauses denote superiority, inferiority, or superlative. In these clauses, the parameter follows the index, the comparee takes the subject slot, and the optional standard is expressed in a prepositional phrase. Identity-marking clauses designate similarity or dissimilarity. In these constructions, the parameter either precedes the index or is omitted. The comparee takes the subject slot while the standard is usually expressed with a prepositional phrase. In similarity clauses, the standard can also be encoded as the clausal subject together with comparee.
\end{styleBodyvafter}

%\setcounter{page}{1}\chapter[Nonverbal clauses]{Nonverbal clauses}
\label{bkm:Ref293423883}
This chapter discusses nonverbal predicate clauses in Papuan Malay, that is, clauses in which the main semantic content is not conveyed by a verb or verbal phrase, but by some other predicate category.



Papuan Malay has three syntactically distinct types of nonverbal predicate clauses, namely, nominal, numeral/quantifier, and prepositional predicate clauses. Nominal predicates have ascriptive or equative function and also encode possession. Numeral and quantifier predicates denote quantities. Prepositional predicates encode locational or nonlocational relations between a figure and the ground. As in verbal clauses, the nonverbal predicate typically follows the subject; no copula intervenes (see Chapter 11).
\end{styleBodyvafter}


Before discussing the three types of nonverbal clauses in more detail, §12.1 explores which constituents can fill the subject slot in nonverbal clauses. Nominal predicate clauses are described in §12.2, numeral and quantifier clauses in §12.3, and prepositional clauses in §12.4. The main points of this chapter are summarized in §12.5. (Negation of nonverbal clauses is discussed in §13.1.)
\end{styleBodyvxvafter}

\section{Nonverbal clause subjects}
\label{bkm:Ref263353131}
In nonverbal clauses, the subject can be a noun or noun phrase, a personal pronoun, or a demonstrative, as shown in (0) to (0). Alternatively, the subject can be elided if it is understood from the context, as shown in (0) and (0).



In the nominal clause in (0) and the quantifier clause in (0), the subject is a noun phrase or a noun, respectively. In the nominal clause in (0) and the prepositional clause in (0) the subjects are encoded as personal pronouns. And in the numeral clause in (0) and the prepositional clause in (0), the subjects are expressed with demonstratives. (For a nominal clause with a demonstrative subject see (0) in §12.2, p. \pageref{bkm:Ref436750907}, for a numeral clause with a personal pronoun subject see (0) in §12.3, p. \pageref{bkm:Ref436750967}, and for a prepositional phrase with a noun phrase subject see (0) in §12.4.1, p. \pageref{bkm:Ref436751015}.)
\end{styleBodyvvafter}

\begin{styleExampleTitle}
Subjects in nonverbal clauses
\end{styleExampleTitle}

\begin{tabular}{lllll}
\lsptoprule
\label{bkm:Ref340309670}
\gll {\bluebold{orang}} {\bluebold{ini}} {muka} {baru}\\ %
& person & \textsc{d.prox} & face & be.new\\
\lspbottomrule
\end{tabular}
\ea
\glt 
‘\bluebold{this person} is a new person’ \textstyleExampleSource{[080919-004-NP.0079]}
\z

\begin{tabular}{lllll}
\lsptoprule
\label{bkm:Ref340310727}
\gll {…} {\bluebold{picaang}} {juga} {banyak}\\ %
&  & splinter & also & many\\
\lspbottomrule
\end{tabular}
\ea
\glt 
‘[at the beach] there are also lots of \bluebold{splinters}’ (Lit. ‘\bluebold{the splinters} (are) also many’) \textstyleExampleSource{[080917-006-CvHt.0008]}
\z

\begin{tabular}{lllllllll}
\lsptoprule
\label{bkm:Ref340309671}
\gll {\bluebold{ko}} {prempuang} {Jayapura,} {de} {bilang,} {\bluebold{ko}} {prempuang} {Demta}\\ %
& \textsc{2sg} & woman & Jayapura & \textsc{3sg} & say & \textsc{2sg} & woman & Demta\\
\lspbottomrule
\end{tabular}
\ea
\glt 
‘‘\bluebold{you}’re a Jayapura girl’, he says, ‘\bluebold{you}’re a Demta girl’’ \textstyleExampleSource{[081006-025-CvEx.0014]}
\z

\begin{tabular}{llllllll}
\lsptoprule
\label{bkm:Ref366948584}
\gll {baru} {Sarles} {ini} {\bluebold{de}} {di} {blakang} {bapa}\\ %
& and.then & Sarles & \textsc{d.prox} & \textsc{3sg} & at & backside & father\\
\lspbottomrule
\end{tabular}
\ea
\glt 
‘but then Sarles here, \bluebold{he} was behind father’ \textstyleExampleSource{[081025-009b-Cv.0014]}
\z

\begin{tabular}{lllllll}
\lsptoprule
\label{bkm:Ref340310726}
\gll {\bluebold{itu}} {satu} {saja} {blum} {brapa} {…}\\ %
& \textsc{d.dist} & one & just & not.yet & several & \\
\lspbottomrule
\end{tabular}
\ea
\glt 
[Conversation about cloths as a bride-price:] ‘\bluebold{that} is just one (cloth and) not yet several (cloths) …’ \textstyleExampleSource{[081006-029-CvEx.0011]}
\z

\begin{tabular}{llllll}
\lsptoprule
\label{bkm:Ref366948586}
\gll {a} {\bluebold{itu}} {di} {Wakde} {sana}\\ %
& ah! & \textsc{d.dist} & at & Wakde & \textsc{l.dist}\\
\lspbottomrule
\end{tabular}
\ea
\glt 
‘ah, \bluebold{that}’s in Wakde over there’ \textstyleExampleSource{[081006-016-Cv.0030]}
\z


If the subject can be inferred from the context it can also be elided. This is illustrated with the two nominal clauses in (0) and the prepositional clause in (0). In the two nominal clauses in (0), the predicates \textitbf{kitong pu ana} ‘our child’ and \textitbf{tong punya dara} ‘our blood’ are co-referential with \textitbf{de} ‘\textsc{3sg}’. As the subject was already introduced at the beginning of the utterance, it is omitted in the nominal clause. In the prepositional clause in (0), the elided subject is \textitbf{ko} ‘\textsc{2sg}’, that is, the addressee.


\begin{styleExampleTitle}
Elision of subjects in nonverbal clauses
\end{styleExampleTitle}

\begin{tabular}{lllllllllllllll}
\lsptoprule
\label{bkm:Ref340309691}
\gll {\multicolumn{2}{l}{\bluebold{de}}} {\multicolumn{3}{l}{minta}} {\multicolumn{2}{l}{apa,}} {kitong} {kasi} {karna} {\bluebold{${\varnothing}$}} {kitong} {punya} {ana}\\ %
& \multicolumn{2}{l}{\textsc{3sg}} & \multicolumn{3}{l}{request} & \multicolumn{2}{l}{what} & \textsc{1pl} & give & because &  & \textsc{1pl} & \textsc{poss} & child\\
& \textstyleExampleSource{…} & \multicolumn{2}{l}{\bluebold{${\varnothing}$}} & masi & \multicolumn{2}{l}{\bluebold{tong}} & \multicolumn{2}{l}{\bluebold{punya}} & \multicolumn{6}{l}{\bluebold{dara}}\\
&  & \multicolumn{2}{l}{} & still & \multicolumn{2}{l}{\textsc{1pl}} & \multicolumn{2}{l}{\textsc{poss}} & \multicolumn{6}{l}{blood}\\
\lspbottomrule
\end{tabular}
\ea
\glt 
‘\bluebold{she} requests something, we give (it to her), because (\bluebold{she}’s) our child, \textstyleExampleSource{…} (\bluebold{she}’s) still our blood’ \textstyleExampleSource{[081006-025-CvEx.0020/0022]}
\z

\begin{tabular}{llllllll}
\lsptoprule
\label{bkm:Ref366950492}
\gll {wa,} {sa} {pikir} {\bluebold{${\varnothing}$}} {masi} {di} {Arbais?}\\ %
& wow! & \textsc{1sg} & think &  & still & at & Arbais\\
\lspbottomrule
\end{tabular}
\ea
\glt
[Addressing a guest:] ‘wow!, I thought (\bluebold{you}) were still in Arbais’ \textstyleExampleSource{[081011-011-Cv.0044]}
\end{styleFreeTranslEngxvpt}

\section{Nominal predicate clauses}
\label{bkm:Ref272583000}\label{bkm:Ref262745424}
In nonverbal clauses with nominal predicates, a noun or a noun phrase conveys the main semantic content.



In Papuan Malay, nominal clauses have three functions: (1) to describe the subject, (2) to identify the subject, and (3) to express possession of an indefinite possessum. Nominal predicates always receive a static reading.
\end{styleBodyvafter}


Nominal predicates conveying a description of the subject are also referred to as “ascriptive predications”, adopting {Hengeveld’s (1992: 101)} terminology: they describe a particular entity that is denoted by the subject of the clause such that ‘\textsc{s} is a member of \textsc{n}/\textsc{np}’. That is, an ascriptive clause asserts that this entity belongs to the class of entities specified in the nonreferential nominal predicate. By contrast, nominal predicates expressing identification are “equative predicates”. They are referential and equate the particular entity denoted by the subject of the clause to the entity specified in the predicate such that ‘\textsc{s} is \textsc{n}/\textsc{np}’. (See {Hengeveld 1992: 101}{;} {Payne 1997: 105}.) In nominal clauses conveying the notion of possession the subject embodies the semantic role of possessor while the predicate functions as an indefinite possessum such that ‘\textsc{possessor} has a \textsc{possessum}’.
\end{styleBodyvafter}


Papuan Malay ascriptive, equative, and possessive nominal predicates are different in terms of their semantics, but not in terms of their structure. That is, Papuan Malay does not distinguish the three nominal predicate types as far as their syntactic or intonational features are concerned; all three are formed by juxtaposition of two noun phrases with the subject preceding the predicate. This is illustrated with the ascriptive clauses in (0) and (0), the equative clauses in (0) and (0), and the possessive clauses in (0) to (0).
\end{styleBodyvafter}


In the ascriptive clause in (0), the subject \textitbf{saya} ‘\textsc{1sg}’ is asserted to belong to the class of \textitbf{manusia} ‘human being’. In the ascriptive clause in (0), the subject \textitbf{ko} ‘\textsc{2sg}’ is part of the class of \textitbf{prempuang Demta} ‘Demta girls’. The equative clause in (0) identifies the predicate \textbf{\textit{ade}} ‘younger sibling’ with the subject \textitbf{dia} ‘\textsc{3sg}’. Along similar lines, the equative clause in (0) identifies the predicate \textitbf{klawar} ‘cave bat’ with the subject \textitbf{itu} ‘\textsc{d.dist}’. The example in (0) also shows that nonverbal predicates can be modified with adverbs, such as \textitbf{masi} ‘still’.
\end{styleBodyvvafter}

\begin{styleExampleTitle}
Ascriptive clauses
\end{styleExampleTitle}

\begin{tabular}{lllll}
\lsptoprule
\label{bkm:Ref340309664}
\gll {misalnya} {saya} {\bluebold{manusia}} {\bluebold{biasa}}\\ %
& for.example & \textsc{1sg} & human.being & be.usual\\
\lspbottomrule
\end{tabular}
\ea
\glt 
[About humans and evil spirits:] ‘for example, I \bluebold{am a normal human being}’ \textstyleExampleSource{[081006-022-CvEx.0025]}
\z

\begin{tabular}{llllllll}
\lsptoprule
\label{bkm:Ref366952968}
\gll {ko} {\bluebold{prempuang}} {\bluebold{Demta},} {ko} {pulang} {ke} {Demta}\\ %
& \textsc{2sg} & woman & Demta & \textsc{2sg} & go.home & to & Demta\\
\lspbottomrule
\end{tabular}
\ea
\glt 
‘you \bluebold{are a Demta girl}, go home to Demta!’ \textstyleExampleSource{[081006-025-CvEx.0014]}
\z

\begin{styleExampleTitle}
Equative clauses
\end{styleExampleTitle}

\begin{tabular}{llll}
\lsptoprule
\label{bkm:Ref340309665}
\gll {dia} {masi} {\bluebold{ade}}\\ %
& \textsc{3sg} & still & ySb\\
\lspbottomrule
\end{tabular}
\ea
\glt 
‘she’s still (my) \bluebold{younger sister}’ \textstyleExampleSource{[080927-009-CvNP.0038]}
\z

\begin{tabular}{llll}
\lsptoprule
\label{bkm:Ref436750907}\label{bkm:Ref340309677}
\gll {o,} {itu} {\bluebold{klawar}}\\ %
& oh! & \textsc{d.dist} & cave.bat\\
\lspbottomrule
\end{tabular}
\ea
\glt 
‘oh, that \bluebold{was a bat}’ \textstyleExampleSource{[081023-001-Cv.0041]}
\z


The nominal clauses in (0) to (0) express possession of an indefinite possessum. In (0), the subject \textitbf{saya} ‘1\textsc{sg}’ has the semantic role of possessor, while the predicate \textitbf{empat ana} ‘four children’ functions as the possessum. In (0), the possessor \textitbf{de} ‘3\textsc{sg}’ is juxtaposed to the possessum \textitbf{ana kecil} ‘small child’. The possessive clauses in (0) and (0) encode inalienable possession relations. The clauses in (0) and (0), by contrast, denote alienable possession relations, namely between a human referent and animate nonhuman \textitbf{ikang} ‘fish’ in (0) and inanimate \textitbf{glang puti} ‘silver/tin bracelets’ in (0). (Alternatively, possession of an indefinite possessum can be encoded by an existential clause; for details see §11.4.2. Possession of a definite possessum is encoded by an adnominal possessive construction; for details see Chapter 9 and also §11.4.1.)


\begin{styleExampleTitle}
Possessive clauses: Possession of an indefinite possessum
\end{styleExampleTitle}

\begin{tabular}{llll}
\lsptoprule
\label{bkm:Ref340309666}
\gll {saya} {\bluebold{empat}} {\bluebold{ana}}\\ %
& \textsc{1sg} & four & child\\
\lspbottomrule
\end{tabular}
\ea
\glt 
‘I \bluebold{(have) four children}’ \textstyleExampleSource{[081006-024-CvEx.0001]}
\z

\begin{tabular}{llllll}
\lsptoprule
\label{bkm:Ref340309667}
\gll {baru} {de} {\bluebold{ana}} {\bluebold{kecil}} {lagi}\\ %
& and.then & \textsc{3sg} & child & be.small & again\\
\lspbottomrule
\end{tabular}
\ea
\glt 
‘moreover, she \bluebold{(has) a small child} again’ \textstyleExampleSource{[081010-001-Cv.0070]}\footnote{\\
\\
\\
\\
\\
\\
\\
\\
\\
\\
\\
\\
\\
\\
\\
\par In a different context, \textitbf{de ana kecil} can also receive the equative reading ‘she (is) a small child’.}
\z

\begin{tabular}{lllll}
\lsptoprule
\label{bkm:Ref340309668}
\gll {de} {\bluebold{satu},} {sa} {\bluebold{satu}}\\ %
& \textsc{3sg} & one & \textsc{1sg} & one\\
\lspbottomrule
\end{tabular}
\ea
\glt 
[Joke about two fishermen:] ‘he \bluebold{(has) one (fish)}, I \bluebold{(have) one (fish)}’ \textstyleExampleSource{[081109-011-JR.0008]}
\z

\begin{tabular}{llllll}
\lsptoprule
\label{bkm:Ref340309669}
\gll {orang} {Biak} {kang} {\bluebold{glang}} {\bluebold{puti}}\\ %
& person & Biak & you.know & bracelet & be.white\\
\lspbottomrule
\end{tabular}
\ea
\glt 
[About bride-price customs:] ‘you know, the Biak people \bluebold{(have) silver/tin bracelets}’ \textstyleExampleSource{[081006-029-CvEx.0007]}
\z


These examples also show that the predicate of a nominal clause can be a noun such as \textitbf{ade} ‘younger sibling’ in (0), or \textitbf{klawar} ‘cave bat’ in (0), or a noun phrase, such as \textitbf{manusia biasa} ‘normal human being’ in (0) or \textitbf{empat ana} ‘four children’ in (0).



If speakers want to emphasize the predicate, they can front it as for instance \textitbf{orang pintar} ‘smart person’ in (0). The predicate is set-off by a boundary intonation in that the stressed penultimate syllable of the verbal modifier \textitbf{pintar} ‘be clever’ is marked with a slight increase in pitch (“~\'{~}~”). In the second clause in (0) the speaker repeats his statement, this time however returning to the canonical subject-predicate word order.
\end{styleBodyvvafter}

\begin{styleExampleTitle}
Fronted nominal predicates
\end{styleExampleTitle}

\begin{tabular}{llllllll}
\lsptoprule
\label{bkm:Ref340309686}
\gll {trus} {\bluebold{orang}} {\bluebold{píntar}} {dia,} {dia} {orang} {pintar}\\ %
& next & person & be.clever & \textsc{3sg} & \textsc{3sg} & person & be.clever\\
\lspbottomrule
\end{tabular}
\ea
\glt
‘and then \bluebold{a smart person} he is, he’s a smart person’ \textstyleExampleSource{[081029-005-Cv.0169]}
\end{styleFreeTranslEngxvpt}

\section{Numeral and quantifier predicate clauses}
\label{bkm:Ref374454442}\label{bkm:Ref374454192}\label{bkm:Ref270511155}
In numeral and quantifier clauses, a numeral or quantifier conveys the main semantic content; again, these predicates receive a static reading. As in nominal clauses, the subject precedes the predicate. Structurally, numeral and quantifier predicates are identical to noun phrases with a postposed numeral or quantifier (see §8.3). Semantically, numeral and quantifier clauses have determining function in that they express specific properties of the subject, namely those of number and quantity, such that ‘\textsc{s} is \textsc{num}/\textsc{qt}’ as illustrated in (0) to (0).



In (0), a husband relates that in a neighboring village a woman gave birth to a snake. His wife contradicts this statement, asserting that it was not one snake but that the \textitbf{ular} ‘snake’ were \textitbf{dua} ‘two’. The analysis of the \textitbf{dua} ‘two’ as a numeral predicate and not as an adnominal modifier is confirmed by the following fact. In the context of these utterances it is possible to insert existential \textitbf{ada} ‘exist’ between the subject \textitbf{ular} ‘snake’ and the predicate \textitbf{dua} ‘two’ which gives the emphatic progressive reading \textitbf{ular ada dua} ‘the snakes were being two’ or ‘the snakes were indeed two’ (see also §5.4.1). If \textitbf{ular dua} was a noun phrase with the reading of ‘two snakes’, existential \textitbf{ada} ‘exist’ would have to precede or follow the noun phrase such that \textitbf{ada ular dua} ‘there were two snakes’ or \textitbf{ular dua ada} ‘the two snakes exist’. In (0), predicatively used \textitbf{satu} ‘one’ and \textitbf{dua blas} ‘twelve’ convey information about the numeric quantities of their respective subjects \textitbf{bulang} ‘moon’ and \textitbf{de} ‘3\textsc{sg}’. The first clause \textitbf{di langit ini bulang satu} cannot be interpreted as a prepositional predicate clause (see §12.4) in which \textitbf{bulang satu} functions as a noun phrase which takes the subject slot. Such a reading would imply that there are several moons with the speaker talking about one of them: \textitbf{bulang satu} ‘a certain moon’ (see §5.9.4 for a more detailed discussion of \textitbf{satu} ‘one’).
\end{styleBodyvvafter}

\begin{styleExampleTitle}
Numeral predicates
\end{styleExampleTitle}

\begin{tabular}{llllll}
\lsptoprule
\label{bkm:Ref340310722}
\gll { & Husband: & dia & melahirkang & ular}\\ %
&  &  & \textsc{3sg} & give.birth & snake\\
\lspbottomrule
\end{tabular}
\begin{styleFreeTranslIndentiicmEng}
Husband: ‘she gave birth to a snake’
\end{styleFreeTranslIndentiicmEng}

\begin{tabular}{lllll} &  & Wife: & ular & \bluebold{dua}\\
\lsptoprule
&  &  & snake & two\\
\lspbottomrule
\end{tabular}
\begin{styleFreeTranslIndentiicmEng}
Wife: ‘\bluebold{two} snakes’ (Lit. ‘the snakes were \bluebold{two}’ \textstyleExampleSource{[081006-022-CvEx.0002-0003]}
\end{styleFreeTranslIndentiicmEng}

\begin{tabular}{llllllllllll}
\lsptoprule
\label{bkm:Ref436750967}\label{bkm:Ref340310723}
\gll {di} {langit} {ini} {bulang} {\bluebold{satu}} {tapi} {di} {kalender} {de} {\bluebold{dua}} {\bluebold{blas}}\\ %
& at & sky & \textsc{d.prox} & month & one & but & at & calendar & \textsc{3sg} & two & tens\\
\lspbottomrule
\end{tabular}
\ea
\glt 
‘in this sky there is\bluebold{ one} moon, but in the calendar there are\bluebold{ twelve}’ (Lit. ‘the moon is \bluebold{one} … it is \bluebold{twelve}’) \textstyleExampleSource{[081109-007-JR.0002]}
\z


In (0) and (0), predicatively-used universal quantifier \textitbf{smua} ‘all’ and mid-range quantifier \textitbf{banyak} ‘many’ express the non-numeric quantities of their respective subjects \textitbf{orang Sulawesi} ‘Sulawesi people’ and \textitbf{pisang masak itu} ‘that ripe banana’.


\begin{styleExampleTitle}
Quantifier predicates
\end{styleExampleTitle}

\begin{tabular}{lllll}
\lsptoprule
\label{bkm:Ref340310725}
\gll {katanya} {orang} {Sulawesi} {\bluebold{smua}}\\ %
& it.is.being.said & person & Sulawesi & all\\
\lspbottomrule
\end{tabular}
\ea
\glt 
‘it’s being said (that) they are \bluebold{all} Sulawesi people’ (Lit. ‘(the) Sulawesi people (are) \bluebold{all}’) \textstyleExampleSource{[081029-005-Cv.0106]}
\z

\begin{tabular}{llllllllll}
\lsptoprule
\label{bkm:Ref354852624}
\gll {baru} {dong} {bawa} {pisang} {masak,} {pisang} {masak} {itu} {\bluebold{banyak}}\\ %
& and.then & \textsc{3pl} & bring & banana & cook & banana & cook & \textsc{d.dist} & many\\
\lspbottomrule
\end{tabular}
\ea
\glt
‘and then they brought ripe bananas, those ripe bananas were \bluebold{many}’ \textstyleExampleSource{[081006-023-CvEx.0071]}
\end{styleFreeTranslEngxvpt}

\section{Prepositional predicate clauses}
\label{bkm:Ref270511166}
Nonverbal clauses with prepositional predicates convey information about the relation between a figure and a ground, such that ‘\textsc{figure} is in relation to \textsc{ground}’. The figure is encoded by the clausal subject and the ground by the complement of the prepositional phrase. This phrase is juxtaposed to the subject and functions as the clausal predicate. Semantically, two types of prepositional predicate clauses can be distinguished: locational clauses (§12.4.1), and nonlocational clauses (§12.4.2). The precise semantic relation between figure and ground is defined by the preposition that heads the prepositional phrase. (For a detailed discussion of prepositions and prepositional phrases see Chapter 10).
\end{styleBodyxvafter}

\subsection{Locational prepositional clauses}
\label{bkm:Ref439952961}\label{bkm:Ref262906839}
Locational predicate clauses typically express information about the locational relation, spatial or figurative, between a figure and the ground, as shown in (0) to (0). In addition, locational predicates can have presentative function, as shown in (0). In Papuan Malay the specific kind of relation is conveyed by prepositions encoding location, namely locative \textitbf{di} ‘at, in’, allative \textitbf{ke} ‘to’, or elative \textitbf{dari} ‘from’, (see also §10.1). The ground can be encoded by a common (proper) noun or a noun phrase. Unlike prepositional phrases in verbal clauses, locative \textitbf{di} ‘at, in’ and allative \textitbf{ke} ‘to’ cannot be omitted from prepositional clauses with nominal complements as this would result in nominal clauses with unacceptable semantics (for more details on the omission of prepositions encoding location, see §10.1.5); the exceptions are preposed prepositional clauses with locative complements, as in (0) to (0).



Spatial locational predicates denote static or dynamic relations between a figure and the ground, depending on the semantics of the preposition. In (0), locative \textitbf{di} ‘at, in’ expresses the spatial location of the figure \textitbf{dia} ‘\textsc{3sg}’ at the ground \textitbf{kampung} ‘village’. In (0) allative \textitbf{ke} ‘to’ signals the motion of the figure \textitbf{dep mama} ‘her mother’ toward the goal \textitbf{Pante-Barat}.\footnote{\\
\\
\\
\\
\\
\\
\\
\\
\\
\\
\\
\\
\\
\\
\\
\par While this kind of prepositional predicate is not possible in English, it does occur in other languages such as colloquial German. Hence, (0) easily translates into \textitbf{ihre Mutter ist nach}\textitbf{ … Pante-Barat}.} In (0), elative \textitbf{dari} ‘from’ conveys the motion of the figure \textitbf{sa} ‘1\textsc{sg}’ away from the source \textitbf{Sawar}.\footnote{\\
\\
\\
\\
\\
\\
\\
\\
\\
\\
\\
\\
\\
\\
\\

‘I \bluebold{left home}, my parents’ (Lit. ‘\bluebold{went out from} the house’) [081115-001b-Cv.0045]\par }
\end{styleBodyvvafter}

\begin{styleExampleTitle}
Static and dynamic spatial locational relations between figure and ground
\end{styleExampleTitle}

\begin{tabular}{lllll}
\lsptoprule
\label{bkm:Ref340310731}
\gll {memang} {dia} {\bluebold{di}} {\bluebold{kampung}}\\ %
& indeed & \textsc{3sg} & at & village\\
\lspbottomrule
\end{tabular}
\ea
\glt 
‘indeed, he was \bluebold{in the village}’ \textstyleExampleSource{[080918-001-CvNP.0014]}
\z

\begin{tabular}{llllll}
\lsptoprule
\label{bkm:Ref436751015}\label{bkm:Ref342152472}
\gll {dep} {mama} {\bluebold{ke}} {ini} {\bluebold{Pante-Barat}}\\ %
& \textsc{3sg}:\textsc{poss} & mother & to & \textsc{d.prox} & Pante-Barat\\
\lspbottomrule
\end{tabular}
\ea
\glt 
‘her mother (went) \bluebold{to}, what’s-its-name, \bluebold{Pante-Barat}’ \textstyleExampleSource{[080919-006-CvNP.025]}
\z

\begin{tabular}{llll}
\lsptoprule
\label{bkm:Ref340310734}
\gll {sa} {\bluebold{dari}} {\bluebold{Sawar}}\\ %
& \textsc{1sg} & from & Sawar\\
\lspbottomrule
\end{tabular}
\ea
\glt 
‘I (just returned) \bluebold{from Sawar}’ \textstyleExampleSource{[080927-004-CvNP.0003]}
\z


Locational predicates also express figurative locational relations between a figure and the ground. In (0), locative \textitbf{di} ‘at, in’ conveys a figurative locational relation between the figure \textitbf{saya} ‘1\textsc{sg}’ and the ground \textitbf{IPS satu} ‘Social Sciences I’. Along similar lines, elative \textitbf{dari} ‘from’ conveys a figurative relation in (0). This example is part of a conversation about a building project that was put on hold due to the lack of funding. The figure \textitbf{smua itu} ‘all that’ refers to the delayed project while the ground \textitbf{uang} ‘money’ denotes the nonspatial source from which this delay originates. Figurative predicates with allative \textitbf{ke} ‘to’ are unattested.


\begin{styleExampleTitle}
Figurative locational relation between figure and ground
\end{styleExampleTitle}

\begin{tabular}{lllll}
\lsptoprule
\label{bkm:Ref340310732}
\gll {sa} {\bluebold{di}} {\bluebold{IPS}} {\bluebold{satu}}\\ %
& \textsc{1sg} & at & social.sciences & one\\
\lspbottomrule
\end{tabular}
\ea
\glt 
[About course tracks in high school:] ‘I am \bluebold{in Social Sciences I}’ \textstyleExampleSource{[081023-004-Cv.0020]}
\z

\begin{tabular}{lllll}
\lsptoprule
\label{bkm:Ref340310736}
\gll {smua} {itu} {\bluebold{dari}} {\bluebold{uang}}\\ %
& all & \textsc{d.dist} & from & money\\
\lspbottomrule
\end{tabular}
\ea
\glt 
‘all that (depends) \bluebold{on the money}’ (Lit. ‘\bluebold{from the money}’) \textstyleExampleSource{[080927-006-CvNP.0034]}
\z


If speakers want to emphasize the predicate, they can front it. The corpus, however, includes only three utterances with fronted prepositional predicates, which are presented in (0) to (0). In each case the locative preposition \textitbf{di} ‘at, in’ is omitted and the complement is a locative, such as proximal \textitbf{sini} ‘\textsc{l.prox}’ in (0), or medial \textitbf{situ} ‘\textsc{l.med}’ in (0) and (0).\footnote{\\
\\
\\
\\
\\
\\
\\
\\
\\
\\
\\
\\
\\
\\
\\
\par One anonymous reviewers suggests an alternative reading of the \textitbf{situ}{}-clause. The clause \textitbf{situ alang-alang} ‘there (\textsc{emph}) (is only) cogongrass’ parallels the preceding clause \textitbf{sebla tida ada ruma} ‘on that side aren’t (any houses)’. Hence, the predicate is not \textitbf{situ} ‘\textsc{l.med}’ but \textitbf{ada} ‘exist’, with the latter having been elided: \textitbf{situ Ø alang-alang} ‘there (\textsc{emph}) (is only) cogongrass’.} Fronted prepositional predicates with distal \textitbf{sana} ‘\textsc{l.dist}’ are also possible, but unattested in the corpus. (For more details on the omission of prepositions encoding location, see §10.1.5.)


\begin{styleExampleTitle}
Fronting of prepositional predicates
\end{styleExampleTitle}

\begin{tabular}{lllll}
\lsptoprule
\label{bkm:Ref364331067}
\gll {\bluebold{Ø}} {\bluebold{sini}} {bua{\Tilde}bua} {banyak}\\ %
&  & \textsc{l.prox} & \textsc{rdp}{\Tilde}fruit & many\\
\lspbottomrule
\end{tabular}
\ea
\glt 
‘\bluebold{here (}\blueboldSmallCaps{emph}\bluebold{)} are many different kinds of fruit (trees)’ \textstyleExampleSource{[080922-001a-CvPh.0418]}\footnote{\\
\\
\\
\\
\\
\\
\\
\\
\\
\\
\\
\\
\\
\\
\\
\par Alternatively, the utterance in (0) could be interpreted as a numeral predicate clause with a locational adjunct, with \textitbf{bua{\Tilde}bua} ‘\textsc{rdp}{}-fruit’ as the subject, \textitbf{banyak} ‘many’ as the predicate, and \textitbf{sini} ‘\textsc{l.prox}’ as a preposed locational adjunct, giving the literal reading ‘here the various fruit (trees) are many’.}
\z

\begin{tabular}{llllllll}
\lsptoprule
\label{bkm:Ref364333002}\label{bkm:Ref340309680}
\gll {sebla} {tida} {ada} {ruma} {\bluebold{Ø}} {\bluebold{situ}} {alang-alang}\\ %
& side & \textsc{neg} & exist & house &  & \textsc{l.med} & cogongrass\\
\lspbottomrule
\end{tabular}
\ea
\glt 
‘on that side aren’t (any) houses, \bluebold{there (}\blueboldSmallCaps{emph}\bluebold{)} (is only) cogongrass’ \textstyleExampleSource{[081025-008-Cv.0149]}
\z

\begin{tabular}{lllllllllll}
\lsptoprule
\label{bkm:Ref338956153}
\gll {…} {\bluebold{Ø}} {\bluebold{situ}} {Natanael} {\bluebold{Ø}} {\bluebold{situ}} {Martin} {\bluebold{Ø}} {\bluebold{situ}} {Aleks}\\ %
&  &  & \textsc{l.med} & Natanael &  & \textsc{l.med} & Martin &  & \textsc{l.med} & Aleks\\
\lspbottomrule
\end{tabular}
\ea
\glt
[Choosing among potential candidates for the upcoming local elections:] ‘[Burwas (village can have) two candidates,] \bluebold{there (}\blueboldSmallCaps{emph}\bluebold{)} is Natanael, \bluebold{there (}\blueboldSmallCaps{emph}\bluebold{)} is Martin, \bluebold{there (}\blueboldSmallCaps{emph}\bluebold{)} is Aleks’ \textstyleExampleSource{[080919-001-Cv.0117]}
\end{styleFreeTranslEngxvpt}

\subsection{Nonlocational prepositional clauses}
\label{bkm:Ref262906841}
Clauses with nonlocational predicates convey information about the nonlocational, static relation between a figure and the ground. Semantically, Papuan Malay distinguishes three types of nonlocational predicates, namely “associative” or “comitative predicates”, “simulative predicates”, and “benefactive predicates”, adopting {Dryer’s (2007a: 248–249)} terminology. Overall, however, nonlocational prepositional clauses does not appear to be very common; the corpus contains only few examples.



Papuan Malay comitative predicates are formed with prepositions encoding accompaniment/instruments or goals, namely comitative \textitbf{dengang} ‘with’, with its short form \textitbf{deng}, and goal preposition \textitbf{sama} ‘to’ (see also §10.2.1 and §10.2.2). In (0), \textitbf{deng(ang)} ‘with’ denotes the accompaniment of the figure \textitbf{Roni} by the ground \textitbf{de pu temang{\Tilde}temang} ‘his friends’. In (0), \textitbf{sama} ‘to’ signals the association of the implied figure \textitbf{ana} ‘child’ with the ground \textitbf{saya} ‘1\textsc{sg}’.
\end{styleBodyvvafter}

\begin{styleExampleTitle}
Comitative predicates
\end{styleExampleTitle}

\begin{tabular}{lllllll}
\lsptoprule
\label{bkm:Ref262913665}
\gll {Roni} {masi} {\bluebold{deng}} {\bluebold{de}} {\bluebold{pu}} {\bluebold{temang{\Tilde}temang}}\\ %
& Roni & still & with & \textsc{3sg} & \textsc{poss} & \textsc{rdp}{\Tilde}friend\\
\lspbottomrule
\end{tabular}
\ea
\glt 
‘Roni is still \bluebold{with his friends}’ \textstyleExampleSource{[081006-031-Cv.0011]}
\z

\begin{tabular}{llllll}
\lsptoprule
\label{bkm:Ref340310741}
\gll {hanya} {tiga} {saja} {\bluebold{sama}} {\bluebold{saya}}\\ %
& only & three & just & to & \textsc{1sg}\\
\lspbottomrule
\end{tabular}
\ea
\glt 
‘just only three (of my children) are \bluebold{with me}’ \textstyleExampleSource{[081006-024-CvEx.0001]}
\z


Simulative predicates are formed with prepositions encoding comparisons, that is, similative \textitbf{sperti} ‘similar to’ and \textitbf{kaya} ‘like’ and equative \textitbf{sebagey} ‘as’ (see also §10.3). In (0) \textitbf{sperti} ‘similar to’ establishes a simulative relation between the figure \textitbf{de} ‘3\textsc{sg}’ and the ground \textitbf{Sofia}. Along similar lines, \textitbf{kaya} ‘like’ denotes a simulative relation between the figure \textitbf{de} ‘\textsc{sg}’ and the ground \textitbf{de pu bapa} ‘his father’ / \textitbf{Siduas} in (0). In (0), \textitbf{sebagey} ‘as’ expresses equatability between the figure \textitbf{sa} ‘\textsc{1sg}’ and the ground \textitbf{kepala acara} ‘as the head of the festivity’. (See §10.3.1 and §10.3.2 for a detailed discussion of the prepositions \textitbf{sperti} ‘similar to’ and \textitbf{kaya} ‘like’ and their semantics.)


\begin{styleExampleTitle}
Simulative predicates
\end{styleExampleTitle}

\begin{tabular}{llll}
\lsptoprule
\label{bkm:Ref340310743}
\gll {de} {\bluebold{sperti}} {\bluebold{Sofia}}\\ %
& \textsc{3sg} & similar.to & Sofia\\
\lspbottomrule
\end{tabular}
\ea
\glt 
‘she’s \bluebold{similar to Sofia}’ \textstyleExampleSource{[081115-001a-Cv.0283]}
\z

\begin{tabular}{lllllllllll}
\lsptoprule
\label{bkm:Ref439699708}
\gll {de} {pu} {muka} {\bluebold{kaya}} {de} {pu} {bapa} {e} {\bluebold{kaya}} {Siduas}\\ %
& \textsc{3sg} & \textsc{poss} & face & like & \textsc{3sg} & \textsc{poss} & father & eh & like & Siduas\\
\lspbottomrule
\end{tabular}
\ea
\glt 
‘his face is \bluebold{like} his father, eh, \bluebold{like} Siduas’ (face)’ \textstyleExampleSource{[080922-001a-CvPh.1446]}
\z

\begin{tabular}{lllllllll}
\lsptoprule
\label{bkm:Ref354853613}
\gll {paling} {sa} {tra} {kerja,} {sa} {\bluebold{sebagey}} {\bluebold{kepala}} {\bluebold{acara}}\\ %
& most & \textsc{1sg} & \textsc{neg} & work & \textsc{1sg} & as & head & festivity\\
\lspbottomrule
\end{tabular}
\ea
\glt 
[About organizing a festivity:] ‘most likely I won’t (have to) work, I’ll be \bluebold{the head of the festivity}’ (Lit. ‘\bluebold{as the head …}’) \textstyleExampleSource{[080919-004-NP.0068]}
\z


Benefactive predicates are formed with the benefactive preposition \textitbf{untuk} ‘for’ (see also §10.2.3). In (0), for instance, \textitbf{untuk} ‘for’ conveys a benefactive relation between the figure \textitbf{itu} ‘\textsc{d.dist}’ and the ground \textitbf{masarakat} ‘community’. In the corpus, however, benefactive predicates are rare.


\begin{styleExampleTitle}
Benefactive predicates
\end{styleExampleTitle}

\begin{tabular}{llllllllll}
\lsptoprule
\label{bkm:Ref340310744}
\gll {uang} {\multicolumn{3}{l}{besarnya}} {itu} {\bluebold{untuk}} {\bluebold{masarakat}} {tapi} {pejabat}\\ %
& money & \multicolumn{3}{l}{be.big:\textsc{ 3possr}} & \textsc{d.dist} & for & community & but & official\\
& \multicolumn{2}{l}{yang} & makang & \multicolumn{6}{l}{banyak}\\
& \multicolumn{2}{l}{\textsc{rel}} & eat & \multicolumn{6}{l}{many}\\
\lspbottomrule
\end{tabular}
\ea
\glt
‘most of that money, that’s \bluebold{for the community} but (it’s) the officials who take lots (of it)’ \textstyleExampleSource{[081029-004-Cv.0002]}
\end{styleFreeTranslEngxvpt}

\section{Summary}
\label{bkm:Ref272583275}
Papuan Malay employs three syntactically distinct types of nonverbal predicate clauses, namely nominal, numeral/quantifier, and prepositional predicate clauses. These clauses are formed by juxtaposition of the two main constituents; no copula intervenes. The three clause types also have distinct semantic functions. Nominal predicates have ascriptive or equative function and also encode possession. Numeral and quantifier predicates have determining function. Prepositional predicates encode locational or nonlocational relations between a figure and the ground.


%\setcounter{page}{1}\chapter[Negative, interrogative, and directive clauses]{Negative, interrogative, and directive clauses}
\label{bkm:Ref370831310}
This chapter describes negative, interrogative, and directive clauses in Papuan Malay. Negative clauses formed with the negators \textitbf{tida}/\textitbf{tra} ‘\textsc{neg}’ and \textitbf{bukang} ‘\textsc{neg}’ are discussed in §13.1. Interrogative clauses, including polar and alternative questions, are described in §13.2. Directive clauses, including imperatives, adhortatives, permissions, obligations, and prohibitives, are the topic of §13.3.
\end{styleBodyxvafter}

\section{Negative clauses}
\label{bkm:Ref363824846}
In Papuan Malay, negative clauses are formed with the negation adverbs \textitbf{tida}/\textitbf{tra} ‘\textsc{neg}’ or \textitbf{bukang} ‘\textsc{neg}’. Negator \textitbf{tida}/\textitbf{tra} ‘\textsc{neg}’ is used for the negation of verbal, existential, and nonverbal prepositional clauses (§13.1.1). Negator \textitbf{bukang} ‘\textsc{neg}’ is used to negate nonverbal clauses, other than prepositional ones, and to mark contrastive negation (§13.1.2). (Negative directives or prohibitives are discussed in §13.3.3.)
\end{styleBodyxvafter}

\subsection{Negation with \textitbf{tida}/\textitbf{tra} ‘\textsc{neg}’}
\label{bkm:Ref363560295}
The negators \textitbf{tida} ‘\textsc{neg}’ and \textitbf{tra} ‘\textsc{neg}’ negate different types of clauses; they always precede the predicate which they negate. Negation of verbal clauses is discussed in §13.1.1.1, of existential clauses in §13.1.1.2, and of nonverbal prepositional clauses in §13.1.1.3. Negator \textitbf{tida} ‘\textsc{neg}’ also provides negative responses to polar questions, as discussed in §13.1.1.4. With the exception of negative responses to polar questions, both negators are used interchangeably.
\end{styleBodyxvafter}

\paragraph[Negation of verbal clauses]{Negation of verbal clauses}
\label{bkm:Ref367461743}
As a negator of verbal clauses, \textitbf{tida}/\textitbf{tra} ‘\textsc{neg}’ negates stative verbs such as \textitbf{baik} ‘be good’ in (0), dynamic verbs such as \textitbf{datang} ‘come’ in (0), bivalent verbs such as \textitbf{pukul} ‘hit’ in (0), or trivalent verbs such as \textitbf{bli} ‘buy’ in (0). The example in (0) illustrates negation of a causative construction.



The contrastive examples in (0) and (0) also show that \textitbf{tida} ‘\textsc{neg}’ and \textitbf{tra} ‘\textsc{neg}’ are used interchangeably with no differences in function or meaning. In the corpus, however, speakers more often use \textitbf{tida} ‘\textsc{neg}’ than \textitbf{tra} ‘\textsc{neg}’ (1,491 vs. 794 tokens) (\textitbf{bukang} ‘\textsc{neg}’ is attested with 208 tokens; see §13.1.2).
\end{styleBodyvvafter}

\begin{styleExampleTitle}
Negation of verbal clauses with \textitbf{tida}/\textitbf{tra} ‘\textsc{neg}’
\end{styleExampleTitle}

\begin{tabular}{lllllllllllll}
\lsptoprule
\label{bkm:Ref363555095}
\gll {\multicolumn{2}{l}{nanti}} {\multicolumn{2}{l}{dia}} {\multicolumn{2}{l}{pikir}} {saya} {\bluebold{tida}} {\bluebold{baik}} {…} {nanti} {de}\\ %
& \multicolumn{2}{l}{very.soon} & \multicolumn{2}{l}{\textsc{3sg}} & \multicolumn{2}{l}{think} & \textsc{1sg} & \textsc{neg} & be.good &  & very.soon & \textsc{3sg}\\
& pikir & \multicolumn{2}{l}{kitong} & \multicolumn{2}{l}{\bluebold{tra}} & \multicolumn{7}{l}{\bluebold{baik}}\\
& think & \multicolumn{2}{l}{\textsc{1pl}} & \multicolumn{2}{l}{\textsc{neg}} & \multicolumn{7}{l}{be.good}\\
\lspbottomrule
\end{tabular}
\ea
\glt 
‘very soon he’ll think that I’m\bluebold{ not good}’ … very soon he’ll think that we are \bluebold{not good}’ \textstyleExampleSource{[080919-004-NP.0052-0053]}
\z

\begin{tabular}{llllllll}
\lsptoprule
\label{bkm:Ref358274063}
\gll {de} {\bluebold{tra}} {datang} {…} {de} {\bluebold{tida}} {datang}\\ %
& \textsc{3sg} & \textsc{neg} & come &  & \textsc{3sg} & \textsc{neg} & come\\
\lspbottomrule
\end{tabular}
\ea
\glt 
‘she did \bluebold{not} come … she did \bluebold{not} come’ \textstyleExampleSource{[081010-001-Cv.0204-0205]}
\z

\begin{tabular}{lllll}
\lsptoprule
\label{bkm:Ref367462010}
\gll {sa} {\bluebold{tida}} {\bluebold{pukul}} {dorang}\\ %
& \textsc{1sg} & \textsc{neg} & hit & \textsc{3pl}\\
\lspbottomrule
\end{tabular}
\ea
\glt 
‘I \bluebold{don’t hit} them’ \textstyleExampleSource{[080917-010-CvEx.0048]}
\z

\begin{tabular}{lllllllllll}
\lsptoprule
\label{bkm:Ref367462012}
\gll {kalo} {bapa} {\bluebold{tra}} {\bluebold{bli}} {sa} {HP,} {biar} {suda} {tida} {apa{\Tilde}apa}\\ %
& if & father & \textsc{neg} & buy & \textsc{1sg} & cell.phone & let & already & \textsc{neg} & \textsc{rdp}{\Tilde}what\\
\lspbottomrule
\end{tabular}
\ea
\glt 
‘if you (‘father’) \bluebold{won’t buy} me a cell phone, just let it be, no problem’ \textstyleExampleSource{[080922-001a-CvPh.0461]}
\z

\begin{tabular}{llllllll}
\lsptoprule
\label{bkm:Ref367462013}
\gll {baru} {kamu} {\bluebold{tra}} {\bluebold{kas}} {\bluebold{kluar}} {uang} {bayar}\\ %
& and.then & \textsc{2pl} & \textsc{neg} & give & go.out & money & pay\\
\lspbottomrule
\end{tabular}
\ea
\glt
[Encouraging teenagers to take gratis English classes:] ‘and then you \bluebold{won’t have to pay} fees’ (Lit. ‘\bluebold{not cause to come out}’) \textstyleExampleSource{[081115-001a-Cv.0160]}
\end{styleFreeTranslEngxvpt}

\paragraph[Negation of existential clauses]{Negation of existential clauses}
\label{bkm:Ref367461744}
Existential clauses are also negated with \textitbf{tida}/\textitbf{tra} ‘\textsc{neg}’, as illustrated in (0) to (0). Examples for negated one-argument clauses are given in (0) to (0), and for two-argument clauses in (0) to (0). (Existential clauses are discussed in detail in §11.4.)



The respective one-argument clauses in (0) and (0) illustrate negated existence and negated availability of indefinite/nonidentifiable theme expressions. The example in (0) demonstrates negation of a definite/identifiable theme expression. One-argument clauses denoting negative possession of a definite/identifiable possessum are unattested in the corpus; instead, the preferred type of existential clause to express negative possession are two-argument clauses, as shown in (0) and (0).
\end{styleBodyvvafter}

\begin{styleExampleTitle}
Negation of one-argument existential clauses with \textitbf{tida}/\textitbf{tra} ‘\textsc{neg}’
\end{styleExampleTitle}

\begin{tabular}{lllll}
\lsptoprule
\label{bkm:Ref339731772}
\gll {\bluebold{tra}} {\bluebold{ada}} {kamar} {mandi}\\ %
& \textsc{neg} & exist & room & bathe\\
\lspbottomrule
\end{tabular}
\ea
\glt 
‘(there) \bluebold{weren’t} (any) bathrooms’ \textstyleExampleSource{[081025-009a-Cv.0059]}
\z

\begin{tabular}{lllll}
\lsptoprule
\label{bkm:Ref339731773}
\gll {\bluebold{tida}} {\bluebold{ada}} {air} {minum}\\ %
& \textsc{neg} & exist & water & drink\\
\lspbottomrule
\end{tabular}
\ea
\glt 
‘(there) \bluebold{was no} drinking water’ \textstyleExampleSource{[081025-009a-Cv.0060]}
\z

\begin{tabular}{lllll}
\lsptoprule
\label{bkm:Ref339731774}
\gll {ketrampilang} {juga} {\bluebold{tra}} {\bluebold{ada}}\\ %
& skill & also & \textsc{neg} & exist\\
\lspbottomrule
\end{tabular}
\ea
\glt 
[About training activities for women:] ‘\bluebold{neither} \bluebold{do} (they) \bluebold{have} (any) skills’ (Lit. ‘(their) skills also \bluebold{don’t exist}’) \textstyleExampleSource{[081010-001-Cv.0145]}
\z


Negation of two-argument existential clauses is shown in (0) and (0). The negated clauses attested in the corpus always express absence of possession, such as the negative possession of \textitbf{ana} ‘child(ren)’ in (0), or \textitbf{air} ‘water’ in (0).


\begin{styleExampleTitle}
Negation of two-argument existential clauses denoting possession
\end{styleExampleTitle}

\begin{tabular}{lllllll}
\lsptoprule
\label{bkm:Ref339731776}
\gll {sodara} {prempuang} {itu} {\bluebold{tida}} {\bluebold{ada}} {ana}\\ %
& sibling & woman & \textsc{d.dist} & \textsc{neg} & exist & child\\
\lspbottomrule
\end{tabular}
\ea
\glt 
‘that sister \bluebold{doesn’t have} children’ \textstyleExampleSource{[081006-024-CvEx.0005]}
\z

\begin{tabular}{lllll}
\lsptoprule
\label{bkm:Ref339731777}
\gll {dong} {\bluebold{tra}} {\bluebold{ada}} {air}\\ %
& \textsc{3pl} & \textsc{neg} & exist & water\\
\lspbottomrule
\end{tabular}
\ea
\glt
‘they \bluebold{didn’t have} water’ \textstyleExampleSource{[080919-008-CvNP.0013]}
\end{styleFreeTranslEngxvpt}

\paragraph[Negation of prepositional predicate clauses]{Negation of prepositional predicate clauses}
\label{bkm:Ref367461745}
Prepositional predicates are also negated with \textitbf{tida}/\textitbf{tra} ‘\textsc{neg}’, as shown in (0) and (0). (Negation of other types of nonverbal clauses is discussed in §13.1.2; for more details on nonverbal clauses see Chapter 12.)


\begin{styleExampleTitle}
Negation of nonverbal prepositional clauses with \textitbf{tida}/\textitbf{tra} ‘\textsc{neg}’
\end{styleExampleTitle}

\begin{tabular}{lllllll}
\lsptoprule
\label{bkm:Ref363551833}
\gll {saya} {\bluebold{tida}} {sperti} {prempuang} {laing} {to?}\\ %
& \textsc{1sg} & \textsc{neg} & similar.to & woman & be.different & right?\\
\lspbottomrule
\end{tabular}
\ea
\glt 
‘I’m \bluebold{not} like other women, right?’ \textstyleExampleSource{[081011-023-Cv.0173]}
\z

\begin{tabular}{lllll}
\lsptoprule
\label{bkm:Ref363551830}
\gll {tong} {\bluebold{tra}} {ke} {kampung}\\ %
& \textsc{1pl} & \textsc{neg} & to & village\\
\lspbottomrule
\end{tabular}
\ea
\glt
‘we do \bluebold{not} (go) to the village’ \textstyleExampleSource{[080917-003a-CvEx.0048]}
\end{styleFreeTranslEngxvpt}

\paragraph[Negation of polar questions]{Negation of polar questions}
\label{bkm:Ref367461746}
In addition, \textitbf{tida}/\textitbf{tra} ‘\textsc{neg}’ provides negative responses to polar questions, when negating verbal constructions, as shown in (0) and (0). Negator \textitbf{tida} ‘\textsc{neg}’ can stand alone as in (0), or it can occur in the negative existential phrase \textitbf{tida ada} ‘no’ (literally ‘(it) doesn’t exist’). Negator \textitbf{tra} ‘\textsc{neg}’, by contrast, cannot stand alone; it always occurs in the negative existential phrase \textitbf{tra ada} ‘no’, as in (0). (See also §13.2.2.1.)


\begin{styleExampleTitle}
Negator \textitbf{tida}/\textitbf{tra} ‘\textsc{neg}’ in responses to polar questions
\end{styleExampleTitle}

\begin{tabular}{lllll}
\lsptoprule
\label{bkm:Ref358290149}
\gll {Speaker-2:} {\bluebold{tida},} {dia} {balap}\\ %
&  & \textsc{neg} & \textsc{3sg} & race\\
\lspbottomrule
\end{tabular}
\ea
\glt 
[About an accident:] [Speaker-1: ‘what did he do? (was he) drunk?’]\\
Speaker-2: ‘\bluebold{no}!, he was racing (his motorbike)’ \textstyleExampleSource{[081014-013-NP.0003-0004]}
\z

\begin{tabular}{lllllll}
\lsptoprule
\label{bkm:Ref358290151}
\gll {Speaker-2:} {\bluebold{tra}} {\bluebold{ada},} {muara} {baru} {…}\\ %
&  & \textsc{neg} & exist & river.mouth & be.new & \\
\lspbottomrule
\end{tabular}
\ea
\glt
[Discussing the depth of a river mouth:] [Speaker-1: ‘isn’t (it) deep?’]\\
Speaker-2: ‘\bluebold{no}!, (this is) the new river mouth [(it’s) the old river mouth that is (deep)]’ \textstyleExampleSource{[080927-003-Cv.0010-0011]}
\end{styleFreeTranslEngxvpt}

\subsection{Negation with \textitbf{bukang} ‘\textsc{neg}’}
\label{bkm:Ref363560297}
Negator \textitbf{bukang} ‘\textsc{neg}’ has three functions. One function is to negate nonverbal clauses, a second one is to mark contrastive negation, and a third function is to provide negative responses to polar questions.



Nonverbal clauses are typically negated with \textitbf{bukang} ‘\textsc{neg}’, which always precedes the nonverbal predicate. Prepositional predicates are the exception; they are negated with \textitbf{tida}/\textitbf{tra} ‘\textsc{neg}’. (In the corpus, \textitbf{bukang} ‘\textsc{neg}’ is attested with 208 tokens, as compared to 1,491 \textitbf{tida} ‘\textsc{neg}’ and 794 \textitbf{tra} ‘\textsc{neg}’ tokens; see §13.1.1.3).
\end{styleBodyvafter}


In (0) and (0), \textitbf{bukang} ‘\textsc{neg}’ negates nominal predicates, and in (0) a quantifier predicate. (Nonverbal clauses are discussed in detail in Chapter 12.)
\end{styleBodyvvafter}

\begin{styleExampleTitle}
Negation of nonverbal clauses with \textitbf{bukang} ‘\textsc{neg}’
\end{styleExampleTitle}

\begin{tabular}{lllllll}
\lsptoprule
\label{bkm:Ref340310745}
\gll {de} {\bluebold{bukang}} {gembala} {sidang} {di} {situ}\\ %
& \textsc{3sg} & \textsc{neg} & pastor & (church.)gathering & at & \textsc{l.med}\\
\lspbottomrule
\end{tabular}
\ea
\glt 
‘he’s \bluebold{not} a congregational pastor there’ \textstyleExampleSource{[080925-003-Cv.0032]}
\z

\begin{tabular}{llllll}
\lsptoprule
\label{bkm:Ref363551828}
\gll {sa} {\bluebold{bukang}} {orang} {yang} {seraka}\\ %
& \textsc{1sg} & \textsc{neg} & person & \textsc{rel} & be.greedy\\
\lspbottomrule
\end{tabular}
\ea
\glt 
‘I’m \bluebold{not} a person who is greedy’ \textstyleExampleSource{[080917-010-CvEx.0214]}
\z

\begin{tabular}{llll}
\lsptoprule
\label{bkm:Ref340310750}
\gll {pisang} {\bluebold{bukang}} {sedikit}\\ %
& banana & \textsc{neg} & few\\
\lspbottomrule
\end{tabular}
\ea
\glt 
‘there (were) \bluebold{quite a few} bananas’ (Lit. ‘the bananas (were) \bluebold{not} few’) \textstyleExampleSource{[080925-003-Cv.0158]}
\z


A second function of \textitbf{bukang} ‘\textsc{neg}’ is to express contrastive negation of an entire proposition. Contrastive negation implies an alternative in the sense of ‘the situation is not that X (but Y)’. Very often the alternative is expressed overtly, but this is not obligatory. Depending on its scope, \textitbf{bukang} ‘\textsc{neg}’ occurs between the subject and the predicate or clause-initially. Its contrastive uses in pre-predicate position are shown with the examples in (0) and (0). Unlike \textitbf{tida}/\textitbf{tra} ‘\textsc{neg}’, contrastive \textitbf{bukang} ‘\textsc{neg}’ also occurs clause-initially, as shown in (0) and (0).


\begin{styleExampleTitle}
Contrastive negation with \textitbf{bukang} ‘\textsc{neg}’
\end{styleExampleTitle}

\begin{tabular}{lllllllllllll}
\lsptoprule
\label{bkm:Ref358281678}
\gll {mama} {ni} {\multicolumn{2}{l}{\bluebold{bukang}}} {\multicolumn{2}{l}{hidup}} {\multicolumn{2}{l}{deng}} {\multicolumn{2}{l}{orang-tua}} {di} {kampung,}\\ %
& mother & \textsc{d.prox} & \multicolumn{2}{l}{\textsc{neg}} & \multicolumn{2}{l}{live} & \multicolumn{2}{l}{with} & \multicolumn{2}{l}{parent} & at & village\\
& mama & ni & hidup & \multicolumn{2}{l}{deng} & \multicolumn{2}{l}{orang} & \multicolumn{2}{l}{di} & \multicolumn{3}{l}{luar}\\
& mother & \textsc{d.prox} & live & \multicolumn{2}{l}{with} & \multicolumn{2}{l}{person} & \multicolumn{2}{l}{at} & \multicolumn{3}{l}{outside}\\
\lspbottomrule
\end{tabular}
\ea
\glt 
‘(the situation was) \bluebold{not} (that) I (‘mother’) here lived with (my) parents in the village, (but) I (‘mother’) here lived with strangers away from home’ \textstyleExampleSource{[081115-001b-Cv.0043]}
\z

\begin{tabular}{lllllllll}
\lsptoprule
\label{bkm:Ref363551834}
\gll {pernikaang} {ini} {\bluebold{bukang}} {dari} {manusia,} {dari} {Tuhang} {to?}\\ %
& marriage & \textsc{d.prox} & \textsc{neg} & from & human.being & from & God & right?\\
\lspbottomrule
\end{tabular}
\ea
\glt 
‘(the situation is) \bluebold{not} (that) marriage is from man, (but it is) from God, right?’ \textstyleExampleSource{[081110-006-CvEx.0239]}
\z

\begin{tabular}{llllllll}
\lsptoprule
\label{bkm:Ref358281677}
\gll {\bluebold{bukang}} {dong} {maing,} {dong} {taguling} {di} {pecek}\\ %
& \textsc{neg} & \textsc{3pl} & play & \textsc{3pl} & be.rolled.over & at & mud\\
\lspbottomrule
\end{tabular}
\ea
\glt 
‘(the situation was) \bluebold{not} (that) they played (football, but) they got rolled over in the mud’ \textstyleExampleSource{[081109-001-Cv.0025]}
\z

\begin{tabular}{lllllllll}
\lsptoprule
\label{bkm:Ref363570061}
\gll {\bluebold{bukang}} {dong} {taru} {ijing} {tapi} {dong} {taru} {hadir}\\ %
& \textsc{neg} & \textsc{3pl} & put & permission & but & \textsc{3pl} & put & attend\\
\lspbottomrule
\end{tabular}
\ea
\glt 
[About students who falsified the attendance book:] ‘(the situation is) \bluebold{not} (that) they wrote down (their absences as) permitted (absences), but they wrote (them) down as (having) attended’ \textstyleExampleSource{[081023-004-Cv.0018]}
\z


This function of \textitbf{bukang} ‘\textsc{neg}’ to signal contrastive negation has also been noted for Ambon Malay {(van Minde 1997: 278–279)}, Manado Malay {\citep[59]{Stoel2005}}, Ternate Malay {(Litamahuputty 1994: 224–225)}, and Standard Malay and Standard Indonesian ({Himmelmann 2005: 127}{;} {Kroeger 2012}).



Speakers also use \textitbf{bukang} ‘\textsc{neg}’ in single word clauses to contradict an interlocutor’s statements. They may submit an alternative to the negated proposition as in (0), or they may reply with bare \textitbf{bukang} ‘\textsc{neg}’.
\end{styleBodyvvafter}

\begin{styleExampleTitle}
Contradiction of an interlocutor’s statements with \textitbf{bukang} ‘\textsc{neg}’
\end{styleExampleTitle}

\begin{tabular}{llllll}
\lsptoprule
\label{bkm:Ref372450505}
\gll {Speaker-2:} {\bluebold{bukang},} {de} {punya} {pacar}\\ %
&  & \textsc{neg} & \textsc{3sg} & \textsc{poss} & date/lover\\
\lspbottomrule
\end{tabular}
\ea
\glt 
[Speaker-1: ‘(it was) her husband!’]\\
Speaker-2: ‘\bluebold{no}, (it was) her lover’ \textstyleExampleSource{[081006-022-CvEx.0043-0045]}
\z


Finally, speakers employ \textitbf{bukang} ‘\textsc{neg}’ to give contrastive negative responses to polar questions, as in the elicited example in (0). This example contrasts with the one in (0) in which the speaker uses \textitbf{tida} ‘\textsc{neg}’ to respond to the same question as in (0). While \textitbf{tida} ‘\textsc{neg}’ in (0) merely negates a verbal construction, \textitbf{bukang} ‘\textsc{neg}’ in (0) marks contrastive negation, similar to its uses in (0) to (0). Again, speakers can add the correct response as in (0) or reply with bare \textitbf{bukang} ‘\textsc{neg}’. (For more details on polar questions see §13.2.2.)


\begin{styleExampleTitle}
Contrastive uses of \textitbf{bukang} ‘\textsc{neg}’ in responses to polar questions
\end{styleExampleTitle}

\begin{tabular}{lllll}
\lsptoprule
\label{bkm:Ref358289359}
\gll {Speaker-2:} {\bluebold{bukang}!,} {dia} {balap}\\ %
&  & \textsc{neg} & \textsc{3sg} & race\\
\lspbottomrule
\end{tabular}
\ea
\glt
[About an accident:] [Speaker-1: ‘what did he do? (was he) drunk?’]\\
Speaker-2: ‘\bluebold{no}!, (it happened because) he was racing (his motorbike)’ \textstyleExampleSource{[Elicited MY131126.001]}
\end{styleFreeTranslEngxvpt}

\section{Interrogative clauses}
\label{bkm:Ref363824847}
In Papuan Malay, three types of interrogative clauses can be distinguished: (1) content, or information questions which elicit new information (§13.2.1), (2) polar questions which elicit yes-no answers (§13.2.2), and (3) alternative questions which require the interlocutor to choose the supposedly right answer from a list of possible answers (§13.2.3).
\end{styleBodyxvafter}

\subsection{Content questions}
\label{bkm:Ref364091899}
In Papuan Malay, content questions eliciting new information are formed with the interrogatives discussed in §5.8. The description of their positions and functions within the clause entails a description of content questions. Therefore, content questions are not further discussed here.
\end{styleBodyxvafter}

\subsection{Polar questions}
\label{bkm:Ref364091900}
Papuan Malay polar questions, that is questions that elicit yes-no answers, can be unmarked and neutral, or marked and biased, as shown in §13.2.2.1 and §13.2.2.2, respectively. Both sections also describe how polar questions are answered.
\end{styleBodyxvafter}

\paragraph[Unmarked neutral polar questions]{Unmarked neutral polar questions}
\label{bkm:Ref364073527}
Generally speaking, unmarked polar questions are “neutral with respect to the answer the speaker expects” {(Sadock and Zwicky 1985: 179)}. That is, neutral questions do not indicate whether speakers would like their interlocutors to answer with ‘yes’ or with ‘no’. More specifically, polar questions can express positive polarity or negative polarity. A negative polar question differs “from the positive question in communicating […] that the speaker already has his own opinion, but that he is interested in getting the hearer’s reaction” {\citep[67]{Grimes1975}}.



These observations also apply to Papuan Malay, as demonstrated in the examples in (0) to (0).
\end{styleBodyvafter}


Syntactically, the examples show that neutral polar questions have the same structure as the corresponding declarative clauses. The only distinction between the two clause types is that polar questions are marked with the rising intonation pattern typical for interrogatives, as shown in (0).
\end{styleBodyvafter}


The examples also show that polar questions can express positive polarity as in (0), (0), (0), and (0), or negative polarity as in (0), (0), and (0).
\end{styleBodyvafter}


Furthermore, the examples in (0) to (0) show how neutral polar questions are answered. Polar questions with positive answers are presented in (0) to (0), and those with negative answers in (0) to (0). An alternative strategy to answer polar questions is illustrated in (0).
\end{styleBodyvafter}


Positive answers to polar questions are typically formed with affirmative \textitbf{yo} ‘yes’ or the interjection \textitbf{mm-mm} ‘mhm’. This applies to positive questions, as in (0) and (0), as well as to negative ones, as in (0). In answering, speakers may also echo part of the question and/or provide additional information, as in (0) and (0).
\end{styleBodyvvafter}

\begin{styleExampleTitle}
Polar questions: Positive answers
\end{styleExampleTitle}

\begin{tabular}{lllllll}
\lsptoprule
\label{bkm:Ref358218262}\label{bkm:Ref363916887}
\gll {\label{bkm:Ref363916943}} {} {\textstyleChBold{{}---{}---}} {\textstyleChBold{{}---{}---}} {\textstyleChBold{{}---{}---}} {\textstyleChBold{\textsuperscript{{}---}}\textstyleChBold{\textsubscript{{}---}}}\\ %
&  & Speaker-1: & trek & de & isi & minyak?\\
&  &  & truck & \textsc{3sg} & fill & oil\\
\lspbottomrule
\end{tabular}
\begin{styleFreeTranslAlphaEng}
Speaker-1: ‘does the truck load gasoline?’
\end{styleFreeTranslAlphaEng}

\begin{tabular}{llllll} & \label{bkm:Ref356583715} & Speaker-2: & \bluebold{yo,} & minyak & tana\\
\lsptoprule
&  &  & yes & oil & ground\\
\lspbottomrule
\end{tabular}
\begin{styleFreeTranslAlphaEng}
Speaker-2: ‘\bluebold{yes}!, kerosene’ \textstyleExampleSource{[080923-009-Cv.0037-0038]}
\end{styleFreeTranslAlphaEng}

\begin{tabular}{llllll}
\lsptoprule
\label{bkm:Ref363916873}
\gll { & Speaker-1: & o, & Ise & sakit?}\\ %
&  &  & oh! & Ise & be.sick\\
\lspbottomrule
\end{tabular}
\begin{styleFreeTranslAlphaEng}
Speaker-1: ‘oh, is Ise sick?’
\end{styleFreeTranslAlphaEng}

\begin{tabular}{llll} &  & Speaker-2: & \bluebold{mm-mm}\\
\lsptoprule
&  &  & mhm\\
\lspbottomrule
\end{tabular}
\begin{styleFreeTranslAlphaEng}
Speaker-1: ‘\bluebold{mhm}!’ \textstyleExampleSource{[080919-006-CvNP.0030-0031]}
\end{styleFreeTranslAlphaEng}

\begin{tabular}{lllllllll}
\lsptoprule
\label{bkm:Ref363916877}\label{bkm:Ref363916891}
\gll { & Speaker-1: & ade & hari & ini & ko & tra & skola?}\\ %
&  &  & ySb & day & \textsc{d.prox} & \textsc{2sg} & \textsc{neg} & go.to.school\\
\lspbottomrule
\end{tabular}
\begin{styleFreeTranslAlphaEng}
Speaker-1: ‘younger sister, don’t you go to school today?’
\end{styleFreeTranslAlphaEng}

\begin{tabular}{lllllll} & \label{bkm:Ref363916955} & Speaker-2: & \bluebold{yo}, & sa & minta & ijing\\
\lsptoprule
&  &  & yes & \textsc{1sg} & request & permission\\
\lspbottomrule
\end{tabular}
\begin{styleFreeTranslAlphaEng}
Speaker-2: ‘\bluebold{yes}, I asked for a leave of absence’ (Lit. ‘request permission (to be absent from school)’) \textstyleExampleSource{[080922-001a-CvPh.0093-0094]}
\end{styleFreeTranslAlphaEng}


Negative answers to neutral positive or negative polar questions are formed in three ways, as discussed in §13.1.1.4 and §13.1.2. Negative replies to polar questions are formed with \textitbf{tida} ‘\textsc{neg}’ as shown in (0), repeated as (0), or with the negative existential phrase \textitbf{tida}/\textitbf{tra ada} ‘(it) doesn’t exist’, as in (0), repeated as (0), when negating verbal constructions. Negative answers to polar questions are formed with \textitbf{bukang} ‘\textsc{neg}’, as in (0), repeated as (0), when negating nonverbal constructions.


\begin{styleExampleTitle}
Polar questions: Negative answers
\end{styleExampleTitle}

\begin{tabular}{lllllll}
\lsptoprule
\label{bkm:Ref363916874}
\gll { & Speaker-1: & dia & biking & apa? & mabuk?}\\ %
&  &  & \textsc{3sg} & make & what & be.drunk\\
\lspbottomrule
\end{tabular}
\begin{styleFreeTranslAlphaEng}
[About an accident:] Speaker-1: ‘what did he do?, (was he) drunk?’
\end{styleFreeTranslAlphaEng}

\begin{tabular}{llllll} &  & Speaker-2: & \bluebold{tida}, & dia & balap\\
\lsptoprule
&  &  & \textsc{neg} & \textsc{3sg} & race\\
\lspbottomrule
\end{tabular}
\begin{styleFreeTranslAlphaEng}
Speaker-2: ‘\bluebold{no}!, he raced (his motorbike)’ \textstyleExampleSource{[081014-013-NP.0003-0004]}
\end{styleFreeTranslAlphaEng}

\begin{tabular}{lllll}
\lsptoprule
\label{bkm:Ref363916879}
\gll { & Speaker-1: & tra & dalam?}\\ %
&  &  & \textsc{neg} & inside\\
\lspbottomrule
\end{tabular}
\begin{styleFreeTranslAlphaEng}
[Discussing the depth of a river mouth:] Speaker-1: ‘isn’t (it) deep?’
\end{styleFreeTranslAlphaEng}

\begin{tabular}{llllllll} &  & Speaker-2: & \bluebold{tra} & \bluebold{ada}, & muara & baru & …\\
\lsptoprule
&  &  & \textsc{neg} & exist & river.mouth & be.new & \\
\lspbottomrule
\end{tabular}
\begin{styleFreeTranslAlphaEng}
Speaker-2: ‘\bluebold{no}!, (this is) the new river mouth [(it’s) the old river mouth that is (deep)]’ \textstyleExampleSource{[080927-003-Cv.0010-0011]}
\end{styleFreeTranslAlphaEng}

\begin{tabular}{llllll}
\lsptoprule
\label{bkm:Ref372457067}
\gll { & Speaker-1: & de & punya & paytua?}\\ %
&  &  & \textsc{3sg} & \textsc{poss} & husband\\
\lspbottomrule
\end{tabular}
\begin{styleFreeTranslAlphaEng}
Speaker-1: ‘(was it) her husband?’
\end{styleFreeTranslAlphaEng}

\begin{tabular}{lllllll} &  & Speaker-2: & \bluebold{bukang}, & de & punya & pacar\\
\lsptoprule
&  &  & \textsc{neg} & \textsc{3sg} & \textsc{poss} & date/lover\\
\lspbottomrule
\end{tabular}
\begin{styleFreeTranslAlphaEng}
Speaker-2: ‘\bluebold{no}, (it was) her lover’ \textstyleExampleSource{[081006-022-CvEx.0044-0045]}
\end{styleFreeTranslAlphaEng}


At times, speakers employ an alternative strategy to respond to polar questions as shown in (0). Speakers may reply to a polar question without giving an explicit answer in the affirmative or negative. Instead they provide additional information and leave it to their interlocutor to interpret this answer as a positive or a negative reply. This is shown with the implied negative answer in (0). When interlocutors do not know the answer, they typically reply with \textitbf{tida}/\textitbf{tra taw} ‘(I) don’t know’.


\begin{styleExampleTitle}
Alternative answers to polar questions
\end{styleExampleTitle}

\begin{tabular}{lllllll}
\lsptoprule
\label{bkm:Ref364071167}\label{bkm:Ref364071172}
\gll { & Speaker-1: & di & sini & tra & pahit?}\\ %
&  &  & at & \textsc{l.prox} & \textsc{neg} & be.bitter\\
\lspbottomrule
\end{tabular}
\begin{styleFreeTranslAlphaEng}
[Discussing various melinjo varieties] Speaker-1: ‘(the melinjo varieties) here are not bitter?’
\end{styleFreeTranslAlphaEng}

\begin{tabular}{lllllll} & \label{bkm:Ref364071150} & Speaker-2: & \bluebold{${\varnothing}$}, & Jayapura & pu & pahit\\
\lsptoprule
&  &  &  & Jayapura & \textsc{poss} & be.bitter\\
\lspbottomrule
\end{tabular}
\begin{styleFreeTranslAlphaEngxxpt}
Speaker-2: ‘(\bluebold{no}!, the ones from) Jayapura are bitter’ \textstyleExampleSource{[080923-004-Cv.0011-0012]}
\end{styleFreeTranslAlphaEngxxpt}

\paragraph[Marked biased polar questions]{Marked biased polar questions}
\label{bkm:Ref364073528}
Marked polar questions are defined as questions which convey a bias toward the expected answer, hence “biased” questions {(Moravcsik 1971} in {Sadock and Zwicky 1985: 180)}. Biased questions allow speakers “to express [… their] belief that a particular answer is likely to be correct and to request assurance that this belief is true” {(Sadock and Zwicky 1985: 180)}. More specifically, positively biased questions signal that the speaker is in favor of a positive answer, while negatively biased questions indicate that the speaker expects a negative answer. 



Papuan Malay biased questions are presented in (0) to (0). While the corpus contains both positively and negatively biased questions, positively biased ones, as in (0) to (0), occur much more often than negatively biased ones, as in (0) or (0).
\end{styleBodyvafter}


Biased questions are usually formed with the tags \textitbf{to} ‘right?’ or \textitbf{e} ‘eh?’. Prosodically, these questions are marked with a rising pitch on the tag (see §5.13.1 for more details concerning the semantics of both tags). The examples in (0) and (0) show positive bias, while (0) and (0) show negative bias, using the negator \textitbf{tida}/\textitbf{tra} ‘\textsc{neg}’. Less often, a positive bias is marked with affirmative \textitbf{yo} ‘yes’ as in (0). Answers to biased polar questions follow the same patterns as answers to unbiased ones, as discussed in §13.2.2.1.
\end{styleBodyvvafter}

\begin{styleExampleTitle}
Positively biased polar questions
\end{styleExampleTitle}

\begin{tabular}{llllllll}
\lsptoprule
\label{bkm:Ref363931009}\label{bkm:Ref363931019}
\gll {\label{bkm:Ref363930983}} {Speaker-1:} {yang} {dekat} {ada} {ruma} {\bluebold{to}?}\\ %
&  &  & \textsc{rel} & near & exist & house & right?\\
\lspbottomrule
\end{tabular}
\begin{styleFreeTranslAlphaEng}
[Asking about a certain tree:] Speaker-1: ‘(the one that’s) close by (where) the houses are, \bluebold{right?}’
\end{styleFreeTranslAlphaEng}

\begin{tabular}{lllllll} &  & Speaker-2: & \bluebold{mm-mm}, & ruma & di & pante\\
\lsptoprule
&  &  & mhm & house & at & coast\\
\lspbottomrule
\end{tabular}
\begin{styleFreeTranslAlphaEng}
Speaker-2: ‘\bluebold{mhm}, the houses along the beach’ \textstyleExampleSource{[080917-009-CvEx.0012-0013]}
\end{styleFreeTranslAlphaEng}

\begin{tabular}{lllllllll}
\lsptoprule
\label{bkm:Ref363931022}
\gll {\label{bkm:Ref363930987}} {Speaker-1:} {o,} {skarang} {orang} {su} {daftar} {\bluebold{e}?}\\ %
&  &  & oh! & now & person & already & enroll & eh\\
\lspbottomrule
\end{tabular}
\begin{styleFreeTranslAlphaEng}
[About local elections:] Speaker-1: ‘oh, now people already (started) enrolling, \bluebold{eh?}’
\end{styleFreeTranslAlphaEng}

\begin{tabular}{llllllll} &  & Speaker-2: & \bluebold{yo}, & tu & sa & pu & urusang\\
\lsptoprule
&  &  & yes & \textsc{d.dist} & \textsc{1sg} & \textsc{poss} & affairs\\
\lspbottomrule
\end{tabular}
\begin{styleFreeTranslAlphaEng}
Speaker-2: ‘\bluebold{yes}!, that’s my responsibility’ \textstyleExampleSource{[081005-001-Cv.0031-0032]}
\end{styleFreeTranslAlphaEng}

\begin{tabular}{lllllllllll}
\lsptoprule
\label{bkm:Ref363925855}\label{bkm:Ref363925859}
\gll {\label{bkm:Ref363925846}} {Speaker-1:} {jadi} {\multicolumn{2}{l}{itu}} {\multicolumn{2}{l}{nomor}} {\multicolumn{2}{l}{rekening}} {itu}\\ %
&  &  & so & \multicolumn{2}{l}{\textsc{d.dist}} & \multicolumn{2}{l}{number} & \multicolumn{2}{l}{bank.account} & \textsc{d.dist}\\
&  &  & \multicolumn{2}{l}{pace} & \multicolumn{2}{l}{Natanael} & \multicolumn{2}{l}{punya} & \multicolumn{2}{l}{\bluebold{yo}?}\\
&  &  & \multicolumn{2}{l}{man} & \multicolumn{2}{l}{Natanael} & \multicolumn{2}{l}{\textsc{poss}} & \multicolumn{2}{l}{yes}\\
\lspbottomrule
\end{tabular}
\begin{styleFreeTranslAlphaEng}
Speaker-1: ‘so, what’s-its-name, that bank account number is Mr. Natanael’s, \bluebold{yes}?’
\end{styleFreeTranslAlphaEng}

\begin{tabular}{lllllll} &  & Speaker-2: & \bluebold{yo}, & bukang & sa & punya\\
\lsptoprule
&  &  & yes & \textsc{neg} & \textsc{1sg} & \textsc{poss}\\
\lspbottomrule
\end{tabular}
\begin{styleFreeTranslAlphaEng}
Speaker-2: ‘\bluebold{yes}!, (it’s) not mine’ \textstyleExampleSource{[080922-001a-CvPh.0078-0079]}
\end{styleFreeTranslAlphaEng}

\begin{styleExampleTitle}
Negatively biased polar questions
\end{styleExampleTitle}

\begin{tabular}{lllllllllll}
\lsptoprule
\label{bkm:Ref363931014}\label{bkm:Ref363931024}
\gll {\label{bkm:Ref363931142}} {Speaker-1:} {ko} {\bluebold{tra}} {taw} {sa} {skola} {dari} {mana} {\bluebold{to}?}\\ %
&  &  & \textsc{2sg} & \textsc{neg} & know & \textsc{1sg} & school & from & where & right?\\
\lspbottomrule
\end{tabular}
\begin{styleFreeTranslAlphaEng}
Speaker-1: ‘you don’t know from which school I am, \bluebold{right?}’
\end{styleFreeTranslAlphaEng}

\begin{tabular}{llllllll} &  & Speaker-2: & \bluebold{${\varnothing}$} & sa & \bluebold{tida} & \bluebold{taw} & ((laughter))\\
\lsptoprule
&  &  &  & \textsc{1sg} & \textsc{neg} & know & \\
\lspbottomrule
\end{tabular}
\begin{styleFreeTranslAlphaEng}
Speaker-2: ‘(\bluebold{yes}!), I \bluebold{don’t know}! ((laughter))’ \textstyleExampleSource{[080922-003-Cv.0031-0032]}
\end{styleFreeTranslAlphaEng}

\begin{tabular}{lllllll}
\lsptoprule
\label{bkm:Ref363931015}\label{bkm:Ref363931025}
\gll {\label{bkm:Ref363931144}} {Speaker-1:} {\bluebold{tida}} {di} {Beneraf} {\bluebold{e}?}\\ %
&  &  & \textsc{neg} & at & Beneraf & eh\\
\lspbottomrule
\end{tabular}
\begin{styleFreeTranslAlphaEng}
Speaker-1: ‘(they) are \bluebold{not} in Beneraf, \bluebold{eh?}’
\end{styleFreeTranslAlphaEng}

\begin{tabular}{llll} &  & Speaker-2: & \bluebold{mm}{}-\bluebold{mm}\\
\lsptoprule
&  &  & mhm\\
\lspbottomrule
\end{tabular}
\begin{styleFreeTranslAlphaEngxxpt}
Speaker-2: ‘\bluebold{mhm}!’ \textstyleExampleSource{[080925-003-Cv.0173-0174]}
\end{styleFreeTranslAlphaEngxxpt}

\subsection{Alternative questions}
\label{bkm:Ref364091902}
In Papuan Malay, alternative questions are formed with the alternative-marking conjunction \textitbf{ka} ‘or’ (see also §14.2.2.2). They require the interlocutor to choose the supposedly right answer from a list of possible answers, as shown in (0) to (0).



The alternatives can be overtly listed as in (0) or (0), in which case they are linked with post-posed \textitbf{ka} ‘or’. The question can also contain just one “proposition and its negation”, as in (0) or (0), in which case the proposition is marked with \textitbf{ka} ‘or’ followed by negator \textitbf{tida} ‘\textsc{neg}’. Rather often, though, the negator is omitted, as in (0) or (0).
\end{styleBodyvvafter}

\begin{tabular}{llllllllll}
\lsptoprule
\label{bkm:Ref364091540}
\gll {bapa} {pake} {kartu} {apa} {ka?} {AS} {\bluebold{ka}?} {Simpati} {\bluebold{ka}?}\\ %
& father & use & card & what & or & AS & or & Simpati & or\\
\lspbottomrule
\end{tabular}
\ea
\glt 
‘you (‘father’) use what (kind of SIM) card? AS \bluebold{or} Simpati?’ \textstyleExampleSource{[081014-016-Cv.0012]}
\z

\begin{tabular}{llllllm{-9.4015896E-4in}lllll}
\lsptoprule
\label{bkm:Ref364091541}
\gll {sa} {\multicolumn{2}{l}{tu}} {\multicolumn{2}{l}{biasa}} {\multicolumn{2}{l}{bilang}} {sama} {ana{\Tilde}ana} {di} {skola,}\\ %
& \textsc{1sg} & \multicolumn{2}{l}{\textsc{d.dist}} & \multicolumn{2}{l}{be.usual} & \multicolumn{2}{l}{say} & with & \textsc{rdp}{\Tilde}child & at & school\\
& \multicolumn{2}{l}{sala} & \multicolumn{2}{l}{\bluebold{ka}?} & \multicolumn{2}{l}{benar} & \multicolumn{5}{l}{\bluebold{ka}?}\\
& \multicolumn{2}{l}{be.wrong} & \multicolumn{2}{l}{or} & \multicolumn{2}{l}{be.true} & \multicolumn{5}{l}{or}\\
\lspbottomrule
\end{tabular}
\ea
\glt 
‘I (\textsc{emph}) usually ask the kids in school, ‘(is this) right \bluebold{or} wrong?’’ \textstyleExampleSource{[081014-015-Cv.0029]}
\z

\begin{tabular}{lllllll}
\lsptoprule
\label{bkm:Ref364091542}
\gll {kira{\Tilde}kira} {bisa} {kenal} {bapa} {\bluebold{ka}} {\bluebold{tida}?}\\ %
& \textsc{rdp}{\Tilde}think & be.able & know & father & or & \textsc{neg}\\
\lspbottomrule
\end{tabular}
\ea
\glt 
‘do you think you can recognize me (‘father’) \bluebold{or not}?’ \textstyleExampleSource{[080922-001a-CvPh.1301]}
\z

\begin{tabular}{lllllllll}
\lsptoprule
\label{bkm:Ref364091543}
\gll {mama} {Rahab} {ada} {datang} {ke} {ruma} {\bluebold{ka}} {\bluebold{tida}?}\\ %
& mother & Rahab & exist & come & to & house & or & \textsc{neg}\\
\lspbottomrule
\end{tabular}
\ea
\glt 
‘did mother Rahab come (\textsc{emph}) to the house \bluebold{or not}?’ \textstyleExampleSource{[081110-003-Cv.0001]}
\z

\begin{tabular}{lllll}
\lsptoprule
\label{bkm:Ref364091544}
\gll {de} {su} {datang} {\bluebold{ka}?}\\ %
& \textsc{3sg} & already & come & or\\
\lspbottomrule
\end{tabular}
\ea
\glt 
‘did he already come \bluebold{or (not)}?’ \textstyleExampleSource{[080925-003-Cv.0138]}
\z

\begin{tabular}{lllll}
\lsptoprule
\label{bkm:Ref364091545}
\gll {ko} {ada} {karet} {\bluebold{ka}?}\\ %
& \textsc{2sg} & exist & rubber & or\\
\lspbottomrule
\end{tabular}
\ea
\glt
‘do you have rubber bands \bluebold{or (not)}?’ \textstyleExampleSource{[081110-004-Cv.0008]}
\end{styleFreeTranslEngxvpt}

\section{Directive clauses}
\label{bkm:Ref363824852}
In Papuan Malay, three different types of directive clauses can be distinguished: imperatives and hortatives (§13.3.1), permissions and obligations (§13.3.2), and prohibitives (§13.3.3). They are used with any kind of predicate. Syntactically, directive clauses have the same structure as declarative clauses.
\end{styleBodyxvafter}

\subsection{Imperatives and hortatives}
\label{bkm:Ref363821875}
Papuan Malay employs imperatives and hortatives to issues commands. Imperatives always involve the second person, given that the addressee is the one who is expected to carry out the requested action, as shown in (0) to (0). In hortatives, by contrast, any person other than the addressee is expected to carry out the requested action. Hence, hortatives involve first and third persons, as shown in (0) to (0). In addition, Papuan Malay also employs a number of strategies to strengthen or soften commands, as demonstrated in (0) to (0).



Imperative constructions have a second person subject, as shown in (0) and (0). The clauses in (0) and (0) are formed with second singular \textitbf{ko} ‘\textsc{2sg}’ subjects. Depending on the context they can receive a declarative or an imperative reading. It is also possible to omit the addressee, as demonstrated in (0) and (0). Single word imperatives, as in (0), are rare, however. (The uses of \textitbf{suda} ‘already’ in directive clauses as in (0) are discussed together with the examples in (0) and (0).)
\end{styleBodyvvafter}

\begin{styleExampleTitle}
Imperatives: Syntactic structure\footnote{\\
\\
\\
\\
\\
\\
\\
\\
\\
\\
\\
\\
\\
\\
\\
\par Documentation: \textitbf{bangung} ‘wake up’ 081006-022-CvEx.0081, 080918-001-CvNP.0038; \textitbf{pulang} ‘go home’ 081006-025-CvEx.0013, 081006-007-Cv.0001.}
\end{styleExampleTitle}

\begin{tabular}{llllllll}
\lsptoprule
\label{bkm:Ref363656350}\label{bkm:Ref363656379}
\gll {\label{bkm:Ref376357420}} {\bluebold{ko}} {bangung} {} {\label{bkm:Ref363656364}} {e} {bangung!}\\ %
&  & \textsc{2sg} & wake.up &  &  & hey! & wake.up\\
&  & \multicolumn{2}{l}{‘\bluebold{you} woke up’ / ‘\bluebold{you} wake up!} &  &  & \multicolumn{2}{l}{‘hey, wake up!’}\\
&  & \multicolumn{2}{l}{} &  &  & \multicolumn{2}{l}{}\\
\label{bkm:Ref363656351}\label{bkm:Ref363656380}
\gll {\label{bkm:Ref376357423}} {\bluebold{ko}} {pulang} {} {\label{bkm:Ref363656367}} {pulang} {suda!}\\ %
&  & \textsc{2sg} & go.home &  &  & go.home & already\\
&  & \multicolumn{2}{l}{‘\bluebold{you} went home’ / \bluebold{you} go home!’} &  &  & \multicolumn{2}{l}{‘go home already!’}\\
\lspbottomrule
\end{tabular}

More examples of imperatives clauses are presented in (0) to (0), with second person singular addressees in (0) and (0), and second person plural addressees in (0). These examples also illustrate that imperatives are formed with trivalent verbs, as in (0), bivalent verbs as in (0), or monovalent verbs, such as stative \textitbf{diam} ‘be quiet’ in (0); see also monovalent dynamic \textitbf{pulang} ‘go home’ in (0).


\begin{styleExampleTitle}
Imperatives formed with tri-, bi-, and monovalent verbs
\end{styleExampleTitle}

\begin{tabular}{lllll}
\lsptoprule
\label{bkm:Ref363662758}
\gll {\bluebold{ko}} {ambil} {sa} {air!}\\ %
& \textsc{2sg} & fetch & \textsc{1sg} & water\\
\lspbottomrule
\end{tabular}
\ea
\glt 
‘\bluebold{you} fetch me water!’ \textstyleExampleSource{[081006-024-CvEx.0092]}
\z

\begin{tabular}{lllllllll}
\lsptoprule
\label{bkm:Ref363662759}
\gll {…} {trus} {\bluebold{kam}} {\bluebold{dua}} {cuci} {celana} {di} {situ!}\\ %
&  & next & \textsc{2pl} & two & wash & trousers & at & \textsc{l.med}\\
\lspbottomrule
\end{tabular}
\ea
\glt 
[A mother addressing her young sons:] ‘[hey, you two go bathe in the sea already!,] then \bluebold{you two} wash (your) trousers there!’ \textstyleExampleSource{[080917-006-CvHt.0007]}
\z

\begin{tabular}{lllllll}
\lsptoprule
\label{bkm:Ref363662760}
\gll {\bluebold{ko}} {jangang} {bicara} {lagi,} {\bluebold{ko}} {diam!}\\ %
& \textsc{2sg} & \textsc{neg.imp} & speak & again & \textsc{2sg} & be.quiet\\
\lspbottomrule
\end{tabular}
\ea
\glt 
‘\bluebold{you} don’t talk again!, \bluebold{you} be quiet!’ \textstyleExampleSource{[081029-004-Cv.0072]}
\z


Hortatives are typically expressed with clause-initial \textitbf{biar} ‘let’. It exhorts the addressee to let or allow the desired future state of affairs come true, as illustrated in (0) to (0).


\begin{styleExampleTitle}
Hortatives with clause-initial \textitbf{biar} ‘let’
\end{styleExampleTitle}

\begin{tabular}{llllllll}
\lsptoprule
\label{bkm:Ref363725195}
\gll {kalo} {nanti} {tong} {maing} {\bluebold{biar}} {\bluebold{sa}} {cadangang!}\\ %
& if & very.soon & \textsc{1pl} & play & let & \textsc{1sg} & reserve\\
\lspbottomrule
\end{tabular}
\ea
\glt 
‘later when we play (volleyball), \bluebold{let me} be a reserve!’ \textstyleExampleSource{[081109-001-Cv.0154]}
\z

\begin{tabular}{llllll}
\lsptoprule
(\stepcounter{}{\the}) & \bluebold{biar} & \bluebold{tong} & tinggal & di & situ!\\
& let & \textsc{1pl} & stay & at & \textsc{l.med}\\
\lspbottomrule
\end{tabular}
\ea
\glt 
‘\bluebold{let us} live there!’ \textstyleExampleSource{[081110-008-CvNP.0091]}
\z

\begin{tabular}{lllllll}
\lsptoprule
(\stepcounter{}{\the}) & yo, & \bluebold{biar} & \bluebold{de} & juga & liat & sa!\\
& yes & let & \textsc{3sg} & also & see & \textsc{1sg}\\
\lspbottomrule
\end{tabular}
\ea
\glt 
‘yes, \bluebold{let her} also see me!’ \textstyleExampleSource{[081015-005-NP.0013]}
\z

\begin{tabular}{llllllll}
\lsptoprule
\label{bkm:Ref363725199}
\gll {\bluebold{biar}} {\bluebold{dong}} {ejek{\Tilde}ejek} {bapa!,} {tida} {apa{\Tilde}apa} {to?}\\ %
& let & \textsc{3pl} & \textsc{rdp}{\Tilde}mock & father & \textsc{neg} & \textsc{rdp}{\Tilde}what & right?\\
\lspbottomrule
\end{tabular}
\ea
\glt 
‘\bluebold{let them} mock me (‘father’)!, it doesn’t matter, right?’ \textstyleExampleSource{[080922-001a-CvPh.0180]}
\z


First person plural hortatives can also be formed without \textitbf{biar} ‘let’, as shown in (0). In this case, the context shows whether the utterance is a hortative such as the first \textitbf{kitong dua pulang} ‘(let) the two of us go home!’ token, or a declarative such as the second occurrence of \textitbf{kitong dua pulang} ‘the two of us went home’.


\begin{styleExampleTitle}
First person plural hortatives without clause-initial \textitbf{biar} ‘let’
\end{styleExampleTitle}

\begin{tabular}{llllllllllll}
\lsptoprule
\label{bkm:Ref363725204}
\gll {dia} {bilang,} {\bluebold{Ø}} {\bluebold{kitong}} {\bluebold{dua}} {pulang!} {…} {trus} {\bluebold{kitong}} {\bluebold{dua}} {pulang}\\ %
& \textsc{3sg} & say &  & \textsc{1pl} & two & go.home &  & next & \textsc{1pl} & two & go.home\\
\lspbottomrule
\end{tabular}
\ea
\glt 
‘he said, ‘\bluebold{(let) the two of us} go home!’ … then \bluebold{the two of us} went home’ \textstyleExampleSource{[081015-005-NP.0035]}
\z


Papuan Malay also uses a number of strategies to strengthen or soften commands. Strengthening is illustrated in (0) to (0) and softening in (0) to (0).



Speakers can add \textitbf{ayo} ‘come on’ or \textitbf{suda} ‘already’ to commands or requests to make them more urgent and to strengthen them. Urgency-marking \textitbf{ayo} ‘come on’ can occur clause-initially, as in the imperative in (0) and in the hortative in (0), or clause-finally, also in (0); \textitbf{ayo} ‘come on!’ is unattested in hortatives with third persons. Urgency-marking \textitbf{suda} ‘already’, by contrast, always takes a post-predicate position as in the imperative in (0) and the hortative in (0).
\end{styleBodyvvafter}

\begin{styleExampleTitle}
Strengthening commands with \textitbf{ayo} ‘come on’ or \textitbf{suda} ‘already’
\end{styleExampleTitle}

\begin{tabular}{llllll}
\lsptoprule
\label{bkm:Ref363734749}
\gll {\bluebold{ayo},} {jalang} {ke} {Ise!,} {\bluebold{ayo}!}\\ %
& come.on! & walk & to & Ise & come.on!\\
\lspbottomrule
\end{tabular}
\ea
\glt 
‘\bluebold{come on!}, go to Ise, \bluebold{come on!}’ \textstyleExampleSource{[080917-008-NP.0065]}
\z

\begin{tabular}{llllllllll}
\lsptoprule
\label{bkm:Ref363734750}
\gll {\bluebold{ayo},} {kitong} {dua} {jalang} {cepat!,} {kitong} {dua} {jalang} {cepat!}\\ %
& come.on! & \textsc{1pl} & two & walk & be.fast & \textsc{1pl} & two & walk & be.fast\\
\lspbottomrule
\end{tabular}
\ea
\glt 
‘\bluebold{come on!}, (let) the two of us walk fast!, (let) the two of us walk fast!’ \textstyleExampleSource{[081015-005-NP.0037]}
\z

\begin{tabular}{lllllllll}
\lsptoprule
\label{bkm:Ref363734751}
\gll {ey,} {kam} {dua} {pi} {mandi} {di} {laut} {\bluebold{suda}!}\\ %
& hey! & \textsc{2pl} & two & go & bathe & at & sea & already\\
\lspbottomrule
\end{tabular}
\ea
\glt 
‘hey, you two go bathe in the sea \bluebold{already}!’ \textstyleExampleSource{[080917-006-CvHt.0007]}
\z

\begin{tabular}{lllllll}
\lsptoprule
\label{bkm:Ref363734752}
\gll {ana} {kecil} {biar} {dong} {makang} {\bluebold{suda}!}\\ %
& child & be.small & let & \textsc{3pl} & eat & just\\
\lspbottomrule
\end{tabular}
\ea
\glt 
‘(as for) the small children, let them eat \bluebold{already}!’ \textstyleExampleSource{[081002-001-CvNP.0051]}
\z


Requests or commands can be softened by adding clause-initial \textitbf{coba} ‘try’ as in (0), \textitbf{mari} ‘hither, (come) here’ as in (0), or \textitbf{tolong} ‘please’ (literally ‘help’) as in (0). This applies most often to imperatives, as in (0) and (0), and less often to hortatives, as in (0).


\begin{styleExampleTitle}
Softening commands with clause-initial \textitbf{coba} ‘try’, \textitbf{mari} ‘hither, (come) here’, or \textitbf{tolong} ‘help’
\end{styleExampleTitle}

\begin{tabular}{lllllll}
\lsptoprule
\label{bkm:Ref363745152}
\gll {sa} {bilang,} {\bluebold{coba}} {ko} {tanya} {dorang!}\\ %
& \textsc{1sg} & say & try & \textsc{2sg} & ask & \textsc{3pl}\\
\lspbottomrule
\end{tabular}
\ea
\glt 
‘I said, ‘\bluebold{try} asking them!’ \textstyleExampleSource{[081025-008-Cv.0076]}
\z

\begin{tabular}{llllll}
\lsptoprule
\label{bkm:Ref363745156}
\gll {a,} {\bluebold{mari}} {kitong} {turung} {olaraga!}\\ %
& ah! & hither & \textsc{1pl} & descend & do.sports\\
\lspbottomrule
\end{tabular}
\ea
\glt 
‘ah, \bluebold{come}, (let) us go down (to the beach) to do sports!’ \textstyleExampleSource{[080917-001-CvNP.0003]}
\z

\begin{tabular}{lllll}
\lsptoprule
\label{bkm:Ref363745157}
\gll {\bluebold{tolong}} {ceritra} {tu} {plang{\Tilde}plang!}\\ %
& help & tell & \textsc{d.dist} & \textsc{rdp}{\Tilde}be.slow\\
\lspbottomrule
\end{tabular}
\ea
\glt 
[Addressing another adult:] ‘\bluebold{please}, talk (\textsc{emph}) slowly!’ \textstyleExampleSource{[081015-005-NP.0015]}
\z


Requests or commands can also be mitigated by adding in post-predicate position the temporal adverb \textitbf{dulu} ‘first, in the past’ as in (0), the focus adverb \textitbf{saja} ‘just’ as in (0), or the clause-final tag \textitbf{e} ‘eh?’ as in (0). (For more details on adverbs see §5.4 and on tags see §5.13.1.)


\begin{styleExampleTitle}
Softening commands with clause-final \textitbf{dulu} ‘first, in the past’, \textitbf{saja} ‘just’, or \textitbf{e} ‘eh’
\end{styleExampleTitle}

\begin{tabular}{lllll}
\lsptoprule
\label{bkm:Ref363741830}
\gll {sabar} {\bluebold{dulu}!,} {sabar} {\bluebold{dulu}!}\\ %
& be.patient & first & be.patient & first\\
\lspbottomrule
\end{tabular}
\ea
\glt 
‘be patient \bluebold{for now}!, be patient \bluebold{for now}!’ \textstyleExampleSource{[080921-004b-CvNP.0051]}
\z

\begin{tabular}{lllll}
\lsptoprule
\label{bkm:Ref372459198}
\gll {sa} {blang,} {jalang} {\bluebold{saja}!}\\ %
& \textsc{1pl} & say & walk & just\\
\lspbottomrule
\end{tabular}
\ea
\glt 
‘I said, ‘just \bluebold{walk}!’’ \textstyleExampleSource{[080917-008-NP.0117]}
\z

\begin{tabular}{lllllll}
\lsptoprule
\label{bkm:Ref363741832}
\gll {ko} {kasi} {sama} {kaka} {mantri} {\bluebold{e}?!}\\ %
& \textsc{2sg} & give & to & oSb & male.nurse & eh\\
\lspbottomrule
\end{tabular}
\ea
\glt
‘give (the keys) to the male nurse, \bluebold{eh}?!’ \textstyleExampleSource{[080922-010a-CvNF.0167]}
\end{styleFreeTranslEngxvpt}

\subsection{Permissions and obligations}
\label{bkm:Ref363821876}
Papuan Malay permissions are expressed with the auxiliary verb \textitbf{bole} ‘may’, as illustrated in (0) to (0), while obligations are formed with the auxiliary verb \textitbf{harus} ‘have to’, as shown in (0) and (0).



Permission-marking \textitbf{bole} ‘may’ most often occurs in single-word clauses, following a clause which depicts the permitted event or state, as in (0) or (0). Less often, \textitbf{bole} ‘may’ occurs between the subject and the predicate, as in (0). Only rarely, \textitbf{bole} ‘may’ occurs clause-initially, where it has scope over the entire clause, as (0).
\end{styleBodyvvafter}

\begin{styleExampleTitle}
Permissions with \textitbf{bole} ‘may’
\end{styleExampleTitle}

\begin{tabular}{lllll}
\lsptoprule
\label{bkm:Ref363809496}
\gll {kamu} {mo} {pacar,} {\bluebold{bole}}\\ %
& \textsc{2pl} & want & date/lover & may\\
\lspbottomrule
\end{tabular}
\ea
\glt 
[Addressing teenagers:] ‘(if) you want to date (someone) you \bluebold{may / are allowed to} (do so)’ \textstyleExampleSource{[081011-023-Cv.0269]}
\z

\begin{tabular}{llllll}
\lsptoprule
\label{bkm:Ref363809494}
\gll {ko} {mancing} {dari} {jembatang,} {\bluebold{bole}}\\ %
& \textsc{2sg} & fish & from & bridge & may\\
\lspbottomrule
\end{tabular}
\ea
\glt 
[Addressing her son:] ‘(if) you’re fishing from the bridge, (you) \bluebold{may} (do so) / \bluebold{are allowed to} (fish)’ \textstyleExampleSource{[081025-003-Cv.0058]}
\z

\begin{tabular}{llllllll}
\lsptoprule
\label{bkm:Ref363809497}
\gll {setiap} {kegiatang} {apa} {saja} {dorang} {\bluebold{bole}} {kerja}\\ %
& every & activity & what & just & \textsc{3pl} & may & work\\
\lspbottomrule
\end{tabular}
\ea
\glt 
‘whatever activity, they \bluebold{may} / \bluebold{are allowed to} carry (it) out’ \textstyleExampleSource{[080923-007-Cv.0013]}
\z

\begin{tabular}{llllllllll}
\lsptoprule
\label{bkm:Ref351225332}
\gll {…} {kalo} {tinggal} {di} {Arbais,} {\bluebold{bole}} {ko} {tokok} {sama{\Tilde}sama}\\ %
&  & if & stay & at & Arbais & may & \textsc{2sg} & tap & \textsc{rdp}{\Tilde}be.same\\
& \multicolumn{2}{l}{dengang} & \multicolumn{7}{l}{kaka}\\
& \multicolumn{2}{l}{with} & \multicolumn{7}{l}{oSb}\\
\lspbottomrule
\end{tabular}
\ea
\glt 
‘[my husband said, ‘(here in my village) don’t extract and crush the sago, you just knead and filter it,] when you’re staying in Arbais, (it is) \bluebold{allowed} (that) you (extract and) crush (the sago) together with (your) older sibling’ \textstyleExampleSource{[081014-007-CvEx.0058]}
\z


Obligation-marking \textitbf{harus} ‘have to’ typically takes a pre-predicate position, as in (0). Alternatively, \textitbf{harus} ‘have to’ can occur clause-initially, where it has scope over the entire clause and reinforces the obligation, as in (0).


\begin{styleExampleTitle}
Obligations with \textitbf{harus} ‘have to’
\end{styleExampleTitle}

\begin{tabular}{llllllllllllll}
\lsptoprule
\label{bkm:Ref363805642}
\gll {\multicolumn{2}{l}{besok}} {\multicolumn{2}{l}{pagi}} {\multicolumn{2}{l}{saya}} {\multicolumn{2}{l}{\bluebold{harus}}} {\multicolumn{2}{l}{cari}} {batrey,} {sa} {\bluebold{harus}}\\ %
& \multicolumn{2}{l}{tomorrow} & \multicolumn{2}{l}{morning} & \multicolumn{2}{l}{\textsc{1sg}} & \multicolumn{2}{l}{have.to} & \multicolumn{2}{l}{search} & battery & \textsc{1sg} & have.to\\
& bli & \multicolumn{2}{l}{pecis,} & \multicolumn{2}{l}{sa} & \multicolumn{2}{l}{\bluebold{harus}} & \multicolumn{2}{l}{ambil} & \multicolumn{4}{l}{senter}\\
& buy & \multicolumn{2}{l}{light.bulb} & \multicolumn{2}{l}{\textsc{1sg}} & \multicolumn{2}{l}{have.to} & \multicolumn{2}{l}{fetch} & \multicolumn{4}{l}{flashlight}\\
\lspbottomrule
\end{tabular}
\ea
\glt 
[Getting ready for hunting:] ‘tomorrow morning I \bluebold{have to} get batteries, I \bluebold{have to} buy small light bulbs, I \bluebold{have to} take a flashlight’ \textstyleExampleSource{[080919-004-NP.0003]}
\z

\begin{tabular}{llllll}
\lsptoprule
\label{bkm:Ref363805644}
\gll {\bluebold{harus}} {kitong} {baik} {deng} {orang}\\ %
& have.to & \textsc{1pl} & be.good & with & person\\
\lspbottomrule
\end{tabular}
\ea
\glt
‘we \bluebold{have to (}\blueboldSmallCaps{emph}\bluebold{)} be / (it’s) \bluebold{obligatory} (that) we are good to (other) people’ \textstyleExampleSource{[081110-008-CvNP.0166]}
\end{styleFreeTranslEngxvpt}

\subsection{Prohibitives}
\label{bkm:Ref438299584}\label{bkm:Ref363821877}
Papuan Malay prohibitives are typically formed with the negative imperative \textitbf{jangang} ‘\textsc{neg.imp}, don’t’. Its main function is to signal the addressee that the action of the verb is forbidden, as illustrated in (0) to (0). Quite often, however, a prohibitive is softened, by using \textitbf{tida}/\textitbf{tra bole} ‘shouldn’t’ (literally ‘may not’), instead of \textitbf{jangang} ‘\textsc{neg.imp}, don’t’, as shown in (0) to (0).



The main function of negative imperative \textitbf{jangang} ‘\textsc{neg.imp}’, with its short form \textitbf{jang}, is to signal a straight-out prohibitive. It occurs between the subject and the predicate, as in (0) and (0), or clause-initially where it has scope over the entire clause and reinforces the prohibitive, as in (0) and (0). Besides, speakers also employ \textitbf{jangang} ‘\textsc{neg.imp}’ as stand-alone clauses which provide a response to a preceding prohibitive, in the sense of ‘(I would) never (do such a thing)’, as in (0).
\end{styleBodyvvafter}

\begin{styleExampleTitle}
Prohibitives with \textitbf{jangang} ‘\textsc{neg.imp}’
\end{styleExampleTitle}

\begin{tabular}{llllllll}
\lsptoprule
\label{bkm:Ref353214813}
\gll {Wili} {ko} {\bluebold{jangang}} {gara{\Tilde}gara} {tanta} {dia} {itu!}\\ %
& Wili & \textsc{2sg} & \textsc{neg.imp} & \textsc{rdp}{\Tilde}irritate & aunt & \textsc{3sg} & \textsc{d.dist}\\
\lspbottomrule
\end{tabular}
\ea
\glt 
[Addressing a young boy:] ‘you Wili \bluebold{don’t} irritate that aunt!’ \textstyleExampleSource{[081023-001-Cv.0038]}
\z

\begin{tabular}{lllll}
\lsptoprule
\label{bkm:Ref358294387}
\gll {kamorang} {\bluebold{jangang}} {pukul} {dia!}\\ %
& \textsc{2pl} & \textsc{neg.imp} & hit & \textsc{3sg}\\
\lspbottomrule
\end{tabular}
\ea
\glt 
‘\bluebold{don’t} beat him!’ \textstyleExampleSource{[081015-005-NP.0024]}
\z

\begin{tabular}{llll}
\lsptoprule
\label{bkm:Ref358294382}
\gll {\bluebold{jangang}} {ko} {pergi!}\\ %
& \textsc{neg.imp} & \textsc{2sg} & go\\
\lspbottomrule
\end{tabular}
\ea
\glt 
‘\bluebold{don’t} you go!’ \textstyleExampleSource{[081025-006-Cv.0192]}
\z

\begin{tabular}{llllll}
\lsptoprule
\label{bkm:Ref358294383}
\gll {Klara,} {\bluebold{jangang}} {ko} {gara{\Tilde}gara} {dia!}\\ %
& Klara & \textsc{neg.imp} & \textsc{2sg} & \textsc{rdp}{\Tilde}irritate & \textsc{3sg}\\
\lspbottomrule
\end{tabular}
\ea
\glt 
‘Klara, \bluebold{don’t} you irritate him!’ \textstyleExampleSource{[080917-003b-CvEx.0027]}
\z

\begin{tabular}{lllllllll}
\lsptoprule
\label{bkm:Ref358294388}
\gll {…} {a,} {\bluebold{jangang}!,} {sa} {tida} {bisa} {buang} {takaroang}\\ %
&  & ah & \textsc{neg.imp} & \textsc{1sg} & \textsc{neg} & be.able & discard & be.chaotic\\
\lspbottomrule
\end{tabular}
\ea
\glt 
‘[he said (to me), ‘don’t throw away (your betel nut waste)’, (I said),] ‘ah \bluebold{never}!, I can’t throw (it) away randomly’’ \textstyleExampleSource{[081025-008-Cv.0012]}
\z


Prohibitives can be softened by employing \textitbf{tida}/\textitbf{tra bole} ‘shouldn’t’ (literally ‘may not’). Most often, \textitbf{tida}/\textitbf{tra bole} ‘may not’ occurs between the subject and the predicate, as in (0) and (0). Alternatively, although rarely, it occurs clause-initially, where it has scope over the entire clause, as in (0). In addition, speakers use \textitbf{tida}/\textitbf{tra bole} ‘may not’ as stand-alone clauses, which refer back to the speakers’ own or their interlocutors’ preceding statements about a state of affairs, as in (0) and (0), respectively.


\begin{styleExampleTitle}
Prohibitives with \textitbf{tida}/\textitbf{tra bole} ‘may not’
\end{styleExampleTitle}

\begin{tabular}{lllllllllll}
\lsptoprule
\label{bkm:Ref363820424}
\gll {sa} {\bluebold{tida}} {\bluebold{bole}} {di} {depang!,} {saya} {harus} {di} {blakang} {skali}\\ %
& \textsc{1sg} & \textsc{neg} & may & at & front & \textsc{1sg} & have.to & at & backside & very\\
\lspbottomrule
\end{tabular}
\ea
\glt 
‘I \bluebold{shouldn’t} be in front, I had to stay in the very back’ \textstyleExampleSource{[081029-005-Cv.0133]}
\z

\begin{tabular}{llllllllll}
\lsptoprule
\label{bkm:Ref363820425}
\gll {mama} {\bluebold{tra}} {\bluebold{bole}} {lipat!,} {mama} {harus} {kas} {panjang} {kaki}\\ %
& mother & \textsc{neg} & may & fold & mother & have.to & give & be.long & foot\\
\lspbottomrule
\end{tabular}
\ea
\glt 
[Addressing someone with a bad knee:] ‘you (‘mother’) \bluebold{shouldn’t} fold (your legs) under, you (‘mother’) have to stretch out (your) legs’ \textstyleExampleSource{[080921-004a-CvNP.0069]}
\z

\begin{tabular}{llllll}
\lsptoprule
\label{bkm:Ref363820427}
\gll {\bluebold{tida}} {\bluebold{bole}} {ko} {ceritra} {orang!}\\ %
& \textsc{neg} & may & \textsc{2sg} & tell & person\\
\lspbottomrule
\end{tabular}
\ea
\glt 
‘you \bluebold{shouldn’t (}\blueboldSmallCaps{emph}\bluebold{)} tell other people’ \textstyleExampleSource{[081110-008-CvNP.0072]}
\z

\begin{tabular}{lllllll}
\lsptoprule
\label{bkm:Ref363820428}
\gll {…} {bunga{\Tilde}bunga} {suda} {habis,} {\bluebold{tida}} {\bluebold{bole}!}\\ %
&  & \textsc{rdp}{\Tilde}flower & already & be.used.up & \textsc{neg} & may\\
\lspbottomrule
\end{tabular}
\ea
\glt 
[Addressing a child who had picked the speaker’s flowers:] ‘[(the flowers) over there (you) already picked (them) until (they were) all gone,] the flowers are already gone, (you) \bluebold{shouldn’t} (have done that)’ \textstyleExampleSource{[081006-021-CvHt.0001]}
\z

\begin{tabular}{lllll}
\lsptoprule
\label{bkm:Ref363820429}
\gll {Speaker-2:} {a,} {\bluebold{tida}} {\bluebold{bole}!}\\ %
&  & ah! & \textsc{neg} & may\\
\lspbottomrule
\end{tabular}
\ea
\glt 
[About membership in a committee:] [Speaker-1: ‘the two of them are the committee’]\\
Speaker-2: ‘ah, (that) \bluebold{shouldn’t} be!’ \textstyleExampleSource{[080917-002-Cv.0015-0016]}
\z

%\setcounter{page}{1}\chapter[Conjunctions and constituent combining]{Conjunctions and constituent combining}
\label{bkm:Ref439247309}\label{bkm:Ref439247217}\label{bkm:Ref439247144}\label{bkm:Ref436244712}\label{bkm:Ref387858178}\label{bkm:Ref374460078}\label{bkm:Ref374435055}\label{bkm:Ref374434202}\label{bkm:Ref374433903}\section{Introduction}

This chapter describes how Papuan Malay combines constituents such as clauses or phrases by overt marking with conjunctions. The Papuan Malay conjunctions can be divided into two major groups, those combining same-type constituents, such as clauses with clauses, and those linking different-type constituents, such as verbs with clauses. In combining constituents, the conjunctions belong to neither of the conjuncts they combine in semantic terms. They do, however, form intonation units with the constituents they mark. Most conjunctions occur at the left periphery of the clause. Typically, an intonational break separates the conjunction from a preceding constituent. A second strategy to combine constituents is juxtaposition which is mentioned only briefly.



Papuan Malay has 21 conjunctions which link same-type constituents and two which combine different-type constituents. Most of the conjunctions conjoining same-type constituents link clauses with clauses. Traditionally, clause-linking conjunctions are divided into coordinating and subordinating ones: “coordinating conjunctions are those that assign equal rank to the conjoined elements” whereas “subordinating conjunctions are those that assign unequal rank to the conjoined elements, marking one of them as subordinate to the other” {(Schachter and Shopen 2007: 45)}. Modifying this terminology by employing the more general term “dependency” rather than “subordination”, {\citet[46]{Haspelmath2007a}} defines the distinction between coordination and dependency as follows:
\end{styleBodyvvafter}

\begin{styleIvI}
In a coordination structure of the type \textstyleChItalic{A(-link-)B}, \textstyleChItalic{A}\textit{ }and \textstyleChItalic{B}\textit{ }are structurally symmetrical in some sense, whereas in a dependency structure of the type \textstyleChItalic{X(-link-)Y}, \textstyleChItalic{X}\textit{ }and \textstyleChItalic{Y}\textit{ }are not symmetrical, but either \textit{X }or \textit{Y }is the head and the other element is a dependent.
\end{styleIvI}


According to {\citet[46]{Haspelmath2007a}}, this distinction between coordination and dependency in terms of symmetry “is often thought of as a difference in the syntactic/structural relations of the elements”. As {Haspelmath }points out, however, that “it is sometimes not evident whether a construction exhibits a coordination relation or a dependency relation”; this applies, for instance, to “languages that lack agreement and case-marking” {(2007a: 46)}.



The lack of a clear opposition between coordination and dependency in terms of structural relations also applies to clause combining in Papuan Malay: clauses marked with a conjunction are not distinct from unmarked clauses in terms of their morphosyntax and word order. This is shown in (0) to (0) with purpose-marking \textitbf{supaya} ‘so that’. Omitting the conjunction from the two purpose clauses in (0) leaves two grammatically complete and correct clauses: \textitbf{saya harus kas makang dia} ‘I have to give him/her food’ and \textitbf{dia kenal saya lebi} ‘he/she can know me better’.
\end{styleBodyvvafter}

\begin{styleExampleTitle}
Purpose-marking \textitbf{supaya} ‘so that’ linking two clauses
\end{styleExampleTitle}

\begin{tabular}{llllllllllllllllll}
\lsptoprule
\label{bkm:Ref356573997}
\gll {saya} {\multicolumn{2}{l}{harus}} {\multicolumn{2}{l}{kas}} {\multicolumn{2}{l}{makang}} {\multicolumn{2}{l}{dia,}} {\multicolumn{2}{l}{\bluebold{supaya}}} {\multicolumn{2}{l}{dia}} {\multicolumn{2}{l}{kenal}} {saya} {lebi}\\ %
& \textsc{1sg} & \multicolumn{2}{l}{have.to} & \multicolumn{2}{l}{give} & \multicolumn{2}{l}{eat} & \multicolumn{2}{l}{\textsc{3sg}} & \multicolumn{2}{l}{so.that} & \multicolumn{2}{l}{\textsc{3sg}} & \multicolumn{2}{l}{know} & \textsc{1sg} & more\\
& \multicolumn{2}{l}{dekat,} & \multicolumn{2}{l}{\bluebold{supaya}} & \multicolumn{2}{l}{de} & \multicolumn{2}{l}{bisa} & \multicolumn{2}{l}{taw} & \multicolumn{2}{l}{saya} & \multicolumn{2}{l}{punya} & \multicolumn{3}{l}{nama}\\
& \multicolumn{2}{l}{near} & \multicolumn{2}{l}{so.that} & \multicolumn{2}{l}{\textsc{3sg}} & \multicolumn{2}{l}{be.able} & \multicolumn{2}{l}{know} & \multicolumn{2}{l}{\textsc{1sg}} & \multicolumn{2}{l}{\textsc{poss}} & \multicolumn{3}{l}{name}\\
\lspbottomrule
\end{tabular}
\ea
\glt 
‘I have to give him/her food \bluebold{so that} he/she can know me better, \bluebold{so that} he/she can know my name’ \textstyleExampleSource{[080919-004-NP.0079]}
\z


When a conjunction is missing an argument, the result is still a grammatically complete and correct clause. In the purpose clause in (0), for instance, the subject \textitbf{obat} ‘medicine’ is elided. This elision, however, does not signify a grammaticalized gap that signals the dependent status of the purpose clause marked with \textitbf{supaya} ‘so that’. Instead, the elision is due to the fact that speakers often omit arguments and other constituents if these can be inferred. In (0) the elided subject \textitbf{obat} ‘medicine’ is understood from the context.


\begin{styleExampleTitle}
Purpose clause with elided subject argument
\end{styleExampleTitle}

\begin{tabular}{lllllllllllllll}
\lsptoprule
\label{bkm:Ref356573998}
\gll {\multicolumn{2}{l}{ibu}} {\multicolumn{2}{l}{itu}} {de} {\multicolumn{3}{l}{mo}} {kasi} {\multicolumn{2}{l}{obat,}} {tapi} {ko} {harus}\\ %
& \multicolumn{2}{l}{woman} & \multicolumn{2}{l}{\textsc{d.dist}} & \textsc{3sg} & \multicolumn{3}{l}{want} & give & \multicolumn{2}{l}{medicine} & but & \textsc{2sg} & have.to\\
& priksa & \multicolumn{2}{l}{dara,} & \multicolumn{3}{l}{\bluebold{supaya}} & Ø & \multicolumn{3}{l}{harus} & \multicolumn{4}{l}{cocok}\\
& check & \multicolumn{2}{l}{blood} & \multicolumn{3}{l}{so.that} &  & \multicolumn{3}{l}{have.to} & \multicolumn{4}{l}{be.suitable}\\
\lspbottomrule
\end{tabular}
\ea
\glt 
‘that lady, she wants to give (you) medicine, but you have to (get your) blood checked \bluebold{so that} (the medicine) fits’ \textstyleExampleSource{[080917-007-CvHt.0003]}
\z


In Papuan Malay, elision of core arguments is not limited to clauses marked with conjunctions. It is a generalized phenomenon, as demonstrated with the reported direct speech in (0) (see also §11.1). The original utterance is given in (0), while in (0) the elided constituents are given in brackets, such as purposive \textitbf{supaya} ‘so that’\footnote{\\
\\
\\
\\
\\
\\
\\
\\
\\
\\
\\
\\
\\
\\
\\
\par Alternatively, the conjunction \textitbf{sampe} ‘until, with the result that’ could fill this slot.} or the subject of the purpose clause, \textitbf{kaki} ‘foot, leg’.


\begin{styleExampleTitle}
Elision as a generalized phenomenon
\end{styleExampleTitle}

\begin{tabular}{lllllllllllllllll}
\lsptoprule
\label{bkm:Ref356573999}\label{bkm:Ref356583696}
\gll {\label{bkm:Ref356583714}} {…} {\multicolumn{3}{l}{malam}} {\multicolumn{4}{l}{Kapolsek}} {\multicolumn{2}{l}{bilang,}} {\multicolumn{2}{l}{kalo}} {\multicolumn{3}{l}{dapat}}\\ %
&  &  & \multicolumn{3}{l}{night} & \multicolumn{4}{l}{head.of.district.police} & \multicolumn{2}{l}{say} & \multicolumn{2}{l}{if} & \multicolumn{3}{l}{get}\\
&  & \multicolumn{3}{l}{tembak} & \multicolumn{2}{l}{kaki} & \multicolumn{10}{l}{pata}\\
&  & \multicolumn{3}{l}{shoot} & \multicolumn{2}{l}{foot} & \multicolumn{10}{l}{break}\\
&  & \multicolumn{3}{l}{} & \multicolumn{2}{l}{} & \multicolumn{10}{l}{}\\
&  & … & \multicolumn{3}{l}{malam} & \multicolumn{4}{l}{Kapolsek} & \multicolumn{2}{l}{bilang,} & \multicolumn{2}{l}{kalo} & \multicolumn{2}{l}{[kam]} & dapat\\
&  &  & \multicolumn{3}{l}{night} & \multicolumn{4}{l}{head.of.district.police} & \multicolumn{2}{l}{say} & \multicolumn{2}{l}{if} & \multicolumn{2}{l}{[\textsc{2pl}]} & get\\
&  & \multicolumn{2}{l}{[dia,]} & \multicolumn{4}{l}{tembak} & [de pu] & \multicolumn{2}{l}{kaki} & \multicolumn{2}{l}{[\bluebold{supaya}]} & \multicolumn{2}{l}{[kaki]} & \multicolumn{2}{l}{pata}\\
&  & \multicolumn{2}{l}{[\textsc{3sg}]} & \multicolumn{4}{l}{shoot} & [\textsc{3sg} \textsc{poss}] & \multicolumn{2}{l}{foot} & \multicolumn{2}{l}{[so.that]} & \multicolumn{2}{l}{[foot]} & \multicolumn{2}{l}{break}\\
\lspbottomrule
\end{tabular}
\ea
\glt 
[Reply to the question about who the police were looking for:] ‘[(they’re looking for Martin …,] (last) night the head of the district police said, ‘if (you) get (him), shoot (his) leg (\bluebold{so that} it) breaks’’ \textstyleExampleSource{[081011-009-Cv.0048/0050]}
\z


This data shows that, in terms of structural relations, the opposition between coordination and dependency does not apply to purpose-marking \textitbf{supaya} ‘so that’. Neither does the distinction apply to the other clause-combining conjunctions.



Given that cross-linguistically this lack of a clear-cut opposition between coordination and dependency in terms of structural relations is not uncommon, {\citet[46]{Haspelmath2007a}} suggests “to define both coordination and dependency in semantic terms”. He also notes, however, that even the distinction on semantic grounds “is often difficult to apply” ({2007a: 47}; see also {Cristofaro 2005: 1–50}{;} {Dixon and Aikhenvald 2009}).
\end{styleBodyvafter}


This difficulty also applies to clause combining in Papuan Malay. Therefore, in discussing clause combining in Papuan Malay at this point in the current research, no attempt is being made to distinguish between coordination and dependency on semantic grounds. Instead, this chapter describes the following aspects: (1) the meaning which the different Papuan Malay conjunctions convey, (2) the position which a given conjunction takes within its clause, and (3) the position which the clause marked with a conjunction takes vis-à-vis the clause it is conjoined with. For lack of a better term, the clause that is not marked with a conjunction is labeled as the “unmarked clause” throughout the remainder of this chapter. This label is used as a working term only for practical purposes.
\end{styleBodyvafter}


In addition to the 21 conjunctions combining same-type constituents, Papuan Malay also has two conjunctions which link different-type constituents, namely complementizer \textitbf{bahwa} ‘so that’ and relativizer \textitbf{yang} ‘\textsc{rel}’. Both are subordinating conjunctions, in that they “serve to integrate a … clause into some larger construction”, adopting {Schachter and Shopen’s (2007: 45)} definition. Complementizer \textitbf{bahwa} ‘that’ marks a clause as an argument of the verb, as illustrated in (0), while relativizer \textitbf{yang} ‘\textsc{rel}’ integrates a relative clause within a noun phrase, as demonstrated in (0).
\end{styleBodyvvafter}

\begin{styleExampleTitle}
Conjunctions combining different-type constituents
\end{styleExampleTitle}

\begin{tabular}{lllllllll}
\lsptoprule
\label{bkm:Ref356576825}
\gll {sa} {cuma} {taw} {\bluebold{bahwa}} {de} {ada} {di} {sini}\\ %
& \textsc{1sg} & just & know & that & \textsc{3sg} & exist & at & \textsc{l.prox}\\
\lspbottomrule
\end{tabular}
\ea
\glt 
‘I just know \bluebold{that} he was here’ \textstyleExampleSource{[080922-010a-CvNF.0180]}
\z

\begin{tabular}{lllllllllll}
\lsptoprule
\label{bkm:Ref356583506}
\gll {baru} {Iskia} {dia} {pegang} {sa} {punya} {lutut} {\bluebold{yang}} {tida} {baik}\\ %
& and.then & Iskia & \textsc{3sg} & hold & \textsc{1sg} & \textsc{poss} & knee & \textsc{rel} & \textsc{neg} & be.good\\
\lspbottomrule
\end{tabular}
\ea
\glt 
‘and then Iskia held my knee \bluebold{that} is not well’ \textstyleExampleSource{[080916-001-CvNP.0003]}
\z


Conjunctions linking same-type constituents are described in §14.2 and those linking different-type constituents are discussed in §14.3. Unless mentioned otherwise, the clausal conjunctions combine clauses with same-subject coreference as well as those with a switch in reference. Juxtaposition is briefly mentioned in §14.4. The main points of this chapter are summarized in §14.5.
\end{styleBodyxvafter}

\section{Conjunctions combining same-type constituents}
\label{bkm:Ref374456904}\label{bkm:Ref374455378}\label{bkm:Ref374434199}\label{bkm:Ref356411659}
This section discusses conjunctions which combine same-type constituents. In terms of the semantic relations which they signal, the conjunctions fall into six groups, that is conjunctions marking addition (§14.2.1), alternative (§14.2.2), time and/or conditions (§14.2.3), consequence (§14.2.4), contrast (§14.2.5), and similarity (§14.2.6).
\end{styleBodyxvafter}

\subsection{Addition}
\label{bkm:Ref356049218}
Addition-marking conjunctions combine constituents denoting events, states, or entities which are “closely linked and … valid simultaneously”, employing {Rudolph’s (1996: 20)} definition.



Papuan Malay employs three addition-marking conjunctions. Most often addition is encoded with the comitative preposition \textitbf{dengang} ‘with’ (670 tokens); as a conjunction, it typically conjoins noun phrases, as discussed in §14.2.1.1. Much less often, Papuan Malay employs conjunctive \textitbf{dang} ‘and’ (24 tokens); it typically joins two clauses, as described in §14.2.1.2. Even less often, addition is encoded with the goal-oriented preposition \textitbf{sama} ‘to’; as a conjunction it links noun phrases with human referents (8 tokens), as shown in §14.2.1.3. (For details on variation in word class membership is discussed in §5.14.)
\end{styleBodyvxvafter}

\paragraph[Comitative dengang ‘with’]{Comitative \textitbf{dengang} ‘with’}
\label{bkm:Ref292894266}
The comitative preposition \textitbf{dengang} ‘with’, with its short form \textitbf{deng}, typically conjoins noun phrases (654 tokens). The conjoined referents can be animate, as in (0), or inanimate, as in (0). The fact that Papuan Malay employs the same marker for “noun phrase conjunction and comitative phrases” suggests that, in terms of {Stassen’s (2011a: 1)} typology, Papuan Malay is a “\textsc{with}{}-language”. Occasionally, \textitbf{deng(ang)} ‘with’ also links verb phrases (16 tokens) as in (0). The linking of clauses with comitative \textitbf{dengang} ‘with’ is unattested in the corpus. (Besides, comitative \textitbf{dengang} ‘with’ is also used to encode inclusory conjunction constructions, as discussed in §6.1.4; for a detailed discussion of preposition \textitbf{dengang} ‘with’, see §10.2.1.)
\end{styleBodyxafter}

\begin{tabular}{llllllll}
\lsptoprule
\label{bkm:Ref341895838}
\gll {bapa} {\bluebold{dengang}} {bapa-tua} {pi} {biking} {kebung} {…}\\ %
& father & with & older.uncle & go & make & garden & \\
\lspbottomrule
\end{tabular}
\ea
\glt 
‘father \bluebold{and} uncle went to work in the garden (together) …’ \textstyleExampleSource{[080922-001a-CvPh.0629]}
\z

\begin{tabular}{llllllllll}
\lsptoprule
\label{bkm:Ref373848882}
\gll {…} {apa} {biologi} {\bluebold{dengang}} {apa} {astronomi} {\bluebold{dengang}} {bahasa} {Inggris}\\ %
&  & what & biology & with & what & astronomy & with & language & England\\
\lspbottomrule
\end{tabular}
\ea
\glt 
[About a school competition] ‘[later they’ll participate in the Olympiad contest in,] what-is-it, biology \bluebold{and}, what-is-it, astronomy \bluebold{and} English’ \textstyleExampleSource{[081115-001a-Cv.0111-0113]}
\z

\begin{tabular}{llllllll}
\lsptoprule
\label{bkm:Ref355092168}
\gll {nene} {jam} {dua} {malam} {datang} {\bluebold{deng}} {menangis}\\ %
& grandmother & hour & two & night & come & with & cry\\
\lspbottomrule
\end{tabular}
\ea
\glt
‘at two o’clock in the morning grandmother came crying’ (Lit. ‘come \bluebold{with} cry’) \textstyleExampleSource{[081014-008-CvNP.0001]}
\end{styleFreeTranslEngxvpt}

\paragraph[Conjunctive dang ‘and’]{Conjunctive \textitbf{dang} ‘and’}
\label{bkm:Ref361251454}
The conjunction \textitbf{dang} ‘and’ typically links two clauses (168 tokens), as in (0). Less often, it links noun phrases (24 tokens), as in (0) and (0), or verb phrases (10 tokens) as in (0). Usually, the noun phrases have human referents as in (0); coordination of inanimate referents, as in (0), is rare.
\end{styleBodyxafter}

\begin{tabular}{llllllllllll}
\lsptoprule
\label{bkm:Ref355093146}
\gll {de} {\multicolumn{2}{l}{pegang}} {de} {punya} {prahu,} {\bluebold{dang}} {de} {dayung,} {\bluebold{dang}} {de}\\ %
& \textsc{3sg} & \multicolumn{2}{l}{hold} & \textsc{3sg} & \textsc{poss} & boat & and & \textsc{3sg} & paddle & and & \textsc{3sg}\\
& \multicolumn{2}{l}{bilang,} & \multicolumn{9}{l}{…}\\
& \multicolumn{2}{l}{say} & \multicolumn{9}{l}{}\\
\lspbottomrule
\end{tabular}
\ea
\glt 
‘he took his boat \bluebold{and} he paddled \bluebold{and} he said, …’ \textstyleExampleSource{[080917-008-NP.0018]}
\z

\begin{tabular}{llllllllll}
\lsptoprule
\label{bkm:Ref355093166}
\gll {sa} {kas} {taw} {mama} {\bluebold{dang}} {mama-ade,} {nanti} {kam} {…}\\ %
& \textsc{1sg} & give & know & mother & and & aunt & very.soon & \textsc{2pl} & \\
\lspbottomrule
\end{tabular}
\ea
\glt 
‘I let mother \bluebold{and} aunt know, ‘later you …’’ \textstyleExampleSource{[080919-007-CvNP.0001]}
\z

\begin{tabular}{lllllll}
\lsptoprule
\label{bkm:Ref341895858}
\gll {de} {suda} {taw} {ruma} {\bluebold{dang}} {kampung}\\ %
& \textsc{3sg} & already & know & house & and & village\\
\lspbottomrule
\end{tabular}
\ea
\glt 
‘he already knew the house \bluebold{and} the village’ \textstyleExampleSource{[080923-006-CvNP.0002]}
\z

\begin{tabular}{llllllll}
\lsptoprule
\label{bkm:Ref355093069}
\gll {pagi} {helikopter} {turung} {\bluebold{dang}} {kembali} {ke} {Anggruk}\\ %
& morning & helicopter & descend & and & return & to & Anggruk\\
\lspbottomrule
\end{tabular}
\ea
\glt
‘in the morning the helicopter came down \bluebold{and} returned to Anggruk’ \textstyleExampleSource{[081011-022-Cv.0228]}
\end{styleFreeTranslEngxvpt}

\paragraph[Goal{}-oriented sama ‘to’]{Goal-oriented \textitbf{sama} ‘to’}
\label{bkm:Ref438300440}
The goal-oriented preposition \textitbf{sama} ‘to’ occasionally links noun phrases with human referents (8 tokens), as in (0). The coordination of clauses or verb phrases with \textitbf{sama} ‘to’ is unattested in the corpus. Goal-oriented \textitbf{sama} ‘to’ has trial word class membership. That is, besides being used as a preposition and an addition-marking conjunction, it is also used as the stative verb \textitbf{sama} ‘be same’ (see §5.14; see also §10.2.2 for a detailed discussion of preposition \textitbf{sama} ‘to’ and how it is distinct from comitative \textitbf{dengang} ‘with’).\footnote{\\
\\
\\
\\
\\
\\
\\
\\
\\
\\
\\
\\
\\
\\
\\
\par As mentioned in §10.2.2, the goal preposition \textitbf{sama} ‘to’ is rather general in its meaning. Typically it translates with ‘to’ but depending on its context it also translates with ‘of, from, with’. For more information regarding the etymology of \textitbf{sama} ‘to’ see Footnote 225 in §10.2.2 (p. \pageref{bkm:Ref438368001}).}
\end{styleBodyxafter}

\begin{tabular}{lllllllllllll}
\lsptoprule
\label{bkm:Ref355105711}
\gll {…} {\multicolumn{2}{l}{Aris}} {\multicolumn{2}{l}{\bluebold{sama}}} {Siduas\textsuperscript{i}} {deng} {de\textsuperscript{i}} {pu} {maytua,} {\bluebold{sama}} {dep\textsuperscript{i},}\\ %
&  & \multicolumn{2}{l}{Aris} & \multicolumn{2}{l}{to} & Siduas & with & \textsc{3sg} & \textsc{poss} & wife & to & \textsc{3sg:poss}\\
& \multicolumn{2}{l}{de\textsuperscript{i}} & \multicolumn{2}{l}{punya} & \multicolumn{8}{l}{maytua}\\
& \multicolumn{2}{l}{\textsc{3sg}} & \multicolumn{2}{l}{\textsc{poss}} & \multicolumn{8}{l}{wife}\\
\lspbottomrule
\end{tabular}
\ea
\glt
‘[all (of you will) be taken (on board …,)] Aris \bluebold{and} Siduas\textsuperscript{i} and his wife\textsuperscript{i}, \bluebold{and} his\textsuperscript{i}, his\textsuperscript{i} wife’ \textstyleExampleSource{[080922-001a-CvPh.0493/0497]}\footnote{\\
\\
\\
\\
\\
\\
\\
\\
\\
\\
\\
\\
\\
\\
\\
\par The subscript letters indicate which personal pronouns have which referents.}
\end{styleFreeTranslEngxvpt}

\subsection{Alternative}
\label{bkm:Ref356654779}
In Papuan Malay, two conjunctions mark alternative, namely disjunctive \textitbf{ato} ‘or’ (§14.2.2.1) and disjunctive \textitbf{ka} ‘or’ (§14.2.2.2).
\end{styleBodyxvafter}

\paragraph[Disjunctive ato ‘or’]{Disjunctive \textitbf{ato} ‘or’}
\label{bkm:Ref361249557}
Generally speaking, the notion of disjunction is defined as “a logical relationship between propositions” in the sense that “[i]f the logical disjunction of two propositions is true, then one or both of the component propositions can be true” {\citep[305]{Payne1997}}.



In Papuan Malay, disjunction is marked with \textitbf{ato} ‘or’ which always occurs at the left periphery of the constituents it combines. Most often, disjunctive \textitbf{ato} ‘or’ joins clauses, as in (0). Also quite often, \textitbf{ato} ‘or’ links noun phrases as in (0). Only rarely \textitbf{ato} ‘or’ links prepositional phrases as in (0), or verb phrases as in (0).
\end{styleBodyvxafter}

\begin{tabular}{lllllllllll}
\lsptoprule
\label{bkm:Ref355117317}
\gll {\multicolumn{2}{l}{kalo}} {saya} {\multicolumn{2}{l}{susa,}} {\bluebold{ato}} {saya} {biking} {acara,} {nanti}\\ %
& \multicolumn{2}{l}{if} & \textsc{1sg} & \multicolumn{2}{l}{be.difficult} & or & \textsc{1sg} & make & ceremony & very.soon\\
& dia & \multicolumn{3}{l}{bantu} & \multicolumn{6}{l}{saya}\\
& \textsc{3sg} & \multicolumn{3}{l}{help} & \multicolumn{6}{l}{\textsc{1sg}}\\
\lspbottomrule
\end{tabular}
\ea
\glt 
‘if I have difficulties \bluebold{or} I make a festivity, then he’ll help me’ \textstyleExampleSource{[080919-004-NP.0065]}
\z

\begin{tabular}{lllllllllllllll}
\lsptoprule
\label{bkm:Ref355117320}
\gll {kalo} {\multicolumn{2}{l}{tong}} {\multicolumn{2}{l}{pu}} {\multicolumn{2}{l}{uang}} {\multicolumn{2}{l}{satu}} {juta,} {\bluebold{ato}} {satu} {juta} {lima}\\ %
& if & \multicolumn{2}{l}{\textsc{1pl}} & \multicolumn{2}{l}{\textsc{poss}} & \multicolumn{2}{l}{money} & \multicolumn{2}{l}{one} & million & or & one & million & five\\
& \multicolumn{2}{l}{ratus,} & \multicolumn{2}{l}{tong} & \multicolumn{2}{l}{bisa} & \multicolumn{2}{l}{bakar} & \multicolumn{6}{l}{natal}\\
& \multicolumn{2}{l}{hundred} & \multicolumn{2}{l}{\textsc{1pl}} & \multicolumn{2}{l}{be.able} & \multicolumn{2}{l}{burn} & \multicolumn{6}{l}{Christmas}\\
\lspbottomrule
\end{tabular}
\ea
\glt 
‘if we had one million \bluebold{or} one million five hundred (thousand rupiah), we could have a Christmas party’ (Lit. ‘burn (the) Christmas (fire)’) \textstyleExampleSource{[081006-017-Cv.0016]}
\z

\begin{tabular}{lllllllllllllllllll}
\lsptoprule
\label{bkm:Ref355117323}
\gll {jadi} {\multicolumn{2}{l}{kalo}} {\multicolumn{2}{l}{dia,}} {\multicolumn{3}{l}{suku}} {\multicolumn{2}{l}{dari}} {\multicolumn{3}{l}{situ,}} {dari} {\multicolumn{3}{l}{Masep}} {suda}\\ %
& so & \multicolumn{2}{l}{if} & \multicolumn{2}{l}{\textsc{3sg}} & \multicolumn{3}{l}{ethnic.group} & \multicolumn{2}{l}{from} & \multicolumn{3}{l}{\textsc{l.med}} & from & \multicolumn{3}{l}{Masep} & already\\
& \multicolumn{2}{l}{bunu} & \multicolumn{2}{l}{orang} & \multicolumn{2}{l}{di,} & a, & \multicolumn{2}{l}{Karfasia,} & \multicolumn{2}{l}{\bluebold{ato}} & di & \multicolumn{3}{l}{Waim,} & na & \multicolumn{2}{l}{…}\\
& \multicolumn{2}{l}{kill} & \multicolumn{2}{l}{person} & \multicolumn{2}{l}{at} & umh & \multicolumn{2}{l}{Karfasia} & \multicolumn{2}{l}{or} & at & \multicolumn{3}{l}{Waim} & well & \multicolumn{2}{l}{}\\
\lspbottomrule
\end{tabular}
\ea
\glt 
‘so if it, the ethnic group from there, from Masep has already killed someone at, umh Karfasia \bluebold{or} at Waim, well …’ \textstyleExampleSource{[081006-027-CvEx.0002]}
\z

\begin{tabular}{llllllllll}
\lsptoprule
\label{bkm:Ref355117290}
\gll {dong} {bilang,} {a,} {tunggu} {minum} {dulu,} {\bluebold{ato}} {makang} {dulu}\\ %
& \textsc{3pl} & say & ah! & wait & drink & first & or & eat & first\\
\lspbottomrule
\end{tabular}
\ea
\glt
‘they said, ‘ah, wait, please drink \bluebold{or} eat’’ (Lit. ‘drink first \bluebold{or} eat first’) \textstyleExampleSource{[080925-003-Cv.0111]}
\end{styleFreeTranslEngxvpt}

\paragraph[Disjunctive ka ‘or’]{Disjunctive \textitbf{ka} ‘or’}
\label{bkm:Ref361249558}
Disjunctive \textitbf{ka} ‘or’ signals series or sequences of alternatives. Occurring at the right periphery of a constituent, it indicates that a list of alternatives is not exhaustive. That is, a few possible options are overtly mentioned, while others are implied. To make the notion of ‘non-exhaustive list of alternatives’ explicit, the conjunction marks an interrogative as the final enumerated constituent. Typically, disjunctive \textitbf{ka} ‘or’ links noun phrases, as in (0) and (0). In (0), the notion of a ‘non-exhaustive list’ is implied, while in (0) it is overtly marked with \textitbf{apa ka} ‘or something else’ (literally ‘what or’). Less often, \textitbf{ka} ‘or’ combines prepositional phrases as in (0), or clauses as in (0); the linking of verbs with \textitbf{ka} ‘or’ is unattested in the corpus. Another function of \textitbf{ka} ‘or’, not discussed here, is to mark interrogative clauses (see §13.2.3).
\end{styleBodyxafter}

\begin{tabular}{lllllllll}
\lsptoprule
\label{bkm:Ref356547270}
\gll {…} {nanti} {banjir} {\bluebold{ka},} {hujang} {\bluebold{ka},} {guntur} {\bluebold{ka}}\\ %
&  & very.soon & flooding & or & rain & or & thunder & or\\
\lspbottomrule
\end{tabular}
\ea
\glt 
‘[it’s not allowed to kill the snake otherwise] later (there’ll be) flooding, \bluebold{or} rains, \bluebold{or} thunder (\bluebold{or something else})’ \textstyleExampleSource{[081006-022-CvEx.0004]}
\z

\begin{tabular}{llllllllllllll}
\lsptoprule
\label{bkm:Ref356484245}
\gll {sa} {\multicolumn{2}{l}{deng}} {\multicolumn{3}{l}{kaka}} {Petrus} {pikir,} {mungking} {klapa} {\bluebold{ka},} {\bluebold{apa}} {\bluebold{ka}}\\ %
& \textsc{1sg} & \multicolumn{2}{l}{with} & \multicolumn{3}{l}{oSb} & Petrus & think & maybe & coconut & or & what & or\\
& \multicolumn{2}{l}{yang} & \multicolumn{2}{l}{ada} & di & \multicolumn{8}{l}{depang}\\
& \multicolumn{2}{l}{\textsc{rel}} & \multicolumn{2}{l}{exist} & at & \multicolumn{8}{l}{front}\\
\lspbottomrule
\end{tabular}
\ea
\glt 
[About a motorbike trip:] ‘I and older brother Petrus thought, ‘maybe it is a coconut \bluebold{or something else} that is in front (of us)’’ \textstyleExampleSource{[081023-004-Cv.0002]}
\z

\begin{tabular}{lllllllll}
\lsptoprule
\label{bkm:Ref356484246}
\gll {ko} {lapor} {di} {umum} {\bluebold{ka},} {di} {keuangang} {\bluebold{ka}}\\ %
& \textsc{2sg} & report & at & general & or & at & finance.affairs & or\\
\lspbottomrule
\end{tabular}
\ea
\glt 
[About a government office:] ‘you (should) report to the general (office), \bluebold{or} the finance (office) (\bluebold{or some other} office)’ \textstyleExampleSource{[081005-001-Cv.0011]}
\z

\begin{tabular}{llllllllllllllllllll}
\lsptoprule
\label{bkm:Ref356484248}
\gll {…} {\multicolumn{2}{l}{waktu}} {\multicolumn{2}{l}{ko}} {\multicolumn{2}{l}{ada}} {\multicolumn{2}{l}{potong}} {\multicolumn{2}{l}{babi}} {\multicolumn{2}{l}{\bluebold{ka},}} {\multicolumn{2}{l}{potong}} {\multicolumn{2}{l}{ikang}} {\bluebold{ka},} {ato}\\ %
&  & \multicolumn{2}{l}{time} & \multicolumn{2}{l}{\textsc{2sg}} & \multicolumn{2}{l}{exist} & \multicolumn{2}{l}{cut} & \multicolumn{2}{l}{pig} & \multicolumn{2}{l}{or} & \multicolumn{2}{l}{cut} & \multicolumn{2}{l}{fish} & or & or\\
& \multicolumn{2}{l}{dapat} & \multicolumn{2}{l}{ikang} & \multicolumn{2}{l}{ka} & \multicolumn{2}{l}{kuskus} & \multicolumn{2}{l}{ka,} & \multicolumn{2}{l}{waktu} & \multicolumn{2}{l}{lewat} & \multicolumn{2}{l}{kasi} & \multicolumn{3}{l}{saja}\\
& \multicolumn{2}{l}{get} & \multicolumn{2}{l}{fish} & \multicolumn{2}{l}{or} & \multicolumn{2}{l}{cuscus} & \multicolumn{2}{l}{or} & \multicolumn{2}{l}{time} & \multicolumn{2}{l}{pass.by} & \multicolumn{2}{l}{give} & \multicolumn{3}{l}{just}\\
\lspbottomrule
\end{tabular}
\ea
\glt
‘[when (your) friends and relatives,] when you are carving a pig \bluebold{or} carving fish (\bluebold{or} carving \bluebold{something else}), or (when you) get a fish or cuscus (or something else), when (they) walk by, just share (it with them)’ \textstyleExampleSource{[080919-004-NP.0060]}
\end{styleFreeTranslEngxvpt}

\subsection{Time and/or condition}
\label{bkm:Ref356409384}
Papuan Malay conjunctions marking temporal relations indicate relative time; that is, the temporal reference point is determined by the context. Providing a reference point for the events or states depicted in the unmarked clause, time-marking conjunctions signal sequence relations, anteriority, or posteriority. Condition-marking conjunctions introduce clauses which expresses conditions, while the unmarked clauses describe event or states which could come about once the conditions have been met.



In many languages, there is no distinction between conditional ‘if’ and temporal ‘when’ clauses, as {\citet[257]{ThompsonEtAl2007}} point out. This also applies to Papuan Malay. Therefore, both types of linkings are discussed here.
\end{styleBodyvafter}


This section describes five conjunctions: sequential \textitbf{trus} ‘next’ (§14.2.3.1) and \textitbf{baru} ‘and then’ (§14.2.3.2), anteriority-marking \textitbf{sampe} ‘until’ (§14.2.3.3) and \textitbf{seblum} ‘before’ (§14.2.3.4), and posteriority-marking/conditional \textitbf{kalo} ‘when, if’ (§14.2.3.5).\footnote{\\
\\
\\
\\
\\
\\
\\
\\
\\
\\
\\
\\
\\
\\
\\

‘\bluebold{when} I … lived in the village, I worked like a man’ (Lit. ‘\bluebold{(at that) time}’) [081014-007-Pr.0048]\par }
\end{styleBodyvxvafter}

\paragraph[Sequential trus ‘next’]{Sequential \textitbf{trus} ‘next’}
\label{bkm:Ref356233464}
The sequential conjunction \textitbf{trus} ‘next’ marks temporal relations between clauses or phrases in an iconic way by organizing events in their logical and temporal order. When combining clauses, \textitbf{trus} ‘next’ always occurs in clause-initial position. The conjunction has dual word class membership; it is also used as the monovalent verb \textitbf{trus} ‘be continuous’ (see §5.14).



In terms of subject reference, an initial investigation of the attested \textitbf{trus} ‘next’ tokens in the corpus suggests the following. The conjunction more often links clauses with a switch in reference (269 tokens), as in (0), than those with same-subject coreference (101 tokens). This quantitative data is in contrast to {Donohue’s (2003: 31)} observations that \textitbf{trus} ‘next’ “is a commonly used connective when there is same-subject coreference condition between clauses”. Less often, \textitbf{trus} ‘next’ combines noun phrases, as in (0), or prepositional phrases, as in (0).
\end{styleBodyvxafter}

\begin{tabular}{llllllllllllllllllllllll}
\lsptoprule
\label{bkm:Ref355282772}
\gll {\multicolumn{2}{l}{waktu}} {\multicolumn{2}{l}{Sofia}} {\multicolumn{5}{l}{lewat}} {\multicolumn{3}{l}{mandi}} {\multicolumn{3}{l}{to?}} {di} {\multicolumn{2}{l}{kamar}} {\multicolumn{2}{l}{mandi,}} {\multicolumn{2}{l}{\bluebold{trus}}} {Nusa}\\ %
& \multicolumn{2}{l}{when} & \multicolumn{2}{l}{Sofia} & \multicolumn{5}{l}{pass.by} & \multicolumn{3}{l}{bathe} & \multicolumn{3}{l}{right?} & at & \multicolumn{2}{l}{room} & \multicolumn{2}{l}{bathe} & \multicolumn{2}{l}{next} & Nusa\\
& juga & \multicolumn{4}{l}{lewat,} & \multicolumn{2}{l}{Sofia} & \multicolumn{4}{l}{ikat} & \multicolumn{3}{l}{handuk,} & \multicolumn{2}{l}{de} & mo & \multicolumn{2}{l}{lewat} & \multicolumn{2}{l}{masuk} & \multicolumn{2}{l}{ke}\\
& also & \multicolumn{4}{l}{pass.by} & \multicolumn{2}{l}{Sofia} & \multicolumn{4}{l}{tie.up} & \multicolumn{3}{l}{towel} & \multicolumn{2}{l}{\textsc{3sg}} & want & \multicolumn{2}{l}{pass.by} & \multicolumn{2}{l}{enter} & \multicolumn{2}{l}{to}\\
& \multicolumn{3}{l}{kamar,} & \multicolumn{3}{l}{\bluebold{trus}} & \multicolumn{2}{l}{Nusa} & \multicolumn{2}{l}{de} & \multicolumn{3}{l}{bicara} & \multicolumn{10}{l}{dia}\\
& \multicolumn{3}{l}{room} & \multicolumn{3}{l}{next} & \multicolumn{2}{l}{Nusa} & \multicolumn{2}{l}{\textsc{3sg}} & \multicolumn{3}{l}{speak} & \multicolumn{10}{l}{\textsc{3sg}}\\
\lspbottomrule
\end{tabular}
\ea
\glt 
‘when Sofia passed by to bathe, right?, in the bathroom, \bluebold{then} Nusa also passed by, Sofia had tied (her) towel (around her waist), she wanted to pass by (and) enter the (bath)room, \bluebold{then} Nusa spoke to her’ \textstyleExampleSource{[081115-001a-Cv.0263]}
\z

\begin{tabular}{lllllllm{-2.4015456E-4in}llllllll}
\lsptoprule
\label{bkm:Ref355342205}
\gll {de} {\multicolumn{2}{l}{pu}} {potong} {\multicolumn{3}{l}{selesay}} {\multicolumn{2}{l}{ambil}} {\multicolumn{2}{l}{ubi,}} {\multicolumn{2}{l}{\bluebold{trus}}} {daung} {petatas}\\ %
& \textsc{3sg} & \multicolumn{2}{l}{\textsc{poss}} & cut & \multicolumn{3}{l}{finish} & \multicolumn{2}{l}{get} & \multicolumn{2}{l}{purple.yam} & \multicolumn{2}{l}{next} & leaf & sweet.potato\\
& \multicolumn{2}{l}{daung} & \multicolumn{3}{l}{singkong,} & \bluebold{trus} & \multicolumn{2}{l}{apa} & \multicolumn{2}{l}{lagi} & \multicolumn{2}{l}{sayur} & \multicolumn{3}{l}{bayam}\\
& \multicolumn{2}{l}{leaf} & \multicolumn{3}{l}{cassava} & next & \multicolumn{2}{l}{what} & \multicolumn{2}{l}{again} & \multicolumn{2}{l}{vegetable} & \multicolumn{3}{l}{amaranth}\\
\lspbottomrule
\end{tabular}
\ea
\glt 
[A recipe:] ‘(once) the cutting up (of the pig meat) is done, take purple yam, \bluebold{then} sweet potato leaves, cassava leaves, \bluebold{then} what else amaranth vegetables’ \textstyleExampleSource{[081014-017-CvPr.0033]}
\z

\begin{tabular}{llllllllllllll}
\lsptoprule
\label{bkm:Ref355342206}
\gll {…} {\multicolumn{3}{l}{jalang}} {\multicolumn{2}{l}{banyak}} {to?,} {di} {atas,} {tenga,} {\bluebold{trus}} {di} {laut,}\\ %
&  & \multicolumn{3}{l}{road} & \multicolumn{2}{l}{many} & right? & at & top & middle & next & at & sea\\
& \multicolumn{2}{l}{\bluebold{trus}} & di & \multicolumn{2}{l}{pante} & \multicolumn{8}{l}{sana}\\
& \multicolumn{2}{l}{then} & at & \multicolumn{2}{l}{coast} & \multicolumn{8}{l}{\textsc{l.dist}}\\
\lspbottomrule
\end{tabular}
\ea
\glt
‘[I was confused (about) the road, you know,] (there) were many roads, right?, in the upper part (of the village), in the middle, \bluebold{and then} at the sea, \bluebold{and then} at the beach over there’ \textstyleExampleSource{[081025-008-Cv.0018]}
\end{styleFreeTranslEngxvpt}

\paragraph[Sequential baru ‘and then’]{Sequential \textitbf{baru} ‘and then’}
\label{bkm:Ref355429821}
The sequential conjunction \textitbf{baru} ‘and then’ most commonly also marks temporal succession by ordering events in their logical and temporal sequence, as shown in (0). In addition, although less often, the conjunction introduces contrast clauses, as illustrated in (0). The conjunction has dual word class membership; it is also used as the stative verb \textitbf{baru} ‘be new’ (see §5.14).



Typically, \textitbf{baru} ‘and then’ occurs in clause-initial position where it marks an immediate subsequent event or action, similar to sequential \textitbf{trus} ‘next’ (§14.2.3.1). Concurrently, however, the conjunction signals another piece of information, as shown in (0) (note that this example presents contiguous text). Depending on the context, the conjunction marks noteworthy parts and/or signals a new aspect or perspective regarding the event or discourse unfolding. In this case \textitbf{baru} translates with ‘but then’, as in (0) or ‘and then’ as in (0). Alternatively, the conjunction signals that the event depicted in its clause does not occur until after the event of the preceding clause. In this case, it translates with ‘only then’, as in (0). In marking contrastive sequentiality, \textitbf{baru} ‘and then’ differs from \textitbf{trus} ‘next’ which indicates neutral sequentiality (see §14.2.3.1).
\end{styleBodyvafter}


As for subject reference, an initial inspection of the \textitbf{baru} ‘and then’ tokens in the corpus suggests that the conjunction more often links clauses with a switch in reference (524 tokens), as in (0), than clauses with same-subject coreference (455 tokens), as in (0, ). In this respect, \textitbf{baru} ‘and then’ behaves like \textitbf{trus} ‘next’ (see §14.2.3.1).
\end{styleBodyvvafter}

\begin{styleExampleTitle}
Combining clauses with \textitbf{baru} ‘and then’ in clause initial position: Sequential reading
\end{styleExampleTitle}

\begin{tabular}{llllllllllll}
\lsptoprule
\label{bkm:Ref355439273}\label{bkm:Ref355439277}
\gll {\label{bkm:Ref309979254}} {tong} {…} {jaga} {dia\textsuperscript{i}} {sampe} {jam} {satu,} {\bluebold{baru}} {tong} {tidor,}\\ %
&  & \textsc{1pl} &  & guard & \textsc{3sg} & until & hour & one & and.then & \textsc{1pl} & sleep\\
\lspbottomrule
\end{tabular}
\begin{styleFreeTranslAlphaEng}
[About a sick relative:] ‘we … watched her until one o’clock, \bluebold{only then} did we sleep’
\end{styleFreeTranslAlphaEng}

\begin{tabular}{lllllllllll} & \label{bkm:Ref355439292} & \bluebold{baru} & Pawlus & de\textsuperscript{j} & sandar & di & de\textsuperscript{i} & pu & badang & begini,\\
\lsptoprule
&  & and.then & Pawlus & \textsc{3sg} & lean & at & \textsc{3sg} & \textsc{poss} & body & like.this\\
\lspbottomrule
\end{tabular}
\begin{styleFreeTranslAlphaEng}
‘\bluebold{but then} Pawlus was leaning against her body like this’
\end{styleFreeTranslAlphaEng}

\begin{tabular}{lllllllllllll} & \label{bkm:Ref355439293} & \multicolumn{2}{l}{\bluebold{baru}} & \multicolumn{2}{l}{de\textsuperscript{j}} & kas & pata & leher & ke & bawa & di & atas\\
\lsptoprule
&  & \multicolumn{2}{l}{and.then} & \multicolumn{2}{l}{\textsc{3sg}} & give & break & neck & to & bottom & at & top\\
&  & de\textsuperscript{i} & \multicolumn{2}{l}{pu} & \multicolumn{8}{l}{bahu}\\
&  & \textsc{3sg} & \multicolumn{2}{l}{\textsc{poss}} & \multicolumn{8}{l}{shoulder}\\
\lspbottomrule
\end{tabular}
\begin{styleFreeTranslAlphaEng}
‘\bluebold{and then} he bent his neck down onto her shoulder’ (Lit. ‘caused his head to be broken’) \textstyleExampleSource{[080916-001-CvNP.0005-0006]}
\end{styleFreeTranslAlphaEng}


Occasionally, the conjunction occurs at the right periphery of a contrast clause. Summarizing what has been said before, it marks the propositional content of its clause as true despite the contents of the preceding unmarked clause. In this case, the conjunction receives the counter-expectational reading ‘after all’, as in (0). As this contrast-marking function of the conjunction is marginal, it is not further discussed in §14.2.5.


\begin{styleExampleTitle}
Combining clauses with \textitbf{baru} ‘and then’ in clause final position: Counter-expectational reading
\end{styleExampleTitle}

\begin{tabular}{lllllllllllll}
\lsptoprule
\label{bkm:Ref355892488}
\gll {sa} {\multicolumn{2}{l}{tra}} {\multicolumn{2}{l}{akang}} {kasi} {kaing,} {sa} {juga} {dinging} {stenga} {mati,}\\ %
& \textsc{1sg} & \multicolumn{2}{l}{\textsc{neg}} & \multicolumn{2}{l}{will} & give & cloth & \textsc{1sg} & also & be.cold & half & dead\\
& \multicolumn{2}{l}{ada} & \multicolumn{2}{l}{anging} & \multicolumn{8}{l}{\bluebold{baru}}\\
& \multicolumn{2}{l}{exist} & \multicolumn{2}{l}{wind} & \multicolumn{8}{l}{and.then}\\
\lspbottomrule
\end{tabular}
\ea
\glt
‘I wasn’t going to give (her my) cloth, I was also half dead (from being) cold, it was windy \bluebold{after all}’ \textstyleExampleSource{[081025-006-Cv.0048]}
\end{styleFreeTranslEngxvpt}

\paragraph[Anteriority{}-marking sampe ‘until’]{Anteriority-marking \textitbf{sampe} ‘until’}
\label{bkm:Ref356288962}
The conjunction \textitbf{sampe} ‘until’ introduces a temporal clause which follows the unmarked clause. The conjunction has trial word class membership; that is, besides being used as a conjunction, it is also used as the bivalent verb \textitbf{sampe} ‘reach’ and as the temporal preposition \textitbf{sampe} ‘until’ (see §5.14; see also §10.1.4 for its prepositional uses).



Usually, \textitbf{sampe} ‘until’ marks anteriority. That is, it signals that the event or state of the unmarked clause occurs prior to that of the temporal clause, as shown in (0). Concurrently, \textitbf{sampe} ‘until’ marks temporal extent in that it indicates that the event or state of the unmarked clause continues until the event or state of the temporal clause comes about. Depending on the context, temporal \textitbf{sampe} ‘until’ can also receive a resultative reading in the sense of ‘with the result that’, as in (0). Given that the resultative reading of \textitbf{sampe} ‘until’ is the derived, marginal one, this result-marking function of \textitbf{sampe} ‘until’ is not further discussed in §14.2.4.
\end{styleBodyvxafter}

\begin{tabular}{lllllllllll}
\lsptoprule
\label{bkm:Ref355803520}
\gll {…} {de} {harus} {taru} {di} {mata-hari,} {\bluebold{sampe}} {de} {jadi} {papeda}\\ %
&  & \textsc{3sg} & have.to & put & at & sun & until & \textsc{3sg} & become & sagu.porridge\\
\lspbottomrule
\end{tabular}
\ea
\glt 
[Before an ancestor had fire to heat water:] ‘[when he wanted to make sagu porridge,] he had to leave (the sago) out in the sun \bluebold{until} it turned into sagu porridge’ \textstyleExampleSource{[080922-010a-CvNF.007-0008]}
\z

\begin{tabular}{llllllllllll}
\lsptoprule
\label{bkm:Ref355803521}
\gll {Fredi} {\multicolumn{2}{l}{pu}} {\multicolumn{2}{l}{tangang}} {dia} {palungku} {kaca,} {jadi} {dia} {rabik,}\\ %
& Fredi & \multicolumn{2}{l}{\textsc{poss}} & \multicolumn{2}{l}{hand} & \textsc{3sg} & punch & glass & so & \textsc{3sg} & tear\\
& \multicolumn{2}{l}{\bluebold{sampe}} & \multicolumn{2}{l}{brapa} & \multicolumn{7}{l}{jahitang}\\
& \multicolumn{2}{l}{until} & \multicolumn{2}{l}{several} & \multicolumn{7}{l}{stitch}\\
\lspbottomrule
\end{tabular}
\ea
\glt
[About an accident:] ‘Fredi’s hand hit glass, so it was torn \bluebold{with the result that} (he got) several stitches’ \textstyleExampleSource{[081006-032-Cv.0066]}
\end{styleFreeTranslEngxvpt}

\paragraph[Anteriority{}-marking seblum ‘before’]{Anteriority-marking \textitbf{seblum} ‘before’}
\label{bkm:Ref356289539}
Anteriority-marking \textitbf{seblum} ‘before’ also introduces a temporal clause.\footnote{\\
\\
\\
\\
\\
\\
\\
\\
\\
\\
\\
\\
\\
\\
\\
\par The conjunction \textitbf{seblum} ‘before’ is historically derived from the aspectual adverb \textitbf{blum} ‘not yet’: \textitbf{se-blum} ‘one-not.yet’ (see §5.4.1).} It indicates – similar to \textitbf{sampe} ‘until’ – that the event or state of the unmarked clause occurs prior to that of the temporal clause. Unlike \textitbf{sampe} ‘until’, however, \textitbf{seblum} ‘before’ does not signal extent. The temporal clause with \textitbf{seblum} ‘before’ can precede or follow the unmarked clause, as shown in (0) and (0), respectively. In the corpus, however, the temporal clause more often precedes the unmarked clause (21 tokens) rather than follows it (8 tokens).
\end{styleBodyxafter}

\begin{tabular}{llllllllllll}
\lsptoprule
\label{bkm:Ref356462163}
\gll {de} {bilang,} {\bluebold{seblum}} {kitong} {pergi} {ke} {kota,} {kitong} {cuci} {muka} {dulu}\\ %
& \textsc{3sg} & say & before & \textsc{1pl} & go & to & city & \textsc{1pl} & wash & front & first\\
\lspbottomrule
\end{tabular}
\ea
\glt 
‘he said, ‘\bluebold{before} we go to the city, we wash (our) faces first’’ \textstyleExampleSource{[080917-008-NP.0126]}
\z

\begin{tabular}{llllllllll}
\lsptoprule
\label{bkm:Ref355451193}
\gll {…} {saya} {suda} {punya} {rencana} {juga,} {\bluebold{seblum}} {sa} {kluar}\\ %
&  & \textsc{1sg} & already & have & plan & also & before & \textsc{1sg} & go.out\\
\lspbottomrule
\end{tabular}
\ea
\glt
‘[when I hunt without taking dogs, I leave in the night,] I also already have a plan \bluebold{before} I leave’ \textstyleExampleSource{[080919-004-NP.0002]}
\end{styleFreeTranslEngxvpt}

\paragraph[Posteriority{}-marking/conditional kalo ‘when, if’]{Posteriority{}-marking/conditional \textitbf{kalo} ‘when, if’}
\label{bkm:Ref356289540}
The conjunction \textitbf{kalo} ‘when, if’ signals temporal relations, namely posteriority, and/or conditional relations between two clauses. The clause it introduces always precedes the unmarked clause.



Whether \textitbf{kalo} ‘when, if’ receives a temporal reading as in (0) and (0), or a conditional reading, as in (0) and (0), is context-dependent. Quite often, though, both interpretations are possible, as shown in (0). As mentioned, this lack of a “distinction between ‘if’ clauses and ‘when’ clauses” is also found in other languages; examples are “Indonesian and certain languages of Papua New Guinea” s {(Thompson et al. 2007: 257)}.
\end{styleBodyvafter}


When marking posteriority, \textitbf{kalo} translates with ‘when’; it signals that the event or state of the unmarked main clause occurs subsequent to that of the temporal clause, as in (0). When the conjunction co-occurs with the retrospective adverb \textitbf{suda} ‘already’, or with its short form \textitbf{su}, it projects these events or states to the future; in this case \textitbf{kalo} translates with ‘once’. That is, in combination with \textitbf{suda} ‘already’, the conjunction signals that the event or state of the unmarked clause will eventuate, once that of the temporal clause has come about, as in (0).
\end{styleBodyvvafter}

\begin{styleExampleTitle}
Combining clauses with \textitbf{kalo} ‘when/after’: Temporal reading
\end{styleExampleTitle}

\begin{tabular}{lllllllllll}
\lsptoprule
\label{bkm:Ref355778340}
\gll {\bluebold{kalo}} {dong} {tendang} {de} {pu} {kaki} {tu,} {dia} {pegang} {bola}\\ %
& when & \textsc{3pl} & kick & \textsc{3sg} & \textsc{poss} & foot & \textsc{d.dist} & \textsc{3sg} & hold & ball\\
\lspbottomrule
\end{tabular}
\ea
\glt 
[About a football match:] ‘\bluebold{when} they kicked those legs of his, he grabbed the ball’ \textstyleExampleSource{[081006-014-Cv.0004]}
\z

\begin{tabular}{llllllllllllllll}
\lsptoprule
\label{bkm:Ref355784925}
\gll {jadi} {\bluebold{kalo}} {dong} {\multicolumn{2}{l}{\bluebold{su}}} {\multicolumn{3}{l}{tinggal}} {di} {\multicolumn{2}{l}{kota}} {\multicolumn{3}{l}{begini,}} {dong}\\ %
& so & if & \textsc{3pl} & \multicolumn{2}{l}{already} & \multicolumn{3}{l}{stay} & at & \multicolumn{2}{l}{city} & \multicolumn{3}{l}{like.this} & \textsc{3pl}\\
& \multicolumn{4}{l}{snang} & \multicolumn{2}{l}{tinggal,} & tida & \multicolumn{3}{l}{maw} & \multicolumn{2}{l}{pulang} & ke & \multicolumn{2}{l}{kampung}\\
& \multicolumn{4}{l}{feel.happy(.about)} & \multicolumn{2}{l}{stay} & \textsc{neg} & \multicolumn{3}{l}{want} & \multicolumn{2}{l}{go.home} & to & \multicolumn{2}{l}{village}\\
\lspbottomrule
\end{tabular}
\ea
\glt 
‘so \bluebold{once} they’ve lived in the city like this, they’re happy to stay (here), (they) don’t want to return home to the village’ \textstyleExampleSource{[080927-009-CvNP.0059]}
\z


In a different context, the conjunction receives a conditional reading and signals, what {\citet[6]{Kaufmann2006}} calls, “indicative conditional” relations or “counterfactual conditional” relations. In such a context \textitbf{kalo} translates with ‘if’. An indicative conditional indicates that it is possible for the condition presented in its clause to be met. In this case the event or state of the unmarked clause will also come about, as shown in (0). When conditional \textitbf{kalo} ‘if’ co-occurs with retrospective \textitbf{suda} ‘already’, the clause receives a counterfactual conditional reading. That is, it signals that the condition was not met in the past. If the condition had been met, however, then the event or state of the unmarked clause would also have come about. This is illustrated in (0).


\begin{styleExampleTitle}
Combining clauses with \textitbf{kalo} ‘if’: Conditional reading
\end{styleExampleTitle}

\begin{tabular}{lllllll}
\lsptoprule
\label{bkm:Ref387670262}
\gll {\bluebold{kalo}} {ko} {alpa,} {kitong} {tra} {jalang}\\ %
& if & \textsc{2sg} & be.absent & \textsc{1pl} & \textsc{neg} & walk\\
\lspbottomrule
\end{tabular}
\ea
\glt 
[Talking to her son about an upcoming trip:] ‘\bluebold{if} you play hooky, we won’t go’ \textstyleExampleSource{[080917-003a-CvEx.0038]}
\z

\begin{tabular}{lllllllllll}
\lsptoprule
\label{bkm:Ref355784927}
\gll {\bluebold{kalo}} {\multicolumn{2}{l}{sa}} {\multicolumn{2}{l}{\bluebold{su}}} {\multicolumn{2}{l}{pake}} {em} {kaca-mata} {tu,}\\ %
& if & \multicolumn{2}{l}{\textsc{1sg}} & \multicolumn{2}{l}{already} & \multicolumn{2}{l}{use} & uh & glasses & \textsc{d.dist}\\
& \multicolumn{2}{l}{mungking} & \multicolumn{2}{l}{sa} & \multicolumn{2}{l}{su} & \multicolumn{4}{l}{gila}\\
& \multicolumn{2}{l}{maybe} & \multicolumn{2}{l}{\textsc{1sg}} & \multicolumn{2}{l}{already} & \multicolumn{4}{l}{be.crazy}\\
\lspbottomrule
\end{tabular}
\ea
\glt 
‘\bluebold{if} I’d been wearing, uh, those (sun)glasses, I might already be crazy’ \textstyleExampleSource{[080919-005-Cv.0007]}
\z


Rather commonly, \textitbf{kalo} ‘when, if’ allows both a temporal and a conditional reading, as in (0).


\begin{styleExampleTitle}
Combining clauses with \textitbf{kalo} ‘when, if’: Temporal and/or conditional reading
\end{styleExampleTitle}

\begin{tabular}{llllll}
\lsptoprule
\label{bkm:Ref355778474}
\gll {\bluebold{kalo}} {bapa} {datang,} {pluk} {bapa}\\ %
& when/if & father & come & embrace & father\\
\lspbottomrule
\end{tabular}
\ea
\glt
‘\bluebold{when/if} you (‘father’) come (here), (I’ll) embrace you (‘father’)’ \textstyleExampleSource{[080922-001a-CvPh.0360]}
\end{styleFreeTranslEngxvpt}

\subsection{Consequence}
\label{bkm:Ref355803638}
A consequence-marking conjunction indicates that the event or state of its clause is the outcome of an event or state depicted in the unmarked clause. Papuan Malay has five such conjunctions: resultative/causal \textitbf{jadi} ‘so, since’ (§14.2.4.1), purposive \textitbf{supaya} ‘so that’ (§14.2.4.2), purposive \textitbf{untuk} ‘for’ (§14.2.4.3), causal \textitbf{karna} ‘because’ (§14.2.4.4), and causal \textitbf{gara-gara} ‘because’ (§14.2.4.5). In addition, although rarely, temporal \textitbf{sampe} ‘until’ has result-marking function in the sense of ‘with the result that’; given that this function is marginal, it is discussed in §14.2.3.3 and not here.
\end{styleBodyxvafter}

\paragraph[Resultative/causal jadi ‘so, since’]{Resultative/causal \textitbf{jadi} ‘so, since’}
\label{bkm:Ref355714776}
The resultative/causal conjunction \textitbf{jadi} ‘so, since’ most often marks a resultative relation between two clauses, as shown in (0). In addition, although less often, the conjunction signals a causal relation, as illustrated in (0). The conjunction has dual word class membership; it is also used as the bivalent verb \textitbf{jadi} ‘become’ (see §5.14).



Typically, \textitbf{jadi} ‘so, since’ occurs in initial position of a result clause that follows the unmarked clause. Here, the conjunction signals that the event or state of its clause results from that of the unmarked clause, as in (0); hence, \textitbf{jadi} translates with ‘so’.
\end{styleBodyvvafter}

\begin{styleExampleTitle}
Combining clauses with \textitbf{jadi} ‘so, since’: Clause-initial position
\end{styleExampleTitle}

\begin{tabular}{llllllllllll}
\lsptoprule
\label{bkm:Ref355625774}
\gll {tong} {\multicolumn{2}{l}{tra}} {\multicolumn{3}{l}{snang}} {dengang} {dia,} {\bluebold{jadi}} {kitong} {malas}\\ %
& \textsc{1pl} & \multicolumn{2}{l}{\textsc{neg}} & \multicolumn{3}{l}{feel.happy(.about)} & with & \textsc{3sg} & so & \textsc{1pl} & be.listless\\
& \multicolumn{2}{l}{datang} & \multicolumn{2}{l}{dia} & pu & \multicolumn{6}{l}{ruma}\\
& \multicolumn{2}{l}{come} & \multicolumn{2}{l}{\textsc{3sg}} & \textsc{poss} & \multicolumn{6}{l}{house}\\
\lspbottomrule
\end{tabular}
\ea
\glt 
‘we don’t feel happy about her, \bluebold{so} we don’t want (to) come to her house’ \textstyleExampleSource{[080927-006-CvNP.0032]}
\z


Alternatively, but less often, the conjunction occurs in clause-final position of a cause clause where it marks a causal relation with the preceding unmarked clause, as in (0). In this position, the conjunction signals that something depicted in its clause is the cause for the event or state of the unmarked clause, and that the result depicted in the unmarked clause is anticipated. Hence, \textitbf{jadi} translates with ‘since’. In that the result is expected, causal \textitbf{jadi} ‘since’ differs from neutral causality-marking \textitbf{karna} ‘because’ (see §14.2.4.4).


\begin{styleExampleTitle}
Combining clauses with \textitbf{jadi} ‘so, since’: Clause-final position
\end{styleExampleTitle}

\begin{tabular}{lllllllllll}
\lsptoprule
\label{bkm:Ref355623305}
\gll {Musa} {ini,} {e,} {de} {loyo{\Tilde}loyo} {ini,} {de} {bangung} {tidor} {\bluebold{jadi}}\\ %
& Musa & \textsc{d.prox} & uh & \textsc{3sg} & \textsc{rdp}{\Tilde}be.weak & this & \textsc{3sg} & wake.up & sleep & so\\
\lspbottomrule
\end{tabular}
\ea
\glt
[About a small boy:] ‘Musa here, uh, right now he’s kind of weak \bluebold{since} he woke up from sleeping’ \textstyleExampleSource{[080922-001a-CvPh.1435/1437]}
\end{styleFreeTranslEngxvpt}

\paragraph[Purposive supaya ‘so that’]{Purposive \textitbf{supaya} ‘so that’}
\label{bkm:Ref355852858}
Purposive \textitbf{supaya} ‘so that’ introduces a purpose clause which follows the unmarked clause. The conjunction signals that the event or state of its clause is the intended outcome of the deliberate activity depicted in the unmarked clause. Most often, \textitbf{supaya} ‘so that’ introduces a purpose clause with an overt subject (96/129 tokens – 74\%), as in (0). Less often, the conjunction introduces a purpose clause with elided subject (33/129 tokens – 26\%), as in (0).
\end{styleBodyxafter}

\begin{tabular}{llllllllll}
\lsptoprule
\label{bkm:Ref355856542}
\gll {mace} {ko} {\multicolumn{2}{l}{sendiri}} {yang} {ikut,} {\bluebold{supaya}} {ko} {atur}\\ %
& wife & \textsc{2sg} & \multicolumn{2}{l}{alone} & \textsc{rel} & follow & so.that & \textsc{2sg} & arrange\\
& \multicolumn{2}{l}{makangang} & di & \multicolumn{6}{l}{sana!}\\
& \multicolumn{2}{l}{food} & at & \multicolumn{6}{l}{\textsc{l.dist}}\\
\lspbottomrule
\end{tabular}
\ea
\glt 
‘you wife yourself (should) go with (them) \bluebold{so that} you organize the catering over there!’ (Lit. ‘(it’s) you wife yourself who …’) \textstyleExampleSource{[081025-009a-Cv.0032]}
\z

\begin{tabular}{llllllll}
\lsptoprule
\label{bkm:Ref355708526}
\gll {e,} {angkat} {muka,} {\bluebold{supaya}} {Ø} {liat} {orang!}\\ %
& hey! & lift & front & so.that &  & see & person\\
\lspbottomrule
\end{tabular}
\ea
\glt
‘hey, lift (your) face \bluebold{so that} (you) see (the other) people!’ \textstyleExampleSource{[081110-008-CvNP.0101]}
\end{styleFreeTranslEngxvpt}

\paragraph[Purposive untuk ‘for’]{Purposive \textitbf{untuk} ‘for’}
\label{bkm:Ref355852861}
As a conjunction, the benefactive preposition \textitbf{untuk} ‘for’ signals a purpose relation between two clauses (for a description of preposition \textitbf{untuk} ‘for’, see §10.2.3). Purposive \textitbf{untuk} ‘for’, like \textitbf{supaya} ‘so that’ (see §14.2.4.2), introduces a purpose clause which expresses the intended outcome of the purposeful activity depicted in the preceding unmarked clause, as shown in (0) and (0). Usually, \textitbf{untuk} ‘for’ introduces a purpose clause with an elided subject (115/163 tokens – 71\%), as shown with the second \textitbf{untuk} ‘for’ token in (0). Much less often the conjunction introduces a purpose clause with an overt subject (48/163tokens – 29\%), as shown with the first \textitbf{untuk} ‘for’ token in (0), or as in (0).
\end{styleBodyxafter}

\begin{tabular}{lllllllllllllllllll}
\lsptoprule
\label{bkm:Ref355856491}
\gll {\multicolumn{2}{l}{tadi}} {\multicolumn{2}{l}{ana}} {\multicolumn{2}{l}{bilang,}} {\multicolumn{2}{l}{…}} {bapa} {\multicolumn{2}{l}{dorang}} {\multicolumn{3}{l}{siap}} {\multicolumn{2}{l}{saja,}} {\bluebold{untuk}} {kita}\\ %
& \multicolumn{2}{l}{earlier} & \multicolumn{2}{l}{child} & \multicolumn{2}{l}{say} & \multicolumn{2}{l}{} & father & \multicolumn{2}{l}{\textsc{3pl}} & \multicolumn{3}{l}{get.ready} & \multicolumn{2}{l}{just} & for & \textsc{1pl}\\
& ke & \multicolumn{2}{l}{sana} & a, & sa & \multicolumn{2}{l}{juga} & \multicolumn{3}{l}{siap,} & \multicolumn{2}{l}{\bluebold{untuk}} & Ø & \multicolumn{2}{l}{bawa} & \multicolumn{3}{l}{kamu}\\
& to & \multicolumn{2}{l}{\textsc{l.dist}} & ah! & \textsc{1sg} & \multicolumn{2}{l}{also} & \multicolumn{3}{l}{get.ready} & \multicolumn{2}{l}{for} &  & \multicolumn{2}{l}{bring} & \multicolumn{3}{l}{\textsc{2pl}}\\
& ke & \multicolumn{17}{l}{sini}\\
& to & \multicolumn{17}{l}{\textsc{l.prox}}\\
\lspbottomrule
\end{tabular}
\ea
\glt 
‘a short while ago you (‘child’) said, ‘… father and the others are ready \bluebold{for} us (to move) to (Sarmi over) there’, ah (in that case) I’m also ready \bluebold{to} bring you (to Sarmi) here’ (Lit. ‘\bluebold{for} (me to) bring you’) \textstyleExampleSource{[080922-001a-CvPh.1241]}
\z

\begin{tabular}{llllllllll}
\lsptoprule
\label{bkm:Ref355856492}
\gll {…} {tida} {bole,} {ini,} {kamu} {datang,} {\bluebold{untuk}} {kamu} {skola}\\ %
&  & \textsc{neg} & may & \textsc{d.prox} & \textsc{2pl} & come & for & \textsc{2pl} & go.to.school\\
\lspbottomrule
\end{tabular}
\ea
\glt 
‘[you shouldn’t hate each other, (you) shouldn’t infuriate each other,] (you) shouldn’t (do all this), what’s-its-name, you came (here) \bluebold{to} go to school’ (Lit. ‘\bluebold{for} you (to) go to school’) \textstyleExampleSource{[081115-001a-Cv.0272]}
\z


The attested data indicate that \textitbf{untuk} ‘for’ differs from \textitbf{supaya} ‘so that’ in that \textitbf{untuk} ‘for’ most often introduces purpose clauses with elided subjects. By contrast, \textitbf{supaya} ‘so that’ most often introduces purpose clauses with overt subjects.
\end{styleBodyxvafter}

\paragraph[Causal karna ‘because’]{Causal \textitbf{karna} ‘because’}
\label{bkm:Ref355697115}
Causal \textitbf{karna} ‘because’ signals a neutral causal relation between two clauses by introducing a cause clause which gives the reason for the event or state depicted in the unmarked clause. Usually the cause clause follows the unmarked clause, as in (0). In combination with adversative \textitbf{tapi} ‘but’ (see §14.2.5.1), however, it can precede the unmarked clause, as in (0). In this case the unmarked clause is often introduced with resultative \textitbf{jadi} ‘so’. Signaling neutral causality, \textitbf{karna} ‘because’ is distinct from causal \textitbf{jadi} ‘since’ which marks expected results (see §14.2.4.1), and from causal \textitbf{gara-gara} ‘because’ which marks emotive causal relations (see §14.2.4.5).
\end{styleBodyxafter}

\begin{tabular}{lllllllll}
\lsptoprule
\label{bkm:Ref356229176}
\gll {saya} {bisa} {pulang,} {\bluebold{karna}} {sa} {su} {dapat} {babi}\\ %
& \textsc{1sg} & be.able & go.home & because & \textsc{1sg} & already & get & pig\\
\lspbottomrule
\end{tabular}
\ea
\glt 
[Hunting a wild pig:] ‘I can return home \bluebold{because} I already got the pig’ \textstyleExampleSource{[080919-004-NP.0024]}
\z

\begin{tabular}{llllllllllllllm{-9.4015896E-4in}llllll}
\lsptoprule
\label{bkm:Ref355696074}
\gll {dong} {\multicolumn{3}{l}{memang}} {\multicolumn{4}{l}{piara}} {\multicolumn{2}{l}{de}} {di} {\multicolumn{3}{l}{situ,}} {\bluebold{tapi}} {\multicolumn{2}{l}{\bluebold{karna}}} {\multicolumn{2}{l}{mama}} {dong}\\ %
& \textsc{3pl} & \multicolumn{3}{l}{indeed} & \multicolumn{4}{l}{raise} & \multicolumn{2}{l}{\textsc{3sg}} & at & \multicolumn{3}{l}{\textsc{l.med}} & but & \multicolumn{2}{l}{because} & \multicolumn{2}{l}{mother} & \textsc{3pl}\\
& \multicolumn{2}{l}{pu} & \multicolumn{3}{l}{bapa-ade} & …, & \multicolumn{3}{l}{\bluebold{tapi}} & \multicolumn{3}{l}{\bluebold{karna}} & \multicolumn{4}{l}{tete} & \multicolumn{2}{l}{meninggal,} & \multicolumn{2}{l}{\bluebold{jadi}}\\
& \multicolumn{2}{l}{\textsc{poss}} & \multicolumn{3}{l}{uncle} &  & \multicolumn{3}{l}{but} & \multicolumn{3}{l}{because} & \multicolumn{4}{l}{grandfather} & \multicolumn{2}{l}{die} & \multicolumn{2}{l}{so}\\
& \multicolumn{2}{l}{dong} & pu & \multicolumn{4}{l}{kluarga} & \multicolumn{4}{l}{ini} & \multicolumn{2}{l}{yang} & \multicolumn{7}{l}{piara}\\
& \multicolumn{2}{l}{\textsc{3pl}} & \textsc{poss} & \multicolumn{4}{l}{family} & \multicolumn{4}{l}{\textsc{d.prox}} & \multicolumn{2}{l}{\textsc{rel}} & \multicolumn{7}{l}{raise}\\
\lspbottomrule
\end{tabular}
\ea
\glt
‘they took indeed care of him there, \bluebold{but because} the uncle of mama and her companions [umh, who’s actually the youngest offspring,] \bluebold{but because} grandfather died, \bluebold{so} (it’s) their family here who took care of him’ \textstyleExampleSource{[080919-006-CvNP.0006-0008]}
\end{styleFreeTranslEngxvpt}

\paragraph[Causal gara{}-gara ‘because’]{Causal \textitbf{gara-gara} ‘because’}
\label{bkm:Ref355696451}
The causal conjunction \textitbf{gara-gara} ‘because’ indicates an emotive causal relation between two clauses by introducing a cause clause which gives the reason for the circumstances depicted in the unmarked clause. The conjunction has dual word class membership; it is also used as the bivalent verb \textitbf{gara} ‘irritate’ (see §5.14).



Most often, the cause clause marked with \textitbf{gara-gara} ‘because’ follows the unmarked clause, as in (0). Alternatively, the cause clause can precede the unmarked clause. In this case adversative \textitbf{tapi} ‘but’ (see §14.2.5.1) precedes \textitbf{gara-gara} ‘because’, as in (0), in the same way as \textitbf{tapi} ‘but’ precedes \textitbf{karna} ‘because’ (see §14.2.4.4). In that \textitbf{gara-gara} ‘because’ signals an emotive causal relation between its clause and the unmarked clause, it is distinct from \textitbf{karna} ‘because’ which marks neutral causal relations.
\end{styleBodyvxafter}

\begin{tabular}{llllllll}
\lsptoprule
\label{bkm:Ref355704554}
\gll {sap} {prut} {sakit,} {\bluebold{gara-gara}} {sa} {makang} {nasi}\\ %
& \textsc{1sg:poss} & stomach & be.sick & because & \textsc{1sg} & eat & cooked.rice\\
\lspbottomrule
\end{tabular}
\ea
\glt 
‘my stomach was sick \bluebold{because} I ate rice’ \textstyleExampleSource{[081025-009a-Cv.0046]}
\z

\begin{tabular}{lllllllllllllll}
\lsptoprule
\label{bkm:Ref355704555}
\gll {…} {\multicolumn{2}{l}{\bluebold{tapi}}} {\multicolumn{2}{l}{\bluebold{gara-gara}}} {\multicolumn{3}{l}{Nofela}} {\multicolumn{2}{l}{bi,}} {\multicolumn{2}{l}{\bluebold{gara-gara}}} {Nofela} {bicara}\\ %
&  & \multicolumn{2}{l}{but} & \multicolumn{2}{l}{because} & \multicolumn{3}{l}{Nofela} & \multicolumn{2}{l}{\textsc{tru}{}-speak} & \multicolumn{2}{l}{because} & Nofela & speak\\
& \multicolumn{2}{l}{deng} & \multicolumn{2}{l}{bapa,} & \multicolumn{2}{l}{bapa} & pu & \multicolumn{2}{l}{hati} & \multicolumn{2}{l}{tergrak} & \multicolumn{3}{l}{…}\\
& \multicolumn{2}{l}{with} & \multicolumn{2}{l}{father} & \multicolumn{2}{l}{father} & \textsc{poss} & \multicolumn{2}{l}{liver} & \multicolumn{2}{l}{be.moved} & \multicolumn{3}{l}{}\\
\lspbottomrule
\end{tabular}
\ea
\glt
[Phone conversation between a father and his daughter:] ‘[(if) I had just spoken to Siduas, maybe I wouldn’t have felt moved to come (and pick you up), right?,] \bluebold{but because} you (‘Nofela’) spoke[\textsc{tru}], \bluebold{because} you (‘Nofela’) spoke with me, my (‘father’s) heart was moved [so I’ll definitely come (and pick you up)]’ \textstyleExampleSource{[080922-001a-CvPh.1082-1083]}
\end{styleFreeTranslEngxvpt}

\subsection{Contrast}
\label{bkm:Ref356308992}
Contrast-marking conjunctions are cross-linguistically defined as conjunctions that signal that the events or states described in two clauses “are valid simultaneously”, but that the information given in one clause “marks a contrast to the information” given in the other clause {\citep[20]{Rudolph1996}}. This section describes four Papuan Malay contrast-marking conjunctions: adversative \textitbf{tapi} ‘but’ and \textitbf{habis} ‘after all’ (§14.2.5.2 and §14.2.5.1), oppositive \textitbf{padahal} ‘but actually’ (§14.2.5.3), and concessive \textitbf{biar} ‘although’ (§14.2.5.4). In addition, temporal \textitbf{baru} ‘and then’ has contrast-marking function in that it signals counter-expectation in the sense of ‘after all’; as this function is marginal it is discussed in §14.2.3.2 and not here.
\end{styleBodyxvafter}

\paragraph[Adversative tapi ‘but’]{Adversative \textitbf{tapi} ‘but’}
\label{bkm:Ref361248129}
Adversative \textitbf{tapi} ‘but’ occurs in interclausal position. It marks an adversative contrast relation between the clause it introduces and the preceding unmarked clause, as shown in (0) and (0).
\end{styleBodyxafter}

\begin{tabular}{llllllll}
\lsptoprule
\label{bkm:Ref355178194}
\gll {de} {bisa} {maing} {gitar,} {\bluebold{tapi}} {de} {malu}\\ %
& \textsc{3sg} & be.able & play & guitar & but & \textsc{3sg} & feel.embarrassed(.about)\\
\lspbottomrule
\end{tabular}
\ea
\glt 
‘she can play the guitar \bluebold{but} she feels shy (about it)’ \textstyleExampleSource{[081014-015-Cv.0008]}
\z

\begin{tabular}{llllllllllllllll}
\lsptoprule
\label{bkm:Ref355178200}
\gll {jadi} {\multicolumn{2}{l}{sa}} {\multicolumn{2}{l}{punya}} {\multicolumn{2}{l}{bapa}} {kasi} {\multicolumn{2}{l}{saya}} {\multicolumn{2}{l}{untuk}} {Iskia,} {\bluebold{tapi}} {Iskia}\\ %
& so & \multicolumn{2}{l}{\textsc{1sg}} & \multicolumn{2}{l}{\textsc{poss}} & \multicolumn{2}{l}{father} & give & \multicolumn{2}{l}{\textsc{1sg}} & \multicolumn{2}{l}{for} & Iskia & but & Iskia\\
& \multicolumn{2}{l}{kawing} & \multicolumn{2}{l}{sala,} & \multicolumn{2}{l}{Iskia} & \multicolumn{2}{l}{kawing} & sa & \multicolumn{2}{l}{punya} & \multicolumn{4}{l}{kaka}\\
& \multicolumn{2}{l}{marry} & \multicolumn{2}{l}{be.wrong} & \multicolumn{2}{l}{Iskia} & \multicolumn{2}{l}{marry} & \textsc{1sg} & \multicolumn{2}{l}{\textsc{poss}} & \multicolumn{4}{l}{oSb}\\
\lspbottomrule
\end{tabular}
\ea
\glt
‘so my father gave me to Iskia, \bluebold{but} Iskia married improperly, Iskia married my older sister’ \textstyleExampleSource{[081006-028-CvEx.0005]}
\end{styleFreeTranslEngxvpt}

\paragraph[Adversative habis ‘after all’]{Adversative \textitbf{habis} ‘after all’}
\label{bkm:Ref361248130}
Adversative \textitbf{habis} ‘after all’ also marks an adversative relation between two clauses. The conjunction has dual word class membership; it is also used as the monovalent verb \textitbf{habis} ‘be used up’ (see §5.14).



Introducing a contrast clause that follows the unmarked clause, \textitbf{habis} ‘after all’ summarizes what has been said before and signals that the propositional content of its clause is true in spite of the content of the preceding unmarked clause, as shown in (0) and (0). At the same time, the conjunction signals that the interlocutor is expected to know that this content is true. Thereby \textitbf{habis} ‘after all’ is distinct from adversative \textitbf{tapi} ‘but’ (see §14.2.5.2). Adversative \textitbf{habis} ‘after all’ is also distinct from counter-expectational \textitbf{baru} ‘after all’ which merely summarizes what has been said before (see §14.2.3.2). The exchange in (0) illustrates that there does not need to be an overt unmarked clause which precedes the contrast clause: speakers also use \textitbf{habis} ‘after all’ to reply to an interlocutor’s statements.
\end{styleBodyvxafter}

\begin{tabular}{llllllllll}
\lsptoprule
\label{bkm:Ref356038066}
\gll {bilang} {bapa,} {kirim} {tong} {uang,} {\bluebold{habis}} {sa} {susa} {to?}\\ %
& say & father & send & \textsc{1pl} & money & after.all & \textsc{1sg} & difficult & right?\\
\lspbottomrule
\end{tabular}
\ea
\glt 
‘say (to) father, ‘send us money, \bluebold{after all}, I have difficulties, right?’’ \textstyleExampleSource{[080922-001a-CvPh.0866]}
\z

\begin{tabular}{lllllllll}
\lsptoprule
\label{bkm:Ref356038065}
\gll { & Speaker-1: & ko & baru & masuk & klas & satu & ini?}\\ %
&  &  & \textsc{2sg} & recently & enter & class & one & \textsc{d.prox}\\
\lspbottomrule
\end{tabular}
\begin{styleFreeTranslAlphaEng}
Speaker-1: ‘recently you got into first grade (of middle school)?’
\end{styleFreeTranslAlphaEng}

\begin{tabular}{lllllll} &  & Speaker-2: & yo, & \bluebold{habis} & sa & gagal\\
\lsptoprule
&  &  & yes & after.all & \textsc{1sg} & fail\\
\lspbottomrule
\end{tabular}
\begin{styleFreeTranslAlphaEngxxpt}
Speaker-2: ‘yes, \bluebold{after all}, I failed (the last exams)’ \textstyleExampleSource{[080922-001a-CvPh.0965-0966]}
\end{styleFreeTranslAlphaEngxxpt}

\paragraph[Oppositive padahal ‘but actually’]{Oppositive \textitbf{padahal} ‘but actually’}
\label{bkm:Ref361248131}
The conjunction \textitbf{padahal} ‘but actually’ introduces a contrast clause, which follows the unmarked clause. Concurrent to marking contrast, the conjunction signals that the propositional content of its clause is surprising and unexpected given the content of the unmarked clause. Thereby, \textitbf{padahal} ‘but actually’ is more oppositive than \textitbf{tapi} ‘but’ (see §14.2.5.1). This is illustrated in (0) and (0).
\end{styleBodyxafter}

\begin{tabular}{lllllllllllll}
\lsptoprule
\label{bkm:Ref356031371}
\gll {ana} {\multicolumn{3}{l}{ini,}} {\multicolumn{2}{l}{sa}} {pikir} {de} {suda} {lewat,} {\bluebold{padahal}} {de}\\ %
& child & \multicolumn{3}{l}{\textsc{d.prox}} & \multicolumn{2}{l}{\textsc{1sg}} & think & \textsc{3sg} & already & pass.by & but.actually & \textsc{3sg}\\
& \multicolumn{2}{l}{tidor} & di & \multicolumn{2}{l}{atas} & \multicolumn{7}{l}{kayu{\Tilde}kayu}\\
& \multicolumn{2}{l}{sleep} & at & \multicolumn{2}{l}{top} & \multicolumn{7}{l}{\textsc{rdp}{\Tilde}wood}\\
\lspbottomrule
\end{tabular}
\ea
\glt 
‘this child, I thought he’d already passed by, \bluebold{but actually} he was sleeping on top of the wood’ \textstyleExampleSource{[081013-004.Cv.0004]}
\z

\begin{tabular}{lllllllllllll}
\lsptoprule
\label{bkm:Ref356031373}
\gll {bulang} {\multicolumn{2}{l}{oktober}} {\multicolumn{2}{l}{sa}} {pu} {\multicolumn{2}{l}{alpa}} {cuma} {dua} {saja,} {bayangkang,}\\ %
& month & \multicolumn{2}{l}{October} & \multicolumn{2}{l}{\textsc{1sg}} & \textsc{poss} & \multicolumn{2}{l}{be.absent} & just & two & just & image\\
& \multicolumn{2}{l}{\bluebold{padahal}} & \multicolumn{2}{l}{sa} & \multicolumn{3}{l}{alpa} & \multicolumn{5}{l}{banyak}\\
& \multicolumn{2}{l}{but.actually} & \multicolumn{2}{l}{\textsc{1sg}} & \multicolumn{3}{l}{be.absent} & \multicolumn{5}{l}{many}\\
\lspbottomrule
\end{tabular}
\ea
\glt
[About the speaker’s school attendance:] ‘imagine!, in October I had just only two (official) absences, \bluebold{but actually} I was absent many times’ (Lit. ‘my absences were many’) \textstyleExampleSource{[081023-004-Cv.0014]}
\end{styleFreeTranslEngxvpt}

\paragraph[Concessive biar ‘although’]{Concessive \textitbf{biar} ‘although’}
\label{bkm:Ref361248132}
Concessive \textitbf{biar} ‘although’ is marks concessive relations between two clauses. The conjunction has dual word class membership; it is also used as the bivalent verb \textitbf{biar} ‘let’ (see §5.14).



Introducing a concession clause, \textitbf{biar} ‘although’ signals that despite the event or state depicted in its clause, the event or state depicted in the unmarked clause occurred. Usually, the concession clause precedes the unmarked clause, whereby the concession is emphasized, as in (0). Alternatively, although less often, it can follow the unmarked clause, in which case the content of the latter clause is emphasized, as in (0).
\end{styleBodyvxafter}

\begin{tabular}{lllllll}
\lsptoprule
\label{bkm:Ref356059222}
\gll {yo,} {\bluebold{biar}} {makangang} {tinggi,} {de} {ambil}\\ %
& yes & although & food & be.high & \textsc{3sg} & fetch\\
\lspbottomrule
\end{tabular}
\ea
\glt 
[About a greedy child:] ‘yes, \bluebold{although} the food is (placed) high (up on a shelf), he takes (it)’ \textstyleExampleSource{[081025-006-Cv.0254]}
\z

\begin{tabular}{llllllllllll}
\lsptoprule
\label{bkm:Ref356059223}
\gll {…} {\multicolumn{2}{l}{jangang}} {\multicolumn{2}{l}{tinggal}} {di} {ruma,} {tida} {bole,} {\bluebold{biar}} {dulu}\\ %
&  & \multicolumn{2}{l}{\textsc{neg.imp}} & \multicolumn{2}{l}{stay} & at & house & \textsc{neg} & may & although & first\\
& \multicolumn{2}{l}{orang-tua} & \multicolumn{2}{l}{dong} & \multicolumn{2}{l}{bilang} & \multicolumn{5}{l}{begini}\\
& \multicolumn{2}{l}{parent} & \multicolumn{2}{l}{\textsc{3pl}} & \multicolumn{2}{l}{say} & \multicolumn{5}{l}{like.this}\\
\lspbottomrule
\end{tabular}
\ea
\glt
‘[so you kids have to go to school,] don’t stay home, (that’s) not allowed, \bluebold{although} the parents said so in the past’ \textstyleExampleSource{[081110-008-CvNP.0036]}
\end{styleFreeTranslEngxvpt}

\subsection{Similarity}
\label{bkm:Ref361251536}
As conjunctions, the similative prepositions \textitbf{sperti} ‘similar to’ and \textitbf{kaya} ‘like’ mark similarity between two clauses. Introducing similarity clauses, both signal that the event or state depicted in the unmarked clause is similar to that described in the similarity clause. The similarity clause always follows the unmarked clause.



Derived from their prepositional semantics, \textitbf{sperti} ‘similar to’ signals likeness in some, often implied, respect, while \textitbf{kaya} ‘like’ marks overall resemblance, as shown in (0) and (0), respectively. (See §10.3.1 and §10.3.2 for a detailed discussion of the prepositions \textitbf{sperti} ‘similar to’ and \textitbf{kaya} ‘like’ and their semantics.)
\end{styleBodyvxafter}

\begin{tabular}{lllllllll}
\lsptoprule
\label{bkm:Ref361153962}
\gll {mama} {dia} {lupa} {kamu,} {\bluebold{sperti}} {kacang} {lupa} {kulit}\\ %
& mother & \textsc{3sg} & forget & \textsc{2pl} & similar.to & bean & forget & skin\\
\lspbottomrule
\end{tabular}
\ea
\glt 
‘mother forgot you (in a way that is) \bluebold{similar to} a bean forgetting its skin’ \textstyleExampleSource{[080922-001a-CvPh.0932]}
\z

\begin{tabular}{llllllllll}
\lsptoprule
\label{bkm:Ref361153963}
\gll {…} {tong} {\multicolumn{3}{l}{taputar}} {\bluebold{kaya}} {kitong} {ni} {ana{\Tilde}ana}\\ %
&  & \textsc{1pl} & \multicolumn{3}{l}{be.turned.around} & like & \textsc{1pl} & \textsc{d.prox} & \textsc{rdp}{\Tilde}child\\
& \multicolumn{3}{l}{perjalangang} & yang & \multicolumn{5}{l}{taputar}\\
& \multicolumn{3}{l}{journey} & \textsc{rel} & \multicolumn{5}{l}{be.turned.around}\\
\lspbottomrule
\end{tabular}
\ea
\glt
‘[we were looking for a bathroom …, good grief! there weren’t (any) bathrooms,] we wandered around \bluebold{like} we here were children on a trip wandering around’ \textstyleExampleSource{[081025-009a-Cv.0059]}
\end{styleFreeTranslEngxvpt}

\section{Conjunctions combining different-type constituents}
\label{bkm:Ref374455400}\label{bkm:Ref356411665}
This section describes two conjunctions which combine different-type constituents. Complementizer \textitbf{bahwa} ‘that’ links a clause to a bivalent verb (§14.3.1), while relativizer \textitbf{yang} ‘\textsc{rel}’ integrates a relative clause within a noun phrase (§14.3.2).
\end{styleBodyxvafter}

\subsection{Complementizer \textitbf{bahwa} ‘that’}
\label{bkm:Ref356320826}
The complementizer \textitbf{bahwa} ‘that’ marks a clause as the complement of a verb. Cross-linguistically, it is typically bivalent “verbs of utterance and cognition” that take complements {\citep[279]{Payne1997}}. This also applies to Papuan Malay. The corpus contains 68 complement clauses with \textitbf{bahwa} ‘that’. In 37 cases (54\%), the complement-taking verb is \textitbf{taw} ‘know’, followed by \textitbf{bilang} ‘say’ (5 tokens), \textitbf{ceritra} ‘tell’(4 tokens), and \textitbf{liat} ‘see’ (3 tokens).



Two structural patterns are attested for complementation with \textitbf{bahwa} ‘that’. Usually, the verb is followed by the clausal complement with \textitbf{bahwa} ‘that’ (61 tokens), as in (0) and (0). Alternatively, although much less often, the verb is followed by an object which is followed by the clausal complement (8 tokens), as in (0).\footnote{\\
\\
\\
\\
\\
\\
\\
\\
\\
\\
\\
\\
\\
\\
\\
\par Typically, speakers report speech in the form of direct speech rather than indirect speech as in (0) (see also §6.2.1.1).}
\end{styleBodyvvafter}

\begin{styleExampleTitle}
\textsc{verb} – \textitbf{bahwa} ‘that’ (\textsc{object}) – \textsc{clausal} \textsc{complement}
\end{styleExampleTitle}

\begin{tabular}{lllllllllll}
\lsptoprule
\label{bkm:Ref356320281}
\gll {sa} {tida} {\bluebold{taw}} {\bluebold{bahwa}} {jam} {tiga} {itu} {de} {su} {meninggal}\\ %
& \textsc{1sg} & \textsc{neg} & know & that & hour & three & \textsc{d.dist} & \textsc{3sg} & already & die\\
\lspbottomrule
\end{tabular}
\ea
\glt 
‘I didn’t \bluebold{know that} by three o’clock she had already died’ \textstyleExampleSource{[080917-001-CvNP.0005]}
\z

\begin{tabular}{lllllllllllll}
\lsptoprule
\label{bkm:Ref356320283}
\gll {kalo} {\multicolumn{2}{l}{blum}} {\multicolumn{2}{l}{nika}} {\multicolumn{2}{l}{itu,}} {greja} {\bluebold{bilang}} {\bluebold{bahwa}} {dong} {dua}\\ %
& if & \multicolumn{2}{l}{not.yet} & \multicolumn{2}{l}{marry} & \multicolumn{2}{l}{\textsc{d.dist}} & church & say & that & \textsc{3pl} & two\\
& \multicolumn{2}{l}{blum} & \multicolumn{2}{l}{jadi} & \multicolumn{2}{l}{swami} & \multicolumn{6}{l}{istri}\\
& \multicolumn{2}{l}{not.yet} & \multicolumn{2}{l}{become} & \multicolumn{2}{l}{husband} & \multicolumn{6}{l}{wife}\\
\lspbottomrule
\end{tabular}
\ea
\glt 
‘if (they) haven’t (officially) married yet, (then) the church \bluebold{says that} the two of them haven’t yet become husband and wife’ \textstyleExampleSource{[081110-006-CvEx.0196]}
\z

\begin{tabular}{llllllllllll}
\lsptoprule
\label{bkm:Ref356320285}
\gll {jadi} {\multicolumn{2}{l}{Raymon}} {\multicolumn{2}{l}{\bluebold{tuntut}}} {\bluebold{sama}} {\multicolumn{2}{l}{\bluebold{kita}}} {to?,} {\bluebold{sama}} {\bluebold{kitorang}}\\ %
& so & \multicolumn{2}{l}{Raymon} & \multicolumn{2}{l}{demand} & from & \multicolumn{2}{l}{\textsc{1pl}} & right? & from & \textsc{1pl}\\
& \multicolumn{2}{l}{\bluebold{bahwa}} & kamu & harus & \multicolumn{3}{l}{ganti} & \multicolumn{4}{l}{lagi}\\
& \multicolumn{2}{l}{that} & \textsc{2pl} & have.to & \multicolumn{3}{l}{replace} & \multicolumn{4}{l}{also}\\
\lspbottomrule
\end{tabular}
\ea
\glt
[About bride-price customs:] ‘so Raymon \bluebold{demanded from us}, right?, \bluebold{from us that} we also had to compensate (for that wife)’ (Lit. ‘… \bluebold{from us that} you had to replace’) \textstyleExampleSource{[081006-024-CvEx.0019]}
\end{styleFreeTranslEngxvpt}

\subsection{Relativizer \textitbf{yang} ‘\textsc{rel}’}
\label{bkm:Ref356320827}
Relativizer \textitbf{yang} ‘\textsc{rel}’ introduces relative clauses which function as modifiers within noun phrases (see also §8.2.8). Typically, the relative clause follows its head nominal, as in (0) and (0). However, \textitbf{yang} ‘\textsc{rel}’ can also introduce a headless relative clause. Cross-linguistically, headless relative clauses can be used “when the head noun is non-specific” or when “the specific reference to the head is clear” {\citep[295]{Payne1997}}. This also applies to Papuan Malay. In (0), for instance, the head nominal is non-specific, while in (0) the reference to the head is clear (“${\varnothing}$” signifies the implied head nominal).


\begin{styleExampleTitle}
Relative clauses with overt head nominal and headless relative clauses
\end{styleExampleTitle}

\begin{tabular}{llllllllll}
\lsptoprule
\label{bkm:Ref356379722}
\gll {kitong} {mo} {hancurkang} {\bluebold{tugu}} {\bluebold{yang}} {ada} {di} {Sarmi} {itu}\\ %
& \textsc{1pl} & want & shatter & monument & \textsc{rel} & exist & at & Sarmi & \textsc{d.dist}\\
\lspbottomrule
\end{tabular}
\ea
\glt 
‘we want to destroy \bluebold{the statue that} is in Sarmi there’ \textstyleExampleSource{[080917-008-NP.0043]}
\z

\begin{tabular}{llllllllllll}
\lsptoprule
\label{bkm:Ref356379723}
\gll {tong} {tra} {ke} {kampung,} {tra} {ada} {\bluebold{${\varnothing}$}} {\bluebold{yang}} {jalang} {ke} {kampung}\\ %
& \textsc{1pl} & \textsc{neg} & to & village & \textsc{neg} & exist &  & \textsc{rel} & walk & to & village\\
\lspbottomrule
\end{tabular}
\ea
\glt 
‘we don’t (go) to the village, there is (\bluebold{nobody}) \bluebold{who} goes to the village’ \textstyleExampleSource{[080917-003a-CvEx.0048]}
\z

\begin{tabular}{lllllll}
\lsptoprule
\label{bkm:Ref356379738}
\gll {\label{bkm:Ref356379753}} {Speaker-1:} {\bluebold{Nelci}} {\bluebold{itu}} {\bluebold{yang}} {mana?}\\ %
&  &  & Nelci & \textsc{d.dist} & \textsc{rel} & where\\
\lspbottomrule
\end{tabular}
\begin{styleFreeTranslAlphaEng}
Speaker-1: ‘\bluebold{which} one is \bluebold{that Nelci}?’
\end{styleFreeTranslAlphaEng}

\begin{tabular}{llllllllll} & \label{bkm:Ref356379754} & Speaker-2: & \bluebold{${\varnothing}$} & \bluebold{yang} & kecil{\Tilde}kecil & … & \bluebold{${\varnothing}$} & \bluebold{yang} & rajing{\Tilde}rajing\\
\lsptoprule
&  &  &  & \textsc{rel} & \textsc{rdp}{\Tilde}be.small &  &  & \textsc{rel} & \textsc{rdp}{\Tilde}be.diligent\\
\lspbottomrule
\end{tabular}
\begin{styleFreeTranslAlphaEng}
Speaker-2: ‘(\bluebold{the one}) \bluebold{who}’s kind of small … (\bluebold{the one}) \bluebold{who}’s very diligent’ \textstyleExampleSource{[081115-001a-Cv.0285-0292]}
\end{styleFreeTranslAlphaEng}


The remainder of this section describes the grammatical positions which can be relativized in Papuan Malay. The data in the corpus shows that, in terms of {Keenan and Comrie’s (1977)} “Accessibility Hierarchy”, Papuan Malay allows relativization on all five positions, namely:


\begin{styleIvI}
\textsc{subject} {\textgreater} \textsc{direct} \textsc{object} {\textgreater} \textsc{indirect} \textsc{object} {\textgreater} \textsc{oblique} {\textgreater} \textsc{possessor}
\end{styleIvI}


Cross-linguistically, as{ Payne (1997: 297, 298) points out}, relativization of these positions involves two different “case recoverability strategies” which allow to identify “the role of the referent of the head noun \textstyleChItalic{within the relative clause}”, namely the “gap strategy” or “pronoun retention”. Both strategies are also found in Papuan Malay. Relativization of subject, direct and oblique object arguments is achieved with the gap strategy, while relativization of obliques and possessors involves pronoun retention.



When core arguments are relativized, a gap is left. This gap, signified with “${\varnothing}$“, occurs where the relativized noun phrase would be situated if it were expressed overtly. Relativization of the subject argument is illustrated in (0), and of the direct object argument of a bivalent verb, namely \textitbf{biking} ‘make’, in (0). The examples in (0) and (0) illustrate the relativization of the direct object positions in double-object constructions; in both examples the trivalent verb is \textitbf{kasi} ‘give’. In (0), the R argument \textitbf{papeda} ‘sagu porridge’ is relativized. In (0), the T argument \textitbf{Efana ini} ‘this Efana’ is relativized. (Verbal clauses with bivalent and trivalent verbs are discussed in detail in §11.1.2 and §11.1.3, respectively.)
\end{styleBodyvvafter}

\begin{styleExampleTitle}
Relativization of the subject and direct object positions
\end{styleExampleTitle}

\begin{tabular}{lllllllllllllll}
\lsptoprule
\label{bkm:Ref356382344}
\gll {tong} {bagi} {buat} {\multicolumn{3}{l}{\bluebold{kitorang}}} {\multicolumn{2}{l}{\bluebold{yang}}} {\bluebold{${\varnothing}$}} {\multicolumn{3}{l}{potong}} {itu} {…,}\\ %
& \textsc{1pl} & divide & for & \multicolumn{3}{l}{\textsc{1pl}} & \multicolumn{2}{l}{\textsc{rel}} &  & \multicolumn{3}{l}{cut} & \textsc{d.dist} & \\
& buat & \multicolumn{3}{l}{\bluebold{sodara{\Tilde}sodara}} & \bluebold{yang} & \multicolumn{2}{l}{\bluebold{${\varnothing}$}} & \multicolumn{3}{l}{tinggal} & di & \multicolumn{3}{l}{kampung}\\
& for & \multicolumn{3}{l}{\textsc{rdp}{\Tilde}sibling} & \textsc{rel} & \multicolumn{2}{l}{} & \multicolumn{3}{l}{stay} & at & \multicolumn{3}{l}{village}\\
\lspbottomrule
\end{tabular}
\ea
\glt 
[About hunting a wild pig:] we divided (the meat) for \bluebold{us who} cut (it) up that day, (and) then for \bluebold{the relatives and friends who} live in the village’ \textstyleExampleSource{[080919-003-NP.0014]}
\z

\begin{tabular}{lllllllllllllll}
\lsptoprule
\label{bkm:Ref356382345}
\gll {saya} {kas} {\multicolumn{3}{l}{makang}} {\multicolumn{2}{l}{anjing}} {\multicolumn{2}{l}{deng}} {\multicolumn{2}{l}{\bluebold{papeda}}} {\bluebold{yang}} {sa} {pu}\\ %
& \textsc{1sg} & give & \multicolumn{3}{l}{eat} & \multicolumn{2}{l}{dog} & \multicolumn{2}{l}{with} & \multicolumn{2}{l}{sagu.porridge} & \textsc{rel} & \textsc{1sg} & \textsc{poss}\\
& bini & \multicolumn{2}{l}{biking} & \bluebold{${\varnothing}$} & \multicolumn{2}{l}{malam} & \multicolumn{2}{l}{untuk} & \multicolumn{2}{l}{anjing} & \multicolumn{4}{l}{dorang}\\
& wife & \multicolumn{2}{l}{make} &  & \multicolumn{2}{l}{night} & \multicolumn{2}{l}{for} & \multicolumn{2}{l}{dog} & \multicolumn{4}{l}{\textsc{3pl}}\\
\lspbottomrule
\end{tabular}
\ea
\glt 
‘I fed the dogs with \bluebold{the sagu porridge which} my wife had prepared for the dogs in the evening’ \textstyleExampleSource{[080919-003-NP.0002]}
\z

\begin{tabular}{llllllllllll}
\lsptoprule
\label{bkm:Ref356394392}
\gll {\bluebold{Fitri}} {\bluebold{yang}} {de} {bapa} {kasi} {\bluebold{${\varnothing}$}} {ijing} {mo} {ikut} {ke} {kampung}\\ %
& Fitri & \textsc{rel} & \textsc{3sg} & father & give &  & permission & want & follow & to & village\\
\lspbottomrule
\end{tabular}
\ea
\glt 
‘(it was) \bluebold{Fitri whom} her husband gave permission to go with (us) to the village’ (Lit. ‘her (daughter’s) father …’) \textstyleExampleSource{[080925-003-Cv.0211]}
\z

\begin{tabular}{llllllll}
\lsptoprule
\label{bkm:Ref439865389}
\gll {\bluebold{Efana}} {\bluebold{ini}} {\bluebold{yang}} {dia\textsuperscript{i}} {kas} {dia\textsuperscript{j}} {\bluebold{${\varnothing}$}}\\ %
& Efana & \textsc{d.prox} & \textsc{rel} & \textsc{3sg} & give & \textsc{3sg} & \\
\lspbottomrule
\end{tabular}
\ea
\glt 
[About an ancestor’s first wife:] ‘(it was) \bluebold{this Efana that} he\textsuperscript{i} (‘Aris’) gave (to) him\textsuperscript{j} (‘Oten’)’ \textstyleExampleSource{[080922-010a-CvNF.0062]}\footnote{\\
\\
\\
\\
\\
\\
\\
\\
\\
\\
\\
\\
\\
\\
\\
\par The subscript letters keep track of what each term refers to.}
\z


Obliques and possessors are relativized via pronoun retention. That is, a retained personal pronoun explicitly marks the relativized position within the relative clause. This is illustrated with the relativization of an oblique argument in (0), and of a possessor in (0). (For a discussion of interrogative \textitbf{mana} ‘where, which’ and its adnominal uses, see §5.8.3; for details on adnominal possessive constructions, see Chapter 9.)


\begin{styleExampleTitle}
Relativization of the oblique object, and possessor positions
\end{styleExampleTitle}

\begin{tabular}{lllllllllllllll}
\lsptoprule
\label{bkm:Ref356394393}
\gll {kalo} {\multicolumn{2}{l}{\bluebold{ana}}} {\multicolumn{2}{l}{\bluebold{mana}}} {\multicolumn{2}{l}{\bluebold{yang}}} {\multicolumn{2}{l}{sa}} {\multicolumn{2}{l}{duduk}} {ceritra} {\bluebold{deng}} {\bluebold{dia},}\\ %
& if & \multicolumn{2}{l}{child} & \multicolumn{2}{l}{where} & \multicolumn{2}{l}{\textsc{rel}} & \multicolumn{2}{l}{\textsc{1sg}} & \multicolumn{2}{l}{sit} & tell & with & \textsc{3sg}\\
& \multicolumn{2}{l}{itu} & \multicolumn{2}{l}{ana} & \multicolumn{2}{l}{itu,} & \multicolumn{2}{l}{de} & \multicolumn{2}{l}{hormat} & \multicolumn{4}{l}{torang}\\
& \multicolumn{2}{l}{\textsc{d.dist}} & \multicolumn{2}{l}{child} & \multicolumn{2}{l}{\textsc{d.dist}} & \multicolumn{2}{l}{\textsc{3sg}} & \multicolumn{2}{l}{respect} & \multicolumn{4}{l}{\textsc{1pl}}\\
\lspbottomrule
\end{tabular}
\ea
\glt 
‘as for \bluebold{which kid with whom} I sit and talk, that is that kid, she respects us’ \textstyleExampleSource{[081115-001a-Cv.0282]}
\z

\begin{tabular}{lllllllllllll}
\lsptoprule
\label{bkm:Ref356394397}
\gll {itu} {\bluebold{kaka}} {\multicolumn{2}{l}{\bluebold{satu}}} {\multicolumn{2}{l}{\bluebold{itu}}} {\multicolumn{2}{l}{\bluebold{yang}}} {\bluebold{dia}} {\bluebold{punya}} {ade} {prempuang}\\ %
& \textsc{d.dist} & oSb & \multicolumn{2}{l}{one} & \multicolumn{2}{l}{\textsc{d.dist}} & \multicolumn{2}{l}{\textsc{rel}} & \textsc{3sg} & \textsc{poss} & ySb & woman\\
& itu & \multicolumn{2}{l}{tinggal} & \multicolumn{2}{l}{deng} & \multicolumn{2}{l}{Natanael} & \multicolumn{5}{l}{tu}\\
& \textsc{d.dist} & \multicolumn{2}{l}{stay} & \multicolumn{2}{l}{with} & \multicolumn{2}{l}{Natanael} & \multicolumn{5}{l}{\textsc{d.dist}}\\
\lspbottomrule
\end{tabular}
\ea
\glt
‘that is \bluebold{that one older brother whose} younger sister is staying with Natanael’ \textstyleExampleSource{[080922-001a-CvPh.0888]}
\end{styleFreeTranslEngxvpt}

\section{Juxtaposition}
\label{bkm:Ref356637572}
Juxtaposition is another strategy in Papuan Malay to link constituents, namely same-type constituents, such as noun phrases, prepositional phrases, verbs, or clauses.



Juxtaposition of noun phrases, as in (0) to (0), occurs considerably less often in the corpus than conjoining with a conjunction. Most often, three, four or five noun phrases are juxtaposed to enumerate entities, while juxtaposition of just two noun phrases occurs less often. These findings reflect the results of {Stassen’s (2000)} typological study of noun phrase conjunction which shows that juxtaposition is “a minor strategy” which is often used “in list-like enumerations”.\footnote{\\
\\
\\
\\
\\
\\
\\
\\
\\
\\
\\
\\
\\
\\
\\
\par According to {Stassen (2000: 7–8)}, “the general trend all over the world is that zero-coordination tends to be marginalized into specific functions or is replaced altogether by overt marking strategies”. {Mithun (1988: 351–357)} suggests that this development is due to the global increase in bilingualism and in literacy. With respect to bilingualism, {\citet[351]{Mithun1988}} observes that “an astonishing number of coordinating conjunctions have been recently borrowed into languages that previously had none”. As for the role of literacy, {\citet[356]{Mithun1988}} notes that, whereas in oral language intonation suffices to signal the syntactic structure of juxtaposed constituents, written language requires the overt and “systematic specification of the precise nature of link” to disambiguate syntactic relations.}
\end{styleBodyvafter}


Papuan Malay combines different prosodic features to indicate the structure of the juxtaposed noun phrases: final vowel lengthening (orthographically represented by a sequence of three vowels), slight increase in pitch of the stressed syllable (“~\'{~}~”), intonation breaks (“{\textbar}”), non-final intonation pattern with level pitch (“---”), and end-of-list intonation with fall pitch (“{\textbackslash}”). The enumeration structure in (0) is indicated with an increase in pitch, and the last item is marked off by the demonstrative \textitbf{itu} ‘\textsc{d.dist}’. In (0), the enumeration is signaled with an increase in pitch as well as intonation breaks; the last item has an end-of-list intonation. In (0), the structure is marked with a slight increase in pitch and final vowel lengthening of the first and third coordinands while the fourth item has an end-of-list intonation. The second and third coordinands form a compact intonation unit, separated from the first and fourth coordinands by intonation breaks. After another intonation break following the fourth coordinand, the fifth coordinand is added as an afterthought.
\end{styleBodyvvafter}

\begin{styleExampleTitle}
Juxtaposition of noun phrases
\end{styleExampleTitle}

\begin{tabular}{lllllllll}
\lsptoprule
\label{bkm:Ref373852630}
\gll {\multicolumn{2}{l}{\textstyleChBold{\textsuperscript{{}---}}\textstyleChBold{\textsubscript{{}---}}}} {\multicolumn{2}{l}{\textstyleChBold{\textsuperscript{{}---}}\textstyleChBold{\textsubscript{{}---}}}} {\textstyleChBold{\textsuperscript{{}---}}\textstyleChBold{\textsubscript{{}---}}} {\textstyleChBold{\textsubscript{{}---{}---}}} {} {}\\ %
& \multicolumn{2}{l}{\bluebold{gúntur}} & \multicolumn{2}{l}{\bluebold{kílat}} & \bluebold{hújang} & \bluebold{itu} & dia & sambar\\
& \multicolumn{2}{l}{thunder} & \multicolumn{2}{l}{lightning} & rain & \textsc{d.dist} & \textsc{3sg} & strike.one.after.the.other\\
& ruma & \multicolumn{2}{l}{itu} & \multicolumn{5}{l}{sampeee}\\
& house & \multicolumn{2}{l}{\textsc{d.dist}} & \multicolumn{5}{l}{reach}\\
\lspbottomrule
\end{tabular}
\ea
\glt 
‘that \bluebold{thunder, lightning, (and) rain}, it hit one house after the other on and on’ \textstyleExampleSource{[081006-022-CvEx.0007]}
\z

\begin{tabular}{llllllll}
\lsptoprule
\label{bkm:Ref373852631}
\gll {\textstyleChBold{\textsuperscript{{}---}}\textstyleChBold{\textsubscript{{}---}}} {} {\textstyleChBold{\textsuperscript{{}---}}\textstyleChBold{\textsubscript{{}---}}} {} {\textstyleChBold{\textsuperscript{{}---}}\textstyleChBold{\textsubscript{{}---}}} {} {\textstyleChBold{{}---{}---}\textbf{\textsubscript{{\textbackslash}}}}\\ %
& \bluebold{káing} & {\textbar} & \bluebold{bántal} & {\textbar} & \bluebold{smúa} & {\textbar} & \bluebold{tíkar}\\
& cloth &  & pillow &  & all &  & plaited.mat\\
\lspbottomrule
\end{tabular}
\ea
\glt 
[Listing laundry items:] ‘\bluebold{the cloths, pillows, everything, the plaited mats}’ \textstyleExampleSource{[081025-006-Cv.0057]}
\z

\begin{tabular}{lllllllllll}
\lsptoprule
\label{bkm:Ref373852632}
\gll { &  & \textstyleChBold{\textsuperscript{{}---}}\textstyleChBold{\textsubscript{{}---{}---}} &  & \textstyleChBold{\textsubscript{{}---{}---}} & \textstyleChBold{\textsuperscript{{}---}}\textstyleChBold{\textsubscript{{}---{}---}} &  & \textstyleChBold{{}---{}---}\textbf{\textsubscript{{\textbackslash}}} &  & \textstyleChBold{{}---{}---{}---}}\\ %
& kita & pake & \bluebold{búmbuuu} & {\textbar} & \bluebold{fetsin} & \bluebold{gáraaam} & {\textbar} & \bluebold{sere} & {\textbar} & \bluebold{ricaaa}\\
& \textsc{1pl} & use & spice &  & MSG & salt &  & lemon.grass &  & red.pepper\\
\lspbottomrule
\end{tabular}
\ea
\glt 
‘we used \bluebold{spices, flavoring spice, salt, lemongrass, red pepper}’ \textstyleExampleSource{[080919-004-NP.0037]}
\z


Juxtaposition of prepositional phrases, verbs, or clauses is illustrated in (0) to (0). Three prepositional phrases introduced with elative \textitbf{dari} ‘from’ are juxtaposed in (0), three verbs in (0), and four clauses in (0) (for easier recognition the first constituent of each of the linked clauses is bolded).


\begin{styleExampleTitle}
Juxtaposition of prepositional phrases, verbs, or clauses
\end{styleExampleTitle}

\begin{tabular}{llllllllllll}
\lsptoprule
\label{bkm:Ref373852633}
\gll {\multicolumn{2}{l}{baru}} {sa} {\multicolumn{2}{l}{punya}} {\multicolumn{2}{l}{bapa}} {dia} {turung} {\bluebold{dari}} {atas}\\ %
& \multicolumn{2}{l}{and.then} & \textsc{1sg} & \multicolumn{2}{l}{\textsc{poss}} & \multicolumn{2}{l}{father} & \textsc{3sg} & descend & from & top\\
& \bluebold{dari} & \multicolumn{3}{l}{pedalamang} & \multicolumn{2}{l}{\bluebold{dari}} & \multicolumn{5}{l}{Siantoa}\\
& from & \multicolumn{3}{l}{interior} & \multicolumn{2}{l}{from} & \multicolumn{5}{l}{Siantoa}\\
\lspbottomrule
\end{tabular}
\ea
\glt 
‘and then my father came down \bluebold{from} the hills, \bluebold{from} the interior, \bluebold{from} Siantoa’ \textstyleExampleSource{[080927-009-CvNP.0010]}
\z

\begin{tabular}{lllllllll}
\lsptoprule
\label{bkm:Ref373852634}
\gll {kepala} {desa} {mantang} {Arbais} {ada} {\bluebold{duduk}} {\bluebold{ceritra}} {\bluebold{minum}}\\ %
& head & village & former & Arbais & exist & sit & tell & drink\\
\lspbottomrule
\end{tabular}
\ea
\glt 
‘the former mayor of Arbais was \bluebold{sitting} (there and) \bluebold{talking} (and) \bluebold{drinking}’ \textstyleExampleSource{[081011-024-Cv.0135]}
\z

\begin{tabular}{llllllllllll}
\lsptoprule
\label{bkm:Ref373852635}
\gll {\multicolumn{2}{l}{\bluebold{Oktofernus}}} {tra} {\multicolumn{2}{l}{makang,}} {\multicolumn{2}{l}{\bluebold{Mateus}}} {tra} {makang,} {\bluebold{Wili}} {tra}\\ %
& \multicolumn{2}{l}{Oktofernus} & \textsc{neg} & \multicolumn{2}{l}{eat} & \multicolumn{2}{l}{Mateus} & \textsc{neg} & eat & Wili & \textsc{neg}\\
& makang, & \multicolumn{2}{l}{e,} & \bluebold{paytua} & \multicolumn{2}{l}{tra} & \multicolumn{5}{l}{makang}\\
& eat & \multicolumn{2}{l}{uh} & husband & \multicolumn{2}{l}{\textsc{neg}} & \multicolumn{5}{l}{eat}\\
\lspbottomrule
\end{tabular}
\ea
\glt
‘\bluebold{Oktofernus} didn’t eat, \bluebold{Mateus} didn’t eat, \bluebold{Wili} didn’t eat, uh, (my) \bluebold{husband} didn’t eat’ \textstyleExampleSource{[080921-003-CvNP.0005]}
\end{styleFreeTranslEngxvpt}

\section{Summary}
\label{bkm:Ref356411666}
Papuan Malay conjunctions typically conjoin same-type constituents. Most of them combine clauses with clauses. Only two link different-type constituents, such as verbs with clauses. Typically, the conjunctions occur at the left periphery of the constituent they mark.



The 21 conjunctions linking same-type constituents are divided into six groups according to the semantic relations they signal:
\end{styleBodyvvafter}

%\setcounter{itemize}{0}
\begin{itemize}
\item \begin{styleIIndented}
Addition: \textitbf{dengang} ‘with’, \textitbf{dang} ‘and’, \textitbf{sama} ‘to’.
\end{styleIIndented}\item \begin{styleIIndented}
Alternative: \textitbf{ato} ‘or’ and \textitbf{ka} ‘or’.
\end{styleIIndented}\item \begin{styleIIndented}
Time and/or condition: \textitbf{trus} ‘next’, \textitbf{baru} ‘and then’, \textitbf{sampe} ‘until’, \textitbf{seblum} ‘before’, and \textitbf{kalo} ‘when, if’.
\end{styleIIndented}\item \begin{styleIIndented}
Consequence: \textitbf{jadi} ‘so, since’, \textitbf{supaya} ‘so that’, \textitbf{untuk} ‘for’, \textitbf{karna} ‘because’, and \textitbf{gara-gara} ‘because’; time-marking \textitbf{sampe} also has result-marking function in the sense of ‘with the result that’.
\end{styleIIndented}\item \begin{styleIIndented}
Contrast: \textitbf{tapi} ‘but’, \textitbf{habis} ‘after all’, \textitbf{padahal} ‘but actually’, and \textitbf{biar} ‘although’; time-marking \textitbf{baru} also marks contrast in the sense of ‘after all’.
\end{styleIIndented}\item \begin{styleIvI}
Similarity: \textitbf{sperti} ‘similar to’ and \textitbf{kaya} ‘like’.
\end{styleIvI}\end{itemize}

A substantial number of the conjunctions have dual word class membership, two have trial class membership. More specifically, seven conjunctions are also used as verbs, namely \textitbf{baru} ‘and then, after all’, \textitbf{biar} ‘although’, \textitbf{habis} ‘after all’, \textitbf{jadi} ‘so, since’, \textitbf{sama} ‘to’, \textitbf{sampe} ‘until’, and \textitbf{trus} ‘next’ (see §5.3). Six conjunctions are also used as prepositions, namely \textitbf{dengang} ‘with’, \textitbf{kaya} ‘like’, \textitbf{sama} ‘to’, \textitbf{sampe} ‘until’, \textitbf{sperti} ‘similar to’, and \textitbf{untuk} ‘for’ (see §5.11 and Chapter 10). Besides, alternative-marking \textitbf{ka} ‘or’ is also used to mark interrogative clauses (see §13.2.3). (Variation in word class membership is discussed in §5.14.)



The main features of the conjunctions are summarized in two tables. Table  ‎14 .1 lists the conjunctions and the different types of constituents they link. For those linking more than one constituent type, the primary type is underlined. Empty cells signal unattested constituent combinations.
\end{styleBodyvvafter}

\begin{stylecaption}
\label{bkm:Ref356404998}Table ‎14.\stepcounter{Table}{\theTable}:  Conjunctions linking same-type constituents and the constituents they combine
\end{stylecaption}

\tablehead{
\multicolumn{2}{l}{\textsc{conjunctions}} & \textsc{cl-cl} & \textsc{np-np} & \textsc{pp-pp} & \arraybslash \textsc{vp-vp}\\
}
\begin{tabular}{llllll}
\lsptoprule
\multicolumn{2}{l}{Addition} &  &  &  & \\
& \textitbf{dengang} ‘with’ &  & \textstyleChUnderl{X} &  & \arraybslash X\\
& \textitbf{dang} ‘and’ & \textstyleChUnderl{X} & X &  & \arraybslash X\\
& \textitbf{sama} ‘to’ &  & X &  & \\
\multicolumn{2}{l}{Alternative} &  &  &  & \\
& \textitbf{ato} ‘or’ & \textstyleChUnderl{X} & X & X & \arraybslash X\\
& \textitbf{ka} ‘or’ & X & \textstyleChUnderl{X} & X & \\
\multicolumn{2}{l}{Time and Condition} &  &  &  & \\
& \textitbf{trus} ‘next’ & \textstyleChUnderl{X} & X & X & \\
& \textitbf{baru} ‘and then’ & X &  &  & \\
& \textitbf{sampe} ‘until’ & X &  &  & \\
& \textitbf{seblum} ‘before’ & X &  &  & \\
& \textitbf{kalo} ‘when, if’ & X &  &  & \\
\multicolumn{2}{l}{Consequence} &  &  &  & \\
& \textitbf{jadi} ‘so, since’ & X &  &  & \\
& \textitbf{supaya} ‘so that’ & X &  &  & \\
& \textitbf{untuk} ‘for’ & X &  &  & \\
& \textitbf{sampe} ‘with the result that’ & X &  &  & \\
& \textitbf{karna} ‘because’ & X &  &  & \\
& \textitbf{gara-gara} ‘because’ & X &  &  & \\
\multicolumn{2}{l}{Contrast} &  &  &  & \\
& \textitbf{tapi} ‘but’ & X &  &  & \\
& \textitbf{habis} ‘after all’ & X &  &  & \\
& \textitbf{baru} ‘after all’ & X &  &  & \\
& \textitbf{padahal} ‘but actually’ & X &  &  & \\
& \textitbf{biar} ‘although’ & X &  &  & \\
\multicolumn{2}{l}{Similarity} &  &  &  & \\
& \textitbf{sperti} ‘similar to’ & X &  &  & \\
& \textitbf{kaya} ‘like & X &  &  & \\
\lspbottomrule
\end{tabular}

Table  ‎14 .2 gives an overview of the positions which the conjunctions take within the clause, and the position the clause marked with a conjunction takes vis-à-vis the unmarked clause. Almost all conjunctions occur in clause-initial position, while only two occur in clause-final position. Typically, the clause marked with a conjunction follows the unmarked clause; only a few conjunctions mark clauses which precede the unmarked clause. Two of the conjunctions have two functions each, which belong to different semantic groupings, namely \textitbf{baru} ‘and then, after all’ and \textitbf{sampe} ‘until, with the result that’. Both conjunctions are listed in each of the respective groupings.


\begin{stylecaption}
\label{bkm:Ref356462326}Table ‎14.\stepcounter{Table}{\theTable}:  Conjunctions linking same-type constituents and their positions
\end{stylecaption}

\tablehead{
\multicolumn{2}{l}{\textsc{conjunctions}} & \textsc{cl} [\textsc{cnj} \textsc{cl}] & [\textsc{cnj} \textsc{cl}] \textsc{cl} & \arraybslash \textsc{cl} [\textsc{cl} \textsc{cnj}]\\
}
\begin{tabular}{lllll}
\lsptoprule
\multicolumn{2}{l}{Addition} &  &  & \\
& \textitbf{dengang} ‘with’ & X &  & \\
& \textitbf{dang} ‘and’ & X &  & \\
& \textitbf{sama} ‘to’ & X &  & \\
& Alternative &  &  & \\
& \textitbf{ato} ‘or’ & X &  & \\
& \textitbf{ka} ‘or’ & X &  & \\
\multicolumn{2}{l}{Time and Condition} &  &  & \\
& \textitbf{trus} ‘next’ & X &  & \\
& \textitbf{baru} ‘and then’ & X &  & \\
& \textitbf{sampe} ‘until’ & X &  & \\
& \textitbf{seblum} ‘before’ & X & X & \\
& \textitbf{kalo} ‘when, if’ &  & X & \\
\multicolumn{2}{l}{Consequence} &  &  & \\
& \textitbf{jadi} ‘so, since’ & X &  & \arraybslash X\\
& \textitbf{supaya} ‘so that’ & X &  & \\
& \textitbf{untuk} ‘for’ & X &  & \\
& \textitbf{sampe} ‘with the result that’ & X &  & \\
& \textitbf{karna} ‘because’ & X & X & \\
& \textitbf{gara-gara} ‘because’ & X &  & \\
\multicolumn{2}{l}{Contrast} &  &  & \\
& \textitbf{tapi} ‘but’ & X &  & \\
& \textitbf{habis} ‘after all’ & X &  & \\
& \textitbf{baru} ‘after all’ &  &  & \arraybslash X\\
& \textitbf{padahal} ‘but actually’ & X &  & \\
& \textitbf{biar} ‘although’ & X & X & \\
\multicolumn{2}{l}{Similarity} &  &  & \\
& \textitbf{sperti} ‘similar to’ & X &  & \\
& \textitbf{kaya} ‘like & X &  & \\
\lspbottomrule
\end{tabular}

The conjunctions combining different-type constituents discussed in this chapter are the complementizer \textitbf{bahwa} ‘that’ and the relativizer \textitbf{yang} ‘\textsc{rel}’. Complementizer \textitbf{bahwa} ‘that’ links a clause to a bivalent verb, while relativizer \textitbf{yang} ‘\textsc{rel}’ integrates a relative clause within a noun phrase.
\end{styleBodyaftervbefore}

%\setcounter{page}{1}\chapter[Appendices]{Appendices}
\section{Word lists}
\label{bkm:Ref376255529}\label{bkm:Ref373929511}
This appendix presents 2,215 lexemes which form the basis for the phonological analysis in Chapter 2.\footnote{\\
\\
\\
\\
\\
\\
\\
\\
\\
\\
\\
\\
\\
\\
\\
\par The 2,215 lexemes are extracted from the 2,458-item list, discussed in 1.11.6. The remaining 243 items include lexemes historically derived by (unproductive) affixation of loan words, such as \textitbf{berkomunikasi} ‘communicate’ as well as collocations such as \textitbf{ade-kaka} ‘siblings’ (see §3.1 and §3.2, respectively).} Included are 1,117 Papuan Malay lexical roots, listed in Appendix A.1 (see also §2.1), and 719 loan words, listed in Appendix A.2 (see also §2.5). Also included are 380 items, historically derived by (unproductive) affixation, listed in Appendix A.3 (see also §2.4.4.2). Upon further investigation some of the words listed as inherited Papuan Malay lexemes in Appendix A.1 and Appendix A.3 may also turn out to be loan words.



The lexemes listed in the following sections are presented in their alphabetical order. For each lexeme the following details are included: their phonetic transcription, word class, and English gloss. As discussed in §2.4.4 and §2.5.3.3, lexical roots in Papuan Malay typically carry penultimate stress. In the following sections, lexemes which do not have penultimate but ultimate or antepenultimate stress are marked with “x” for easier recognition.
\end{styleBodyvxvafter}

\subsection{Papuan Malay roots}
\label{bkm:Ref376621290}
\tablehead{ & Lexeme & Transcription & Word class & English gloss\\
}
\begin{tabular}{lllll} & \textstyleChBold{A} &  &  & \\
\lsptoprule
& \textitbf{abu} & \textstyleChCharisSIL{ˈa.bu} & \textsc{v.mo(st)} & be dusty\\
& \textitbf{ada} & \textstyleChCharisSIL{ˈa.da} & \textsc{v.bi} & exist\\
& \textitbf{ade} & \textstyleChCharisSIL{ˈa.dɛ} & \textsc{n} & younger sibling\\
& \textitbf{aduk} & \textstyleChCharisSIL{ˈa.dʊk̚} & \textsc{v.bi} & beat\\
& \textitbf{agak} & \textstyleChCharisSIL{ˈa.gɐk̚} & \textsc{adv} & rather\\
& \textitbf{air} & \textstyleChCharisSIL{ˈa.ɪr̥} & \textsc{n} & water\\
& \textitbf{ajak} & \textstyleChCharisSIL{ˈa.dʒɐk} & \textsc{v.bi} & invite\\
& \textitbf{ajar} & \textstyleChCharisSIL{ˈa.dʒɐr̥} & \textsc{v.bi} & teach\\
& \textitbf{alas} & \textstyleChCharisSIL{ˈa.lɐs} & \textsc{v.bi} & put down as base\\
& \textitbf{ambil} & \textstyleChCharisSIL{ˈɐm.bɪl} & \textsc{v.tri} & fetch\\
& \textitbf{ampas} & \textstyleChCharisSIL{ˈɐm.pɐs} & \textsc{n} & waste\\
& \textitbf{ampung} & \textstyleChCharisSIL{ˈɐm.pʊn} & \textsc{n} & forgiveness\\
& \textitbf{ana} & \textstyleChCharisSIL{ˈa.nɐk} & \textsc{n} & child\\
& \textitbf{ancam} & \textstyleChCharisSIL{ˈɐn.tʃɐm} & \textsc{v.bi} & threaten\\
& \textitbf{andal} & \textstyleChCharisSIL{ˈɐn.dɐl} & \textsc{v.mo(st)} & be reliable\\
& \textitbf{ane} & \textstyleChCharisSIL{ˈa.nɛ} & \textsc{v.mo(st)} & be strange\\
& \textitbf{anggap} & \textstyleChCharisSIL{ˈɐŋ.gɐp} & \textsc{v.bi} & regard as\\
& \textitbf{anging} & \textstyleChCharisSIL{ˈa.ŋɪn} & \textsc{n} & wind\\
& \textitbf{angkat} & \textstyleChCharisSIL{ˈɐŋ.kɐt̚} & \textsc{v.bi} & lift\\
& \textitbf{anjing} & \textstyleChCharisSIL{ˈɐn.dʒɪŋ} & \textsc{n} & dog\\
& \textitbf{antar} & \textstyleChCharisSIL{ˈɐn.tɐr̥} & \textsc{v.bi} & bring\\
& \textitbf{anyam} & \textstyleChCharisSIL{ˈa.ɲɐm} & \textsc{v.bi} & plait\\
& \textitbf{apa} & \textstyleChCharisSIL{ˈa.pa} & \textsc{int} & what\\
& \textitbf{api} & \textstyleChCharisSIL{ˈa.pi} & \textsc{n} & fire\\
& \textitbf{arang} & \textstyleChCharisSIL{ˈa.ɾɐŋ} & \textsc{n} & charcoal\\
& \textitbf{asap} & \textstyleChCharisSIL{ˈa.sɐp} & \textsc{n} & smoke\\
& \textitbf{asing} & \textstyleChCharisSIL{ˈa.sɪn} & \textsc{v.mo(st)} & be salty\\
& \textitbf{asing} & \textstyleChCharisSIL{ˈa.sɪŋ} & \textsc{v.mo(st)} & be foreign\\
& \textitbf{atas} & \textstyleChCharisSIL{ˈa.tɐs} & \textsc{n-loc} & top\\
& \textitbf{atur} & \textstyleChCharisSIL{ˈa.tʊr̥} & \textsc{v.bi} & arrange\\
& \textitbf{awang} & \textstyleChCharisSIL{ˈa.wɐŋ} & \textsc{n} & cloud\\
& \textitbf{awas} & \textstyleChCharisSIL{ˈa.wɐs} & \textsc{v.mo(dy)} & watch out\\
& \textitbf{ayam} & \textstyleChCharisSIL{ˈa.jɐm} & \textsc{n} & chicken\\
& \textitbf{ayung} & \textstyleChCharisSIL{ˈa.jʊn} & \textsc{v.bi} & hit\\
& \textstyleChBold{B} &  &  & \\
& \textitbf{babat} & \textstyleChCharisSIL{ˈba.bɐt̚} & \textsc{v.bi} & clear away\\
& \textitbf{babi} & \textstyleChCharisSIL{ˈba.bi} & \textsc{n} & pig\\
& \textitbf{bagus} & \textstyleChCharisSIL{ˈba.gʊs} & \textsc{v.mo(st)} & be good\\
& \textitbf{baik} & \textstyleChCharisSIL{ˈba.ɪk̚} & \textsc{v.mo(st)} & be good\\
& \textitbf{bakar} & \textstyleChCharisSIL{ˈba.kɐr̥} & \textsc{v.bi} & burn\\
& \textitbf{baku} & \textstyleChCharisSIL{ˈba.ku} & \textsc{recp} & reciprocal\\
& \textitbf{balap} & \textstyleChCharisSIL{ˈba.lɐp̚} & \textsc{v.bi} & race\\
& \textitbf{balas} & \textstyleChCharisSIL{ˈba.lɐs} & \textsc{v.bi} & reply\\
& \textitbf{balay} & \textstyleChCharisSIL{ˈba.lɐj} & \textsc{n} & meeting hall\\
& \textitbf{balik} & \textstyleChCharisSIL{ˈba.lɪk̚} & \textsc{v.bi} & turn around\\
& \textitbf{balok} & \textstyleChCharisSIL{ˈba.lɔ̞k̚} & \textsc{n} & wooden beam\\
& \textitbf{balut} & \textstyleChCharisSIL{ˈba.lʊt} & \textsc{v.bi} & bandage\\
& \textitbf{bambu} & \textstyleChCharisSIL{ˈbɐm.bu} & \textsc{n} & bamboo\\
& \textitbf{banci} & \textstyleChCharisSIL{ˈbɐn.tʃi} & \textsc{n} & homosexual male\\
& \textitbf{bandar} & \textstyleChCharisSIL{ˈbɐn.dɐr̥} & \textsc{n} & stick\\
& \textitbf{banding} & \textstyleChCharisSIL{ˈbɐn.dɪŋ} & \textsc{v.bi} & compare\\
& \textitbf{bangga} & \textstyleChCharisSIL{ˈbɐŋ.ga} & \textsc{v.mo(st)} & be proud\\
& \textitbf{bangkit} & \textstyleChCharisSIL{ˈbɐŋ.kɪt̚} & \textsc{v.mo(st)} & be resurrected\\
& \textitbf{bangung} & \textstyleChCharisSIL{ˈba.ŋʊn} & \textsc{v.bi} & build\\
& \textitbf{bangung} & \textstyleChCharisSIL{ˈba.ŋʊn} & \textsc{v} & wake up\\
& \textitbf{banjir} & \textstyleChCharisSIL{ˈbɐn.dʒɪr̥} & \textsc{n} & flooding\\
& \textitbf{bantal} & \textstyleChCharisSIL{ˈbɐn.tɐl} & \textsc{n} & pillow\\
& \textitbf{banting} & \textstyleChCharisSIL{ˈbɐn.tɪŋ} & \textsc{v.bi} & throw\\
& \textitbf{bantu} & \textstyleChCharisSIL{ˈbɐn.tu} & \textsc{v.bi} & help\\
& \textitbf{banyak} & \textstyleChCharisSIL{ˈba.ɲɐk̚} & \textsc{qt} & many\\
& \textitbf{bapa} & \textstyleChCharisSIL{ˈba.pa} & \textsc{n} & father\\
& \textitbf{barang} & \textstyleChCharisSIL{ˈba.ɾɐŋ} & \textsc{n} & stuff\\
& \textitbf{barapeng} & \textstyleChCharisSIL{ba.ˈɾa.pɛ̞n} & \textsc{v.bi} & cook with hot stones\\
& \textitbf{baring} & \textstyleChCharisSIL{ˈba.ɾɪŋ} & \textsc{v.mo(dy)} & lie down\\
& \textitbf{baris} & \textstyleChCharisSIL{ˈba.ɾɪs} & \textsc{n} & row\\
& \textitbf{baru} & \textstyleChCharisSIL{ˈba.ɾʊ} & \textsc{v.mo(st)}

\textsc{adv}

\textsc{cnj} & be new

recently

and then, after all\\
& \textitbf{barusang} & \textstyleChCharisSIL{ba.ˈɾu.sɐn} & \textsc{adv} & just now\\
& \textitbf{basa} & \textstyleChCharisSIL{ˈba.sa} & \textsc{v.mo(st)} & be wet\\
& \textitbf{batang} & \textstyleChCharisSIL{ˈba.tɐŋ} & \textsc{n} & stick\\
& \textitbf{batas} & \textstyleChCharisSIL{ˈba.tɐs} & \textsc{n} & border\\
& \textitbf{batu} & \textstyleChCharisSIL{ˈba.tu} & \textsc{n} & stone\\
& \textitbf{bawa} & \textstyleChCharisSIL{ˈba.wa} & \textsc{n-loc} & bottom\\
& \textitbf{bawa} & \textstyleChCharisSIL{ˈba.wa} & \textsc{v.tri} & bring\\
& \textitbf{bawang} & \textstyleChCharisSIL{ˈba.wɐn} & \textsc{n} & onion\\
& \textitbf{bayam} & \textstyleChCharisSIL{ˈba.jɐm} & \textsc{n} & amaranth\\
& \textitbf{bayang} & \textstyleChCharisSIL{ˈba.jɐŋ} & \textsc{n} & image\\
& \textitbf{bayar} & \textstyleChCharisSIL{ˈba.jɐr̥} & \textsc{v.bi} & pay\\
& \textitbf{bayi} & \textstyleChCharisSIL{ˈba.ji} & \textsc{n} & baby\\
& \textitbf{bayi} & \textstyleChCharisSIL{ˈba.ji} & \textsc{n} & palm stem\\
\textstyleExampleSource{x} & \textitbf{bebang} & \textstyleChCharisSIL{bɛ.ˈbɐn} & \textsc{n} & burden\\
& \textitbf{bebas} & \textstyleChCharisSIL{ˈbɛ.bɐs} & \textsc{v.mo(st)} & be free\\
& \textitbf{bebek} & \textstyleChCharisSIL{ˈbɛ.bɛ̞k} & \textsc{n} & duck\\
& \textitbf{begini} & \textstyleChCharisSIL{bɛ.ˈgi.ni} & \textsc{adv} & like this\\
& \textitbf{begitu} & \textstyleChCharisSIL{bɛ.ˈgi.tu} & \textsc{adv} & like that\\
\textstyleExampleSource{x} & \textitbf{bekal} & \textstyleChCharisSIL{bɛ.ˈkɐl} & \textsc{v.mo(st)} & be equipped\\
\textstyleExampleSource{x} & \textitbf{bekas} & \textstyleChCharisSIL{bə.ˈkɐs} & \textsc{n} & trace\\
& \textitbf{belalang} & \textstyleChCharisSIL{bɛ.ˈla.lɐŋ} & \textsc{n} & grasshopper\\
& \textitbf{belok} & \textstyleChCharisSIL{ˈbɛ.lɔ̞k} & \textsc{v.mo(dy)} & turn\\
\textstyleExampleSource{x} & \textitbf{benang} & \textstyleChCharisSIL{bɛ.ˈnɐŋ} & \textsc{n} & thread\\
\textstyleExampleSource{x} & \textitbf{benar} & \textstyleChCharisSIL{bɛ.ˈnɐr} & \textsc{v.mo(st)} & be true\\
& \textitbf{bencong} & \textstyleChCharisSIL{ˈbɛ̞n.tʃɔ̞ŋ} & \textsc{n} & transvestite\\
\textstyleExampleSource{x} & \textitbf{bengkak} & \textstyleChCharisSIL{bɛ̞ŋ.ˈkɐk̚} & \textsc{v.mo(st)} & be swollen\\
& \textitbf{bengkok} & \textstyleChCharisSIL{ˈbɛ̞ŋ.kɔ̞k̚} & \textsc{v.mo(st)} & be crooked\\
\textstyleExampleSource{x} & \textitbf{bentuk} & \textstyleChCharisSIL{bɛ̞n.ˈtʊk} & \textsc{v.bi} & form\\
& \textitbf{bera} & \textstyleChCharisSIL{ˈbɛ̞.ɾa} & \textsc{v.bi} & defecate\\
& \textitbf{beres} & \textstyleChCharisSIL{ˈbɛ̞.ɾɛ̞s} & \textsc{v.mo(st)} & be in order\\
& \textitbf{berhala} & \textstyleChCharisSIL{bɛ̞r.ˈha.la} & \textsc{n} & idol\\
\textstyleExampleSource{x} & \textitbf{bernang} & \textstyleChCharisSIL{bɛ̞r.ˈnɐŋ} & \textsc{v.mo(dy)} & swim\\
\textstyleExampleSource{x} & \textitbf{bersi} & \textstyleChCharisSIL{bɛ̞r.ˈsi} & \textsc{v.mo(st)} & be clean\\
\textstyleExampleSource{x} & \textitbf{besar} & \textstyleChCharisSIL{bɛ.ˈsɐr} & \textsc{v.mo(st)} & be big\\
& \textitbf{besi} & \textstyleChCharisSIL{ˈbɛ.si} & \textsc{n} & metal\\
& \textitbf{besok} & \textstyleChCharisSIL{ˈbɛ.sɔ̞k̚} & \textsc{n} & tomorrow\\
& \textitbf{bete} & \textstyleChCharisSIL{ˈbɛ.tɛ} & \textsc{n} & taro\\
\textstyleExampleSource{x} & \textitbf{betul} & \textstyleChCharisSIL{bɛ.ˈtʊl} & \textsc{v.mo(st)} & be true\\
& \textitbf{biang} & \textstyleChCharisSIL{ˈbɪ.ɐŋ} & \textsc{n} & main root stock\\
& \textitbf{biar} & \textstyleChCharisSIL{ˈbi.ɐr} & \textsc{v.bi}

\textsc{cnj} & let

although\\
& \textitbf{bibit} & \textstyleChCharisSIL{ˈbɪ.bɪt} & \textsc{n} & seed\\
& \textitbf{biking} & \textstyleChCharisSIL{ˈbi.kɪn} & \textsc{v.bi} & make\\
& \textitbf{bilang} & \textstyleChCharisSIL{ˈbi.lɐŋ} & \textsc{v.bi} & say\\
& \textitbf{bimbing} & \textstyleChCharisSIL{ˈbɪm.bɪŋ} & \textsc{v.bi} & lead\\
& \textitbf{binatang} & \textstyleChCharisSIL{bi.ˈna.tɐn} & \textsc{n} & animal\\
& \textitbf{bingung} & \textstyleChCharisSIL{ˈbi.ŋʊŋ} & \textsc{v.mo(st)} & be confused\\
& \textitbf{bini} & \textstyleChCharisSIL{ˈbi.ni} & \textsc{n} & wife\\
& \textitbf{bintang} & \textstyleChCharisSIL{ˈbɪn.tɐŋ} & \textsc{n} & star\\
& \textitbf{biru} & \textstyleChCharisSIL{ˈbi.ɾu} & \textsc{v.mo(st)} & be blue\\
& \textitbf{bisa} & \textstyleChCharisSIL{ˈbi.sa} & \textsc{v.mo(st)} & be able\\
& \textitbf{bisik} & \textstyleChCharisSIL{ˈbi.sɪk̚} & \textsc{v.bi} & whisper\\
& \textitbf{bisu} & \textstyleChCharisSIL{ˈbi.su} & \textsc{v.mo(st)} & be mute\\
& \textitbf{bla} & \textstyleChCharisSIL{ˈbla} & \textsc{v.bi} & split\\
& \textitbf{blakang} & \textstyleChCharisSIL{ˈbla.kɐŋ} & \textsc{n-loc} & backside\\
& \textitbf{blanga} & \textstyleChCharisSIL{ˈbla.ŋa} & \textsc{n} & cooking pot\\
& \textitbf{blanja} & \textstyleChCharisSIL{ˈblɐn.dʒa} & \textsc{v.bi} & shop\\
& \textitbf{blas} & \textstyleChCharisSIL{ˈblɐs} & \textsc{num.c} & teens\\
& \textitbf{bli} & \textstyleChCharisSIL{ˈbli} & \textsc{v.tri} & buy\\
& \textitbf{blimbing} & \textstyleChCharisSIL{ˈblɪm.bɪŋ} & \textsc{n} & star fruit\\
& \textitbf{blum} & \textstyleChCharisSIL{ˈblʊm} & \textsc{adv} & not yet\\
& \textitbf{bobo} & \textstyleChCharisSIL{ˈbɔ.bɔ} & \textsc{n} & Nipah palm fruit schnapps\\
& \textitbf{bocor} & \textstyleChCharisSIL{ˈbɔ.tʃɔ̞r} & \textsc{v.mo(dy)} & leak\\
& \textitbf{bodo} & \textstyleChCharisSIL{ˈbɔ.dɔ} & \textsc{v.mo(st)} & be stupid\\
& \textitbf{bole} & \textstyleChCharisSIL{ˈbɔ.lɛ} & \textsc{v.aux} & may\\
& \textitbf{bongkar} & \textstyleChCharisSIL{ˈbɔ̞ŋ.kɐr̥} & \textsc{v.bi} & unload\\
& \textitbf{bongkok} & \textstyleChCharisSIL{ˈbɔ̞ŋ.kɔ̞k̚} & \textsc{v.mo(st)} & be bent over\\
& \textitbf{bongso} & \textstyleChCharisSIL{ˈbɔ̞ŋ.sɔ̞} & \textsc{n} & youngest offspring\\
& \textitbf{borgol} & \textstyleChCharisSIL{ˈbɔ̞r.gɔ̞l} & \textsc{v.bi} & handcuff\\
& \textitbf{bosang} & \textstyleChCharisSIL{ˈbɔ.sɐn} & \textsc{v.mo(st)} & be bored\\
& \textitbf{botak} & \textstyleChCharisSIL{ˈbɔ.tɐk̚} & \textsc{v.mo(st)} & be bald\\
& \textitbf{brani} & \textstyleChCharisSIL{ˈbɾa.ni} & \textsc{v.mo(st)} & be courageous\\
& \textitbf{brapa} & \textstyleChCharisSIL{ˈbra.pa} & \textsc{int} & several\\
& \textitbf{bras} & \textstyleChCharisSIL{ˈbrɐs} & \textsc{n} & hulled rice\\
& \textitbf{brat} & \textstyleChCharisSIL{ˈbɾɐt̚} & \textsc{v.mo(st)} & be heavy\\
& \textitbf{bua} & \textstyleChCharisSIL{ˈbu.a} & \textsc{n} & fruit\\
& \textitbf{buang} & \textstyleChCharisSIL{ˈbu.ɐŋ} & \textsc{v.bi} & discard\\
& \textitbf{buat} & \textstyleChCharisSIL{ˈbu.ɐt} & \textsc{v.bi} & make\\
& \textitbf{buat} & \textstyleChCharisSIL{ˈbʊ.ɐt} & \textsc{prep} & for\\
& \textitbf{buaya} & \textstyleChCharisSIL{bʊ.ˈa.ja} & \textsc{n} & crocodile\\
& \textitbf{bubar} & \textstyleChCharisSIL{ˈbu.bɐr̥} & \textsc{v.bi} & scatter\\
& \textitbf{bujang} & \textstyleChCharisSIL{ˈbu.dʒɐŋ} & \textsc{v.mo(st)} & be unmarried\\
& \textitbf{bujuk} & \textstyleChCharisSIL{ˈbu.dʒʊk} & \textsc{v.bi} & trick\\
& \textitbf{buka} & \textstyleChCharisSIL{ˈbu.ka} & \textsc{v.bi} & open\\
& \textitbf{bukang} & \textstyleChCharisSIL{ˈbʊ.kɐn} & \textsc{adv} & \textsc{neg}\\
& \textitbf{bukit} & \textstyleChCharisSIL{ˈbu.kɪt} & \textsc{n} & mountain\\
& \textitbf{bulang} & \textstyleChCharisSIL{ˈbu.lɐn} & \textsc{n} & month\\
& \textitbf{bulat} & \textstyleChCharisSIL{ˈbu.lɐt̚} & \textsc{v.mo(st)} & be round\\
& \textitbf{bule} & \textstyleChCharisSIL{ˈbu.lɛ} & \textsc{n} & white person\\
& \textitbf{bulu} & \textstyleChCharisSIL{ˈbu.lu} & \textsc{n} & body hair\\
& \textitbf{bumbu} & \textstyleChCharisSIL{ˈbʊm.bʊ} & \textsc{n} & spice\\
& \textitbf{bunga} & \textstyleChCharisSIL{ˈbu.ŋa} & \textsc{n} & flower\\
& \textitbf{bungkus} & \textstyleChCharisSIL{ˈbʊŋ.kʊs} & \textsc{v.bi} & pack\\
& \textitbf{buntu} & \textstyleChCharisSIL{ˈbʊn.tʊ} & \textsc{v.mo(st)} & be blocked\\
& \textitbf{bunu} & \textstyleChCharisSIL{ˈbu.nu} & \textsc{v.bi} & kill\\
& \textitbf{bunyi} & \textstyleChCharisSIL{ˈbu.ɲi} & \textsc{n} & sound\\
& \textitbf{buru} & \textstyleChCharisSIL{ˈbu.ɾu} & \textsc{v.bi} & hunt\\
& \textitbf{burung} & \textstyleChCharisSIL{ˈbʊ.ɾʊŋ} & \textsc{n} & bird\\
& \textitbf{busa} & \textstyleChCharisSIL{ˈbu.sa} & \textsc{n} & foam\\
& \textitbf{busuk} & \textstyleChCharisSIL{ˈbʊ.sʊk̚} & \textsc{v.mo(st)} & be rotten\\
& \textitbf{busur} & \textstyleChCharisSIL{ˈbu.sʊr̥} & \textsc{n} & bow\\
& \textitbf{buta} & \textstyleChCharisSIL{ˈbu.ta} & \textsc{v.mo(st)} & be blind\\
& \textitbf{butu} & \textstyleChCharisSIL{ˈbu.tu} & \textsc{v.bi} & need\\
& \textstyleChBold{C} &  &  & \\
& \textitbf{cabang} & \textstyleChCharisSIL{ˈtʃa.bɐŋ} & \textsc{n} & branch\\
& \textitbf{cabut} & \textstyleChCharisSIL{ˈtʃa.bʊt} & \textsc{v.bi} & pull out\\
& \textitbf{cacat} & \textstyleChCharisSIL{ˈtʃa.tʃɐt̚} & \textsc{v.mo(st)} & be disabled\\
& \textitbf{cakar} & \textstyleChCharisSIL{ˈtʃa.kɐr̥} & \textsc{v.bi} & scratch\\
& \textitbf{calong} & \textstyleChCharisSIL{ˈtʃa.lɔ̞n} & \textsc{n} & candidate\\
& \textitbf{camat} & \textstyleChCharisSIL{ˈtʃa.mɐt} & \textsc{n} & subdistrict head\\
& \textitbf{campur} & \textstyleChCharisSIL{ˈtsɐm.pʊr̥} & \textsc{v.bi} & mix\\
& \textitbf{canggi} & \textstyleChCharisSIL{ˈtʃɐŋ.gi} & \textsc{v.mo(st)} & be sophisticated\\
& \textitbf{cangkul} & \textstyleChCharisSIL{ˈtʃɐŋ.kʊl} & \textsc{n} & mattock\\
& \textitbf{cantik} & \textstyleChCharisSIL{ˈtʃɐn.tɪk̚} & \textsc{v.mo(st)} & be beautiful\\
& \textitbf{cape} & \textstyleChCharisSIL{ˈtʃa.pɛ} & \textsc{v.mo(st)} & be tired\\
& \textitbf{catat} & \textstyleChCharisSIL{ˈtʃa.tɐt} & \textsc{v.bi} & note\\
& \textitbf{cebo} & \textstyleChCharisSIL{ˈtʃɛ.bɔ} & \textsc{v.bi} & wash after defecating\\
& \textitbf{cece} & \textstyleChCharisSIL{ˈtʃɛ.tʃɛ} & \textsc{n} & great-grandchild\\
\textstyleExampleSource{x} & \textitbf{cegat} & \textstyleChCharisSIL{tʃɛ.ˈgɐt̚} & \textsc{v.bi} & hold up\\
& \textitbf{cengeng} & \textstyleChCharisSIL{ˈtʃɛ̞.ŋɛ̞ŋ} & \textsc{v.mo(st)} & be a crybaby\\
\textstyleExampleSource{x} & \textitbf{cepat} & \textstyleChCharisSIL{tʃɛ.ˈpɐt̚} & \textsc{v.mo(st)} & be fast\\
\textstyleExampleSource{x} & \textitbf{cerey} & \textstyleChCharisSIL{tʃɛ.ˈɾɛ̞j} & \textsc{v.bi} & divorce\\
& \textitbf{cetak} & \textstyleChCharisSIL{ˈtʃɛ.tɐk} & \textsc{v.bi} & print\\
& \textitbf{cewe} & \textstyleChCharisSIL{ˈtʃɛ.wɛ} & \textsc{n} & girl\\
& \textitbf{cici} & \textstyleChCharisSIL{ˈtʃi.tʃi} & \textsc{n} & great-great-grandchild\\
& \textitbf{cincang} & \textstyleChCharisSIL{ˈtʃɪn.tʃɐŋ} & \textsc{v.bi} & chop up\\
& \textitbf{ciri} & \textstyleChCharisSIL{ˈtʃi.ɾi} & \textsc{n} & feature\\
& \textitbf{cium} & \textstyleChCharisSIL{ˈtʃi.ʊm} & \textsc{v.bi} & kiss\\
& \textitbf{coba} & \textstyleChCharisSIL{ˈtʃɔ.ba} & \textsc{v.bi}

\textsc{adv} & try

if only\\
& \textitbf{cobe} & \textstyleChCharisSIL{ˈtʃɔ.bɛ} & \textsc{n} & mortar\\
& \textitbf{coblos} & \textstyleChCharisSIL{ˈtʃɔ.blɔ̞s} & \textsc{v.bi} & punch\\
& \textitbf{cocok} & \textstyleChCharisSIL{ˈtʃɔ.tʃɔ̞k̚} & \textsc{v.mo(st)} & be suitable\\
& \textitbf{colo} & \textstyleChCharisSIL{ˈtʃɔ.lɔ} & \textsc{v.bi} & immerse\\
& \textitbf{conto} & \textstyleChCharisSIL{ˈtʃɔ̞n.tɔ̞} & \textsc{n} & example\\
& \textitbf{crewet} & \textstyleChCharisSIL{ˈtʃɾɛ.wɛ̞t̚} & \textsc{v.mo(st)} & be chatty\\
& \textitbf{cucu} & \textstyleChCharisSIL{ˈtʃu.tʃu} & \textsc{n} & grandchild\\
& \textitbf{cuki} & \textstyleChCharisSIL{ˈtʃu.ki} & \textsc{v.bi} & fuck\\
& \textitbf{cukup} & \textstyleChCharisSIL{ˈtʃu.kʊp} & \textsc{v.mo(st)} & be enough\\
& \textitbf{cukur} & \textstyleChCharisSIL{ˈtʃu.kʊr̥} & \textsc{v.bi} & level\\
& \textitbf{culik} & \textstyleChCharisSIL{ˈtʃu.lɪk} & \textsc{v.bi} & kidnap\\
& \textitbf{curang} & \textstyleChCharisSIL{ˈtʃu.ɾɐŋ} & \textsc{v.mo(st)} & be dishonest\\
& \textstyleChBold{D} &  &  & \\
& \textitbf{dada} & \textstyleChCharisSIL{ˈda.da} & \textsc{n} & chest\\
& \textitbf{daging} & \textstyleChCharisSIL{ˈda.gɪŋ} & \textsc{n} & meat\\
& \textitbf{daki} & \textstyleChCharisSIL{ˈda.ki} & \textsc{n} & grime\\
& \textitbf{dalam} & \textstyleChCharisSIL{ˈda.lɐm} & \textsc{n-loc} & inside\\
& \textitbf{damay} & \textstyleChCharisSIL{ˈda.mɐj} & \textsc{n} & peace\\
& \textitbf{dang} & \textstyleChCharisSIL{ˈdɐn} & \textsc{cnj} & and\\
& \textitbf{dano} & \textstyleChCharisSIL{ˈda.nɔ} & \textsc{n} & lake\\
& \textitbf{dapat} & \textstyleChCharisSIL{ˈda.pɐt̚} & \textsc{v.bi} & get\\
& \textitbf{dapur} & \textstyleChCharisSIL{ˈda.pʊr̥} & \textsc{n} & kitchen\\
& \textitbf{dara} & \textstyleChCharisSIL{ˈda.ɾa} & \textsc{n} & blood\\
& \textitbf{darat} & \textstyleChCharisSIL{ˈda.ɾɐt} & \textsc{n} & land\\
& \textitbf{dari} & \textstyleChCharisSIL{ˈda.ɾi} & \textsc{prep} & from\\
& \textitbf{dasar} & \textstyleChCharisSIL{ˈda.sɐr̥} & \textsc{n} & base\\
& \textitbf{datang} & \textstyleChCharisSIL{ˈda.tɐŋ} & \textsc{v.mo(dy)} & come\\
& \textitbf{daung} & \textstyleChCharisSIL{ˈda.ʊn} & \textsc{n} & leaf\\
& \textitbf{dayung} & \textstyleChCharisSIL{ˈda.jʊŋ} & \textsc{n/v.bi} & paddle\\
& \textitbf{de} & \textstyleChCharisSIL{ˈdɛ} & \textsc{pro} & \textsc{3sg}\\
\textstyleExampleSource{x} & \textitbf{dekat} & \textstyleChCharisSIL{dɛ.ˈkɐt̚} & \textsc{v.bi} & near\\
\textstyleExampleSource{x} & \textitbf{dengang} & \textstyleChCharisSIL{dɛ.ˈŋɐn} & \textsc{prep/cnj} & with\\
\textstyleExampleSource{x} & \textitbf{dengar} & \textstyleChCharisSIL{dɛ.ˈŋɐr̥} & \textsc{v.bi} & hear\\
& \textitbf{depang} & \textstyleChCharisSIL{ˈdɛ.pɐn} & \textsc{n-loc} & front\\
\textstyleExampleSource{x} & \textitbf{desak} & \textstyleChCharisSIL{dɛ.ˈsɐk} & \textsc{v.bi} & urge\\
& \textitbf{di} & \textstyleChCharisSIL{ˈdi} & \textsc{prep} & at\\
& \textitbf{dia} & \textstyleChCharisSIL{ˈdɪ.a} & \textsc{pro} & \textsc{3sg}\\
& \textitbf{diam} & \textstyleChCharisSIL{ˈdi.ɐm} & \textsc{v.mo(st)} & be quiet\\
& \textitbf{didik} & \textstyleChCharisSIL{ˈdi.dɪk} & \textsc{v.bi} & educate\\
& \textitbf{dinding} & \textstyleChCharisSIL{ˈdɪn.dɪŋ} & \textsc{n} & interior wall\\
& \textitbf{dinging} & \textstyleChCharisSIL{ˈdi.ŋɪn} & \textsc{v.mo(st)} & be cold\\
& \textitbf{diri} & \textstyleChCharisSIL{ˈdi.ɾi} & \textsc{n} & self\\
& \textitbf{dlapang} & \textstyleChCharisSIL{ˈdla.pɐn} & \textsc{num.c} & eight\\
& \textitbf{dong} & \textstyleChCharisSIL{ˈdɔ̞ŋ} & \textsc{pro} & \textsc{3pl}\\
& \textitbf{dongeng} & \textstyleChCharisSIL{ˈdɔ.ŋɛ̞n} & \textsc{n} & legend\\
& \textitbf{dorang} & \textstyleChCharisSIL{ˈdɔ.ɾɐŋ} & \textsc{pro} & \textsc{3pl}\\
& \textitbf{dorong} & \textstyleChCharisSIL{ˈdɔ.ɾɔ̞ŋ} & \textsc{v.bi} & push\\
& \textitbf{dua} & \textstyleChCharisSIL{ˈdu.a} & \textsc{num.c} & two\\
& \textitbf{duduk} & \textstyleChCharisSIL{ˈdʊ.dʊk} & \textsc{v.mo(dy)} & sit\\
& \textitbf{dukung} & \textstyleChCharisSIL{ˈdʊ.kʊŋ} & \textsc{v.bi} & support\\
& \textitbf{dulu} & \textstyleChCharisSIL{ˈdu.lu} & \textsc{v.mo(st)}

\textsc{adv} & be prior

first, in the past\\
& \textitbf{dumpul} & \textstyleChCharisSIL{ˈdʊm.pʊl} & \textsc{v.bi} & hit\\
& \textitbf{duri} & \textstyleChCharisSIL{ˈdu.ɾi} & \textsc{n} & thorn\\
& \textitbf{dusung} & \textstyleChCharisSIL{ˈdʊ.sʊn} & \textsc{n} & garden\\
& \textstyleChBold{E} &  &  & \\
& \textitbf{ejek} & \textstyleChCharisSIL{ˈɛ.dʒɛ̞k̚} & \textsc{v.bi} & mock\\
& \textitbf{ekor} & \textstyleChCharisSIL{ˈɛ.kɔ̞r̥} & \textsc{n} & tail\\
\textstyleExampleSource{x} & \textitbf{emas} & \textstyleChCharisSIL{ɛ.ˈmɐs} & \textsc{n} & gold\\
\textstyleExampleSource{x} & \textitbf{empat} & \textstyleChCharisSIL{əm.ˈpɐt̚} & \textsc{num.c} & four\\
& \textitbf{enak} & \textstyleChCharisSIL{ˈɛ.nɐk̚} & \textsc{v.mo(st)} & be pleasant\\
\textstyleExampleSource{x} & \textitbf{enam} & \textstyleChCharisSIL{ɛ.ˈnɐm} & \textsc{num.c} & six\\
& \textitbf{epeng} & \textstyleChCharisSIL{ˈɛ̞.pɛ̞ŋ} & \textsc{v.mo(st)} & be important\\
& \textstyleChBold{G} &  &  & \\
& \textitbf{gaba} & \textstyleChCharisSIL{ˈga.ba} & \textsc{n} & unhulled paddy\\
& \textitbf{gabung} & \textstyleChCharisSIL{ˈga.bʊŋ} & \textsc{v.bi} & join\\
& \textitbf{gagal} & \textstyleChCharisSIL{ˈga.gɐl} & \textsc{v.bi} & fail\\
& \textitbf{gale} & \textstyleChCharisSIL{ˈga.lɛ̞} & \textsc{v.bi} & dig up\\
& \textitbf{gambar} & \textstyleChCharisSIL{ˈgɐm.bɐr̥} & \textsc{n/v.bi} & drawing / draw\\
& \textitbf{gampang} & \textstyleChCharisSIL{ˈgɐm.pɐŋ} & \textsc{v.mo(st)} & be easy\\
& \textitbf{ganas} & \textstyleChCharisSIL{ˈga.nɐs} & \textsc{v.bi} & feel furious (about)\\
& \textitbf{gandeng} & \textstyleChCharisSIL{ˈgɐn.dɛ̞ŋ} & \textsc{v.bi} & hold\\
& \textitbf{ganggu} & \textstyleChCharisSIL{ˈgɐŋ.gu} & \textsc{v.bi} & disturb\\
& \textitbf{ganjal} & \textstyleChCharisSIL{ˈgɐn.dʒɐl} & \textsc{v.bi} & prop up\\
& \textitbf{ganteng} & \textstyleChCharisSIL{ˈgɐn.tɛ̞ŋ} & \textsc{v.mo(st)} & be handsome\\
& \textitbf{ganti} & \textstyleChCharisSIL{ˈgɐn.ti} & \textsc{v.bi} & replace\\
& \textitbf{gantong} & \textstyleChCharisSIL{ˈgɐn.tɔ̞ŋ} & \textsc{v.bi} & hang\\
& \textitbf{gara} & \textstyleChCharisSIL{ˈga.ɾa} & \textsc{v.bi} & irritate\\
& \textitbf{garam} & \textstyleChCharisSIL{ˈga.ɾɐm} & \textsc{n} & salt\\
& \textitbf{garo} & \textstyleChCharisSIL{ˈga.ɾɔ} & \textsc{v.bi} & scratch\\
& \textitbf{gatal} & \textstyleChCharisSIL{ˈga.tɐl} & \textsc{v.mo(st)} & be itchy\\
& \textitbf{gate} & \textstyleChCharisSIL{ˈga.tɛ} & \textsc{v.bi} & hook\\
& \textitbf{gawang} & \textstyleChCharisSIL{ˈga.wɐŋ} & \textsc{n} & goal posts\\
& \textitbf{gawat} & \textstyleChCharisSIL{ˈga.wɐt̚} & \textsc{v.mo(st)} & be terrible\\
& \textitbf{gaya} & \textstyleChCharisSIL{ˈga.ja} & \textsc{n} & manner\\
& \textitbf{gedi} & \textstyleChCharisSIL{ˈgɛ.di} & \textsc{n} & aibika\\
\textstyleExampleSource{x} & \textitbf{gedung} & \textstyleChCharisSIL{gɛ.ˈdʊŋ} & \textsc{n} & building\\
\textstyleExampleSource{x} & \textitbf{geli} & \textstyleChCharisSIL{gɛ.ˈli} & \textsc{v.bi} & tickle\\
\textstyleExampleSource{x} & \textitbf{gementar} & \textstyleChCharisSIL{ˌgɛ.mɛ̞n.ˈtɐr̥} & \textsc{v.mo(dy)} & tremble\\
\textstyleExampleSource{x} & \textitbf{gemuk} & \textstyleChCharisSIL{gɛ.ˈmʊk̚} & \textsc{v.mo(st)} & be fat\\
\textstyleExampleSource{x} & \textitbf{gencar} & \textstyleChCharisSIL{gɛ̞n.ˈtʃɐr} & \textsc{v.mo(st)} & be incessant\\
& \textitbf{gendong} & \textstyleChCharisSIL{ˈgɛ̞n.dɔ̞ŋ} & \textsc{v.bi} & hold\\
& \textitbf{gepe} & \textstyleChCharisSIL{ˈgɛ.pɛ} & \textsc{v.bi} & clamp\\
\textstyleExampleSource{x} & \textitbf{gertak} & \textstyleChCharisSIL{gɛ̞r.ˈtɐk} & \textsc{v.bi} & intimidate\\
& \textitbf{giawas} & \textstyleChCharisSIL{gɪ.ˈa.wɐs} & \textsc{n} & guava\\
& \textitbf{gigi} & \textstyleChCharisSIL{ˈgi.gi} & \textsc{n} & tooth\\
& \textitbf{gigit} & \textstyleChCharisSIL{ˈgi.gɪt} & \textsc{v.bi} & bite\\
& \textitbf{gila} & \textstyleChCharisSIL{ˈgi.la} & \textsc{v.mo(st)} & be crazy\\
& \textitbf{giling} & \textstyleChCharisSIL{ˈgɪ.lɪŋ} & \textsc{v.bi} & grind\\
& \textitbf{glang} & \textstyleChCharisSIL{ˈglɐŋ} & \textsc{n} & bracelet\\
& \textitbf{glap} & \textstyleChCharisSIL{ˈglɐp̚} & \textsc{v.mo(st)} & be dark\\
& \textitbf{glombang} & \textstyleChCharisSIL{ˈglɔ̞m.bɐŋ} & \textsc{n} & wave\\
& \textitbf{gnemo} & \textstyleChCharisSIL{ˈgnɛ.mɔ} & \textsc{n} & melinjo\\
& \textitbf{goblok} & \textstyleChCharisSIL{ˈgɔ.blɔ̞k̚} & \textsc{v.mo(st)} & be stupid\\
& \textitbf{gode} & \textstyleChCharisSIL{ˈgɔ.dɛ} & \textsc{v.mo(st)} & be fat\\
& \textitbf{gonceng} & \textstyleChCharisSIL{ˈgɔ̞n.tʃɛ̞ŋ} & \textsc{v.bi} & give a ride\\
& \textitbf{gondrong} & \textstyleChCharisSIL{ˈgɔ̞n.dɾɔ̞ŋ} & \textsc{v.mo(st)} & be long haired\\
& \textitbf{gora} & \textstyleChCharisSIL{ˈgɔ.ɾa} & \textsc{n} & water apple\\
& \textitbf{goreng} & \textstyleChCharisSIL{ˈgɔ.ɾɛ̞ŋ} & \textsc{v.bi} & fry\\
& \textitbf{goso} & \textstyleChCharisSIL{ˈgɔ.sɔ} & \textsc{v.bi} & rub\\
& \textitbf{goyang} & \textstyleChCharisSIL{ˈgɔ.jɐŋ} & \textsc{v.bi} & shake\\
& \textitbf{gugat} & \textstyleChCharisSIL{ˈgu.gɐt} & \textsc{v.bi} & demand\\
& \textitbf{gugur} & \textstyleChCharisSIL{ˈgʊ.gʊr̥} & \textsc{v.mo(dy)} & fall (prematurely)\\
& \textitbf{guling} & \textstyleChCharisSIL{ˈgu.lɪŋ} & \textsc{v.mo(dy)} & roll over\\
& \textitbf{guntur} & \textstyleChCharisSIL{ˈgʊn.tʊr̥} & \textsc{n} & thunder\\
& \textitbf{gunung} & \textstyleChCharisSIL{ˈgʊ.nʊŋ} & \textsc{n} & mountain\\
& \textitbf{gurango} & \textstyleChCharisSIL{gʊ.ˈɾa.ŋɔ} & \textsc{n} & shark\\
& \textitbf{gurita} & \textstyleChCharisSIL{gu.ˈɾi.ta} & \textsc{n} & octopus\\
& \textstyleChBold{H} &  &  & \\
& \textitbf{habis} & \textstyleChCharisSIL{ˈha.bɪs} & \textsc{v.mo(st)}

\textsc{cnj} & be used up

after all\\
& \textitbf{hajar} & \textstyleChCharisSIL{ˈha.dʒɐr̥} & \textsc{v.bi} & beat up\\
& \textitbf{halamang} & \textstyleChCharisSIL{ha.ˈla.mɐn} & \textsc{n} & yard\\
& \textitbf{halus} & \textstyleChCharisSIL{ˈha.lʊs} & \textsc{v.mo(st)} & be soft\\
& \textitbf{hamba} & \textstyleChCharisSIL{ˈhɐm.ba} & \textsc{n} & servant\\
& \textitbf{hambat} & \textstyleChCharisSIL{ˈhɐm.bɐt̚} & \textsc{v.bi} & block\\
& \textitbf{hambur} & \textstyleChCharisSIL{ˈhɐm.bʊr̥} & \textsc{v.bi} & scatter\\
& \textitbf{hampir} & \textstyleChCharisSIL{ˈhɐm.pɪr} & \textsc{adv} & almost\\
& \textitbf{hancur} & \textstyleChCharisSIL{ˈhɐn.tʃʊr} & \textsc{v} & be shattered\\
& \textitbf{hangus} & \textstyleChCharisSIL{ˈha.ŋʊs} & \textsc{v.mo(st)} & be singed\\
& \textitbf{hantam} & \textstyleChCharisSIL{ˈhɐn.tɐm} & \textsc{v.bi} & strike\\
& \textitbf{hanya} & \textstyleChCharisSIL{ˈha.ɲa} & \textsc{adv} & only\\
& \textitbf{hapus} & \textstyleChCharisSIL{ˈha.pʊs} & \textsc{v.bi} & completely remove\\
& \textitbf{harap} & \textstyleChCharisSIL{ˈha.ɾɐp} & \textsc{v.bi} & hope\\
& \textitbf{harus} & \textstyleChCharisSIL{ˈha.ɾʊs} & \textsc{v.aux} & have to\\
& \textitbf{hati} & \textstyleChCharisSIL{ˈha.ti} & \textsc{n} & liver\\
& \textitbf{haus} & \textstyleChCharisSIL{ˈha.ʊs} & \textsc{v.mo(st)} & be thirsty\\
& \textitbf{hela} & \textstyleChCharisSIL{ˈhɛ̞.la} & \textsc{v.bi} & haul\\
& \textitbf{hias} & \textstyleChCharisSIL{ˈhɪ.ɐs} & \textsc{v.bi} & decorate\\
& \textitbf{hidup} & \textstyleChCharisSIL{ˈhi.dʊp̚} & \textsc{v.mo(dy)} & live\\
& \textitbf{hijow} & \textstyleChCharisSIL{ˈhi.dʒɔ̞w} & \textsc{v.mo(st)} & be green\\
& \textitbf{hilang} & \textstyleChCharisSIL{ˈhɪ.lɐŋ} & \textsc{v.mo(st)} & be lost\\
& \textitbf{hinggap} & \textstyleChCharisSIL{ˈhɪŋ.gɐp̚} & \textsc{v.mo(dy)} & perch\\
& \textitbf{hitam} & \textstyleChCharisSIL{ˈhi.tɐm} & \textsc{v.mo(st)} & be black\\
& \textitbf{hitung} & \textstyleChCharisSIL{ˈhi.tʊŋ} & \textsc{v.bi} & count\\
& \textitbf{hoki} & \textstyleChCharisSIL{ˈhɔ.ki} & \textsc{n} & plant stem\\
& \textitbf{hosa} & \textstyleChCharisSIL{ˈhɔ̞.sa} & \textsc{v.mo(dy)} & pant\\
& \textitbf{hujang} & \textstyleChCharisSIL{ˈhu.dʒɐn} & \textsc{n} & rain\\
& \textitbf{hutang} & \textstyleChCharisSIL{ˈhʊ.tɐŋ} & \textsc{n} & debt\\
& \textitbf{hutang} & \textstyleChCharisSIL{ˈhu.tɐn} & \textsc{n} & forest\\
& \textstyleChBold{I} &  &  & \\
& \textitbf{ibu} & \textstyleChCharisSIL{ˈi.bu} & \textsc{n} & woman\\
& \textitbf{ikang} & \textstyleChCharisSIL{ˈi.kɐn} & \textsc{n} & fish\\
& \textitbf{ikat} & \textstyleChCharisSIL{ˈi.kɐt̚} & \textsc{v.bi} & tie up\\
& \textitbf{ikut} & \textstyleChCharisSIL{ˈi.kʊt̚} & \textsc{v.bi} & follow\\
& \textitbf{ingat} & \textstyleChCharisSIL{ˈi.ŋɐt̚} & \textsc{v.bi} & remember\\
& \textitbf{inging} & \textstyleChCharisSIL{ˈi.ŋɪn} & \textsc{v.bi} & wish\\
& \textitbf{ingus} & \textstyleChCharisSIL{ˈi.ŋʊs} & \textsc{n} & snot\\
& \textitbf{ini} & \textstyleChCharisSIL{ˈi.ni} & \textsc{dem} & \textsc{d.prox}\\
& \textitbf{injak} & \textstyleChCharisSIL{ˈɪn.dʒa} & \textsc{v.bi} & step on\\
& \textitbf{ipar} & \textstyleChCharisSIL{ˈi.pɐr} & \textsc{n} & sibling in-law\\
& \textitbf{iris} & \textstyleChCharisSIL{ˈɪ.ɾɪs} & \textsc{v.bi} & cut\\
& \textitbf{isap} & \textstyleChCharisSIL{ˈi.sɐp̚} & \textsc{v.bi} & smoke\\
& \textitbf{isi} & \textstyleChCharisSIL{ˈi.si} & \textsc{n/v.bi} & filling / fill\\
& \textitbf{itu} & \textstyleChCharisSIL{ˈi.tu} & \textsc{dem} & \textsc{d.dist}\\
& \textstyleChBold{J} &  &  & \\
& \textitbf{jadi} & \textstyleChCharisSIL{ˈdʒa.di} & \textsc{v.bi}

\textsc{cnj} & become

so, since\\
& \textitbf{jago} & \textstyleChCharisSIL{ˈdʒa.gɔ} & \textsc{n} & candidate\\
& \textitbf{jagung} & \textstyleChCharisSIL{ˈdʒa.gʊn} & \textsc{n} & corn\\
& \textitbf{jahat} & \textstyleChCharisSIL{ˈdʒa.hɐt̚} & \textsc{v.mo(st)} & be bad\\
& \textitbf{jahe} & \textstyleChCharisSIL{ˈdʒa.hɛ} & \textsc{n} & ginger\\
& \textitbf{jahit} & \textstyleChCharisSIL{ˈdʒa.hɪt̚} & \textsc{v.bi} & sew\\
& \textitbf{jaja} & \textstyleChCharisSIL{ˈdʒa.dʒa} & \textsc{v.bi} & colonize\\
& \textitbf{jajang} & \textstyleChCharisSIL{ˈdʒa.dʒɐn} & \textsc{n} & snack\\
& \textitbf{jalang} & \textstyleChCharisSIL{ˈdʒa.lɐn} & \textsc{n/v.mo(dy)} & street / walk\\
& \textitbf{jalur} & \textstyleChCharisSIL{ˈdʒa.lʊr} & \textsc{n} & traffic lane\\
& \textitbf{jangang} & \textstyleChCharisSIL{ˈdʒa.ŋɐn} & \textsc{adv} & \textsc{neg.imp}\\
& \textitbf{jangkrik} & \textstyleChCharisSIL{ˈdʒɐŋ.krɪk̚} & \textsc{n} & cricket\\
& \textitbf{janji} & \textstyleChCharisSIL{ˈdʒɐn.dʒi} & \textsc{v.bi} & promise\\
& \textitbf{jantung} & \textstyleChCharisSIL{ˈdʒɐn.tʊŋ} & \textsc{n} & heart\\
& \textitbf{jarak} & \textstyleChCharisSIL{ˈdʒa.ɾɐk} & \textsc{n} & distance between\\
& \textitbf{jarang} & \textstyleChCharisSIL{ˈdʒa.ɾɐŋ} & \textsc{adv} & rarely\\
& \textitbf{jari} & \textstyleChCharisSIL{ˈdʒa.ɾi} & \textsc{n} & digit\\
& \textitbf{jaring} & \textstyleChCharisSIL{ˈdʒa.ɾɪŋ} & \textsc{n} & net\\
& \textitbf{jata} & \textstyleChCharisSIL{ˈdʒa.ta} & \textsc{n} & allotment\\
& \textitbf{jatu} & \textstyleChCharisSIL{ˈdʒa.tu} & \textsc{v.mo(dy)} & fall\\
& \textitbf{jaw} & \textstyleChCharisSIL{ˈdʒɐw} & \textsc{v.mo(st)} & be far\\
& \textitbf{jaya} & \textstyleChCharisSIL{ˈdʒa.ja} & \textsc{v.mo(st)} & be glorious\\
\textstyleExampleSource{x} & \textitbf{jelas} & \textstyleChCharisSIL{dʒɛ.ˈlɐs} & \textsc{v.mo(st)} & be clear\\
& \textitbf{jelek} & \textstyleChCharisSIL{dʒɛ.ˈlɛ̞k̚} & \textsc{v.mo(st)} & be bad\\
& \textitbf{jembatang} & \textstyleChCharisSIL{dʒɛ̞m.ˈba.tɐn} & \textsc{n} & bridge\\
\textstyleExampleSource{x} & \textitbf{jempol} & \textstyleChCharisSIL{dʒɛ̞m.ˈpɔ̞l} & \textsc{n} & thumb\\
\textstyleExampleSource{x} & \textitbf{jemput} & \textstyleChCharisSIL{dʒɛ̞m.ˈpʊt} & \textsc{v.bi} & pick up\\
\textstyleExampleSource{x} & \textitbf{jemur} & \textstyleChCharisSIL{dʒɛ.ˈmʊr} & \textsc{v.bi} & dry in sun\\
& \textitbf{jenggot} & \textstyleChCharisSIL{ˈdʒɛ̞ŋ.gɔ̞t̚} & \textsc{n} & beard\\
& \textitbf{jengkel} & \textstyleChCharisSIL{ˈdʒɛ̞ŋ.kɛ̞l} & \textsc{v.bi} & annoy\\
\textstyleExampleSource{x} & \textitbf{jerat} & \textstyleChCharisSIL{dʒɛ.ˈɾɐt̚} & \textsc{n} & trap\\
& \textitbf{jerawat} & \textstyleChCharisSIL{dʒɛ.ˈɾa.wɐt̚} & \textsc{n} & acne\\
& \textitbf{jeruk} & \textstyleChCharisSIL{ˈdʒɛ.ɾʊk̚} & \textsc{n} & citrus fruit\\
& \textitbf{jintang} & \textstyleChCharisSIL{ˈdʒɪn.tɐŋ} & \textsc{n} & caraway seed\\
& \textitbf{jual} & \textstyleChCharisSIL{ˈdʒu.ɐl} & \textsc{v.bi} & sell\\
& \textitbf{juara} & \textstyleChCharisSIL{dʒu.ˈa.ɾa} & \textsc{n} & champion\\
& \textitbf{jubi} & \textstyleChCharisSIL{ˈdʒu.bi} & \textsc{n/v.bi} & bow and arrow / bow shoot\\
& \textitbf{juga} & \textstyleChCharisSIL{ˈdʒu.ga} & \textsc{adv} & also\\
& \textitbf{jujur} & \textstyleChCharisSIL{ˈdʒu.dʒʊr} & \textsc{v.mo(st)} & be honest\\
& \textitbf{jungkir} & \textstyleChCharisSIL{ˈdʒʊŋ.kɪr} & \textsc{v.bi} & flip over\\
& \textitbf{jurang} & \textstyleChCharisSIL{ˈdʒu.ɾɐn} & \textsc{n} & steep decline\\
& \textitbf{jurus} & \textstyleChCharisSIL{ˈdʒu.ɾʊs} & \textsc{n} & steps\\
& \textstyleChBold{K} &  &  & \\
& \textitbf{ka} & \textstyleChCharisSIL{ˈka} & \textsc{cnj} & or\\
& \textitbf{kabur} & \textstyleChCharisSIL{ˈka.bʊr} & \textsc{v.mo(st)} & be hazy\\
& \textitbf{kabut} & \textstyleChCharisSIL{ˈka.bʊt̚} & \textsc{n} & fog\\
& \textitbf{kacang} & \textstyleChCharisSIL{ˈka.tʃɐŋ} & \textsc{n} & bean\\
& \textitbf{kaco} & \textstyleChCharisSIL{ˈka.tʃɔ} & \textsc{v.mo(st)} & be confused\\
& \textitbf{kadang} & \textstyleChCharisSIL{ˈka.dɐŋ} & \textsc{adv} & sometimes\\
& \textitbf{kaget} & \textstyleChCharisSIL{ˈka.gɛ̞t̚} & \textsc{v.bi} & feel startled (by)\\
& \textitbf{kaing} & \textstyleChCharisSIL{ˈka.ɪn} & \textsc{n} & cloth\\
& \textitbf{kaka} & \textstyleChCharisSIL{ˈka.ka} & \textsc{n} & older sibling\\
& \textitbf{kaki} & \textstyleChCharisSIL{ˈka.ki} & \textsc{n} & foot\\
& \textitbf{kala} & \textstyleChCharisSIL{ˈka.la} & \textsc{v.mo(st)} & be defeated\\
& \textitbf{kalangang} & \textstyleChCharisSIL{ka.ˈla.ŋɐn} & \textsc{n} & circle\\
& \textitbf{kaleng} & \textstyleChCharisSIL{ˈka.lɛ̞ŋ} & \textsc{n} & tin can\\
& \textitbf{kali} & \textstyleChCharisSIL{ˈka.li} & \textsc{n} & river\\
& \textitbf{kalong} & \textstyleChCharisSIL{ˈka.lɔ̞ŋ} & \textsc{n} & necklace\\
& \textitbf{kam} & \textstyleChCharisSIL{ˈkɐm} & \textsc{pro} & \textsc{2pl}\\
& \textitbf{kamorang} & \textstyleChCharisSIL{ka.ˈmɔ.ɾɐŋ} & \textsc{pro} & \textsc{2pl}\\
& \textitbf{kampung} & \textstyleChCharisSIL{ˈkɐm.pʊŋ} & \textsc{n} & village\\
& \textitbf{kamu} & \textstyleChCharisSIL{ˈka.mu} & \textsc{pro} & \textsc{2pl}\\
& \textitbf{kanang} & \textstyleChCharisSIL{ˈka.nɐn} & \textsc{adv} & right\\
& \textitbf{kancing} & \textstyleChCharisSIL{ˈkɐn.tʃɪŋ} & \textsc{v.bi} & lock\\
& \textitbf{kandam} & \textstyleChCharisSIL{ˈkɐn.dɐm} & \textsc{n} & stable\\
& \textitbf{kandung} & \textstyleChCharisSIL{ˈkɐn.dʊŋ} & \textsc{n} & womb\\
& \textitbf{kangkung} & \textstyleChCharisSIL{ˈkɐŋ.kʊŋ} & \textsc{n} & water spinach\\
& \textitbf{kapak} & \textstyleChCharisSIL{ˈka.pɐk} & \textsc{n} & axe\\
& \textitbf{kapang} & \textstyleChCharisSIL{ˈka.pɐn} & \textsc{int} & when\\
& \textitbf{kapur} & \textstyleChCharisSIL{ˈka.pʊr̥} & \textsc{n} & lime\\
& \textitbf{karang} & \textstyleChCharisSIL{ˈka.ɾɐŋ} & \textsc{n} & lime stone\\
& \textitbf{karet} & \textstyleChCharisSIL{ˈka.ɾɛ̞t} & \textsc{n} & rubber\\
& \textitbf{karong} & \textstyleChCharisSIL{ˈka.ɾɔ̞ŋ} & \textsc{n} & bag\\
& \textitbf{kasar} & \textstyleChCharisSIL{ˈka.sɐr̥} & \textsc{v.mo(st)} & be coarse\\
& \textitbf{kasbi} & \textstyleChCharisSIL{ˈkɐs.bi} & \textsc{n} & cassava\\
& \textitbf{kasi} & \textstyleChCharisSIL{ˈkɐs} & \textsc{v.tri} & give\\
& \textitbf{kasi} & \textstyleChCharisSIL{ˈka.si} & \textsc{n} & love\\
& \textitbf{kaswari} & \textstyleChCharisSIL{ka.ˈswa.ɾi} & \textsc{n} & cassowary\\
& \textitbf{kawang} & \textstyleChCharisSIL{ˈka.wɐn} & \textsc{n} & friend\\
& \textitbf{kaya} & \textstyleChCharisSIL{ˈka.ja} & \textsc{prep/cnj} & like\\
& \textitbf{kayu} & \textstyleChCharisSIL{ˈka.ju} & \textsc{n} & wood\\
& \textitbf{ke} & \textstyleChCharisSIL{ˈkɛ} & \textsc{prep} & to\\
\textstyleExampleSource{x} & \textitbf{kebung} & \textstyleChCharisSIL{kɛ.ˈbʊn} & \textsc{n} & garden\\
\textstyleExampleSource{x} & \textitbf{kecil} & \textstyleChCharisSIL{kɛ.ˈtʃɪl} & \textsc{v.mo(st)} & be small\\
& \textitbf{kecuali} & \textstyleChCharisSIL{ˌkɛ.tʃʊ.ˈa.li} & \textsc{adv} & except\\
\textstyleExampleSource{x} & \textitbf{kejar} & \textstyleChCharisSIL{kɛ.ˈdʒɐr} & \textsc{v.bi} & chase\\
& \textitbf{kemaring} & \textstyleChCharisSIL{kɛ.ˈma.ɾɪn} & \textsc{n} & yesterday\\
& \textitbf{kembali} & \textstyleChCharisSIL{kɛ̞m.ˈba.li} & \textsc{v.mo(dy)} & return\\
\textstyleExampleSource{x} & \textitbf{kembang} & \textstyleChCharisSIL{kɛ̞m.ˈbɐŋ} & \textsc{n} & flower\\
\textstyleExampleSource{x} & \textitbf{kembar} & \textstyleChCharisSIL{kɛ̞m.ˈbɐr̥} & \textsc{n} & twin\\
& \textitbf{kempes} & \textstyleChCharisSIL{ˈkɛ̞m.pɛ̞s} & \textsc{v.mo(st)} & be deflated\\
\textstyleExampleSource{x} & \textitbf{kena} & \textstyleChCharisSIL{kɛ.ˈna} & \textsc{v.bi} & hit\\
\textstyleExampleSource{x} & \textitbf{kenal} & \textstyleChCharisSIL{kɛ.ˈnɐl} & \textsc{v.bi} & know\\
\textstyleExampleSource{x} & \textitbf{kencang} & \textstyleChCharisSIL{kɛ̞n.ˈtʃɐŋ} & \textsc{v.mo(st)} & be speedy\\
& \textitbf{kendali} & \textstyleChCharisSIL{kɛ̞n.ˈda.li} & \textsc{n} & reins\\
& \textitbf{kendara} & \textstyleChCharisSIL{kɛ̞n.ˈda.ɾa} & \textsc{v.bi} & ride\\
\textstyleExampleSource{x} & \textitbf{kental} & \textstyleChCharisSIL{kɛ̞n.ˈtɐl} & \textsc{v.mo(st)} & be fluent\\
& \textitbf{kentara} & \textstyleChCharisSIL{kɛ̞n.ˈta.ɾa} & \textsc{v.mo(st)} & be visible\\
\textstyleExampleSource{x} & \textitbf{kenyang} & \textstyleChCharisSIL{kɛ.ˈɲɐŋ} & \textsc{v.mo(st)} & be satisfied\\
& \textitbf{kepiting} & \textstyleChCharisSIL{kɛ.ˈpi.tɪŋ} & \textsc{n} & crab\\
\textstyleExampleSource{x} & \textitbf{kepung} & \textstyleChCharisSIL{kɛ.ˈpʊn} & \textsc{v.bi} & surround\\
\textstyleExampleSource{x} & \textitbf{kera} & \textstyleChCharisSIL{kɛ.ˈɾa} & \textsc{n} & ape\\
\textstyleExampleSource{x} & \textitbf{ketang} & \textstyleChCharisSIL{kɛ.ˈtɐŋ} & \textsc{n} & crab\\
& \textitbf{ketapang} & \textstyleChCharisSIL{kɛ.ˈta.pɐŋ} & \textsc{n} & tropical-almond\\
\textstyleExampleSource{x} & \textitbf{ketuk} & \textstyleChCharisSIL{kɛ.ˈtʊk̚} & \textsc{v.bi} & knock\\
& \textitbf{kewa} & \textstyleChCharisSIL{ˈkɛ.wa} & \textsc{n} & dance party\\
& \textitbf{kicaw} & \textstyleChCharisSIL{ˈki.tʃɐw} & \textsc{v.mo(st)} & be naughty\\
& \textitbf{kikir} & \textstyleChCharisSIL{ˈki.kɪr̥} & \textsc{v.mo(st)} & be stingy\\
& \textitbf{kikis} & \textstyleChCharisSIL{ˈki.kɪs} & \textsc{v.bi} & scrape\\
& \textitbf{kilat} & \textstyleChCharisSIL{ˈki.lɐt̚} & \textsc{n} & lightning\\
& \textitbf{kincing} & \textstyleChCharisSIL{ˈkɪn.tʃɪŋ} & \textsc{v.bi} & pee\\
& \textitbf{kintal} & \textstyleChCharisSIL{ˈkɪn.tɐl} & \textsc{n} & yard\\
& \textitbf{kipas} & \textstyleChCharisSIL{ˈki.pɐs} & \textsc{v.bi} & beat\\
& \textitbf{kira} & \textstyleChCharisSIL{ˈki.ɾa} & \textsc{v.mo(dy)} & think\\
& \textitbf{kiri} & \textstyleChCharisSIL{ˈki.ɾi} & \textsc{adv} & left\\
& \textitbf{kirim} & \textstyleChCharisSIL{ˈki.ɾɪm} & \textsc{v.tri} & send\\
& \textitbf{kita} & \textstyleChCharisSIL{ˈki.ta} & \textsc{pro} & \textsc{1pl}\\
\textstyleExampleSource{x} & \textitbf{kitong} & \textstyleChCharisSIL{ki.ˈtɔ̞ŋ} & \textsc{pro} & \textsc{1pl}\\
& \textitbf{kitorang} & \textstyleChCharisSIL{ki.ˈtɔ.ɾɐŋ} & \textsc{pro} & \textsc{1pl}\\
& \textitbf{kladi} & \textstyleChCharisSIL{ˈkla.di} & \textsc{n} & taro root\\
& \textitbf{klambu} & \textstyleChCharisSIL{ˈklɐm.bu} & \textsc{n} & mosquito net\\
& \textitbf{klapa} & \textstyleChCharisSIL{ˈkla.pa} & \textsc{n} & coconut\\
& \textitbf{klawar} & \textstyleChCharisSIL{ˈkla.wɐr̥} & \textsc{n} & cave bat\\
& \textitbf{klereng} & \textstyleChCharisSIL{ˈklɛ̞.ɾɛ̞ŋ} & \textsc{n} & marbles\\
& \textitbf{kliling} & \textstyleChCharisSIL{ˈklɪ.lɪŋ} & \textsc{v.bi} & travel around\\
& \textitbf{klompok} & \textstyleChCharisSIL{ˈklɔ̞m.pɔ̞k} & \textsc{n} & group\\
& \textitbf{kluar} & \textstyleChCharisSIL{ˈklʊ.ɐr} & \textsc{v.mo(dy)} & go out\\
& \textitbf{knapa} & \textstyleChCharisSIL{ˈkna.pa} & \textsc{int} & why\\
& \textitbf{ko} & \textstyleChCharisSIL{ˈkɔ} & \textsc{pro} & \textsc{2sg}\\
& \textitbf{koco} & \textstyleChCharisSIL{ˈkɔ.tʃɔ} & \textsc{v.bi} & tell off\\
& \textitbf{kodok} & \textstyleChCharisSIL{ˈkɔ̞.dɔ̞k} & \textsc{n} & frog\\
& \textitbf{kolam} & \textstyleChCharisSIL{ˈkɔ̞.lɐm} & \textsc{n} & big hole\\
& \textitbf{kolong} & \textstyleChCharisSIL{ˈkɔ̞.lɔ̞ŋ} & \textsc{n} & space below\\
& \textitbf{korek} & \textstyleChCharisSIL{ˈkɔ.ɾɛ̞k} & \textsc{n/v.bi} & matches / scrape\\
& \textitbf{kos} & \textstyleChCharisSIL{ˈkɔ̞s} & \textsc{n} & boarding house\\
& \textitbf{kosong} & \textstyleChCharisSIL{ˈkɔ̞.sɔ̞ŋ} & \textsc{v.mo(st)} & be empty\\
& \textitbf{kota} & \textstyleChCharisSIL{ˈkɔ.ta} & \textsc{n} & city\\
& \textitbf{kotor} & \textstyleChCharisSIL{ˈkɔ̞.tɔ̞r̥} & \textsc{v.mo(st)} & be dirty\\
& \textitbf{kras} & \textstyleChCharisSIL{ˈkrɐs} & \textsc{v.mo(st)} & be harsh\\
& \textitbf{kring} & \textstyleChCharisSIL{ˈkrɪŋ} & \textsc{v.mo(st)} & be dry\\
& \textitbf{kringat} & \textstyleChCharisSIL{ˈkri.ŋɐt̚} & \textsc{v.mo(dy)} & sweat\\
& \textitbf{kriting} & \textstyleChCharisSIL{ˈkrɪ.tɪŋ} & \textsc{v.mo(st)} & be curly\\
& \textitbf{kuca} & \textstyleChCharisSIL{ˈku.tʃa} & \textsc{v.bi} & rub with hands\\
& \textitbf{kucing} & \textstyleChCharisSIL{ˈku.tʃɪŋ} & \textsc{n} & cat\\
& \textitbf{kuku} & \textstyleChCharisSIL{ˈku.ku} & \textsc{n} & digit nail\\
& \textitbf{kukus} & \textstyleChCharisSIL{ˈku.kʊs} & \textsc{v.bi} & steam\\
& \textitbf{kulit} & \textstyleChCharisSIL{ˈku.lɪt} & \textsc{n} & skin\\
& \textitbf{kumis} & \textstyleChCharisSIL{ˈku.mɪs} & \textsc{n} & mustache\\
& \textitbf{kumpul} & \textstyleChCharisSIL{ˈkʊm.pʊl} & \textsc{v.bi} & gather\\
\textstyleExampleSource{x} & \textitbf{kumur} & \textstyleChCharisSIL{kʊ.ˈmʊr} & \textsc{v.bi} & rinse mouth\\
& \textitbf{kunya} & \textstyleChCharisSIL{ˈkʊ.ɲa} & \textsc{v.bi} & chew\\
& \textitbf{kupas} & \textstyleChCharisSIL{ˈku.pɐs} & \textsc{v.bi} & peel\\
& \textitbf{kurang} & \textstyleChCharisSIL{ˈku.ɾɐŋ} & \textsc{v.bi} & lack\\
& \textitbf{kurung} & \textstyleChCharisSIL{ˈkʊ.ɾʊŋ} & \textsc{v.bi} & imprison\\
& \textitbf{kurus} & \textstyleChCharisSIL{ˈkʊ.ɾʊs} & \textsc{v.mo(st)} & be thin\\
\textstyleExampleSource{x} & \textitbf{kuskus} & \textstyleChCharisSIL{kʊs.ˈkʊs} & \textsc{n} & cuscus\\
& \textitbf{kutik} & \textstyleChCharisSIL{ˈku.tɪk̚} & \textsc{v.bi} & snap\\
& \textitbf{kutu} & \textstyleChCharisSIL{ˈku.tu} & \textsc{n} & louse\\
& \textitbf{kutuk} & \textstyleChCharisSIL{ˈku.tʊk} & \textsc{n/v.bi} & curse\\
& \textitbf{kwali} & \textstyleChCharisSIL{ˈkwa.li} & \textsc{n} & frying pan\\
& \textstyleChBold{L} &  &  & \\
& \textitbf{lada} & \textstyleChCharisSIL{ˈla.da} & \textsc{n} & pepper\\
& \textitbf{ladang} & \textstyleChCharisSIL{ˈla.dɐŋ} & \textsc{n} & field\\
& \textitbf{lagi} & \textstyleChCharisSIL{ˈla.gi} & \textsc{adv} & also, again\\
& \textitbf{lagu} & \textstyleChCharisSIL{ˈla.gu} & \textsc{n} & song\\
& \textitbf{laing} & \textstyleChCharisSIL{ˈla.ɪn} & \textsc{v.mo(st)} & be different\\
& \textitbf{laju} & \textstyleChCharisSIL{ˈla.dʒu} & \textsc{v.mo(st)} & be quick\\
& \textitbf{laki} & \textstyleChCharisSIL{ˈla.ki} & \textsc{n} & husband\\
& \textitbf{laki{\Tilde}laki} & \textstyleChCharisSIL{ˌla.kiˈla.ki} & \textsc{n} & man\\
& \textitbf{lalapang} & \textstyleChCharisSIL{la.ˈla.pɐn} & \textsc{n} & k. o. vegetable dish\\
& \textitbf{lalat} & \textstyleChCharisSIL{ˈla.lɐt} & \textsc{n} & fly\\
& \textitbf{laley} & \textstyleChCharisSIL{ˈla.lɛ̞j} & \textsc{v.mo(st)} & be careless\\
& \textitbf{lama} & \textstyleChCharisSIL{ˈla.ma} & \textsc{v.mo(st)} & be long (of duration)\\
& \textitbf{lamar} & \textstyleChCharisSIL{ˈla.mɐr} & \textsc{v.bi} & apply for\\
& \textitbf{lambat} & \textstyleChCharisSIL{ˈlɐm.bɐt} & \textsc{v.mo(st)} & be slow\\
& \textitbf{lancar} & \textstyleChCharisSIL{ˈlɐn.tʃɐr̥} & \textsc{v.mo(st)} & be fluent\\
& \textitbf{langar} & \textstyleChCharisSIL{ˈla.ŋɐr} & \textsc{v.bi} & collide with\\
& \textitbf{langit} & \textstyleChCharisSIL{ˈla.ŋɪt̚} & \textsc{n} & sky\\
& \textitbf{langka} & \textstyleChCharisSIL{ˈlɐŋ.ka} & \textsc{n} & step\\
& \textitbf{langsung} & \textstyleChCharisSIL{ˈlɐŋ.sʊŋ} & \textsc{adv} & directly\\
& \textitbf{lanjut} & \textstyleChCharisSIL{ˈlɐn.dʒʊt̚} & \textsc{v.bi} & continue\\
& \textitbf{lante} & \textstyleChCharisSIL{ˈlɐn.tɛ} & \textsc{n} & floor\\
& \textitbf{lantik} & \textstyleChCharisSIL{ˈlɐn.tɪk̚} & \textsc{v.bi} & inaugurate someone\\
& \textitbf{lapang} & \textstyleChCharisSIL{ˈla.pɐŋ} & \textsc{v.mo(st)} & be spacious\\
& \textitbf{lapar} & \textstyleChCharisSIL{ˈla.pɐr̥} & \textsc{v.mo(st)} & be hungry\\
& \textitbf{lapis} & \textstyleChCharisSIL{ˈla.pɪs} & \textsc{qt} & all\\
& \textitbf{lapuk} & \textstyleChCharisSIL{ˈla.pʊk} & \textsc{v.mo(dy)} & decompose\\
& \textitbf{larang} & \textstyleChCharisSIL{ˈla.ɾɐŋ} & \textsc{v.bi} & forbid\\
& \textitbf{lari} & \textstyleChCharisSIL{ˈla.ɾi} & \textsc{v.mo(dy)} & run\\
& \textitbf{larut} & \textstyleChCharisSIL{ˈla.ɾʊt} & \textsc{v.mo(st)} & be protracted\\
& \textitbf{lati} & \textstyleChCharisSIL{ˈla.ti} & \textsc{v.bi} & practice\\
& \textitbf{lauk} & \textstyleChCharisSIL{ˈla.ʊk} & \textsc{n} & side dish\\
& \textitbf{laut} & \textstyleChCharisSIL{ˈla.ʊt̚} & \textsc{n} & sea\\
& \textitbf{lawang} & \textstyleChCharisSIL{ˈla.wɐn} & \textsc{v.bi} & oppose\\
& \textitbf{layak} & \textstyleChCharisSIL{ˈla.jɐk} & \textsc{v.mo(st)} & be suitable\\
& \textitbf{layang} & \textstyleChCharisSIL{ˈla.jɐn} & \textsc{v.bi} & serve\\
& \textitbf{layar} & \textstyleChCharisSIL{ˈla.jɐr} & \textsc{v.mo(dy)} & sail\\
& \textitbf{lebar} & \textstyleChCharisSIL{ˈlɛ.bɐr̥} & \textsc{v.mo(st)} & be wide\\
& \textitbf{lebarang} & \textstyleChCharisSIL{lɛ.ˈba.ɾɐn} & \textsc{n} & end of fasting month\\
\textstyleExampleSource{x} & \textitbf{lebi} & \textstyleChCharisSIL{lɛ.ˈbi} & \textsc{adv} & more\\
\textstyleExampleSource{x} & \textitbf{lega} & \textstyleChCharisSIL{lɛ.ˈga} & \textsc{v.mo(st)} & be relieved\\
& \textitbf{leher} & \textstyleChCharisSIL{ˈlɛ.hɛ̞r̥} & \textsc{n} & neck\\
\textstyleExampleSource{x} & \textitbf{lema} & \textstyleChCharisSIL{lɛ.ˈma} & \textsc{v.mo(st)} & be weak\\
\textstyleExampleSource{x} & \textitbf{lemba} & \textstyleChCharisSIL{lɛ̞m.ˈba} & \textsc{n} & valley\\
& \textitbf{lembaga} & \textstyleChCharisSIL{lɛ̞m.ˈba.ga} & \textsc{n} & institute\\
\textstyleExampleSource{x} & \textitbf{lembar} & \textstyleChCharisSIL{lɛ̞m.ˈbɐr} & \textsc{n} & sheet\\
& \textitbf{lembek} & \textstyleChCharisSIL{ˈlɛ̞m.bɛ̞k} & \textsc{v.mo(st)} & be soft\\
& \textitbf{lempar} & \textstyleChCharisSIL{ˈlɛ̞m.pɐr} & \textsc{v.bi} & throw\\
\textstyleExampleSource{x} & \textitbf{lengkap} & \textstyleChCharisSIL{lɛ̞ŋ.ˈkɐp} & \textsc{v.mo(st)} & be complete\\
\textstyleExampleSource{x} & \textitbf{lepas} & \textstyleChCharisSIL{lə.ˈpɐs} & \textsc{v.bi} & free\\
& \textitbf{lewat} & \textstyleChCharisSIL{ˈlɛ.wɐt̚} & \textsc{v.bi} & pass by\\
& \textitbf{liar} & \textstyleChCharisSIL{ˈlɪ.ɐr̥} & \textsc{v.mo(st)} & be wild\\
& \textitbf{liat} & \textstyleChCharisSIL{ˈlɪ.ɐt̚} & \textsc{v.bi} & see\\
& \textitbf{libur} & \textstyleChCharisSIL{ˈli.bʊr̥} & \textsc{v.mo(dy)} & take vacation\\
& \textitbf{licing} & \textstyleChCharisSIL{ˈli.tʃɪn} & \textsc{v.mo(st)} & be straight\\
& \textitbf{lida} & \textstyleChCharisSIL{ˈli.da} & \textsc{n} & tongue\\
& \textitbf{liling} & \textstyleChCharisSIL{ˈlɪ.lɪn} & \textsc{n} & candle\\
& \textitbf{lima} & \textstyleChCharisSIL{ˈli.ma} & \textsc{num.c} & five\\
& \textitbf{limpa} & \textstyleChCharisSIL{ˈlɪm.pa} & \textsc{v.mo(st)} & be abundant\\
& \textitbf{lingkar} & \textstyleChCharisSIL{ˈlɪŋ.kɐr̥} & \textsc{v.bi} & circle\\
& \textitbf{lipat} & \textstyleChCharisSIL{ˈli.pɐt̚} & \textsc{v.bi} & fold\\
& \textitbf{lobe} & \textstyleChCharisSIL{ˈlɔ.bɛ} & \textsc{v.bi} & night hunt\\
& \textitbf{lomba} & \textstyleChCharisSIL{ˈlɔ̞m.ba} & \textsc{n} & contest\\
& \textitbf{lompat} & \textstyleChCharisSIL{ˈlɔ̞m.pa} & \textsc{v.bi} & jump\\
& \textitbf{loncat} & \textstyleChCharisSIL{ˈlɔ̞n.tʃɐt̚} & \textsc{v.bi} & jump\\
& \textitbf{longgar} & \textstyleChCharisSIL{ˈlɔ̞ŋ.gɐr̥} & \textsc{v.mo(st)} & be thin\\
& \textitbf{loyang} & \textstyleChCharisSIL{ˈlɔ.jɐŋ} & \textsc{n} & large bowl\\
& \textitbf{loyo} & \textstyleChCharisSIL{ˈlɔ.jɔ} & \textsc{v.mo(st)} & be weak\\
& \textitbf{luar} & \textstyleChCharisSIL{ˈlʊ.ɐr̥} & \textsc{n-loc} & outside\\
& \textitbf{luas} & \textstyleChCharisSIL{ˈlʊ.ɐs} & \textsc{v.mo(st)} & be vast\\
& \textitbf{lubang} & \textstyleChCharisSIL{ˈlʊ.bɐŋ} & \textsc{n} & hole\\
& \textitbf{lucu} & \textstyleChCharisSIL{ˈlu.tʃu} & \textsc{v.mo(st)} & be funny\\
& \textitbf{luda} & \textstyleChCharisSIL{ˈlu.da} & \textsc{v.bi} & spit\\
& \textitbf{luka} & \textstyleChCharisSIL{ˈlʊ.ka} & \textsc{n} & wound\\
& \textitbf{lulus} & \textstyleChCharisSIL{ˈlu.lʊs} & \textsc{v.bi} & pass (a test)\\
& \textitbf{lumayang} & \textstyleChCharisSIL{lu.ˈma.jɐn} & \textsc{v.mo(st)} & be moderate\\
& \textitbf{lunas} & \textstyleChCharisSIL{ˈlu.nɐs} & \textsc{v.mo(st)} & be paid\\
& \textitbf{luncur} & \textstyleChCharisSIL{ˈlʊn.tʃʊr} & \textsc{v.bi} & slide\\
& \textitbf{lupa} & \textstyleChCharisSIL{ˈlu.pa} & \textsc{v.bi} & forget\\
& \textitbf{lur} & \textstyleChCharisSIL{ˈlʊr} & \textsc{v.bi} & spy on\\
& \textitbf{luru} & \textstyleChCharisSIL{ˈlu.ɾu} & \textsc{v.bi} & chase after\\
& \textitbf{lurus} & \textstyleChCharisSIL{ˈlʊ.ɾʊs} & \textsc{v.mo(st)} & be straight\\
& \textitbf{lusa} & \textstyleChCharisSIL{ˈlu.sa} & \textsc{n} & day after tomorrow\\
& \textitbf{lutut} & \textstyleChCharisSIL{ˈlu.tʊt} & \textsc{n} & knee\\
& \textstyleChBold{M} &  &  & \\
& \textitbf{mabuk} & \textstyleChCharisSIL{ˈma.bʊk̚} & \textsc{v.mo(st)} & be drunk\\
& \textitbf{macang} & \textstyleChCharisSIL{ˈma.tʃɐm} & \textsc{n} & variety\\
& \textitbf{mace} & \textstyleChCharisSIL{ˈma.tʃɛ} & \textsc{n} & woman\\
& \textitbf{macet} & \textstyleChCharisSIL{ˈma.tʃɛ̞t} & \textsc{v.mo(st)} & be stuck\\
& \textitbf{mahal} & \textstyleChCharisSIL{ˈma.hɐl} & \textsc{v.mo(st)} & be expensive\\
& \textitbf{maing} & \textstyleChCharisSIL{ˈma.ɪn} & \textsc{v.bi} & play\\
& \textitbf{maju} & \textstyleChCharisSIL{ˈma.dʒu} & \textsc{v.mo(dy)} & advance\\
& \textitbf{makang} & \textstyleChCharisSIL{ˈma.kɐn} & \textsc{v.bi} & eat\\
& \textitbf{maki} & \textstyleChCharisSIL{ˈma.ki} & \textsc{v.bi} & abuse verbally\\
& \textitbf{making} & \textstyleChCharisSIL{ˈma.kɪn} & \textsc{adv} & increasingly\\
& \textitbf{mala} & \textstyleChCharisSIL{ˈma.la} & \textsc{adv} & even\\
& \textitbf{malam} & \textstyleChCharisSIL{ˈma.lɐm} & \textsc{n} & night\\
& \textitbf{malas} & \textstyleChCharisSIL{ˈma.lɐs} & \textsc{v.mo(st)} & be listless\\
& \textitbf{maling} & \textstyleChCharisSIL{ˈma.lɪŋ} & \textsc{n} & thief\\
& \textitbf{malu} & \textstyleChCharisSIL{ˈma.lu} & \textsc{v.bi} & feel embarrassed (about)\\
& \textitbf{mampu} & \textstyleChCharisSIL{ˈmɐm.pu} & \textsc{v.mo(st)} & be capable\\
& \textitbf{mana} & \textstyleChCharisSIL{ˈma.na} & \textsc{int} & where\\
& \textitbf{mancing} & \textstyleChCharisSIL{ˈmɐn.tʃɪŋ} & \textsc{v.bi} & fish with rod\\
& \textitbf{mandi} & \textstyleChCharisSIL{ˈmɐn.di} & \textsc{v.mo(dy)} & bathe\\
& \textitbf{mandiri} & \textstyleChCharisSIL{mɐn.ˈdi.ɾi} & \textsc{v.mo(dy)} & stand alone\\
& \textitbf{mandul} & \textstyleChCharisSIL{ˈmɐn.dʊl} & \textsc{v.mo(st)} & be sterile\\
& \textitbf{mangkok} & \textstyleChCharisSIL{ˈmɐŋ.kɔ̞k} & \textsc{n} & cup\\
& \textitbf{manis} & \textstyleChCharisSIL{ˈma.nɪs} & \textsc{v.mo(st)} & be sweet\\
& \textitbf{manja} & \textstyleChCharisSIL{ˈmɐn.dʒa} & \textsc{v.bi} & spoil\\
& \textitbf{mantang} & \textstyleChCharisSIL{ˈmɐn.tɐn} & \textsc{v.mo(st)} & be former\\
& \textitbf{mantap} & \textstyleChCharisSIL{ˈmɐn.tɐp̚} & \textsc{v.mo(st)} & be good\\
& \textitbf{mantu} & \textstyleChCharisSIL{ˈmɐn.tu} & \textsc{n} & in-law\\
& \textitbf{mara} & \textstyleChCharisSIL{ˈma.ɾa} & \textsc{v.bi} & feel angry (about)\\
& \textitbf{mari} & \textstyleChCharisSIL{ˈma.ɾi} & \textsc{loc} & hither\\
& \textitbf{masa} & \textstyleChCharisSIL{ˈma.sa} & \textsc{v.mo(st)} & impossible\\
& \textitbf{masak} & \textstyleChCharisSIL{ˈma.sɐk} & \textsc{v.bi} & cook\\
& \textitbf{masi} & \textstyleChCharisSIL{ˈma.si} & \textsc{adv} & still\\
& \textitbf{masuk} & \textstyleChCharisSIL{ˈma.sʊk̚} & \textsc{v.bi} & enter\\
& \textitbf{mata} & \textstyleChCharisSIL{ˈma.ta} & \textsc{n} & eye\\
& \textitbf{mati} & \textstyleChCharisSIL{ˈma.tɪ} & \textsc{v.mo(dy)} & die\\
& \textitbf{maw} & \textstyleChCharisSIL{ˈmɐw} & \textsc{v.bi} & want\\
& \textitbf{mayana} & \textstyleChCharisSIL{ma.ˈja.na} & \textsc{n} & painted nettle\\
& \textitbf{mayang} & \textstyleChCharisSIL{ˈma.jɐŋ} & \textsc{n} & palm blossom\\
\textstyleExampleSource{x} & \textitbf{mekar} & \textstyleChCharisSIL{mɛ.ˈkɐr} & \textsc{v.mo(dy)} & blossom\\
& \textitbf{melulu} & \textstyleChCharisSIL{mɛ.ˈlu.lu} & \textsc{adv} & exclusively\\
& \textitbf{memang} & \textstyleChCharisSIL{ˈmɛ.mɐŋ} & \textsc{adv} & indeed\\
\textstyleExampleSource{x} & \textitbf{menang} & \textstyleChCharisSIL{mɛ.ˈnɐŋ} & \textsc{v.bi} & win\\
\textstyleExampleSource{x} & \textitbf{menta} & \textstyleChCharisSIL{mɛ̞n.ˈta} & \textsc{v.mo(st)} & be uncooked\\
& \textitbf{mera} & \textstyleChCharisSIL{ˈmɛ.ɾa} & \textsc{v.mo(st)} & be red\\
& \textitbf{mesti} & \textstyleChCharisSIL{ˈmɛ̞s.ti} & \textsc{v.aux} & have to\\
& \textitbf{meti} & \textstyleChCharisSIL{ˈmɛ.ti} & \textsc{n} & low tide\\
& \textitbf{mewa} & \textstyleChCharisSIL{ˈmɛ.wa} & \textsc{v.mo(st)} & be luxurious\\
& \textitbf{mimpi} & \textstyleChCharisSIL{ˈmɪm.pi} & \textsc{v.bi} & dream (of)\\
& \textitbf{minang} & \textstyleChCharisSIL{ˈmi.nɐŋ} & \textsc{v.bi} & propose\\
& \textitbf{minta} & \textstyleChCharisSIL{ˈmɪn.ta} & \textsc{v.bi} & request\\
& \textitbf{minum} & \textstyleChCharisSIL{ˈmi.nʊm} & \textsc{v.bi} & drink\\
& \textitbf{minyak} & \textstyleChCharisSIL{ˈmi.ɲɐk} & \textsc{n} & oil\\
& \textitbf{miring} & \textstyleChCharisSIL{ˈmɪ.ɾɪŋ} & \textsc{v.mo(st)} & be sideways\\
& \textitbf{mirip} & \textstyleChCharisSIL{ˈmɪ.ɾɪp̚} & \textsc{v.bi} & resemble\\
& \textitbf{mo} & \textstyleChCharisSIL{ˈmɔ} & \textsc{v.bi} & want\\
& \textitbf{molo} & \textstyleChCharisSIL{ˈmɔ.lɔ} & \textsc{v.bi} & dive, drown\\
& \textitbf{mono} & \textstyleChCharisSIL{ˈmɔ.nɔ} & \textsc{v.mo(st)} & be stupid\\
& \textitbf{monyet} & \textstyleChCharisSIL{ˈmɔ.ɲɛ̞t} & \textsc{n} & monkey\\
& \textitbf{moyang} & \textstyleChCharisSIL{ˈmɔ.jɐŋ} & \textsc{n} & ancestor\\
& \textitbf{muara} & \textstyleChCharisSIL{mʊ.ˈa.ɾa} & \textsc{n} & river mouth\\
& \textitbf{muat} & \textstyleChCharisSIL{ˈmʊ.ɐt̚} & \textsc{v.bi} & hold\\
& \textitbf{muda} & \textstyleChCharisSIL{ˈmu.da} & \textsc{v.mo(st)} & be easy\\
& \textitbf{mujair} & \textstyleChCharisSIL{mu.ˈdʒa.ɪr} & \textsc{n} & tilapiine fish\\
& \textitbf{muka} & \textstyleChCharisSIL{ˈmu.ka} & \textsc{n} & front\\
& \textitbf{mulut} & \textstyleChCharisSIL{ˈmʊ.lʊt̚} & \textsc{n} & mouth\\
& \textitbf{muncul} & \textstyleChCharisSIL{ˈmʊn.tʃʊl} & \textsc{v.mo(dy)} & appear\\
& \textitbf{mundur} & \textstyleChCharisSIL{ˈmʊn.dʊr} & \textsc{v.mo(dy)} & back up\\
& \textitbf{mungking} & \textstyleChCharisSIL{ˈmʊŋ.kɪn} & \textsc{adv} & maybe\\
& \textitbf{munta} & \textstyleChCharisSIL{ˈmʊn.ta} & \textsc{v.bi} & vomit\\
& \textitbf{mura} & \textstyleChCharisSIL{ˈmʊ.ɾa} & \textsc{v.mo(st)} & be cheap\\
& \textitbf{murni} & \textstyleChCharisSIL{ˈmʊr.ni} & \textsc{v.mo(st)} & be pure\\
& \textitbf{musu} & \textstyleChCharisSIL{ˈmu.su} & \textsc{n/v.bi} & enemy / hate\\
& \textstyleChBold{N} &  &  & \\
& \textitbf{naik} & \textstyleChCharisSIL{ˈna.ɪk̚} & \textsc{v.bi} & ascend\\
& \textitbf{nakal} & \textstyleChCharisSIL{ˈna.kɐl} & \textsc{v.mo(st)} & be mischievous\\
& \textitbf{nangka} & \textstyleChCharisSIL{ˈnɐŋ.ka} & \textsc{n} & jackfruit\\
& \textitbf{nanti} & \textstyleChCharisSIL{ˈnɐn.ti} & \textsc{adv} & very soon\\
& \textitbf{nasi} & \textstyleChCharisSIL{ˈna.si} & \textsc{n} & cooked rice\\
& \textitbf{nekat} & \textstyleChCharisSIL{ˈnɛ.kɐt̚} & \textsc{v.bi} & determine\\
& \textitbf{nene} & \textstyleChCharisSIL{ˈnɛ.nɛ} & \textsc{n} & grandmother\\
& \textitbf{nika} & \textstyleChCharisSIL{ˈni.ka} & \textsc{v.bi} & marry officially\\
& \textitbf{nokeng} & \textstyleChCharisSIL{ˈnɔ.kɛ̞n} & \textsc{n} & stringbag\\
& \textitbf{nontong} & \textstyleChCharisSIL{ˈnɔ̞n.tɔ̞ŋ} & \textsc{v.bi} & watch for entertainment\\
& \textitbf{nyamang} & \textstyleChCharisSIL{ˈɲa.mɐn} & \textsc{v.mo(st)} & be comfortable\\
& \textitbf{nyawa} & \textstyleChCharisSIL{ˈɲa.wa} & \textsc{n} & soul\\
& \textitbf{nyonyor} & \textstyleChCharisSIL{ˈɲɔ.ɲɔ̞r̥} & \textsc{v.mo(st)} & be black and blue\\
& \textstyleChBold{O} &  &  & \\
& \textitbf{obat} & \textstyleChCharisSIL{ˈɔ.bɐt} & \textsc{n} & medicine\\
& \textitbf{ojek} & \textstyleChCharisSIL{ˈɔ.dʒɛ̞k̚} & \textsc{n/v.bi} & motorbike taxi / take motorbike taxi\\
& \textitbf{olaraga} & \textstyleChCharisSIL{ˌɔ.la.ˈɾa.ga} & \textsc{v.mo(dy)} & do sports\\
& \textitbf{oleng} & \textstyleChCharisSIL{ˈɔ.lɛ̞ŋ} & \textsc{v.mo(dy)} & shake\\
& \textitbf{ombak} & \textstyleChCharisSIL{ˈɔ̞m.bɐk} & \textsc{n} & wave\\
& \textitbf{omel} & \textstyleChCharisSIL{ˈɔ.mɛ̞l} & \textsc{v.bi} & complain\\
& \textitbf{orang} & \textstyleChCharisSIL{ˈɔ.ɾɐŋ} & \textsc{n} & person\\
& \textitbf{otak} & \textstyleChCharisSIL{ˈɔ.tɐk} & \textsc{n} & brain\\
& \textitbf{otot} & \textstyleChCharisSIL{ˈɔ̞.tɔ̞t̚} & \textsc{n} & muscle\\
& \textstyleChBold{P} &  &  & \\
& \textitbf{pacar} & \textstyleChCharisSIL{ˈpa.tʃɐr̥} & \textsc{n/v.bi} & lover / date\\
& \textitbf{pace} & \textstyleChCharisSIL{ˈpa.tʃɛ} & \textsc{n} & man\\
& \textitbf{padam} & \textstyleChCharisSIL{ˈpa.dɐm} & \textsc{v.bi} & extinguish\\
& \textitbf{padat} & \textstyleChCharisSIL{ˈpa.dɐt} & \textsc{v.mo(st)} & be solid\\
& \textitbf{padede} & \textstyleChCharisSIL{pa.ˈdɛ̞.dɛ} & \textsc{v.bi} & whine (for)\\
& \textitbf{padu} & \textstyleChCharisSIL{ˈpa.du} & \textsc{v.mo(st)} & be fused\\
& \textitbf{pagi} & \textstyleChCharisSIL{ˈpa.gi} & \textsc{n} & morning\\
& \textitbf{paha} & \textstyleChCharisSIL{ˈpa.ha} & \textsc{n} & thigh\\
& \textitbf{pahit} & \textstyleChCharisSIL{ˈpa.hɪt} & \textsc{v.mo(st)} & be bitter\\
& \textitbf{pajak} & \textstyleChCharisSIL{ˈpa.dʒɐk} & \textsc{n} & tax\\
& \textitbf{pake} & \textstyleChCharisSIL{ˈpa.kɛ} & \textsc{v.bi} & use\\
& \textitbf{paku} & \textstyleChCharisSIL{ˈpa.ku} & \textsc{n} & nail\\
& \textitbf{paling} & \textstyleChCharisSIL{ˈpa.lɪŋ} & \textsc{adv} & most\\
& \textitbf{palungku} & \textstyleChCharisSIL{pa.ˈlʊŋ.kʊ} & \textsc{v.bi} & punch\\
& \textitbf{pamali} & \textstyleChCharisSIL{pa.ˈma.li} & \textsc{n} & taboo\\
& \textitbf{pamang} & \textstyleChCharisSIL{ˈpa.mɐn} & \textsc{n} & uncle\\
& \textitbf{pamer} & \textstyleChCharisSIL{ˈpa.mɛ̞r} & \textsc{v.bi} & show off\\
& \textitbf{pana} & \textstyleChCharisSIL{ˈpa.na} & \textsc{n/v.bi} & arrow / bow shoot\\
& \textitbf{panas} & \textstyleChCharisSIL{ˈpa.nɐs} & \textsc{v.mo(st)} & be hot\\
& \textitbf{panggal} & \textstyleChCharisSIL{ˈpɐŋ.gɐl} & \textsc{n} & fragment\\
& \textitbf{panggil} & \textstyleChCharisSIL{ˈpɐŋ.gɪl} & \textsc{v.bi} & call\\
& \textitbf{pangkal} & \textstyleChCharisSIL{ˈpɐŋ.kɐl} & \textsc{n} & base\\
& \textitbf{pangkat} & \textstyleChCharisSIL{ˈpɐŋ.kɐt} & \textsc{n} & rank\\
& \textitbf{pangku} & \textstyleChCharisSIL{ˈpɐŋ.ku} & \textsc{v.bi} & hold on lap\\
& \textitbf{panjang} & \textstyleChCharisSIL{ˈpɐn.dʒɐŋ} & \textsc{v.mo(st)} & be long\\
& \textitbf{panjat} & \textstyleChCharisSIL{ˈpɐn.dʒɐt̚} & \textsc{v.bi} & climb\\
& \textitbf{panta} & \textstyleChCharisSIL{ˈpɐn.ta} & \textsc{n} & buttock\\
& \textitbf{pantas} & \textstyleChCharisSIL{ˈpɐn.tɐs} & \textsc{v.mo(st)} & be proper\\
& \textitbf{pante} & \textstyleChCharisSIL{ˈpɐn.tɛ} & \textsc{n} & coast\\
& \textitbf{pantul} & \textstyleChCharisSIL{ˈpɐn.tʊl} & \textsc{v.mo(dy)} & bounce back\\
& \textitbf{pantung} & \textstyleChCharisSIL{ˈpɐn.tʊn} & \textsc{n} & k. o. traditional poetry\\
& \textitbf{papang} & \textstyleChCharisSIL{ˈpa.pɐn} & \textsc{n} & plank\\
& \textitbf{papeda} & \textstyleChCharisSIL{pa.ˈpɛ.da} & \textsc{n} & sagu porridge\\
& \textitbf{para} & \textstyleChCharisSIL{ˈpa.ɾa} & \textsc{v.mo(st)} & be in serious condition\\
& \textitbf{parang} & \textstyleChCharisSIL{ˈpa.ɾɐŋ} & \textsc{n} & short machete\\
& \textitbf{parit} & \textstyleChCharisSIL{ˈpa.ɾɪt} & \textsc{n} & ditch\\
& \textitbf{parut} & \textstyleChCharisSIL{ˈpa.ɾʊt̚} & \textsc{v.bi} & scrape\\
& \textitbf{pasang} & \textstyleChCharisSIL{ˈpa.sɐŋ} & \textsc{n} & pair\\
& \textitbf{pasang} & \textstyleChCharisSIL{ˈpa.sɐŋ} & \textsc{v.bi} & install\\
& \textitbf{pasar} & \textstyleChCharisSIL{ˈpa.sɐr} & \textsc{n} & market\\
& \textitbf{pasir} & \textstyleChCharisSIL{ˈpa.sɪr̥} & \textsc{n} & sand\\
& \textitbf{pasti} & \textstyleChCharisSIL{ˈpɐs.ti} & \textsc{adv} & definitely\\
& \textitbf{pasukang} & \textstyleChCharisSIL{pa.ˈsu.kɐn} & \textsc{n} & troops\\
& \textitbf{pata} & \textstyleChCharisSIL{ˈpa.ta} & \textsc{v.bi} & break\\
& \textitbf{patung} & \textstyleChCharisSIL{ˈpa.tʊŋ} & \textsc{n} & statue\\
& \textitbf{pecek} & \textstyleChCharisSIL{ˈpɛ.tʃɛ̞k} & \textsc{n} & mud\\
& \textitbf{pecis} & \textstyleChCharisSIL{ˈpɛ.tʃɪs} & \textsc{n} & light bulb\\
\textstyleExampleSource{x} & \textitbf{pedang} & \textstyleChCharisSIL{pɛ.ˈdɐŋ} & \textsc{n} & sword\\
\textstyleExampleSource{x} & \textitbf{pedis} & \textstyleChCharisSIL{pɛ.ˈdɪs} & \textsc{v.mo(st)} & be spicy\\
\textstyleExampleSource{x} & \textitbf{pegang} & \textstyleChCharisSIL{pɛ.ˈgɐŋ} & \textsc{v.bi} & hold\\
& \textitbf{pegaway} & \textstyleChCharisSIL{pə.ˈga.wɐj} & \textsc{n} & government employee\\
& \textitbf{pele} & \textstyleChCharisSIL{ˈpɛ.lɛ} & \textsc{v.bi} & cover\\
\textstyleExampleSource{x} & \textitbf{pelepa} & \textstyleChCharisSIL{ˌpɛ.lɛ.ˈpa} & \textsc{n} & palm stem and midrib\\
& \textitbf{pendek} & \textstyleChCharisSIL{ˈpɛ̞n.dɛ̞k} & \textsc{v.mo(st)} & be short\\
\textstyleExampleSource{x} & \textitbf{penting} & \textstyleChCharisSIL{pɛ̞n.ˈtɪŋ} & \textsc{v.mo(st)} & be important\\
\textstyleExampleSource{x} & \textitbf{penu} & \textstyleChCharisSIL{pɛ.ˈnu} & \textsc{v.mo(st)} & be full\\
\textstyleExampleSource{x} & \textitbf{pergi} & \textstyleChCharisSIL{pɛ̞r.ˈgi} & \textsc{v.mo(dy)} & go\\
\textstyleExampleSource{x} & \textitbf{perna} & \textstyleChCharisSIL{pɛ̞r.ˈna} & \textsc{adv} & ever\\
\textstyleExampleSource{x} & \textitbf{pesang} & \textstyleChCharisSIL{pɛ.ˈsɐn} & \textsc{v.bi} & order\\
& \textitbf{pesawat} & \textstyleChCharisSIL{pɛ.ˈsa.wɐt} & \textsc{n} & airplane\\
& \textitbf{pesisir} & \textstyleChCharisSIL{pɛ.ˈsɪ.sɪr̥} & \textsc{n} & shore\\
& \textitbf{petatas} & \textstyleChCharisSIL{pɛ.ˈta.tɐs} & \textsc{n} & sweet potato\\
& \textitbf{pete} & \textstyleChCharisSIL{ˈpɛ.tɛ} & \textsc{v.bi} & pick\\
& \textitbf{piatu} & \textstyleChCharisSIL{pi.ˈa.tu} & \textsc{v.mo(st)} & be motherless\\
& \textitbf{pica} & \textstyleChCharisSIL{ˈpi.tʃa} & \textsc{v.mo(st)} & be broken\\
& \textitbf{pijit} & \textstyleChCharisSIL{ˈpi.dʒɪt̚} & \textsc{v.bi} & massage\\
& \textitbf{pikol} & \textstyleChCharisSIL{ˈpi.kɔ̞l} & \textsc{v.bi} & shoulder\\
& \textitbf{pikung} & \textstyleChCharisSIL{ˈpi.kʊn} & \textsc{v.mo(st)} & be senile\\
& \textitbf{pili} & \textstyleChCharisSIL{ˈpi.li} & \textsc{v.bi} & choose\\
& \textitbf{pimping} & \textstyleChCharisSIL{ˈpɪm.pɪn} & \textsc{v.bi} & lead\\
& \textitbf{pinang} & \textstyleChCharisSIL{ˈpɪ.nɐŋ} & \textsc{n} & betel nut\\
& \textitbf{pinda} & \textstyleChCharisSIL{ˈpɪn.da} & \textsc{v.bi} & move\\
& \textitbf{pinggang} & \textstyleChCharisSIL{ˈpɪŋ.gɐŋ} & \textsc{n} & loins\\
& \textitbf{pinggir} & \textstyleChCharisSIL{ˈpɪŋ.gɪr̥} & \textsc{n-loc} & border\\
& \textitbf{pinjam} & \textstyleChCharisSIL{ˈpɪn.dʒɐm} & \textsc{v.bi} & borrow\\
& \textitbf{pintar} & \textstyleChCharisSIL{ˈpɪn.tɐr̥} & \textsc{v.mo(st)} & be clever\\
& \textitbf{pintu} & \textstyleChCharisSIL{ˈpɪn.tu} & \textsc{n} & door\\
& \textitbf{pisa} & \textstyleChCharisSIL{ˈpi.sa} & \textsc{v.mo(st)} & be separate\\
& \textitbf{pisang} & \textstyleChCharisSIL{ˈpi.sɐŋ} & \textsc{n} & banana\\
& \textitbf{pisow} & \textstyleChCharisSIL{ˈpi.sɔ̞w} & \textsc{n} & knife\\
& \textitbf{pita} & \textstyleChCharisSIL{ˈpi.ta} & \textsc{n} & ribbon of volleyball net\\
& \textitbf{plaka} & \textstyleChCharisSIL{ˈpla.ka} & \textsc{v.bi} & fall over\\
& \textitbf{plang} & \textstyleChCharisSIL{ˈplɐn} & \textsc{v.mo(st)} & be slow\\
& \textitbf{pluk} & \textstyleChCharisSIL{ˈplʊk̚} & \textsc{v.bi} & embrace\\
& \textitbf{pohong} & \textstyleChCharisSIL{ˈpɔ̞.hɔ̞n} & \textsc{n} & tree\\
& \textitbf{potong} & \textstyleChCharisSIL{ˈpɔ̞.tɔ̞ŋ} & \textsc{v.bi} & cut\\
& \textitbf{prahu} & \textstyleChCharisSIL{ˈpra.hʊ} & \textsc{n} & boat\\
& \textitbf{prang} & \textstyleChCharisSIL{ˈprɐŋ} & \textsc{n} & war\\
& \textitbf{prempuang} & \textstyleChCharisSIL{prɛ̞m.ˈpʊ.ɐn} & \textsc{n} & woman\\
& \textitbf{printa} & \textstyleChCharisSIL{ˈprɪn.ta} & \textsc{n/v.bi} & command\\
& \textitbf{prut} & \textstyleChCharisSIL{ˈprʊt} & \textsc{n} & stomach\\
& \textitbf{pu} & \textstyleChCharisSIL{ˈpu} & \textsc{v.poss} & possessive\\
& \textitbf{puas} & \textstyleChCharisSIL{ˈpʊ.ɐs} & \textsc{v.mo(st)} & be satisfied\\
& \textitbf{pukul} & \textstyleChCharisSIL{ˈpʊ.kʊl} & \textsc{v.bi} & hit\\
& \textitbf{pulang} & \textstyleChCharisSIL{ˈpu.lɐŋ} & \textsc{v.bi} & go home\\
& \textitbf{pulow} & \textstyleChCharisSIL{ˈpu.lɔ̞w} & \textsc{n} & island\\
& \textitbf{pulu} & \textstyleChCharisSIL{ˈpu.lu} & \textsc{num.c} & tens\\
& \textitbf{puntung} & \textstyleChCharisSIL{ˈpʊn.tʊŋ} & \textsc{n} & butt\\
& \textitbf{punya} & \textstyleChCharisSIL{ˈpu.ɲa} & \textsc{v.poss/v.bi} & \textsc{poss} / have\\
& \textitbf{puri} & \textstyleChCharisSIL{ˈpu.ɾi} & \textsc{n} & anchovy-like fish\\
& \textitbf{pusing} & \textstyleChCharisSIL{ˈpu.sɪŋ} & \textsc{v.mo(st)} & be dizzy, be confused\\
& \textitbf{putar} & \textstyleChCharisSIL{ˈpu.tɐr} & \textsc{v.bi} & turn around\\
& \textitbf{puti} & \textstyleChCharisSIL{ˈpu.ti} & \textsc{v.mo(st)} & be white\\
& \textitbf{putus} & \textstyleChCharisSIL{ˈpu.tʊs} & \textsc{v.mo(st)} & be broken\\
& \textstyleChBold{R} &  &  & \\
& \textitbf{raba} & \textstyleChCharisSIL{ˈra.ba} & \textsc{v.bi} & grope\\
& \textitbf{rabik} & \textstyleChCharisSIL{ˈra.bɪk̚} & \textsc{v.bi} & tear\\
& \textitbf{ragu} & \textstyleChCharisSIL{ˈra.gu} & \textsc{v.bi} & doubt\\
& \textitbf{rajing} & \textstyleChCharisSIL{ˈra.dʒɪn} & \textsc{v.mo(st)} & be diligent\\
& \textitbf{rakit} & \textstyleChCharisSIL{ˈra.kɪt} & \textsc{n} & raft\\
& \textitbf{rakus} & \textstyleChCharisSIL{ˈra.kʊs} & \textsc{v.bi} & crave\\
& \textitbf{ramas} & \textstyleChCharisSIL{ˈra.mɐs} & \textsc{v.bi} & press\\
& \textitbf{rambut} & \textstyleChCharisSIL{ˈrɐm.bʊt̚} & \textsc{n} & hair\\
& \textitbf{rame} & \textstyleChCharisSIL{ˈra.mɛ} & \textsc{v.mo(st)} & be bustling\\
& \textitbf{rampas} & \textstyleChCharisSIL{ˈrɐm.pɐs} & \textsc{v.bi} & seize\\
& \textitbf{rangkap} & \textstyleChCharisSIL{ˈrɐŋ.kɐp̚} & \textsc{v.mo(st)} & be doubled\\
& \textitbf{rapat} & \textstyleChCharisSIL{ˈra.pɐt} & \textsc{n/v.mo(dy)} & meeting / move close\\
& \textitbf{rapi} & \textstyleChCharisSIL{ˈra.pi} & \textsc{v.mo(st)} & be neat\\
& \textitbf{rata} & \textstyleChCharisSIL{ˈra.ta} & \textsc{v.mo(st)} & be even\\
& \textitbf{ratus} & \textstyleChCharisSIL{ˈra.tʊs} & \textsc{num.c} & hundred\\
& \textitbf{rawa} & \textstyleChCharisSIL{ˈra.wa} & \textsc{n} & swamp\\
& \textitbf{rawang} & \textstyleChCharisSIL{ˈra.wɐn} & \textsc{v.mo(st)} & be haunted\\
& \textitbf{rawat} & \textstyleChCharisSIL{ˈra.wɐt̚} & \textsc{v.bi} & take care of\\
\textstyleExampleSource{x} & \textitbf{rebus} & \textstyleChCharisSIL{rɛ.ˈbʊs} & \textsc{v.bi} & boil\\
\textstyleExampleSource{x} & \textitbf{rebut} & \textstyleChCharisSIL{rɛ.ˈbʊt̚} & \textsc{v.bi} & race each other\\
\textstyleExampleSource{x} & \textitbf{renda} & \textstyleChCharisSIL{rɛ̞n.ˈda} & \textsc{v.mo(st)} & be low\\
\textstyleExampleSource{x} & \textitbf{rendam} & \textstyleChCharisSIL{rɛ̞n.ˈdɐm} & \textsc{v.bi} & soak\\
& \textitbf{repot} & \textstyleChCharisSIL{ˈrɛ.pɔ̞t} & \textsc{v.mo(st)} & be busy\\
& \textitbf{ribu} & \textstyleChCharisSIL{ˈri.bu} & \textsc{num.c} & thousand\\
& \textitbf{ribut} & \textstyleChCharisSIL{ˈri.bʊt} & \textsc{v.bi} & trouble\\
& \textitbf{rica} & \textstyleChCharisSIL{ˈri.tʃa} & \textsc{n} & red pepper\\
& \textitbf{rindu} & \textstyleChCharisSIL{ˈrɪn.du} & \textsc{v.bi} & long for\\
& \textitbf{ringang} & \textstyleChCharisSIL{ˈri.ŋɐn} & \textsc{v.mo(st)} & be light\\
& \textitbf{rokok} & \textstyleChCharisSIL{ˈrɔ̞.kɔ̞k} & \textsc{n} & cigarette\\
& \textitbf{ruang} & \textstyleChCharisSIL{ˈrʊ.ɐŋ} & \textsc{n} & room\\
& \textitbf{rubu} & \textstyleChCharisSIL{ˈru.bu} & \textsc{v.mo(dy)} & collapse\\
& \textitbf{rugi} & \textstyleChCharisSIL{ˈru.gi} & \textsc{v.bi} & lose out\\
& \textitbf{ruma} & \textstyleChCharisSIL{ˈrʊ.ma} & \textsc{n} & house\\
& \textitbf{rumput} & \textstyleChCharisSIL{ˈrʊm.pʊt} & \textsc{n} & grass\\
& \textitbf{runcing} & \textstyleChCharisSIL{ˈrʊn.tʃɪŋ} & \textsc{v.mo(st)} & be pointed\\
& \textitbf{rusak} & \textstyleChCharisSIL{ˈrʊ.sɐk} & \textsc{v.mo(st)} & be damaged\\
& \textitbf{rusuk} & \textstyleChCharisSIL{ˈrʊ.sʊk} & \textsc{n} & rib\\
& \textstyleChBold{S} &  &  & \\
& \textitbf{sa} & \textstyleChCharisSIL{ˈsa} & \textsc{pro} & \textsc{1sg}\\
& \textitbf{sabit} & \textstyleChCharisSIL{ˈsa.bɪt} & \textsc{n} & sickle\\
& \textitbf{sadap} & \textstyleChCharisSIL{ˈsa.dɐp̚} & \textsc{v.mo(st)} & be delicious\\
& \textitbf{sadar} & \textstyleChCharisSIL{ˈsa.dɐr̥} & \textsc{v.mo(st)} & be aware\\
& \textitbf{sagu} & \textstyleChCharisSIL{ˈsa.gu} & \textsc{n} & sago\\
& \textitbf{saing} & \textstyleChCharisSIL{ˈsa.ɪŋ} & \textsc{v.mo(dy)} & compete\\
& \textitbf{sakit} & \textstyleChCharisSIL{ˈsa.kɪt̚} & \textsc{v.mo(st)} & be sick\\
& \textitbf{sala} & \textstyleChCharisSIL{ˈsa.la} & \textsc{v.mo(st)} & be wrong\\
& \textitbf{salip} & \textstyleChCharisSIL{ˈsa.lɪp̚} & \textsc{n/v.bi} & cross / crucify\\
& \textitbf{sama} & \textstyleChCharisSIL{ˈsa.ma} & \textsc{v.mo(st)}

\textsc{prep/cnj} & be same

to\\
& \textitbf{sambar} & \textstyleChCharisSIL{ˈsɐm.bɐr} & \textsc{v.bi} & seize\\
& \textitbf{sambil} & \textstyleChCharisSIL{ˈsɐm.bɪl} & \textsc{cnj} & while\\
& \textitbf{sambung} & \textstyleChCharisSIL{ˈsɐm.bʊŋ} & \textsc{v.bi} & continue\\
& \textitbf{sambut} & \textstyleChCharisSIL{ˈsɐm.bʊt̚} & \textsc{v.bi} & welcome\\
& \textitbf{sampa} & \textstyleChCharisSIL{ˈsɐm.pa} & \textsc{n} & trash\\
& \textitbf{sampe} & \textstyleChCharisSIL{ˈsɐm.pɛ} & \textsc{v.bi}

\textsc{prep}

\textsc{cnj} & reach

until

until, with the result that\\
& \textitbf{samping} & \textstyleChCharisSIL{ˈsɐm.pɪŋ} & \textsc{n-loc} & side\\
& \textitbf{sana} & \textstyleChCharisSIL{ˈsa.na} & \textsc{loc} & \textsc{l.dist}\\
& \textitbf{sandar} & \textstyleChCharisSIL{ˈsɐn.dɐr̥} & \textsc{v.mo(dy)} & lean\\
& \textitbf{sanggup} & \textstyleChCharisSIL{ˈsɐŋ.gʊp} & \textsc{v.mo(st)} & be capable\\
& \textitbf{santang} & \textstyleChCharisSIL{ˈsɐn.tɐn} & \textsc{n} & coconut milk\\
& \textitbf{sante} & \textstyleChCharisSIL{ˈsɐn.tɛ} & \textsc{v.mo(dy)} & relax\\
& \textitbf{sapi} & \textstyleChCharisSIL{ˈsa.pɪ} & \textsc{n} & cow\\
& \textitbf{sapu} & \textstyleChCharisSIL{ˈsa.pu} & \textsc{n/v.bi} & broom / sweep\\
& \textitbf{sarana} & \textstyleChCharisSIL{sa.ˈɾa.na} & \textsc{n} & facility\\
& \textitbf{sarang} & \textstyleChCharisSIL{ˈsa.ɾɐn} & \textsc{n} & suggestion\\
& \textitbf{sarat} & \textstyleChCharisSIL{ˈsa.ɾɐt̚} & \textsc{v.mo(st)} & be loaded\\
& \textitbf{saring} & \textstyleChCharisSIL{ˈsa.ɾɪŋ} & \textsc{v.bi} & filter\\
& \textitbf{sarung} & \textstyleChCharisSIL{ˈsa.ɾʊn} & \textsc{n} & protective sleeve\\
& \textitbf{sasarang} & \textstyleChCharisSIL{sa.ˈsa.ɾɐŋ} & \textsc{n} & target\\
& \textitbf{satu} & \textstyleChCharisSIL{ˈsa.tu} & \textsc{num.c} & one\\
& \textitbf{saya} & \textstyleChCharisSIL{ˈsa.ja} & \textsc{pro} & \textsc{1sg}\\
& \textitbf{sayang} & \textstyleChCharisSIL{ˈsa.jɐŋ} & \textsc{v.bi} & love\\
& \textitbf{sayur} & \textstyleChCharisSIL{ˈsa.jʊr̥} & \textsc{n} & vegetable\\
\textstyleExampleSource{x} & \textitbf{sebut} & \textstyleChCharisSIL{sɛ.ˈbʊt̚} & \textsc{v.bi} & name\\
\textstyleExampleSource{x} & \textitbf{sedi} & \textstyleChCharisSIL{sɛ.ˈdi} & \textsc{v.mo(st)} & be sad\\
& \textitbf{sedikit} & \textstyleChCharisSIL{sɛ.ˈdɪ.kɪt̚} & \textsc{qt} & few\\
\textstyleExampleSource{x} & \textitbf{sedot} & \textstyleChCharisSIL{sɛ.ˈdɔ̞t̚} & \textsc{v.bi} & suck\\
\textstyleExampleSource{x} & \textitbf{segar} & \textstyleChCharisSIL{sɛ.ˈgɐr̥} & \textsc{v.mo(st)} & be fresh\\
& \textitbf{seher} & \textstyleChCharisSIL{ˈsɛ̞.hɛ̞r̥} & \textsc{n} & piston\\
\textstyleExampleSource{x} & \textitbf{sejak} & \textstyleChCharisSIL{sɛ.ˈdʒɐk̚} & \textsc{adv} & since\\
\textstyleExampleSource{x} & \textitbf{selesay} & \textstyleChCharisSIL{ˌsɛ.lɛ.ˈsɐj} & \textsc{v.bi} & finish\\
& \textitbf{semang} & \textstyleChCharisSIL{ˈsɛ.mɐn} & \textsc{n} & outrigger\\
\textstyleExampleSource{x} & \textitbf{semba} & \textstyleChCharisSIL{sɛ̞m.ˈba} & \textsc{v.bi} & worship\\
& \textitbf{sembayang} & \textstyleChCharisSIL{səm.ˈba.jɐŋ} & \textsc{v.bi} & worship\\
& \textitbf{sembilang} & \textstyleChCharisSIL{səm.ˈbi.lɐn} & \textsc{num.c} & nine\\
\textstyleExampleSource{x} & \textitbf{sembu} & \textstyleChCharisSIL{sɛ̞m.ˈbu} & \textsc{v.mo(st)} & be healed\\
& \textitbf{sembuni} & \textstyleChCharisSIL{sɛ̞m.ˈbu.ni} & \textsc{v.bi} & hide\\
\textstyleExampleSource{x} & \textitbf{sempat} & \textstyleChCharisSIL{sɛ̞m.ˈpɐt̚} & \textsc{v.mo(dy)} & have enough time\\
\textstyleExampleSource{x} & \textitbf{sempit} & \textstyleChCharisSIL{sɛ̞m.ˈpɪt} & \textsc{v.mo(st)} & be narrow\\
& \textitbf{sendiri} & \textstyleChCharisSIL{sɛ̞n.ˈdi.ɾi} & \textsc{v.mo(st)} & be alone\\
& \textitbf{sendok} & \textstyleChCharisSIL{ˈsɛ̞n.dɔ̞k} & \textsc{n/v.bi} & spoon\\
& \textitbf{seneng} & \textstyleChCharisSIL{ˈsɛ.nɛ̞n} & \textsc{v.bi} & sign\\
& \textitbf{sentu} & \textstyleChCharisSIL{ˈsɛ̞n.tu} & \textsc{v.bi} & touch\\
\textstyleExampleSource{x} & \textitbf{senyum} & \textstyleChCharisSIL{sɛ.ˈɲʊm} & \textsc{v.mo(dy)} & smile\\
\textstyleExampleSource{x} & \textitbf{sepi} & \textstyleChCharisSIL{sɛ.ˈpi} & \textsc{v.mo(st)} & be quiet\\
& \textitbf{seraka} & \textstyleChCharisSIL{sɛ.ˈɾa.ka} & \textsc{v.mo(st)} & be greedy\\
\textstyleExampleSource{x} & \textitbf{serang} & \textstyleChCharisSIL{sɛ.ˈɾɐŋ} & \textsc{v.bi} & attack\\
\textstyleExampleSource{x} & \textitbf{serey} & \textstyleChCharisSIL{sɛ.ˈɾɛ̞j} & \textsc{n} & lemongrass\\
\textstyleExampleSource{x} & \textitbf{sesak} & \textstyleChCharisSIL{sɛ.ˈsɐk} & \textsc{v.mo(st)} & be crowded\\
& \textitbf{sial} & \textstyleChCharisSIL{ˈsi.ɐl} & \textsc{v.mo(st)} & be unfortunate\\
& \textitbf{siang} & \textstyleChCharisSIL{ˈsi.ɐŋ} & \textsc{n} & midday\\
& \textitbf{siap} & \textstyleChCharisSIL{ˈsi.ɐp̚} & \textsc{v.bi} & get ready\\
& \textitbf{siapa} & \textstyleChCharisSIL{si.ˈa.pa} & \textsc{int} & who\\
& \textitbf{sibuk} & \textstyleChCharisSIL{ˈsi.bʊk} & \textsc{v.mo(st)} & be busy\\
& \textitbf{sidang} & \textstyleChCharisSIL{ˈsi.dɐŋ} & \textsc{n} & meeting\\
& \textitbf{sikakar} & \textstyleChCharisSIL{si.ˈka.kɐr̥} & \textsc{v.bi} & hold onto tightfisted\\
& \textitbf{sikap} & \textstyleChCharisSIL{ˈsi.kɐp̚} & \textsc{n} & attitude\\
& \textitbf{sikat} & \textstyleChCharisSIL{ˈsi.kɐt} & \textsc{n} & brush\\
& \textitbf{simpang} & \textstyleChCharisSIL{ˈsɪm.pɐn} & \textsc{v.bi} & store, prepare\\
& \textitbf{singgung} & \textstyleChCharisSIL{ˈsɪŋ.gʊŋ} & \textsc{v.bi} & offend\\
& \textitbf{singkat} & \textstyleChCharisSIL{ˈsɪŋ.kɐt} & \textsc{v.mo(st)} & be brief\\
& \textitbf{sini} & \textstyleChCharisSIL{ˈsi.ni} & \textsc{loc} & \textsc{l.prox}\\
& \textitbf{siram} & \textstyleChCharisSIL{ˈsi.ɾɐm} & \textsc{v.bi} & pour over\\
& \textitbf{siri} & \textstyleChCharisSIL{ˈsi.ɾi} & \textsc{n} & betel vine\\
& \textitbf{sisi} & \textstyleChCharisSIL{ˈsi.si} & \textsc{n} & side\\
& \textitbf{sisir} & \textstyleChCharisSIL{ˈsi.sɪr} & \textsc{v.bi} & comb\\
& \textitbf{situ} & \textstyleChCharisSIL{ˈsi.tu} & \textsc{loc} & \textsc{l.med}\\
& \textitbf{skarang} & \textstyleChCharisSIL{ˈska.ɾɐŋ} & \textsc{v.mo(st)}

\textsc{adv} & be current

now\\
& \textitbf{slak} & \textstyleChCharisSIL{ˈslɐk} & \textsc{v.mo(dy)} & want to\\
& \textitbf{slatang} & \textstyleChCharisSIL{ˈsla.tɐn} & \textsc{n} & south\\
& \textitbf{slimut} & \textstyleChCharisSIL{ˈslɪ.mʊt} & \textsc{n} & blanket\\
& \textitbf{smangat} & \textstyleChCharisSIL{ˈsma.ŋɐt} & \textsc{v.mo(st)} & be enthusiastic\\
& \textitbf{smut} & \textstyleChCharisSIL{ˈsmʊt} & \textsc{n} & ant\\
& \textitbf{snang} & \textstyleChCharisSIL{ˈsnɐŋ} & \textsc{v.bi} & feel happy (about)\\
& \textitbf{sobek} & \textstyleChCharisSIL{ˈsɔ.bɛ̞k̚} & \textsc{v.bi} & tear\\
& \textitbf{sombong} & \textstyleChCharisSIL{ˈsɔ̞m.bɔ̞ŋ} & \textsc{v.mo(st)} & be arrogant\\
& \textitbf{sopang} & \textstyleChCharisSIL{ˈsɔ.pɐn} & \textsc{v.mo(st)} & be respectful\\
& \textitbf{sore} & \textstyleChCharisSIL{ˈsɔ.ɾɛ} & \textsc{n} & afternoon\\
& \textitbf{sorong} & \textstyleChCharisSIL{ˈsɔ̞.ɾɔ̞ŋ} & \textsc{v.bi} & slide\\
& \textitbf{sperti} & \textstyleChCharisSIL{ˈspɛ̞r.ti} & \textsc{prep/cnj} & similar to\\
& \textitbf{srabut} & \textstyleChCharisSIL{ˈsra.bʊt} & \textsc{n} & fiber\\
& \textitbf{sring} & \textstyleChCharisSIL{ˈsrɪŋ} & \textsc{adv} & often\\
& \textitbf{subu} & \textstyleChCharisSIL{ˈsu.bu} & \textsc{n} & very early morning\\
& \textitbf{subur} & \textstyleChCharisSIL{ˈsu.bʊr̥} & \textsc{v.mo(st)} & be fertile\\
& \textitbf{sudut} & \textstyleChCharisSIL{ˈsʊ.dʊt} & \textsc{n} & direction\\
& \textitbf{suka} & \textstyleChCharisSIL{ˈsu.ka} & \textsc{v.bi} & enjoy\\
& \textitbf{suku} & \textstyleChCharisSIL{ˈsu.ku} & \textsc{n} & ethnic group\\
& \textitbf{sulit} & \textstyleChCharisSIL{ˈsu.lɪt} & \textsc{v.mo(st)} & be difficult\\
& \textitbf{sumbang} & \textstyleChCharisSIL{ˈsʊm.bɐŋ} & \textsc{v.bi} & donate\\
& \textitbf{sumber} & \textstyleChCharisSIL{ˈsʊm.bɛ̞r̥} & \textsc{n} & source\\
& \textitbf{sumbur} & \textstyleChCharisSIL{ˈsʊm.bʊr} & \textsc{v.bi} & spit at\\
& \textitbf{sumur} & \textstyleChCharisSIL{ˈsʊ.mʊr̥} & \textsc{n} & well\\
& \textitbf{sungay} & \textstyleChCharisSIL{ˈsu.ŋɐj} & \textsc{n} & river\\
& \textitbf{sunggu} & \textstyleChCharisSIL{ˈsʊŋ.gu} & \textsc{v.mo(st)} & be true\\
& \textitbf{suntik} & \textstyleChCharisSIL{ˈsʊn.tɪk̚} & \textsc{v.bi} & inject\\
& \textitbf{surat} & \textstyleChCharisSIL{ˈsʊ.ɾɐt} & \textsc{n} & letter\\
& \textitbf{suru} & \textstyleChCharisSIL{ˈsu.ɾu} & \textsc{v.bi} & order\\
& \textitbf{susa} & \textstyleChCharisSIL{ˈsu.sa} & \textsc{v.mo(st)} & be difficult\\
& \textitbf{susu} & \textstyleChCharisSIL{ˈsu.su} & \textsc{n} & milk\\
& \textitbf{susung} & \textstyleChCharisSIL{ˈsu.sʊn} & \textsc{v.bi} & arrange\\
& \textitbf{swak} & \textstyleChCharisSIL{ˈswɐk} & \textsc{v.mo(st)} & be exhausted\\
& \textitbf{swanggi} & \textstyleChCharisSIL{ˈswɐŋ.gi} & \textsc{n} & nocturnal evil spirit, sorcerer\\
& \textstyleChBold{T} &  &  & \\
& \textitbf{tabrak} & \textstyleChCharisSIL{ˈta.brɐk} & \textsc{v.bi} & hit against\\
& \textitbf{tadi} & \textstyleChCharisSIL{ˈta.di} & \textsc{adv} & earlier\\
& \textitbf{tahang} & \textstyleChCharisSIL{ˈta.hɐn} & \textsc{v.bi} & hold (out/back)\\
& \textitbf{tahap} & \textstyleChCharisSIL{ˈta.hɐp̚} & \textsc{n} & phase\\
& \textitbf{tajam} & \textstyleChCharisSIL{ˈta.dʒɐm} & \textsc{v.mo(st)} & be sharp\\
& \textitbf{takut} & \textstyleChCharisSIL{ˈta.kʊt̚} & \textsc{v.bi} & feel afraid (of)\\
& \textitbf{tali} & \textstyleChCharisSIL{ˈta.li} & \textsc{n} & cord\\
& \textitbf{talut} & \textstyleChCharisSIL{ˈta.lʊt} & \textsc{v.bi} & build protection wall\\
& \textitbf{tamba} & \textstyleChCharisSIL{ˈtɐm.ba} & \textsc{v.bi} & add\\
& \textitbf{tampar} & \textstyleChCharisSIL{ˈtɐm.pɐr̥} & \textsc{v.bi} & beat\\
& \textitbf{tampeleng} & \textstyleChCharisSIL{tɐm.ˈpɛ̞.lɛ̞ŋ} & \textsc{v.bi} & slap on face or ears\\
& \textitbf{tampil} & \textstyleChCharisSIL{ˈtɐm.pɪl} & \textsc{v.mo(dy)} & perform\\
& \textitbf{tampung} & \textstyleChCharisSIL{ˈtɐm.pʊŋ} & \textsc{v.bi} & receive\\
& \textitbf{tamu} & \textstyleChCharisSIL{ˈta.mu} & \textsc{n} & guest\\
& \textitbf{tana} & \textstyleChCharisSIL{ˈta.na} & \textsc{n} & ground\\
& \textitbf{tanam} & \textstyleChCharisSIL{ˈta.nɐm} & \textsc{v.bi} & plant\\
& \textitbf{tanda} & \textstyleChCharisSIL{ˈtɐn.da} & \textsc{n} & sign\\
& \textitbf{tandang} & \textstyleChCharisSIL{ˈtɐn.dɐŋ} & \textsc{n} & banana plant stem\\
& \textitbf{tanding} & \textstyleChCharisSIL{ˈtɐn.dɪŋ} & \textsc{v.mo(dy)} & compete\\
& \textitbf{tangang} & \textstyleChCharisSIL{ˈta.ŋɐn} & \textsc{n} & hand\\
& \textitbf{tangga} & \textstyleChCharisSIL{ˈtɐŋ.ga} & \textsc{n} & ladder\\
& \textitbf{tanggal} & \textstyleChCharisSIL{ˈtɐŋ.gɐl} & \textsc{n} & date\\
& \textitbf{tanggap} & \textstyleChCharisSIL{ˈtɐŋ.gɐp̚} & \textsc{v.mo(st)} & be perceptive\\
& \textitbf{tanggulang} & \textstyleChCharisSIL{tɐŋ.ˈgʊ.lɐŋ} & \textsc{v.bi} & cope with\\
& \textitbf{tanggung} & \textstyleChCharisSIL{ˈtɐŋ.gʊŋ} & \textsc{v.bi} & bear\\
& \textitbf{tangkap} & \textstyleChCharisSIL{ˈtɐŋ.kɐp̚} & \textsc{v.bi} & catch\\
& \textitbf{tangkis} & \textstyleChCharisSIL{ˈtɐŋ.kɪs} & \textsc{v.bi} & ward off\\
& \textitbf{tanjung} & \textstyleChCharisSIL{ˈtɐn.dʒʊŋ} & \textsc{n} & cape\\
& \textitbf{tanjung} & \textstyleChCharisSIL{ˈtɐn.dʒʊŋ} & \textsc{v.mo(st)} & be tired\\
& \textitbf{tantang} & \textstyleChCharisSIL{ˈtɐn.tɐŋ} & \textsc{v.bi} & challenge\\
& \textitbf{tanya} & \textstyleChCharisSIL{ˈta.ɲa} & \textsc{v.bi} & ask\\
& \textitbf{tapis} & \textstyleChCharisSIL{ˈta.pɪs} & \textsc{v.bi} & sieve\\
& \textitbf{tara} & \textstyleChCharisSIL{ˈta.ɾa} & \textsc{v.mo(st)} & be matching\\
& \textitbf{tarik} & \textstyleChCharisSIL{ˈta.ɾɪk} & \textsc{v.bi} & pull\\
& \textitbf{taru} & \textstyleChCharisSIL{ˈta.ɾu} & \textsc{v.bi} & put\\
& \textitbf{tatap} & \textstyleChCharisSIL{ˈta.tɐp̚} & \textsc{v.bi} & gaze at\\
& \textitbf{taung} & \textstyleChCharisSIL{ˈta.ʊn} & \textsc{n} & year\\
& \textitbf{taw} & \textstyleChCharisSIL{ˈtɐw} & \textsc{v.bi} & know\\
& \textitbf{tawar} & \textstyleChCharisSIL{ˈta.wɐr̥} & \textsc{v.bi} & bargain\\
& \textitbf{tawong} & \textstyleChCharisSIL{ˈta.wɔ̞n} & \textsc{n} & bee\\
& \textitbf{tay} & \textstyleChCharisSIL{ˈtɐj} & \textsc{n} & excrement\\
\textstyleExampleSource{x} & \textitbf{tebal} & \textstyleChCharisSIL{tɛ.ˈbɐl} & \textsc{v.mo(st)} & be thick\\
\textstyleExampleSource{x} & \textitbf{tebang} & \textstyleChCharisSIL{tɛ.ˈbɐŋ} & \textsc{v.bi} & fell\\
\textstyleExampleSource{x} & \textitbf{tedu} & \textstyleChCharisSIL{tɛ.ˈdu} & \textsc{v.mo(st)} & be calm\\
\textstyleExampleSource{x} & \textitbf{tegang} & \textstyleChCharisSIL{tɛ.ˈgɐŋ} & \textsc{v.mo(st)} & be tight\\
\textstyleExampleSource{x} & \textitbf{tegas} & \textstyleChCharisSIL{tɛ.ˈgɐs} & \textsc{v.mo(st)} & be firm\\
\textstyleExampleSource{x} & \textitbf{tegur} & \textstyleChCharisSIL{tɛ.ˈgʊr̥} & \textsc{v.bi} & reprimand\\
\textstyleExampleSource{x} & \textitbf{tekang} & \textstyleChCharisSIL{tɛ.ˈkɐn} & \textsc{v.bi} & press\\
& \textitbf{telinga} & \textstyleChCharisSIL{tɛ.ˈli.ŋa} & \textsc{n} & ear\\
\textstyleExampleSource{x} & \textitbf{telur} & \textstyleChCharisSIL{tɛ.ˈlʊr̥} & \textsc{n} & egg\\
\textstyleExampleSource{x} & \textitbf{temang} & \textstyleChCharisSIL{tɛ.ˈmɐn} & \textsc{n} & friend\\
& \textitbf{tembak} & \textstyleChCharisSIL{ˈtɛ̞m.bɐk̚} & \textsc{v.bi} & shoot\\
\textstyleExampleSource{x} & \textitbf{tembus} & \textstyleChCharisSIL{tɛ̞m.ˈbʊs} & \textsc{v.mo(dy)} & emerge\\
\textstyleExampleSource{x} & \textitbf{tempat} & \textstyleChCharisSIL{tɛ̞m.ˈpɐt} & \textsc{n} & place\\
& \textitbf{tempe} & \textstyleChCharisSIL{ˈtɛ̞m.pɛ} & \textsc{n} & tempeh\\
& \textitbf{tenaga} & \textstyleChCharisSIL{tɛ.ˈna.ga} & \textsc{n} & energy\\
\textstyleExampleSource{x} & \textitbf{tenang} & \textstyleChCharisSIL{tɛ.ˈnɐŋ} & \textsc{v.mo(st)} & be quiet\\
\textstyleExampleSource{x} & \textitbf{tendang} & \textstyleChCharisSIL{tɛ̞n.ˈdɐŋ} & \textsc{v.bi} & kick\\
\textstyleExampleSource{x} & \textitbf{tenga} & \textstyleChCharisSIL{tɛ.ˈŋa} & \textsc{n-loc} & middle\\
\textstyleExampleSource{x} & \textitbf{tenggelam} & \textstyleChCharisSIL{ˌtɛ̞ŋ.gɛ.ˈlɐm} & \textsc{v.mo(dy)} & sink\\
\textstyleExampleSource{x} & \textitbf{tengking} & \textstyleChCharisSIL{tɛ̞ŋ.ˈkɪŋ} & \textsc{v.bi} & exorcize\\
& \textitbf{tengok} & \textstyleChCharisSIL{ˈtɛ.ŋɔ̞k} & \textsc{v.bi} & view something\\
& \textitbf{tepu} & \textstyleChCharisSIL{ˈtɛ.pu} & \textsc{v.bi} & clap\\
\textstyleExampleSource{x} & \textitbf{tepung} & \textstyleChCharisSIL{tɛ.ˈpʊŋ} & \textsc{n} & flour\\
\textstyleExampleSource{x} & \textitbf{terbang} & \textstyleChCharisSIL{tɛ̞r.ˈbɐŋ} & \textsc{v.mo(dy)} & fly\\
\textstyleExampleSource{x} & \textitbf{terbit} & \textstyleChCharisSIL{tɛ̞r.ˈbɪt} & \textsc{v.mo(dy)} & rise\\
\textstyleExampleSource{x} & \textitbf{tetap} & \textstyleChCharisSIL{tɛ.ˈtɐp̚} & \textsc{v.mo(st)} & be unchanged\\
& \textitbf{tete} & \textstyleChCharisSIL{ˈtɛ.tɛ} & \textsc{n} & grandfather\\
& \textitbf{tiang} & \textstyleChCharisSIL{ˈtɪ.ɐŋ} & \textsc{n} & pole\\
& \textitbf{tiap} & \textstyleChCharisSIL{ˈti.ɐp̚} & \textsc{qt} & every\\
& \textitbf{tiarap} & \textstyleChCharisSIL{ti.ˈa.ɾɐp̚} & \textsc{v.mo(dy)} & lie face downward\\
& \textitbf{tiba} & \textstyleChCharisSIL{ˈti.ba} & \textsc{v.mo(dy)} & arrive\\
& \textitbf{tida} & \textstyleChCharisSIL{ˈti.dɐk̚} & \textsc{adv} & \textsc{neg}\\
& \textitbf{tidor} & \textstyleChCharisSIL{ˈti.dɔ̞r̥} & \textsc{v.mo(dy)} & sleep\\
& \textitbf{tiga} & \textstyleChCharisSIL{ˈti.ga} & \textsc{num.c} & three\\
& \textitbf{tikam} & \textstyleChCharisSIL{ˈti.kɐm} & \textsc{v.bi} & stab\\
& \textitbf{tikar} & \textstyleChCharisSIL{ˈti.kɐr̥} & \textsc{n} & plaited mat\\
& \textitbf{tikung} & \textstyleChCharisSIL{ˈti.kʊŋ} & \textsc{v.mo(dy)} & curve\\
& \textitbf{tikus} & \textstyleChCharisSIL{ˈti.kʊs} & \textsc{n} & rat\\
& \textitbf{timba} & \textstyleChCharisSIL{ˈtɪm.ba} & \textsc{v.bi} & fetch\\
& \textitbf{timbang} & \textstyleChCharisSIL{ˈtɪm.bɐn} & \textsc{v.bi} & weigh\\
& \textitbf{timbul} & \textstyleChCharisSIL{ˈtɪm.bʊl} & \textsc{v.mo(dy)} & emerge\\
& \textitbf{timbung} & \textstyleChCharisSIL{ˈtɪm.bʊn} & \textsc{v.bi} & pile up\\
& \textitbf{timpang} & \textstyleChCharisSIL{ˈtɪm.pɐŋ} & \textsc{v.mo(st)} & be unbalanced\\
& \textitbf{timur} & \textstyleChCharisSIL{ˈti.mʊr} & \textsc{n} & east\\
& \textitbf{tindak} & \textstyleChCharisSIL{ˈtɪn.dɐk} & \textsc{v.bi} & act\\
& \textitbf{tindas} & \textstyleChCharisSIL{ˈtɪn.dɐs} & \textsc{v.bi} & crush\\
& \textitbf{tindis} & \textstyleChCharisSIL{ˈtɪn.dɪs} & \textsc{v.bi} & overlap\\
& \textitbf{tinggal} & \textstyleChCharisSIL{ˈtɪŋ.gɐl} & \textsc{v.mo(dy)} & stay\\
& \textitbf{tinggi} & \textstyleChCharisSIL{ˈtɪŋ.gi} & \textsc{v.mo(st)} & be high\\
& \textitbf{tingkat} & \textstyleChCharisSIL{ˈtɪŋ.kɐt} & \textsc{n} & floor\\
& \textitbf{tinju} & \textstyleChCharisSIL{ˈtɪn.dʒu} & \textsc{v.bi} & box\\
& \textitbf{tipis} & \textstyleChCharisSIL{ˈti.pɪs} & \textsc{v.mo(st)} & be thin\\
& \textitbf{tipu} & \textstyleChCharisSIL{ˈti.pu} & \textsc{v.bi} & cheat\\
& \textitbf{tiru} & \textstyleChCharisSIL{ˈti.ɾu} & \textsc{v.bi} & imitate\\
& \textitbf{titik} & \textstyleChCharisSIL{ˈtɪ.tɪk} & \textsc{n} & period\\
& \textitbf{titip} & \textstyleChCharisSIL{ˈti.tɪp} & \textsc{v.bi} & deposit\\
& \textitbf{tiup} & \textstyleChCharisSIL{ˈti.ʊp̚} & \textsc{v.bi} & blow\\
& \textitbf{tladang} & \textstyleChCharisSIL{ˈtla.dɐn} & \textsc{v.mo(st)} & be exemplary\\
& \textitbf{tlanjang} & \textstyleChCharisSIL{ˈtlɐn.dʒɐŋ} & \textsc{v.mo(st)} & be naked\\
& \textitbf{tobo} & \textstyleChCharisSIL{ˈtɔ.bɔ} & \textsc{v.mo(dy)} & dive\\
& \textitbf{toki} & \textstyleChCharisSIL{ˈtɔ.ki} & \textsc{v.bi} & beat\\
& \textitbf{tokok} & \textstyleChCharisSIL{ˈtɔ.kɔ̞k} & \textsc{v.bi} & tap\\
& \textitbf{tolak} & \textstyleChCharisSIL{ˈtɔ.lɐk} & \textsc{n} & standard\\
& \textitbf{tolak} & \textstyleChCharisSIL{ˈtɔ.lɐk} & \textsc{v.bi} & push away\\
& \textitbf{tolong} & \textstyleChCharisSIL{ˈtɔ̞.lɔ̞ŋ} & \textsc{v.bi} & help\\
& \textitbf{tombak} & \textstyleChCharisSIL{ˈtɔ̞m.bɐk̚} & \textsc{n} & spear\\
& \textitbf{tong} & \textstyleChCharisSIL{ˈtɔ̞ŋ} & \textsc{pro} & \textsc{1pl}\\
& \textitbf{tongkat} & \textstyleChCharisSIL{ˈtɔ̞ŋ.kɐt} & \textsc{n/v.bi} & cane\\
& \textitbf{torang} & \textstyleChCharisSIL{ˈtɔ.ɾɐŋ} & \textsc{pro} & \textsc{1pl}\\
& \textitbf{tra} & \textstyleChCharisSIL{ˈtra} & \textsc{adv} & \textsc{neg}\\
& \textitbf{trampil} & \textstyleChCharisSIL{ˈtɾɐm.pɪl} & \textsc{v.mo(st)} & be skilled\\
& \textitbf{trang} & \textstyleChCharisSIL{ˈtrɐŋ} & \textsc{v.mo(st)} & be clear\\
& \textitbf{trik} & \textstyleChCharisSIL{ˈtɾɪk̚} & \textsc{v.mo(st)} & be intense\\
& \textitbf{trima} & \textstyleChCharisSIL{ˈtri.ma} & \textsc{v.bi} & receive\\
& \textitbf{trus} & \textstyleChCharisSIL{ˈtɾʊs} & \textsc{v.mo(st)}

\textsc{cnj} & be continuous

next\\
& \textitbf{tua} & \textstyleChCharisSIL{ˈtʊ.a} & \textsc{v.mo(st)} & be old\\
& \textitbf{tuang} & \textstyleChCharisSIL{ˈtʊ.ɐn} & \textsc{n} & head\\
& \textitbf{tuay} & \textstyleChCharisSIL{ˈtʊ.ɐj} & \textsc{v.bi} & harvest\\
& \textitbf{tubir} & \textstyleChCharisSIL{ˈtu.bɪr} & \textsc{v.mo(st)} & be steep\\
& \textitbf{tubu} & \textstyleChCharisSIL{ˈtu.bu} & \textsc{n} & body\\
& \textitbf{tugas} & \textstyleChCharisSIL{ˈtʊ.gɐs} & \textsc{n} & duty\\
& \textitbf{tugu} & \textstyleChCharisSIL{ˈtu.gu} & \textsc{n} & monument\\
& \textitbf{Tuhang} & \textstyleChCharisSIL{ˈtʊ.hɐn} & \textsc{n} & God\\
& \textitbf{tuju} & \textstyleChCharisSIL{ˈtu.dʒu} & \textsc{num.c} & seven\\
& \textitbf{tukang} & \textstyleChCharisSIL{ˈtʊ.kɐn} & \textsc{n} & craftsman\\
& \textitbf{tukar} & \textstyleChCharisSIL{ˈtu.kɐr̥} & \textsc{v.bi} & exchange\\
& \textitbf{tulang} & \textstyleChCharisSIL{ˈtu.lɐŋ} & \textsc{n} & bone\\
& \textitbf{tuli} & \textstyleChCharisSIL{ˈtu.li} & \textsc{v.mo(st)} & be deaf\\
& \textitbf{tulis} & \textstyleChCharisSIL{ˈtu.lɪs} & \textsc{v.bi} & write\\
& \textitbf{tum} & \textstyleChCharisSIL{ˈtʊm} & \textsc{v.mo(dy)} & dive\\
& \textitbf{tumbu} & \textstyleChCharisSIL{ˈtʊm.bu} & \textsc{v.mo(dy)} & grow\\
& \textitbf{tumbuk} & \textstyleChCharisSIL{ˈtʊm.bʊk} & \textsc{v.bi} & pound\\
& \textitbf{tumis} & \textstyleChCharisSIL{ˈtu.mɪs} & \textsc{v.bi} & sauté very hot\\
& \textitbf{tumpang} & \textstyleChCharisSIL{ˈtʊm.pɐŋ} & \textsc{v.bi} & join in\\
& \textitbf{tumpuk} & \textstyleChCharisSIL{ˈtʊm.pʊk} & \textsc{v.bi} & pile\\
& \textitbf{tunda} & \textstyleChCharisSIL{ˈtʊn.da} & \textsc{v.bi} & delay\\
& \textitbf{tunduk} & \textstyleChCharisSIL{ˈtʊn.dʊk̚} & \textsc{v.mo(dy)} & bow\\
& \textitbf{tunggu} & \textstyleChCharisSIL{ˈtʊŋ.gu} & \textsc{v.bi} & wait (for)\\
& \textitbf{tunjuk} & \textstyleChCharisSIL{ˈtʊn.dʒʊk̚} & \textsc{v.bi} & show\\
& \textitbf{tuntung} & \textstyleChCharisSIL{ˈtʊn.tʊn} & \textsc{v.bi} & guide\\
& \textitbf{tuntut} & \textstyleChCharisSIL{ˈtʊn.tʊt} & \textsc{v.bi} & demand\\
& \textitbf{tupay} & \textstyleChCharisSIL{ˈtu.pɐj} & \textsc{n} & squirrel\\
& \textitbf{turung} & \textstyleChCharisSIL{ˈtʊ.ɾʊn} & \textsc{v.bi} & descend\\
& \textitbf{tusu} & \textstyleChCharisSIL{ˈtu.su} & \textsc{v.bi} & stab\\
& \textitbf{tutup} & \textstyleChCharisSIL{ˈtu.tʊp} & \textsc{v.bi} & close\\
& \textstyleChBold{U} &  &  & \\
& \textitbf{uang} & \textstyleChCharisSIL{ˈʊ.ɐŋ} & \textsc{n} & money\\
& \textitbf{uba} & \textstyleChCharisSIL{ˈu.ba} & \textsc{v.bi} & change\\
& \textitbf{ubi} & \textstyleChCharisSIL{ˈu.bi} & \textsc{n} & purple yam\\
& \textitbf{udang} & \textstyleChCharisSIL{ˈu.dɐŋ} & \textsc{n} & shrimp\\
& \textitbf{udik} & \textstyleChCharisSIL{ˈu.dɪk} & \textsc{v.bi} & observe in amazement\\
& \textitbf{uji} & \textstyleChCharisSIL{ˈu.dʒi} & \textsc{v.bi} & examine\\
& \textitbf{ujung} & \textstyleChCharisSIL{ˈu.dʒʊŋ} & \textsc{n} & end\\
& \textitbf{ukir} & \textstyleChCharisSIL{ˈu.kɪr̥} & \textsc{v.bi} & carve\\
& \textitbf{ukur} & \textstyleChCharisSIL{ˈʊ.kʊr} & \textsc{v.bi} & measure\\
& \textitbf{ulang} & \textstyleChCharisSIL{ˈu.lɐŋ} & \textsc{v.bi} & repeat\\
& \textitbf{ular} & \textstyleChCharisSIL{ˈu.lɐr̥} & \textsc{n} & snake\\
& \textitbf{umpang} & \textstyleChCharisSIL{ˈʊm.pɐn} & \textsc{v.bi} & pass ball\\
& \textitbf{undang} & \textstyleChCharisSIL{ˈʊn.dɐŋ} & \textsc{v.bi} & invite\\
& \textitbf{unggung} & \textstyleChCharisSIL{ˈʊŋ.gʊn} & \textsc{n} & campfire\\
& \textitbf{untuk} & \textstyleChCharisSIL{ˈʊn.tʊk̚} & \textsc{prep/cnj} & for\\
& \textitbf{untung} & \textstyleChCharisSIL{ˈʊn.tʊŋ} & \textsc{n} & fortune\\
& \textitbf{urat} & \textstyleChCharisSIL{ˈu.ɾɐt̚} & \textsc{n} & vein\\
& \textitbf{urus} & \textstyleChCharisSIL{ˈʊ.ɾʊs} & \textsc{v.bi} & arrange\\
& \textitbf{urut} & \textstyleChCharisSIL{ˈʊ.ɾʊt} & \textsc{v.bi} & massage\\
& \textitbf{usa} & \textstyleChCharisSIL{ˈu.sa} & \textsc{v.bi} & need to\\
& \textitbf{usir} & \textstyleChCharisSIL{ˈu.sɪr} & \textsc{v.bi} & chase away\\
& \textitbf{usus} & \textstyleChCharisSIL{ˈʊ.sʊs} & \textsc{n} & intestines\\
& \textitbf{utus} & \textstyleChCharisSIL{ˈʊ.tʊs} & \textsc{v.bi} & delegate\\
& \textstyleChBold{W} &  &  & \\
& \textitbf{wada} & \textstyleChCharisSIL{ˈwa.da} & \textsc{n} & umbrella organization\\
& \textitbf{waras} & \textstyleChCharisSIL{ˈwa.ɾɐs} & \textsc{v.mo(st)} & be sane\\
& \textitbf{warung} & \textstyleChCharisSIL{ˈwa.ɾʊŋ} & \textsc{n} & food stall\\
& \textitbf{watak} & \textstyleChCharisSIL{ˈwa.tɐk} & \textsc{n} & character\\
& \textitbf{wewenang} & \textstyleChCharisSIL{wɛ.ˈwɛ.nɐŋ} & \textsc{n} & authority\\
& \textstyleChBold{Y} &  &  & \\
& \textitbf{yahanam} & \textstyleChCharisSIL{ja.ˈha.nɐm} & \textsc{v.mo(st)} & be rebellious\\
& \textitbf{yang} & \textstyleChCharisSIL{ˈjɐŋ} & \textsc{cnj} & \textsc{rel}\\
\lspbottomrule
\end{tabular}
\subsection{Loan words}
\label{bkm:Ref376623220}
\tablehead{ & Lexeme & Transcription & Word class & English gloss\\
}
\begin{tabular}{lllll} & \textstyleChBold{A} &  &  & \\
\lsptoprule
& \textitbf{abadi} & \textstyleChCharisSIL{a.ˈba.di} & \textsc{v.mo(st)} & be eternal\\
& \textitbf{acara} & \textstyleChCharisSIL{a.ˈtʃa.ɾa} & \textsc{n} & ceremony\\
& \textitbf{adat} & \textstyleChCharisSIL{ˈa.dɐt} & \textsc{n} & tradition\\
& \textitbf{adil} & \textstyleChCharisSIL{ˈa.dɪl} & \textsc{v.mo(st)} & be fair\\
& \textitbf{adopsi} & \textstyleChCharisSIL{a.ˈdɔ̞p̚.si} & \textsc{n} & adoption\\
& \textitbf{agama} & \textstyleChCharisSIL{a.ˈga.ma} & \textsc{n} & religion\\
& \textitbf{agenda} & \textstyleChCharisSIL{a.ˈgɛ̞n.da} & \textsc{n} & agenda\\
\textstyleExampleSource{x} & \textitbf{agraria} & \textstyleChCharisSIL{a.ˈgɾa.ɾi.ˌa} & \textsc{n} & agrarian affairs\\
& \textitbf{ajaip} & \textstyleChCharisSIL{a.ˈdʒa.ɪp̚} & \textsc{v.mo(st)} & be miraculous\\
& \textitbf{akal} & \textstyleChCharisSIL{ˈa.kɐl} & \textsc{n} & reason\\
& \textitbf{akibat} & \textstyleChCharisSIL{a.ˈki.bɐt} & \textsc{n} & consequence\\
& \textitbf{akta} & \textstyleChCharisSIL{ˈɐk̚.ta} & \textsc{n} & certificate\\
& \textitbf{aktif} & \textstyleChCharisSIL{ˈɐk̚.tɪf} & \textsc{v.mo(st)} & be active\\
& \textitbf{alam} & \textstyleChCharisSIL{ˈa.lɐm} & \textsc{n} & world\\
& \textitbf{alamat} & \textstyleChCharisSIL{a.ˈla.mɐt} & \textsc{n} & address\\
& \textitbf{alat} & \textstyleChCharisSIL{ˈa.lɐt} & \textsc{n} & equipment\\
& \textitbf{alergi} & \textstyleChCharisSIL{a.ˈlɛ̞r.gi} & \textsc{v.mo(st)} & be allergic\\
& \textitbf{alkitap} & \textstyleChCharisSIL{ɐl.ˈki.tɐp̚} & \textsc{n} & Bible\\
& \textitbf{alpa} & \textstyleChCharisSIL{ˈɐl.pa} & \textsc{v.mo(st)} & be absent\\
& \textitbf{alumni} & \textstyleChCharisSIL{a.ˈlʊm.ni} & \textsc{n} & alumnus\\
& \textitbf{amang} & \textstyleChCharisSIL{ˈa.mɐn} & \textsc{v.mo(st)} & be safe\\
& \textitbf{amplop} & \textstyleChCharisSIL{ˈɐm.plɔ̞p} & \textsc{n} & envelope\\
& \textitbf{anggota} & \textstyleChCharisSIL{ɐŋ.ˈgɔ.ta} & \textsc{n} & member\\
& \textitbf{antifirus} & \textstyleChCharisSIL{ˌɐn.ti.ˈfi.ɾʊs} & \textsc{n} & antivirus\\
& \textitbf{aparat} & \textstyleChCharisSIL{a.ˈpa.ɾɐt} & \textsc{n} & apparatus\\
& \textitbf{apel} & \textstyleChCharisSIL{ˈa.pɛ̞l} & \textsc{n} & apple\\
& \textitbf{apsen} & \textstyleChCharisSIL{ˈɐp̚.sɛ̞n} & \textsc{v.mo(st)} & be absent\\
& \textitbf{aroa} & \textstyleChCharisSIL{a.ˈɾɔ̞.a} & \textsc{n} & departed spirit\\
& \textitbf{arti} & \textstyleChCharisSIL{ˈɐr.ti} & \textsc{n} & meaning\\
& \textitbf{asal} & \textstyleChCharisSIL{ˈa.sɐl} & \textsc{n} & origin\\
& \textitbf{asar} & \textstyleChCharisSIL{ˈa.sɐr̥} & \textsc{v.bi} & smoke\\
& \textitbf{asik} & \textstyleChCharisSIL{ˈa.sɪk̚} & \textsc{v.mo(st)} & be passionate\\
& \textitbf{asli} & \textstyleChCharisSIL{ˈɐs.li} & \textsc{v.mo(st)} & be original\\
& \textitbf{aspal} & \textstyleChCharisSIL{ˈɐs.pɐl} & \textsc{n} & asphalt\\
& \textitbf{asrama} & \textstyleChCharisSIL{ɐs.ˈra.ma} & \textsc{n} & dormitory\\
& \textitbf{astronomi} & \textstyleChCharisSIL{ˌɐs.trɔ.ˈnɔ.mi} & \textsc{n} & astronomy\\
\textstyleExampleSource{x} & \textitbf{ato} & \textstyleChCharisSIL{a.ˈtɔ} & \textsc{cnj} & or\\
& \textitbf{ayat} & \textstyleChCharisSIL{ˈa.jɐt} & \textsc{n} & verse\\
& \textstyleChBold{B} &  &  & \\
& \textitbf{baca} & \textstyleChCharisSIL{ˈba.tʃa} & \textsc{v.bi} & read\\
& \textitbf{badang} & \textstyleChCharisSIL{ˈba.dɐn} & \textsc{n} & body\\
& \textitbf{bagi} & \textstyleChCharisSIL{ˈba.gi} & \textsc{v.bi} & divide\\
& \textitbf{bahas} & \textstyleChCharisSIL{ˈba.hɐs} & \textsc{v.bi} & discuss\\
& \textitbf{bahasa} & \textstyleChCharisSIL{ba.ˈha.sa} & \textsc{n} & language\\
& \textitbf{bahaya} & \textstyleChCharisSIL{ba.ˈha.ja} & \textsc{v.mo(st)} & be dangerous\\
& \textitbf{bahu} & \textstyleChCharisSIL{ˈba.hʊ} & \textsc{n} & shoulder\\
& \textitbf{bahwa} & \textstyleChCharisSIL{ˈbɐh.wa} & \textsc{cnj} & that\\
& \textitbf{baju} & \textstyleChCharisSIL{ˈba.dʒʊ} & \textsc{n} & shirt\\
& \textitbf{bak} & \textstyleChCharisSIL{ˈbɐk} & \textsc{n} & basin\\
& \textitbf{ban} & \textstyleChCharisSIL{ˈbɐn} & \textsc{n} & tire\\
& \textitbf{bangku} & \textstyleChCharisSIL{ˈbɐŋ.kʊ} & \textsc{n} & bench\\
& \textitbf{bangsa} & \textstyleChCharisSIL{ˈbɐŋ.sa} & \textsc{n} & people group\\
& \textitbf{bangsat} & \textstyleChCharisSIL{ˈbɐŋ.sɐt} & \textsc{n} & rascal\\
& \textitbf{baptis} & \textstyleChCharisSIL{ˈbɐp̚.tɪs} & \textsc{v.bi} & baptize\\
& \textitbf{barat} & \textstyleChCharisSIL{ˈba.ɾɐt} & \textsc{n} & west\\
& \textitbf{batal} & \textstyleChCharisSIL{ˈba.tɐl} & \textsc{v.bi} & cancel\\
\textstyleExampleSource{x} & \textitbf{batrey} & \textstyleChCharisSIL{ba.ˈtɾɛ̞j} & \textsc{n} & battery\\
& \textitbf{baw} & \textstyleChCharisSIL{ˈbɐw} & \textsc{n} & smell\\
& \textitbf{beda} & \textstyleChCharisSIL{ˈbɛ.da} & \textsc{v.mo(st)} & be different\\
& \textitbf{bel} & \textstyleChCharisSIL{ˈbɛ̞l} & \textsc{v.bi} & ring\\
& \textitbf{bendera} & \textstyleChCharisSIL{bɛ̞n.ˈdɛ̞.ɾa} & \textsc{n} & flag\\
& \textitbf{bengkel} & \textstyleChCharisSIL{ˈbɛ̞ŋ.kɛ̞l} & \textsc{n} & repair shop\\
\textstyleExampleSource{x} & \textitbf{bensing} & \textstyleChCharisSIL{bɛ̞n.ˈsɪn} & \textsc{n} & gasoline\\
& \textitbf{berkat} & \textstyleChCharisSIL{ˈbɛ̞r.kɐt} & \textsc{n} & blessing\\
& \textitbf{biasa} & \textstyleChCharisSIL{bi.ˈa.sa} & \textsc{v.mo(st)} & be usual\\
& \textitbf{biaya} & \textstyleChCharisSIL{bi.ˈa.ja} & \textsc{n/v.bi} & pay\\
& \textitbf{bicara} & \textstyleChCharisSIL{bi.ˈtʃa.ɾa} & \textsc{v.bi} & speak\\
& \textitbf{bijaksana} & \textstyleChCharisSIL{ˌbi.dʒɐk̚.ˈsa.na} & \textsc{v.mo(st)} & be wise\\
& \textitbf{biji} & \textstyleChCharisSIL{ˈbi.dʒi} & \textsc{n} & seed\\
& \textitbf{biodata} & \textstyleChCharisSIL{ˌbi.ɔ.ˈda.ta} & \textsc{n} & biodata\\
& \textitbf{biologi} & \textstyleChCharisSIL{ˌbi.ɔ.ˈlɔ.gi} & \textsc{n} & biology\\
& \textitbf{bis} & \textstyleChCharisSIL{ˈbɪs} & \textsc{n} & bus\\
\textstyleExampleSource{x} & \textitbf{biskwit} & \textstyleChCharisSIL{bɪs.ˈkwɪt} & \textsc{n} & cracker\\
& \textitbf{bisnis} & \textstyleChCharisSIL{ˈbɪs.nɪs} & \textsc{n} & business\\
\textstyleExampleSource{x} & \textitbf{bistir} & \textstyleChCharisSIL{bɪs.ˈtɪr̥} & \textsc{n} & subdistrict head\\
& \textitbf{bola} & \textstyleChCharisSIL{ˈbɔ.la} & \textsc{n} & ball\\
& \textitbf{bolpen} & \textstyleChCharisSIL{ˈbɔ̞l.pɛ̞n} & \textsc{n} & ballpoint pen\\
& \textitbf{boneka} & \textstyleChCharisSIL{bɔ.ˈnɛ.ka} & \textsc{n} & doll\\
& \textitbf{bos} & \textstyleChCharisSIL{ˈbɔ̞s} & \textsc{n} & boss\\
& \textitbf{botol} & \textstyleChCharisSIL{ˈbɔ̞.tɔ̞l} & \textsc{n} & bottle\\
& \textitbf{brita} & \textstyleChCharisSIL{ˈbri.ta} & \textsc{n} & news\\
& \textitbf{budaya} & \textstyleChCharisSIL{bʊ.ˈda.ja} & \textsc{n} & culture\\
& \textitbf{bukti} & \textstyleChCharisSIL{ˈbʊk̚.ti} & \textsc{n} & proof\\
& \textitbf{buku} & \textstyleChCharisSIL{ˈbu.ku} & \textsc{n} & book\\
& \textitbf{bumi} & \textstyleChCharisSIL{ˈbu.mi} & \textsc{n} & earth\\
& \textitbf{bupati} & \textstyleChCharisSIL{bu.ˈpa.ti} & \textsc{n} & regent\\
& \textitbf{busi} & \textstyleChCharisSIL{ˈbu.si} & \textsc{n} & sparkplug\\
& \textstyleChBold{C} &  &  & \\
& \textitbf{cahaya} & \textstyleChCharisSIL{tʃa.ˈha.ja} & \textsc{n} & glow\\
& \textitbf{cap} & \textstyleChCharisSIL{ˈtʃɐp̚} & \textsc{n} & stamp\\
& \textitbf{cara} & \textstyleChCharisSIL{ˈtʃa.ɾa} & \textsc{n} & manner\\
& \textitbf{cari} & \textstyleChCharisSIL{ˈtʃa.ɾi} & \textsc{v.bi} & search\\
& \textitbf{cek} & \textstyleChCharisSIL{ˈtʃɛ̞k} & \textsc{v.bi} & check\\
& \textitbf{celana} & \textstyleChCharisSIL{tʃɛ.ˈla.na} & \textsc{n} & trousers\\
& \textitbf{cemara} & \textstyleChCharisSIL{tʃɛ.ˈma.ɾa} & \textsc{n} & casuarina tree\\
& \textitbf{ceria} & \textstyleChCharisSIL{tʃɛ.ˈɾi.a} & \textsc{v.mo(st)} & be cheerful\\
& \textitbf{ceritra} & \textstyleChCharisSIL{tʃɛ.ˈɾi.tra} & \textsc{v.tri} & tell\\
& \textitbf{cinta} & \textstyleChCharisSIL{ˈtʃɪn.ta} & \textsc{n} & love\\
& \textitbf{coklat} & \textstyleChCharisSIL{ˈtʃɔ̞k̚.lɐt} & \textsc{n} & chocolate\\
& \textitbf{cuci} & \textstyleChCharisSIL{ˈtʃu.tʃi} & \textsc{v.bi} & wash\\
& \textitbf{cuma} & \textstyleChCharisSIL{ˈtʃu.ma} & \textsc{adv} & just\\
& \textitbf{curi} & \textstyleChCharisSIL{ˈtʃu.ɾi} & \textsc{v.bi} & steal\\
& \textitbf{cuti} & \textstyleChCharisSIL{ˈtʃu.ti} & \textsc{v.mo(dy)} & take leave\\
& \textstyleChBold{D} &  &  & \\
& \textitbf{daera} & \textstyleChCharisSIL{da.ˈɛ̞.ɾa} & \textsc{n} & area\\
& \textitbf{daftar} & \textstyleChCharisSIL{ˈdɐf.tɐr̥} & \textsc{n/v.bi} & list\\
& \textitbf{dana} & \textstyleChCharisSIL{ˈda.na} & \textsc{n} & fund\\
& \textitbf{dansa} & \textstyleChCharisSIL{ˈdɐn.sa} & \textsc{v.mo(dy)} & dance\\
& \textitbf{dasyat} & \textstyleChCharisSIL{ˈda.sʲɐt} & \textsc{v.mo(st)} & be terrifying\\
& \textitbf{daya} & \textstyleChCharisSIL{ˈda.ja} & \textsc{n} & energy\\
\textstyleExampleSource{x} & \textitbf{debat} & \textstyleChCharisSIL{dɛ.ˈbɐt} & \textsc{n} & debate\\
& \textitbf{debel} & \textstyleChCharisSIL{ˈdɛ̞.bɛ̞l} & \textsc{v.bi} & double\\
& \textitbf{delegasi} & \textstyleChCharisSIL{ˌdɛ.lɛ.ˈga.si} & \textsc{n} & delegation\\
& \textitbf{demo} & \textstyleChCharisSIL{ˈdɛ.mɔ} & \textsc{n} & demonstration\\
& \textitbf{desember} & \textstyleChCharisSIL{dɛ.ˈsɛ̞m.bɛ̞r̥} & \textsc{n} & December\\
& \textitbf{dewan} & \textstyleChCharisSIL{ˈdɛ.wɐn} & \textsc{n} & council (member)\\
& \textitbf{dewasa} & \textstyleChCharisSIL{dɛ.ˈwa.sa} & \textsc{n} & adult\\
& \textitbf{dialek} & \textstyleChCharisSIL{dɪ.ˈa.lɛ̞k} & \textsc{n} & dialect\\
& \textitbf{dikta} & \textstyleChCharisSIL{ˈdɪk̚.ta} & \textsc{n} & written lectures summary\\
& \textitbf{dinas} & \textstyleChCharisSIL{ˈdi.nɐs} & \textsc{n} & department\\
& \textitbf{disain} & \textstyleChCharisSIL{di.ˈsa.ɪn} & \textsc{n} & design\\
& \textitbf{disko} & \textstyleChCharisSIL{ˈdɪs.kɔ} & \textsc{n} & discotheque\\
& \textitbf{distrik} & \textstyleChCharisSIL{ˈdɪs.trɪk} & \textsc{n} & district\\
& \textitbf{doa} & \textstyleChCharisSIL{ˈdɔ.a} & \textsc{n} & prayer\\
& \textitbf{dobrak} & \textstyleChCharisSIL{ˈdɔ.brɐk̚} & \textsc{v.bi} & smash\\
& \textitbf{dokter} & \textstyleChCharisSIL{ˈdɔ̞k̚.tɛ̞r̥} & \textsc{n} & doctor\\
\textstyleExampleSource{x} & \textitbf{dokumen} & \textstyleChCharisSIL{ˌdɔ.ku.ˈmɛ̞n} & \textsc{n} & document\\
& \textitbf{dol} & \textstyleChCharisSIL{ˈdɔ̞l} & \textsc{v.mo(st)} & be damaged\\
& \textitbf{domba} & \textstyleChCharisSIL{ˈdɔ̞m.ba} & \textsc{n} & sheep\\
\textstyleExampleSource{x} & \textitbf{dominan} & \textstyleChCharisSIL{ˌdɔ.mi.ˈnɐn} & \textsc{v.mo(st)} & dominate, master\\
& \textitbf{dompet} & \textstyleChCharisSIL{ˈdɔ̞m.pɛ̞t} & \textsc{n} & wallet\\
& \textitbf{donat} & \textstyleChCharisSIL{ˈdɔ.nɐt} & \textsc{n} & doughnut\\
& \textitbf{dosa} & \textstyleChCharisSIL{ˈdɔ.sa} & \textsc{n} & sin\\
\textstyleExampleSource{x} & \textitbf{doseng} & \textstyleChCharisSIL{dɔ.ˈsɛ̞n} & \textsc{n} & lecturer\\
& \textitbf{doser} & \textstyleChCharisSIL{ˈdɔ.sɛ̞r̥} & \textsc{n} & bulldozer\\
& \textitbf{dramben} & \textstyleChCharisSIL{ˈdrɐm.bɛ̞n} & \textsc{n} & marching band\\
& \textitbf{drop} & \textstyleChCharisSIL{ˈdrɔ̞p̚} & \textsc{v.bi} & drop\\
& \textitbf{drum} & \textstyleChCharisSIL{ˈdrʊm} & \textsc{n} & drum\\
& \textitbf{duka} & \textstyleChCharisSIL{ˈdu.ka} & \textsc{n} & grief\\
& \textitbf{dunia} & \textstyleChCharisSIL{du.ˈni.a} & \textsc{n} & world\\
& \textstyleChBold{E} &  &  & \\
& \textitbf{egois} & \textstyleChCharisSIL{ɛ.ˈgɔ.ɪs} & \textsc{v.mo(st)} & be egoistic\\
& \textitbf{ekstra} & \textstyleChCharisSIL{ˈɛ̞k̚.stɾa} & \textsc{v.mo(st)} & be extra\\
& \textitbf{emansipasi} & \textstyleChCharisSIL{ɛ.ˌmɐn.si.ˈpa.si} & \textsc{n} & emancipation\\
& \textitbf{ember} & \textstyleChCharisSIL{ˈɛ̞m.bɛ̞r̥} & \textsc{n} & bucket\\
& \textitbf{emosi} & \textstyleChCharisSIL{ɛ.ˈmɔ.si} & \textsc{v.bi} & feel angry (about)\\
& \textitbf{erport} & \textstyleChCharisSIL{ˈɛ̞r.pɔ̞rt} & \textsc{n} & airport\\
& \textitbf{etnis} & \textstyleChCharisSIL{ˈɛ̞t̚.nɪs} & \textsc{n} & ethnic\\
& \textstyleChBold{F} &  &  & \\
& \textitbf{fajar} & \textstyleChCharisSIL{ˈfa.dʒɐr̥} & \textsc{n} & dawn\\
& \textitbf{fam} & \textstyleChCharisSIL{ˈfɐm} & \textsc{n} & family name\\
\textstyleExampleSource{x} & \textitbf{fasilitas} & \textstyleChCharisSIL{fa.ˌsi.li.ˈtɐs} & \textsc{n} & facility\\
& \textitbf{februari} & \textstyleChCharisSIL{ˌfɛ.brʊ.ˈa.ɾi} & \textsc{n} & February\\
& \textitbf{federasi} & \textstyleChCharisSIL{ˌfɛ.dɛ.ˈɾa.si} & \textsc{n} & federation\\
\textstyleExampleSource{x} & \textitbf{ferban} & \textstyleChCharisSIL{fɛ̞r.ˈbɐn} & \textsc{n} & bandage\\
& \textitbf{fetsin} & \textstyleChCharisSIL{ˈfɛ̞t̚.sɪn} & \textsc{n} & flavoring spice\\
& \textitbf{fideo} & \textstyleChCharisSIL{fi.ˈdɛ.ɔ} & \textsc{n} & video\\
& \textitbf{figur} & \textstyleChCharisSIL{ˈfi.gʊr̥} & \textsc{n} & figure\\
& \textitbf{filem} & \textstyleChCharisSIL{ˈfɪ.lɛ̞m} & \textsc{n} & film\\
& \textitbf{final} & \textstyleChCharisSIL{ˈfi.nɐl} & \textsc{n} & finals\\
& \textitbf{firman} & \textstyleChCharisSIL{ˈfɪr.mɐn} & \textsc{n} & divine saying\\
& \textitbf{fit} & \textstyleChCharisSIL{ˈfɪt} & \textsc{n} & Fit-drinking water\\
\textstyleExampleSource{x} & \textitbf{fitamin} & \textstyleChCharisSIL{ˌfi.ta.ˈmɪn} & \textsc{n} & vitamins\\
& \textitbf{fokal} & \textstyleChCharisSIL{ˈfɔ.kɐl} & \textsc{n} & song\\
& \textitbf{fol} & \textstyleChCharisSIL{ˈfɔ̞l} & \textsc{v.mo(st)} & be full\\
& \textitbf{foli} & \textstyleChCharisSIL{ˈfɔ.li} & \textsc{n} & volleyball\\
& \textitbf{fondasi} & \textstyleChCharisSIL{fɔ̞n.ˈda.si} & \textsc{n} & foundation\\
& \textitbf{formasi} & \textstyleChCharisSIL{fɔ̞r.ˈma.si} & \textsc{n} & formation\\
& \textitbf{formulir} & \textstyleChCharisSIL{fɔ̞r.ˈmu.lɪr̥} & \textsc{n} & form\\
& \textitbf{forum} & \textstyleChCharisSIL{ˈfɔ.ɾʊm} & \textsc{n} & forum\\
& \textitbf{foto} & \textstyleChCharisSIL{ˈfɔ.tɔ} & \textsc{n} & photograph\\
& \textitbf{fotokopi} & \textstyleChCharisSIL{ˌfɔ.tɔ.ˈkɔ.pi} & \textsc{n} & photocopy\\
& \textitbf{frey} & \textstyleChCharisSIL{ˈfrɛ̞j} & \textsc{v.mo(st)} & be blank\\
& \textitbf{fungsi} & \textstyleChCharisSIL{ˈfʊŋ.si} & \textsc{n} & function\\
& \textstyleChBold{G} &  &  & \\
& \textitbf{gaja} & \textstyleChCharisSIL{ˈga.dʒa} & \textsc{n} & elephant\\
& \textitbf{gaji} & \textstyleChCharisSIL{ˈga.dʒi} & \textsc{n} & salary\\
& \textitbf{gas} & \textstyleChCharisSIL{ˈgɐs} & \textsc{n} & gas\\
& \textitbf{gembala} & \textstyleChCharisSIL{gɛ̞m.ˈba.la} & \textsc{n} & pastor\\
& \textitbf{generasi} & \textstyleChCharisSIL{ˌgɛ.nɛ.ˈɾa.si} & \textsc{n} & generation\\
& \textitbf{gergaji} & \textstyleChCharisSIL{gɛ̞r.ˈga.dʒi} & \textsc{n} & saw\\
& \textitbf{gisi} & \textstyleChCharisSIL{ˈgi.si} & \textsc{n} & nutrient\\
\textstyleExampleSource{x} & \textitbf{gitar} & \textstyleChCharisSIL{gi.ˈtɐr̥} & \textsc{n} & guitar\\
& \textitbf{glas} & \textstyleChCharisSIL{ˈglɐs} & \textsc{n} & glass\\
& \textitbf{glojo} & \textstyleChCharisSIL{ˈglɔ.dʒɔ} & \textsc{v.bi} & crave\\
& \textitbf{got} & \textstyleChCharisSIL{ˈgɔ̞t} & \textsc{n} & gutter\\
& \textitbf{gratis} & \textstyleChCharisSIL{ˈgɾa.tɪs} & \textsc{v.mo(st)} & be gratis\\
& \textitbf{greja} & \textstyleChCharisSIL{ˈgrɛ.dʒa} & \textsc{n} & church\\
& \textitbf{grobak} & \textstyleChCharisSIL{ˈgrɔ.bɐk} & \textsc{n} & wheelbarrow\\
& \textitbf{grup} & \textstyleChCharisSIL{ˈgrʊp} & \textsc{n} & group\\
\textstyleExampleSource{x} & \textitbf{gubernur} & \textstyleChCharisSIL{ˌgu.bɛ̞r.ˈnʊr̥} & \textsc{n} & governor\\
& \textitbf{gudang} & \textstyleChCharisSIL{ˈgʊ.dɐŋ} & \textsc{n} & storeroom\\
& \textitbf{gula} & \textstyleChCharisSIL{ˈgu.la} & \textsc{n} & sugar\\
& \textitbf{guru} & \textstyleChCharisSIL{ˈgu.ɾu} & \textsc{n} & teacher\\
& \textstyleChBold{H} &  &  & \\
& \textitbf{hadia} & \textstyleChCharisSIL{ha.ˈdɪ.a} & \textsc{n} & gift\\
& \textitbf{hadir} & \textstyleChCharisSIL{ˈha.dɪr̥} & \textsc{v.mo(dy)} & attend\\
& \textitbf{hafal} & \textstyleChCharisSIL{ˈha.fɐl} & \textsc{v.bi} & memorize\\
& \textitbf{hak} & \textstyleChCharisSIL{ˈhɐk} & \textsc{n} & right\\
& \textitbf{hal} & \textstyleChCharisSIL{ˈhɐl} & \textsc{n} & thing\\
\textstyleExampleSource{x} & \textitbf{halal} & \textstyleChCharisSIL{ha.ˈlɐl} & \textsc{v.mo(st)} & be permitted\\
& \textitbf{hamil} & \textstyleChCharisSIL{ˈha.mɪl} & \textsc{v.mo(st)} & be pregnant\\
& \textitbf{handuk} & \textstyleChCharisSIL{ˈhɐn.dʊk} & \textsc{n} & towel\\
& \textitbf{hantu} & \textstyleChCharisSIL{ˈhɐn.tu} & \textsc{n} & ghost\\
& \textitbf{harga} & \textstyleChCharisSIL{ˈhɐr.ga} & \textsc{n} & price\\
& \textitbf{hari} & \textstyleChCharisSIL{ˈha.ɾi} & \textsc{n} & day\\
& \textitbf{harta} & \textstyleChCharisSIL{ˈhɐr.ta} & \textsc{n} & wealth\\
& \textitbf{hasil} & \textstyleChCharisSIL{ˈha.sɪl} & \textsc{n} & product\\
& \textitbf{hebat} & \textstyleChCharisSIL{ˈhɛ.bɐt̚} & \textsc{v.mo(st)} & be great\\
& \textitbf{heking} & \textstyleChCharisSIL{ˈhɛ.kɪŋ} & \textsc{v.mo(dy)} & hiking\\
& \textitbf{hektar} & \textstyleChCharisSIL{ˈhɛ̞k̚.tɐr} & \textsc{n} & hectare\\
& \textitbf{helem} & \textstyleChCharisSIL{ˈhɛ.lɛ̞m} & \textsc{n} & helmet\\
& \textitbf{helikopter} & \textstyleChCharisSIL{ˌhɛ.li.ˈkɔ̞p̚.tɛ̞r} & \textsc{n} & helicopter\\
& \textitbf{hemat} & \textstyleChCharisSIL{ˈhɛ.mɐt} & \textsc{v.bi} & economize\\
& \textitbf{herang} & \textstyleChCharisSIL{ˈhɛ.ɾɐn} & \textsc{v.bi} & feel surprised (about)\\
& \textitbf{hikmat} & \textstyleChCharisSIL{ˈhɪk̚.mɐt̚} & \textsc{n} & wisdom\\
& \textitbf{hina} & \textstyleChCharisSIL{ˈhi.na} & \textsc{v.bi} & humiliate\\
& \textitbf{hobi} & \textstyleChCharisSIL{ˈhɔ.bi} & \textsc{n} & hobby\\
\textstyleExampleSource{x} & \textitbf{honay} & \textstyleChCharisSIL{hɔ.ˈnɐj} & \textsc{n} & traditional Dani hut\\
& \textitbf{honor} & \textstyleChCharisSIL{ˈhɔ̞.nɔ̞r} & \textsc{n} & honorarium\\
\textstyleExampleSource{x} & \textitbf{honorer} & \textstyleChCharisSIL{ˌhɔ.nɔ.ˈɾɛ̞r} & \textsc{v.mo(st)} & be honorary\\
& \textitbf{hordeng} & \textstyleChCharisSIL{ˈhɔ̞r.dɛ̞ŋ} & \textsc{n} & curtains\\
& \textitbf{hormat} & \textstyleChCharisSIL{ˈhɔ̞r.mɐt} & \textsc{v.bi} & respect\\
& \textitbf{hotba} & \textstyleChCharisSIL{ˈhɔ̞t̚.ba} & \textsc{v.bi} & preach\\
\textstyleExampleSource{x} & \textitbf{hotel} & \textstyleChCharisSIL{hɔ.ˈtɛ̞l} & \textsc{n} & hotel\\
& \textitbf{hukum} & \textstyleChCharisSIL{ˈhu.kʊm} & \textsc{n} & law\\
& \textitbf{humur} & \textstyleChCharisSIL{ˈhʊ.mʊr} & \textsc{n} & joke\\
& \textstyleChBold{I} &  &  & \\
& \textitbf{ibada} & \textstyleChCharisSIL{i.ˈba.da} & \textsc{v.mo(dy)} & worship\\
& \textitbf{iblis} & \textstyleChCharisSIL{ˈɪ.blɪs} & \textsc{n} & devil\\
& \textitbf{ide} & \textstyleChCharisSIL{ˈi.dɛ} & \textsc{n} & idea\\
& \textitbf{ijasa} & \textstyleChCharisSIL{i.ˈdʒa.sa} & \textsc{n} & diploma\\
& \textitbf{ijing} & \textstyleChCharisSIL{ˈi.dʒɪn} & \textsc{n} & permission\\
& \textitbf{ilmu} & \textstyleChCharisSIL{ˈɪl.mu} & \textsc{n} & knowledge\\
\textstyleExampleSource{x} & \textitbf{iman} & \textstyleChCharisSIL{i.ˈmɐn} & \textsc{n} & faith\\
& \textitbf{informasi} & \textstyleChCharisSIL{ˌɪn.fɔ̞r.ˈma.si} & \textsc{n} & information\\
& \textitbf{infus} & \textstyleChCharisSIL{ˈɪn.fʊs} & \textsc{n} & give an infusion\\
& \textitbf{injil} & \textstyleChCharisSIL{ˈɪn.dʒɪl} & \textsc{n} & Gospel\\
& \textitbf{insentif} & \textstyleChCharisSIL{ɪn.ˈsɛ̞n.tɪf} & \textsc{n} & incentive\\
\textstyleExampleSource{x} & \textitbf{insinyur} & \textstyleChCharisSIL{ˌɪn.si.ˈɲʊr} & \textsc{n} & engineer\\
& \textitbf{instansi} & \textstyleChCharisSIL{ɪn.ˈstɐn.si} & \textsc{n} & level\\
& \textitbf{intel} & \textstyleChCharisSIL{ˈɪn.tɛ̞l} & \textsc{n} & intelligence service\\
& \textitbf{intro} & \textstyleChCharisSIL{ˈɪn.trɔ} & \textsc{v.bi} & play musical introduction\\
& \textitbf{istila} & \textstyleChCharisSIL{ɪs.ˈtɪ.la} & \textsc{n} & term\\
& \textitbf{istimewa} & \textstyleChCharisSIL{ˌɪs.ti.ˈmɛ.wa} & \textsc{v.mo(st)} & be special\\
\textstyleExampleSource{x} & \textitbf{istirahat} & \textstyleChCharisSIL{ˈɪs.ti.ˈɾa.hɐt̚} & \textsc{v.mo(dy)} & rest\\
& \textitbf{istri} & \textstyleChCharisSIL{ˈɪs.tri} & \textsc{n} & wife\\
& \textstyleChBold{J} &  &  & \\
& \textitbf{jaga} & \textstyleChCharisSIL{ˈdʒa.ga} & \textsc{v.bi} & guard\\
& \textitbf{jam} & \textstyleChCharisSIL{ˈdʒɐm} & \textsc{n} & hour\\
& \textitbf{jaman} & \textstyleChCharisSIL{ˈdʒa.mɐn} & \textsc{n} & period\\
& \textitbf{jambu} & \textstyleChCharisSIL{ˈdʒɐm.bu} & \textsc{n} & rose apple\\
& \textitbf{janda} & \textstyleChCharisSIL{ˈdʒɐn.da} & \textsc{n} & widow\\
& \textitbf{januari} & \textstyleChCharisSIL{ˌdʒa.nʊ.ˈa.ɾi} & \textsc{n} & January\\
& \textitbf{jatwal} & \textstyleChCharisSIL{ˈdʒɐt̚.wɐl} & \textsc{n} & schedule\\
& \textitbf{jawap} & \textstyleChCharisSIL{ˈdʒa.wɐp} & \textsc{v.bi} & answer\\
& \textitbf{jeket} & \textstyleChCharisSIL{ˈdʒɛ̞.kɛ̞t̚} & \textsc{n} & jacket\\
& \textitbf{jemaat} & \textstyleChCharisSIL{dʒɛ.ˈma.ɐt̚} & \textsc{n} & congregation\\
& \textitbf{jenasa} & \textstyleChCharisSIL{dʒɛ.ˈna.sa} & \textsc{n} & corpse\\
& \textitbf{jendela} & \textstyleChCharisSIL{dʒɛ̞n.ˈdɛ̞.la} & \textsc{n} & window\\
\textstyleExampleSource{x} & \textitbf{jenis} & \textstyleChCharisSIL{dʒɛ.ˈnɪs} & \textsc{n} & kind\\
\textstyleExampleSource{x} & \textitbf{jeriken} & \textstyleChCharisSIL{ˈdʒɛ.ɾi.ˌkɛ̞n} & \textsc{n} & jerry can\\
& \textitbf{jing} & \textstyleChCharisSIL{ˈdʒɪn} & \textsc{n} & genie\\
& \textitbf{jiwa} & \textstyleChCharisSIL{ˈdʒi.wa} & \textsc{n} & soul\\
& \textitbf{jonson} & \textstyleChCharisSIL{ˈdʒɔ̞n.sɔ̞n} & \textsc{n} & motorboat\\
& \textitbf{jumat} & \textstyleChCharisSIL{ˈdʒu.mɐt̚} & \textsc{n} & Friday\\
& \textitbf{jumla} & \textstyleChCharisSIL{ˈdʒʊm.la} & \textsc{n} & sum\\
& \textitbf{justru} & \textstyleChCharisSIL{ˈdʒʊs.tɾu} & \textsc{adv} & precisely\\
& \textitbf{juta} & \textstyleChCharisSIL{ˈdʒu.ta} & \textsc{num.c} & million\\
& \textstyleChBold{K} &  &  & \\
& \textitbf{kabul} & \textstyleChCharisSIL{ˈka.bʊl} & \textsc{v.bi} & grant\\
& \textitbf{kaca} & \textstyleChCharisSIL{ˈka.tʃa} & \textsc{n} & glass\\
& \textitbf{kader} & \textstyleChCharisSIL{ˈka.dɛ̞r} & \textsc{n} & cadre\\
& \textitbf{kafir} & \textstyleChCharisSIL{ˈka.fɪr} & \textsc{n} & unbeliever\\
\textstyleExampleSource{x} & \textitbf{kakaw} & \textstyleChCharisSIL{ka.ˈkɐw} & \textsc{n} & cacao\\
& \textitbf{kalender} & \textstyleChCharisSIL{ka.ˈlɛ̞n.dɛ̞r} & \textsc{n} & calendar\\
& \textitbf{kali} & \textstyleChCharisSIL{ˈka.li} & \textsc{n} & time\\
& \textitbf{kalo} & \textstyleChCharisSIL{ˈka.lɔ̞} & \textsc{cnj} & if\\
& \textitbf{kamar} & \textstyleChCharisSIL{ˈka.mɐr} & \textsc{n} & room\\
& \textitbf{kamis} & \textstyleChCharisSIL{ˈka.mɪs} & \textsc{n} & Thursday\\
& \textitbf{kampus} & \textstyleChCharisSIL{ˈkɐm.pʊs} & \textsc{n} & campus\\
& \textitbf{kantong} & \textstyleChCharisSIL{ˈkɐn.tɔ̞ŋ} & \textsc{n} & bag\\
& \textitbf{kantor} & \textstyleChCharisSIL{ˈkɐn.tɔ̞r̥} & \textsc{n} & office\\
& \textitbf{kapal} & \textstyleChCharisSIL{ˈka.pɐl} & \textsc{n} & ship\\
& \textitbf{kapas} & \textstyleChCharisSIL{ˈka.pɐs} & \textsc{n} & cotton\\
& \textitbf{karakter} & \textstyleChCharisSIL{ka.ˈɾɐk̚.tɛ̞r} & \textsc{n} & character\\
& \textitbf{karate} & \textstyleChCharisSIL{ka.ˈɾa.tɛ} & \textsc{n/v.bi} & karate\\
& \textitbf{kariawang} & \textstyleChCharisSIL{ˌka.ɾi.ˈa.wɐn} & \textsc{n} & employee\\
\textstyleExampleSource{x} & \textitbf{karna} & \textstyleChCharisSIL{kɐr.ˈna} & \textsc{cnj} & because\\
& \textitbf{karpet} & \textstyleChCharisSIL{ˈkɐr.pɛ̞t} & \textsc{n} & plastic carpet\\
\textstyleExampleSource{x} & \textitbf{kartapel} & \textstyleChCharisSIL{ˌkɐr.ta.ˈpɛ̞l} & \textsc{n} & slingshot\\
& \textitbf{kartu} & \textstyleChCharisSIL{ˈkɐr.tu} & \textsc{n} & card\\
& \textitbf{karunia} & \textstyleChCharisSIL{ˌka.ɾu.ˈni.a} & \textsc{n} & gift\\
\textstyleExampleSource{x} & \textitbf{kaset} & \textstyleChCharisSIL{ka.ˈsɛ̞t̚} & \textsc{n} & cassette\\
& \textitbf{kasir} & \textstyleChCharisSIL{ˈka.sɪr̥} & \textsc{n} & cashier\\
& \textitbf{kata} & \textstyleChCharisSIL{ˈka.ta} & \textsc{n} & word\\
& \textitbf{kawal} & \textstyleChCharisSIL{ˈka.wɐl} & \textsc{v.bi} & escort\\
& \textitbf{kawing} & \textstyleChCharisSIL{ˈka.wɪn} & \textsc{v.bi} & marry unofficially\\
& \textitbf{kaya} & \textstyleChCharisSIL{ˈka.ja} & \textsc{v.mo(st)} & be rich\\
& \textitbf{kecap} & \textstyleChCharisSIL{ˈkɛ.tʃɐp̚} & \textsc{n} & soy sauce\\
& \textitbf{kejora} & \textstyleChCharisSIL{kɛ.ˈdʒɔ.ɾa} & \textsc{n} & morning star\\
& \textitbf{kem} & \textstyleChCharisSIL{ˈkɛ̞m} & \textsc{n} & camp\\
& \textitbf{kepala} & \textstyleChCharisSIL{kɛ.ˈpa.la} & \textsc{n} & head\\
& \textitbf{keponakan} & \textstyleChCharisSIL{ˌkɛ.pɔ.ˈna.kɐn} & \textsc{n} & nephew, niece\\
\textstyleExampleSource{x} & \textitbf{kerja} & \textstyleChCharisSIL{kɛ̞r.ˈdʒa} & \textsc{v.bi} & work\\
\textstyleExampleSource{x} & \textitbf{kertas} & \textstyleChCharisSIL{kɛ̞r.ˈtɐs} & \textsc{n} & paper\\
\textstyleExampleSource{x} & \textitbf{ketik} & \textstyleChCharisSIL{kɛ.ˈtɪk̚} & \textsc{v.bi} & type\\
& \textitbf{ketumbar} & \textstyleChCharisSIL{kɛ.ˈtʊm.bɐr̥} & \textsc{n} & coriander\\
\textstyleExampleSource{x} & \textitbf{kilogram} & \textstyleChCharisSIL{ˌki.lɔ.ˈgɾɐm} & \textsc{n} & kilogram\\
& \textitbf{kilometer} & \textstyleChCharisSIL{ˌki.lɔ.ˈmɛ̞.tɛ̞r̥} & \textsc{n} & kilometer\\
& \textitbf{kios} & \textstyleChCharisSIL{ˈki.ɔ̞s} & \textsc{n} & kiosk\\
& \textitbf{kip} & \textstyleChCharisSIL{ˈkɪp̚} & \textsc{v.bi} & unloading truck\\
& \textitbf{klakson} & \textstyleChCharisSIL{ˈklɐk̚.sɔ̞n} & \textsc{v.bi} & blow horn\\
& \textitbf{klas} & \textstyleChCharisSIL{ˈklɐs} & \textsc{n} & class\\
& \textitbf{klasis} & \textstyleChCharisSIL{ˈkla.sɪs} & \textsc{n} & ecclesiastical district\\
& \textitbf{klet} & \textstyleChCharisSIL{ˈklɛ̞t} & \textsc{n} & dress\\
& \textitbf{klinik} & \textstyleChCharisSIL{ˈklɪ.nɪk} & \textsc{n} & clinic\\
& \textitbf{kluarga} & \textstyleChCharisSIL{klʊ.ˈɐr.ga} & \textsc{n} & family\\
& \textitbf{knalpot} & \textstyleChCharisSIL{ˈknɐl.pɔ̞t} & \textsc{n} & muffler\\
& \textitbf{kode} & \textstyleChCharisSIL{ˈkɔ.dɛ} & \textsc{n} & code\\
& \textitbf{kolor} & \textstyleChCharisSIL{ˈkɔ̞.lɔ̞r} & \textsc{n} & undershorts\\
\textstyleExampleSource{x} & \textitbf{komandan} & \textstyleChCharisSIL{ˌkɔ.mɐn.ˈdɐn} & \textsc{n} & commandant\\
& \textitbf{komando} & \textstyleChCharisSIL{kɔ.ˈmɐn.dɔ} & \textsc{v.bi} & command\\
\textstyleExampleSource{x} & \textitbf{kombong} & \textstyleChCharisSIL{kɔ̞m.ˈbɔ̞ŋ} & \textsc{v.mo(st)} & be inflated\\
\textstyleExampleSource{x} & \textitbf{komentar} & \textstyleChCharisSIL{ˌkɔ.mɛ̞n.ˈtɐr̥} & \textsc{n} & commentary\\
& \textitbf{komitmen} & \textstyleChCharisSIL{kɔ.ˈmɪt̚.mɛ̞n} & \textsc{n} & commitment\\
& \textitbf{kompi} & \textstyleChCharisSIL{ˈkɔ̞m.pi} & \textsc{n} & military company\\
& \textitbf{kompleks} & \textstyleChCharisSIL{ˈkɔ̞m.plɛ̞ks} & \textsc{n} & complex\\
& \textitbf{komplotang} & \textstyleChCharisSIL{kɔ̞m.ˈplɔ.tɐn} & \textsc{n} & (half) circle\\
& \textitbf{komputer} & \textstyleChCharisSIL{kɔ̞m.ˈpu.tɛ̞r̥} & \textsc{n} & computer\\
& \textitbf{komunikasi} & \textstyleChCharisSIL{ˌkɔ.mu.ni.ˈka.si} & \textsc{n} & communication\\
& \textitbf{kondisi} & \textstyleChCharisSIL{kɔ̞n.ˈdi.si} & \textsc{n} & condition\\
& \textitbf{konsep} & \textstyleChCharisSIL{ˈkɔ̞n.sɛ̞p̚} & \textsc{n} & concept\\
& \textitbf{konsumsi} & \textstyleChCharisSIL{kɔ̞n.ˈsʊm.si} & \textsc{n} & consumption\\
\textstyleExampleSource{x} & \textitbf{kontak} & \textstyleChCharisSIL{kɔ̞n.ˈtɐk̚} & \textsc{v.bi} & contact\\
\textstyleExampleSource{x} & \textitbf{kontan} & \textstyleChCharisSIL{kɔ̞n.ˈtɐn} & \textsc{n} & cash\\
& \textitbf{kontener} & \textstyleChCharisSIL{kɔ̞n.ˈtɛ̞.nɛ̞r̥} & \textsc{n} & container (ship)\\
& \textitbf{kontrak} & \textstyleChCharisSIL{ˈkɔ̞n.tɾɐk} & \textsc{n} & contract\\
& \textitbf{kontrol} & \textstyleChCharisSIL{ˈkɔ̞n.trɔ̞l} & \textsc{v.bi} & control\\
& \textitbf{kopeng} & \textstyleChCharisSIL{ˈkɔ.pɛ̞ŋ} & \textsc{v.bi} & head\\
& \textitbf{koper} & \textstyleChCharisSIL{ˈkɔ.pɛ̞r̥} & \textsc{n} & suitcase\\
& \textitbf{kopi} & \textstyleChCharisSIL{ˈkɔ.pi} & \textsc{n} & coffee\\
& \textitbf{kopling} & \textstyleChCharisSIL{ˈkɔ̞p̚.lɪŋ} & \textsc{n} & clutch\\
& \textitbf{kor} & \textstyleChCharisSIL{ˈkɔ̞r} & \textsc{n} & choir\\
& \textitbf{korbang} & \textstyleChCharisSIL{ˈkɔ̞r.bɐn} & \textsc{n} & sacrifice\\
& \textitbf{kordinasi} & \textstyleChCharisSIL{ˌkɔ̞r.di.ˈna.si} & \textsc{v.bi} & coordinate\\
& \textitbf{kordinator} & \textstyleChCharisSIL{ˌkɔ̞r.di.ˈna.tɔ̞r̥} & \textsc{n} & coordinator\\
& \textitbf{koreksi} & \textstyleChCharisSIL{kɔ.ˈɾɛ̞k̚.si} & \textsc{n} & correction\\
& \textitbf{korupsi} & \textstyleChCharisSIL{kɔ.ˈɾʊp̚.si} & \textsc{n} & corruption\\
& \textitbf{kostum} & \textstyleChCharisSIL{ˈkɔ̞s.tʊm} & \textsc{n} & costume\\
& \textitbf{koteka} & \textstyleChCharisSIL{kɔ.ˈtɛ.ka} & \textsc{n} & penis sheath\\
& \textitbf{kram} & \textstyleChCharisSIL{ˈkrɐm} & \textsc{n} & cramps\\
& \textitbf{kremasi} & \textstyleChCharisSIL{krɛ.ˈma.si} & \textsc{n} & cremation\\
& \textitbf{krempeng} & \textstyleChCharisSIL{ˈkrɛ̞m.pɛ̞ŋ} & \textsc{v.mo(st)} & be thin\\
& \textitbf{kreta} & \textstyleChCharisSIL{ˈkrɛ.ta} & \textsc{n} & carriage\\
& \textitbf{kubur} & \textstyleChCharisSIL{ˈkʊ.bʊr} & \textsc{v.bi} & bury\\
\textstyleExampleSource{x} & \textitbf{kudus} & \textstyleChCharisSIL{kʊ.ˈdʊs} & \textsc{v.mo(st)} & be sacred\\
& \textitbf{kulia} & \textstyleChCharisSIL{ku.ˈli.a} & \textsc{v.bi} & study\\
& \textitbf{kunci} & \textstyleChCharisSIL{ˈkʊn.tʃi} & \textsc{n} & key\\
& \textitbf{kursi} & \textstyleChCharisSIL{ˈkʊr.si} & \textsc{n} & chair\\
& \textitbf{kursus} & \textstyleChCharisSIL{ˈkʊr.sʊs} & \textsc{n} & course\\
& \textitbf{kusus} & \textstyleChCharisSIL{ˈkʊ.sʊs} & \textsc{v.mo(st)} & be special\\
& \textitbf{kwa} & \textstyleChCharisSIL{ˈkwa} & \textsc{n} & broth\\
& \textitbf{kwasa} & \textstyleChCharisSIL{ˈkwa.sa} & \textsc{n} & power\\
& \textitbf{kwat} & \textstyleChCharisSIL{ˈkwɐt} & \textsc{v.mo(st)} & be strong\\
& \textitbf{kwatir} & \textstyleChCharisSIL{ˈkwa.tɪr̥} & \textsc{v.bi} & frighten\\
& \textitbf{kwe} & \textstyleChCharisSIL{ˈkwɛ} & \textsc{n} & cake\\
& \textstyleChBold{L} &  &  & \\
& \textitbf{labu} & \textstyleChCharisSIL{ˈla.bu} & \textsc{n} & gourd\\
& \textitbf{lahir} & \textstyleChCharisSIL{ˈla.hɪr} & \textsc{v.mo(dy)} & give birth\\
& \textitbf{lampu} & \textstyleChCharisSIL{ˈlɐm.pu} & \textsc{n} & lamp\\
& \textitbf{lap} & \textstyleChCharisSIL{ˈlɐp̚} & \textsc{v.bi} & wipe\\
& \textitbf{lapor} & \textstyleChCharisSIL{ˈla.pɔ̞r̥} & \textsc{v.bi} & report\\
& \textitbf{lego} & \textstyleChCharisSIL{ˈlɛ.gɔ} & \textsc{v.bi} & throw away\\
& \textitbf{lem} & \textstyleChCharisSIL{ˈlɛ̞m} & \textsc{v.bi} & glue\\
& \textitbf{lemari} & \textstyleChCharisSIL{lɛ.ˈma.ɾi} & \textsc{n} & cupboard\\
& \textitbf{lep} & \textstyleChCharisSIL{ˈlɛ̞p} & \textsc{n} & laboratory\\
& \textitbf{lesmen} & \textstyleChCharisSIL{ˈlɛ̞s.mɛ̞n} & \textsc{n} & line judge\\
& \textitbf{lipstik} & \textstyleChCharisSIL{ˈlɪp̚.stɪk} & \textsc{n} & lipstick\\
& \textitbf{liter} & \textstyleChCharisSIL{ˈli.tɛ̞r̥} & \textsc{n} & liter\\
& \textitbf{lobi} & \textstyleChCharisSIL{ˈlɔ.bi} & \textsc{n} & lobby\\
& \textitbf{logat} & \textstyleChCharisSIL{ˈlɔ.gɐt̚} & \textsc{n} & speech variety\\
& \textitbf{lokasi} & \textstyleChCharisSIL{lɔ.ˈka.si} & \textsc{n} & location\\
& \textitbf{lonceng} & \textstyleChCharisSIL{ˈlɔ̞n.tʃɛ̞ŋ} & \textsc{n} & bell\\
& \textitbf{los} & \textstyleChCharisSIL{ˈlɔ̞s} & \textsc{v.bi} & loosen\\
& \textstyleChBold{M} &  &  & \\
\textstyleExampleSource{x} & \textitbf{maaf} & \textstyleChCharisSIL{ma.ˈɐf} & \textsc{n} & pardon\\
& \textitbf{mahir} & \textstyleChCharisSIL{ˈma.hɪr̥} & \textsc{v.bi} & master\\
\textstyleExampleSource{x} & \textitbf{majelis} & \textstyleChCharisSIL{ˌma.dʒɛ.ˈlɪs} & \textsc{n} & church elder\\
& \textitbf{makam} & \textstyleChCharisSIL{ˈma.kɐm} & \textsc{n} & grave\\
& \textitbf{makna} & \textstyleChCharisSIL{ˈmɐk̚.na} & \textsc{n} & meaning\\
& \textitbf{maksut} & \textstyleChCharisSIL{ˈmɐk̚.sʊt} & \textsc{n} & purpose\\
& \textitbf{malaria} & \textstyleChCharisSIL{ˌma.la.ˈɾi.a} & \textsc{n} & malaria\\
& \textitbf{malaykat} & \textstyleChCharisSIL{ma.ˈlɐj.kɐt̚} & \textsc{n} & angel\\
& \textitbf{mama} & \textstyleChCharisSIL{ˈma.ma} & \textsc{n} & mother\\
& \textitbf{manfaat} & \textstyleChCharisSIL{mɐn.ˈfa.ɐt̚} & \textsc{v.mo(dy)} & benefit\\
& \textitbf{mangga} & \textstyleChCharisSIL{ˈmɐŋ.ga} & \textsc{n} & mango\\
& \textitbf{mantri} & \textstyleChCharisSIL{ˈmɐn.tɾi} & \textsc{n} & male nurse\\
& \textitbf{manusia} & \textstyleChCharisSIL{ˌma.nu.ˈsi.a} & \textsc{n} & human being\\
& \textitbf{marga} & \textstyleChCharisSIL{ˈmɐr.ga} & \textsc{n} & clan\\
& \textitbf{martabat} & \textstyleChCharisSIL{mɐr.ˈta.bɐt} & \textsc{n} & status\\
& \textitbf{masala} & \textstyleChCharisSIL{ma.ˈsa.la} & \textsc{n} & problem\\
& \textitbf{masarakat} & \textstyleChCharisSIL{ˌma.sa.ˈɾa.kɐt̚} & \textsc{n} & community\\
& \textitbf{matematika} & \textstyleChCharisSIL{ma.ˌtɛ.ma.ˈti.ka} & \textsc{n} & mathematics\\
& \textitbf{materi} & \textstyleChCharisSIL{ma.ˈtɛ̞.ɾi} & \textsc{n} & material\\
& \textitbf{maut} & \textstyleChCharisSIL{ˈma.ʊt} & \textsc{n} & death\\
& \textitbf{mayat} & \textstyleChCharisSIL{ˈma.jɐt} & \textsc{n} & corpse\\
\textstyleExampleSource{x} & \textitbf{mayoritas} & \textstyleChCharisSIL{ma.ˌjɔ.ɾi.ˈtɐs} & \textsc{n} & majority\\
& \textitbf{meja} & \textstyleChCharisSIL{ˈmɛ.dʒa} & \textsc{n} & table\\
\textstyleExampleSource{x} & \textitbf{mental} & \textstyleChCharisSIL{mɛ̞n.ˈtɐl} & \textsc{n} & emotion\\
\textstyleExampleSource{x} & \textitbf{mentri} & \textstyleChCharisSIL{mɛ̞n.ˈtri} & \textsc{n} & cabinet minister\\
& \textitbf{merdeka} & \textstyleChCharisSIL{mɛ̞r.ˈdɛ.ka} & \textsc{v.mo(st)} & be independent\\
\textstyleExampleSource{x} & \textitbf{mesing} & \textstyleChCharisSIL{mɛ.ˈsɪn} & \textsc{n} & engine\\
& \textitbf{meter} & \textstyleChCharisSIL{ˈmɛ̞.tɛ̞r̥} & \textsc{n} & meter\\
& \textitbf{milyar} & \textstyleChCharisSIL{ˈmɪl.jɐr} & \textsc{num.c} & billion\\
& \textitbf{mimbar} & \textstyleChCharisSIL{ˈmɪm.bɐr̥} & \textsc{n} & pulpit\\
& \textitbf{minggu} & \textstyleChCharisSIL{ˈmɪŋ.gu} & \textsc{n} & week, Sunday\\
& \textitbf{mini} & \textstyleChCharisSIL{ˈmi.ni} & \textsc{v.mo(st)} & be mini\\
\textstyleExampleSource{x} & \textitbf{minit} & \textstyleChCharisSIL{mɪ.ˈnɪt} & \textsc{n} & minute\\
& \textitbf{misionaris} & \textstyleChCharisSIL{mi.ˌsi.ɔ.ˈna.rɪs} & \textsc{n} & missionary\\
& \textitbf{miskin} & \textstyleChCharisSIL{ˈmɪs.kɪn} & \textsc{v.mo(st)} & be poor\\
& \textitbf{mobil} & \textstyleChCharisSIL{ˈmɔ.bɪl} & \textsc{n} & car\\
\textstyleExampleSource{x} & \textitbf{modal} & \textstyleChCharisSIL{ˈmɔ.dɐl} & \textsc{n} & means\\
& \textitbf{model} & \textstyleChCharisSIL{mɔ.ˈdɛ̞l} & \textsc{n} & model\\
& \textitbf{mop} & \textstyleChCharisSIL{ˈmɔ̞p̚} & \textsc{n} & joke\\
& \textitbf{motor} & \textstyleChCharisSIL{ˈmɔ.tɔ̞r̥} & \textsc{n} & motorbike\\
& \textitbf{mujisat} & \textstyleChCharisSIL{mu.ˈdʒi.sɐt} & \textsc{n} & miracle\\
& \textitbf{mulay} & \textstyleChCharisSIL{ˈmu.lɐj} & \textsc{n} & start\\
& \textitbf{mulia} & \textstyleChCharisSIL{mu.ˈli.a} & \textsc{v.mo(st)} & be sublime\\
& \textitbf{murit} & \textstyleChCharisSIL{ˈmu.ɾɪt} & \textsc{n} & pupil\\
& \textitbf{musim} & \textstyleChCharisSIL{ˈmu.sɪm} & \textsc{n} & season\\
& \textstyleChBold{N} &  &  & \\
& \textitbf{nabi} & \textstyleChCharisSIL{ˈna.bi} & \textsc{n} & prophet\\
& \textitbf{nama} & \textstyleChCharisSIL{ˈna.ma} & \textsc{n} & name\\
& \textitbf{napas} & \textstyleChCharisSIL{ˈna.pɐs} & \textsc{n} & breath\\
& \textitbf{nasihat} & \textstyleChCharisSIL{na.ˈsɪ.hɐt} & \textsc{n/v.bi} & advice\\
& \textitbf{nasional} & \textstyleChCharisSIL{ˌnɐ.si.ɔ.ˈnɐl} & \textsc{v.mo(st)} & be national\\
& \textitbf{nasip} & \textstyleChCharisSIL{ˈna.sɪp} & \textsc{n} & destiny\\
& \textitbf{natal} & \textstyleChCharisSIL{ˈna.tɐl} & \textsc{n} & Christmas\\
& \textitbf{neces} & \textstyleChCharisSIL{ˈnɛ.tʃɛ̞s} & \textsc{v.mo(st)} & be neat\\
& \textitbf{negara} & \textstyleChCharisSIL{nɛ.ˈga.ɾa} & \textsc{n} & state\\
\textstyleExampleSource{x} & \textitbf{negri} & \textstyleChCharisSIL{nɛ.ˈgri} & \textsc{n} & state\\
& \textitbf{neraka} & \textstyleChCharisSIL{nɛ.ˈɾa.ka} & \textsc{n} & hell\\
& \textitbf{net} & \textstyleChCharisSIL{ˈnɛ̞t} & \textsc{n} & (sport) net\\
& \textitbf{nilay} & \textstyleChCharisSIL{ˈni.lɐj} & \textsc{n} & value\\
& \textitbf{nofember} & \textstyleChCharisSIL{nɔ.ˈfɛ̞m.bɛ̞r̥} & \textsc{n} & November\\
& \textitbf{nomor} & \textstyleChCharisSIL{ˈnɔ̞.mɔ̞r̥} & \textsc{n} & number\\
& \textitbf{nona} & \textstyleChCharisSIL{ˈnɔ.na} & \textsc{n} & girl\\
& \textitbf{nyonya} & \textstyleChCharisSIL{ˈɲɔ.ɲa} & \textsc{n} & lady\\
& \textitbf{nyora} & \textstyleChCharisSIL{ˈɲɔ.ɾa} & \textsc{n} & teacher’s wife\\
& \textstyleChBold{O} &  &  & \\
& \textitbf{odol} & \textstyleChCharisSIL{ˈɔ̞.dɔ̞l} & \textsc{n} & toothpaste\\
& \textitbf{ofor} & \textstyleChCharisSIL{ˈɔ̞.fɔ̞r} & \textsc{v.bi} & give\\
& \textitbf{oktober} & \textstyleChCharisSIL{ɔ̞k̚.ˈtɔ̞.bɛ̞r̥} & \textsc{n} & October\\
& \textitbf{oli} & \textstyleChCharisSIL{ˈɔ.li} & \textsc{n} & oil\\
& \textitbf{olimpiade} & \textstyleChCharisSIL{ɔ.ˌlɪm.pi.ˈa.dɛ} & \textsc{n} & olympiad\\
& \textitbf{om} & \textstyleChCharisSIL{ˈɔ̞m} & \textsc{n} & uncle\\
& \textitbf{oma} & \textstyleChCharisSIL{ˈɔ.ma} & \textsc{n} & great-great-grandmother\\
& \textitbf{ondoafi} & \textstyleChCharisSIL{ˌɔ̞n.dɔ̞.ˈa.fi} & \textsc{n} & traditional chief\\
& \textitbf{ongkos} & \textstyleChCharisSIL{ˈɔ̞ŋ.kɔ̞s} & \textsc{n} & expenses\\
& \textitbf{opa} & \textstyleChCharisSIL{ˈɔ.pa} & \textsc{n} & great-great-grandfather\\
& \textitbf{opname} & \textstyleChCharisSIL{ɔ̞p̚.ˈna.mɛ} & \textsc{v.mo(st)} & be hospitalized\\
& \textitbf{oprasi} & \textstyleChCharisSIL{ɔ.ˈpɾa.si} & \textsc{n} & operation\\
& \textitbf{otomatis} & \textstyleChCharisSIL{ˌɔ.tɔ.ˈma.tɪs} & \textsc{v.mo(st)} & be automatic\\
\textstyleExampleSource{x} & \textitbf{otonom} & \textstyleChCharisSIL{ˌɔ̞.tɔ̞.ˈnɔ̞m} & \textsc{v.mo(st)} & be autonomous\\
& \textitbf{otonomi} & \textstyleChCharisSIL{ˌɔ.tɔ.ˈnɔ.mi} & \textsc{n} & autonomy\\
& \textstyleChBold{P} &  &  & \\
& \textitbf{pagar} & \textstyleChCharisSIL{ˈpa.gɐr̥} & \textsc{n} & fence\\
& \textitbf{paham} & \textstyleChCharisSIL{ˈpa.hɐm} & \textsc{n} & understanding\\
& \textitbf{pakem} & \textstyleChCharisSIL{ˈpa.kɛ̞m} & \textsc{n} & break disk\\
& \textitbf{paket} & \textstyleChCharisSIL{ˈpa.kɛ̞t} & \textsc{n} & package\\
& \textitbf{paksa} & \textstyleChCharisSIL{ˈpɐk̚.sa} & \textsc{v.bi} & force\\
& \textitbf{pakwel} & \textstyleChCharisSIL{ˈpɐk̚.wɛ̞l} & \textsc{n} & k. o. crowbar\\
& \textitbf{panci} & \textstyleChCharisSIL{ˈpɐn.tʃi} & \textsc{n} & pan\\
& \textitbf{panitia} & \textstyleChCharisSIL{ˌpa.nɪ.ˈti.a} & \textsc{n} & committee\\
& \textitbf{parte} & \textstyleChCharisSIL{ˈpɐr.tɛ} & \textsc{n} & party\\
& \textitbf{pas} & \textstyleChCharisSIL{ˈpɐs} & \textsc{v.mo(st)}

\textsc{adv} & be exact

precisely\\
& \textitbf{pena} & \textstyleChCharisSIL{ˈpɛ.na} & \textsc{n} & pen\\
& \textitbf{pendeta} & \textstyleChCharisSIL{pɛ̞n.ˈdɛ.ta} & \textsc{n} & pastor\\
& \textitbf{penjara} & \textstyleChCharisSIL{pɛ̞n.ˈdʒa.ɾa} & \textsc{n/v.bi} & jail\\
& \textitbf{pepaya} & \textstyleChCharisSIL{pɛ.ˈpa.ja} & \textsc{n} & papaya\\
& \textitbf{percaya} & \textstyleChCharisSIL{pɛ̞r.ˈtʃa.ja} & \textsc{v.bi} & trust\\
& \textitbf{peristiwa} & \textstyleChCharisSIL{ˌpɛ.ɾɪs.ˈti.wa} & \textsc{n} & incident\\
& \textitbf{perkosa} & \textstyleChCharisSIL{pɛ̞r.ˈkɔ.sa} & \textsc{v.bi} & rape\\
\textstyleExampleSource{x} & \textitbf{perlu} & \textstyleChCharisSIL{pɛ̞r.ˈlu} & \textsc{v.bi} & need\\
& \textitbf{permanen} & \textstyleChCharisSIL{pɛ̞r.ˈma.nɛ̞n} & \textsc{v.mo(st)} & be permanent\\
& \textitbf{permisi} & \textstyleChCharisSIL{pɛ̞r.ˈmi.si} & \textsc{v.mo(dy)} & ask permission\\
\textstyleExampleSource{x} & \textitbf{persen} & \textstyleChCharisSIL{pɛ̞r.ˈsɛ̞n} & \textsc{n} & percent\\
\textstyleExampleSource{x} & \textitbf{persis} & \textstyleChCharisSIL{pɛ̞r.ˈsɪs} & \textsc{v.mo(st)} & be precise\\
& \textitbf{pertama} & \textstyleChCharisSIL{pɛ̞r.ˈta.ma} & \textsc{num.o} & first\\
& \textitbf{pesta} & \textstyleChCharisSIL{ˈpɛ̞s.ta} & \textsc{n} & party\\
\textstyleExampleSource{x} & \textitbf{peta} & \textstyleChCharisSIL{pɛ.ˈta} & \textsc{n} & map\\
\textstyleExampleSource{x} & \textitbf{peti} & \textstyleChCharisSIL{pɛ.ˈti} & \textsc{n} & box\\
\textstyleExampleSource{x} & \textitbf{petromaks} & \textstyleChCharisSIL{ˌpɛ.tɾɔ.ˈmɐks} & \textsc{n} & kerosene lantern\\
& \textitbf{piara} & \textstyleChCharisSIL{pi.ˈa.ɾa} & \textsc{v.bi} & raise\\
& \textitbf{pikir} & \textstyleChCharisSIL{ˈpi.kɪr} & \textsc{v.bi} & think\\
& \textitbf{piknik} & \textstyleChCharisSIL{ˈpɪk̚.nɪk̚} & \textsc{v.mo(dy)} & picnic\\
& \textitbf{pilot} & \textstyleChCharisSIL{ˈpi.lɔ̞t} & \textsc{n} & pilot\\
& \textitbf{piring} & \textstyleChCharisSIL{ˈpɪ.ɾɪŋ} & \textsc{n} & plate\\
\textstyleExampleSource{x} & \textitbf{pisikologi} & \textstyleChCharisSIL{ˌpi.si.ˌkɔ.lɔ.ˈgi} & \textsc{n} & psychology\\
& \textitbf{plastik} & \textstyleChCharisSIL{ˈplɐs.tɪk} & \textsc{n} & plastic\\
& \textitbf{plat} & \textstyleChCharisSIL{ˈplɐt} & \textsc{v.mo(st)} & be flattened\\
\textstyleExampleSource{x} & \textitbf{pleton} & \textstyleChCharisSIL{plɛ.ˈtɔ̞n} & \textsc{n} & platoon\\
& \textitbf{plita} & \textstyleChCharisSIL{ˈpli.ta} & \textsc{n} & oil lamp\\
& \textitbf{polisi} & \textstyleChCharisSIL{pɔ.ˈli.si} & \textsc{n} & police\\
& \textitbf{politik} & \textstyleChCharisSIL{pɔ.ˈlɪ.tɪk} & \textsc{n} & politics\\
& \textitbf{pondok} & \textstyleChCharisSIL{ˈpɔ̞n.dɔ̞k} & \textsc{n} & shelter\\
& \textitbf{porsi} & \textstyleChCharisSIL{ˈpɔ̞r.si} & \textsc{n} & portion\\
& \textitbf{pos} & \textstyleChCharisSIL{ˈpɔ̞s} & \textsc{n} & post\\
& \textitbf{posisi} & \textstyleChCharisSIL{pɔ.ˈsi.si} & \textsc{n} & position\\
& \textitbf{praktek} & \textstyleChCharisSIL{ˈprɐk̚.tɛ̞k} & \textsc{n} & practicum\\
\textstyleExampleSource{x} & \textitbf{presiden} & \textstyleChCharisSIL{ˌprɛ.si.ˈdɛ̞n} & \textsc{n} & president\\
& \textitbf{pribadi} & \textstyleChCharisSIL{pri.ˈba.di} & \textsc{n} & personal property\\
& \textitbf{priksa} & \textstyleChCharisSIL{ˈprɪk̚.sa} & \textsc{v.bi} & check\\
& \textitbf{prinsip} & \textstyleChCharisSIL{ˈprɪn.sɪp̚} & \textsc{n} & principle\\
& \textitbf{priode} & \textstyleChCharisSIL{pri.ˈɔ.dɛ} & \textsc{n} & period\\
\textstyleExampleSource{x} & \textitbf{prioritas} & \textstyleChCharisSIL{pri.ˌɔ.ɾi.ˈtɐs} & \textsc{n} & priority\\
\textstyleExampleSource{x} & \textitbf{profesor} & \textstyleChCharisSIL{ˌprɔ.fɛ.ˈsɔ̞r̥} & \textsc{n} & professor\\
& \textitbf{program} & \textstyleChCharisSIL{ˈprɔ.grɐm} & \textsc{n/v.bi} & program\\
& \textitbf{propinsi} & \textstyleChCharisSIL{prɔ.ˈpɪn.sɪ} & \textsc{n} & province\\
\textstyleExampleSource{x} & \textitbf{proposal} & \textstyleChCharisSIL{ˌprɔ.pɔ.ˈsɐl} & \textsc{n} & proposal\\
& \textitbf{proses} & \textstyleChCharisSIL{ˈprɔ.sɛ̞s} & \textsc{n/v.bi} & process\\
\textstyleExampleSource{x} & \textitbf{protes} & \textstyleChCharisSIL{prɔ.ˈtɛ̞s} & \textsc{v.bi} & protest\\
& \textitbf{proyek} & \textstyleChCharisSIL{ˈprɔ.jɛ̞k} & \textsc{n} & project\\
& \textitbf{puasa} & \textstyleChCharisSIL{pʊ.ˈa.sa} & \textsc{v.bi} & fast\\
& \textitbf{puji} & \textstyleChCharisSIL{ˈpu.dʒi} & \textsc{v.bi} & praise\\
& \textitbf{pul} & \textstyleChCharisSIL{ˈpʊl} & \textsc{n} & pool\\
& \textitbf{pulsa} & \textstyleChCharisSIL{ˈpʊl.sa} & \textsc{n} & pulse\\
& \textstyleChBold{R} &  &  & \\
& \textitbf{rabu} & \textstyleChCharisSIL{ˈra.bu} & \textsc{n} & Wednesday\\
& \textitbf{radio} & \textstyleChCharisSIL{ra.ˈdɪ.ɔ} & \textsc{n} & radio\\
& \textitbf{rahasia} & \textstyleChCharisSIL{ˌra.ha.ˈsɪ.a} & \textsc{n} & secret\\
& \textitbf{raja} & \textstyleChCharisSIL{ˈra.dʒa} & \textsc{n} & king\\
& \textitbf{rakyat} & \textstyleChCharisSIL{ˈrɐk̚.jɐt} & \textsc{n} & citizenry\\
& \textitbf{rangsel} & \textstyleChCharisSIL{ˈrɐŋ.sɛ̞l} & \textsc{n} & backpack\\
& \textitbf{raport} & \textstyleChCharisSIL{ˈra.pɔ̞rt} & \textsc{n} & school report book\\
& \textitbf{rasa} & \textstyleChCharisSIL{ˈra.sa} & \textsc{v.bi} & feel\\
& \textitbf{rasul} & \textstyleChCharisSIL{ˈra.sʊl} & \textsc{n} & prophet\\
& \textitbf{reaksi} & \textstyleChCharisSIL{rɛ.ˈɐk̚.si} & \textsc{n} & reaction\\
& \textitbf{referendum} & \textstyleChCharisSIL{ˌrɛ.fɛ.ˈɾɛ̞n.dʊm} & \textsc{n} & referendum\\
& \textitbf{reformasi} & \textstyleChCharisSIL{ˌrɛ.fɔ̞r.ˈma.si} & \textsc{n} & reformation\\
\textstyleExampleSource{x} & \textitbf{rejeki} & \textstyleChCharisSIL{ˌrɛ.dʒɛ.ˈki} & \textsc{n} & livelihood\\
\textstyleExampleSource{x} & \textitbf{rekam} & \textstyleChCharisSIL{rɛ.ˈkɐm} & \textsc{v.bi} & record\\
\textstyleExampleSource{x} & \textitbf{rekening} & \textstyleChCharisSIL{ˌrɛ.kɛ.ˈnɪŋ} & \textsc{n} & bank account\\
& \textitbf{rekreasi} & \textstyleChCharisSIL{ˌrɛ.krɛ.ˈa.si} & \textsc{n} & recreation\\
& \textitbf{rel} & \textstyleChCharisSIL{ˈrɛ̞l} & \textsc{n} & railway track\\
& \textitbf{rela} & \textstyleChCharisSIL{ˈrɛ.la} & \textsc{v.mo(st)} & be willing\\
& \textitbf{rem} & \textstyleChCharisSIL{ˈrɛ̞m} & \textsc{n/v.bi} & brake\\
& \textitbf{rencana} & \textstyleChCharisSIL{rɛ̞n.ˈtʃa.na} & \textsc{n/v.bi} & plan\\
& \textitbf{rengking} & \textstyleChCharisSIL{ˈrɛ̞ŋ.kɪŋ} & \textsc{n} & ranking\\
\textstyleExampleSource{x} & \textitbf{republik} & \textstyleChCharisSIL{ˌrɛ.pu.ˈblɪk} & \textsc{n} & republic\\
& \textitbf{resiko} & \textstyleChCharisSIL{rɛ.ˈsi.kɔ} & \textsc{n} & risk\\
\textstyleExampleSource{x} & \textitbf{resmi} & \textstyleChCharisSIL{rɛ̞s.ˈmi} & \textsc{v.mo(st)} & be official\\
\textstyleExampleSource{x} & \textitbf{retrit} & \textstyleChCharisSIL{rɛ.ˈtɾɪt} & \textsc{n} & retreat\\
& \textitbf{ring} & \textstyleChCharisSIL{ˈrɪn} & \textsc{n} & ring\\
& \textitbf{ro} & \textstyleChCharisSIL{ˈrɔ} & \textsc{n} & spirit\\
& \textitbf{roda} & \textstyleChCharisSIL{ˈrɔ.da} & \textsc{n} & wheel\\
& \textitbf{rok} & \textstyleChCharisSIL{ˈrɔ̞k} & \textsc{n} & skirt\\
& \textitbf{rol} & \textstyleChCharisSIL{ˈrɔ̞l} & \textsc{n} & roll\\
& \textitbf{rotang} & \textstyleChCharisSIL{ˈrɔ̞.tɐn} & \textsc{n} & rattan\\
& \textitbf{rupa} & \textstyleChCharisSIL{ˈrʊ.pa} & \textsc{n} & form\\
& \textitbf{rupia} & \textstyleChCharisSIL{ru.ˈpi.a} & \textsc{n} & rupiah\\
& \textstyleChBold{S} &  &  & \\
& \textitbf{sabar} & \textstyleChCharisSIL{ˈsa.bɐr̥} & \textsc{v.mo(st)} & be patient\\
& \textitbf{sabung} & \textstyleChCharisSIL{ˈsa.bʊn} & \textsc{n} & soap\\
& \textitbf{saja} & \textstyleChCharisSIL{ˈsa.dʒa} & \textsc{adv} & just\\
& \textitbf{sak} & \textstyleChCharisSIL{ˈsɐk} & \textsc{n} & bag\\
& \textitbf{saksi} & \textstyleChCharisSIL{ˈsɐk̚.sɪ} & \textsc{n} & witness\\
& \textitbf{salam} & \textstyleChCharisSIL{ˈsa.lɐm} & \textsc{v.bi} & greet\\
& \textitbf{salju} & \textstyleChCharisSIL{ˈsɐl.dʒu} & \textsc{n} & snow\\
& \textitbf{salon} & \textstyleChCharisSIL{ˈsa.lɔ̞n} & \textsc{n} & console\\
& \textitbf{sama} & \textstyleChCharisSIL{ˈsa.ma} & \textsc{v.mo(st)} & be same\\
& \textitbf{sandal} & \textstyleChCharisSIL{ˈsɐn.dɐl} & \textsc{n} & sandal\\
& \textitbf{sangka} & \textstyleChCharisSIL{ˈsɐŋ.ka} & \textsc{v.bi} & assume\\
& \textitbf{saptu} & \textstyleChCharisSIL{ˈsɐp̚.tu} & \textsc{n} & Saturday\\
& \textitbf{sarjana} & \textstyleChCharisSIL{sɐr.ˈdʒa.na} & \textsc{n} & academic degree\\
& \textitbf{sasar} & \textstyleChCharisSIL{ˈsa.sɐr̥} & \textsc{v.mo(st)} & be insane\\
\textstyleExampleSource{x} & \textitbf{sebap} & \textstyleChCharisSIL{sɛ.ˈbɐp̚} & \textsc{cnj} & because\\
\textstyleExampleSource{x} & \textitbf{sebentar} & \textstyleChCharisSIL{ˌsɛ.bɛ̞n.ˈtɐr} & \textsc{adv} & in a moment\\
& \textitbf{segala} & \textstyleChCharisSIL{sɛ.ˈga.la} & \textsc{qt} & all\\
& \textitbf{sehat} & \textstyleChCharisSIL{ˈsɛ.hɐt̚} & \textsc{v.mo(st)} & be healthy\\
& \textitbf{sejara} & \textstyleChCharisSIL{sɛ.ˈdʒa.ɾa} & \textsc{n} & history\\
& \textitbf{sekertaria} & \textstyleChCharisSIL{sɛ.ˌkɛ̞r.ta.ˈri.a} & \textsc{n} & secretariat\\
& \textitbf{sekertaris} & \textstyleChCharisSIL{ˌsɛ.kɛ̞r.ˈta.ɾɪs} & \textsc{n} & secretary\\
& \textitbf{seksi} & \textstyleChCharisSIL{ˈsɛ̞k̚.si} & \textsc{n} & section\\
& \textitbf{sel} & \textstyleChCharisSIL{ˈsɛ̞l} & \textsc{n} & cell\\
& \textitbf{semester} & \textstyleChCharisSIL{sɛ.ˈmɛ̞s.tɛ̞r̥} & \textsc{n} & semester\\
& \textitbf{sempurna} & \textstyleChCharisSIL{sɛ̞m.ˈpʊr.na} & \textsc{v.mo(st)} & be perfect\\
& \textitbf{seng} & \textstyleChCharisSIL{ˈsɛ̞ŋ} & \textsc{n} & corrugated iron\\
& \textitbf{sengaja} & \textstyleChCharisSIL{sɛ.ˈŋa.dʒa} & \textsc{v.bi} & do intentionally\\
& \textitbf{sengsara} & \textstyleChCharisSIL{sɛ̞ŋ.ˈsa.ɾa} & \textsc{v.mo(dy)} & suffer\\
\textstyleExampleSource{x} & \textitbf{sening} & \textstyleChCharisSIL{sɛ.ˈnɪn} & \textsc{n} & Monday\\
& \textitbf{senjata} & \textstyleChCharisSIL{sɛ̞n.ˈdʒa.ta} & \textsc{n} & rifle\\
& \textitbf{senter} & \textstyleChCharisSIL{ˈsɛ̞n.tɛ̞r̥} & \textsc{n/v.bi} & flashlight / light with flashlight\\
& \textitbf{senyor} & \textstyleChCharisSIL{ˈsɛ.ɲɔ̞r̥} & \textsc{n} & senior\\
& \textitbf{serfen} & \textstyleChCharisSIL{ˈsɛ̞r.fɛ̞n} & \textsc{v.bi} & serve\\
& \textitbf{serfis} & \textstyleChCharisSIL{ˈsɛ̞r.fɪs} & \textsc{v.bi} & process documents\\
\textstyleExampleSource{x} & \textitbf{serius} & \textstyleChCharisSIL{ˌsɛ.ɾi.ˈʊs} & \textsc{v.mo(st)} & be serious\\
\textstyleExampleSource{x} & \textitbf{sersang} & \textstyleChCharisSIL{sɛ̞r.ˈsɐŋ} & \textsc{n} & sergeant\\
& \textitbf{set} & \textstyleChCharisSIL{ˈsɛ̞t} & \textsc{n} & set\\
& \textitbf{setang} & \textstyleChCharisSIL{ˈsɛ.tɐn} & \textsc{n} & evil spirit\\
& \textitbf{setia} & \textstyleChCharisSIL{sɛ.ˈti.a} & \textsc{v.mo(st)} & be faithful\\
& \textitbf{sidi} & \textstyleChCharisSIL{ˈsi.di} & \textsc{n} & CD player\\
& \textitbf{sifat} & \textstyleChCharisSIL{ˈsi.fɐt} & \textsc{n} & characteristic\\
& \textitbf{sihir} & \textstyleChCharisSIL{ˈsɪ.hɪr} & \textsc{v.bi} & practice black magic on s.o.\\
& \textitbf{silet} & \textstyleChCharisSIL{ˈsi.lɛ̞t} & \textsc{n} & razor blade\\
& \textitbf{singga} & \textstyleChCharisSIL{ˈsɪŋ.ga} & \textsc{v.mo(dy)} & stop by\\
& \textitbf{sinode} & \textstyleChCharisSIL{si.ˈnɔ.dɛ} & \textsc{n} & synod\\
& \textitbf{sipil} & \textstyleChCharisSIL{ˈsi.pɪl} & \textsc{v.mo(st)} & be civil\\
& \textitbf{sisa} & \textstyleChCharisSIL{ˈsi.sa} & \textsc{n} & residue\\
& \textitbf{siswa} & \textstyleChCharisSIL{ˈsɪs.wa} & \textsc{n} & student\\
& \textitbf{siswi} & \textstyleChCharisSIL{ˈsis.wi} & \textsc{n} & female student\\
& \textitbf{skaf} & \textstyleChCharisSIL{ˈskɐf} & \textsc{v.bi} & plane wood\\
& \textitbf{skola} & \textstyleChCharisSIL{ˈskɔ.la} & \textsc{n/v.mo(dy)} & school / go to school\\
& \textitbf{skop} & \textstyleChCharisSIL{ˈskɔ̞p̚} & \textsc{n/v.bi} & shovel\\
& \textitbf{skot} & \textstyleChCharisSIL{ˈskɔ̞t̚} & \textsc{v.bi} & hit\\
& \textitbf{skripsi} & \textstyleChCharisSIL{ˈskrɪp̚.sɪ} & \textsc{n} & minithesis\\
& \textitbf{skutu} & \textstyleChCharisSIL{ˈsku.tu} & \textsc{n} & partner\\
& \textitbf{slamat} & \textstyleChCharisSIL{ˈsla.mɐt̚} & \textsc{v.mo(st)} & be safe\\
& \textitbf{slang} & \textstyleChCharisSIL{ˈslɐŋ} & \textsc{n} & hose\\
& \textitbf{slasa} & \textstyleChCharisSIL{ˈsla.sa} & \textsc{n} & Tuesday\\
& \textitbf{slenger} & \textstyleChCharisSIL{ˈslɛ.ŋɛ̞r̥} & \textsc{n} & sling\\
& \textitbf{smeng} & \textstyleChCharisSIL{ˈsmɛ̞n} & \textsc{n/v.bi} & cement\\
& \textitbf{smes} & \textstyleChCharisSIL{ˈsmɛ̞s} & \textsc{v.bi} & smash\\
& \textitbf{smua} & \textstyleChCharisSIL{ˈsmʊ.a} & \textsc{qt} & all\\
& \textitbf{snek} & \textstyleChCharisSIL{ˈsnɛ̞k} & \textsc{n} & snack\\
& \textitbf{soak} & \textstyleChCharisSIL{ˈsɔ.ɐk} & \textsc{v.mo(st)} & be weak\\
& \textitbf{soal} & \textstyleChCharisSIL{ˈsɔ̞.ɐl} & \textsc{n} & problem\\
& \textitbf{sodara} & \textstyleChCharisSIL{sɔ.ˈda.ɾa} & \textsc{n} & sibling\\
& \textitbf{solar} & \textstyleChCharisSIL{ˈsɔ.lɐr̥} & \textsc{n} & diesel fuel\\
& \textitbf{sono} & \textstyleChCharisSIL{ˈsɔ.nɔ} & \textsc{v.mo(dy)} & sleep soundly\\
\textstyleExampleSource{x} & \textitbf{sopir} & \textstyleChCharisSIL{sɔ.ˈpɪr̥} & \textsc{n} & driver\\
& \textitbf{sorga} & \textstyleChCharisSIL{ˈsɔ̞r.ga} & \textsc{n} & heaven\\
& \textitbf{sos} & \textstyleChCharisSIL{ˈsɔ̞s} & \textsc{n} & sauce\\
\textstyleExampleSource{x} & \textitbf{sosial} & \textstyleChCharisSIL{ˌsɔ.si.ˈɐl} & \textsc{v.mo(st)} & be social\\
& \textitbf{spak} & \textstyleChCharisSIL{ˈspɐk̚} & \textsc{v.bi} & kick\\
& \textitbf{spang} & \textstyleChCharisSIL{ˈspɐŋ} & \textsc{v.bi} & spank\\
& \textitbf{spatu} & \textstyleChCharisSIL{ˈspa.tu} & \textsc{n} & shoe\\
& \textitbf{speda} & \textstyleChCharisSIL{ˈspɛ.da} & \textsc{n} & bicycle\\
& \textitbf{spit} & \textstyleChCharisSIL{ˈspɪt} & \textsc{n} & speedboat\\
& \textitbf{sprey} & \textstyleChCharisSIL{ˈsprɛ̞j} & \textsc{n} & bedsheet\\
& \textitbf{spul} & \textstyleChCharisSIL{ˈspʊl} & \textsc{v.bi} & rinse\\
& \textitbf{staf} & \textstyleChCharisSIL{ˈstɐf} & \textsc{n} & staff\\
& \textitbf{standar} & \textstyleChCharisSIL{ˈstɐn.dɐr̥} & \textsc{n} & motorbike kickstand\\
& \textitbf{stang} & \textstyleChCharisSIL{ˈstɐŋ} & \textsc{v.mo(dy)} & boast\\
& \textitbf{star} & \textstyleChCharisSIL{ˈstɐr} & \textsc{v.bi} & start engine\\
& \textitbf{status} & \textstyleChCharisSIL{ˈsta.tʊs} & \textsc{n} & status\\
& \textitbf{stel} & \textstyleChCharisSIL{ˈstɛ̞l} & \textsc{v.bi} & tune\\
& \textitbf{step} & \textstyleChCharisSIL{ˈstɛ̞p̚} & \textsc{v.mo(dy)} & fall unconsciously\\
& \textitbf{stir} & \textstyleChCharisSIL{ˈstɪr̥} & \textsc{n/v.bi} & steering wheel / steer\\
& \textitbf{stop} & \textstyleChCharisSIL{ˈstɔ̞p̚} & \textsc{v.bi} & stop\\
& \textitbf{stor} & \textstyleChCharisSIL{ˈstɔ̞r} & \textsc{v.bi} & deposit\\
& \textitbf{strap} & \textstyleChCharisSIL{ˈstrɐp̚} & \textsc{v.bi} & punish\\
& \textitbf{strategi} & \textstyleChCharisSIL{stra.ˈtɛ.gi} & \textsc{n} & strategy\\
& \textitbf{stres} & \textstyleChCharisSIL{ˈstrɛ̞s} & \textsc{v.mo(st)} & be stressed\\
& \textitbf{strika} & \textstyleChCharisSIL{ˈstrɪ.ka} & \textsc{n/v.bi} & iron\\
& \textitbf{strom} & \textstyleChCharisSIL{ˈstɾɔ̞m} & \textsc{n} & electric current\\
& \textitbf{suda} & \textstyleChCharisSIL{ˈsu.da} & \textsc{adv} & already\\
& \textitbf{sukses} & \textstyleChCharisSIL{ˈsʊk̚.sɛ̞s} & \textsc{n} & success\\
& \textitbf{supaya} & \textstyleChCharisSIL{su.ˈpa.ja} & \textsc{cnj} & so that\\
\textstyleExampleSource{x} & \textitbf{supermi} & \textstyleChCharisSIL{ˌsu.pɛ̞r.ˈmi} & \textsc{n} & instant noodles\\
& \textitbf{suster} & \textstyleChCharisSIL{ˈsʊs.tɛ̞r̥} & \textsc{n} & nurse\\
& \textitbf{suting} & \textstyleChCharisSIL{ˈsu.tɪŋ} & \textsc{v.bi} & shoot\\
& \textitbf{swami} & \textstyleChCharisSIL{ˈswa.mi} & \textsc{n} & husband\\
& \textitbf{swara} & \textstyleChCharisSIL{ˈswa.ɾa} & \textsc{n} & voice\\
& \textitbf{syarat} & \textstyleChCharisSIL{ˈsʲa.ɾɐt} & \textsc{n} & condition\\
& \textitbf{syukur} & \textstyleChCharisSIL{ˈsʲu.kʊr̥} & \textsc{n} & thanks to God\\
& \textstyleChBold{T} &  &  & \\
& \textitbf{taat} & \textstyleChCharisSIL{ˈta.ɐt} & \textsc{v.mo(st)} & be obedient\\
& \textitbf{takraw} & \textstyleChCharisSIL{ˈta.krɐw} & \textsc{n} & Takraw ball game\\
& \textitbf{talenta} & \textstyleChCharisSIL{ta.ˈlɛ̞n.ta} & \textsc{n} & gift\\
& \textitbf{tang} & \textstyleChCharisSIL{ˈtɐŋ} & \textsc{n} & pliers\\
& \textitbf{tanpa} & \textstyleChCharisSIL{ˈtɐn.pa} & \textsc{prep} & without\\
& \textitbf{tanta} & \textstyleChCharisSIL{ˈtɐn.ta} & \textsc{n} & aunt\\
& \textitbf{tapi} & \textstyleChCharisSIL{ˈta.pi} & \textsc{cnj} & but\\
& \textitbf{taplak} & \textstyleChCharisSIL{ˈtɐp̚.lɐk} & \textsc{n} & tablecloth\\
& \textitbf{target} & \textstyleChCharisSIL{ˈtɐr.gɛ̞t} & \textsc{n} & target\\
& \textitbf{tas} & \textstyleChCharisSIL{ˈtɐs} & \textsc{n} & bag\\
& \textitbf{taykondo} & \textstyleChCharisSIL{tɐj.ˈkɔ̞n.dɔ} & \textsc{n} & taekwondo\\
& \textitbf{te} & \textstyleChCharisSIL{ˈtɛ} & \textsc{n} & tea\\
& \textitbf{teko} & \textstyleChCharisSIL{ˈtɛ.kɔ} & \textsc{n} & teapot\\
& \textitbf{telaga} & \textstyleChCharisSIL{tɛ.ˈla.ga} & \textsc{n} & lake\\
& \textitbf{telefisi} & \textstyleChCharisSIL{ˌtɛ.lɛ.ˈfi.si} & \textsc{n} & television\\
& \textitbf{telpon} & \textstyleChCharisSIL{ˈtɛ̞l.pɔ̞n} & \textsc{v.bi} & phone\\
& \textitbf{tembaga} & \textstyleChCharisSIL{tɛ̞m.ˈba.ga} & \textsc{n} & copper\\
& \textitbf{tempo} & \textstyleChCharisSIL{ˈtɛ̞m.pɔ} & \textsc{v.mo(st)} & be quick\\
& \textitbf{tempramen} & \textstyleChCharisSIL{tɛ̞m.ˈpɾa.mɛ̞n} & \textsc{n} & temperament\\
& \textitbf{tempres} & \textstyleChCharisSIL{ˈtɛ̞m.pɾɛ̞s} & \textsc{n} & medical compress\\
& \textitbf{tenda} & \textstyleChCharisSIL{ˈtɛ̞n.da} & \textsc{n} & tent\\
& \textitbf{tengki} & \textstyleChCharisSIL{ˈtɛ̞ŋ.ki} & \textsc{n} & tank\\
& \textitbf{tenis} & \textstyleChCharisSIL{ˈtɛ.nɪs} & \textsc{n} & tennis\\
& \textitbf{tentara} & \textstyleChCharisSIL{tɛ̞n.ˈta.ɾa} & \textsc{n} & soldier\\
\textstyleExampleSource{x} & \textitbf{teologia} & \textstyleChCharisSIL{ˌtɛ.ɔ.ˈlɔ.gɪ.ˌa} & \textsc{n} & theology\\
& \textitbf{teras} & \textstyleChCharisSIL{ˈtɛ.ɾɐs} & \textsc{n} & porch\\
& \textitbf{termos} & \textstyleChCharisSIL{ˈtɛ̞r.mɔ̞s} & \textsc{n} & thermos bottle\\
\textstyleExampleSource{x} & \textitbf{terpal} & \textstyleChCharisSIL{tɛ̞r.ˈpɐl} & \textsc{n} & canvas\\
& \textitbf{terpol} & \textstyleChCharisSIL{ˈtɛ̞r.pɔ̞l} & \textsc{n} & container\\
& \textitbf{tes} & \textstyleChCharisSIL{ˈtɛ̞s} & \textsc{v.bi} & test\\
& \textitbf{tifa} & \textstyleChCharisSIL{ˈti.fa} & \textsc{n} & k. o. drum\\
& \textitbf{tim} & \textstyleChCharisSIL{ˈtɪm} & \textsc{n} & delegation\\
& \textitbf{tipe} & \textstyleChCharisSIL{ˈti.pɛ} & \textsc{n} & type\\
& \textitbf{to} & \textstyleChCharisSIL{ˈtɔ} & \textsc{tag} & right?\\
& \textitbf{toa} & \textstyleChCharisSIL{ˈtɔ̞.a} & \textsc{n} & field loudspeaker\\
& \textitbf{tobat} & \textstyleChCharisSIL{ˈtɔ.bɐt} & \textsc{v.mo(dy)} & repent\\
& \textitbf{toko} & \textstyleChCharisSIL{ˈtɔ.kɔ} & \textsc{n} & shop\\
& \textitbf{top} & \textstyleChCharisSIL{ˈtɔ̞p} & \textsc{v.mo(st)} & be good\\
& \textitbf{topi} & \textstyleChCharisSIL{ˈtɔ.pi} & \textsc{n} & hat\\
& \textitbf{toser} & \textstyleChCharisSIL{ˈtɔ.sɛ̞r} & \textsc{v.bi} & pass ball\\
& \textitbf{tradisi} & \textstyleChCharisSIL{tra.ˈdɪ.si} & \textsc{n} & tradition\\
& \textitbf{transfer} & \textstyleChCharisSIL{ˈtɾɐns.fɛ̞r} & \textsc{v.bi} & transfer\\
& \textitbf{trawma} & \textstyleChCharisSIL{ˈtrɐw.ma} & \textsc{n} & trauma\\
& \textitbf{trek} & \textstyleChCharisSIL{ˈtrɛ̞k} & \textsc{n} & truck\\
& \textitbf{trening} & \textstyleChCharisSIL{ˈtɾɛ.nɪŋ} & \textsc{n} & tracksuit\\
& \textitbf{trilyun} & \textstyleChCharisSIL{ˈtrɪl.jʊn} & \textsc{num.c} & trillion\\
& \textstyleChBold{U} &  &  & \\
& \textitbf{umat} & \textstyleChCharisSIL{ˈʊ.mɐt} & \textsc{n} & congregation\\
& \textitbf{umum} & \textstyleChCharisSIL{ˈʊ.mʊm} & \textsc{v.mo(st)} & be public\\
& \textitbf{umur} & \textstyleChCharisSIL{ˈʊ.mʊr} & \textsc{n} & age\\
& \textitbf{ungsi} & \textstyleChCharisSIL{ˈʊŋ.si} & \textsc{v.bi} & flee\\
\textstyleExampleSource{x} & \textitbf{unifersitas} & \textstyleChCharisSIL{ˌu.ni.ˌfɛ̞r.si.ˈtɐs} & \textsc{n} & university\\
& \textitbf{usaha} & \textstyleChCharisSIL{u.ˈsa.ha} & \textsc{v.bi} & attempt\\
& \textitbf{usia} & \textstyleChCharisSIL{u.ˈsi.a} & \textsc{n} & age\\
& \textitbf{usul} & \textstyleChCharisSIL{ˈʊ.sʊl} & \textsc{n} & proposal\\
& \textitbf{utama} & \textstyleChCharisSIL{u.ˈta.ma} & \textsc{v.mo(st)} & be prominent\\
& \textstyleChBold{W} &  &  & \\
& \textitbf{wakil} & \textstyleChCharisSIL{ˈwa.kɪl} & \textsc{n} & deputy\\
& \textitbf{waktu} & \textstyleChCharisSIL{ˈwɐk̚.tu} & \textsc{n} & time\\
& \textitbf{walikota} & \textstyleChCharisSIL{ˌwa.li.ˈkɔ.ta} & \textsc{n} & mayor\\
& \textitbf{wanita} & \textstyleChCharisSIL{wa.ˈni.ta} & \textsc{n} & woman\\
& \textitbf{warna} & \textstyleChCharisSIL{ˈwɐr.na} & \textsc{n} & color\\
& \textitbf{wasit} & \textstyleChCharisSIL{ˈwa.sɪt} & \textsc{n} & referee\\
& \textitbf{wawancara} & \textstyleChCharisSIL{ˌwa.wɐn.ˈtʃa.ɾa} & \textsc{v.bi} & interview\\
& \textitbf{wesel} & \textstyleChCharisSIL{ˈwɛ.sɛ̞l} & \textsc{v.bi} & transfer money\\
& \textitbf{wilaya} & \textstyleChCharisSIL{wi.ˈla.ja} & \textsc{n} & district\\
& \textitbf{wisuda} & \textstyleChCharisSIL{wi.ˈsu.da} & \textsc{n} & graduation ceremony\\
& \textstyleChBold{Y} &  &  & \\
& \textitbf{yakin} & \textstyleChCharisSIL{ˈja.kɪn} & \textsc{v.mo(st)} & be certain\\
& \textitbf{yatim} & \textstyleChCharisSIL{ˈja.tɪm} & \textsc{v.mo(st)} & be fatherless\\
& \textitbf{yayasan} & \textstyleChCharisSIL{ja.ˈja.sɐn} & \textsc{n} & foundation\\
& \textitbf{yo} & \textstyleChCharisSIL{ˈjɔ} & \textsc{adv} & yes\\
\lspbottomrule
\end{tabular}
\subsection{Lexical items historically derived by (unproductive) affixation}
\label{bkm:Ref376621295}
\tablehead{ & Lexeme & Transcription & Word class & English gloss\\
}
\begin{tabular}{lllll} & \textstyleChBold{A} &  &  & \\
\lsptoprule
& \textitbf{anaang} & \textstyleChCharisSIL{a.ˈna.kɐn} & \textsc{n} & offspring\\
& \textstyleChBold{B} &  &  & \\
& \textitbf{babingung} & \textstyleChCharisSIL{ba.ˈbi.ŋʊŋ} & \textsc{v.mo(st)} & be confused\\
& \textitbf{badani} & \textstyleChCharisSIL{ba.ˈda.ni} & \textsc{v.mo(st)} & be physical\\
& \textitbf{badara} & \textstyleChCharisSIL{ba.ˈda.ɾa} & \textsc{v.mo(st)} & be bloody\\
& \textitbf{badiam} & \textstyleChCharisSIL{ba.ˈdi.ɐm} & \textsc{v.mo(st)} & be quiet\\
& \textitbf{baduri} & \textstyleChCharisSIL{ba.ˈdu.ɾi} & \textsc{v.mo(st)} & be thorny\\
& \textitbf{bergaya} & \textstyleChCharisSIL{ba.ˈga.ja} & \textsc{v.mo(dy)} & put on airs\\
& \textitbf{bagigit} & \textstyleChCharisSIL{bɛ̞r.ˈgi.gɪt̚} & \textsc{v.mo(dy)} & bite\\
& \textitbf{baisi} & \textstyleChCharisSIL{ba.ˈɪ.si} & \textsc{v.mo(st)} & be muscular\\
& \textitbf{bajalang} & \textstyleChCharisSIL{ba.ˈdʒa.lɐn} & \textsc{v.mo(dy)} & walk\\
& \textitbf{bakumis} & \textstyleChCharisSIL{ba.ˈku.mɪs} & \textsc{v.mo(st)} & be with beard\\
& \textitbf{bangungang} & \textstyleChCharisSIL{ba.ˈŋu.ŋɐn} & \textsc{n} & building\\
& \textitbf{bantuang} & \textstyleChCharisSIL{bɐn.ˈtʊ.ɐn} & \textsc{n} & help\\
& \textitbf{baribut} & \textstyleChCharisSIL{ba.ˈɾi.bʊt̚} & \textsc{v.bi} & trouble\\
& \textitbf{bayangang} & \textstyleChCharisSIL{ba.ˈja.ŋɐn} & \textsc{n} & shadow\\
& \textitbf{bayangkang} & \textstyleChCharisSIL{ba.ˈjɐŋ.kɐn} & \textsc{v.bi} & imagine\\
& \textitbf{bebrapa} & \textstyleChCharisSIL{bɛ.ˈbra.pa} & \textsc{qt} & be several\\
x & \textitbf{berbentuk} & \textstyleChCharisSIL{ˌbɛ̞r.bɛ̞n.ˈtʊk̚} & \textsc{v.mo(st)} & be with shape of\\
& \textitbf{berbua} & \textstyleChCharisSIL{bɛ̞r.ˈbu.a} & \textsc{v.mo(st)} & be with fruit\\
& \textitbf{berbuat} & \textstyleChCharisSIL{bɛ̞r.ˈbʊ.ɐt̚} & \textsc{v.bi} & make\\
& \textitbf{berburu} & \textstyleChCharisSIL{bɛ̞r.ˈbu.ɾu} & \textsc{v.bi} & hunt\\
& \textitbf{berdasarkang} & \textstyleChCharisSIL{ˌbɛ̞r.da.ˈsɐr.kɐn} & \textsc{v.mo(st)} & be based on\\
x & \textitbf{berdebar} & \textstyleChCharisSIL{ˌbɛ̞r.dɛ.ˈbɐr} & \textsc{v.mo(dy)} & pulsate\\
& \textitbf{berdiri} & \textstyleChCharisSIL{bɛ̞r.ˈdi.ɾi} & \textsc{v.mo(dy)} & stand\\
x & \textitbf{berempat} & \textstyleChCharisSIL{ˌbɛ.ɾɛ̞m.ˈpɐt̚} & \textsc{v.mo(st)} & be four\\
& \textitbf{bergabung} & \textstyleChCharisSIL{bɛ̞r.ˈga.bʊŋ} & \textsc{v.mo(dy)} & join\\
& \textitbf{bergaul} & \textstyleChCharisSIL{bɛ̞r.ˈga.ʊl} & \textsc{v.mo(dy)} & associate\\
x & \textitbf{bergrak} & \textstyleChCharisSIL{bɛ̞r.ˈgrɐk̚} & \textsc{v.mo(dy)} & move\\
& \textitbf{bergumul} & \textstyleChCharisSIL{bɛ̞r.ˈgu.mʊl} & \textsc{v.mo(dy)} & struggle\\
& \textitbf{berharap} & \textstyleChCharisSIL{bɛ̞r.ˈha.ɾɐp̚} & \textsc{v.bi} & hope\\
& \textitbf{berhasil} & \textstyleChCharisSIL{bɛ̞r.ˈha.sɪl} & \textsc{v.mo(dy)} & succeed\\
& \textitbf{berhubungang} & \textstyleChCharisSIL{ˌbɛ̞r.hu.ˈbu.ŋɐn} & \textsc{v.mo(dy)} & have sexual intercourse\\
& \textitbf{berjuang} & \textstyleChCharisSIL{bɛ̞r.ˈdʒʊ.ɐŋ} & \textsc{v.mo(dy)} & struggle\\
x & \textitbf{berkebung} & \textstyleChCharisSIL{ˌbɛ̞r.kɛ̞.ˈbʊn} & \textsc{v.mo(dy)} & do farming\\
& \textitbf{berkumpul} & \textstyleChCharisSIL{bɛ̞r.ˈkʊm.pʊl} & \textsc{v.mo(dy)} & gather\\
& \textitbf{berlabu} & \textstyleChCharisSIL{bɛ̞r.ˈla.bu} & \textsc{v.mo(dy)} & anchor\\
& \textitbf{berlaku} & \textstyleChCharisSIL{bɛ̞r.ˈla.ku} & \textsc{v.mo(st)} & be valid\\
& \textitbf{berlindung} & \textstyleChCharisSIL{bɛ̞r.ˈlɪn.dʊŋ} & \textsc{v.mo(dy)} & take shelter\\
& \textitbf{bermaing} & \textstyleChCharisSIL{bɛ̞r.ˈma.ɪn} & \textsc{v.mo(dy)} & play\\
& \textitbf{bermalam} & \textstyleChCharisSIL{bɛ̞r.ˈma.lɐm} & \textsc{v.mo(dy)} & overnight\\
& \textitbf{berpakeang} & \textstyleChCharisSIL{ˌbɛ̞r.pa.ˈkɛ.ɐn} & \textsc{v.mo(st)} & be with clothes\\
& \textitbf{berpikir} & \textstyleChCharisSIL{bɛ̞r.ˈpi.kɪr} & \textsc{v.mo(dy)} & think\\
& \textitbf{berpisa} & \textstyleChCharisSIL{bɛ̞r.ˈpi.sa} & \textsc{v.mo(st)} & be separate\\
& \textitbf{bersaling} & \textstyleChCharisSIL{bɛ̞r.ˈsa.lɪŋ} & \textsc{v.mo(dy)} & give birth\\
& \textitbf{bersandar} & \textstyleChCharisSIL{bɛ̞r.ˈsɐn.dɐr} & \textsc{v.mo(dy)} & lean\\
& \textitbf{bersangkutang} & \textstyleChCharisSIL{ˌbɛ̞r.sɐŋ.ˈku.tɐn} & \textsc{v.mo(st)} & be concerned with\\
& \textitbf{bersatu} & \textstyleChCharisSIL{bɛ̞r.ˈsa.tu} & \textsc{v.mo(st)} & be one\\
& \textitbf{bersina} & \textstyleChCharisSIL{bɛ̞r.ˈsi.na} & \textsc{v.mo(dy)} & commit adultery\\
& \textitbf{bertahang} & \textstyleChCharisSIL{bɛ̞r.ˈta.hɐn} & \textsc{v.mo(dy)} & hold (out/back)\\
x & \textitbf{bertemang} & \textstyleChCharisSIL{ˌbɛ̞r.tɛ̞.ˈmɐn} & \textsc{v.mo(st)} & be friends\\
& \textitbf{bertemu} & \textstyleChCharisSIL{bɛ̞r.tɛ̞.ˈmu} & \textsc{v.mo(dy)} & meet\\
& \textitbf{bertengkar} & \textstyleChCharisSIL{bɛ̞r.tɛ̞ŋ.ˈkɐr} & \textsc{v.mo(dy)} & quarrel\\
& \textitbf{bertentangang} & \textstyleChCharisSIL{ˌbɛ̞r.tɛ̞n.ˈta.ŋɐn} & \textsc{v.mo(st)} & be in conflict\\
& \textitbf{bertindak} & \textstyleChCharisSIL{bɛ̞r.ˈtɪn.dɐk} & \textsc{v.mo(dy)} & act\\
& \textitbf{bertriak} & \textstyleChCharisSIL{ˌba.ta.ˈɾɪ.a} & \textsc{v.bi} & scream (at)\\
& \textitbf{bertukarang} & \textstyleChCharisSIL{ˌbɛ̞r.tu.ˈka.ɾɐn} & \textsc{v.mo(dy)} & mutually exchange\\
& \textitbf{brade} & \textstyleChCharisSIL{ˈbɾa.dɛ} & \textsc{v.mo(st)} & be younger sibling\\
& \textitbf{brali} & \textstyleChCharisSIL{ˈbɾa.li} & \textsc{v.mo(dy)} & shift\\
& \textitbf{branak} & \textstyleChCharisSIL{ˈbɾa.nɐk̚} & \textsc{v.bi} & give birth (to)\\
& \textitbf{brangkat} & \textstyleChCharisSIL{ˈbrɐŋ.kɐt̚} & \textsc{v.mo(dy)} & leave\\
& \textitbf{brenti} & \textstyleChCharisSIL{ˈbrɛ̞n.ti} & \textsc{v.mo(dy)} & stop\\
& \textitbf{brikut} & \textstyleChCharisSIL{ˈbri.kʊt} & \textsc{v.mo(dy)} & follow\\
& \textitbf{brontakkang} & \textstyleChCharisSIL{bɾɔ̞n.ˈta.kɐn} & \textsc{v.bi} & fight\\
& \textitbf{bruba} & \textstyleChCharisSIL{ˈbru.ba} & \textsc{v.mo(dy)} & change\\
& \textitbf{buatang} & \textstyleChCharisSIL{bʊ.ˈa.tɐn} & \textsc{n} & deed\\
& \textitbf{buruang} & \textstyleChCharisSIL{bu.ˈɾʊ.ɐn} & \textsc{n} & prey\\
& \textstyleChBold{C} &  &  & \\
& \textitbf{cadangang} & \textstyleChCharisSIL{tʃa.ˈda.ŋɐn} & \textsc{n} & reserve\\
& \textitbf{campurang} & \textstyleChCharisSIL{tʃɐm.ˈpu.ɾɐn} & \textsc{n} & mixture\\
& \textitbf{catatang} & \textstyleChCharisSIL{tʃa.ˈta.tɐn} & \textsc{n} & note\\
& \textitbf{cobaang} & \textstyleChCharisSIL{tʃɔ.ˈba.ɐn} & \textsc{n} & trial\\
& \textstyleChBold{D} &  &  & \\
& \textitbf{didikang} & \textstyleChCharisSIL{di.ˈdi.kɐn} & \textsc{n} & upbringing\\
& \textitbf{duluang} & \textstyleChCharisSIL{dʊ.ˈlʊ.ɐn} & \textsc{v.mo(st)} & be prior to others\\
& \textstyleChBold{G} &  &  & \\
& \textitbf{gambarang} & \textstyleChCharisSIL{gɐm.ˈba.ɾɐn} & \textsc{n} & illustration\\
& \textitbf{gangguang} & \textstyleChCharisSIL{gɐŋ.ˈgʊ.ɐn} & \textsc{n} & disturbance\\
& \textitbf{golongang} & \textstyleChCharisSIL{gɔ̞.ˈlɔ̞.ŋɐn} & \textsc{n} & group\\
& \textitbf{grakang} & \textstyleChCharisSIL{ˈgra.kɐn} & \textsc{n} & movement\\
& \textstyleChBold{H} &  &  & \\
& \textitbf{halangang} & \textstyleChCharisSIL{ha.ˈla.ŋɐn} & \textsc{n} & hindrance\\
& \textitbf{harapang} & \textstyleChCharisSIL{ha.ˈɾa.pɐn} & \textsc{n} & hope\\
& \textitbf{harapkang} & \textstyleChCharisSIL{ha.ˈɾɐp̚.kɐn} & \textsc{v.bi} & hope for\\
& \textitbf{harusnya} & \textstyleChCharisSIL{ha.ˈɾʊs.ɲa} & \textsc{adv} & appropriately\\
& \textitbf{hubungang} & \textstyleChCharisSIL{hu.ˈbu.ŋɐn} & \textsc{n} & connection\\
& \textitbf{hubungi} & \textstyleChCharisSIL{hu.ˈbu.ŋi} & \textsc{v.bi} & contact\\
& \textstyleChBold{I} &  &  & \\
& \textitbf{ikatang} & \textstyleChCharisSIL{i.ˈka.tɐn} & \textsc{n} & tie\\
& \textitbf{ingatang} & \textstyleChCharisSIL{i.ˈŋa.tɐn} & \textsc{n} & memory\\
& \textstyleChBold{J} &  &  & \\
& \textitbf{jabatang} & \textstyleChCharisSIL{dʒa.ˈba.tɐn} & \textsc{n} & position\\
& \textitbf{jahitang} & \textstyleChCharisSIL{dʒa.ˈhi.tɐn} & \textsc{n} & stitch\\
& \textitbf{jajaang} & \textstyleChCharisSIL{dʒa.ˈdʒa.ɐn} & \textsc{n} & colony\\
& \textitbf{jalangang} & \textstyleChCharisSIL{dʒa.ˈla.nɐn} & \textsc{n} & route\\
& \textitbf{jalangkang} & \textstyleChCharisSIL{dʒa.ˈlɐn.kɐn} & \textsc{v.bi} & put into operation\\
& \textitbf{jalani} & \textstyleChCharisSIL{dʒa.ˈla.ni} & \textsc{v.bi} & undergo\\
& \textitbf{jelaskang} & \textstyleChCharisSIL{dʒɛ.ˈlɐs.kɐn} & \textsc{v.bi} & explain\\
& \textitbf{jemputang} & \textstyleChCharisSIL{dʒɛ̞m.ˈpu.tɐn} & \textsc{n} & pick up service\\
& \textitbf{jualang} & \textstyleChCharisSIL{dʒʊ.ˈa.lɐn} & \textsc{n/v.bi} & merchandise / sell\\
& \textitbf{jurusang} & \textstyleChCharisSIL{dʒu.ˈɾu.sɐn} & \textsc{n} & department\\
& \textstyleChBold{K} &  &  & \\
& \textitbf{kabulkang} & \textstyleChCharisSIL{ka.ˈbʊl.kɐn} & \textsc{v.bi} & fulfill a request\\
& \textitbf{kasiang} & \textstyleChCharisSIL{ka.ˈsi.ɐn} & \textsc{n/v.bi} & pity\\
& \textitbf{keadaang} & \textstyleChCharisSIL{ˌkɛ.a.ˈda.ɐn} & \textsc{n} & condition\\
& \textitbf{kebaikang} & \textstyleChCharisSIL{ˌkɛ.ba.ˈɪ.kɐn} & \textsc{n} & goodness\\
& \textitbf{kebalikang} & \textstyleChCharisSIL{ˌkɛ.ba.ˈlɛ̞.ɐn} & \textsc{n} & opposite\\
& \textitbf{kebanyakang} & \textstyleChCharisSIL{ˌkɛ.ba.ˈɲa.kɐn} & \textsc{n} & majority\\
& \textitbf{kebenarang} & \textstyleChCharisSIL{ˌkɛ.bɛ.ˈna.ɾɐn} & \textsc{n} & truth\\
& \textitbf{kebetulang} & \textstyleChCharisSIL{ˌkɛ.bɛ.ˈtʊ.lɐn} & \textsc{n} & chance\\
& \textitbf{kebodoang} & \textstyleChCharisSIL{ˌkɛ.bɔ.ˈdɔ.ɐn} & \textsc{n} & stupidity\\
& \textitbf{kebrapa} & \textstyleChCharisSIL{kɛ.ˈbra.pa} & \textsc{int} & how manyeth\\
& \textitbf{kebutuang} & \textstyleChCharisSIL{ˌkɛ.bu.ˈtʊ.ɐn} & \textsc{n} & need\\
& \textitbf{kecamatang} & \textstyleChCharisSIL{ˌkɛ.tʃa.ˈma.tɐn} & \textsc{n} & subdistrict\\
& \textitbf{kedua} & \textstyleChCharisSIL{kɛ.ˈdu.a} & \textsc{num.o} & second\\
& \textitbf{kedudukang} & \textstyleChCharisSIL{ˌkɛ.du.ˈdu.kɐn} & \textsc{n} & position\\
x & \textitbf{keempat} & \textstyleChCharisSIL{ˌkɛ.ɛ̞m.ˈpɐt̚} & \textsc{num.o} & fourth\\
& \textitbf{kegiatang} & \textstyleChCharisSIL{ˌkɛ.gi.ˈa.tɐn} & \textsc{n} & activity\\
& \textitbf{keglapang} & \textstyleChCharisSIL{kɛ.ˈgla.pɐn} & \textsc{n} & darkness\\
& \textitbf{kehidupang} & \textstyleChCharisSIL{ˌkɛ.hi.ˈdu.pɐn} & \textsc{n} & life\\
& \textitbf{keingingang} & \textstyleChCharisSIL{ˌkɛ.i.ˈŋi.nɐn} & \textsc{n} & wish\\
& \textitbf{kejahatang} & \textstyleChCharisSIL{ˌkɛ.dʒa.ˈha.tɐn} & \textsc{n} & evilness\\
& \textitbf{kekurangang} & \textstyleChCharisSIL{ˌkɛ.ku.ˈɾa.ŋɐn} & \textsc{n} & shortage\\
& \textitbf{kelakuang} & \textstyleChCharisSIL{ˌkɛ.la.ˈkʊ.ɐn} & \textsc{n} & behavior\\
& \textitbf{kelaleyang} & \textstyleChCharisSIL{ˌkɛ.la.ˈlɛ.jɐn} & \textsc{n} & neglect\\
& \textitbf{kelebiang} & \textstyleChCharisSIL{ˌkɛ.lɛ.ˈbi.ɐn} & \textsc{n} & surplus\\
& \textitbf{kelemaang} & \textstyleChCharisSIL{ˌkɛ.lɛ.ˈma.ɐn} & \textsc{n} & weakness\\
& \textitbf{kelewatang} & \textstyleChCharisSIL{ˌkɛ.lɛ.ˈwa.tɐn} & \textsc{v.mo(acl)} & be overly abundant\\
& \textitbf{keliarang} & \textstyleChCharisSIL{ˌkɛ.li.ˈa.ɾɐn} & \textsc{v.mo(dy)} & roam about\\
& \textitbf{keliatang} & \textstyleChCharisSIL{ˌkɛ.li.ˈa.tɐn} & \textsc{v.mo(acl)} & be visible\\
& \textitbf{kemajuang} & \textstyleChCharisSIL{ˌkɛ.ma.ˈdʒu.ɐn} & \textsc{n} & progress\\
& \textitbf{kemaluang} & \textstyleChCharisSIL{ˌkɛ.ma.ˈlʊ.ɐn} & \textsc{n} & genitals\\
& \textitbf{kematiang} & \textstyleChCharisSIL{ˌkɛ.ma.ˈti.ɐn} & \textsc{n} & death\\
& \textitbf{kemawang} & \textstyleChCharisSIL{kɛ.ˈmaw.ʷɐn} & \textsc{n} & will\\
& \textitbf{kemenangang} & \textstyleChCharisSIL{ˌkɛ.mɛ.ˈna.ŋɐn} & \textsc{n} & victory\\
& \textitbf{kenalang} & \textstyleChCharisSIL{kɛ.ˈna.lɐn} & \textsc{n} & acquaintance\\
& \textitbf{kendaraang} & \textstyleChCharisSIL{ˌkɛ̞n.da.ˈɾa.ɐn} & \textsc{n} & vehicle\\
& \textitbf{kepentingang} & \textstyleChCharisSIL{ˌkɛ.pɛ̞n.ˈti.ŋɐn} & \textsc{n} & importance\\
& \textitbf{keputusang} & \textstyleChCharisSIL{ˌkɛ.pu.ˈtu.sɐn} & \textsc{n} & decision\\
& \textitbf{kesadarang} & \textstyleChCharisSIL{ˌkɛ.sa.ˈda.ɾɐn} & \textsc{n} & awareness\\
& \textitbf{kesalaang} & \textstyleChCharisSIL{ˌkɛ.sa.ˈla.ɐn} & \textsc{n} & mistake\\
& \textitbf{kesehatang} & \textstyleChCharisSIL{ˌkɛ.sɛ.ˈha.tɐn} & \textsc{n} & health\\
& \textitbf{kesempatang} & \textstyleChCharisSIL{ˌkɛ.sɛ̞m.ˈpa.tɐn} & \textsc{n} & opportunity\\
& \textitbf{kesulitang} & \textstyleChCharisSIL{ˌkɛ.su.ˈli.tɐn} & \textsc{n} & difficulty\\
& \textitbf{ketakutang} & \textstyleChCharisSIL{ˌkɛ.ta.ˈku.tɐn} & \textsc{n} & fear\\
& \textitbf{ketawa} & \textstyleChCharisSIL{kɛ.ˈta.wa} & \textsc{v.bi} & laugh\\
& \textitbf{ketawang} & \textstyleChCharisSIL{kɛ.ˈtaw.ʷɐn} & \textsc{v.mo(acl)} & be found out\\
x & \textitbf{ketemu} & \textstyleChCharisSIL{ˌkɛ.tɛ.ˈmu} & \textsc{v.bi} & meet\\
& \textitbf{ketiga} & \textstyleChCharisSIL{kɛ.ˈti.ga} & \textsc{num.o} & third\\
& \textitbf{ketindisang} & \textstyleChCharisSIL{ˌkɛ.tɪn.ˈdi.sɐn} & \textsc{n} & k. o. trap\\
& \textitbf{ketinggalang} & \textstyleChCharisSIL{ˌkɛ.tɪŋ.ˈga.lɐn} & \textsc{v.mo(acl)} & be left behind\\
& \textitbf{ketrangang} & \textstyleChCharisSIL{kɛ.ˈtɾa.ŋɐn} & \textsc{n} & explanation\\
& \textitbf{ketua} & \textstyleChCharisSIL{kɛ.ˈtʊ.a} & \textsc{n} & chairperson\\
& \textitbf{keturungang} & \textstyleChCharisSIL{ˌkɛ.tu.ˈɾu.nɐn} & \textsc{n} & descendant\\
& \textitbf{keuntungang} & \textstyleChCharisSIL{ˌkɛ.ʊn.ˈtʊ.ŋɐn} & \textsc{n} & advantage\\
& \textitbf{kunjungang} & \textstyleChCharisSIL{kʊn.ˈdʒu.ŋɐn} & \textsc{n} & visit\\
& \textitbf{kunjungi} & \textstyleChCharisSIL{kʊn.ˈdʒu.ŋi} & \textsc{v.bi} & visit\\
& \textitbf{kutukang} & \textstyleChCharisSIL{ku.ˈtu.kɐn} & \textsc{n} & curse\\
& \textstyleChBold{L} &  &  & \\
& \textitbf{lalapang} & \textstyleChCharisSIL{la.ˈla.pɐn} & \textsc{n} & k. o. vegetable dish\\
& \textitbf{lamarang} & \textstyleChCharisSIL{la.ˈma.ɾɐn} & \textsc{n} & application, proposal\\
& \textitbf{lapangang} & \textstyleChCharisSIL{la.ˈpa.ŋɐn} & \textsc{n} & field\\
& \textitbf{latiang} & \textstyleChCharisSIL{la.ˈti.ɐn} & \textsc{n/v.bi} & practice\\
& \textitbf{lautang} & \textstyleChCharisSIL{la.ˈʊ.tɐn} & \textsc{n} & ocean\\
& \textitbf{layani} & \textstyleChCharisSIL{la.ˈja.ni} & \textsc{v.bi} & serve\\
& \textitbf{liburang} & \textstyleChCharisSIL{li.ˈbu.ɾɐn} & \textsc{n} & vacation\\
& \textitbf{lingkarang} & \textstyleChCharisSIL{lɪŋ.ˈka.ɾɐn} & \textsc{n} & circle\\
& \textstyleChBold{M} &  &  & \\
& \textitbf{maingang} & \textstyleChCharisSIL{ma.ˈɪ.nɐn} & \textsc{n} & toy\\
& \textitbf{makangang} & \textstyleChCharisSIL{ma.ˈka.nɐn} & \textsc{n} & food\\
& \textitbf{makanya} & \textstyleChCharisSIL{ma.ˈka.ɲa} & \textsc{adv} & for that reason\\
& \textitbf{masakang} & \textstyleChCharisSIL{ma.ˈsa.kɐn} & \textsc{n} & cooking\\
& \textitbf{melalui} & \textstyleChCharisSIL{ˌmɛ.la.ˈlu.i} & \textsc{v.bi} & pass by\\
& \textitbf{melamar} & \textstyleChCharisSIL{mɛ.ˈla.mɐr} & \textsc{v.bi} & apply for, propose\\
& \textitbf{melancong} & \textstyleChCharisSIL{mɛ.ˈlɐn.tʃɔ̞ŋ} & \textsc{v.mo(dy)} & take a pleasure trip\\
& \textitbf{melawang} & \textstyleChCharisSIL{mɛ.ˈla.wɐn} & \textsc{v.bi} & oppose\\
& \textitbf{melayani} & \textstyleChCharisSIL{ˌmɛ.la.ˈja.ni} & \textsc{v.bi} & serve\\
x & \textitbf{melekat} & \textstyleChCharisSIL{ˌmɛ.lɛ.ˈkɐt̚} & \textsc{v.mo(dy)} & stick\\
& \textitbf{meleset} & \textstyleChCharisSIL{mɛ.ˈlɛ̞.sɛ̞t̚} & \textsc{v.mo(dy)} & miss a target\\
& \textitbf{melintang} & \textstyleChCharisSIL{mɛ.ˈlɪn.tɐŋ} & \textsc{v.mo(dy)} & lie across\\
& \textitbf{melulu} & \textstyleChCharisSIL{mɛ.ˈlu.lu} & \textsc{adv} & exclusively\\
& \textitbf{menangis} & \textstyleChCharisSIL{mɛ.ˈna.ŋɪs} & \textsc{v.bi} & cry (for)\\
& \textitbf{menari} & \textstyleChCharisSIL{mɛ.ˈna.ɾi} & \textsc{v.mo(dy)} & dance\\
& \textitbf{mendadak} & \textstyleChCharisSIL{mɛ̞n.ˈda.dɐk̚} & \textsc{v.mo(st)} & be sudden\\
& \textitbf{mendarat} & \textstyleChCharisSIL{mɛ̞n.ˈda.ɾɐt} & \textsc{v.mo(dy)} & land\\
& \textitbf{mendidi} & \textstyleChCharisSIL{mɛ̞n.ˈdi.di} & \textsc{v.mo(dy)} & boil\\
& \textitbf{mendukung} & \textstyleChCharisSIL{mɛ̞n.ˈdʊ.kʊŋ} & \textsc{v.bi} & support\\
& \textitbf{mengaku} & \textstyleChCharisSIL{mɛ.ˈŋa.ku} & \textsc{v.bi} & confess\\
& \textitbf{mengala} & \textstyleChCharisSIL{mɛ.ˈŋa.la} & \textsc{v.mo(dy)} & yield\\
& \textitbf{mengalir} & \textstyleChCharisSIL{mɛ.ˈŋa.lɪr} & \textsc{v.mo(dy)} & flow\\
& \textitbf{mengantuk} & \textstyleChCharisSIL{mɛ.ˈŋɐn.tʊk̚} & \textsc{v.mo(st)} & be sleepy\\
& \textitbf{mengasii} & \textstyleChCharisSIL{ˌmɛ.ŋa.ˈsi.i} & \textsc{v.bi} & love\\
& \textitbf{mengelu} & \textstyleChCharisSIL{mɛ.ˈŋɛ.lu} & \textsc{v.mo(dy)} & complain\\
& \textitbf{meninggal} & \textstyleChCharisSIL{mɛ.ˈnɪŋ.gɐl} & \textsc{v.mo(dy)} & die\\
& \textitbf{menjadi} & \textstyleChCharisSIL{mɛ̞n.ˈdʒa.di} & \textsc{v.bi} & become\\
& \textitbf{menjelang} & \textstyleChCharisSIL{mɛ̞n.dʒɛ.ˈlɐŋ} & \textsc{v.bi} & approach\\
& \textitbf{menuju} & \textstyleChCharisSIL{mɛ.ˈnu.dʒu} & \textsc{v.bi} & aim at\\
& \textitbf{menurut} & \textstyleChCharisSIL{mɛ.ˈnʊ.ɾʊt̚} & \textsc{prep} & according to\\
& \textitbf{menyala} & \textstyleChCharisSIL{mɛ.ˈɲa.la} & \textsc{v.mo(dy)} & shine\\
& \textitbf{menyangkal} & \textstyleChCharisSIL{mɛ.ˈɲɐŋ.kɐl} & \textsc{v.bi} & deny\\
& \textitbf{menyanyi} & \textstyleChCharisSIL{mɛ.ˈɲa.ɲi} & \textsc{v.bi} & sing\\
& \textitbf{menyapu} & \textstyleChCharisSIL{mɛ.ˈɲa.pu} & \textsc{v.bi} & sweep\\
x & \textitbf{menyebrang} & \textstyleChCharisSIL{ˌmɛ.ɲɛ.ˈbrɐŋ} & \textsc{v.bi} & cross\\
x & \textitbf{menyesal} & \textstyleChCharisSIL{ˌmɛ.ɲɛ.ˈsɐl} & \textsc{v.mo(st)} & regret\\
& \textitbf{menyusul} & \textstyleChCharisSIL{mɛ.ˈɲu.sʊl} & \textsc{v.bi} & follow\\
& \textitbf{merangkap} & \textstyleChCharisSIL{mɛ.ˈɾɐŋ.kɐp̚} & \textsc{v.bi} & double as\\
& \textitbf{merantaw} & \textstyleChCharisSIL{mɛ.ˈɾɐn.tɐw} & \textsc{v.mo(dy)} & wander about\\
& \textitbf{merayap} & \textstyleChCharisSIL{mɛ.ˈɾa.jɐp̚} & \textsc{v.bi} & creep (over)\\
& \textitbf{minumang} & \textstyleChCharisSIL{mi.ˈnu.mɐn} & \textsc{n} & beverage\\
& \textitbf{muatang} & \textstyleChCharisSIL{mʊ.ˈa.tɐn} & \textsc{n} & cargo, contents\\
& \textstyleChBold{O} &  &  & \\
& \textitbf{obatang} & \textstyleChCharisSIL{ɔ.ˈba.tɐn} & \textsc{n} & magic spell\\
& \textstyleChBold{P} &  &  & \\
& \textitbf{paginya} & \textstyleChCharisSIL{pa.ˈgi.ɲa} & \textsc{adv} & next morning\\
& \textitbf{pakeang} & \textstyleChCharisSIL{pa.ˈkɛ.ɐn} & \textsc{n} & clothes\\
& \textitbf{pamalas} & \textstyleChCharisSIL{pɛ.ˈma.lɐs} & \textsc{n/v.mo(st)} & listless person / be very listless\\
& \textitbf{pamerang} & \textstyleChCharisSIL{pa.ˈmɛ.ɾɐn} & \textsc{n} & exhibition\\
& \textitbf{panakut} & \textstyleChCharisSIL{pɛ.ˈna.kʊt̚} & \textsc{n/v.bi} & coward / be very fearful(.of)\\
& \textitbf{pandangang} & \textstyleChCharisSIL{pɐn.ˈda.ŋɐn} & \textsc{n} & view\\
& \textitbf{pandiam} & \textstyleChCharisSIL{pɛ̞n.ˈdi.ɐm} & \textsc{n/v.mo(st)} & taciturn person / be very quiet\\
& \textitbf{panggayu} & \textstyleChCharisSIL{pɐŋ.ˈga.ju} & \textsc{n/v.bi} & paddle\\
& \textitbf{panggilang} & \textstyleChCharisSIL{pɐŋ.ˈgi.lɐn} & \textsc{n} & call, summons\\
& \textitbf{pangkalang} & \textstyleChCharisSIL{pɐŋ.ˈka.lɐn} & \textsc{n} & base\\
& \textitbf{pasangang} & \textstyleChCharisSIL{pa.ˈsa.ŋɐn} & \textsc{n} & pair\\
& \textitbf{pedalamang} & \textstyleChCharisSIL{ˌpɛ.da.ˈla.mɐn} & \textsc{n} & interior\\
& \textitbf{pembangungang} & \textstyleChCharisSIL{ˌpɛ̞m.ba.ˈŋu.nɐn} & \textsc{n} & building\\
& \textitbf{pembantu} & \textstyleChCharisSIL{pəm.ˈbɐn.tu} & \textsc{n} & house helper\\
& \textitbf{pembayarang} & \textstyleChCharisSIL{ˌpɛ̞m.ba.ˈja.ɾɐn} & \textsc{n} & payment\\
& \textitbf{pembunuang} & \textstyleChCharisSIL{ˌpɛ̞m.bu.ˈnu.ɐn} & \textsc{n} & killing\\
& \textitbf{pemekarang} & \textstyleChCharisSIL{ˌpɛ.mɛ.ˈka.ɾɐn} & \textsc{n} & development\\
& \textitbf{pemiliang} & \textstyleChCharisSIL{ˌpɛ.mi.ˈli.ɐn} & \textsc{n} & election\\
& \textitbf{pemimping} & \textstyleChCharisSIL{pɛ.ˈmɪm.pɪn} & \textsc{n} & leader\\
& \textitbf{pemrinta} & \textstyleChCharisSIL{pəm.ˈrɪn.ta} & \textsc{n} & government\\
& \textitbf{pemrintaang} & \textstyleChCharisSIL{ˌpəm.rɪn.ˈta.ɐn} & \textsc{n} & governance\\
& \textitbf{pemuda} & \textstyleChCharisSIL{pɛ.ˈmu.da} & \textsc{n} & youth\\
& \textitbf{penani} & \textstyleChCharisSIL{pɛ.ˈna.ni} & \textsc{n} & farmer\\
& \textitbf{penantar} & \textstyleChCharisSIL{pɛ.ˈnɐn.tɐr} & \textsc{n} & escort\\
& \textitbf{penasarang} & \textstyleChCharisSIL{ˌpɛ.na.ˈsa.ɾɐn} & \textsc{v.mo(st)} & be curious\\
& \textitbf{pendatang} & \textstyleChCharisSIL{pɛ̞n.ˈda.tɐŋ} & \textsc{n} & stranger\\
& \textitbf{pendidikang} & \textstyleChCharisSIL{ˌpɛ̞n.di.ˈdi.kɐn} & \textsc{n} & education\\
& \textitbf{pendiriang} & \textstyleChCharisSIL{ˌpɛ̞n.di.ˈɾɪ.ɐn} & \textsc{n} & convictions\\
& \textitbf{penduduk} & \textstyleChCharisSIL{pɛ̞n.ˈdʊ.dʊk̚} & \textsc{n} & inhabitant\\
& \textitbf{penentuang} & \textstyleChCharisSIL{ˌpɛ.nɛ̞n.ˈtʊ.ɐn} & \textsc{n} & determination\\
& \textitbf{pengakuang} & \textstyleChCharisSIL{ˌpɛ.ŋa.ˈkʊ.ɐn} & \textsc{n} & confession\\
& \textitbf{pengarui} & \textstyleChCharisSIL{ˌpɛ.ŋa.ˈɾu.i} & \textsc{v.bi} & influence\\
& \textitbf{pengasu} & \textstyleChCharisSIL{pɛ.ˈŋa.su} & \textsc{n} & Sunday school teacher\\
& \textitbf{pengetawang} & \textstyleChCharisSIL{ˌpɛ.ŋɛ.ˈtaw.ʷɐn} & \textsc{n} & knowledge\\
& \textitbf{pengganti} & \textstyleChCharisSIL{pəŋ.ˈgɐn.ti} & \textsc{n} & replacement\\
& \textitbf{pengirimang} & \textstyleChCharisSIL{ˌpɛ.ŋi.ˈɾi.mɐn} & \textsc{n} & dispatch\\
& \textitbf{pengurus} & \textstyleChCharisSIL{pɛ.ˈŋʊ.ɾʊs} & \textsc{n} & manager\\
& \textitbf{peninju} & \textstyleChCharisSIL{pɛ.ˈnɪn.dʒu} & \textsc{n} & boxer\\
& \textitbf{penjelasang} & \textstyleChCharisSIL{ˌpɛ̞n.dʒɛ.ˈla.sɐn} & \textsc{n} & explanation\\
& \textitbf{penolakang} & \textstyleChCharisSIL{ˌpɛ.nɔ.ˈla.kɐn} & \textsc{n} & rejection\\
& \textitbf{penolong} & \textstyleChCharisSIL{pɛ.ˈnɔ̞.lɔ̞ŋ} & \textsc{n} & helper\\
& \textitbf{penugas} & \textstyleChCharisSIL{pɛ.ˈnu.gɐs} & \textsc{n} & official\\
& \textitbf{penumpang} & \textstyleChCharisSIL{pɛ.ˈnʊm.pɐŋ} & \textsc{n} & passenger\\
& \textitbf{penunggu} & \textstyleChCharisSIL{pɛ.ˈnʊŋ.gʊ} & \textsc{n} & tutelary spirit\\
& \textitbf{penunjuk} & \textstyleChCharisSIL{pɛ.ˈnʊn.dʒʊk̚} & \textsc{n} & guide\\
& \textitbf{penutupang} & \textstyleChCharisSIL{ˌpɛ.nu.ˈtu.pɐn} & \textsc{n} & closure\\
& \textitbf{penyalaang} & \textstyleChCharisSIL{ˌpɛ.ɲa.ˈla.ɐn} & \textsc{n} & ignition\\
& \textitbf{penyeraang} & \textstyleChCharisSIL{ˌpɛ.ɲɛ.ˈɾa.ɐn} & \textsc{n} & dedication\\
& \textitbf{penyesalang} & \textstyleChCharisSIL{ˌpɛ.ɲɛ.ˈsa.lɐn} & \textsc{n} & remorse\\
& \textitbf{perbedaang} & \textstyleChCharisSIL{ˌpɛ̞r.bɛ.ˈda.ɐn} & \textsc{n} & difference\\
& \textitbf{perbuatang} & \textstyleChCharisSIL{ˌpɛ̞r.bʊ.ˈa.tɐn} & \textsc{n} & act, action\\
& \textitbf{perhitungang} & \textstyleChCharisSIL{ˌpɛ̞r.hi.ˈtu.ŋɐn} & \textsc{n} & calculation\\
& \textitbf{peringatang} & \textstyleChCharisSIL{ˌpɛ.ɾi.ˈŋa.tɐn} & \textsc{n} & reminder, warning\\
& \textitbf{perjalangang} & \textstyleChCharisSIL{ˌpɛ̞r.dʒa.ˈla.nɐn} & \textsc{n} & journey\\
& \textitbf{perjanjiang} & \textstyleChCharisSIL{ˌpɛ̞r.dʒɐn.ˈdʒi.ɐn} & \textsc{n} & promise\\
& \textitbf{perlengkapang} & \textstyleChCharisSIL{ˌpɛ̞r.lɛ̞ŋ.ˈka.pɐn} & \textsc{n} & equipment\\
& \textitbf{permaingang} & \textstyleChCharisSIL{ˌpɛ̞r.ma.ˈɪ.nɐn} & \textsc{n} & game\\
& \textitbf{persiapang} & \textstyleChCharisSIL{ˌpɛ̞r.si.ˈa.pɐn} & \textsc{n} & preparation\\
& \textitbf{pertahangang} & \textstyleChCharisSIL{ˌpɛ̞r.ta.ˈha.nɐn} & \textsc{n} & defense\\
& \textitbf{pertandingang} & \textstyleChCharisSIL{ˌpɛ̞r.tɐn.ˈdi.ŋɐn} & \textsc{n} & competition\\
& \textitbf{pertanyaang} & \textstyleChCharisSIL{ˌpɛ̞r.ta.ˈɲa.ɐn} & \textsc{n} & question\\
& \textitbf{pertemuang} & \textstyleChCharisSIL{ˌpɛ̞r.tɛ.ˈmʊ.ɐn} & \textsc{n} & meeting\\
& \textitbf{pertumbuang} & \textstyleChCharisSIL{ˌpɛ̞r.tʊm.ˈbʊ.ɐn} & \textsc{n} & growth\\
& \textitbf{picaang} & \textstyleChCharisSIL{pi.ˈtʃa.ɐn} & \textsc{n} & splinter\\
& \textitbf{pimpingang} & \textstyleChCharisSIL{pɪm.ˈpi.nɐn} & \textsc{n} & leadership\\
& \textitbf{plabuang} & \textstyleChCharisSIL{pla.ˈbu.ɐn} & \textsc{n} & harbor\\
& \textitbf{plantikang} & \textstyleChCharisSIL{plɐn.ˈti.kɐn} & \textsc{n} & inauguration\\
& \textitbf{platiang} & \textstyleChCharisSIL{pla.ˈtɪ.ɐn} & \textsc{n} & training\\
& \textitbf{playangang} & \textstyleChCharisSIL{pla.ˈja.nɐn} & \textsc{n} & service\\
& \textitbf{pokoknya} & \textstyleChCharisSIL{pɔ.ˈkɔ̞k̚.ɲa} & \textsc{adv} & the main thing is\\
& \textitbf{praliang} & \textstyleChCharisSIL{pɾa.ˈlɪ.ɐn} & \textsc{n} & transition\\
& \textitbf{pranaang} & \textstyleChCharisSIL{pra.ˈna.kɐn} & \textsc{n} & mixed ethnic origins\\
& \textitbf{praturang} & \textstyleChCharisSIL{pra.ˈtu.ɾɐn} & \textsc{n} & regulation\\
& \textitbf{prubaang} & \textstyleChCharisSIL{pru.ˈba.ɐn} & \textsc{n} & change\\
& \textitbf{pukulang} & \textstyleChCharisSIL{pu.ˈku.lɐn} & \textsc{n} & stroke\\
& \textitbf{puluang} & \textstyleChCharisSIL{pu.ˈlu.ɐn} & \textsc{n} & tens\\
& \textitbf{putarang} & \textstyleChCharisSIL{pu.ˈta.ɾɐn} & \textsc{n} & circle\\
& \textstyleChBold{R} &  &  & \\
& \textitbf{rambutang} & \textstyleChCharisSIL{rɐm.ˈbu.tɐn} & \textsc{n} & rambutan\\
& \textitbf{ramuang} & \textstyleChCharisSIL{ra.ˈmʊ.ɐn} & \textsc{n} & ingredients\\
& \textitbf{ratusang} & \textstyleChCharisSIL{ra.ˈtu.sɐn} & \textsc{n} & hundreds\\
& \textitbf{renungang} & \textstyleChCharisSIL{rɛ.ˈnu.ŋɐn} & \textsc{n} & meditation\\
& \textitbf{rombongang} & \textstyleChCharisSIL{rɔ̞m.ˈbɔ̞.ŋɐn} & \textsc{n} & group of people\\
& \textitbf{ruangang} & \textstyleChCharisSIL{ru.ˈa.ŋɐn} & \textsc{n} & room\\
& \textstyleChBold{S} &  &  & \\
& \textitbf{salakang} & \textstyleChCharisSIL{sa.ˈla.kɐn} & \textsc{v.bi} & blame\\
& \textitbf{saringang} & \textstyleChCharisSIL{sa.ˈɾi.ŋɐn} & \textsc{n} & filter\\
& \textitbf{sebapnya} & \textstyleChCharisSIL{sɛ.ˈbɐp̚.ɲa} & \textsc{adv} & for that reason\\
& \textitbf{sebenarnya} & \textstyleChCharisSIL{ˌsɛ.bɛ.ˈnɐr.ɲa} & \textsc{adv} & actually\\
x & \textitbf{sebla} & \textstyleChCharisSIL{sɛ.ˈbla} & \textsc{n-loc} & side\\
x & \textitbf{seblas} & \textstyleChCharisSIL{sɛ.ˈblɐs} & \textsc{num.c} & eleven\\
x & \textitbf{seblum} & \textstyleChCharisSIL{sɛ.ˈblʊm} & \textsc{cnj} & before\\
& \textitbf{sekitar} & \textstyleChCharisSIL{sɛ.ˈki.tɐr} & \textsc{n} & vicinity\\
& \textitbf{selaing} & \textstyleChCharisSIL{sɛ.ˈla.ɪn} & \textsc{adv} & besides\\
& \textitbf{sembarang} & \textstyleChCharisSIL{sɛ̞m.ˈba.ɾɐŋ} & \textsc{qt} & any (kind of)\\
& \textitbf{sepakat} & \textstyleChCharisSIL{sɛ.ˈpa.kɐt̚} & \textsc{v.mo(st)} & be agreed\\
& \textitbf{sepanggal} & \textstyleChCharisSIL{sɛ.ˈpɐŋ.gɐl} & \textsc{n} & a fragment\\
& \textitbf{serakang} & \textstyleChCharisSIL{sɛ.ˈɾa.kɐn} & \textsc{v.bi} & surrender\\
& \textitbf{serangang} & \textstyleChCharisSIL{sɛ.ˈɾa.ŋɐn} & \textsc{n} & attack\\
& \textitbf{seswai} & \textstyleChCharisSIL{sɛ.ˈswa.ɪ} & \textsc{v.mo(st)} & be appropriate\\
& \textitbf{seswaikang} & \textstyleChCharisSIL{ˌsɛ.swa.ˈɪ.kɐn} & \textsc{v.bi} & adjust\\
& \textitbf{setiap} & \textstyleChCharisSIL{sɛ.ˈti.ɐp̚} & \textsc{qt} & every\\
& \textitbf{sialang} & \textstyleChCharisSIL{sɪ.ˈa.lɐn} & \textsc{n} & s. o. unfortunate\\
& \textitbf{siapkang} & \textstyleChCharisSIL{sɪ.ˈɐp̚.kɐn} & \textsc{v.bi} & prepare\\
& \textitbf{silakang} & \textstyleChCharisSIL{si.ˈla.kɐn} & \textsc{v.bi} & invite\\
& \textitbf{slalu} & \textstyleChCharisSIL{ˈsla.lu} & \textsc{adv} & always\\
& \textitbf{slama} & \textstyleChCharisSIL{sɛ.ˈla.ma} & \textsc{adv} & as long as, while\\
& \textitbf{sorenya} & \textstyleChCharisSIL{sɔ.ˈɾɛ.ɲa} & \textsc{adv} & this afternoon\\
& \textitbf{spertinya} & \textstyleChCharisSIL{spɛ̞r.ˈti.ɲa} & \textsc{adv} & it seems\\
& \textitbf{spulu} & \textstyleChCharisSIL{ˈspu.lu} & \textsc{num.c} & ten\\
& \textitbf{sratus} & \textstyleChCharisSIL{ˈsɾa.tʊs} & \textsc{num.c} & one hundred\\
& \textitbf{sribu} & \textstyleChCharisSIL{ˈsri.bu} & \textsc{num.c} & one thousand\\
& \textitbf{stuju} & \textstyleChCharisSIL{ˈstu.dʒu} & \textsc{v.mo(dy)} & agree\\
& \textitbf{sumbangang} & \textstyleChCharisSIL{sʊm.ˈba.ŋɐn} & \textsc{n} & donation\\
& \textstyleChBold{T} &  &  & \\
& \textitbf{tabalik} & \textstyleChCharisSIL{ta.ˈba.lɛ} & \textsc{v.mo(acl)} & be turned over\\
& \textitbf{tabanting} & \textstyleChCharisSIL{ta.ˈbɐn.tɪŋ} & \textsc{v.mo(acl)} & be tossed around\\
x & \textitbf{tabla} & \textstyleChCharisSIL{ta.ˈbla} & \textsc{v.mo(acl)} & be cracked open\\
& \textitbf{tacukur} & \textstyleChCharisSIL{ta.ˈtʃu.kʊr} & \textsc{v.mo(acl)} & be scalped\\
& \textitbf{tagait} & \textstyleChCharisSIL{ta.ˈgɐ.ɪt̚} & \textsc{v.mo(acl)} & be hooked\\
& \textitbf{tagoyang} & \textstyleChCharisSIL{ta.ˈgɔ̞.jɐŋ} & \textsc{v.mo(acl)} & be shaken\\
& \textitbf{taguling} & \textstyleChCharisSIL{ta.ˈgu.lɪŋ} & \textsc{v.mo(acl)} & be rolled over\\
& \textitbf{tahambur} & \textstyleChCharisSIL{ta.ˈhɐm.bʊr} & \textsc{v.mo(acl)} & be scattered about\\
& \textitbf{tahangang} & \textstyleChCharisSIL{ta.ˈha.nɐn} & \textsc{n} & detention\\
& \textitbf{takancing} & \textstyleChCharisSIL{ta.ˈkɐn.tʃɪŋ} & \textsc{v.mo(acl)} & be locked\\
& \textitbf{takumpul} & \textstyleChCharisSIL{ta.ˈkʊm.pʊl} & \textsc{v.mo(acl)} & be collected\\
& \textitbf{takupas} & \textstyleChCharisSIL{ta.ˈku.pɐs} & \textsc{v.mo(acl)} & be peeled\\
& \textitbf{talipat} & \textstyleChCharisSIL{tɛ̞r.ˈli.pɐt̚} & \textsc{v.mo(acl)} & be folded\\
& \textitbf{tamasuk} & \textstyleChCharisSIL{ta.ˈma.sʊk̚} & \textsc{v.mo(acl)} & be included\\
& \textitbf{tambaang} & \textstyleChCharisSIL{tɐm.ˈba.ɐn} & \textsc{n} & extra amount\\
& \textitbf{tanamang} & \textstyleChCharisSIL{ta.ˈna.mɐn} & \textsc{n} & plants\\
& \textitbf{tanggapang} & \textstyleChCharisSIL{tɐŋ.ˈga.pɐn} & \textsc{n} & response, idea\\
& \textitbf{tanggulangi} & \textstyleChCharisSIL{ˌtɐŋ.gu.ˈla.ŋi} & \textsc{v.bi} & ward off, cope with\\
& \textitbf{tantangang} & \textstyleChCharisSIL{tɐn.ˈta.ŋɐn} & \textsc{n} & challenge\\
& \textitbf{tapisang} & \textstyleChCharisSIL{ta.ˈpi.sɐn} & \textsc{n} & filter\\
& \textitbf{taputar} & \textstyleChCharisSIL{ta.ˈpu.tɐr} & \textsc{v.mo(acl)} & be turned around\\
& \textitbf{tasala} & \textstyleChCharisSIL{ta.ˈsa.la} & \textsc{v.mo(acl)} & be mistaken\\
& \textitbf{tatikam} & \textstyleChCharisSIL{ta.ˈti.kɐm} & \textsc{v.mo(acl)} & be stabbed\\
& \textitbf{tatongkat} & \textstyleChCharisSIL{ta.ˈtɔ̞ŋ.kɐt̚} & \textsc{v.mo(acl)} & be beaten\\
& \textitbf{tatutup} & \textstyleChCharisSIL{ta.ˈtʊ.tʊp̚} & \textsc{v.mo(acl)} & be closed\\
& \textitbf{tendangang} & \textstyleChCharisSIL{tɛ̞n.ˈda.ŋɐn} & \textsc{n} & kicking\\
& \textitbf{tentukang} & \textstyleChCharisSIL{tɛ̞n.ˈtʊ.kɐn} & \textsc{v.bi} & determine\\
& \textitbf{terbakar} & \textstyleChCharisSIL{tɛ̞r.ˈba.kɐr} & \textsc{v.mo(acl)} & be burnt\\
& \textitbf{terbuka} & \textstyleChCharisSIL{tɛ̞r.ˈbu.ka} & \textsc{v.mo(acl)} & be opened\\
& \textitbf{terendam} & \textstyleChCharisSIL{tɛ̞.ˈɾɛ̞n.dɐm} & \textsc{v.mo(acl)} & be soaked\\
& \textitbf{terganggu} & \textstyleChCharisSIL{tɛ̞r.ˈgɐŋ.gu} & \textsc{v.mo(acl)} & be disturbed\\
& \textitbf{tergantong} & \textstyleChCharisSIL{ta.ˈgɐn.tɔ̞ŋ} & \textsc{v.mo(acl)} & be dependent\\
& \textitbf{terjadi} & \textstyleChCharisSIL{tɛ̞r.ˈdʒa.di} & \textsc{v.mo(acl)} & happen\\
& \textitbf{terjatu} & \textstyleChCharisSIL{tɛ̞r.ˈdʒa.tu} & \textsc{v.mo(acl)} & be fallen\\
& \textitbf{terkenal} & \textstyleChCharisSIL{tɛ̞r.kɛ.ˈnɐl} & \textsc{v.mo(acl)} & be well-known\\
& \textitbf{terlambat} & \textstyleChCharisSIL{tɛ̞r.ˈlɐm.bɐt̚} & \textsc{v.mo(acl)} & be late\\
& \textitbf{terlanjur} & \textstyleChCharisSIL{tɛ̞r.ˈlɐn.dʒʊr} & \textsc{v.mo(st)} & be beyond bounds\\
& \textitbf{terlempar} & \textstyleChCharisSIL{tɛ̞r.ˈlɛ̞m.pɐr} & \textsc{v.mo(acl)} & be thrown\\
x & \textitbf{terlepas} & \textstyleChCharisSIL{ˌtɛ̞r.lɛ.ˈpɐs} & \textsc{v.mo(acl)} & be loose\\
& \textitbf{terpukul} & \textstyleChCharisSIL{tɛ̞r.ˈpʊ.kʊl} & \textsc{v.mo(acl)} & be beaten\\
& \textitbf{tersendiri} & \textstyleChCharisSIL{ˌtɛ̞r.sɛ̞n.ˈdi.ɾi} & \textsc{v.mo(acl)} & be separate\\
x & \textitbf{tersera} & \textstyleChCharisSIL{ˌtɛ̞r.sɛ.ˈɾa} & \textsc{v.bi} & up to s. o.\\
& \textitbf{tersinggung} & \textstyleChCharisSIL{tɛ̞r.ˈsɪŋ.gʊŋ} & \textsc{v.mo(acl)} & be offended\\
& \textitbf{tertarik} & \textstyleChCharisSIL{tɛ̞r.ˈta.ɾɪk̚} & \textsc{v.mo(acl)} & be pulled\\
& \textitbf{tertawa} & \textstyleChCharisSIL{tɛ̞r.ˈta.wa} & \textsc{v.bi} & laugh\\
x & \textitbf{tertentu} & \textstyleChCharisSIL{ˌtɛ̞r.tɛ̞n.ˈtu} & \textsc{v.mo(st)} & be specific\\
& \textitbf{tertolak} & \textstyleChCharisSIL{tɛ̞r.ˈtɔ.lɐk̚} & \textsc{v.mo(acl)} & be rejected\\
& \textitbf{tertukar} & \textstyleChCharisSIL{tɛ̞r.ˈtu.kɐr} & \textsc{v.mo(acl)} & get changed\\
& \textitbf{tikungang} & \textstyleChCharisSIL{ti.ˈku.ŋɐn} & \textsc{n} & bend in road\\
& \textitbf{timbulkang} & \textstyleChCharisSIL{tɪm.ˈbʊl.kɐn} & \textsc{v.bi} & emerge\\
& \textitbf{tindakang} & \textstyleChCharisSIL{tɪn.ˈda.kɐn} & \textsc{n} & action\\
& \textitbf{tingkatang} & \textstyleChCharisSIL{tɪŋ.ˈka.tɐn} & \textsc{n} & level\\
& \textitbf{titipang} & \textstyleChCharisSIL{ti.ˈti.pɐn} & \textsc{n} & entrusted goods\\
& \textitbf{trangkang} & \textstyleChCharisSIL{ˈtrɐŋ.kɐn} & \textsc{v.bi} & clarify\\
& \textitbf{trangkat} & \textstyleChCharisSIL{ˈtrɐŋ.kɐt̚} & \textsc{v.mo(acl)} & be lifted\\
& \textitbf{trapkang} & \textstyleChCharisSIL{ˈtrɐp̚.kɐn} & \textsc{v.bi} & implement, apply\\
& \textitbf{trapung} & \textstyleChCharisSIL{ˈtra.pʊŋ} & \textsc{v.mo(acl)} & be drifting\\
& \textitbf{tujuang} & \textstyleChCharisSIL{tu.ˈdʒu.ɐn} & \textsc{n} & purpose\\
& \textitbf{tulisang} & \textstyleChCharisSIL{tu.ˈli.sɐn} & \textsc{n} & writing\\
& \textitbf{tumpukang} & \textstyleChCharisSIL{tʊm.ˈpʊ.kɐn} & \textsc{n} & pile\\
& \textitbf{turungang} & \textstyleChCharisSIL{tu.ˈɾu.nɐn} & \textsc{n} & descendant\\
& \textstyleChBold{U} &  &  & \\
& \textitbf{ucapkang} & \textstyleChCharisSIL{u.ˈtʃɐp̚.kɐn} & \textsc{v.bi} & express\\
& \textitbf{ujiang} & \textstyleChCharisSIL{u.ˈdʒɪ.ɐn} & \textsc{n/v.bi} & examination / examine\\
& \textitbf{ukirang} & \textstyleChCharisSIL{u.ˈki.ɾɐn} & \textsc{n} & carved object\\
& \textitbf{ukurang} & \textstyleChCharisSIL{u.ˈku.ɾɐn} & \textsc{n} & measurement\\
& \textitbf{ulangang} & \textstyleChCharisSIL{ʊ.ˈla.ŋɐn} & \textsc{n/v.bi} & test\\
& \textitbf{utusang} & \textstyleChCharisSIL{u.ˈtu.sɐn} & \textsc{n} & messenger\\
\lspbottomrule
\end{tabular}
%\setcounter{page}{1}\section{Texts}
\label{bkm:Ref373929512}
This appendix presents a sample of twelve texts. Included are three spontaneous conversations, one spontaneous narrative, two elicited narratives, two expositories, two hortatories, and two elicited jokes. For each text the following meta data are provided: the file name, the text type, the interlocutors, and the length (in minutes). For additional information see also §1.11 and Appendix B.
\end{styleBodyxvafter}

\subsection{Conversation: Playing volleyball; morning chores}

\begin{tabular}{ll}
\lsptoprule
File name: & 081023-001-Cv\\
Text type: & Conversation, spontaneous\\
Interlocutors: & 1 younger male, 2 younger females\\
Length (min.): & 4:52\\
\lspbottomrule
\end{tabular}
\begin{tabular}{llllllllllllllllllllllllllllll}
\lsptoprule
0001 & \multicolumn{2}{l}{Oten:} & \multicolumn{4}{l}{\textsc{[up]}} & \multicolumn{4}{l}{blang,} & \multicolumn{2}{l}{kam} & \multicolumn{4}{l}{dari} & \multicolumn{3}{l}{mana?} & \multicolumn{2}{l}{trus} & \multicolumn{3}{l}{\textsc{[up]}} & \multicolumn{3}{l}{tong} & \multicolumn{2}{l}{dari}\\
& \multicolumn{2}{l}{} & \multicolumn{4}{l}{} & \multicolumn{4}{l}{say} & \multicolumn{2}{l}{\textsc{2pl}} & \multicolumn{4}{l}{from} & \multicolumn{3}{l}{where} & \multicolumn{2}{l}{next} & \multicolumn{3}{l}{} & \multicolumn{3}{l}{\textsc{1pl}} & \multicolumn{2}{l}{from}\\
& \multicolumn{3}{l}{Arbais,} & \multicolumn{4}{l}{kam} & \multicolumn{3}{l}{pu} & \multicolumn{4}{l}{nama} & \multicolumn{4}{l}{siapa{\Tilde}siapa?} & \multicolumn{4}{l}{Herman} & \multicolumn{3}{l}{de} & \multicolumn{3}{l}{bilang} & de\\
& \multicolumn{3}{l}{Arbais} & \multicolumn{4}{l}{\textsc{2pl}} & \multicolumn{3}{l}{\textsc{poss}} & \multicolumn{4}{l}{name} & \multicolumn{4}{l}{\textsc{rdp}{\Tilde}who} & \multicolumn{4}{l}{Herman} & \multicolumn{3}{l}{\textsc{3sg}} & \multicolumn{3}{l}{say} & \textsc{3sg}\\
& pu & \multicolumn{4}{l}{nama,} & \multicolumn{3}{l}{pace} & \multicolumn{3}{l}{de} & \multicolumn{4}{l}{tulis} & \multicolumn{2}{l}{di} & \multicolumn{3}{l}{kertas,} & \multicolumn{3}{l}{su,} & \multicolumn{3}{l}{situ} & \multicolumn{3}{l}{de}\\
& \textsc{poss} & \multicolumn{4}{l}{name} & \multicolumn{3}{l}{man} & \multicolumn{3}{l}{\textsc{3sg}} & \multicolumn{4}{l}{write} & \multicolumn{2}{l}{at} & \multicolumn{3}{l}{paper} & \multicolumn{3}{l}{already} & \multicolumn{3}{l}{\textsc{l.med}} & \multicolumn{3}{l}{\textsc{3sg}}\\
& ada, & \multicolumn{3}{l}{de} & \multicolumn{5}{l}{su} & \multicolumn{4}{l}{biking} & \multicolumn{16}{l}{daftar}\\
& exist & \multicolumn{3}{l}{\textsc{3sg}} & \multicolumn{5}{l}{already} & \multicolumn{4}{l}{make} & \multicolumn{16}{l}{list}\\
\lspbottomrule
\end{tabular}
\ea
\glt 
Oten: [\textsc{up}] said, ‘where are you from?’, then [\textsc{up}], ‘we are from Arbais’, ‘what are your names?’ Herman gave his name, the man wrote (it) on a paper, that’s it, there it was!, he (the man) had already made a list
\z

\begin{tabular}{lllllllllllllllllllllllllll}
\lsptoprule
0002 & \multicolumn{2}{l}{su} & \multicolumn{2}{l}{biking} & \multicolumn{5}{l}{daftar,} & \multicolumn{3}{l}{pertama} & \multicolumn{3}{l}{di} & \multicolumn{3}{l}{atas} & \multicolumn{2}{l}{sa} & \multicolumn{2}{l}{liat} & \multicolumn{3}{l}{nama} & tu\\
& \multicolumn{2}{l}{already} & \multicolumn{2}{l}{make} & \multicolumn{5}{l}{list} & \multicolumn{3}{l}{first} & \multicolumn{3}{l}{at} & \multicolumn{3}{l}{top} & \multicolumn{2}{l}{\textsc{1sg}} & \multicolumn{2}{l}{see} & \multicolumn{3}{l}{name} & \textsc{d.dist}\\
& Lukas & \multicolumn{4}{l}{ini} & \multicolumn{3}{l}{T.,} & \multicolumn{5}{l}{bencong} & \multicolumn{3}{l}{satu,} & \multicolumn{3}{l}{Lukas} & \multicolumn{2}{l}{T.} & \multicolumn{3}{l}{dia,} & \multicolumn{2}{l}{trus}\\
& Lukas & \multicolumn{4}{l}{\textsc{d.prox}} & \multicolumn{3}{l}{T.} & \multicolumn{5}{l}{transvestite} & \multicolumn{3}{l}{one} & \multicolumn{3}{l}{Lukas} & \multicolumn{2}{l}{T.} & \multicolumn{3}{l}{\textsc{3sg}} & \multicolumn{2}{l}{next}\\
& \multicolumn{2}{l}{suda} & \multicolumn{4}{l}{spulu,} & \multicolumn{5}{l}{pas} & \multicolumn{3}{l}{tong,} & \multicolumn{3}{l}{trus} & \multicolumn{3}{l}{tamba} & \multicolumn{3}{l}{kaka} & \multicolumn{3}{l}{dari}\\
& \multicolumn{2}{l}{already} & \multicolumn{4}{l}{one-tens} & \multicolumn{5}{l}{precisely} & \multicolumn{3}{l}{\textsc{1pl}} & \multicolumn{3}{l}{next} & \multicolumn{3}{l}{add} & \multicolumn{3}{l}{oSb} & \multicolumn{3}{l}{from}\\
& \multicolumn{3}{l}{Mamberamo} & \multicolumn{4}{l}{satu,} & \multicolumn{3}{l}{Agus,} & \multicolumn{4}{l}{Agus} & \multicolumn{12}{l}{Y.}\\
& \multicolumn{3}{l}{Mamberamo} & \multicolumn{4}{l}{one} & \multicolumn{3}{l}{Agus} & \multicolumn{4}{l}{Agus} & \multicolumn{12}{l}{Y.}\\
\lspbottomrule
\end{tabular}
\ea
\glt 
(he) had already made a list, the first one on top, I saw that name, Lukas, what’s-his-name, T., a certain transvestite, Lukas T., then (there were) already ten (names on that list), at that moment we, then add a certain older brother from (the) Mambramo (area), Agus, Agus Y.
\z

\begin{tabular}{llllllllllll}
\lsptoprule
0003 & \multicolumn{2}{l}{tadi} & di & \multicolumn{2}{l}{pasar} & sa & ada & pegang & tangang & deng & dia,\\
& \multicolumn{2}{l}{earlier} & at & \multicolumn{2}{l}{market} & \textsc{1sg} & exist & hold & hand & with & \textsc{3sg}\\
& de & \multicolumn{2}{l}{pake} & baju & \multicolumn{7}{l}{mera}\\
& \textsc{3sg} & \multicolumn{2}{l}{use} & shirt & \multicolumn{7}{l}{be.red}\\
\lspbottomrule
\end{tabular}
\ea
\glt 
earlier in the market I was holding hands with him, he was wearing a red shirt
\z

\begin{tabular}{llll}
\lsptoprule
0004 & Klara: & o, & [\textsc{up}]\\
&  & oh! & \\
\lspbottomrule
\end{tabular}
\ea
\glt 
Klara: oh, [\textsc{up}]
\z

\begin{tabular}{llll}
\lsptoprule
0005 & Oten: & badang & besar{\Tilde}besar\\
&  & body & \textsc{rdp}{\Tilde}be.big\\
\lspbottomrule
\end{tabular}
\ea
\glt 
Oten: (his) body is very big
\z

\begin{tabular}{lll}
\lsptoprule
0006 & Klara: & ((laughter))\\
\lspbottomrule
\end{tabular}
\ea
\glt 
Klara: ((laughter))
\z

\begin{tabular}{lllllllllllllllllllllllllllll}
\lsptoprule
0007 & \multicolumn{2}{l}{Oten:} & \multicolumn{4}{l}{pace} & \multicolumn{3}{l}{de} & \multicolumn{2}{l}{tulis} & \multicolumn{3}{l}{tong} & \multicolumn{3}{l}{pu} & \multicolumn{3}{l}{nama} & \multicolumn{3}{l}{selesay,} & \multicolumn{3}{l}{de} & \multicolumn{2}{l}{bilang,}\\
& \multicolumn{2}{l}{} & \multicolumn{4}{l}{man} & \multicolumn{3}{l}{\textsc{3sg}} & \multicolumn{2}{l}{write} & \multicolumn{3}{l}{\textsc{1pl}} & \multicolumn{3}{l}{\textsc{poss}} & \multicolumn{3}{l}{name} & \multicolumn{3}{l}{finish} & \multicolumn{3}{l}{\textsc{3sg}} & \multicolumn{2}{l}{say}\\
& \multicolumn{5}{l}{besok,} & \multicolumn{3}{l}{jam,} & \multicolumn{4}{l}{seblum} & \multicolumn{3}{l}{jam} & \multicolumn{3}{l}{tiga} & \multicolumn{3}{l}{kamu} & \multicolumn{3}{l}{su} & \multicolumn{3}{l}{ada} & di\\
& \multicolumn{5}{l}{tomorrow} & \multicolumn{3}{l}{hour} & \multicolumn{4}{l}{before} & \multicolumn{3}{l}{hour} & \multicolumn{3}{l}{three} & \multicolumn{3}{l}{\textsc{2pl}} & \multicolumn{3}{l}{already} & \multicolumn{3}{l}{exist} & at\\
& \multicolumn{3}{l}{sini} & \multicolumn{4}{l}{untuk} & \multicolumn{3}{l}{latiang,} & \multicolumn{3}{l}{trus} & \multicolumn{3}{l}{sa} & \multicolumn{3}{l}{tanya,} & \multicolumn{3}{l}{tong} & \multicolumn{3}{l}{latiang} & \multicolumn{3}{l}{ini}\\
& \multicolumn{3}{l}{\textsc{l.prox}} & \multicolumn{4}{l}{for} & \multicolumn{3}{l}{practice} & \multicolumn{3}{l}{next} & \multicolumn{3}{l}{\textsc{1sg}} & \multicolumn{3}{l}{ask} & \multicolumn{3}{l}{\textsc{1pl}} & \multicolumn{3}{l}{practice} & \multicolumn{3}{l}{\textsc{d.prox}}\\
& mo & \multicolumn{3}{l}{ke} & \multicolumn{24}{l}{mana?}\\
& want & \multicolumn{3}{l}{to} & \multicolumn{24}{l}{where}\\
\lspbottomrule
\end{tabular}
\ea
\glt 
Oten: after the man had written down our names, he said, ‘tomorrow, o’clock, before three o’clock you’ll already be here to practice’, then I asked, ‘we (do) our very practicing to go where?’
\z

\begin{tabular}{llm{-9.4015896E-4in}lllllllllllllllllllllllllllllll}
\lsptoprule
0008 & de & \multicolumn{3}{l}{blang,} & \multicolumn{4}{l}{a,} & \multicolumn{5}{l}{latiang} & \multicolumn{4}{l}{saja,} & \multicolumn{7}{l}{katanya} & \multicolumn{4}{l}{bupati} & \multicolumn{4}{l}{bilang,} & ada\\
& \textsc{3sg} & \multicolumn{3}{l}{say} & \multicolumn{4}{l}{ah!} & \multicolumn{5}{l}{practice} & \multicolumn{4}{l}{just} & \multicolumn{7}{l}{it.is.being.said} & \multicolumn{4}{l}{regent} & \multicolumn{4}{l}{say} & exist\\
& \multicolumn{2}{l}{mo} & \multicolumn{3}{l}{pergi} & \multicolumn{4}{l}{maing} & \multicolumn{2}{l}{di} & \multicolumn{5}{l}{ini} & \multicolumn{4}{l}{Serui} & \multicolumn{2}{l}{ka} & \multicolumn{4}{l}{itu} & \multicolumn{3}{l}{yang} & \multicolumn{2}{l}{de} & \multicolumn{2}{l}{ada}\\
& \multicolumn{2}{l}{want} & \multicolumn{3}{l}{go} & \multicolumn{4}{l}{play} & \multicolumn{2}{l}{at} & \multicolumn{5}{l}{\textsc{d.prox}} & \multicolumn{4}{l}{Serui} & \multicolumn{2}{l}{or} & \multicolumn{4}{l}{\textsc{d.dist}} & \multicolumn{3}{l}{\textsc{rel}} & \multicolumn{2}{l}{\textsc{3sg}} & \multicolumn{2}{l}{exist}\\
& \multicolumn{3}{l}{cari} & \multicolumn{4}{l}{ana{\Tilde}ana} & \multicolumn{5}{l}{untuk} & \multicolumn{3}{l}{pergi} & \multicolumn{4}{l}{maing,} & \multicolumn{4}{l}{suda,} & \multicolumn{4}{l}{baru} & \multicolumn{3}{l}{sa} & \multicolumn{3}{l}{bilang}\\
& \multicolumn{3}{l}{search} & \multicolumn{4}{l}{\textsc{rdp}{\Tilde}child} & \multicolumn{5}{l}{for} & \multicolumn{3}{l}{go} & \multicolumn{4}{l}{play} & \multicolumn{4}{l}{already} & \multicolumn{4}{l}{and.then} & \multicolumn{3}{l}{\textsc{1sg}} & \multicolumn{3}{l}{say}\\
& \multicolumn{2}{l}{masi} & \multicolumn{4}{l}{bisa} & \multicolumn{4}{l}{ada} & \multicolumn{4}{l}{yang} & \multicolumn{4}{l}{masuk} & \multicolumn{3}{l}{ato} & \multicolumn{4}{l}{su} & \multicolumn{2}{l}{tra} & \multicolumn{6}{l}{ada?}\\
& \multicolumn{2}{l}{still} & \multicolumn{4}{l}{be.able} & \multicolumn{4}{l}{exist} & \multicolumn{4}{l}{\textsc{rel}} & \multicolumn{4}{l}{enter} & \multicolumn{3}{l}{or} & \multicolumn{4}{l}{already} & \multicolumn{2}{l}{\textsc{neg}} & \multicolumn{6}{l}{exist}\\
\lspbottomrule
\end{tabular}
\ea
\glt 
he said, ‘ah, just practice, it’s being said that the regent says that we are going to go to play maybe on, what’s-its-name, Serui (Island), that’s why he’s looking for young people to go play’, alright, and then I said, ‘can one still be included (on that list) or already not any longer?’
\z

\begin{tabular}{llllllllllllllllllllllllllllllll}
\lsptoprule
0009 & de & \multicolumn{4}{l}{blang,} & \multicolumn{4}{l}{kalo} & \multicolumn{3}{l}{ada} & \multicolumn{4}{l}{yang} & \multicolumn{3}{l}{mo} & \multicolumn{5}{l}{masuk,} & \multicolumn{4}{l}{bisa,} & \multicolumn{3}{l}{trus}\\
& \textsc{3sg} & \multicolumn{4}{l}{say} & \multicolumn{4}{l}{if} & \multicolumn{3}{l}{exist} & \multicolumn{4}{l}{\textsc{rel}} & \multicolumn{3}{l}{want} & \multicolumn{5}{l}{enter} & \multicolumn{4}{l}{be.able} & \multicolumn{3}{l}{next}\\
& \multicolumn{2}{l}{kaka} & \multicolumn{2}{l}{wa,} & \multicolumn{4}{l}{yang} & \multicolumn{6}{l}{nanti} & \multicolumn{4}{l}{kasi} & \multicolumn{5}{l}{latiang} & \multicolumn{4}{l}{itu} & \multicolumn{3}{l}{kaka} & polisi\\
& \multicolumn{2}{l}{oSb} & \multicolumn{2}{l}{\textsc{spm}} & \multicolumn{4}{l}{\textsc{rel}} & \multicolumn{6}{l}{very.soon} & \multicolumn{4}{l}{give} & \multicolumn{5}{l}{practice} & \multicolumn{4}{l}{\textsc{d.dist}} & \multicolumn{3}{l}{oSb} & police\\
& \multicolumn{2}{l}{yang} & \multicolumn{5}{l}{baru{\Tilde}baru} & \multicolumn{4}{l}{deng} & \multicolumn{4}{l}{Hurki} & \multicolumn{5}{l}{jalang} & \multicolumn{2}{l}{ke} & \multicolumn{4}{l}{Jakarta} & \multicolumn{3}{l}{sana,} & \multicolumn{2}{l}{ka}\\
& \multicolumn{2}{l}{\textsc{rel}} & \multicolumn{5}{l}{just.now} & \multicolumn{4}{l}{with} & \multicolumn{4}{l}{Hurki} & \multicolumn{5}{l}{walk} & \multicolumn{2}{l}{to} & \multicolumn{4}{l}{Jakarta} & \multicolumn{3}{l}{\textsc{l.dist}} & \multicolumn{2}{l}{oSb}\\
& \multicolumn{3}{l}{Sarles,} & \multicolumn{3}{l}{ka} & \multicolumn{4}{l}{Sarles} & \multicolumn{3}{l}{juga,} & \multicolumn{4}{l}{de} & \multicolumn{4}{l}{pu} & \multicolumn{4}{l}{maim} & \multicolumn{6}{l}{pisow}\\
& \multicolumn{3}{l}{Sarles} & \multicolumn{3}{l}{oSb} & \multicolumn{4}{l}{Sarles} & \multicolumn{3}{l}{also} & \multicolumn{4}{l}{\textsc{3sg}} & \multicolumn{4}{l}{\textsc{poss}} & \multicolumn{4}{l}{play} & \multicolumn{6}{l}{knife}\\
\lspbottomrule
\end{tabular}
\ea
\glt 
he said, ‘if there is someone who wants to be included, (he/she) can (be included), then, older brother [\textsc{spm}], (the one) who will give the training, what’s-his-name, the older brother (who’s a) police (officer) who just now went to Jakarta over there together with Hurki, older brother Sarles, older brother Sarles also, he has a fast and smart way of playing’ (Lit. ‘the knife playing of’)
\z

\begin{tabular}{llll}
\lsptoprule
0010 & Klara: & bola & fol\\
&  & ball & volleyball\\
\lspbottomrule
\end{tabular}
\ea
\glt 
Klara: volleyball
\z

\begin{tabular}{llllllll}
\lsptoprule
0011 & Oten: & yo, & bola & foli & ini, & tanta & Nelci\\
&  & yes & ball & volleyball & \textsc{d.prox} & aunt & Nelci\\
\lspbottomrule
\end{tabular}
\ea
\glt 
Oten: yes, this volleyball, [addressing Nelci] aunt Nelci
\z

\begin{tabular}{llllllllll}
\lsptoprule
0012 & Klara: & yo, & net & laki{\Tilde}laki, & tong & yang & bli, & yang & sebla\\
&  & yes & (sport.)net & \textsc{rdp}{\Tilde}husband & \textsc{1pl} & \textsc{rel} & buy & \textsc{rel} & side\\
& darat & \multicolumn{8}{l}{[\textsc{up}]}\\
& land & \multicolumn{8}{l}{}\\
\lspbottomrule
\end{tabular}
\ea
\glt 
Klara: yes, the (volleyball) net for men, (it was) us who (bought it), (the one) which is off the beach [\textsc{up}]
\z

\begin{tabular}{llllllllll}
\lsptoprule
0013 & Oten: & yang & sebla, & yo & sebla, & di & pinggir & kali & tu\\
&  & \textsc{rel} & side & yes & side & at & border & river & \textsc{d.dist}\\
\lspbottomrule
\end{tabular}
\ea
\glt 
Oten: (the one) which is off (the beach), yes, off (the beach), on the banks of that river
\z

\begin{tabular}{lllllllllllllll}
\lsptoprule
0014 & Klara: & \multicolumn{2}{l}{itu} & \multicolumn{2}{l}{kalo} & \multicolumn{2}{l}{memang} & \multicolumn{2}{l}{bola{\Tilde}bola} & \multicolumn{2}{l}{tinggi} & kalo & smes & itu\\
&  & \multicolumn{2}{l}{\textsc{d.dist}} & \multicolumn{2}{l}{if} & \multicolumn{2}{l}{indeed} & \multicolumn{2}{l}{\textsc{rdp}{\Tilde}ball} & \multicolumn{2}{l}{be.high} & if & smash & \textsc{d.dist}\\
& \multicolumn{2}{l}{memang} & \multicolumn{2}{l}{masuk} & \multicolumn{2}{l}{kali,} & \multicolumn{2}{l}{bola{\Tilde}bola} & \multicolumn{2}{l}{terlalu} & \multicolumn{4}{l}{[\textsc{up}]}\\
& \multicolumn{2}{l}{indeed} & \multicolumn{2}{l}{enter} & \multicolumn{2}{l}{river} & \multicolumn{2}{l}{\textsc{rdp}{\Tilde}ball} & \multicolumn{2}{l}{too} & \multicolumn{4}{l}{}\\
\lspbottomrule
\end{tabular}
\ea
\glt 
Klara: so, if indeed the balls are high, if (one) smashes them, indeed they go into the river, the balls are too [\textsc{up}]
\z

\begin{tabular}{llllllll}
\lsptoprule
0015 & Oten: & sa & lompat & itu & frey, & tangang & lewat\\
&  & \textsc{1sg} & jump & \textsc{d.dist} & be.free & hand & pass.by\\
\lspbottomrule
\end{tabular}
\ea
\glt 
Oten: I jump high, free (of the net), (my) hands surpass (the net)
\z

\begin{tabular}{lll}
\lsptoprule
0016 & MY: & [\textsc{up}]\\
\lspbottomrule
\end{tabular}
\ea
\glt 
MY: [\textsc{up}]
\z

\begin{tabular}{llllllllllllllll}
\lsptoprule
0017 & Klara: & \multicolumn{3}{l}{kemaring} & \multicolumn{2}{l}{saya,} & \multicolumn{2}{l}{Herman,} & \multicolumn{3}{l}{Maa,} & \multicolumn{2}{l}{Markus,} & siapa & ni,\\
&  & \multicolumn{3}{l}{yesterday} & \multicolumn{2}{l}{\textsc{1sg}} & \multicolumn{2}{l}{Herman} & \multicolumn{3}{l}{\textsc{tru-}Markus} & \multicolumn{2}{l}{Markus} & who & \textsc{d.prox}\\
& \multicolumn{2}{l}{Nofita,} & sa & \multicolumn{2}{l}{bilang} & \multicolumn{2}{l}{begini,} & \multicolumn{2}{l}{sa} & juga & \multicolumn{2}{l}{naik} & \multicolumn{3}{l}{frey}\\
& \multicolumn{2}{l}{Nofita} & \textsc{1sg} & \multicolumn{2}{l}{say} & \multicolumn{2}{l}{like.this} & \multicolumn{2}{l}{\textsc{1sg}} & also & \multicolumn{2}{l}{ascend} & \multicolumn{3}{l}{be.free}\\
\lspbottomrule
\end{tabular}
\ea
\glt 
Klara: yesterday, I, Herman, Markus[\textsc{tru}], Markus, (and) who-is-it, Nofita, I said like this, ‘I also jump free (of the net)’
\z

\begin{tabular}{lllllll}
\lsptoprule
0018 & Oten: & Nofita, & Nofita & pu & bagi{\Tilde}bagi & tu\\
&  & Nofita & Nofita & \textsc{poss} & \textsc{rdp}{\Tilde}divide & \textsc{d.dist}\\
\lspbottomrule
\end{tabular}
\ea
\glt 
Oten: Nofita, Nofita tosses well (Lit. ‘Nofita’s dividing’)
\z

\begin{tabular}{lllllll}
\lsptoprule
0019 & Nelci: & kalo & Nofita & kena & itu & tubir\\
&  & if & Nofita & hit & \textsc{d.dist} & steep\\
\lspbottomrule
\end{tabular}
\ea
\glt 
Nelci: whenever Nofita hits (the ball, it comes down in a) steep (angle)
\z

\begin{tabular}{lllllllllllllllllllllllllll}
\lsptoprule
0020 & \multicolumn{2}{l}{Klara:} & \multicolumn{2}{l}{adu,} & \multicolumn{3}{l}{tong} & \multicolumn{4}{l}{maing} & \multicolumn{2}{l}{tu} & \multicolumn{5}{l}{hancur,} & \multicolumn{2}{l}{tong} & \multicolumn{3}{l}{maing} & \multicolumn{3}{l}{net}\\
& \multicolumn{2}{l}{} & \multicolumn{2}{l}{oh.no!} & \multicolumn{3}{l}{\textsc{1pl}} & \multicolumn{4}{l}{play} & \multicolumn{2}{l}{\textsc{d.dist}} & \multicolumn{5}{l}{be.shattered} & \multicolumn{2}{l}{\textsc{1pl}} & \multicolumn{3}{l}{play} & \multicolumn{3}{l}{(sport.)net}\\
& sebla & \multicolumn{4}{l}{baru} & \multicolumn{3}{l}{ada,} & \multicolumn{2}{l}{a,} & \multicolumn{2}{l}{sebla} & \multicolumn{2}{l}{darat} & \multicolumn{2}{l}{tapi} & \multicolumn{3}{l}{dong} & \multicolumn{3}{l}{bilang} & \multicolumn{3}{l}{begini,} & sebla\\
& side & \multicolumn{4}{l}{and.then} & \multicolumn{3}{l}{exist} & \multicolumn{2}{l}{ah!} & \multicolumn{2}{l}{Side} & \multicolumn{2}{l}{land} & \multicolumn{2}{l}{but} & \multicolumn{3}{l}{\textsc{3pl}} & \multicolumn{3}{l}{say} & \multicolumn{3}{l}{like.this} & side\\
& \multicolumn{3}{l}{net} & \multicolumn{3}{l}{darat} & \multicolumn{3}{l}{tu} & \multicolumn{6}{l}{tinggi{\Tilde}tinggi} & \multicolumn{2}{l}{to?,} & \multicolumn{4}{l}{tinggi} & \multicolumn{3}{l}{itu} & \multicolumn{2}{l}{suda}\\
& \multicolumn{3}{l}{(sport.)net} & \multicolumn{3}{l}{land} & \multicolumn{3}{l}{\textsc{d.dist}} & \multicolumn{6}{l}{\textsc{rdp}{\Tilde}be.high} & \multicolumn{2}{l}{right?} & \multicolumn{4}{l}{be.high} & \multicolumn{3}{l}{\textsc{d.dist}} & \multicolumn{2}{l}{already}\\
\lspbottomrule
\end{tabular}
\ea
\glt 
Klara: oh no!, we did our very playing poorly, we played the net off (the beach), and then there is (one), ah, off the beach, but they talked like this, the net off the beach is very high, right?, its height is fixed
\z

\begin{tabular}{llll}
\lsptoprule
0021 & Oten: & yo, & de\\
&  & yes & \textsc{3sg}\\
\lspbottomrule
\end{tabular}
\ea
\glt 
Oten: yes, it
\z

\begin{tabular}{llllll}
\lsptoprule
0022 & Klara: & de & tinggi & itu & suda\\
&  & \textsc{3sg} & be.high & \textsc{d.dist} & already\\
\lspbottomrule
\end{tabular}
\ea
\glt 
Klara: its height is fixed
\z

\begin{tabular}{llll}
\lsptoprule
0023 & Oten: & de & pu\\
&  & \textsc{3sg} & \textsc{poss}\\
\lspbottomrule
\end{tabular}
\ea
\glt 
Oten: its
\z

\begin{tabular}{lllll}
\lsptoprule
0024 & Klara: & pita & di & atas\\
&  & ribbon.of.volleyball.net & at & top\\
\lspbottomrule
\end{tabular}
\ea
\glt 
Klara: the upper ribbon of the volleyball net
\z

\begin{tabular}{lllllll}
\lsptoprule
0025 & Oten: & yang & pita & di & bawa & itu\\
&  & \textsc{rel} & ribbon.of.volleyball.net & at & bottom & \textsc{d.dist}\\
\lspbottomrule
\end{tabular}
\ea
\glt 
Oten: (its) lower ribbon
\z

\begin{tabular}{lll}
\lsptoprule
0026 & Klara: & batas\\
&  & border\\
\lspbottomrule
\end{tabular}
\ea
\glt 
Klara: (its) height
\z

\begin{tabular}{lllllllllll}
\lsptoprule
0027 & \multicolumn{2}{l}{Oten:} & \multicolumn{2}{l}{sa} & berdiri & pas & batas & ini, & angkat & tangang\\
& \multicolumn{2}{l}{} & \multicolumn{2}{l}{\textsc{1sg}} & stand & be.exact & border & \textsc{d.prox} & lift & hand\\
& tapi & \multicolumn{2}{l}{[\textsc{up}]} & \multicolumn{7}{l}{lewat}\\
& but & \multicolumn{2}{l}{} & \multicolumn{7}{l}{pass.by}\\
\lspbottomrule
\end{tabular}
\ea
\glt 
Oten: (when) I’m standing the lower ribbon is exactly on this height, (when I) lift (my) hand [\textsc{up}]
\z

\begin{tabular}{llllllllllm{-9.4015896E-4in}llllllllllll}
\lsptoprule
0028 & \multicolumn{2}{l}{Klara:} & \multicolumn{5}{l}{makanya} & \multicolumn{2}{l}{kalo} & \multicolumn{3}{l}{bola} & \multicolumn{3}{l}{su} & \multicolumn{2}{l}{mo} & \multicolumn{3}{l}{turung,} & \multicolumn{2}{l}{jang}\\
& \multicolumn{2}{l}{} & \multicolumn{5}{l}{for.that.reason} & \multicolumn{2}{l}{if} & \multicolumn{3}{l}{ball} & \multicolumn{3}{l}{already} & \multicolumn{2}{l}{want} & \multicolumn{3}{l}{descend} & \multicolumn{2}{l}{\textsc{neg.imp}}\\
& ko & \multicolumn{3}{l}{lompat,} & \multicolumn{2}{l}{bola} & \multicolumn{4}{l}{tinggi} & \multicolumn{3}{l}{tu} & \multicolumn{3}{l}{yang} & \multicolumn{2}{l}{ko} & \multicolumn{3}{l}{lompat} & deng\\
& \textsc{2sg} & \multicolumn{3}{l}{jump} & \multicolumn{2}{l}{ball} & \multicolumn{4}{l}{be.high} & \multicolumn{3}{l}{\textsc{d.dist}} & \multicolumn{3}{l}{\textsc{rel}} & \multicolumn{2}{l}{\textsc{2sg}} & \multicolumn{3}{l}{jump} & with\\
& \multicolumn{3}{l}{akang} & \multicolumn{2}{l}{to?,} & \multicolumn{3}{l}{karna} & \multicolumn{3}{l}{bola} & \multicolumn{3}{l}{turung} & \multicolumn{2}{l}{tra} & \multicolumn{3}{l}{akang} & \multicolumn{3}{l}{sampe}\\
& \multicolumn{3}{l}{it[SI]} & \multicolumn{2}{l}{right?} & \multicolumn{3}{l}{because} & \multicolumn{3}{l}{ball} & \multicolumn{3}{l}{descend} & \multicolumn{2}{l}{\textsc{neg}} & \multicolumn{3}{l}{will[SI]} & \multicolumn{3}{l}{reach}\\
\lspbottomrule
\end{tabular}
\ea
\glt 
Klara: so, when the ball is already coming down, don’t jump, (when) the ball is really high, you jump for it, right?, because the ball (that’s) coming down won’t hit the ground
\z

\begin{tabular}{llllllllll}
\lsptoprule
0029 & Oten: & tadi & \multicolumn{2}{l}{tong} & cara & maing & juga, & bola{\Tilde}bola & pul,\\
&  & earlier & \multicolumn{2}{l}{\textsc{1pl}} & manner & play & also & \textsc{rdp}{\Tilde}ball & pool\\
& kejar, & \multicolumn{2}{l}{tangang} & \multicolumn{6}{l}{kembali}\\
& chase & \multicolumn{2}{l}{hand} & \multicolumn{6}{l}{return}\\
\lspbottomrule
\end{tabular}
\ea
\glt 
Oten: earlier the way we played (was) also (good in some way), we played beautifully, chasing and passing (the ball)
\z

\begin{tabular}{llllllllllllllllll}
\lsptoprule
0030 & \multicolumn{3}{l}{Klara:} & \multicolumn{3}{l}{memang,} & \multicolumn{3}{l}{baru} & \multicolumn{2}{l}{net} & de & \multicolumn{2}{l}{spang} & itu, & \multicolumn{2}{l}{mantap}\\
& \multicolumn{3}{l}{} & \multicolumn{3}{l}{indeed} & \multicolumn{3}{l}{and.then} & \multicolumn{2}{l}{(sport.)net} & \textsc{3sg} & \multicolumn{2}{l}{spank} & \textsc{d.dist} & \multicolumn{2}{l}{be.good}\\
& \multicolumn{2}{l}{skali} & \multicolumn{3}{l}{to?,} & \multicolumn{2}{l}{jadi} & tong & \multicolumn{2}{l}{kemaring} & \multicolumn{2}{l}{maing} & deng & \multicolumn{3}{l}{net} & itu\\
& \multicolumn{2}{l}{very} & \multicolumn{3}{l}{right?} & \multicolumn{2}{l}{so} & \textsc{1pl} & \multicolumn{2}{l}{yesterday} & \multicolumn{2}{l}{play} & with & \multicolumn{3}{l}{(sport.)net} & \textsc{d.dist}\\
& dua & \multicolumn{3}{l}{kali} & \multicolumn{13}{l}{saja}\\
& two & \multicolumn{3}{l}{time} & \multicolumn{13}{l}{just}\\
\lspbottomrule
\end{tabular}
\ea
\glt 
Klara: indeed, and then the net was really tight, (it was) very good, right?, so yesterday we played at that net only twice
\z

\begin{tabular}{lllll}
\lsptoprule
0031 & Wili: & sa & yang & [\textsc{up}]\\
&  & \textsc{1sg} & \textsc{rel} & \\
\lspbottomrule
\end{tabular}
\ea
\glt 
Wili: it was me who [\textsc{up}]
\z

\begin{tabular}{llllll}
\lsptoprule
0032 & Nelci: & siapa & yang & ganggu & [\textsc{up}]\\
&  & who & \textsc{rel} & disturb & \\
\lspbottomrule
\end{tabular}
\ea
\glt 
Nelci: who was it who disturbed [\textsc{up}]
\z

\begin{tabular}{lllll}
\lsptoprule
0033 & Klara: & tong & maing & [\textsc{up}]\\
&  & \textsc{1pl} & play & \\
\lspbottomrule
\end{tabular}
\ea
\glt 
Klara: we were playing [\textsc{up}]
\z

\begin{tabular}{lllllllllllllllll}
\lsptoprule
0034 & \multicolumn{2}{l}{Oten:} & \multicolumn{2}{l}{lo,} & \multicolumn{2}{l}{de} & \multicolumn{2}{l}{yang} & \multicolumn{2}{l}{gara,} & ko & \multicolumn{3}{l}{jang} & mo & bilang\\
& \multicolumn{2}{l}{} & \multicolumn{2}{l}{right![SI]} & \multicolumn{2}{l}{\textsc{3sg}} & \multicolumn{2}{l}{\textsc{rel}} & \multicolumn{2}{l}{irritate} & \textsc{2sg} & \multicolumn{3}{l}{\textsc{neg.imp}} & want & say\\
& saya & \multicolumn{2}{l}{laing,} & \multicolumn{2}{l}{ko} & \multicolumn{2}{l}{apa?} & \multicolumn{2}{l}{siapa?} & siapa & \multicolumn{2}{l}{lu,} & siapa & \multicolumn{3}{l}{gua?}\\
& \textsc{1sg} & \multicolumn{2}{l}{again} & \multicolumn{2}{l}{\textsc{2sg}} & \multicolumn{2}{l}{what} & \multicolumn{2}{l}{who} & who & \multicolumn{2}{l}{\textsc{2sg}[JI]} & who & \multicolumn{3}{l}{\textsc{1sg}[JI]}\\
\lspbottomrule
\end{tabular}
\ea
\glt 
Oten: right!, it was him who irritated (you), don’t you accuse me again, who in the world do you think you are?, who are you?, who am I?\footnote{\\
\\
\\
\\
\\
\\
\\
\\
\\
\\
\\
\\
\\
\\
\\
\par The use of the second singular person serves as a rhetorical figure of speech (“apostrophe”) and refers to the absent person who irritated the players (see ‘‘np 2sg’ noun phrases as rhetorical figures of speech (“apostrophes”)’ in §6.2.1.1).}
\z

\begin{tabular}{lllllllllllll}
\lsptoprule
0035 & Klara: & \multicolumn{3}{l}{net} & \multicolumn{2}{l}{sebla} & \multicolumn{2}{l}{kitong,} & itu & yang & langsung & tong\\
&  & \multicolumn{3}{l}{(sport.)net} & \multicolumn{2}{l}{side} & \multicolumn{2}{l}{\textsc{1pl}} & \textsc{d.dist} & \textsc{rel} & immediately & \textsc{1pl}\\
& \multicolumn{2}{l}{turung} & maing & \multicolumn{2}{l}{di} & \multicolumn{2}{l}{net} & \multicolumn{5}{l}{ini}\\
& \multicolumn{2}{l}{descend} & play & \multicolumn{2}{l}{at} & \multicolumn{2}{l}{(sport.)net} & \multicolumn{5}{l}{\textsc{d.prox}}\\
\lspbottomrule
\end{tabular}
\ea
\glt 
Klara: the net on the other side, we, that’s where we immediately went to play at this net
\z

\begin{tabular}{lll}
\lsptoprule
0036 & Oten: & o\\
&  & oh!\\
\lspbottomrule
\end{tabular}
\ea
\glt 
Oten: oh!
\z

\begin{tabular}{lllll}
\lsptoprule
0037 & Klara: & net & prempuang & to?\\
&  & (sport.)net & woman & right?\\
\lspbottomrule
\end{tabular}
\ea
\glt 
Klara: the women’s net, right?
\z

\begin{tabular}{lllllllll}
\lsptoprule
0038 & Oten: & Wili & ko & jang & gara{\Tilde}gara & tanta & dia & itu\\
&  & Wili & \textsc{2sg} & \textsc{neg.imp} & \textsc{rdp}{\Tilde}irritate & aunt & \textsc{3sg} & \textsc{d.dist}\\
\lspbottomrule
\end{tabular}
\ea
\glt 
Oten: you Wili don’t irritate that aunt
\z

\begin{tabular}{lll}
\lsptoprule
0039 & Wili: & mm-mm\\
&  & mhm\\
\lspbottomrule
\end{tabular}
\ea
\glt 
Wili: mhm
\z

\begin{tabular}{lllllllllllll}
\lsptoprule
0040 & Klara: & \multicolumn{2}{l}{Wili} & \multicolumn{2}{l}{ko} & \multicolumn{2}{l}{masuk} & suda, & ko & tadi & dengar & itu\\
&  & \multicolumn{2}{l}{Wili} & \multicolumn{2}{l}{\textsc{2sg}} & \multicolumn{2}{l}{enter} & already & \textsc{2sg} & earlier & hear & \textsc{d.dist}\\
& \multicolumn{2}{l}{burung} & \multicolumn{2}{l}{itu} & \multicolumn{2}{l}{ka} & \multicolumn{6}{l}{tida?}\\
& \multicolumn{2}{l}{bird} & \multicolumn{2}{l}{\textsc{d.dist}} & \multicolumn{2}{l}{or} & \multicolumn{6}{l}{\textsc{neg}}\\
\lspbottomrule
\end{tabular}
\ea
\glt 
Klara: you Wili go inside!, earlier you heard, what’s-its-name, that bird or not?
\z

\begin{tabular}{lllllllllll}
\lsptoprule
0041 & \multicolumn{2}{l}{Oten:} & \multicolumn{2}{l}{o,} & itu & klawar, & de & makang & ini, & mangga\\
& \multicolumn{2}{l}{} & \multicolumn{2}{l}{oh!} & \textsc{d.dist} & cave.bat & \textsc{3sg} & eat & \textsc{d.prox} & mango\\
& ka, & \multicolumn{2}{l}{apa,} & \multicolumn{7}{l}{ketapang}\\
& or & \multicolumn{2}{l}{what} & \multicolumn{7}{l}{tropical-almond}\\
\lspbottomrule
\end{tabular}
\ea
\glt 
Oten: oh, that was a bat, it was eating, what’s-its-name, maybe mangos, what-is-it, tropical-almonds
\z

\begin{tabular}{lllllllllllllllll}
\lsptoprule
0042 & Klara: & \multicolumn{2}{l}{tida,} & \multicolumn{3}{l}{ana{\Tilde}ana} & \multicolumn{2}{l}{kecil} & \multicolumn{2}{l}{kaya} & \multicolumn{2}{l}{begini,} & \multicolumn{3}{l}{ana{\Tilde}ana} & kecil\\
&  & \multicolumn{2}{l}{\textsc{neg}} & \multicolumn{3}{l}{\textsc{rdp}{\Tilde}child} & \multicolumn{2}{l}{be.small} & \multicolumn{2}{l}{like} & \multicolumn{2}{l}{like.this} & \multicolumn{3}{l}{\textsc{rdp}{\Tilde}child} & be.small\\
& \multicolumn{2}{l}{nanti} & \multicolumn{3}{l}{de} & \multicolumn{2}{l}{bangung} & \multicolumn{2}{l}{terlambat,} & \multicolumn{2}{l}{lebi} & \multicolumn{2}{l}{bagus} & ko & \multicolumn{2}{l}{masuk}\\
& \multicolumn{2}{l}{very.soon} & \multicolumn{3}{l}{\textsc{3sg}} & \multicolumn{2}{l}{wake.up} & \multicolumn{2}{l}{be.late} & \multicolumn{2}{l}{more} & \multicolumn{2}{l}{be.good} & \textsc{2sg} & \multicolumn{2}{l}{enter}\\
& tidor & \multicolumn{3}{l}{sana} & \multicolumn{12}{l}{suda}\\
& sleep & \multicolumn{3}{l}{\textsc{l.dist}} & \multicolumn{12}{l}{already}\\
\lspbottomrule
\end{tabular}
\ea
\glt 
Klara: no, young children like him, young children, later he’ll wake up too late, it’s better you go inside and just sleep over there
\z

\begin{tabular}{llll}
\lsptoprule
0043 & Oten: & baru & ko?\\
&  & and.then & \textsc{2sg}\\
\lspbottomrule
\end{tabular}
\ea
\glt 
Oten: and (what about) you?
\z

\begin{tabular}{lllllllllll}
\lsptoprule
0044 & Klara: & ko & \multicolumn{2}{l}{jang} & \multicolumn{2}{l}{bergabung,} & sa & masi & bisa & pu\\
&  & \textsc{2sg} & \multicolumn{2}{l}{\textsc{neg.imp}} & \multicolumn{2}{l}{join} & \textsc{1sg} & still & be.able & \textsc{poss}\\
& \multicolumn{2}{l}{kesadarang} & sa & \multicolumn{2}{l}{bangung} & \multicolumn{5}{l}{tempo}\\
& \multicolumn{2}{l}{awareness} & \textsc{1sg} & \multicolumn{2}{l}{wake.up} & \multicolumn{5}{l}{quick}\\
\lspbottomrule
\end{tabular}
\ea
\glt 
Klara: don’t stay with us any longer, I have enough mindfulness, I wake up early
\z

\begin{tabular}{llllllllll}
\lsptoprule
0045 & Oten: & dari & tadi & siang & sa & yang & kasi & bangung & ko\\
&  & from & earlier & midday & \textsc{1sg} & \textsc{rel} & give & wake.up & \textsc{2sg}\\
\lspbottomrule
\end{tabular}
\ea
\glt 
Oten: earlier this noon, (it was) me who woke you up
\z

\begin{tabular}{lllllllll}
\lsptoprule
0046 & Nelci: & i, & malam & de & bangung, & e & yo & hampir\\
&  & ugh! & night & \textsc{3sg} & wake.up & uh & yes & almost\\
\lspbottomrule
\end{tabular}
\ea
\glt 
Nelci: ugh! (last) night she got up, uh yes, (it was) almost
\z

\begin{tabular}{llllllllll}
\lsptoprule
0047 & Oten: & lo & hampir & siang & sa & yang & bangung & lebi & cepat\\
&  & right![SI] & almost & day & \textsc{1sg} & \textsc{rel} & wake.up & more & be.fast\\
\lspbottomrule
\end{tabular}
\ea
\glt 
Oten: right, it was almost daylight, (it was) me who woke up earlier
\z

\begin{tabular}{llll}
\lsptoprule
0048 & Klara: & em? & e?\\
&  & uh & uh\\
\lspbottomrule
\end{tabular}
\ea
\glt 
Klara: uh, uh
\z

\begin{tabular}{llll}
\lsptoprule
0049 & Oten: & knapa & ka?\\
&  & why & or\\
\lspbottomrule
\end{tabular}
\ea
\glt 
Oten: what happened?
\z

\begin{tabular}{llllllll}
\lsptoprule
0050 & Nelci: & sa & bangung & stenga & empat, & stenga & lima\\
&  & \textsc{1sg} & wake.up & half & four & half & five\\
\lspbottomrule
\end{tabular}
\ea
\glt 
Nelci: I got up at half past three, half past four
\z

\begin{tabular}{llllllllllllllllllllllllllllm{-9.4015896E-4in}lllm{-9.4015896E-4in}l}
\lsptoprule
0051 & \multicolumn{2}{l}{Klara:} & \multicolumn{3}{l}{sa} & \multicolumn{6}{l}{bangung,} & \multicolumn{4}{l}{sa} & \multicolumn{4}{l}{kluar} & \multicolumn{5}{l}{pas} & \multicolumn{3}{l}{ana} & \multicolumn{4}{l}{ini,} & \multicolumn{2}{l}{Nusa}\\
& \multicolumn{2}{l}{} & \multicolumn{3}{l}{\textsc{1sg}} & \multicolumn{6}{l}{wake.up} & \multicolumn{4}{l}{\textsc{1sg}} & \multicolumn{4}{l}{go.out} & \multicolumn{5}{l}{precisely} & \multicolumn{3}{l}{child} & \multicolumn{4}{l}{\textsc{d.prox}} & \multicolumn{2}{l}{Nusa}\\
& juga & \multicolumn{3}{l}{kluar} & \multicolumn{4}{l}{dari} & \multicolumn{4}{l}{dalam,} & \multicolumn{4}{l}{de} & \multicolumn{4}{l}{kas} & \multicolumn{5}{l}{bangung} & \multicolumn{3}{l}{ana} & \multicolumn{4}{l}{ini,} & dong\\
& also & \multicolumn{3}{l}{go.out} & \multicolumn{4}{l}{from} & \multicolumn{4}{l}{inside} & \multicolumn{4}{l}{\textsc{3sg}} & \multicolumn{4}{l}{give} & \multicolumn{5}{l}{wake.up} & \multicolumn{3}{l}{child} & \multicolumn{4}{l}{\textsc{d.prox}} & \textsc{3pl}\\
& \multicolumn{2}{l}{dua} & \multicolumn{4}{l}{kluar} & \multicolumn{3}{l}{cuci} & \multicolumn{5}{l}{piring,} & \multicolumn{4}{l}{dong} & \multicolumn{3}{l}{dua} & \multicolumn{2}{l}{biking} & \multicolumn{3}{l}{te} & \multicolumn{4}{l}{pagi,} & \multicolumn{3}{l}{memang}\\
& \multicolumn{2}{l}{two} & \multicolumn{4}{l}{go.out} & \multicolumn{3}{l}{wash} & \multicolumn{5}{l}{plate} & \multicolumn{4}{l}{\textsc{3pl}} & \multicolumn{3}{l}{two} & \multicolumn{2}{l}{make} & \multicolumn{3}{l}{tea} & \multicolumn{4}{l}{morning} & \multicolumn{3}{l}{indeed}\\
& \multicolumn{3}{l}{hampir} & \multicolumn{4}{l}{siang} & \multicolumn{3}{l}{tu} & \multicolumn{3}{l}{dong} & \multicolumn{4}{l}{dua} & \multicolumn{5}{l}{yang} & \multicolumn{3}{l}{kluar} & \multicolumn{4}{l}{bangung} & \multicolumn{4}{l}{pagi}\\
& \multicolumn{3}{l}{almost} & \multicolumn{4}{l}{day} & \multicolumn{3}{l}{\textsc{d.dist}} & \multicolumn{3}{l}{\textsc{3pl}} & \multicolumn{4}{l}{two} & \multicolumn{5}{l}{\textsc{rel}} & \multicolumn{3}{l}{go.out} & \multicolumn{4}{l}{wake.up} & \multicolumn{4}{l}{morning}\\
\lspbottomrule
\end{tabular}
\ea
\glt 
Klara: I got up, I went outside, in that moment this kid here, Nusa came outside, she woke up this kid,\footnote{\\
\\
\\
\\
\\
\\
\\
\\
\\
\\
\\
\\
\\
\\
\\
\par \textitbf{Klara} refers to \textitbf{Nelci} (see line 0050 and 0053).} the two of them went outside (and) washed the plates, the two of them made the morning tea, indeed it was almost daylight, (it was) the two of them who came outside and woke up the morning
\z

\begin{tabular}{llllll}
\lsptoprule
0052 & Oktofina: & e, & mama & bilang & masuk\\
&  & hey! & mother & say & enter\\
\lspbottomrule
\end{tabular}
\ea
\glt 
Oktofina [addressing Wili]: hey, mother said (you should) go inside
\z

\begin{tabular}{llllllll}
\lsptoprule
0053 & Nelci: & Nusa & cuci & piring, & sa & goreng & nasi\\
&  & Nusa & wash & plate & \textsc{1sg} & fry & cooked.rice\\
\lspbottomrule
\end{tabular}
\ea
\glt 
Nelci: Nusa washed the plates (and) I fried the cooked rice
\z

\begin{tabular}{llllllllllll}
\lsptoprule
0054 & Wili: & de & tipu, & sa & tadi & liat & dia & tida & ada & di & dalam\\
&  & \textsc{3sg} & cheat & \textsc{1sg} & earlier & see & \textsc{3sg} & \textsc{neg} & exist & at & inside\\
\lspbottomrule
\end{tabular}
\ea
\glt 
Wili: she’s deceiving (me), earlier I saw (that) she (mother) wasn’t inside
\z

\begin{tabular}{llllll}
\lsptoprule
0055 & Oktofina: & a, & betul, & ma & bilang\\
&  & ah! & be.true & mother & say\\
\lspbottomrule
\end{tabular}
\ea
\glt 
Oktofina: ah, it’s true, mother said
\z

\begin{tabular}{llllll}
\lsptoprule
0056 & Oten: & siapa & yang & bla & kayu?\\
&  & who & \textsc{rel} & split & wood\\
\lspbottomrule
\end{tabular}
\ea
\glt 
Oten: who was it who split (the fire)wood?
\z

\begin{tabular}{llllllllll}
\lsptoprule
0057 & Klara: & a, & omong & kosong, & ko & masuk & tidor & sana & suda\\
&  & ah! & gossip[SI] & empty & \textsc{2sg} & enter & sleep & \textsc{l.dist} & already\\
\lspbottomrule
\end{tabular}
\ea
\glt 
Klara [addressing Wili]: ah, nonsense, you just go inside (and) sleep over there
\z

\begin{tabular}{llllllllll}
\lsptoprule
0058 & Nelci: & yo, & itu & om & siapa & ni & Hendrikus & pu & maytua\\
&  & yes & \textsc{d.dist} & uncle & who & \textsc{d.prox} & Hendrikus & \textsc{poss} & wife\\
\lspbottomrule
\end{tabular}
\ea
\glt 
Nelci: yes, that was uncle, who is this, Hendrikus’ wife
\z

\begin{tabular}{lll}
\lsptoprule
0059 & Oten: & hm\\
&  & pfft\\
\lspbottomrule
\end{tabular}
\ea
\glt 
Oten: pfft!
\z

\begin{tabular}{lllll}
\lsptoprule
0060 & Nelci: & kapang & ko & bla?\\
&  & when & \textsc{2sg} & split\\
\lspbottomrule
\end{tabular}
\ea
\glt 
Nelci [addressing Oten]: when did you split (the firewood)?
\z

\begin{tabular}{llll}
\lsptoprule
0061 & Klara: & RW, & RW\\
&  & cooked.dog.meat & cooked.dog.meat\\
\lspbottomrule
\end{tabular}
\ea
\glt 
Klara [responding to another interlocutor]: cooked dog meat, cooked dog meat
\z

\begin{tabular}{llllll}
\lsptoprule
0062 & Oten: & sa & yang & bla & sore\\
&  & \textsc{1sg} & \textsc{rel} & split & afternoon\\
\lspbottomrule
\end{tabular}
\ea
\glt 
Oten: (it was) me who split (the firewood) in the afternoon
\z

\begin{tabular}{llllll}
\lsptoprule
0063 & Klara: & RW, & tra & ada & RW\\
&  & cooked.dog.meat & \textsc{neg} & exist & cooked.dog.meat\\
\lspbottomrule
\end{tabular}
\ea
\glt 
Klara: cooked dog meat, there’s no cooked dog meat
\z

\begin{tabular}{llm{4.5984238E-4in}lllllllllllllll}
\lsptoprule
0064 & \multicolumn{2}{l}{Nelci:} & \multicolumn{3}{l}{pagi,} & \multicolumn{2}{l}{bukang} & ko, & \multicolumn{2}{l}{itu} & \multicolumn{3}{l}{su} & \multicolumn{2}{l}{kemaring} & \multicolumn{2}{l}{sore}\\
& \multicolumn{2}{l}{} & \multicolumn{3}{l}{morning} & \multicolumn{2}{l}{\textsc{neg}} & \textsc{2sg} & \multicolumn{2}{l}{\textsc{d.dist}} & \multicolumn{3}{l}{already} & \multicolumn{2}{l}{yesterday} & \multicolumn{2}{l}{afternoon}\\
& yang & \multicolumn{2}{l}{ko} & bla, & \multicolumn{2}{l}{ini} & \multicolumn{3}{l}{pagi} & \multicolumn{2}{l}{lagi} & om & \multicolumn{2}{l}{Hendrikus} & \multicolumn{2}{l}{yang} & bla\\
& \textsc{rel} & \multicolumn{2}{l}{\textsc{2sg}} & split & \multicolumn{2}{l}{\textsc{d.prox}} & \multicolumn{3}{l}{morning} & \multicolumn{2}{l}{again} & uncle & \multicolumn{2}{l}{Hendrikus} & \multicolumn{2}{l}{\textsc{rel}} & split\\
\lspbottomrule
\end{tabular}
\ea
\glt
Nelci: in the morning!! (that) wasn’t you, that was already yesterday afternoon that you split (firewood), this morning, (it was) again uncle Hendrikus who split (the firewood)
\end{styleFreeTranslEngxvpt}

\subsection{Conversation: Buying soap; bringing gasoline to Webro}

\begin{tabular}{ll}
\lsptoprule
File name: & 081110-002-Cv\\
Text type: & Conversation, spontaneous\\
Interlocutors: & 2 older males, 2 older females\\
Length (min.): & 3:55\\
\lspbottomrule
\end{tabular}
\begin{tabular}{lllll}
\lsptoprule
0001 & Ida: & slamat & sore & smua\\
&  & be.safe & afternoon & all\\
\lspbottomrule
\end{tabular}
\ea
\glt 
Ida: good afternoon you all
\z

\begin{tabular}{llll}
\lsptoprule
0002 & Natalia: & sore, & sore\\
&  & afternoon & afternoon\\
\lspbottomrule
\end{tabular}
\ea
\glt 
Natalia: afternoon, afternoon
\z

\begin{tabular}{llllllll}
\lsptoprule
0003 & Natalia: & eh, & bagemana & ipar? & sore, & dari & Jayapura?\\
&  & hey! & how & sibling.in-law & afternoon & from & Jayapura\\
\lspbottomrule
\end{tabular}
\ea
\glt 
Natalia [greeting another visitor]: hey, how is it going brother-in-law?, good afternoon! (did you just get here) from Jayapura?
\z

\begin{tabular}{lll}
\lsptoprule
0004 & MO-1: & [\textsc{up}]\\
\lspbottomrule
\end{tabular}
\ea
\glt 
MO-1: [\textsc{up}]
\z

\begin{tabular}{lllllllll}
\lsptoprule
0005 & Natalia: & aah, & yo! & baru & mana & tong & pu & ipar\\
&  & ah! & yes & and.then & where & \textsc{1pl} & \textsc{poss} & sibling.in-law\\
& \multicolumn{8}{l}{prempuang?}\\
& \multicolumn{8}{l}{woman}\\
\lspbottomrule
\end{tabular}
\ea
\glt 
Natalia: ah, yes! so where is our sister-in-law?
\z

\begin{tabular}{lllllllll}
\lsptoprule
0006 & Ida: & \multicolumn{2}{l}{ipar} & prempuang & yang & baru & lewat & deng\\
&  & \multicolumn{2}{l}{sibling.in-law} & woman & \textsc{rel} & recently & pass.by & with\\
& \multicolumn{2}{l}{ojek} & \multicolumn{6}{l}{((laughter))}\\
& \multicolumn{2}{l}{motorbike.taxi} & \multicolumn{6}{l}{}\\
\lspbottomrule
\end{tabular}
\ea
\glt 
Ida: (it’s our) sister-in-law who passed by with a motorbike taxi a short while ago ((laughter))
\z

\begin{tabular}{lllll}
\lsptoprule
0007 & Natalia: & ey! & baru & lewat?\\
&  & hey! & recently & pass.by\\
\lspbottomrule
\end{tabular}
\ea
\glt 
Natalia: hey! did (she) pass by a short while ago?
\z

\begin{tabular}{llllll}
\lsptoprule
0008 & MO-1: & tadi & lewat & deng & ojek\\
&  & earlier & pass.by & with & motorbike.taxi\\
\lspbottomrule
\end{tabular}
\ea
\glt 
MO-1: earlier she passed by on a motorbike taxi
\z

\begin{tabular}{lll}
\lsptoprule
0009 & Natalia: & yo?\\
&  & yes\\
\lspbottomrule
\end{tabular}
\ea
\glt 
Natalia: yes?
\z

\begin{tabular}{lllllll}
\lsptoprule
0010 & Ida: & de & tadi & lewat & deng & ojek\\
&  & \textsc{3sg} & earlier & pass.by & with & motorbike.taxi\\
\lspbottomrule
\end{tabular}
\ea
\glt 
Ida: earlier she passed by with a motorbike taxi
\z

\begin{tabular}{llllllllll}
\lsptoprule
0011 & Natalia: & ibu, & de & su & bawa & de & pu & maytua? & ((laughter))\\
&  & woman & \textsc{3sg} & already & bring & \textsc{3sg} & \textsc{poss} & wife & \\
\lspbottomrule
\end{tabular}
\ea
\glt 
Natalia: mother, did he already bring his wife? ((laughter))
\z

\begin{tabular}{lllllllll}
\lsptoprule
0012 & Ida: & tra & taw, & tanya & dia, & sa & tra & taw\\
&  & \textsc{neg} & know & ask & \textsc{3sg} & \textsc{1sg} & \textsc{neg} & know\\
\lspbottomrule
\end{tabular}
\ea
\glt 
Ida: I don’t know, ask him, I don’t know
\z

\begin{tabular}{lll}
\lsptoprule
0013 & MO-1: & [\textsc{up}]\\
\lspbottomrule
\end{tabular}
\ea
\glt 
MO-1: [\textsc{up}]
\z

\begin{tabular}{lllllll}
\lsptoprule
0014 & MO-2: & sa & ada & lewat & deng & mobil\\
&  & \textsc{1sg} & exist & pass.by & with & car\\
\lspbottomrule
\end{tabular}
\ea
\glt 
MO-2: I was passing by in a car
\z

\begin{tabular}{llllllllllllllllllllllllllllllll}
\lsptoprule
0015 & \multicolumn{3}{l}{Natalia:} & \multicolumn{5}{l}{bahaya!,} & \multicolumn{5}{l}{((pause))} & \multicolumn{3}{l}{ko} & \multicolumn{4}{l}{punya} & \multicolumn{4}{l}{barang} & \multicolumn{3}{l}{itu} & \multicolumn{3}{l}{masi} & ada?,\\
& \multicolumn{3}{l}{} & \multicolumn{5}{l}{danger} & \multicolumn{5}{l}{} & \multicolumn{3}{l}{\textsc{2sg}} & \multicolumn{4}{l}{\textsc{poss}} & \multicolumn{4}{l}{stuff} & \multicolumn{3}{l}{\textsc{d.dist}} & \multicolumn{3}{l}{still} & exist\\
& \multicolumn{2}{l}{ini} & \multicolumn{4}{l}{sa} & \multicolumn{3}{l}{mo} & \multicolumn{3}{l}{pi,} & \multicolumn{3}{l}{dong} & \multicolumn{4}{l}{ada} & \multicolumn{4}{l}{pesang,} & \multicolumn{3}{l}{sa} & \multicolumn{3}{l}{mo} & \multicolumn{2}{l}{bawa}\\
& \multicolumn{2}{l}{\textsc{d.prox}} & \multicolumn{4}{l}{\textsc{1sg}} & \multicolumn{3}{l}{want} & \multicolumn{3}{l}{go} & \multicolumn{3}{l}{\textsc{3pl}} & \multicolumn{4}{l}{exist} & \multicolumn{4}{l}{order} & \multicolumn{3}{l}{\textsc{1sg}} & \multicolumn{3}{l}{want} & \multicolumn{2}{l}{bring}\\
& \multicolumn{2}{l}{titip} & \multicolumn{3}{l}{di} & \multicolumn{5}{l}{depang} & \multicolumn{4}{l}{situ,} & \multicolumn{4}{l}{bawa} & \multicolumn{3}{l}{ke} & \multicolumn{4}{l}{depang,} & \multicolumn{3}{l}{bukang} & \multicolumn{3}{l}{titip}\\
& \multicolumn{2}{l}{deposit} & \multicolumn{3}{l}{at} & \multicolumn{5}{l}{front} & \multicolumn{4}{l}{\textsc{l.med}} & \multicolumn{4}{l}{bring} & \multicolumn{3}{l}{to} & \multicolumn{4}{l}{front} & \multicolumn{3}{l}{\textsc{neg}} & \multicolumn{3}{l}{deposit}\\
& tapi & \multicolumn{3}{l}{sa} & \multicolumn{3}{l}{pi} & \multicolumn{4}{l}{bawa,} & \multicolumn{6}{l}{kemaring} & \multicolumn{5}{l}{sampe} & sa & \multicolumn{8}{l}{sibuk}\\
& but & \multicolumn{3}{l}{\textsc{1sg}} & \multicolumn{3}{l}{go} & \multicolumn{4}{l}{bring} & \multicolumn{6}{l}{yesterday} & \multicolumn{5}{l}{until} & \textsc{1sg} & \multicolumn{8}{l}{be.busy}\\
\lspbottomrule
\end{tabular}
\ea
\glt 
Natalia: great!, ((pause)) is your stuff still (here)?, right now, I want to go, they ordered (s.th.), I want to bring (and) deposit (it) in front over there, (I want to) bring (it) to the front, not to deposit (it) but I want to go and bring (it), yesterday, (when I) arrived, I was (too) busy (to do it)
\z

\begin{tabular}{lllllllllllll}
\lsptoprule
0016 & Ida: & \multicolumn{2}{l}{ini} & \multicolumn{2}{l}{sa} & ada & cari, & yo, & ini, & sa & ada & cari\\
&  & \multicolumn{2}{l}{\textsc{d.prox}} & \multicolumn{2}{l}{\textsc{1sg}} & exist & search & yes & \textsc{d.prox} & \textsc{1sg} & exist & search\\
& \multicolumn{2}{l}{uang,} & \multicolumn{2}{l}{ini,} & \multicolumn{8}{l}{ojek}\\
& \multicolumn{2}{l}{money} & \multicolumn{2}{l}{\textsc{d.prox}} & \multicolumn{8}{l}{motorbike.taxi}\\
\lspbottomrule
\end{tabular}
\ea
\glt 
Ida: what’s-its-name, I’m looking for, yes, what’s-its-name, I’m looking for money, what’s-its-name, (for) the motorbike taxi
\z

\begin{tabular}{lllllllll}
\lsptoprule
0017 & \multicolumn{2}{l}{Natalia:} & perjalangang, & kemaring & sa & mo & bawa, & kemaring\\
& \multicolumn{2}{l}{} & journey & yesterday & \textsc{1sg} & want & bring & yesterday\\
& dulu & \multicolumn{7}{l}{karna}\\
& be.prior & \multicolumn{7}{l}{because}\\
\lspbottomrule
\end{tabular}
\ea
\glt 
Natalia: (for your) trip, yesterday I wanted to bring (the stuff), the day before yesterday because
\z

\begin{tabular}{lllllllllll}
\lsptoprule
0018 & Ida: & [\textsc{up}] & sabung & saja, & kam & pu & sabung & ada & di & situ\\
&  &  & soap & just & \textsc{2pl} & \textsc{poss} & soap & exist & at & \textsc{l.med}\\
\lspbottomrule
\end{tabular}
\ea
\glt 
Ida: [\textsc{up}] just (laundry) soap, your (laundry) soap is there
\z

\begin{tabular}{llllllll}
\lsptoprule
0019 & Natalia: & damay! & kitong & tra & ada & sabung & ini\\
&  & peace & \textsc{1pl} & \textsc{neg} & exist & soap & \textsc{d.prox}\\
\lspbottomrule
\end{tabular}
\ea
\glt 
Natalia: my goodness!, we don’t have any soap right now!
\z

\begin{tabular}{llllllll}
\lsptoprule
0020 & Ida: & yo, & suda, & kalo & begitu & tinggal & suda!\\
&  & yes & already & if & like.that & stay & already\\
\lspbottomrule
\end{tabular}
\ea
\glt 
Ida: yes!, alright!, if it’s like that, no problem!
\z

\begin{tabular}{lllllllllllllllllllllllllllll}
\lsptoprule
0021 & \multicolumn{3}{l}{Natalia:} & \multicolumn{3}{l}{simpang,} & \multicolumn{4}{l}{sa} & \multicolumn{4}{l}{simpang} & \multicolumn{6}{l}{sratus} & \multicolumn{5}{l}{ribu} & \multicolumn{3}{l}{tu,}\\
& \multicolumn{3}{l}{} & \multicolumn{3}{l}{store} & \multicolumn{4}{l}{\textsc{1sg}} & \multicolumn{4}{l}{store/prepare} & \multicolumn{6}{l}{one:hundred} & \multicolumn{5}{l}{thousand} & \multicolumn{3}{l}{\textsc{d.dist}}\\
& de & \multicolumn{3}{l}{pu} & \multicolumn{3}{l}{bapa} & \multicolumn{5}{l}{ar} & \multicolumn{5}{l}{ambil,} & \multicolumn{2}{l}{de} & \multicolumn{4}{l}{ada} & \multicolumn{5}{l}{du}\\
& \textsc{3sg} & \multicolumn{3}{l}{\textsc{poss}} & \multicolumn{3}{l}{father} & \multicolumn{5}{l}{\textsc{spm}{}-fetch} & \multicolumn{5}{l}{fetch} & \multicolumn{2}{l}{\textsc{3sg}} & \multicolumn{4}{l}{exist} & \multicolumn{5}{l}{\textsc{tru}{}-be.prior}\\
& \multicolumn{5}{l}{d} & \multicolumn{4}{l}{ikut} & \multicolumn{4}{l}{platiang} & \multicolumn{3}{l}{satu} & \multicolumn{5}{l}{minggu} & \multicolumn{3}{l}{di} & \multicolumn{2}{l}{atas,} & \multicolumn{2}{l}{karna}\\
& \multicolumn{5}{l}{\textsc{tru}{}-be.prior} & \multicolumn{4}{l}{follow} & \multicolumn{4}{l}{training} & \multicolumn{3}{l}{one} & \multicolumn{5}{l}{week} & \multicolumn{3}{l}{at} & \multicolumn{2}{l}{top} & \multicolumn{2}{l}{because}\\
& \multicolumn{2}{l}{tadi} & \multicolumn{3}{l}{sa} & \multicolumn{3}{l}{mo} & \multicolumn{3}{l}{cuci} & \multicolumn{4}{l}{pakeang} & \multicolumn{3}{l}{ada} & \multicolumn{4}{l}{taro} & \multicolumn{5}{l}{tinggal,} & [\textsc{up}]\\
& \multicolumn{2}{l}{earlier} & \multicolumn{3}{l}{\textsc{1sg}} & \multicolumn{3}{l}{want} & \multicolumn{3}{l}{wash} & \multicolumn{4}{l}{use-\textsc{pat}} & \multicolumn{3}{l}{exist} & \multicolumn{4}{l}{put} & \multicolumn{5}{l}{stay} & \\
\lspbottomrule
\end{tabular}
\ea
\glt 
Natalia: (I) set aside, I set aside one hundred thousand, my husband\footnote{\\
\\
\\
\\
\\
\\
\\
\\
\\
\\
\\
\\
\\
\\
\\
\par Lit. ‘her father’ (\textitbf{de} ‘\textsc{3sg}’ refers to the speaker’s daughter).} took[\textsc{spm}] took it, he was[\textsc{tru}] was[\textsc{tru}] attending a one-week training (course) up there (at the regent’s office), because earlier I wanted to wash (his) clothes, (but I) had to put it off, [\textsc{up}]
\z

\begin{tabular}{lllllllllll}
\lsptoprule
0022 & Ida: & supaya, & sa & mo & cuci & dong & dua & pu & pakeang & itu\\
&  & so.that & \textsc{1sg} & want & wash & \textsc{3pl} & two & \textsc{poss} & use-\textsc{pat} & \textsc{d.dist}\\
& \multicolumn{10}{l}{yang}\\
& \multicolumn{10}{l}{\textsc{rel}}\\
\lspbottomrule
\end{tabular}
\ea
\glt 
Ida: so that, I want to wash both of their clothes which
\z

\begin{tabular}{lllll}
\lsptoprule
0023 & Natalia: & tra & ada & ma\\
&  & \textsc{neg} & exist & mother\\
\lspbottomrule
\end{tabular}
\ea
\glt 
Natalia: (there) isn’t (any), mother
\z

\begin{tabular}{llllll}
\lsptoprule
0024 & Ida: & su & tra & ada & sabung\\
&  & already & \textsc{neg} & exist & soap\\
\lspbottomrule
\end{tabular}
\ea
\glt 
Ida: alright, there’s no soap
\z

\begin{tabular}{lllllllllllllllll}
\lsptoprule
0025 & \multicolumn{2}{l}{Natalia:} & \multicolumn{4}{l}{tunggu,} & \multicolumn{2}{l}{sabar,} & \multicolumn{2}{l}{kalo} & \multicolumn{2}{l}{mo} & \multicolumn{2}{l}{sabar,} & kalo & masi\\
& \multicolumn{2}{l}{} & \multicolumn{4}{l}{wait} & \multicolumn{2}{l}{be.patient} & \multicolumn{2}{l}{if} & \multicolumn{2}{l}{want} & \multicolumn{2}{l}{be.patient} & if & still\\
& \multicolumn{4}{l}{besok} & mo & \multicolumn{2}{l}{naik,} & \multicolumn{2}{l}{ini} & \multicolumn{2}{l}{bawa} & \multicolumn{2}{l}{ke} & mari, & \multicolumn{2}{l}{nanti}\\
& \multicolumn{4}{l}{tomorrow} & want & \multicolumn{2}{l}{ascend} & \multicolumn{2}{l}{\textsc{d.prox}} & \multicolumn{2}{l}{bring} & \multicolumn{2}{l}{to} & hither & \multicolumn{2}{l}{very.soon}\\
& sa & \multicolumn{2}{l}{yang} & \multicolumn{13}{l}{cuci}\\
& \textsc{1sg} & \multicolumn{2}{l}{\textsc{rel}} & \multicolumn{13}{l}{wash}\\
\lspbottomrule
\end{tabular}
\ea
\glt 
Natalia: wait, be patient, if you want to be patient, if tomorrow (you) still want to go up (to the regent’s office and), what’s-its-name, bring (the clothes) there, I’ll wash (them)
\z

\begin{tabular}{llllllllll}
\lsptoprule
0026 & Ida: & tra & ada, & ini & suda & selesay, & jadi & besok & [\textsc{up}]\\
&  & \textsc{neg} & exist & \textsc{d.prox} & already & finish & so & tomorrow & \\
\lspbottomrule
\end{tabular}
\ea
\glt 
Ida: no, this (meeting) is already over, so tomorrow [\textsc{up}]
\z

\begin{tabular}{llllll}
\lsptoprule
0027 & Natalia: & i, & kam & su & selesay?\\
&  & ugh! & \textsc{2pl} & already & finish\\
\lspbottomrule
\end{tabular}
\ea
\glt 
Natalia: ugh, you already finished?
\z

\begin{tabular}{lllllllll}
\lsptoprule
0028 & Ida: & a, & itu & bukang & apa, & hanya & penyeraang & uang\\
&  & ah! & \textsc{d.dist} & \textsc{neg} & what & only & dedication & money\\
\lspbottomrule
\end{tabular}
\ea
\glt 
Ida: ah, that’s not, what-is-it, (it’s) only the distribution (of) the funds
\z

\begin{tabular}{lll}
\lsptoprule
0029 & Natalia: & o\\
&  & oh!\\
\lspbottomrule
\end{tabular}
\ea
\glt 
Natalia: oh!
\z

\begin{tabular}{llll}
\lsptoprule
0030 & Ida: & saja & [\textsc{up}]\\
&  & just & \\
\lspbottomrule
\end{tabular}
\ea
\glt 
Ida: just [\textsc{up}]
\z

\begin{tabular}{llll}
\lsptoprule
0031 & Natalia: & o, & yo\\
&  & oh! & yes\\
\lspbottomrule
\end{tabular}
\ea
\glt 
Natalia: oh, yes
\z

\begin{tabular}{lllllllll}
\lsptoprule
0032 & Ida: & ibu & bupati & bicarakang & uang & ke & ibu & distrik\\
&  & woman & regent & speak-app & money & to & woman & district\\
\lspbottomrule
\end{tabular}
\ea
\glt 
Ida: Ms. Regent talked to Ms. District (about the) money
\z

\begin{tabular}{lllll}
\lsptoprule
0033 & Natalia: & o, & begitu, & PKK\\
&  & oh! & like.that & family.welfare.program\\
\lspbottomrule
\end{tabular}
\ea
\glt 
Natalia: oh it’s like that, (about) the family welfare program
\z

\begin{tabular}{lll}
\lsptoprule
0034 & Ida: & yo\\
&  & yes\\
\lspbottomrule
\end{tabular}
\ea
\glt 
Ida: yes
\z

\begin{tabular}{lllllllllllllllllm{-2.4015456E-4in}lllllll}
\lsptoprule
0035 & \multicolumn{3}{l}{Natalia:} & \multicolumn{3}{l}{o,} & \multicolumn{3}{l}{kalo} & \multicolumn{4}{l}{begitu} & \multicolumn{3}{l}{siang,} & \multicolumn{2}{l}{tu} & \multicolumn{2}{l}{yang,} & \multicolumn{3}{l}{sa,} & siri\\
& \multicolumn{3}{l}{} & \multicolumn{3}{l}{oh!} & \multicolumn{3}{l}{if} & \multicolumn{4}{l}{like.that} & \multicolumn{3}{l}{midday} & \multicolumn{2}{l}{\textsc{d.dist}} & \multicolumn{2}{l}{\textsc{rel}} & \multicolumn{3}{l}{\textsc{1sg}} & betel.vine\\
& sa & \multicolumn{4}{l}{bawa} & \multicolumn{2}{l}{ke} & \multicolumn{4}{l}{sana} & \multicolumn{3}{l}{dulu,} & \multicolumn{3}{l}{depang} & dulu, & \multicolumn{4}{l}{ini} & \multicolumn{2}{l}{su}\\
& \textsc{1sg} & \multicolumn{4}{l}{bring} & \multicolumn{2}{l}{to} & \multicolumn{4}{l}{\textsc{l.dist}} & \multicolumn{3}{l}{first} & \multicolumn{3}{l}{front} & first & \multicolumn{4}{l}{\textsc{d.prox}} & \multicolumn{2}{l}{already}\\
& \multicolumn{2}{l}{mo} & \multicolumn{6}{l}{sore} & \multicolumn{2}{l}{jadi,} & \multicolumn{2}{l}{sa} & \multicolumn{3}{l}{masak} & \multicolumn{4}{l}{sayur} & \multicolumn{2}{l}{[\textsc{up}],} & \multicolumn{3}{l}{bapa}\\
& \multicolumn{2}{l}{want} & \multicolumn{6}{l}{afternoon} & \multicolumn{2}{l}{so} & \multicolumn{2}{l}{\textsc{1sg}} & \multicolumn{3}{l}{cook} & \multicolumn{4}{l}{vegetable} & \multicolumn{2}{l}{} & \multicolumn{3}{l}{father}\\
& dong & \multicolumn{3}{l}{dari} & \multicolumn{20}{l}{Yawar}\\
& \textsc{3pl} & \multicolumn{3}{l}{from} & \multicolumn{20}{l}{Yawar}\\
\lspbottomrule
\end{tabular}
\ea
\glt 
Natalia: oh, if it’s like that, (I assume the meeting was over) at midday, that’s why, I, the betel vine I’ll bring (it) over there first, (I’ll bring it) to the front first, because now it’s already turning afternoon, I’m cooking the vegetables [\textsc{up}], the men from Yawar
\z

\begin{tabular}{llllllllllllllllll}
\lsptoprule
0036 & Ida: & \multicolumn{2}{l}{hari} & \multicolumn{2}{l}{ini} & \multicolumn{2}{l}{yo} & \multicolumn{2}{l}{suda} & \multicolumn{2}{l}{selesay,} & jadi & \multicolumn{2}{l}{ibu} & \multicolumn{2}{l}{distrik} & de\\
&  & \multicolumn{2}{l}{day} & \multicolumn{2}{l}{\textsc{d.prox}} & \multicolumn{2}{l}{yes} & \multicolumn{2}{l}{already} & \multicolumn{2}{l}{finish} & so & \multicolumn{2}{l}{woman} & \multicolumn{2}{l}{district} & \textsc{3sg}\\
& \multicolumn{2}{l}{kasi} & \multicolumn{2}{l}{kitong} & \multicolumn{2}{l}{dua} & \multicolumn{2}{l}{pu} & \multicolumn{2}{l}{uang} & \multicolumn{3}{l}{ojek} & \multicolumn{2}{l}{pulang} & \multicolumn{2}{l}{pergi}\\
& \multicolumn{2}{l}{give} & \multicolumn{2}{l}{\textsc{1pl}} & \multicolumn{2}{l}{two} & \multicolumn{2}{l}{\textsc{poss}} & \multicolumn{2}{l}{money} & \multicolumn{3}{l}{motorbike.taxi} & \multicolumn{2}{l}{go.home} & \multicolumn{2}{l}{go}\\
\lspbottomrule
\end{tabular}
\ea
\glt 
Ida: today, yes, (the meeting) is already over, so Ms. District gave the two of us money (for) our return fare for the motorbike taxis
\z

\begin{tabular}{lll}
\lsptoprule
0037 & Natalia: & kasiang\\
&  & pity\\
\lspbottomrule
\end{tabular}
\ea
\glt 
Natalia: poor thing!
\z

\begin{tabular}{llllllll}
\lsptoprule
0038 & Ida: & jadi & sa & in & mo & ini, & ini\\
&  & so & \textsc{1sg} & \textsc{tru-d.prox} & want & \textsc{d.prox} & \textsc{d.prox}\\
\lspbottomrule
\end{tabular}
\ea
\glt 
Ida: so, I here[\textsc{tru}] want this (or) this (but I can’t with these limited funds)
\z

\begin{tabular}{llllllllllllllllllllm{-2.4015456E-4in}lllll}
\lsptoprule
0039 & \multicolumn{3}{l}{Natalia:} & \multicolumn{4}{l}{tong} & \multicolumn{2}{l}{dua} & \multicolumn{2}{l}{tra} & \multicolumn{2}{l}{ada,} & \multicolumn{4}{l}{yang} & \multicolumn{3}{l}{pertama} & \multicolumn{2}{l}{itu} & \multicolumn{2}{l}{sa} & su\\
& \multicolumn{3}{l}{} & \multicolumn{4}{l}{\textsc{1pl}} & \multicolumn{2}{l}{two} & \multicolumn{2}{l}{\textsc{neg}} & \multicolumn{2}{l}{exist} & \multicolumn{4}{l}{\textsc{rel}} & \multicolumn{3}{l}{first} & \multicolumn{2}{l}{\textsc{d.dist}} & \multicolumn{2}{l}{\textsc{1sg}} & already\\
& kasi & \multicolumn{5}{l}{dorang,} & \multicolumn{6}{l}{makanya} & \multicolumn{3}{l}{wa} & \multicolumn{3}{l}{mana} & \multicolumn{3}{l}{itu,} & \multicolumn{2}{l}{dong} & \multicolumn{2}{l}{su}\\
& give & \multicolumn{5}{l}{\textsc{3pl}} & \multicolumn{6}{l}{for.that.reason} & \multicolumn{3}{l}{\textsc{spm}} & \multicolumn{3}{l}{where} & \multicolumn{3}{l}{\textsc{d.dist}} & \multicolumn{2}{l}{\textsc{3pl}} & \multicolumn{2}{l}{already}\\
& \multicolumn{2}{l}{mo} & \multicolumn{2}{l}{bli} & \multicolumn{4}{l}{batu,} & \multicolumn{2}{l}{jadi} & \multicolumn{4}{l}{skarang} & \multicolumn{2}{l}{sa,} & \multicolumn{3}{l}{itu,} & \multicolumn{6}{l}{simpang}\\
& \multicolumn{2}{l}{want} & \multicolumn{2}{l}{buy} & \multicolumn{4}{l}{stone} & \multicolumn{2}{l}{so} & \multicolumn{4}{l}{now} & \multicolumn{2}{l}{\textsc{1sg}} & \multicolumn{3}{l}{\textsc{d.dist}} & \multicolumn{6}{l}{store}\\
& \multicolumn{5}{l}{sratus} & \multicolumn{20}{l}{ribu}\\
& \multicolumn{5}{l}{one:hundred} & \multicolumn{20}{l}{thousand}\\
\lspbottomrule
\end{tabular}
\ea
\glt 
Natalia: the two of us haven’t (gotten any money left), I already gave the first (one hundred thousand) to them, that is to say [\textsc{spm}] what-is-it, they already wanted to buy stones, so now I (already), what’s-its-name, set aside one hundred thousand (rupiah)
\z

\begin{tabular}{lllll}
\lsptoprule
0040 & Ida: & yo & suda & kegiatang\\
&  & yes & already & activity\\
\lspbottomrule
\end{tabular}
\ea
\glt 
Ida: yes, well, the activity
\z

\begin{tabular}{llllllllllllllllllllllll}
\lsptoprule
0041 & \multicolumn{2}{l}{Natalia:} & \multicolumn{3}{l}{de} & \multicolumn{3}{l}{bapa,} & \multicolumn{2}{l}{dua} & \multicolumn{4}{l}{ratus} & \multicolumn{3}{l}{de} & \multicolumn{3}{l}{pu} & bapa, & \multicolumn{2}{l}{trus}\\
& \multicolumn{2}{l}{} & \multicolumn{3}{l}{\textsc{3sg}} & \multicolumn{3}{l}{father} & \multicolumn{2}{l}{two} & \multicolumn{4}{l}{hundred} & \multicolumn{3}{l}{\textsc{3sg}} & \multicolumn{3}{l}{\textsc{poss}} & father & \multicolumn{2}{l}{next}\\
& \multicolumn{3}{l}{dep} & \multicolumn{4}{l}{bapa-ade} & \multicolumn{4}{l}{Martin} & \multicolumn{2}{l}{dia} & \multicolumn{3}{l}{bawa} & \multicolumn{2}{l}{lari} & \multicolumn{4}{l}{prempuang,} & adu,\\
& \multicolumn{3}{l}{\textsc{3sg}:\textsc{poss}} & \multicolumn{4}{l}{uncle} & \multicolumn{4}{l}{Martin} & \multicolumn{2}{l}{\textsc{3sg}} & \multicolumn{3}{l}{bring} & \multicolumn{2}{l}{run} & \multicolumn{4}{l}{woman} & oh.no!\\
& in & \multicolumn{3}{l}{tong} & \multicolumn{2}{l}{lagi} & \multicolumn{3}{l}{masala} & \multicolumn{3}{l}{lagi,} & \multicolumn{3}{l}{de} & \multicolumn{4}{l}{bapa-ade} & \multicolumn{4}{l}{Martin}\\
& \textsc{d.prox} & \multicolumn{3}{l}{\textsc{1pl}} & \multicolumn{2}{l}{again} & \multicolumn{3}{l}{problem} & \multicolumn{3}{l}{again} & \multicolumn{3}{l}{\textsc{3sg}} & \multicolumn{4}{l}{uncle} & \multicolumn{4}{l}{Martin}\\
\lspbottomrule
\end{tabular}
\ea
\glt 
Natalia: my husband\footnote{\\
\\
\\
\\
\\
\\
\\
\\
\\
\\
\\
\\
\\
\\
\\
\par Lit. ‘her father’ (\textitbf{de} ‘\textsc{3sg}’ refers to the speaker’s daughter).}, two hundred (thousand for) my husband, and my brother-in-law Martin\footnote{\\
\\
\\
\\
\\
\\
\\
\\
\\
\\
\\
\\
\\
\\
\\
\par Lit. ‘her uncle’ (\textitbf{de} ‘\textsc{3sg}’ refers to the speaker’s daughter).} took a woman away (with him), oh no!, here, we are having problems again, my brother-in-law Martin
\z

\begin{tabular}{llll}
\lsptoprule
0042 & Ida: & naik & motor?\\
&  & ascend & motorbike\\
\lspbottomrule
\end{tabular}
\ea
\glt 
Ida: (he) took (her) on a motorbike?
\z

\begin{tabular}{lllll}
\lsptoprule
0043 & Natalia: & prempuang, & kapal & Papua-Lima\\
&  & woman & ship & Papua-Lima\\
\lspbottomrule
\end{tabular}
\ea
\glt 
Natalia: the woman, (she came with) the Papua-Lima ship
\z

\begin{tabular}{llll}
\lsptoprule
0044 & Ida: & ya & Tuhang\\
&  & yes & God\\
\lspbottomrule
\end{tabular}
\ea
\glt 
Ida: oh God!
\z

\begin{tabular}{llllll}
\lsptoprule
0045 & Natalia: & de & bawa & prempuang & Bagayserwar\\
&  & \textsc{3sg} & bring & woman & Bagayserwar\\
\lspbottomrule
\end{tabular}
\ea
\glt 
Natalia: he brought a woman (from) Bagayserwar
\z

\begin{tabular}{llll}
\lsptoprule
0046 & Ida: & ya & ampung\\
&  & yes & forgiveness\\
\lspbottomrule
\end{tabular}
\ea
\glt 
Ida: for mercy’s sake!
\z

\begin{tabular}{lllllllllll}
\lsptoprule
0047 & Natalia: & kemaring & de & pigi, & sa & pikir & mungking & de & sendiri & pigi\\
&  & yesterday & \textsc{3sg} & go & \textsc{1sg} & think & maybe & \textsc{3sg} & alone & go\\
\lspbottomrule
\end{tabular}
\ea
\glt 
Natalia: yesterday he left, I thought, maybe he went by himself
\z

\begin{tabular}{llllllllllll}
\lsptoprule
0048 & Ida: & \multicolumn{2}{l}{i,} & \multicolumn{2}{l}{e,} & jang & ceritra & banyak, & kasi & sayur & sa\\
&  & \multicolumn{2}{l}{ugh!} & \multicolumn{2}{l}{hey!} & \textsc{neg.imp} & tell & many & give & vegetable & \textsc{1sg}\\
& \multicolumn{2}{l}{makang,} & \multicolumn{2}{l}{sa} & \multicolumn{7}{l}{lapar}\\
& \multicolumn{2}{l}{eat} & \multicolumn{2}{l}{\textsc{1sg}} & \multicolumn{7}{l}{be.hungry}\\
\lspbottomrule
\end{tabular}
\ea
\glt 
Ida: ugh, hey, don’t talk a lot, give me vegetables to eat, I’m hungry
\z

\begin{tabular}{llllllll}
\lsptoprule
0049 & Natalia: & wa, & ko & datang & langsung & ko & lapar?\\
&  & wow! & \textsc{2sg} & come & immediately & \textsc{2sg} & be.hungry\\
\lspbottomrule
\end{tabular}
\ea
\glt 
Natalia: wow, you come (here, and) immediately you’re hungry?
\z

\begin{tabular}{lll}
\lsptoprule
0050 & All: & ((laughter))\\
\lspbottomrule
\end{tabular}
\ea
\glt 
All: ((laughter))
\z

\begin{tabular}{lllllll}
\lsptoprule
0051 & Natalia: & nasi & ada & itu, & timba & suda\\
&  & cooked.rice & exist & \textsc{d.dist} & spoon & already\\
\lspbottomrule
\end{tabular}
\ea
\glt 
Natalia: the cooked rice is over there, just spoon (it)!
\z

\begin{tabular}{lllllll}
\lsptoprule
0052 & Ida: & ah, & sa & tida & makang & nasi\\
&  & ah! & \textsc{1sg} & \textsc{neg} & eat & cooked.rice\\
\lspbottomrule
\end{tabular}
\ea
\glt 
Ida: ah, I don’t eat rice
\z

\begin{tabular}{llll}
\lsptoprule
0053 & Natalia: & habis & apa?\\
&  & after.all & what\\
\lspbottomrule
\end{tabular}
\ea
\glt 
Natalia: so what (do you want)?
\z

\begin{tabular}{lllllll}
\lsptoprule
0054 & Ida: & sa & mo & makang & sayur & saja\\
&  & \textsc{1sg} & want & eat & vegetable & just\\
\lspbottomrule
\end{tabular}
\ea
\glt 
Ida: I just want to eat vegetables
\z

\begin{tabular}{llllllllllllll}
\lsptoprule
0055 & \multicolumn{2}{l}{Natalia:} & yo, & ambil & \multicolumn{2}{l}{piring} & \multicolumn{2}{l}{suda} & di & dalam, & sa & deng & Angela\\
& \multicolumn{2}{l}{} & yes & fetch & \multicolumn{2}{l}{plate} & \multicolumn{2}{l}{already} & at & inside & \textsc{1sg} & with & Angela\\
& ada & \multicolumn{2}{l}{duduk,} & \multicolumn{2}{l}{mama} & \multicolumn{2}{l}{ambil} & \multicolumn{6}{l}{piring}\\
& exist & \multicolumn{2}{l}{sit} & \multicolumn{2}{l}{mother} & \multicolumn{2}{l}{fetch} & \multicolumn{6}{l}{plate}\\
\lspbottomrule
\end{tabular}
\ea
\glt 
Natalia: alright, just get a plate from inside, I and Angela are sitting around, take a plate, mama
\z

\begin{tabular}{lllll}
\lsptoprule
0056 & Ida: & yo, & suda, & sebentar\\
&  & yes & already & in.a.moment\\
\lspbottomrule
\end{tabular}
\ea
\glt 
Ida: yes, alright, (I’ll get one) in a moment
\z

\begin{tabular}{lllllllllllllllllll}
\lsptoprule
0057 & \multicolumn{3}{l}{Natalia:} & \multicolumn{2}{l}{suda,} & \multicolumn{3}{l}{isi} & \multicolumn{2}{l}{sayur} & \multicolumn{2}{l}{suda,} & \multicolumn{2}{l}{masak} & \multicolumn{2}{l}{pertama} & \multicolumn{2}{l}{habis,}\\
& \multicolumn{3}{l}{} & \multicolumn{2}{l}{already} & \multicolumn{3}{l}{fill} & \multicolumn{2}{l}{vegetable} & \multicolumn{2}{l}{already} & \multicolumn{2}{l}{cook} & \multicolumn{2}{l}{first} & \multicolumn{2}{l}{be.used.up}\\
& e & \multicolumn{3}{l}{bapa} & \multicolumn{2}{l}{dong} & \multicolumn{3}{l}{dari} & \multicolumn{2}{l}{Wari,} & \multicolumn{2}{l}{Aruswar} & tra & dapat, & \multicolumn{2}{l}{itu} & yang\\
& uh & \multicolumn{3}{l}{father} & \multicolumn{2}{l}{\textsc{3pl}} & \multicolumn{3}{l}{from} & \multicolumn{2}{l}{Wari} & \multicolumn{2}{l}{Aruswar} & \textsc{neg} & get & \multicolumn{2}{l}{\textsc{d.dist}} & \textsc{rel}\\
& \multicolumn{2}{l}{sa} & \multicolumn{2}{l}{ada} & \multicolumn{3}{l}{masak} & \multicolumn{11}{l}{kangkung}\\
& \multicolumn{2}{l}{\textsc{1sg}} & \multicolumn{2}{l}{exist} & \multicolumn{3}{l}{cook} & \multicolumn{11}{l}{water.spinach}\\
\lspbottomrule
\end{tabular}
\ea
\glt 
Natalia: alright, just fill (the plate with) vegetables, (the food that I) cooked first is finished, uh, the men from Wari, Aruswar didn’t get (any of the food), that’s why I’m cooking water spinach
\z

\begin{tabular}{lllllllllllllllllllll}
\lsptoprule
0058 & de & \multicolumn{4}{l}{pu} & \multicolumn{2}{l}{tanta} & \multicolumn{3}{l}{dong} & \multicolumn{2}{l}{dari} & \multicolumn{2}{l}{Tarfia} & \multicolumn{3}{l}{dorang} & \multicolumn{2}{l}{ini,} & dep\\
& \textsc{3sg} & \multicolumn{4}{l}{\textsc{poss}} & \multicolumn{2}{l}{aunt} & \multicolumn{3}{l}{\textsc{3pl}} & \multicolumn{2}{l}{from} & \multicolumn{2}{l}{Tarfia} & \multicolumn{3}{l}{\textsc{3pl}} & \multicolumn{2}{l}{\textsc{d.prox}} & \textsc{3sg}:\textsc{poss}\\
& \multicolumn{3}{l}{ma,} & \multicolumn{3}{l}{apa,} & \multicolumn{3}{l}{dong} & \multicolumn{2}{l}{pu} & \multicolumn{4}{l}{bapa-ade} & bli & \multicolumn{2}{l}{[Is],} & \multicolumn{2}{l}{suda,}\\
& \multicolumn{3}{l}{\textsc{tru}{}-aunt} & \multicolumn{3}{l}{what} & \multicolumn{3}{l}{\textsc{3pl}} & \multicolumn{2}{l}{\textsc{poss}} & \multicolumn{4}{l}{uncle} & buy & \multicolumn{2}{l}{} & \multicolumn{2}{l}{already}\\
& \multicolumn{2}{l}{trus} & \multicolumn{2}{l}{sa} & \multicolumn{4}{l}{masak} & \multicolumn{5}{l}{nasi} & \multicolumn{7}{l}{pertama}\\
& \multicolumn{2}{l}{next} & \multicolumn{2}{l}{\textsc{1sg}} & \multicolumn{4}{l}{cook} & \multicolumn{5}{l}{cooked.rice} & \multicolumn{7}{l}{first}\\
\lspbottomrule
\end{tabular}
\ea
\glt 
my sister-in-law\footnote{\\
\\
\\
\\
\\
\\
\\
\\
\\
\\
\\
\\
\\
\\
\\
\par Lit. ‘her aunt’ (\textitbf{de} ‘\textsc{3sg}’ refers to the speaker’s daughter).} and the others from Tarfia, my sister-in-law, what-is-it, their uncle bought [Is], well, then I cooked the first meal
\z

\begin{tabular}{llll}
\lsptoprule
0059 & Ida: & baru & [Is]?\\
&  & and.then & \\
\lspbottomrule
\end{tabular}
\ea
\glt 
Ida: and then [Is]?
\z

\begin{tabular}{llllllllllllllllll} & Natalia: & \multicolumn{4}{l}{mama-tua} & \multicolumn{2}{l}{de,} & e, & \multicolumn{2}{l}{kemaring} & \multicolumn{2}{l}{dia} & \multicolumn{2}{l}{kas} & taw & saya & ni,\\
\lsptoprule
&  & \multicolumn{4}{l}{aunt} & \multicolumn{2}{l}{\textsc{3sg}} & uh & \multicolumn{2}{l}{yesterday} & \multicolumn{2}{l}{\textsc{3sg}} & \multicolumn{2}{l}{give} & know & \textsc{1sg} & \textsc{d.prox}\\
& \multicolumn{2}{l}{mama-tua,} & \multicolumn{2}{l}{sa} & \multicolumn{2}{l}{masi} & \multicolumn{3}{l}{sibuk,} & \multicolumn{2}{l}{tunggu,} & \multicolumn{2}{l}{sa} & \multicolumn{2}{l}{blum} & \multicolumn{2}{l}{pigi,}\\
& \multicolumn{2}{l}{aunt} & \multicolumn{2}{l}{\textsc{1sg}} & \multicolumn{2}{l}{still} & \multicolumn{3}{l}{be.busy} & \multicolumn{2}{l}{wait} & \multicolumn{2}{l}{\textsc{1sg}} & \multicolumn{2}{l}{not.yet} & \multicolumn{2}{l}{go}\\
& \multicolumn{3}{l}{sebentar} & \multicolumn{14}{l}{baru}\\
& \multicolumn{3}{l}{in.a.moment} & \multicolumn{14}{l}{and.then}\\
\lspbottomrule
\end{tabular}
\ea
\glt 
Natalia: the aunt, yesterday, she let me here know, aunt, I was still busy, (you) waited, I hadn’t gone yet, a moment later and then
\z

\begin{tabular}{lllllllllll}
\lsptoprule
0061 & Ida: & \multicolumn{2}{l}{sebentar} & bilang & kaka & Nelci & yang & ganti & sa & pu\\
&  & \multicolumn{2}{l}{in.a.moment} & say & oSb & Nelci & \textsc{rel} & replace & \textsc{1sg} & \textsc{poss}\\
& \multicolumn{2}{l}{karong} & \multicolumn{8}{l}{lagi}\\
& \multicolumn{2}{l}{bag} & \multicolumn{8}{l}{again}\\
\lspbottomrule
\end{tabular}
\ea
\glt 
Ida: then tell older sister Nelci who also replaced my bag
\z

\begin{tabular}{llllllllllllllll}
\lsptoprule
0062 & Natalia: & \multicolumn{2}{l}{e,} & \multicolumn{2}{l}{Ise} & o, & \multicolumn{2}{l}{Ise,} & \multicolumn{2}{l}{sa} & \multicolumn{2}{l}{lupa,} & kamu & bawa & pulang\\
&  & \multicolumn{2}{l}{uh} & \multicolumn{2}{l}{Ise} & oh! & \multicolumn{2}{l}{Ise} & \multicolumn{2}{l}{\textsc{1sg}} & \multicolumn{2}{l}{forget} & \textsc{2pl} & bring & go.home\\
& \multicolumn{2}{l}{mama-tua} & \multicolumn{2}{l}{pu} & \multicolumn{3}{l}{cobe,} & \multicolumn{2}{l}{kam} & \multicolumn{2}{l}{bawa} & \multicolumn{4}{l}{[Is]}\\
& \multicolumn{2}{l}{aunt} & \multicolumn{2}{l}{\textsc{poss}} & \multicolumn{3}{l}{mortar} & \multicolumn{2}{l}{\textsc{2pl}} & \multicolumn{2}{l}{bring} & \multicolumn{4}{l}{}\\
\lspbottomrule
\end{tabular}
\ea
\glt 
Natalia [addressing her daughter Ise]: uh, Ise, I forgot, return aunt’s mortar, return [Is]
\z

\begin{tabular}{lllllllllllll}
\lsptoprule
0063 & Ida: & \multicolumn{2}{l}{itu} & yang & sa & \multicolumn{2}{l}{tadi} & bilang & tu, & tadi & sa & bilang\\
&  & \multicolumn{2}{l}{\textsc{d.dist}} & \textsc{rel} & \textsc{1sg} & \multicolumn{2}{l}{earlier} & say & \textsc{d.dist} & earlier & \textsc{1sg} & say\\
& \multicolumn{2}{l}{mama-tua,} & \multicolumn{2}{l}{tolong} & \multicolumn{2}{l}{karna} & \multicolumn{6}{l}{besok}\\
& \multicolumn{2}{l}{aunt} & \multicolumn{2}{l}{help} & \multicolumn{2}{l}{because} & \multicolumn{6}{l}{tomorrow}\\
\lspbottomrule
\end{tabular}
\ea
\glt 
Ida: that’s what I said earlier, earlier I said to aunt, ‘please, because tomorrow’
\z

\begin{tabular}{lllllll}
\lsptoprule
0064 & Natalia: & mo & pulang, & [\textsc{up}] & sa & bawa\\
&  & want & go.home &  & \textsc{1sg} & bring\\
\lspbottomrule
\end{tabular}
\ea
\glt 
Natalia: (I) want to go home, [\textsc{up}] I bring
\z

\begin{tabular}{lllllllll}
\lsptoprule
0065 & Ida: & skarang & kamu & kasi & terpol{\Tilde}terpol & taru & di & sini\\
&  & now & \textsc{2pl} & give & \textsc{rdp}{\Tilde}container & put & at & \textsc{l.prox}\\
\lspbottomrule
\end{tabular}
\ea
\glt 
Ida: now you give (me) the jerry cans, put (them) here
\z

\begin{tabular}{lllllllll}
\lsptoprule
0066 & Natalia: & terpol & [Is], & ey, & yang & besar{\Tilde}besar & itu & jangang\\
&  & container &  & hey! & \textsc{rel} & \textsc{rdp}{\Tilde}be.big & \textsc{d.dist} & \textsc{neg.imp}\\
\lspbottomrule
\end{tabular}
\ea
\glt 
Natalia: the jerry cans, [Is], hey, those big ones, don’t (take them)!
\z

\begin{tabular}{lllll}
\lsptoprule
0067 & Ida: & a, & yang & kecil{\Tilde}kecil\\
&  & ah! & \textsc{rel} & \textsc{rdp}{\Tilde}be.small\\
\lspbottomrule
\end{tabular}
\ea
\glt 
Ida: ah, (I take the ones) that are small
\z

\begin{tabular}{llllllllllll}
\lsptoprule
0068 & \multicolumn{2}{l}{Natalia:} & ey, & ada & tu, & silakang, & ko & mo & bawa & pergi, & ko\\
& \multicolumn{2}{l}{} & hey! & exist & \textsc{d.dist} & please & \textsc{2sg} & want & bring & go & \textsc{2sg}\\
& bawa & \multicolumn{10}{l}{duluang}\\
& bring & \multicolumn{10}{l}{be.prior-\textsc{pat}}\\
\lspbottomrule
\end{tabular}
\ea
\glt 
Natalia: hey, (they) are there, please, (if) you want to take (them) away, take (them) and go ahead
\z

\begin{tabular}{lllll}
\lsptoprule
0069 & Ida: & yo & itu & smua\\
&  & yes & \textsc{d.dist} & all\\
\lspbottomrule
\end{tabular}
\ea
\glt 
Ida: yes, all (of them)
\z

\begin{tabular}{lllll}
\lsptoprule
0070 & Natalia: & ko & bawa & duluang\\
&  & \textsc{2sg} & bring & be.prior-\textsc{pat}\\
\lspbottomrule
\end{tabular}
\ea
\glt 
Natalia: take them (and) go ahead
\z

\begin{tabular}{lllllllllllll}
\lsptoprule
0071 & Ida: & \multicolumn{2}{l}{smua} & kasi & ke & mari, & sa & mo & bawa, & [\textsc{up}] & di & sana\\
&  & \multicolumn{2}{l}{all} & give & to & hither & \textsc{1sg} & want & bring &  & at & \textsc{l.dist}\\
& \multicolumn{2}{l}{tida} & \multicolumn{10}{l}{ada}\\
& \multicolumn{2}{l}{\textsc{neg}} & \multicolumn{10}{l}{exist}\\
\lspbottomrule
\end{tabular}
\ea
\glt 
Ida: give all of them to (me) here, I want to take (them) [\textsc{up}], over there aren’t (any)
\z

\begin{tabular}{llllllll}
\lsptoprule
0072 & Natalia: & sa & stembay, & sa & stembay, & ini, & bensing\\
&  & \textsc{1sg} & stand.by.for & \textsc{1sg} & stand.by.for & \textsc{d.prox} & gasoline\\
\lspbottomrule
\end{tabular}
\ea
\glt 
Natalia: I stand by, I stand by (with), what’s-its-name, the gasoline
\z

\begin{tabular}{llllllll}
\lsptoprule
0073 & Ida: & ko & stembay & bensing, & ko & bli & bensing\\
&  & \textsc{2sg} & stand.by.for & gasoline & \textsc{2sg} & buy & gasoline\\
\lspbottomrule
\end{tabular}
\ea
\glt 
Ida: you stand by (with) the gasoline, you buy gasoline
\z

\begin{tabular}{lll}
\lsptoprule
0074 & Natalia: & yo\\
&  & yes\\
\lspbottomrule
\end{tabular}
\ea
\glt 
Natalia: yes
\z

\begin{tabular}{llllllllllllll}
\lsptoprule
0075 & Ida: & \multicolumn{2}{l}{terpol} & \multicolumn{3}{l}{itu,} & \multicolumn{4}{l}{LNG} & pu & terpol & itu\\
&  & \multicolumn{2}{l}{container} & \multicolumn{3}{l}{\textsc{d.dist}} & \multicolumn{4}{l}{liquefied.natural.gas} & \textsc{poss} & container & \textsc{d.dist}\\
& \multicolumn{2}{l}{tinggal,} & \multicolumn{2}{l}{itu} & ko & \multicolumn{2}{l}{isi} & bensing & di & \multicolumn{4}{l}{situ}\\
& \multicolumn{2}{l}{stay} & \multicolumn{2}{l}{\textsc{d.dist}} & \textsc{2sg} & \multicolumn{2}{l}{fill} & gasoline & at & \multicolumn{4}{l}{\textsc{l.med}}\\
\lspbottomrule
\end{tabular}
\ea
\glt 
Ida: those jerry cans, that LNG jerry can stays behind, that (metal one), you fill the gasoline in there
\z

\begin{tabular}{lll}
\lsptoprule
0076 & Natalia: & yo\\
&  & yes\\
\lspbottomrule
\end{tabular}
\ea
\glt 
Natalia: yes
\z

\begin{tabular}{lllll}
\lsptoprule
0077 & Ida: & empat & liter & saja\\
&  & four & liter & just\\
\lspbottomrule
\end{tabular}
\ea
\glt 
Ida: just four liters
\z

\begin{tabular}{llllllllllll}
\lsptoprule
0078 & Natalia: & \multicolumn{2}{l}{ey,} & \multicolumn{2}{l}{empat} & \multicolumn{2}{l}{e,} & kasiang, & mama & kampung & di\\
&  & \multicolumn{2}{l}{hey!} & \multicolumn{2}{l}{four} & \multicolumn{2}{l}{uh} & pity & mother & village & at\\
& \multicolumn{2}{l}{ba} & \multicolumn{2}{l}{laut} & \multicolumn{2}{l}{mo} & \multicolumn{5}{l}{bli}\\
& \multicolumn{2}{l}{\textsc{tru}{}-bottom} & \multicolumn{2}{l}{sea} & \multicolumn{2}{l}{want} & \multicolumn{5}{l}{buy}\\
\lspbottomrule
\end{tabular}
\ea
\glt 
Natalia: hey four (liters), uh, poor thing, Ms. Mayor down[\textsc{tru}] (at the) seaside wants to buy
\z

\begin{tabular}{lllll}
\lsptoprule
0079 & Ida: & o & yo & suda\\
&  & oh! & yes & already\\
\lspbottomrule
\end{tabular}
\ea
\glt 
Ida: yes, that’s it
\z

\begin{tabular}{lllllllll}
\lsptoprule
0080 & Natalia: & sa & tra & bisa & kasi & sembarang & orang, & mama\\
&  & \textsc{1sg} & \textsc{neg} & be.able & give & any(.kind.of) & person & mother\\
& \multicolumn{8}{l}{kampung}\\
& \multicolumn{8}{l}{village}\\
\lspbottomrule
\end{tabular}
\ea
\glt 
Natalia: I can’t give (the gasoline to just) any person, (but) Ms. Mayor
\z

\begin{tabular}{lll}
\lsptoprule
0081 & Ida: & [Is]\\
\lspbottomrule
\end{tabular}
\ea
\glt 
Ida: [Is]
\z

\begin{tabular}{llll}
\lsptoprule
0082 & Natalia: & Nusa & mama\\
&  & Nusa & mother\\
\lspbottomrule
\end{tabular}
\ea
\glt 
Natalia: Nusa’s mother
\z

\begin{tabular}{lll}
\lsptoprule
0083 & Ida: & [Is]\\
\lspbottomrule
\end{tabular}
\ea
\glt 
Ida: [Is]
\z

\begin{tabular}{llllllllllllllllllm{4.5984238E-4in}llllll}
\lsptoprule
0084 & \multicolumn{3}{l}{Natalia:} & \multicolumn{3}{l}{kitong} & \multicolumn{4}{l}{lima} & \multicolumn{3}{l}{liter,} & \multicolumn{2}{l}{itu} & \multicolumn{2}{l}{saja,} & \multicolumn{3}{l}{yang} & \multicolumn{4}{l}{laing{\Tilde}laing}\\
& \multicolumn{3}{l}{} & \multicolumn{3}{l}{\textsc{1pl}} & \multicolumn{4}{l}{five} & \multicolumn{3}{l}{liter} & \multicolumn{2}{l}{\textsc{d.dist}} & \multicolumn{2}{l}{just} & \multicolumn{3}{l}{\textsc{rel}} & \multicolumn{4}{l}{\textsc{rdp}{\Tilde}be.different}\\
& \multicolumn{2}{l}{mmm,} & \multicolumn{2}{l}{sa} & \multicolumn{4}{l}{su} & \multicolumn{3}{l}{tra} & \multicolumn{3}{l}{maw,} & \multicolumn{2}{l}{sembuni} & \multicolumn{3}{l}{mati,} & \multicolumn{3}{l}{jadi} & ko & bawa\\
& \multicolumn{2}{l}{uh} & \multicolumn{2}{l}{\textsc{1sg}} & \multicolumn{4}{l}{already} & \multicolumn{3}{l}{\textsc{neg}} & \multicolumn{3}{l}{want} & \multicolumn{2}{l}{hide} & \multicolumn{3}{l}{die} & \multicolumn{3}{l}{so} & \textsc{2sg} & bring\\
& \multicolumn{4}{l}{laing,} & \multicolumn{5}{l}{laing} & \multicolumn{3}{l}{sa} & \multicolumn{6}{l}{tahang,} & e, & \multicolumn{2}{l}{sa} & \multicolumn{3}{l}{tahang,}\\
& \multicolumn{4}{l}{be.different} & \multicolumn{5}{l}{be.different} & \multicolumn{3}{l}{\textsc{1sg}} & \multicolumn{6}{l}{hold(.out/back)} & uh & \multicolumn{2}{l}{\textsc{1sg}} & \multicolumn{3}{l}{hold(.out/back)}\\
& kas & \multicolumn{4}{l}{tinggal,} & \multicolumn{2}{l}{sa} & \multicolumn{17}{l}{spulu}\\
& give & \multicolumn{4}{l}{stay} & \multicolumn{2}{l}{\textsc{1sg}} & \multicolumn{17}{l}{one-tens}\\
\lspbottomrule
\end{tabular}
\ea
\glt 
Natalia: we’ll (buy) five liters, that’s it, the others, uh, I already don’t want (to buy gasoline for them), hide (it) from sight, so you take some, I keep some, uh, I keep (some), leave it, I’ll (buy) ten (liters)
\z

\begin{tabular}{lll}
\lsptoprule
0085 & MO-1: & [Is]\\
\lspbottomrule
\end{tabular}
\ea
\glt 
MO-1: [Is]
\z

\begin{tabular}{lllllllllllllll}
\lsptoprule
0086 & Ida: & \multicolumn{2}{l}{[Is],} & \multicolumn{2}{l}{sa} & liat & \multicolumn{3}{l}{dulu,} & \multicolumn{2}{l}{nanti} & sa & sendiri & yang\\
&  & \multicolumn{2}{l}{} & \multicolumn{2}{l}{\textsc{1sg}} & see & \multicolumn{3}{l}{first} & \multicolumn{2}{l}{very.soon} & \textsc{1sg} & be.alone & \textsc{rel}\\
& \multicolumn{2}{l}{pili} & \multicolumn{2}{l}{mana} & \multicolumn{3}{l}{yang} & sa & \multicolumn{2}{l}{m} & \multicolumn{4}{l}{bawa}\\
& \multicolumn{2}{l}{choose} & \multicolumn{2}{l}{where} & \multicolumn{3}{l}{\textsc{rel}} & \textsc{1sg} & \multicolumn{2}{l}{\textsc{tru}{}-want} & \multicolumn{4}{l}{bring}\\
\lspbottomrule
\end{tabular}
\ea
\glt 
Ida: [Is] I’ll have a look first, then (it’ll be) me who’ll choose which (jerry can) I want[\textsc{tru}] to take
\z

\begin{tabular}{lllllllm{-2.4015456E-4in}llllllll}
\lsptoprule
0087 & \multicolumn{3}{l}{Natalia:} & \multicolumn{2}{l}{yang} & \multicolumn{2}{l}{itu,} & yang & itu & tu, & \multicolumn{2}{l}{adu} & \multicolumn{2}{l}{ini,} & ana{\Tilde}ana\\
& \multicolumn{3}{l}{} & \multicolumn{2}{l}{\textsc{rel}} & \multicolumn{2}{l}{\textsc{d.dist}} & \textsc{rel} & \textsc{d.dist} & \textsc{d.dist} & \multicolumn{2}{l}{oh.no!} & \multicolumn{2}{l}{\textsc{d.prox}} & \textsc{rdp}{\Tilde}child\\
& \multicolumn{2}{l}{ini} & \multicolumn{3}{l}{dong} & \multicolumn{2}{l}{tra} & \multicolumn{2}{l}{menyimpang,} & \multicolumn{2}{l}{ini} & \multicolumn{2}{l}{bapa-tua} & \multicolumn{2}{l}{kampung,}\\
& \multicolumn{2}{l}{\textsc{d.prox}} & \multicolumn{3}{l}{\textsc{3pl}} & \multicolumn{2}{l}{\textsc{neg}} & \multicolumn{2}{l}{store/prepare} & \multicolumn{2}{l}{\textsc{d.prox}} & \multicolumn{2}{l}{uncle} & \multicolumn{2}{l}{village}\\
& u, & \multicolumn{3}{l}{Arbais,} & \multicolumn{2}{l}{Arbais} & \multicolumn{9}{l}{punya}\\
& uh & \multicolumn{3}{l}{Arbais} & \multicolumn{2}{l}{Arbais} & \multicolumn{9}{l}{\textsc{poss}}\\
\lspbottomrule
\end{tabular}
\ea
\glt 
Natalia: that one, that one there, oh no, what’s-its-name, these children they didn’t store (the jerry cans well), this one is (the jerry can) of uncle Mayor, umh (from) Arbais, Arbais
\z

\begin{tabular}{llllllllll}
\lsptoprule
0088 & Ida: & yo, & sa & tra & minta & yang & besar, & yang & kecil\\
&  & yes & \textsc{1sg} & \textsc{neg} & request & \textsc{rel} & be.big & \textsc{rel} & be.small\\
\lspbottomrule
\end{tabular}
\ea
\glt
Ida: yes, I don’t ask for (the one) that is one, (I ask for the one) that is small
\end{styleFreeTranslEngxvpt}

\subsection{Conversation: Wanting bananas}

\begin{tabular}{ll}
\lsptoprule
File name: & 081011-003-Cv\\
Text type: & Conversation, spontaneous\\
Interlocutors: & 1 male child, 2 younger females, 2 older females\\
Length (min.): & 0:35\\
\lspbottomrule
\end{tabular}
\begin{tabular}{llllllll}
\lsptoprule
0001 & Fanceria: & kecil & malam & dia & menangis & pisang & goreng\\
&  & be.small & night & \textsc{3sg} & cry & banana & fry\\
\lspbottomrule
\end{tabular}
\ea
\glt 
Fanceria: (this) little (boy Nofi), (last) night he cried (for) fried bananas
\z

\begin{tabular}{llllllllll}
\lsptoprule
0002 & Marta: & \multicolumn{2}{l}{yo,} & dong & dua & deng & Wili & tu & biking\\
&  & \multicolumn{2}{l}{yes} & \textsc{3pl} & two & with & Wili & \textsc{d.dist} & make\\
& \multicolumn{2}{l}{pusing} & \multicolumn{7}{l}{mama}\\
& \multicolumn{2}{l}{be.dizzy} & \multicolumn{7}{l}{mother}\\
\lspbottomrule
\end{tabular}
\ea
\glt 
Marta: yes! he and Wili there worried (their) mother
\z

\begin{tabular}{llllllllll}
\lsptoprule
0003 & Fanceria: & ay, & pisang & di & sana & itu & yang & mo & bli\\
&  & aw! & banana & at & \textsc{l.dist} & \textsc{d.dist} & \textsc{rel} & want & buy\\
\lspbottomrule
\end{tabular}
\ea
\glt 
Fanceria: aw! (it was) the bananas (from) over there which (Nofi) wanted to buy
\z

\begin{tabular}{llllllllll}
\lsptoprule
0004 & Marta: & [\textsc{up}] & ni & tra & rasa & sakit, & dapat & pukul & trus\\
&  &  & \textsc{d.prox} & \textsc{neg} & feel & be.sick & get & hit & be.continuous\\
\lspbottomrule
\end{tabular}
\ea
\glt 
Marta: [\textsc{up}] here doesn’t feel sick, (he) gets beaten continuously
\z

\begin{tabular}{lllll}
\lsptoprule
0005 & Nofi: & sa & pu & sribu\\
&  & \textsc{1sg} & \textsc{poss} & one-thousand\\
\lspbottomrule
\end{tabular}
\ea
\glt 
Nofi: (that’s) my one thousand (rupiah bill)
\z

\begin{tabular}{llllll}
\lsptoprule
0006 & Fanceria: & yo, & ini & kertas & ((laughter))\\
&  & yes & \textsc{d.prox} & paper & \\
\lspbottomrule
\end{tabular}
\ea
\glt 
Fanceria: yes, this is (only) paper (but not money) ((laughter))
\z

\begin{tabular}{lllll}
\lsptoprule
0007 & Nofi: & ko & gila & ka?\\
&  & \textsc{2sg} & be.crazy & or\\
\lspbottomrule
\end{tabular}
\ea
\glt 
Nofi: are you crazy?
\z

\begin{tabular}{lllllll}
\lsptoprule
0008 & Nofita: & terlalu & nakal & ana{\Tilde}ana & di & sini\\
&  & too & be.mischievous & \textsc{rdp}{\Tilde}child & at & \textsc{l.prox}\\
\lspbottomrule
\end{tabular}
\ea
\glt 
Nofita: (they are) too mischievous the children here
\z

\begin{tabular}{lllll}
\lsptoprule
0009 & Fanceria: & a, & Nofi & [\textsc{up}]\\
&  & ah! & Nofi & \\
\lspbottomrule
\end{tabular}
\ea
\glt 
Fanceria: ah, Nofi [\textsc{up}]
\z

\begin{tabular}{lll}
\lsptoprule
0010 & Marta: & [Is]\\
\lspbottomrule
\end{tabular}
\ea
\glt 
Marta: [Is]
\z

\begin{tabular}{lll}
\lsptoprule
0011 & Fanceria: & mm-mm\\
&  & mhm\\
\lspbottomrule
\end{tabular}
\ea
\glt 
Fanceria: mhm
\z

\begin{tabular}{lllllllll}
\lsptoprule
0012 & Marta: & tida & ada & pisang & goreng, & menangis & pisang & goreng\\
&  & \textsc{neg} & exist & banana & fry & cry & banana & fry\\
\lspbottomrule
\end{tabular}
\ea
\glt 
Marta: (when) there aren’t (any) fried bananas, (then Nofi) cries (for) fried bananas
\z

\begin{tabular}{lll}
\lsptoprule
0013 & Fanceria: & ((laughter))\\
\lspbottomrule
\end{tabular}
\ea
\glt 
Fanceria: ((laughter))
\z

\begin{tabular}{llllllll}
\lsptoprule
0014 & Nofita: & ada & pisang & goreng, & tra & maw & makang\\
&  & exist & banana & fry & \textsc{neg} & want & eat\\
\lspbottomrule
\end{tabular}
\ea
\glt 
Nofita: (when) there are fried bananas, (he) doesn’t want to eat (them)
\z

\begin{tabular}{llllllll}
\lsptoprule
0015 & Marta: & ada & pisang & goreng, & tida & maw & makang\\
&  & exist & banana & fry & \textsc{neg} & want & eat\\
& pisang & \multicolumn{6}{l}{goreng}\\
& banana & \multicolumn{6}{l}{fry}\\
\lspbottomrule
\end{tabular}
\ea
\glt 
Marta: (when) there are fried bananas, (he) doesn’t want to eat fried bananas
\z

\begin{tabular}{lllllll}
\lsptoprule
0016 & Klara: & putar & balik, & ana & kecil & itu\\
&  & turn.around & turn.around & child & be.small & \textsc{d.dist}\\
\lspbottomrule
\end{tabular}
\ea
\glt 
Klara: (Nofi) constantly changes (his) opinion, that small child
\z

\begin{tabular}{llllllllll}
\lsptoprule
0017 & Fanceria: & \multicolumn{2}{l}{pisang} & goreng, & pisang & Sorong & sana & tu & iii,\\
&  & \multicolumn{2}{l}{banana} & fry & banana & Sorong & \textsc{l.dist} & \textsc{d.dist} & oh\\
& \multicolumn{2}{l}{besar{\Tilde}besar} & \multicolumn{7}{l}{manis}\\
& \multicolumn{2}{l}{\textsc{rdp}{\Tilde}be.big} & \multicolumn{7}{l}{be.sweet}\\
\lspbottomrule
\end{tabular}
\ea
\glt
Fanceria: fried bananas, those bananas (from) Sorong over there, oooh, (they) are all big (and) sweet
\end{styleFreeTranslEngxvpt}

\subsection{Narrative: A drunkard in the hospital at night}

\begin{tabular}{ll}
\lsptoprule
File name: & 080916-001-CvNP\\
Text type: & Conversation, spontaneous: Personal narrative\\
Interlocutors: & 2 older females\\
Length (min.): & 2:33\\
\lspbottomrule
\end{tabular}
\begin{tabular}{lllllllllllllllllllll}
\lsptoprule
0001 & Marta: & \multicolumn{2}{l}{...} & \multicolumn{2}{l}{de} & \multicolumn{2}{l}{bilang,} & \multicolumn{3}{l}{mama-ade} & \multicolumn{3}{l}{bangung} & \multicolumn{4}{l}{pergi} & \multicolumn{2}{l}{makang} & di\\
&  & \multicolumn{2}{l}{...} & \multicolumn{2}{l}{\textsc{3sg}} & \multicolumn{2}{l}{say} & \multicolumn{3}{l}{aunt} & \multicolumn{3}{l}{wake.up} & \multicolumn{4}{l}{go} & \multicolumn{2}{l}{eat} & at\\
& \multicolumn{3}{l}{warung,} & \multicolumn{2}{l}{sa} & \multicolumn{2}{l}{bilang,} & \multicolumn{2}{l}{Tuhang} & \multicolumn{3}{l}{ini} & \multicolumn{3}{l}{jaw} & \multicolumn{3}{l}{malam} & \multicolumn{2}{l}{begini}\\
& \multicolumn{3}{l}{food.stall} & \multicolumn{2}{l}{\textsc{1sg}} & \multicolumn{2}{l}{say} & \multicolumn{2}{l}{God} & \multicolumn{3}{l}{\textsc{d.prox}} & \multicolumn{3}{l}{far} & \multicolumn{3}{l}{night} & \multicolumn{2}{l}{like.this}\\
& \multicolumn{2}{l}{makang} & \multicolumn{2}{l}{di} & \multicolumn{2}{l}{warung} & \multicolumn{2}{l}{ini} & \multicolumn{3}{l}{suda} & \multicolumn{3}{l}{jam} & \multicolumn{2}{l}{dua} & \multicolumn{4}{l}{malam}\\
& \multicolumn{2}{l}{eat} & \multicolumn{2}{l}{at} & \multicolumn{2}{l}{food.stall} & \multicolumn{2}{l}{\textsc{d.prox}} & \multicolumn{3}{l}{already} & \multicolumn{3}{l}{hour} & \multicolumn{2}{l}{two} & \multicolumn{4}{l}{night}\\
\lspbottomrule
\end{tabular}
\ea
\glt 
Marta: … he (Pawlus) said (to me), ‘aunt get-up, go and eat at the food stall’, I said, ‘God, it’s too late at night to eat at the food stall, this is already two o’clock at night’
\z

\begin{tabular}{llllllllllllllllllllllllllllll}
\lsptoprule
0002 & sa & \multicolumn{5}{l}{bilang,} & \multicolumn{4}{l}{ap} & \multicolumn{4}{l}{[\textsc{up}]} & \multicolumn{4}{l}{Pawlus} & \multicolumn{3}{l}{kalo} & \multicolumn{3}{l}{ko} & \multicolumn{5}{l}{simpang}\\
& \textsc{1sg} & \multicolumn{5}{l}{say} & \multicolumn{4}{l}{\textsc{tru}{}-what} & \multicolumn{4}{l}{} & \multicolumn{4}{l}{Pawlus} & \multicolumn{3}{l}{if} & \multicolumn{3}{l}{\textsc{2sg}} & \multicolumn{5}{l}{store}\\
& \multicolumn{2}{l}{musu} & \multicolumn{2}{l}{di} & \multicolumn{4}{l}{luar,} & \multicolumn{3}{l}{yo} & \multicolumn{4}{l}{suda,} & \multicolumn{2}{l}{biar} & \multicolumn{5}{l}{mama} & \multicolumn{3}{l}{mati} & \multicolumn{2}{l}{ko} & \multicolumn{2}{l}{hidup}\\
& \multicolumn{2}{l}{enemy} & \multicolumn{2}{l}{at} & \multicolumn{4}{l}{outside} & \multicolumn{3}{l}{yes} & \multicolumn{4}{l}{already} & \multicolumn{2}{l}{let} & \multicolumn{5}{l}{mother} & \multicolumn{3}{l}{die} & \multicolumn{2}{l}{\textsc{2sg}} & \multicolumn{2}{l}{live}\\
& \multicolumn{3}{l}{suda,} & \multicolumn{4}{l}{de} & \multicolumn{2}{l}{bilang,} & \multicolumn{3}{l}{tida,} & \multicolumn{4}{l}{mama} & \multicolumn{4}{l}{pergi} & \multicolumn{6}{l}{makang,} & \multicolumn{2}{l}{sa} & bilang\\
& \multicolumn{3}{l}{already} & \multicolumn{4}{l}{\textsc{3sg}} & \multicolumn{2}{l}{say} & \multicolumn{3}{l}{\textsc{neg}} & \multicolumn{4}{l}{mother} & \multicolumn{4}{l}{go} & \multicolumn{6}{l}{eat} & \multicolumn{2}{l}{\textsc{1sg}} & say\\
& ko & \multicolumn{4}{l}{kluar} & \multicolumn{3}{l}{pergi} & \multicolumn{5}{l}{bungkus} & \multicolumn{6}{l}{nasi} & \multicolumn{4}{l}{untuk} & \multicolumn{6}{l}{saya}\\
& \textsc{2sg} & \multicolumn{4}{l}{go.out} & \multicolumn{3}{l}{go} & \multicolumn{5}{l}{pack} & \multicolumn{6}{l}{cooked.rice} & \multicolumn{4}{l}{for} & \multicolumn{6}{l}{\textsc{1sg}}\\
\lspbottomrule
\end{tabular}
\ea
\glt 
I said, ‘what[\textsc{tru}] [\textsc{up}] Pawlus, if you have enemies outside, alright, let me (‘mother’) die and you just live’, he said, ‘no, you (‘mother’) go and eat’, I said, ‘you go out, go, and (get) wrapped-up rice for me’
\z

\begin{tabular}{lllllllllllllllllllllllllll}
\lsptoprule
0003 & \multicolumn{4}{l}{baru} & \multicolumn{3}{l}{Iskia} & \multicolumn{3}{l}{dia} & \multicolumn{3}{l}{pegang} & \multicolumn{3}{l}{sa} & \multicolumn{4}{l}{punya} & \multicolumn{3}{l}{lutut} & yang & \multicolumn{2}{l}{tida}\\
& \multicolumn{4}{l}{and.then} & \multicolumn{3}{l}{Iskia} & \multicolumn{3}{l}{\textsc{3sg}} & \multicolumn{3}{l}{hold} & \multicolumn{3}{l}{\textsc{1sg}} & \multicolumn{4}{l}{\textsc{poss}} & \multicolumn{3}{l}{knee} & \textsc{rel} & \multicolumn{2}{l}{\textsc{neg}}\\
& \multicolumn{3}{l}{baik,} & \multicolumn{3}{l}{sa} & \multicolumn{3}{l}{pu} & \multicolumn{3}{l}{lutut} & \multicolumn{3}{l}{yang} & \multicolumn{4}{l}{suda} & \multicolumn{3}{l}{sakit} & \multicolumn{3}{l}{ini,} & bekas\\
& \multicolumn{3}{l}{be.good} & \multicolumn{3}{l}{\textsc{1sg}} & \multicolumn{3}{l}{\textsc{poss}} & \multicolumn{3}{l}{knee} & \multicolumn{3}{l}{\textsc{rel}} & \multicolumn{4}{l}{already} & \multicolumn{3}{l}{be.sick} & \multicolumn{3}{l}{\textsc{d.prox}} & trace\\
& \multicolumn{2}{l}{ini} & \multicolumn{6}{l}{baru} & \multicolumn{3}{l}{dia} & \multicolumn{3}{l}{gepe} & \multicolumn{4}{l}{begini} & \multicolumn{3}{l}{deng} & \multicolumn{4}{l}{kuku,} & de\\
& \multicolumn{2}{l}{\textsc{d.prox}} & \multicolumn{6}{l}{and.then} & \multicolumn{3}{l}{\textsc{3sg}} & \multicolumn{3}{l}{clamp} & \multicolumn{4}{l}{like.this} & \multicolumn{3}{l}{with} & \multicolumn{4}{l}{digit.nail} & \textsc{3sg}\\
& kasi, & \multicolumn{4}{l}{de} & \multicolumn{4}{l}{balut} & \multicolumn{6}{l}{putar} & \multicolumn{2}{l}{sa} & \multicolumn{4}{l}{punya} & \multicolumn{5}{l}{lutut}\\
& give & \multicolumn{4}{l}{\textsc{3sg}} & \multicolumn{4}{l}{bandage} & \multicolumn{6}{l}{turn.around} & \multicolumn{2}{l}{\textsc{1sg}} & \multicolumn{4}{l}{\textsc{poss}} & \multicolumn{5}{l}{knee}\\
\lspbottomrule
\end{tabular}
\ea
\glt 
and then Iskia held my knee that is not well, this knee which has already been sick, this scar (is still hurting), then he clamped (it) like this, he put, he bandaged my knee
\z

\begin{tabular}{lllllllllllllllllllllllllll}
\lsptoprule
0004 & \multicolumn{2}{l}{ibu} & \multicolumn{2}{l}{Marta} & \multicolumn{4}{l}{bertriak} & \multicolumn{3}{l}{sampe,} & \multicolumn{2}{l}{sa} & \multicolumn{3}{l}{bilang,} & \multicolumn{4}{l}{Tuhang} & \multicolumn{3}{l}{tolong} & \multicolumn{2}{l}{saja} & apa\\
& \multicolumn{2}{l}{woman} & \multicolumn{2}{l}{Marta} & \multicolumn{4}{l}{scream} & \multicolumn{3}{l}{reach} & \multicolumn{2}{l}{\textsc{1sg}} & \multicolumn{3}{l}{say} & \multicolumn{4}{l}{God} & \multicolumn{3}{l}{help} & \multicolumn{2}{l}{just} & what\\
& yang & \multicolumn{2}{l}{su} & \multicolumn{3}{l}{gigit} & \multicolumn{3}{l}{sa} & \multicolumn{3}{l}{pu} & \multicolumn{3}{l}{lutut?,} & \multicolumn{4}{l}{baru} & \multicolumn{2}{l}{dia} & \multicolumn{3}{l}{tertawa,} & \multicolumn{2}{l}{de}\\
& \textsc{rel} & \multicolumn{2}{l}{already} & \multicolumn{3}{l}{bite} & \multicolumn{3}{l}{\textsc{1sg}} & \multicolumn{3}{l}{\textsc{poss}} & \multicolumn{3}{l}{knee} & \multicolumn{4}{l}{and.then} & \multicolumn{2}{l}{\textsc{3sg}} & \multicolumn{3}{l}{laugh} & \multicolumn{2}{l}{\textsc{3sg}}\\
& \multicolumn{5}{l}{tertawa{\Tilde}tertawa,} & \multicolumn{2}{l}{sa} & \multicolumn{3}{l}{blang,} & \multicolumn{4}{l}{adu} & \multicolumn{3}{l}{Tuhang} & ko & \multicolumn{4}{l}{begini} & \multicolumn{4}{l}{ka?}\\
& \multicolumn{5}{l}{\textsc{rdp}{\Tilde}laugh} & \multicolumn{2}{l}{\textsc{1sg}} & \multicolumn{3}{l}{say} & \multicolumn{4}{l}{oh.no!} & \multicolumn{3}{l}{God} & \textsc{2sg} & \multicolumn{4}{l}{like.this} & \multicolumn{4}{l}{or}\\
\lspbottomrule
\end{tabular}
\ea
\glt 
I (‘Ms. Marta’) screamed strongly, I said, ‘God help me!, what (is it) that has bitten my knee?’ but then he laughed, he laughed intensely, I said, ‘oh God, why does this have to happen?’ (Lit. ‘you God are like this?’)
\z

\begin{tabular}{lllllllllllllllllllllll}
\lsptoprule
0005 & \multicolumn{3}{l}{baru} & \multicolumn{4}{l}{Pawlus} & \multicolumn{2}{l}{dia} & \multicolumn{3}{l}{mabuk} & \multicolumn{2}{l}{s} & \multicolumn{4}{l}{ini,} & \multicolumn{2}{l}{ibu} & \multicolumn{2}{l}{guru}\\
& \multicolumn{3}{l}{and.then} & \multicolumn{4}{l}{Pawlus} & \multicolumn{2}{l}{\textsc{3sg}} & \multicolumn{3}{l}{be.drunk} & \multicolumn{2}{l}{\textsc{spm}} & \multicolumn{4}{l}{\textsc{d.prox}} & \multicolumn{2}{l}{woman} & \multicolumn{2}{l}{teacher}\\
& \multicolumn{2}{l}{Maria} & \multicolumn{4}{l}{ini} & \multicolumn{4}{l}{kasiang,} & de & \multicolumn{4}{l}{suda} & \multicolumn{2}{l}{tidor,} & \multicolumn{4}{l}{kang} & dia\\
& \multicolumn{2}{l}{Maria} & \multicolumn{4}{l}{\textsc{d.prox}} & \multicolumn{4}{l}{love-\textsc{pat}} & \textsc{3sg} & \multicolumn{4}{l}{already} & \multicolumn{2}{l}{sleep} & \multicolumn{4}{l}{you.know} & \textsc{3sg}\\
& hosa & \multicolumn{3}{l}{to?,} & \multicolumn{4}{l}{tong} & \multicolumn{3}{l}{ja} & \multicolumn{2}{l}{jaga} & \multicolumn{3}{l}{dia} & \multicolumn{3}{l}{sampe} & \multicolumn{2}{l}{jam} & satu,\\
& pant & \multicolumn{3}{l}{right?} & \multicolumn{4}{l}{\textsc{1pl}} & \multicolumn{3}{l}{\textsc{tru}{}-guard} & \multicolumn{2}{l}{guard} & \multicolumn{3}{l}{\textsc{3sg}} & \multicolumn{3}{l}{until} & \multicolumn{2}{l}{hour} & one\\
& \multicolumn{3}{l}{baru} & \multicolumn{2}{l}{tong} & \multicolumn{17}{l}{tidor}\\
& \multicolumn{3}{l}{and.then} & \multicolumn{2}{l}{\textsc{1pl}} & \multicolumn{17}{l}{sleep}\\
\lspbottomrule
\end{tabular}
\ea
\glt 
and then Pawlus was drunk [\textsc{spm}], what’s-her-name, Ms. Teacher Maria here, poor thing, she was already sleeping, you know?, she has breathing difficulties, right?, we watched[\textsc{tru}] watched her until one o’clock, only then did we sleep
\z

\begin{tabular}{lllllllllllllllllllllllllllll}
\lsptoprule
0006 & \multicolumn{3}{l}{baru} & \multicolumn{3}{l}{Pawlus} & \multicolumn{3}{l}{de} & \multicolumn{3}{l}{sandar} & \multicolumn{3}{l}{di} & \multicolumn{4}{l}{de} & \multicolumn{2}{l}{pu} & \multicolumn{5}{l}{badang} & \multicolumn{2}{l}{begini,}\\
& \multicolumn{3}{l}{and.then} & \multicolumn{3}{l}{Pawlus} & \multicolumn{3}{l}{\textsc{3sg}} & \multicolumn{3}{l}{lean} & \multicolumn{3}{l}{at} & \multicolumn{4}{l}{\textsc{3sg}} & \multicolumn{2}{l}{\textsc{poss}} & \multicolumn{5}{l}{body} & \multicolumn{2}{l}{like.this}\\
& \multicolumn{3}{l}{baru} & de & \multicolumn{3}{l}{kas} & \multicolumn{3}{l}{pata} & \multicolumn{4}{l}{leher} & \multicolumn{3}{l}{ke} & \multicolumn{5}{l}{bawa} & \multicolumn{2}{l}{di} & \multicolumn{3}{l}{atas} & de\\
& \multicolumn{3}{l}{and.then} & \textsc{3sg} & \multicolumn{3}{l}{give} & \multicolumn{3}{l}{break} & \multicolumn{4}{l}{neck} & \multicolumn{3}{l}{to} & \multicolumn{5}{l}{bottom} & \multicolumn{2}{l}{at} & \multicolumn{3}{l}{top} & \textsc{3sg}\\
& pu & \multicolumn{4}{l}{bahu,} & \multicolumn{3}{l}{de} & \multicolumn{3}{l}{bilang,} & \multicolumn{5}{l}{adu} & \multicolumn{4}{l}{Tuhang} & \multicolumn{5}{l}{tolong,} & \multicolumn{3}{l}{ini}\\
& \textsc{poss} & \multicolumn{4}{l}{shoulder} & \multicolumn{3}{l}{\textsc{3sg}} & \multicolumn{3}{l}{say} & \multicolumn{5}{l}{oh.no!} & \multicolumn{4}{l}{God} & \multicolumn{5}{l}{help} & \multicolumn{3}{l}{\textsc{d.prox}}\\
& \multicolumn{2}{l}{siapa?,} & \multicolumn{2}{l}{Tuhang} & \multicolumn{5}{l}{tolong,} & \multicolumn{4}{l}{ini} & \multicolumn{5}{l}{siapa?,} & \multicolumn{5}{l}{ini} & \multicolumn{5}{l}{siapa?}\\
& \multicolumn{2}{l}{who} & \multicolumn{2}{l}{God} & \multicolumn{5}{l}{help} & \multicolumn{4}{l}{\textsc{d.prox}} & \multicolumn{5}{l}{who} & \multicolumn{5}{l}{\textsc{d.prox}} & \multicolumn{5}{l}{who}\\
\lspbottomrule
\end{tabular}
\ea
\glt 
but then Pawlus leaned on her body like this, and then he bent his neck down onto her shoulder, she said, ‘oh God, who is this?, God help me, who is this? who is this?’ (Lit. ‘caused his head to be broken’)
\z

\begin{tabular}{lllllllllllllllllllll}
\lsptoprule
0007 & \multicolumn{3}{l}{baru} & \multicolumn{3}{l}{de} & \multicolumn{2}{l}{su} & \multicolumn{2}{l}{tekang} & \multicolumn{2}{l}{dia} & \multicolumn{2}{l}{ke} & \multicolumn{2}{l}{bawa} & \multicolumn{2}{l}{sini,} & \multicolumn{2}{l}{hampir}\\
& \multicolumn{3}{l}{and.then} & \multicolumn{3}{l}{\textsc{3sg}} & \multicolumn{2}{l}{already} & \multicolumn{2}{l}{press} & \multicolumn{2}{l}{\textsc{3sg}} & \multicolumn{2}{l}{to} & \multicolumn{2}{l}{bottom} & \multicolumn{2}{l}{\textsc{l.prox}} & \multicolumn{2}{l}{almost}\\
& de & \multicolumn{3}{l}{mati,} & \multicolumn{3}{l}{mace} & \multicolumn{2}{l}{de} & \multicolumn{2}{l}{berdiri,} & \multicolumn{2}{l}{de} & \multicolumn{2}{l}{berdiri} & \multicolumn{2}{l}{sampe} & \multicolumn{2}{l}{di} & luar,\\
& \textsc{3sg} & \multicolumn{3}{l}{die} & \multicolumn{3}{l}{woman} & \multicolumn{2}{l}{\textsc{3sg}} & \multicolumn{2}{l}{stand} & \multicolumn{2}{l}{\textsc{3sg}} & \multicolumn{2}{l}{stand} & \multicolumn{2}{l}{reach} & \multicolumn{2}{l}{at} & outside\\
& \multicolumn{2}{l}{dia} & \multicolumn{3}{l}{lapor} & \multicolumn{15}{l}{ke}\\
& \multicolumn{2}{l}{\textsc{3sg}} & \multicolumn{3}{l}{report} & \multicolumn{15}{l}{to}\\
\lspbottomrule
\end{tabular}
\ea
\glt 
but he had already pressed her down, she almost died, the lady got up, she got up and went outside and reported (everything) to
\z

\begin{tabular}{llllllllll}
\lsptoprule
0008 & Efana: & \multicolumn{2}{l}{mabuk,} & tra, & macang & tida & punya & istri & saja,\\
&  & \multicolumn{2}{l}{be.drunk} & \textsc{neg} & variety & \textsc{neg} & \textsc{poss} & wife[SI] & just\\
& \multicolumn{2}{l}{mabuk} & \multicolumn{7}{l}{takaroang}\\
& \multicolumn{2}{l}{be.drunk} & \multicolumn{7}{l}{be.chaotic}\\
\lspbottomrule
\end{tabular}
\ea
\glt 
Efana: to be drunk!, doesn’t, like he doesn’t have a wife, (getting) drunk at random (like this)!
\z

\begin{tabular}{lllllllllllllllllllllllllllll}
\lsptoprule
0009 & \multicolumn{3}{l}{Marta:} & \multicolumn{2}{l}{de} & \multicolumn{4}{l}{lapor} & \multicolumn{2}{l}{ke} & \multicolumn{4}{l}{suster,} & \multicolumn{2}{l}{suster} & \multicolumn{3}{l}{kluar,} & \multicolumn{3}{l}{dia} & \multicolumn{3}{l}{lapor} & \multicolumn{2}{l}{sama}\\
& \multicolumn{3}{l}{} & \multicolumn{2}{l}{\textsc{3sg}} & \multicolumn{4}{l}{report} & \multicolumn{2}{l}{to} & \multicolumn{4}{l}{nurse} & \multicolumn{2}{l}{nurse} & \multicolumn{3}{l}{go.out} & \multicolumn{3}{l}{\textsc{3sg}} & \multicolumn{3}{l}{report} & \multicolumn{2}{l}{to}\\
& \multicolumn{2}{l}{polisi,} & \multicolumn{5}{l}{penjagaang} & \multicolumn{3}{l}{di} & \multicolumn{4}{l}{luar,} & \multicolumn{4}{l}{tinggal} & \multicolumn{3}{l}{tunggu} & \multicolumn{3}{l}{dorang} & \multicolumn{3}{l}{dua,} & dong\\
& \multicolumn{2}{l}{police} & \multicolumn{5}{l}{guard} & \multicolumn{3}{l}{at} & \multicolumn{4}{l}{outside} & \multicolumn{4}{l}{stay} & \multicolumn{3}{l}{wait} & \multicolumn{3}{l}{\textsc{3pl}} & \multicolumn{3}{l}{two} & \textsc{3pl}\\
& dua & \multicolumn{3}{l}{di} & \multicolumn{2}{l}{dalam,} & \multicolumn{6}{l}{sampe} & \multicolumn{3}{l}{dong} & dua & \multicolumn{3}{l}{pu} & \multicolumn{3}{l}{kluar} & \multicolumn{3}{l}{dang} & \multicolumn{3}{l}{polisi}\\
& two & \multicolumn{3}{l}{at} & \multicolumn{2}{l}{inside} & \multicolumn{6}{l}{until} & \multicolumn{3}{l}{\textsc{3pl}} & two & \multicolumn{3}{l}{\textsc{poss}} & \multicolumn{3}{l}{go.out} & \multicolumn{3}{l}{and} & \multicolumn{3}{l}{police}\\
& \multicolumn{3}{l}{pegang} & \multicolumn{2}{l}{dang} & \multicolumn{3}{l}{dong} & \multicolumn{5}{l}{borgol} & \multicolumn{3}{l}{dorang} & \multicolumn{12}{l}{dua}\\
& \multicolumn{3}{l}{hold} & \multicolumn{2}{l}{and} & \multicolumn{3}{l}{\textsc{3pl}} & \multicolumn{5}{l}{handcuff} & \multicolumn{3}{l}{\textsc{3pl}} & \multicolumn{12}{l}{two}\\
\lspbottomrule
\end{tabular}
\ea
\glt 
Marta: she reported (everything) to the nurse, the nurse went outside, she reported (everything) to the police, the security outside, it remained for the two of them to wait, the two of them (who were) inside, until the two of them came out and the police got (them) and they handcuffed the two of them\footnote{\\
\\
\\
\\
\\
\\
\\
\\
\\
\\
\\
\\
\\
\\
\\
\par One of the two detained persons is \textitbf{Pawlus}. It is unclear whether the second person is \textitbf{Iskia} or someone else.}
\z

\begin{tabular}{lllllllllllllll}
\lsptoprule
0010 & skarang & \multicolumn{2}{l}{ada} & \multicolumn{2}{l}{di} & sel, & \multicolumn{2}{l}{masuk} & \multicolumn{2}{l}{sel} & \multicolumn{2}{l}{ada} & tidor, & siram\\
& now & \multicolumn{2}{l}{exist} & \multicolumn{2}{l}{at} & cell & \multicolumn{2}{l}{enter} & \multicolumn{2}{l}{cell} & \multicolumn{2}{l}{exist} & sleep & pour.over\\
& \multicolumn{2}{l}{dengang} & \multicolumn{2}{l}{air} & \multicolumn{3}{l}{baru} & \multicolumn{2}{l}{dong} & \multicolumn{2}{l}{dua} & \multicolumn{3}{l}{tidor}\\
& \multicolumn{2}{l}{with} & \multicolumn{2}{l}{water} & \multicolumn{3}{l}{and.then} & \multicolumn{2}{l}{\textsc{3pl}} & \multicolumn{2}{l}{two} & \multicolumn{3}{l}{sleep}\\
\lspbottomrule
\end{tabular}
\ea
\glt 
now they were in a cell, (they) went into a cell to sleep, (the police) splashed (them) with water and the two of them slept
\z

\begin{tabular}{llll}
\lsptoprule
0011 & Efana: & ditahang, & dikurung\\
&  & \textsc{uv}{}-hold(.out/back) & \textsc{uv}{}-imprison\\
\lspbottomrule
\end{tabular}
\ea
\glt 
Efana: (they were) detained, imprisoned
\z

\begin{tabular}{llllllllll}
\lsptoprule
0012 & Marta: & \multicolumn{2}{l}{ditahangang,} & \multicolumn{2}{l}{polisi} & \multicolumn{2}{l}{kurung,} & mm-mm & tobat\\
&  & \multicolumn{2}{l}{\textsc{uv}{}-hold(.out/back)-\textsc{pat}} & \multicolumn{2}{l}{police} & \multicolumn{2}{l}{imprison} & mhm & repent\\
& to?, & karna & \multicolumn{2}{l}{orang{\Tilde}orang} & \multicolumn{2}{l}{kejahatang} & \multicolumn{3}{l}{nakal}\\
& right? & because & \multicolumn{2}{l}{\textsc{rdp}{\Tilde}person} & \multicolumn{2}{l}{evilness} & \multicolumn{3}{l}{be.mischievous}\\
\lspbottomrule
\end{tabular}
\ea
\glt
Marta: (they were) detained, the police imprisoned (them), mhm, to hell with them, right?, because (they are) bad, mischievous people
\end{styleFreeTranslEngxvpt}

\subsection{Narrative: A motorbike accident}

\begin{tabular}{ll}
\lsptoprule
File name: & 081015-005-NP\\
Text type: & Elicited text: Personal narrative\footnotemark{}\\
Interlocutors: & 2 older males, 3 older females\\
Length (min.): & 10:29\\
\lspbottomrule
\end{tabular}
\footnotetext{\\
\\
\\
\\
\\
\\
\\
\\
\\
\\
\\
\\
\\
\\
\\
The previous evening, the narrator had already told the same story, but due to logistical problems, the author was not able to record the text. The next morning, however, the narrator was willing to retell her story, with the same audience being present.}

\begin{tabular}{llllll}
\lsptoprule
0001 & Maria: & saya, & Martina, & Tinus, & kitong\\
&  & \textsc{1sg} & Martina & Tinus & \textsc{1pl}\\
\lspbottomrule
\end{tabular}
\ea
\glt 
Maria: I, Martina, Tinus, we
\z

\begin{tabular}{lllll}
\lsptoprule
0002 & Hurki: & kitong & tiga & orang\\
&  & \textsc{1pl} & three & person\\
\lspbottomrule
\end{tabular}
\ea
\glt 
Hurki: we (were) three people
\z

\begin{tabular}{llllllll}
\lsptoprule
0003 & Maria: & tiga & orang, & tra & ada, & tra & usa\\
&  & three & person & \textsc{neg} & exist & \textsc{neg} & need.to\\
\lspbottomrule
\end{tabular}
\ea
\glt 
Maria: three people, no, no need (to mention that)
\z

\begin{tabular}{lllll}
\lsptoprule
0004 & Marta: & kitong & tiga & orang\\
&  & \textsc{1pl} & three & person\\
\lspbottomrule
\end{tabular}
\ea
\glt 
Marta: we (were) three people
\z

\begin{tabular}{lllllllllllll}
\lsptoprule
0005 & \multicolumn{2}{l}{Maria:} & \multicolumn{3}{l}{nene,} & \multicolumn{2}{l}{kitorang} & tiga & orang & ((pause)), & kitong & lari\\
& \multicolumn{2}{l}{} & \multicolumn{3}{l}{grandmother} & \multicolumn{2}{l}{\textsc{1pl}} & three & person &  & \textsc{1pl} & run\\
& ke & \multicolumn{2}{l}{mari} & sampe & \multicolumn{2}{l}{di} & \multicolumn{6}{l}{jalangang}\\
& to & \multicolumn{2}{l}{hither} & reach & \multicolumn{2}{l}{at} & \multicolumn{6}{l}{route}\\
\lspbottomrule
\end{tabular}
\ea
\glt 
Maria: (we) grandmothers, we were three people ((pause)), we drove (along the beach back to Sarmi) here (until we) reached the road (Lit. ‘reached the route’)
\z

\begin{tabular}{llllll}
\lsptoprule
0006 & Hurki: & sampe & di & tenga & jalang\\
&  & reach & at & middle & walk\\
\lspbottomrule
\end{tabular}
\ea
\glt 
Hurki: (until we) reached the middle of the road
\z

\begin{tabular}{llllllllllll}
\lsptoprule
0007 & Maria: & \multicolumn{2}{l}{a,} & \multicolumn{2}{l}{hssst,} & tida & bole & begitu, & itu & suda & baik\\
&  & \multicolumn{2}{l}{ah!} & \multicolumn{2}{l}{shhh!} & \textsc{neg} & may & like.that & \textsc{d.dist} & already & good\\
& \multicolumn{2}{l}{maksut} & \multicolumn{2}{l}{jadi} & \multicolumn{7}{l}{((laughter))}\\
& \multicolumn{2}{l}{purpose} & \multicolumn{2}{l}{so} & \multicolumn{7}{l}{}\\
\lspbottomrule
\end{tabular}
\ea
\glt 
Maria: ah, shhh!, (you) shouldn’t (correct me), that’s already good (enough), since the meaning (is already clear) ((laughter))
\z

\begin{tabular}{llllll}
\lsptoprule
0008 & Hurki: & adu, & sampe & di & jalangang\\
&  & oh.no! & reach & at & journey\\
\lspbottomrule
\end{tabular}
\ea
\glt 
Hurki: oh boy! (until we) reached the road (Lit. ‘reached the route’)
\z

\begin{tabular}{lllllllllllllllllllllll}
\lsptoprule
0009 & Maria: & \multicolumn{4}{l}{ini} & \multicolumn{4}{l}{sampe} & \multicolumn{2}{l}{di} & \multicolumn{3}{l}{jalangang,} & \multicolumn{3}{l}{trus} & \multicolumn{3}{l}{tukang} & \multicolumn{2}{l}{ojek}\\
&  & \multicolumn{4}{l}{\textsc{d.prox}} & \multicolumn{4}{l}{reach} & \multicolumn{2}{l}{at} & \multicolumn{3}{l}{route} & \multicolumn{3}{l}{next} & \multicolumn{3}{l}{craftsman} & \multicolumn{2}{l}{motorbike.taxi}\\
& \multicolumn{2}{l}{ini} & \multicolumn{2}{l}{dia} & \multicolumn{3}{l}{tida} & \multicolumn{3}{l}{liat} & \multicolumn{3}{l}{kolam} & \multicolumn{2}{l}{ini,} & \multicolumn{4}{l}{langsung} & \multicolumn{2}{l}{dia} & tabrak\\
& \multicolumn{2}{l}{\textsc{d.prox}} & \multicolumn{2}{l}{\textsc{3sg}} & \multicolumn{3}{l}{\textsc{neg}} & \multicolumn{3}{l}{see} & \multicolumn{3}{l}{big.hole} & \multicolumn{2}{l}{\textsc{d.prox}} & \multicolumn{4}{l}{immediately} & \multicolumn{2}{l}{\textsc{3sg}} & hit.against\\
& \multicolumn{3}{l}{itu,} & \multicolumn{3}{l}{kolam} & \multicolumn{2}{l}{ke} & \multicolumn{4}{l}{sana,} & \multicolumn{4}{l}{langsung} & \multicolumn{2}{l}{mama} & \multicolumn{4}{l}{jatu}\\
& \multicolumn{3}{l}{\textsc{d.dist}} & \multicolumn{3}{l}{big.hole} & \multicolumn{2}{l}{to} & \multicolumn{4}{l}{\textsc{l.dist}} & \multicolumn{4}{l}{immediately} & \multicolumn{2}{l}{mother} & \multicolumn{4}{l}{fall}\\
\lspbottomrule
\end{tabular}
\ea
\glt 
Maria: what’s-its-name, until (we) reached the road, then this motorbike taxi driver, he didn’t see this big hole, immediately, he hit, what’s-its-name, the hole headlong, (and) immediately, I (‘mother’) fell off
\z

\begin{tabular}{lllllllllllllllllllll}
\lsptoprule
0010 & sa & \multicolumn{2}{l}{jatu} & \multicolumn{2}{l}{ke} & \multicolumn{5}{l}{blakang,} & Tinus & \multicolumn{3}{l}{ini} & \multicolumn{3}{l}{de} & lari & \multicolumn{2}{l}{trus,}\\
& \textsc{1sg} & \multicolumn{2}{l}{fall} & \multicolumn{2}{l}{to} & \multicolumn{5}{l}{backside} & Tinus & \multicolumn{3}{l}{\textsc{d.prox}} & \multicolumn{3}{l}{\textsc{3sg}} & run & \multicolumn{2}{l}{be.continuous}\\
& \multicolumn{2}{l}{saya} & \multicolumn{2}{l}{suda} & \multicolumn{3}{l}{jatu} & di & \multicolumn{4}{l}{blakang,} & sa & \multicolumn{3}{l}{jatu} & \multicolumn{3}{l}{begini,} & langsung\\
& \multicolumn{2}{l}{\textsc{1sg}} & \multicolumn{2}{l}{already} & \multicolumn{3}{l}{fall} & at & \multicolumn{4}{l}{backside} & \textsc{1sg} & \multicolumn{3}{l}{fall} & \multicolumn{3}{l}{like.this} & immediately\\
& sa & \multicolumn{5}{l}{taguling,} & \multicolumn{3}{l}{sa} & \multicolumn{4}{l}{guling{\Tilde}guling} & \multicolumn{2}{l}{di} & \multicolumn{5}{l}{situ}\\
& \textsc{1sg} & \multicolumn{5}{l}{be.rolled.over} & \multicolumn{3}{l}{\textsc{1sg}} & \multicolumn{4}{l}{\textsc{rdp}{\Tilde}roll.over} & \multicolumn{2}{l}{at} & \multicolumn{5}{l}{\textsc{l.med}}\\
\lspbottomrule
\end{tabular}
\ea
\glt 
I fell off backwards, Tinus here, he continued on, I had already fallen off the back (of the motorbike-taxi), as I fell, I rolled over immediately, I rolled over and over there
\z

\begin{tabular}{lllllllllllllllllllllllllllllllllll}
\lsptoprule
0011 & \multicolumn{2}{l}{Tinus,} & \multicolumn{4}{l}{dorang} & \multicolumn{4}{l}{dua} & \multicolumn{5}{l}{dengang} & \multicolumn{5}{l}{Martina} & \multicolumn{5}{l}{ini,} & \multicolumn{3}{l}{dong} & \multicolumn{2}{l}{dua} & \multicolumn{4}{l}{lari}\\
& \multicolumn{2}{l}{Tinus} & \multicolumn{4}{l}{\textsc{3pl}} & \multicolumn{4}{l}{two} & \multicolumn{5}{l}{with} & \multicolumn{5}{l}{Martina} & \multicolumn{5}{l}{\textsc{d.prox}} & \multicolumn{3}{l}{\textsc{3pl}} & \multicolumn{2}{l}{two} & \multicolumn{4}{l}{run}\\
& \multicolumn{5}{l}{trus,} & \multicolumn{6}{l}{dong} & \multicolumn{2}{l}{dua} & \multicolumn{4}{l}{lari} & \multicolumn{4}{l}{sampe} & di & \multicolumn{5}{l}{kali,} & \multicolumn{4}{l}{baru} & \multicolumn{3}{l}{Martina}\\
& \multicolumn{5}{l}{be.continuous} & \multicolumn{6}{l}{\textsc{3pl}} & \multicolumn{2}{l}{two} & \multicolumn{4}{l}{run} & \multicolumn{4}{l}{reach} & at & \multicolumn{5}{l}{river} & \multicolumn{4}{l}{and.then} & \multicolumn{3}{l}{Martina}\\
& \multicolumn{3}{l}{ini} & de & \multicolumn{5}{l}{kas} & \multicolumn{4}{l}{taw} & \multicolumn{5}{l}{sama} & \multicolumn{5}{l}{tukang} & \multicolumn{8}{l}{ojek} & \multicolumn{3}{l}{ini,}\\
& \multicolumn{3}{l}{\textsc{d.prox}} & \textsc{3sg} & \multicolumn{5}{l}{give} & \multicolumn{4}{l}{know} & \multicolumn{5}{l}{to} & \multicolumn{5}{l}{craftsman} & \multicolumn{8}{l}{motorbike.taxi} & \multicolumn{3}{l}{\textsc{d.prox}}\\
& de & \multicolumn{3}{l}{bilang,} & \multicolumn{4}{l}{a,} & \multicolumn{8}{l}{tukang} & \multicolumn{8}{l}{ojek,} & \multicolumn{4}{l}{itu} & \multicolumn{5}{l}{kitong} & pu\\
& \textsc{3sg} & \multicolumn{3}{l}{say} & \multicolumn{4}{l}{ah!} & \multicolumn{8}{l}{craftsman} & \multicolumn{8}{l}{motorbike.taxi} & \multicolumn{4}{l}{\textsc{d.dist}} & \multicolumn{5}{l}{\textsc{1pl}} & \textsc{poss}\\
& \multicolumn{3}{l}{kawang} & \multicolumn{4}{l}{suda} & \multicolumn{5}{l}{jatu,} & \multicolumn{2}{l}{yang} & \multicolumn{5}{l}{tadi} & \multicolumn{3}{l}{kitong} & \multicolumn{4}{l}{lari} & \multicolumn{3}{l}{ke} & \multicolumn{3}{l}{mari} & \multicolumn{2}{l}{tu}\\
& \multicolumn{3}{l}{friend} & \multicolumn{4}{l}{already} & \multicolumn{5}{l}{fall} & \multicolumn{2}{l}{\textsc{rel}} & \multicolumn{5}{l}{earlier} & \multicolumn{3}{l}{\textsc{1pl}} & \multicolumn{4}{l}{run} & \multicolumn{3}{l}{to} & \multicolumn{3}{l}{hither} & \multicolumn{2}{l}{\textsc{d.dist}}\\
\lspbottomrule
\end{tabular}
\ea
\glt 
Tinus, he and Martina here, the two of them drove continuously, the two of them drove on all the way to the river, but then Martina here, she let this motorbike taxi driver know, she said, ‘ah, motorbike taxi driver, what’s-her-name, our friend already fell off, with whom we were driving here earlier’
\z

\begin{tabular}{lllllllllllll}
\lsptoprule
0012 & Nofita: & [Is] & \multicolumn{2}{l}{ko} & \multicolumn{2}{l}{liat{\Tilde}liat} & \multicolumn{2}{l}{ke} & sini, & baru & ko & ceritra,\\
&  &  & \multicolumn{2}{l}{\textsc{2sg}} & \multicolumn{2}{l}{\textsc{rdp}{\Tilde}see} & \multicolumn{2}{l}{to} & \textsc{l.prox} & and.then & \textsc{2sg} & tell\\
& ceritra, & \multicolumn{2}{l}{ko} & \multicolumn{2}{l}{ceritra} & \multicolumn{2}{l}{suda} & \multicolumn{5}{l}{[\textsc{up}]}\\
& tell & \multicolumn{2}{l}{\textsc{2sg}} & \multicolumn{2}{l}{tell} & \multicolumn{2}{l}{already} & \multicolumn{5}{l}{}\\
\lspbottomrule
\end{tabular}
\ea
\glt 
Nofita: [Is] you (have to) look over here, and then you tell the story, tell the story!, just tell the story! [\textsc{up}]
\z

\begin{tabular}{lllllllll}
\lsptoprule
0013 & Maria: & yo, & biar & de & juga & liat & sa & ((laughter))\\
&  & yes & let & \textsc{3sg} & also & see & \textsc{1sg} & \\
\lspbottomrule
\end{tabular}
\ea
\glt 
Maria: yes, (but) let her also see me\footnote{\\
\\
\\
\\
\\
\\
\\
\\
\\
\\
\\
\\
\\
\\
\\
\par The personal pronoun \textitbf{de} ‘\textsc{3sg}’ refers to the recording author.}
\z

\begin{tabular}{llllllllllllll}
\lsptoprule
0014 & \multicolumn{2}{l}{skarang} & \multicolumn{3}{l}{tukang} & \multicolumn{3}{l}{ojek} & \multicolumn{2}{l}{ini} & \multicolumn{2}{l}{de} & pulang\\
& \multicolumn{2}{l}{now} & \multicolumn{3}{l}{craftsman} & \multicolumn{3}{l}{motorbike.taxi} & \multicolumn{2}{l}{\textsc{d.prox}} & \multicolumn{2}{l}{\textsc{3sg}} & go.home\\
& lagi & \multicolumn{2}{l}{sampe} & di & \multicolumn{2}{l}{tempat} & yang & \multicolumn{2}{l}{dia} & \multicolumn{2}{l}{buang} & \multicolumn{2}{l}{saya}\\
& again & \multicolumn{2}{l}{reach} & at & \multicolumn{2}{l}{place} & \textsc{rel} & \multicolumn{2}{l}{\textsc{3sg}} & \multicolumn{2}{l}{discard} & \multicolumn{2}{l}{\textsc{1sg}}\\
\lspbottomrule
\end{tabular}
\ea
\glt 
now this motorbike taxi driver, he returned again all the way to the place where he’d thrown me off
\z

\begin{tabular}{lllllllll}
\lsptoprule
0015 & Iskia: & minta & maaf & e?, & tolong & ceritra & tu & plang{\Tilde}plang\\
&  & request & pardon & eh? & help & tell & \textsc{d.dist} & \textsc{rdp}{\Tilde}be.slow\\
\lspbottomrule
\end{tabular}
\ea
\glt 
Iskia: excuse me, eh?, please talk slowly
\z

\begin{tabular}{lllllllllllllllllll}
\lsptoprule
0016 & \multicolumn{2}{l}{Maria:} & \multicolumn{2}{l}{de} & \multicolumn{2}{l}{buang} & \multicolumn{2}{l}{saya,} & \multicolumn{2}{l}{trus} & \multicolumn{2}{l}{dorang} & \multicolumn{2}{l}{dua} & \multicolumn{2}{l}{turung} & \multicolumn{2}{l}{dari}\\
& \multicolumn{2}{l}{} & \multicolumn{2}{l}{\textsc{3sg}} & \multicolumn{2}{l}{discard} & \multicolumn{2}{l}{\textsc{1sg}} & \multicolumn{2}{l}{next} & \multicolumn{2}{l}{\textsc{3pl}} & \multicolumn{2}{l}{two} & \multicolumn{2}{l}{descend} & \multicolumn{2}{l}{from}\\
& \multicolumn{3}{l}{motor,} & \multicolumn{2}{l}{dorang} & \multicolumn{2}{l}{dua} & \multicolumn{2}{l}{liat} & \multicolumn{2}{l}{sa} & \multicolumn{2}{l}{begini,} & \multicolumn{2}{l}{sa} & \multicolumn{2}{l}{su} & plaka\\
& \multicolumn{3}{l}{motorbike} & \multicolumn{2}{l}{\textsc{3pl}} & \multicolumn{2}{l}{two} & \multicolumn{2}{l}{see} & \multicolumn{2}{l}{\textsc{1sg}} & \multicolumn{2}{l}{like.this} & \multicolumn{2}{l}{\textsc{1sg}} & \multicolumn{2}{l}{already} & fall.over\\
& ke & \multicolumn{17}{l}{bawa}\\
& to & \multicolumn{17}{l}{bottom}\\
\lspbottomrule
\end{tabular}
\ea
\glt 
Maria: he’d thrown me off, then the two of them got off the motorbike, the two of them saw me like this, I had already fallen over to the ground
\z

\begin{tabular}{llllllllllllllll}
\lsptoprule
0017 & \multicolumn{2}{l}{dong} & dua & \multicolumn{3}{l}{bilang,} & \multicolumn{2}{l}{adu} & \multicolumn{2}{l}{kasiang,} & ko & jatu & ka?, & yo, & dorang\\
& \multicolumn{2}{l}{\textsc{3pl}} & two & \multicolumn{3}{l}{say} & \multicolumn{2}{l}{oh.no!} & \multicolumn{2}{l}{love-\textsc{pat}} & \textsc{2sg} & fall & or & yes & \textsc{3pl}\\
& dua & \multicolumn{3}{l}{angkat} & saya, & \multicolumn{2}{l}{trus} & \multicolumn{2}{l}{sa} & tida & \multicolumn{5}{l}{swara}\\
& two & \multicolumn{3}{l}{lift} & \textsc{1sg} & \multicolumn{2}{l}{next} & \multicolumn{2}{l}{\textsc{1sg}} & \textsc{neg} & \multicolumn{5}{l}{voice}\\
\lspbottomrule
\end{tabular}
\ea
\glt 
the two of them said, ‘oh no, poor thing!, did you fall?’ ‘yes’, the two of them lifted me, and I couldn’t speak (Lit. ‘didn’t (have) a voice’)
\z

\begin{tabular}{llllllllllllllll}
\lsptoprule
0018 & \multicolumn{2}{l}{dorang} & dua & \multicolumn{5}{l}{goyang{\Tilde}goyang} & \multicolumn{2}{l}{saya,} & \multicolumn{2}{l}{dong} & dua & \multicolumn{2}{l}{goyang{\Tilde}goyang}\\
& \multicolumn{2}{l}{\textsc{3pl}} & two & \multicolumn{5}{l}{\textsc{rdp}{\Tilde}shake} & \multicolumn{2}{l}{\textsc{1sg}} & \multicolumn{2}{l}{\textsc{3pl}} & two & \multicolumn{2}{l}{\textsc{rdp}{\Tilde}shake}\\
& saya, & \multicolumn{2}{l}{trus} & \multicolumn{2}{l}{sa} & \multicolumn{2}{l}{angkat} & \multicolumn{2}{l}{muka,} & \multicolumn{2}{l}{trus} & \multicolumn{2}{l}{Martina} & de & tanya\\
& \textsc{1sg} & \multicolumn{2}{l}{next} & \multicolumn{2}{l}{\textsc{1sg}} & \multicolumn{2}{l}{lift} & \multicolumn{2}{l}{front} & \multicolumn{2}{l}{next} & \multicolumn{2}{l}{Martina} & \textsc{3sg} & ask\\
& saya, & \multicolumn{3}{l}{mama} & \multicolumn{2}{l}{ko} & rasa & \multicolumn{8}{l}{bagemana?}\\
& \textsc{1sg} & \multicolumn{3}{l}{mother} & \multicolumn{2}{l}{\textsc{2sg}} & feel & \multicolumn{8}{l}{how}\\
\lspbottomrule
\end{tabular}
\ea
\glt 
the two of them shook me repeatedly, the two of them shook me repeatedly, then I lifted (my) face, then Martina asked me, ‘mother, how do you feel?’
\z

\begin{tabular}{llllllllllllllllllllllllllllllllll}
\lsptoprule
0019 & \multicolumn{2}{l}{sa} & \multicolumn{5}{l}{bilang} & \multicolumn{6}{l}{begini,} & \multicolumn{4}{l}{sa} & \multicolumn{4}{l}{pusing,} & \multicolumn{3}{l}{mata} & \multicolumn{3}{l}{saya} & \multicolumn{4}{l}{ini} & \multicolumn{2}{l}{glap,}\\
& \multicolumn{2}{l}{\textsc{1sg}} & \multicolumn{5}{l}{say} & \multicolumn{6}{l}{like.this} & \multicolumn{4}{l}{\textsc{1sg}} & \multicolumn{4}{l}{be.dizzy} & \multicolumn{3}{l}{eye} & \multicolumn{3}{l}{\textsc{1sg}} & \multicolumn{4}{l}{\textsc{d.prox}} & \multicolumn{2}{l}{be.dark}\\
& \multicolumn{3}{l}{trus} & \multicolumn{3}{l}{Tinus} & \multicolumn{5}{l}{ini} & \multicolumn{4}{l}{de} & \multicolumn{4}{l}{bilang} & \multicolumn{4}{l}{begini} & \multicolumn{4}{l}{sama} & \multicolumn{3}{l}{saya,} & \multicolumn{3}{l}{sa}\\
& \multicolumn{3}{l}{next} & \multicolumn{3}{l}{Tinus} & \multicolumn{5}{l}{\textsc{d.prox}} & \multicolumn{4}{l}{\textsc{3sg}} & \multicolumn{4}{l}{say} & \multicolumn{4}{l}{like.this} & \multicolumn{4}{l}{to} & \multicolumn{3}{l}{\textsc{1sg}} & \multicolumn{3}{l}{\textsc{1sg}}\\
& \multicolumn{4}{l}{bisa} & \multicolumn{4}{l}{bawa} & \multicolumn{2}{l}{ko} & \multicolumn{4}{l}{ke} & \multicolumn{4}{l}{Webro} & \multicolumn{3}{l}{ka?,} & trus & \multicolumn{3}{l}{sa} & \multicolumn{4}{l}{bilang} & \multicolumn{4}{l}{begini,}\\
& \multicolumn{4}{l}{be.able} & \multicolumn{4}{l}{bring} & \multicolumn{2}{l}{\textsc{2sg}} & \multicolumn{4}{l}{to} & \multicolumn{4}{l}{Webro} & \multicolumn{3}{l}{or} & next & \multicolumn{3}{l}{\textsc{1sg}} & \multicolumn{4}{l}{say} & \multicolumn{4}{l}{like.this}\\
& yo, & \multicolumn{4}{l}{sa} & \multicolumn{4}{l}{jatu,} & \multicolumn{3}{l}{sa} & \multicolumn{4}{l}{rasa} & \multicolumn{4}{l}{kepala} & \multicolumn{2}{l}{pusing,} & \multicolumn{4}{l}{bawa} & \multicolumn{2}{l}{saya} & \multicolumn{4}{l}{ke} & Webro\\
& yes & \multicolumn{4}{l}{\textsc{1sg}} & \multicolumn{4}{l}{fall} & \multicolumn{3}{l}{\textsc{1sg}} & \multicolumn{4}{l}{feel} & \multicolumn{4}{l}{head} & \multicolumn{2}{l}{be.dizzy} & \multicolumn{4}{l}{bring} & \multicolumn{2}{l}{\textsc{1sg}} & \multicolumn{4}{l}{to} & Webro\\
\lspbottomrule
\end{tabular}
\ea
\glt 
I said like this, ‘I’m dizzy, my eyes here are dark’, then Tinus here, he said to me like this, ‘can I bring you to Webro?’, then I said like this, ‘yes, I fell, my head feels dizzy, bring me to Webro’
\z

\begin{tabular}{llllllllllllll}
\lsptoprule
0020 & trus & \multicolumn{2}{l}{kitorang} & \multicolumn{2}{l}{tiga,} & \multicolumn{2}{l}{kitorang} & tiga & naik & di & motor, & sa & di\\
& next & \multicolumn{2}{l}{\textsc{1pl}} & \multicolumn{2}{l}{three} & \multicolumn{2}{l}{\textsc{1pl}} & three & ascend & at & motorbike & \textsc{1sg} & at\\
& \multicolumn{2}{l}{blakang,} & \multicolumn{2}{l}{Martina} & \multicolumn{2}{l}{di} & \multicolumn{7}{l}{tenga}\\
& \multicolumn{2}{l}{backside} & \multicolumn{2}{l}{Martina} & \multicolumn{2}{l}{at} & \multicolumn{7}{l}{middle}\\
\lspbottomrule
\end{tabular}
\ea
\glt 
then, we three, we three got onto the motorbike, I (was) in the back (and) Martina was in the middle
\z

\begin{tabular}{lllllllllllllll}
\lsptoprule
0021 & trus & \multicolumn{3}{l}{tukang} & \multicolumn{2}{l}{ojek} & \multicolumn{2}{l}{ini} & de & \multicolumn{2}{l}{bawa,} & \multicolumn{2}{l}{de} & bawa\\
& next & \multicolumn{3}{l}{craftsman} & \multicolumn{2}{l}{motorbike.taxi} & \multicolumn{2}{l}{\textsc{d.prox}} & \textsc{3sg} & \multicolumn{2}{l}{bring} & \multicolumn{2}{l}{\textsc{3sg}} & bring\\
& \multicolumn{2}{l}{kitorang} & \multicolumn{3}{l}{menyebrang,} & \multicolumn{2}{l}{menyebrang} & \multicolumn{3}{l}{ka} & \multicolumn{2}{l}{kali,} & \multicolumn{2}{l}{menyebra}\\
& \multicolumn{2}{l}{\textsc{1pl}} & \multicolumn{3}{l}{cross} & \multicolumn{2}{l}{cross} & \multicolumn{3}{l}{\textsc{tru}{}-river} & \multicolumn{2}{l}{river} & \multicolumn{2}{l}{\textsc{tru}{}-cross}\\
& \multicolumn{3}{l}{menyebrang} & \multicolumn{11}{l}{kali}\\
& \multicolumn{3}{l}{cross} & \multicolumn{11}{l}{river}\\
\lspbottomrule
\end{tabular}
\ea
\glt 
then this motorbike taxi driver, he took, he took us (and we) crossed, crossed the river[\textsc{tru}] river, (we) crossed[\textsc{tru}] crossed the river
\z

\begin{tabular}{llllllllllllllllll}
\lsptoprule
0022 & \multicolumn{2}{l}{sampe} & di & \multicolumn{2}{l}{Webro} & sa & \multicolumn{2}{l}{pu} & \multicolumn{2}{l}{bapa,} & sa & \multicolumn{2}{l}{pu} & \multicolumn{2}{l}{kaka} & \multicolumn{2}{l}{dorang}\\
& \multicolumn{2}{l}{reach} & at & \multicolumn{2}{l}{Webro} & \textsc{1sg} & \multicolumn{2}{l}{\textsc{poss}} & \multicolumn{2}{l}{father} & \textsc{1sg} & \multicolumn{2}{l}{\textsc{poss}} & \multicolumn{2}{l}{oSb} & \multicolumn{2}{l}{\textsc{3pl}}\\
& tanya & \multicolumn{3}{l}{saya,} & \multicolumn{3}{l}{sodara{\Tilde}sodara} & \multicolumn{2}{l}{dorang,} & \multicolumn{3}{l}{knapa?,} & \multicolumn{2}{l}{ko} & \multicolumn{2}{l}{sakit} & ka?\\
& ask & \multicolumn{3}{l}{\textsc{1sg}} & \multicolumn{3}{l}{\textsc{rdp}{\Tilde}sibling} & \multicolumn{2}{l}{\textsc{3pl}} & \multicolumn{3}{l}{why} & \multicolumn{2}{l}{\textsc{2sg}} & \multicolumn{2}{l}{be.sick} & or\\
\lspbottomrule
\end{tabular}
\ea
\glt 
having arrived in Webro, my father (and) my older siblings asked me, (my) relatives and friends (asked me), ‘what happened? are you hurt?’
\z

\begin{tabular}{lllllllllllllllllllllllll}
\lsptoprule
0023 & sa & \multicolumn{4}{l}{bilang} & \multicolumn{2}{l}{begini,} & \multicolumn{7}{l}{ojek} & \multicolumn{3}{l}{yang} & \multicolumn{3}{l}{buang} & \multicolumn{2}{l}{saya,} & \multicolumn{2}{l}{dong}\\
& \textsc{1sg} & \multicolumn{4}{l}{say} & \multicolumn{2}{l}{like.this} & \multicolumn{7}{l}{motorbike.taxi} & \multicolumn{3}{l}{\textsc{rel}} & \multicolumn{3}{l}{discard} & \multicolumn{2}{l}{\textsc{1sg}} & \multicolumn{2}{l}{\textsc{3pl}}\\
& \multicolumn{2}{l}{bilang,} & \multicolumn{6}{l}{ojek} & \multicolumn{3}{l}{mana?,} & a, & \multicolumn{4}{l}{sa} & \multicolumn{3}{l}{pu} & \multicolumn{4}{l}{motor} & ini,\\
& \multicolumn{2}{l}{say} & \multicolumn{6}{l}{motorbike.taxi} & \multicolumn{3}{l}{where} & ah! & \multicolumn{4}{l}{\textsc{1sg}} & \multicolumn{3}{l}{\textsc{poss}} & \multicolumn{4}{l}{motorbike} & \textsc{d.prox}\\
& sa & \multicolumn{3}{l}{pu} & \multicolumn{5}{l}{tukang} & \multicolumn{6}{l}{ojek} & \multicolumn{3}{l}{yang} & \multicolumn{3}{l}{buang} & \multicolumn{3}{l}{saya,}\\
& \textsc{1sg} & \multicolumn{3}{l}{\textsc{poss}} & \multicolumn{5}{l}{craftsman} & \multicolumn{6}{l}{motorbike.taxi} & \multicolumn{3}{l}{\textsc{rel}} & \multicolumn{3}{l}{discard} & \multicolumn{3}{l}{\textsc{1sg}}\\
& \multicolumn{3}{l}{kurang} & \multicolumn{3}{l}{ajar,} & \multicolumn{2}{l}{kitong} & \multicolumn{2}{l}{pukul} & \multicolumn{3}{l}{dia} & \multicolumn{11}{l}{suda}\\
& \multicolumn{3}{l}{lack} & \multicolumn{3}{l}{teach} & \multicolumn{2}{l}{\textsc{1pl}} & \multicolumn{2}{l}{hit} & \multicolumn{3}{l}{\textsc{3sg}} & \multicolumn{11}{l}{already}\\
\lspbottomrule
\end{tabular}
\ea
\glt 
I said like this, ‘(it was) the motorbike taxi driver who threw me off’, they said, ‘which motorbike taxi?’, ‘ah, (it’s) my motorbike here, (it’s) my motorbike taxi driver who threw me off’, ‘damn him! let us beat him up!’
\z

\begin{tabular}{llllllllllllllllll}
\lsptoprule
0024 & trus & \multicolumn{2}{l}{sa} & \multicolumn{4}{l}{bilang} & \multicolumn{2}{l}{begini,} & \multicolumn{2}{l}{jangang,} & \multicolumn{2}{l}{jangang} & \multicolumn{2}{l}{pukul} & \multicolumn{2}{l}{dia,}\\
& next & \multicolumn{2}{l}{\textsc{1sg}} & \multicolumn{4}{l}{say} & \multicolumn{2}{l}{like.this} & \multicolumn{2}{l}{\textsc{neg.imp}} & \multicolumn{2}{l}{\textsc{neg.imp}} & \multicolumn{2}{l}{hit} & \multicolumn{2}{l}{\textsc{3sg}}\\
& \multicolumn{2}{l}{kasiang,} & \multicolumn{3}{l}{itu} & \multicolumn{5}{l}{manusia,} & \multicolumn{2}{l}{kamorang} & \multicolumn{2}{l}{jangang} & \multicolumn{2}{l}{pukul} & dia,\\
& \multicolumn{2}{l}{pity} & \multicolumn{3}{l}{\textsc{d.dist}} & \multicolumn{5}{l}{human.being} & \multicolumn{2}{l}{\textsc{2pl}} & \multicolumn{2}{l}{\textsc{neg.imp}} & \multicolumn{2}{l}{hit} & \textsc{3sg}\\
& saya & \multicolumn{3}{l}{tida} & \multicolumn{2}{l}{mati,} & \multicolumn{2}{l}{saya} & \multicolumn{9}{l}{ada}\\
& \textsc{1sg} & \multicolumn{3}{l}{\textsc{neg}} & \multicolumn{2}{l}{die} & \multicolumn{2}{l}{\textsc{1sg}} & \multicolumn{9}{l}{exist}\\
\lspbottomrule
\end{tabular}
\ea
\glt 
then I said like this, ‘don’t, don’t beat him!, poor thing, he’s a human being, don’t beat him, I’m not dead, I’m alive’ (Lit. ‘I exist’)
\z

\begin{tabular}{llllllllllllllllllllllll}
\lsptoprule
0025 & trus & \multicolumn{2}{l}{sa} & \multicolumn{3}{l}{tidor,} & \multicolumn{2}{l}{tidor,} & \multicolumn{3}{l}{dorang} & \multicolumn{3}{l}{dua} & \multicolumn{3}{l}{pulang} & \multicolumn{2}{l}{ke} & \multicolumn{2}{l}{Waim,} & \multicolumn{2}{l}{tukang}\\
& next & \multicolumn{2}{l}{\textsc{1sg}} & \multicolumn{3}{l}{sleep} & \multicolumn{2}{l}{sleep} & \multicolumn{3}{l}{\textsc{3pl}} & \multicolumn{3}{l}{two} & \multicolumn{3}{l}{go.home} & \multicolumn{2}{l}{to} & \multicolumn{2}{l}{Waim} & \multicolumn{2}{l}{craftsman}\\
& \multicolumn{5}{l}{ojek} & \multicolumn{2}{l}{sama} & \multicolumn{2}{l}{Martina,} & \multicolumn{4}{l}{dong} & \multicolumn{3}{l}{dua} & \multicolumn{4}{l}{pulang} & \multicolumn{2}{l}{sendiri} & ke\\
& \multicolumn{5}{l}{motorbike.taxi} & \multicolumn{2}{l}{to} & \multicolumn{2}{l}{Martina} & \multicolumn{4}{l}{\textsc{3pl}} & \multicolumn{3}{l}{two} & \multicolumn{4}{l}{go.home} & \multicolumn{2}{l}{alone} & to\\
& \multicolumn{2}{l}{Waim,} & \multicolumn{2}{l}{sa} & \multicolumn{3}{l}{tinggal} & \multicolumn{3}{l}{karna} & \multicolumn{2}{l}{sa} & \multicolumn{3}{l}{rasa} & \multicolumn{3}{l}{masi} & \multicolumn{5}{l}{pusing}\\
& \multicolumn{2}{l}{Waim} & \multicolumn{2}{l}{\textsc{1sg}} & \multicolumn{3}{l}{stay} & \multicolumn{3}{l}{because} & \multicolumn{2}{l}{\textsc{1sg}} & \multicolumn{3}{l}{feel} & \multicolumn{3}{l}{still} & \multicolumn{5}{l}{be.dizzy}\\
\lspbottomrule
\end{tabular}
\ea
\glt 
then I slept, (I) slept, the two of them went home to Waim, the motorbike taxi driver and Martina, the two of them went home alone to Waim, I stayed (in Webro) because I still felt dizzy
\z

\begin{tabular}{lllll}
\lsptoprule
0026 & Nofita: & sap & badang & sakit\\
&  & \textsc{1sg}:\textsc{poss} & body & be.sick\\
\lspbottomrule
\end{tabular}
\ea
\glt 
Nofita: my body was hurting
\z

\begin{tabular}{llllll}
\lsptoprule
0027 & Maria: & badang & sakit, & saya & tidor\\
&  & body & be.sick & \textsc{1sg} & sleep\\
\lspbottomrule
\end{tabular}
\ea
\glt 
Maria: (my) body was hurting, I slept
\z

\begin{tabular}{lllll}
\lsptoprule
0028 & Nofita: & masak & air & panas\\
&  & cook & water & be.hot\\
\lspbottomrule
\end{tabular}
\ea
\glt 
Nofita: (they) boiled hot water
\z

\begin{tabular}{lllll}
\lsptoprule
0029 & Maria: & masak & air & panas\\
&  & cook & water & be.hot\\
\lspbottomrule
\end{tabular}
\ea
\glt 
Maria: (they) boiled hot water
\z

\begin{tabular}{lllllll}
\lsptoprule
0030 & Nofita: & Roni & yang & masak & air & panas\\
&  & Roni & \textsc{rel} & cook & water & be.hot\\
\lspbottomrule
\end{tabular}
\ea
\glt 
Nofita: (it was) Roni who boiled hot water
\z

\begin{tabular}{llllllllllllllllllllllll}
\lsptoprule
0031 & Maria: & \multicolumn{4}{l}{Roni,} & \multicolumn{4}{l}{ana} & \multicolumn{4}{l}{mantri} & \multicolumn{4}{l}{ini,} & de & \multicolumn{3}{l}{masak} & air & panas,\\
&  & \multicolumn{4}{l}{Roni} & \multicolumn{4}{l}{child} & \multicolumn{4}{l}{male.nurse} & \multicolumn{4}{l}{\textsc{d.prox}} & \textsc{3sg} & \multicolumn{3}{l}{cook} & water & be.hot\\
& \multicolumn{2}{l}{dorang} & \multicolumn{5}{l}{tolong,} & \multicolumn{4}{l}{dorang} & \multicolumn{2}{l}{bawa} & \multicolumn{2}{l}{air,} & \multicolumn{4}{l}{dorang} & \multicolumn{2}{l}{bawa} & \multicolumn{2}{l}{daung,}\\
& \multicolumn{2}{l}{\textsc{3pl}} & \multicolumn{5}{l}{help} & \multicolumn{4}{l}{\textsc{3pl}} & \multicolumn{2}{l}{bring} & \multicolumn{2}{l}{water} & \multicolumn{4}{l}{\textsc{3pl}} & \multicolumn{2}{l}{bring} & \multicolumn{2}{l}{leaf}\\
& \multicolumn{4}{l}{baru} & \multicolumn{4}{l}{dorang} & \multicolumn{4}{l}{urut} & \multicolumn{2}{l}{sa} & \multicolumn{2}{l}{deng} & \multicolumn{4}{l}{itu,} & dong & \multicolumn{2}{l}{bilang,}\\
& \multicolumn{4}{l}{and.then} & \multicolumn{4}{l}{\textsc{3pl}} & \multicolumn{4}{l}{massage} & \multicolumn{2}{l}{\textsc{1sg}} & \multicolumn{2}{l}{with} & \multicolumn{4}{l}{\textsc{d.dist}} & \textsc{3pl} & \multicolumn{2}{l}{say}\\
& \multicolumn{3}{l}{badang} & \multicolumn{3}{l}{mana} & \multicolumn{4}{l}{yang} & \multicolumn{13}{l}{sakit?}\\
& \multicolumn{3}{l}{body} & \multicolumn{3}{l}{where} & \multicolumn{4}{l}{\textsc{rel}} & \multicolumn{13}{l}{be.sick}\\
\lspbottomrule
\end{tabular}
\ea
\glt 
Maria: Roni, this young male nurse, he boiled hot water, they helped, they brought water, they brought leaves, then they massaged me with those (leaves), they said, ‘which (part of your) body is hurting?’
\z

\begin{tabular}{llllllllllllllllllllllll}
\lsptoprule
0032 & \multicolumn{2}{l}{adu,} & \multicolumn{2}{l}{sa} & \multicolumn{3}{l}{pu} & \multicolumn{4}{l}{bahu} & \multicolumn{3}{l}{sakit,} & \multicolumn{2}{l}{sa} & \multicolumn{2}{l}{pu} & \multicolumn{3}{l}{pinggang} & \multicolumn{2}{l}{sakit,}\\
& \multicolumn{2}{l}{oh.no!} & \multicolumn{2}{l}{\textsc{1sg}} & \multicolumn{3}{l}{\textsc{poss}} & \multicolumn{4}{l}{shoulder} & \multicolumn{3}{l}{be.sick} & \multicolumn{2}{l}{\textsc{1sg}} & \multicolumn{2}{l}{\textsc{poss}} & \multicolumn{3}{l}{loins} & \multicolumn{2}{l}{be.sick}\\
& sa & \multicolumn{2}{l}{pu} & \multicolumn{6}{l}{blakang} & \multicolumn{3}{l}{sakit,} & trus & \multicolumn{2}{l}{ana} & \multicolumn{4}{l}{mantri} & \multicolumn{3}{l}{ini,} & de\\
& \textsc{1sg} & \multicolumn{2}{l}{\textsc{poss}} & \multicolumn{6}{l}{backside} & \multicolumn{3}{l}{be.sick} & next & \multicolumn{2}{l}{child} & \multicolumn{4}{l}{male.nurse} & \multicolumn{3}{l}{\textsc{d.prox}} & \textsc{3sg}\\
& \multicolumn{5}{l}{urut{\Tilde}urut} & \multicolumn{3}{l}{saya,} & \multicolumn{2}{l}{de} & \multicolumn{5}{l}{pegang{\Tilde}pegang} & \multicolumn{2}{l}{di} & \multicolumn{3}{l}{bahu,} & \multicolumn{3}{l}{de}\\
& \multicolumn{5}{l}{\textsc{rdp}{\Tilde}massage} & \multicolumn{3}{l}{\textsc{1sg}} & \multicolumn{2}{l}{\textsc{3sg}} & \multicolumn{5}{l}{\textsc{rdp}{\Tilde}hold} & \multicolumn{2}{l}{at} & \multicolumn{3}{l}{shoulder} & \multicolumn{3}{l}{\textsc{3sg}}\\
& \multicolumn{6}{l}{pegang{\Tilde}pegang} & \multicolumn{17}{l}{blakang}\\
& \multicolumn{6}{l}{\textsc{rdp}{\Tilde}hold} & \multicolumn{17}{l}{backside}\\
\lspbottomrule
\end{tabular}
\ea
\glt 
‘ouch, my shoulder is hurting, my loins are hurting, my back is hurting’, then this young male nurse, he massaged me, he massaged (my) shoulder, he massaged (my) back
\z

\begin{tabular}{llllllllll}
\lsptoprule
0033 & suda, & saya & tidor & sampe & sore, & sa & pu & laki & datang,\\
& already & \textsc{1sg} & sleep & until & afternoon & \textsc{1sg} & \textsc{poss} & husband & come\\
& \multicolumn{9}{l}{Lukas}\\
& \multicolumn{9}{l}{Lukas}\\
\lspbottomrule
\end{tabular}
\ea
\glt 
eventually I slept until the afternoon, (then) my husband came, Lukas
\z

\begin{tabular}{lllll}
\lsptoprule
0034 & Nofita: & sa & pu & pacar\\
&  & \textsc{1sg} & \textsc{poss} & lover\\
\lspbottomrule
\end{tabular}
\ea
\glt 
Nofita: my lover
\z

\begin{tabular}{llllllllllllllllllllllllll}
\lsptoprule
0035 & \multicolumn{3}{l}{Maria:} & \multicolumn{2}{l}{a} & \multicolumn{4}{l}{ini} & \multicolumn{2}{l}{orang} & \multicolumn{3}{l}{Papua} & \multicolumn{3}{l}{bilang,} & \multicolumn{2}{l}{sa} & \multicolumn{4}{l}{pu} & \multicolumn{2}{l}{laki,}\\
& \multicolumn{3}{l}{} & \multicolumn{2}{l}{ah!} & \multicolumn{4}{l}{\textsc{d.prox}} & \multicolumn{2}{l}{person} & \multicolumn{3}{l}{Papua} & \multicolumn{3}{l}{say} & \multicolumn{2}{l}{\textsc{1sg}} & \multicolumn{4}{l}{\textsc{poss}} & \multicolumn{2}{l}{husband}\\
& sa & \multicolumn{3}{l}{pu} & \multicolumn{4}{l}{laki} & \multicolumn{4}{l}{datang,} & dia & \multicolumn{3}{l}{bilang,} & \multicolumn{4}{l}{kitong} & \multicolumn{2}{l}{dua} & \multicolumn{3}{l}{pulang,}\\
& \textsc{1sg} & \multicolumn{3}{l}{\textsc{poss}} & \multicolumn{4}{l}{husband} & \multicolumn{4}{l}{come} & \textsc{3sg} & \multicolumn{3}{l}{say} & \multicolumn{4}{l}{\textsc{1pl}} & \multicolumn{2}{l}{two} & \multicolumn{3}{l}{go.home}\\
& \multicolumn{2}{l}{sa} & \multicolumn{4}{l}{tanya,} & kitong & \multicolumn{3}{l}{dua} & \multicolumn{3}{l}{pulang} & \multicolumn{2}{l}{ke} & \multicolumn{3}{l}{mana?,} & \multicolumn{3}{l}{pulang} & \multicolumn{3}{l}{ke} & Waim\\
& \multicolumn{2}{l}{\textsc{1sg}} & \multicolumn{4}{l}{ask} & \textsc{1pl} & \multicolumn{3}{l}{two} & \multicolumn{3}{l}{go.home} & \multicolumn{2}{l}{to} & \multicolumn{3}{l}{where} & \multicolumn{3}{l}{go.home} & \multicolumn{3}{l}{to} & Waim\\
\lspbottomrule
\end{tabular}
\ea
\glt 
Maria: ah, this (is what) Papuans say ‘my husband’, my husband came, he said, ‘we two go home’, I asked, ‘where do we two go home to?’, ‘(we) go home to Waim’
\z

\begin{tabular}{lllllllllllllllllllllll}
\lsptoprule
0036 & \multicolumn{2}{l}{trus} & \multicolumn{2}{l}{kitong} & dua & \multicolumn{4}{l}{pulang,} & \multicolumn{3}{l}{sampe} & \multicolumn{2}{l}{di} & \multicolumn{3}{l}{jalangang} & \multicolumn{3}{l}{sa} & \multicolumn{2}{l}{istirahat,}\\
& \multicolumn{2}{l}{next} & \multicolumn{2}{l}{\textsc{1pl}} & two & \multicolumn{4}{l}{go.home} & \multicolumn{3}{l}{reach} & \multicolumn{2}{l}{at} & \multicolumn{3}{l}{route} & \multicolumn{3}{l}{\textsc{1sg}} & \multicolumn{2}{l}{rest}\\
& de & \multicolumn{3}{l}{bilang,} & \multicolumn{2}{l}{kitong} & \multicolumn{2}{l}{dua} & \multicolumn{3}{l}{jalang} & \multicolumn{4}{l}{suda,} & \multicolumn{4}{l}{mata-hari} & \multicolumn{3}{l}{suda}\\
& \textsc{3sg} & \multicolumn{3}{l}{say} & \multicolumn{2}{l}{\textsc{1pl}} & \multicolumn{2}{l}{two} & \multicolumn{3}{l}{street} & \multicolumn{4}{l}{already} & \multicolumn{4}{l}{sun} & \multicolumn{3}{l}{already}\\
& \multicolumn{3}{l}{masuk,} & \multicolumn{2}{l}{nanti} & \multicolumn{2}{l}{kitong} & \multicolumn{3}{l}{dua} & \multicolumn{3}{l}{dapat} & \multicolumn{3}{l}{glap,} & \multicolumn{2}{l}{jalang} & \multicolumn{3}{l}{cepat} & suda\\
& \multicolumn{3}{l}{enter} & \multicolumn{2}{l}{very.soon} & \multicolumn{2}{l}{\textsc{1pl}} & \multicolumn{3}{l}{two} & \multicolumn{3}{l}{get} & \multicolumn{3}{l}{be.dark} & \multicolumn{2}{l}{walk} & \multicolumn{3}{l}{be.fast} & already\\
\lspbottomrule
\end{tabular}
\ea
\glt 
and then we two went home, on the way I rested, he said, ‘let the two of us walk (on)! the sun is already going down, in a short while, we’ll be in the dark, walk fast already!’
\z

\begin{tabular}{llllllllllllllllllllllll}
\lsptoprule
0037 & sa & \multicolumn{4}{l}{dengang} & \multicolumn{2}{l}{pace} & \multicolumn{3}{l}{ini} & \multicolumn{2}{l}{kitong} & \multicolumn{3}{l}{dua} & \multicolumn{3}{l}{jalang,} & \multicolumn{2}{l}{ayo,} & \multicolumn{2}{l}{kitong} & dua\\
& \textsc{1sg} & \multicolumn{4}{l}{with} & \multicolumn{2}{l}{man} & \multicolumn{3}{l}{\textsc{d.prox}} & \multicolumn{2}{l}{\textsc{1pl}} & \multicolumn{3}{l}{two} & \multicolumn{3}{l}{walk} & \multicolumn{2}{l}{come.on!} & \multicolumn{2}{l}{\textsc{1pl}} & two\\
& \multicolumn{3}{l}{jalang} & \multicolumn{3}{l}{cepat,} & \multicolumn{3}{l}{kitong} & \multicolumn{2}{l}{dua} & \multicolumn{3}{l}{jalang} & \multicolumn{3}{l}{cepat,} & \multicolumn{2}{l}{kitong} & \multicolumn{2}{l}{dua} & \multicolumn{2}{l}{jalang,}\\
& \multicolumn{3}{l}{walk} & \multicolumn{3}{l}{be.fast} & \multicolumn{3}{l}{\textsc{1pl}} & \multicolumn{2}{l}{two} & \multicolumn{3}{l}{walk} & \multicolumn{3}{l}{be.fast} & \multicolumn{2}{l}{\textsc{1pl}} & \multicolumn{2}{l}{two} & \multicolumn{2}{l}{walk}\\
& \multicolumn{2}{l}{sampe} & \multicolumn{2}{l}{di} & \multicolumn{4}{l}{Waim,} & \multicolumn{4}{l}{dorang{\Tilde}dorang} & di & \multicolumn{3}{l}{situ,} & \multicolumn{4}{l}{masarakat} & \multicolumn{2}{l}{dong} & datang\\
& \multicolumn{2}{l}{reach} & \multicolumn{2}{l}{at} & \multicolumn{4}{l}{Waim} & \multicolumn{4}{l}{\textsc{rdp}{\Tilde}\textsc{3pl}} & at & \multicolumn{3}{l}{\textsc{l.med}} & \multicolumn{4}{l}{community} & \multicolumn{2}{l}{\textsc{3pl}} & come\\
\lspbottomrule
\end{tabular}
\ea
\glt 
I and the man here, we two walked, ‘come on! we two walk fast already!, we two walk fast already!’, the two of us walked, having arrived in Waim, all of them there, the whole community came
\z

\begin{tabular}{llllllllllllllllllllllllll}
\lsptoprule
0038 & dong & \multicolumn{5}{l}{bilang,} & \multicolumn{3}{l}{ibu} & \multicolumn{3}{l}{desa} & \multicolumn{3}{l}{ko} & \multicolumn{3}{l}{jatu} & \multicolumn{3}{l}{ka?} & \multicolumn{2}{l}{yo} & sa & jatu,\\
& \textsc{3pl} & \multicolumn{5}{l}{say} & \multicolumn{3}{l}{woman} & \multicolumn{3}{l}{village[SI]} & \multicolumn{3}{l}{\textsc{2sg}} & \multicolumn{3}{l}{fall} & \multicolumn{3}{l}{or} & \multicolumn{2}{l}{yes} & \textsc{1sg} & fall\\
& \multicolumn{3}{l}{knapa?} & \multicolumn{2}{l}{sa} & \multicolumn{3}{l}{jatu} & \multicolumn{2}{l}{dari} & \multicolumn{3}{l}{motor,} & \multicolumn{3}{l}{ko} & \multicolumn{3}{l}{pu} & \multicolumn{4}{l}{tulang} & \multicolumn{2}{l}{su}\\
& \multicolumn{3}{l}{why} & \multicolumn{2}{l}{\textsc{1sg}} & \multicolumn{3}{l}{fall} & \multicolumn{2}{l}{from} & \multicolumn{3}{l}{motorbike} & \multicolumn{3}{l}{\textsc{2sg}} & \multicolumn{3}{l}{\textsc{poss}} & \multicolumn{4}{l}{bone} & \multicolumn{2}{l}{already}\\
& \multicolumn{2}{l}{pata} & \multicolumn{2}{l}{ka?} & \multicolumn{3}{l}{tra} & \multicolumn{2}{l}{ada,} & \multicolumn{2}{l}{kosong,} & \multicolumn{3}{l}{tra} & \multicolumn{3}{l}{ada,} & \multicolumn{3}{l}{tulang} & \multicolumn{2}{l}{tra} & \multicolumn{3}{l}{pata}\\
& \multicolumn{2}{l}{break} & \multicolumn{2}{l}{or} & \multicolumn{3}{l}{\textsc{neg}} & \multicolumn{2}{l}{exist} & \multicolumn{2}{l}{be.empty} & \multicolumn{3}{l}{\textsc{neg}} & \multicolumn{3}{l}{exist} & \multicolumn{3}{l}{bone} & \multicolumn{2}{l}{\textsc{neg}} & \multicolumn{3}{l}{break}\\
\lspbottomrule
\end{tabular}
\ea
\glt 
they said, ‘Ms. Mayor, did you fall?’, ‘yes, I fell’, ‘what happened?’, ‘I fell off the motorbike’, ‘are your bones already broken?’, ‘no, nothing (like that), no, the bones aren’t broken’
\z

\begin{tabular}{lllllllllll}
\lsptoprule
0039 & \multicolumn{2}{l}{suda,} & saya & sampe, & sa & tidor, & tidor, & sa & bangung, & suda\\
& \multicolumn{2}{l}{already} & \textsc{1sg} & reach & \textsc{1sg} & sleep & sleep & \textsc{1sg} & wake.up & already\\
& dong & \multicolumn{9}{l}{bilang}\\
& \textsc{3pl} & \multicolumn{9}{l}{say}\\
\lspbottomrule
\end{tabular}
\ea
\glt 
eventually I arrived, I slept, (I) slept, I woke up, then they said
\z

\begin{tabular}{llll}
\lsptoprule
0040 & Nofita: & minum & obat\\
&  & drink & medicine\\
\lspbottomrule
\end{tabular}
\ea
\glt 
Nofita: take (your) medicine
\z

\begin{tabular}{lllllllllllllll}
\lsptoprule
0041 & Maria: & \multicolumn{2}{l}{ko} & \multicolumn{2}{l}{minum} & \multicolumn{2}{l}{obat,} & \multicolumn{2}{l}{suda} & sa & \multicolumn{2}{l}{ambil} & sa & pu\\
&  & \multicolumn{2}{l}{\textsc{2sg}} & \multicolumn{2}{l}{drink} & \multicolumn{2}{l}{medicine} & \multicolumn{2}{l}{already} & \textsc{1sg} & \multicolumn{2}{l}{fetch} & \textsc{1sg} & \textsc{poss}\\
& \multicolumn{2}{l}{obat,} & \multicolumn{2}{l}{tulang} & \multicolumn{2}{l}{sakit} & \multicolumn{2}{l}{punya,} & \multicolumn{3}{l}{bahu} & yang & \multicolumn{2}{l}{sakit}\\
& \multicolumn{2}{l}{medicine} & \multicolumn{2}{l}{bone} & \multicolumn{2}{l}{be.sick} & \multicolumn{2}{l}{\textsc{poss}} & \multicolumn{3}{l}{shoulder} & \textsc{rel} & \multicolumn{2}{l}{be.sick}\\
\lspbottomrule
\end{tabular}
\ea
\glt 
Maria: ‘take (your) medicine!’, then I took my medicine for (my) hurting bone, (it was my) shoulder which was hurting (Lit. ‘the hurting bone’s (medicine)’)
\z

\begin{tabular}{lllllllllllllllllll}
\lsptoprule
0042 & sa & \multicolumn{3}{l}{minum,} & \multicolumn{2}{l}{sa} & \multicolumn{3}{l}{minum,} & \multicolumn{2}{l}{sampe} & \multicolumn{2}{l}{tenga} & \multicolumn{2}{l}{malam} & \multicolumn{2}{l}{sa} & minum\\
& \textsc{1sg} & \multicolumn{3}{l}{drink} & \multicolumn{2}{l}{\textsc{1sg}} & \multicolumn{3}{l}{drink} & \multicolumn{2}{l}{until} & \multicolumn{2}{l}{middle} & \multicolumn{2}{l}{night} & \multicolumn{2}{l}{\textsc{1sg}} & drink\\
& \multicolumn{2}{l}{lagi,} & \multicolumn{3}{l}{pagi} & \multicolumn{2}{l}{sa} & \multicolumn{3}{l}{bangung,} & \multicolumn{2}{l}{sa} & \multicolumn{2}{l}{makang} & \multicolumn{2}{l}{sagu,} & \multicolumn{2}{l}{makang}\\
& \multicolumn{2}{l}{again} & \multicolumn{3}{l}{morning} & \multicolumn{2}{l}{\textsc{1sg}} & \multicolumn{3}{l}{wake.up} & \multicolumn{2}{l}{\textsc{1sg}} & \multicolumn{2}{l}{eat} & \multicolumn{2}{l}{sago} & \multicolumn{2}{l}{eat}\\
& \multicolumn{3}{l}{kasbi,} & sa & \multicolumn{4}{l}{minum} & \multicolumn{10}{l}{lagi}\\
& \multicolumn{3}{l}{cassava} & \textsc{1sg} & \multicolumn{4}{l}{drink} & \multicolumn{10}{l}{again}\\
\lspbottomrule
\end{tabular}
\ea
\glt 
I took (medicine), I took (medicine), when it was the middle of the night, I took (medicine) again, in the morning I woke up, I ate sago, (I) ate cassava, I took (medicine) again
\z

\begin{tabular}{lllllllllll}
\lsptoprule
0043 & trus & \multicolumn{2}{l}{sa} & tinggal & sampe & besok, & suda & sa & rasa & badang\\
& next & \multicolumn{2}{l}{\textsc{1sg}} & stay & until & tomorrow & already & \textsc{1sg} & feel & body\\
& \multicolumn{2}{l}{suda} & \multicolumn{8}{l}{baik}\\
& \multicolumn{2}{l}{already} & \multicolumn{8}{l}{be.good}\\
\lspbottomrule
\end{tabular}
\ea
\glt 
then I stayed until the next day, by then my body already felt good
\z

\begin{tabular}{lllllllllllllllllllllll}
\lsptoprule
0044 & \multicolumn{3}{l}{baru} & \multicolumn{2}{l}{sa} & \multicolumn{4}{l}{punya} & \multicolumn{2}{l}{ana} & \multicolumn{3}{l}{ini,} & \multicolumn{2}{l}{mantri,} & \multicolumn{2}{l}{de} & \multicolumn{2}{l}{pi} & \multicolumn{2}{l}{ambil}\\
& \multicolumn{3}{l}{and.then} & \multicolumn{2}{l}{\textsc{1sg}} & \multicolumn{4}{l}{\textsc{poss}} & \multicolumn{2}{l}{child} & \multicolumn{3}{l}{\textsc{d.prox}} & \multicolumn{2}{l}{male.nurse} & \multicolumn{2}{l}{\textsc{3sg}} & \multicolumn{2}{l}{go} & \multicolumn{2}{l}{fetch}\\
& saya, & \multicolumn{3}{l}{kitong} & \multicolumn{3}{l}{dua} & \multicolumn{3}{l}{lari} & \multicolumn{2}{l}{deng} & \multicolumn{3}{l}{motor,} & \multicolumn{2}{l}{dengang} & \multicolumn{2}{l}{Roni,} & \multicolumn{2}{l}{sa} & pu\\
& \textsc{1sg} & \multicolumn{3}{l}{\textsc{1pl}} & \multicolumn{3}{l}{two} & \multicolumn{3}{l}{run} & \multicolumn{2}{l}{with} & \multicolumn{3}{l}{motorbike} & \multicolumn{2}{l}{with} & \multicolumn{2}{l}{Roni} & \multicolumn{2}{l}{\textsc{1sg}} & \textsc{poss}\\
& \multicolumn{2}{l}{ana} & \multicolumn{4}{l}{mantri} & \multicolumn{2}{l}{di} & \multicolumn{5}{l}{Jayapura} & \multicolumn{9}{l}{ini}\\
& \multicolumn{2}{l}{child} & \multicolumn{4}{l}{male.nurse} & \multicolumn{2}{l}{at} & \multicolumn{5}{l}{Jayapura} & \multicolumn{9}{l}{\textsc{d.prox}}\\
\lspbottomrule
\end{tabular}
\ea
\glt 
and then, my child here, the male nurse, he came to pick me up, the two of us drove with (his) motorbike, with Roni, my young male nurse from Jayapura
\z

\begin{tabular}{lll}
\lsptoprule
0045 & MO: & malam\\
&  & night\\
\lspbottomrule
\end{tabular}
\ea
\glt 
[A guest arrives] MO: good evening
\z

\begin{tabular}{lllllllllllllllllllll}
\lsptoprule
0046 & Maria: & \multicolumn{4}{l}{kitorang} & \multicolumn{2}{l}{dua} & \multicolumn{2}{l}{datang} & \multicolumn{3}{l}{sampe} & di & \multicolumn{3}{l}{sini,} & \multicolumn{2}{l}{ibu} & \multicolumn{2}{l}{pendeta}\\
&  & \multicolumn{4}{l}{\textsc{1pl}} & \multicolumn{2}{l}{two} & \multicolumn{2}{l}{come} & \multicolumn{3}{l}{reach} & at & \multicolumn{3}{l}{\textsc{l.prox}} & \multicolumn{2}{l}{woman} & \multicolumn{2}{l}{pastor}\\
& \multicolumn{2}{l}{ini} & \multicolumn{2}{l}{dia} & \multicolumn{2}{l}{tanya,} & \multicolumn{2}{l}{ko} & \multicolumn{2}{l}{jatu} & ka? & \multicolumn{3}{l}{yo} & sa & \multicolumn{2}{l}{jatu} & \multicolumn{2}{l}{dari} & motor,\\
& \multicolumn{2}{l}{\textsc{d.prox}} & \multicolumn{2}{l}{\textsc{3sg}} & \multicolumn{2}{l}{ask} & \multicolumn{2}{l}{\textsc{2sg}} & \multicolumn{2}{l}{fall} & or & \multicolumn{3}{l}{yes} & \textsc{1sg} & \multicolumn{2}{l}{fall} & \multicolumn{2}{l}{from} & motorbike\\
& \multicolumn{3}{l}{kasiang} & \multicolumn{17}{l}{sayang}\\
& \multicolumn{3}{l}{pity} & \multicolumn{17}{l}{love}\\
\lspbottomrule
\end{tabular}
\ea
\glt 
the two of us came all the way here, Ms. Pastor here, she asked (me), ‘did you fall?’, ‘yes, I fell off the motorbike’, ‘poor thing, (my) dear’
\z

\begin{tabular}{llllllllllllllllllllll}
\lsptoprule
0047 & sa & \multicolumn{3}{l}{tinggal} & \multicolumn{3}{l}{di} & \multicolumn{3}{l}{sini,} & \multicolumn{2}{l}{sa} & \multicolumn{2}{l}{ke} & \multicolumn{2}{l}{ruma-sakit,} & \multicolumn{2}{l}{sa} & \multicolumn{2}{l}{ceritra} & sama\\
& \textsc{1sg} & \multicolumn{3}{l}{stay} & \multicolumn{3}{l}{at} & \multicolumn{3}{l}{\textsc{l.prox}} & \multicolumn{2}{l}{\textsc{1sg}} & \multicolumn{2}{l}{to} & \multicolumn{2}{l}{hospital} & \multicolumn{2}{l}{\textsc{1sg}} & \multicolumn{2}{l}{tell} & to\\
& \multicolumn{2}{l}{dokter,} & \multicolumn{4}{l}{dokter,} & \multicolumn{3}{l}{sa} & \multicolumn{2}{l}{jatu} & \multicolumn{2}{l}{dari} & \multicolumn{2}{l}{motor,} & \multicolumn{2}{l}{dokter} & \multicolumn{2}{l}{dorang} & \multicolumn{2}{l}{bilang}\\
& \multicolumn{2}{l}{doctor} & \multicolumn{4}{l}{doctor} & \multicolumn{3}{l}{\textsc{1sg}} & \multicolumn{2}{l}{fall} & \multicolumn{2}{l}{from} & \multicolumn{2}{l}{motorbike} & \multicolumn{2}{l}{doctor} & \multicolumn{2}{l}{\textsc{3pl}} & \multicolumn{2}{l}{say}\\
& \multicolumn{3}{l}{begini,} & \multicolumn{2}{l}{ko} & \multicolumn{3}{l}{jatu} & \multicolumn{13}{l}{bagemana?}\\
& \multicolumn{3}{l}{like.this} & \multicolumn{2}{l}{\textsc{2sg}} & \multicolumn{3}{l}{fall} & \multicolumn{13}{l}{how}\\
\lspbottomrule
\end{tabular}
\ea
\glt 
I stayed here, I went to the hospital, I talked to the doctor, ‘doctor, I fell off a motorbike’, the doctor and his companions said like this, ‘how did you fall off?’
\z

\begin{tabular}{llllllm{-9.4015896E-4in}llllllllllllllll}
\lsptoprule
0048 & sa & \multicolumn{5}{l}{bilang,} & \multicolumn{2}{l}{sa} & \multicolumn{2}{l}{jatu} & \multicolumn{3}{l}{balik} & \multicolumn{5}{l}{begini,} & \multicolumn{3}{l}{trus} & tulang\\
& \textsc{1sg} & \multicolumn{5}{l}{say} & \multicolumn{2}{l}{\textsc{1sg}} & \multicolumn{2}{l}{fall} & \multicolumn{3}{l}{turn.around} & \multicolumn{5}{l}{like.this} & \multicolumn{3}{l}{next} & bone\\
& \multicolumn{3}{l}{pata,} & \multicolumn{2}{l}{sa} & \multicolumn{4}{l}{bilang,} & \multicolumn{2}{l}{tulang} & \multicolumn{3}{l}{bahu} & \multicolumn{2}{l}{yang} & \multicolumn{4}{l}{pata,} & \multicolumn{2}{l}{tulang}\\
& \multicolumn{3}{l}{break} & \multicolumn{2}{l}{\textsc{1sg}} & \multicolumn{4}{l}{say} & \multicolumn{2}{l}{bone} & \multicolumn{3}{l}{shoulder} & \multicolumn{2}{l}{\textsc{rel}} & \multicolumn{4}{l}{break} & \multicolumn{2}{l}{bone}\\
& \multicolumn{2}{l}{rusuk,} & \multicolumn{2}{l}{o,} & \multicolumn{3}{l}{a} & \multicolumn{3}{l}{mama} & \multicolumn{2}{l}{itu} & \multicolumn{3}{l}{hanya} & \multicolumn{2}{l}{ko} & \multicolumn{2}{l}{jatu} & \multicolumn{3}{l}{kaget}\\
& \multicolumn{2}{l}{rib} & \multicolumn{2}{l}{oh!} & \multicolumn{3}{l}{ah!} & \multicolumn{3}{l}{mother} & \multicolumn{2}{l}{\textsc{d.dist}} & \multicolumn{3}{l}{only} & \multicolumn{2}{l}{\textsc{2sg}} & \multicolumn{2}{l}{fall} & \multicolumn{3}{l}{feel.startled(.by)}\\
\lspbottomrule
\end{tabular}
\ea
\glt 
I said, ‘I fell backwards like this, then the bone broke’, I said, ‘(it’s my) shoulder bone which is broken, (my) ribs’, ‘oh! ah, mother that is just because you’re under shock’
\z

\begin{tabular}{llllllllllllllllllllll}
\lsptoprule
0049 & sa & \multicolumn{3}{l}{bilang} & \multicolumn{2}{l}{begini,} & \multicolumn{3}{l}{adu} & \multicolumn{4}{l}{dokter,} & \multicolumn{3}{l}{ini} & \multicolumn{2}{l}{sa} & \multicolumn{2}{l}{jatu} & sengsara\\
& \textsc{1sg} & \multicolumn{3}{l}{say} & \multicolumn{2}{l}{like.this} & \multicolumn{3}{l}{oh.no!} & \multicolumn{4}{l}{doctor} & \multicolumn{3}{l}{\textsc{d.prox}} & \multicolumn{2}{l}{\textsc{1sg}} & \multicolumn{2}{l}{fall} & suffer\\
& \multicolumn{3}{l}{ini,} & \multicolumn{2}{l}{harus} & \multicolumn{3}{l}{tolong} & \multicolumn{3}{l}{saya,} & a & \multicolumn{3}{l}{mama,} & \multicolumn{2}{l}{sa} & \multicolumn{2}{l}{kasi} & \multicolumn{2}{l}{obat,}\\
& \multicolumn{3}{l}{\textsc{d.prox}} & \multicolumn{2}{l}{have.to} & \multicolumn{3}{l}{help} & \multicolumn{3}{l}{\textsc{1sg}} & ah! & \multicolumn{3}{l}{mother} & \multicolumn{2}{l}{\textsc{1sg}} & \multicolumn{2}{l}{give} & \multicolumn{2}{l}{medicine}\\
& \multicolumn{2}{l}{mama} & \multicolumn{3}{l}{minum,} & \multicolumn{2}{l}{sa} & \multicolumn{3}{l}{bilang,} & \multicolumn{4}{l}{dokter} & \multicolumn{7}{l}{trima-kasi}\\
& \multicolumn{2}{l}{mother} & \multicolumn{3}{l}{drink} & \multicolumn{2}{l}{\textsc{1sg}} & \multicolumn{3}{l}{say} & \multicolumn{4}{l}{doctor} & \multicolumn{7}{l}{thank.you}\\
\lspbottomrule
\end{tabular}
\ea
\glt 
I said like this, ‘oh no!, doctor, what’s-its-name, I fell really painfully, (you) have to help me’, ‘ah mother, I give (you) medicine (and) you (‘mother’) take (it)’, I said, ‘doctor, thank you’
\z

\begin{tabular}{llllllllllllllllllllllllll}
\lsptoprule
0050 & \multicolumn{2}{l}{sa} & \multicolumn{4}{l}{pulang} & \multicolumn{3}{l}{sampe} & \multicolumn{2}{l}{di} & \multicolumn{4}{l}{sini,} & sa & \multicolumn{3}{l}{bilang} & \multicolumn{3}{l}{ibu} & \multicolumn{3}{l}{pendeta,}\\
& \multicolumn{2}{l}{\textsc{1sg}} & \multicolumn{4}{l}{go.home} & \multicolumn{3}{l}{reach} & \multicolumn{2}{l}{at} & \multicolumn{4}{l}{\textsc{l.prox}} & \textsc{1sg} & \multicolumn{3}{l}{say} & \multicolumn{3}{l}{woman} & \multicolumn{3}{l}{pastor}\\
& \multicolumn{3}{l}{ibu} & \multicolumn{2}{l}{ko} & \multicolumn{3}{l}{kas} & sa & \multicolumn{4}{l}{air,} & \multicolumn{2}{l}{sa} & \multicolumn{3}{l}{minum} & \multicolumn{3}{l}{obat,} & \multicolumn{3}{l}{sa} & tinggal\\
& \multicolumn{3}{l}{woman} & \multicolumn{2}{l}{\textsc{2sg}} & \multicolumn{3}{l}{give} & \textsc{1sg} & \multicolumn{4}{l}{water} & \multicolumn{2}{l}{\textsc{1sg}} & \multicolumn{3}{l}{drink} & \multicolumn{3}{l}{medicine} & \multicolumn{3}{l}{\textsc{1sg}} & stay\\
& di & \multicolumn{3}{l}{sini} & \multicolumn{3}{l}{satu} & \multicolumn{3}{l}{minggu,} & \multicolumn{2}{l}{e,} & \multicolumn{2}{l}{dua} & \multicolumn{3}{l}{minggu,} & \multicolumn{3}{l}{baru} & \multicolumn{3}{l}{sa} & \multicolumn{2}{l}{pulang}\\
& at & \multicolumn{3}{l}{\textsc{l.prox}} & \multicolumn{3}{l}{one} & \multicolumn{3}{l}{week} & \multicolumn{2}{l}{uh} & \multicolumn{2}{l}{two} & \multicolumn{3}{l}{week} & \multicolumn{3}{l}{and.then} & \multicolumn{3}{l}{\textsc{1sg}} & \multicolumn{2}{l}{go.home}\\
\lspbottomrule
\end{tabular}
\ea
\glt 
I went home all the way to here, I told Ms. Pastor, ‘Madam, give me water (so that) I (can) take (my) medicine’, I stayed here for one week, uh, two weeks, only then did I return home
\z

\begin{tabular}{llllllllllllllllllllllllll}
\lsptoprule
0051 & sa & \multicolumn{5}{l}{pulang} & \multicolumn{2}{l}{ke} & \multicolumn{3}{l}{Waim} & \multicolumn{3}{l}{lagi,} & \multicolumn{3}{l}{baru} & \multicolumn{3}{l}{kitorang} & \multicolumn{4}{l}{tinggal,} & baru\\
& \textsc{1sg} & \multicolumn{5}{l}{go.home} & \multicolumn{2}{l}{to} & \multicolumn{3}{l}{Waim} & \multicolumn{3}{l}{again} & \multicolumn{3}{l}{and.then} & \multicolumn{3}{l}{\textsc{1pl}} & \multicolumn{4}{l}{stay} & and.then\\
& \multicolumn{2}{l}{sa} & \multicolumn{3}{l}{pu} & \multicolumn{5}{l}{masarakat} & \multicolumn{3}{l}{dong} & \multicolumn{3}{l}{tanya} & \multicolumn{3}{l}{saya,} & \multicolumn{2}{l}{ibu} & \multicolumn{2}{l}{ko} & \multicolumn{2}{l}{su}\\
& \multicolumn{2}{l}{\textsc{1sg}} & \multicolumn{3}{l}{\textsc{poss}} & \multicolumn{5}{l}{community} & \multicolumn{3}{l}{\textsc{3pl}} & \multicolumn{3}{l}{ask} & \multicolumn{3}{l}{\textsc{1sg}} & \multicolumn{2}{l}{woman} & \multicolumn{2}{l}{\textsc{2sg}} & \multicolumn{2}{l}{already}\\
& \multicolumn{4}{l}{sembu} & \multicolumn{3}{l}{ka?} & \multicolumn{2}{l}{sa} & \multicolumn{3}{l}{bilang,} & \multicolumn{3}{l}{sa} & \multicolumn{3}{l}{su} & \multicolumn{4}{l}{sembu,} & \multicolumn{3}{l}{trima-kasi,}\\
& \multicolumn{4}{l}{be.healed} & \multicolumn{3}{l}{or} & \multicolumn{2}{l}{\textsc{1sg}} & \multicolumn{3}{l}{say} & \multicolumn{3}{l}{\textsc{1sg}} & \multicolumn{3}{l}{already} & \multicolumn{4}{l}{be.healed} & \multicolumn{3}{l}{thank.you}\\
& \multicolumn{3}{l}{sampe} & \multicolumn{2}{l}{di} & \multicolumn{20}{l}{sini}\\
& \multicolumn{3}{l}{reach} & \multicolumn{2}{l}{at} & \multicolumn{20}{l}{\textsc{l.prox}}\\
\lspbottomrule
\end{tabular}
\ea
\glt
I went home to Waim again, and then we stayed (there), and then my community asked me, ‘Madam, have you recovered?’, I said, ‘I’ve recovered’, thank you!, this is all (Lit. ‘reach here’)
\end{styleFreeTranslEngxvpt}

\subsection{Narrative: Pig hunting with dogs}

\begin{tabular}{ll}
\lsptoprule
File name: & 080919-003-NP\\
Text type: & Elicited text: Personal narrative\footnotemark{}\\
Interlocutors: & 1 older male, 1 older female\\
Length (min.): & 4:20\\
\lspbottomrule
\end{tabular}
\footnotetext{\\
\\
\\
\\
\\
\\
\\
\\
\\
\\
\\
\\
\\
\\
\\
This narrative is one of the three personal narratives mentioned in §1.11.4.1, which the author recorded with the help of her host Sarlota Merne. Being aware of the target language variety, she was present during these elicitations and explained to the narrator that he should narrate his story in \textitbf{logat Papua} ‘Papuan speech variety’. Being one of the early recordings, the text includes quite a few instances of code-switches with Indonesian, which are marked with “[SI]”.}

\begin{tabular}{llllllllllllllllllllllll}
\lsptoprule
0001 & Iskia: & \multicolumn{2}{l}{jadi} & \multicolumn{3}{l}{satu} & \multicolumn{3}{l}{waktu} & \multicolumn{3}{l}{saya} & \multicolumn{2}{l}{ada} & di & \multicolumn{3}{l}{ruma,} & \multicolumn{2}{l}{malam} & \multicolumn{2}{l}{hari} & saya\\
&  & \multicolumn{2}{l}{so} & \multicolumn{3}{l}{one} & \multicolumn{3}{l}{time} & \multicolumn{3}{l}{\textsc{1sg}} & \multicolumn{2}{l}{exist} & at & \multicolumn{3}{l}{house} & \multicolumn{2}{l}{night} & \multicolumn{2}{l}{day} & \textsc{1sg}\\
& \multicolumn{2}{l}{suda} & \multicolumn{2}{l}{pikir,} & \multicolumn{3}{l}{sa} & \multicolumn{4}{l}{bilang} & \multicolumn{2}{l}{sama} & \multicolumn{3}{l}{ibu,} & \multicolumn{4}{l}{besok} & sa & \multicolumn{2}{l}{bawa}\\
& \multicolumn{2}{l}{already} & \multicolumn{2}{l}{think} & \multicolumn{3}{l}{\textsc{1sg}} & \multicolumn{4}{l}{say} & \multicolumn{2}{l}{to} & \multicolumn{3}{l}{woman} & \multicolumn{4}{l}{tomorrow} & \textsc{1sg} & \multicolumn{2}{l}{bring}\\
& \multicolumn{2}{l}{anjing} & \multicolumn{3}{l}{cari} & \multicolumn{3}{l}{babi,} & \multicolumn{2}{l}{sa} & \multicolumn{7}{l}{snang} & \multicolumn{2}{l}{makang} & \multicolumn{4}{l}{babi}\\
& \multicolumn{2}{l}{dog} & \multicolumn{3}{l}{search} & \multicolumn{3}{l}{pig} & \multicolumn{2}{l}{\textsc{1sg}} & \multicolumn{7}{l}{feel.happy(.about)} & \multicolumn{2}{l}{eat} & \multicolumn{4}{l}{pig}\\
\lspbottomrule
\end{tabular}
\ea
\glt 
Iskia: so, one time I was at home, at night I had already thought, I told (my) wife, ‘tomorrow I take the dogs and look for pigs’, I like eating pig
\z

\begin{tabular}{lllm{-9.4015896E-4in}lllllllllllll}
\lsptoprule
0002 & tong & \multicolumn{3}{l}{tidor} & malam & \multicolumn{2}{l}{sampe} & \multicolumn{3}{l}{pagi} & \multicolumn{2}{l}{saya} & kas & \multicolumn{2}{l}{makang} & anjing\\
& \textsc{1pl} & \multicolumn{3}{l}{sleep} & night & \multicolumn{2}{l}{until} & \multicolumn{3}{l}{morning} & \multicolumn{2}{l}{\textsc{1sg}} & give & \multicolumn{2}{l}{eat} & dog\\
& \multicolumn{2}{l}{deng} & \multicolumn{3}{l}{papeda} & yang & \multicolumn{2}{l}{sa} & pu & \multicolumn{2}{l}{bini} & \multicolumn{2}{l}{biking} & malam & \multicolumn{2}{l}{untuk}\\
& \multicolumn{2}{l}{with} & \multicolumn{3}{l}{sagu.porridge} & \textsc{rel} & \multicolumn{2}{l}{\textsc{1sg}} & \textsc{poss} & \multicolumn{2}{l}{wife} & \multicolumn{2}{l}{make} & night & \multicolumn{2}{l}{for}\\
& \multicolumn{3}{l}{anjing} & \multicolumn{13}{l}{dorang}\\
& \multicolumn{3}{l}{dog} & \multicolumn{13}{l}{\textsc{3pl}}\\
\lspbottomrule
\end{tabular}
\ea
\glt 
we slept through the night until morning, I fed the dogs with papeda which my wife had prepared in the evening for the dogs
\z

\begin{tabular}{llllllllllllllllllll}
\lsptoprule
0003 & jadi & \multicolumn{3}{l}{pagi} & \multicolumn{3}{l}{saya} & \multicolumn{3}{l}{bangung,} & \multicolumn{3}{l}{sa} & \multicolumn{2}{l}{kasi} & \multicolumn{2}{l}{makang} & anjing, & sa\\
& so & \multicolumn{3}{l}{morning} & \multicolumn{3}{l}{\textsc{1sg}} & \multicolumn{3}{l}{wake.up} & \multicolumn{3}{l}{\textsc{1sg}} & \multicolumn{2}{l}{give} & \multicolumn{2}{l}{eat} & dog & \textsc{1sg}\\
& \multicolumn{2}{l}{pegang} & sa & \multicolumn{3}{l}{pu} & \multicolumn{6}{l}{parang,} & \multicolumn{2}{l}{sa} & \multicolumn{2}{l}{punya} & \multicolumn{3}{l}{jubi,}\\
& \multicolumn{2}{l}{hold} & \textsc{1sg} & \multicolumn{3}{l}{\textsc{poss}} & \multicolumn{6}{l}{short.machete} & \multicolumn{2}{l}{\textsc{1sg}} & \multicolumn{2}{l}{\textsc{poss}} & \multicolumn{3}{l}{bow.and.arrow}\\
& sa & \multicolumn{4}{l}{tokiang} & \multicolumn{3}{l}{pana,} & sa & \multicolumn{2}{l}{toki} & \multicolumn{8}{l}{pana}\\
& \textsc{1sg} & \multicolumn{4}{l}{\textsc{spm}{}-beat} & \multicolumn{3}{l}{arrow} & \textsc{1sg} & \multicolumn{2}{l}{beat} & \multicolumn{8}{l}{arrow}\\
\lspbottomrule
\end{tabular}
\ea
\glt 
so, in the morning I got up, I fed the dogs, I took my short machete, my bow and arrows, I banged[\textsc{spm}] (my) arrows, I banged my arrows
\z

\begin{tabular}{lll}
\lsptoprule
0004 & Sarlota: & jubi\\
&  & bow.and.arrow\\
\lspbottomrule
\end{tabular}
\ea
\glt 
Sarlota: (I banged my) bow and arrows
\z

\begin{tabular}{lllllllll}
\lsptoprule
0005 & Iskia: & jubi, & anjing & ikut & saya & masuk & di & hutang\\
&  & bow.and.arrow & dog & follow & \textsc{1sg} & enter & at & forest\\
\lspbottomrule
\end{tabular}
\ea
\glt 
Iskia: (I banged my) bow and arrows, the dogs followed me entering the forest
\z

\begin{tabular}{llllllllllllllllll}
\lsptoprule
0006 & saya & \multicolumn{3}{l}{jalang} & \multicolumn{2}{l}{sampe} & di & \multicolumn{3}{l}{blakang} & \multicolumn{2}{l}{kebung,} & \multicolumn{2}{l}{anjing} & \multicolumn{2}{l}{mulay} & gong-gong\\
& \textsc{1sg} & \multicolumn{3}{l}{walk} & \multicolumn{2}{l}{reach} & at & \multicolumn{3}{l}{backside} & \multicolumn{2}{l}{garden} & \multicolumn{2}{l}{dog} & \multicolumn{2}{l}{start} & bark(.at)\\
& \multicolumn{2}{l}{babi} & o, & \multicolumn{2}{l}{tida} & \multicolumn{3}{l}{lama} & lagi & \multicolumn{2}{l}{dong} & \multicolumn{2}{l}{su} & \multicolumn{2}{l}{kasi} & \multicolumn{2}{l}{berdiri}\\
& \multicolumn{2}{l}{pig} & oh! & \multicolumn{2}{l}{\textsc{neg}} & \multicolumn{3}{l}{be.long} & again & \multicolumn{2}{l}{\textsc{3pl}} & \multicolumn{2}{l}{already} & \multicolumn{2}{l}{give} & \multicolumn{2}{l}{stand}\\
\lspbottomrule
\end{tabular}
\ea
\glt 
I walked all the way to the back of (my) garden, the dogs start barking (because they smelt) a pig, oh, not long after that they already had (the pig) standing (still)
\z

\begin{tabular}{lllllllllllllllllllll}
\lsptoprule
0007 & sa & \multicolumn{2}{l}{lari} & \multicolumn{2}{l}{suda,} & \multicolumn{3}{l}{mendekati} & \multicolumn{2}{l}{babi} & \multicolumn{2}{l}{di} & \multicolumn{3}{l}{mana} & \multicolumn{2}{l}{anjing} & \multicolumn{3}{l}{dong}\\
& \textsc{1sg} & \multicolumn{2}{l}{run} & \multicolumn{2}{l}{already} & \multicolumn{3}{l}{near} & \multicolumn{2}{l}{pig} & \multicolumn{2}{l}{at} & \multicolumn{3}{l}{where} & \multicolumn{2}{l}{dog} & \multicolumn{3}{l}{\textsc{3pl}}\\
& \multicolumn{4}{l}{gong-gong,} & \multicolumn{2}{l}{baru} & sa & \multicolumn{2}{l}{mulay} & \multicolumn{5}{l}{pana} & \multicolumn{2}{l}{dia,} & \multicolumn{4}{l}{pana}\\
& \multicolumn{4}{l}{bark(.at)} & \multicolumn{2}{l}{and.then} & \textsc{1sg} & \multicolumn{2}{l}{start} & \multicolumn{5}{l}{bow.shoot} & \multicolumn{2}{l}{\textsc{3sg}} & \multicolumn{4}{l}{bow.shoot}\\
& \multicolumn{2}{l}{dengang} & \multicolumn{4}{l}{jubi,} & sa & \multicolumn{4}{l}{jubi} & \multicolumn{2}{l}{dia,} & \multicolumn{5}{l}{langsung} & babi & mati\\
& \multicolumn{2}{l}{with} & \multicolumn{4}{l}{bow.and.arrow} & \textsc{1sg} & \multicolumn{4}{l}{bow.shoot} & \multicolumn{2}{l}{\textsc{3sg}} & \multicolumn{5}{l}{immediately} & pig & die\\
\lspbottomrule
\end{tabular}
\ea
\glt 
I just ran closing in on the pig where the dogs were barking, then I started bow shooting (it), bow shooting (it) with (my) bow and arrows, I bow shot it, immediately the pig died
\z

\begin{tabular}{llllllllllllllllllll}
\lsptoprule
0008 & wa, & \multicolumn{3}{l}{babi} & \multicolumn{3}{l}{besar} & \multicolumn{2}{l}{skali,} & \multicolumn{2}{l}{sa} & \multicolumn{2}{l}{sendiri} & \multicolumn{2}{l}{tra} & \multicolumn{2}{l}{bisa} & angkat, & sa\\
& wow! & \multicolumn{3}{l}{pig} & \multicolumn{3}{l}{be.big} & \multicolumn{2}{l}{very} & \multicolumn{2}{l}{\textsc{1sg}} & \multicolumn{2}{l}{alone} & \multicolumn{2}{l}{\textsc{neg}} & \multicolumn{2}{l}{be.able} & lift & \textsc{1sg}\\
& \multicolumn{2}{l}{pikir,} & \multicolumn{3}{l}{adu,} & \multicolumn{3}{l}{babi} & \multicolumn{2}{l}{ni} & \multicolumn{2}{l}{sa} & \multicolumn{2}{l}{harus} & \multicolumn{2}{l}{angkat} & \multicolumn{3}{l}{bagemana,}\\
& \multicolumn{2}{l}{think} & \multicolumn{3}{l}{oh.no!} & \multicolumn{3}{l}{pig} & \multicolumn{2}{l}{\textsc{d.prox}} & \multicolumn{2}{l}{\textsc{1sg}} & \multicolumn{2}{l}{have.to} & \multicolumn{2}{l}{lift} & \multicolumn{3}{l}{how}\\
& \multicolumn{3}{l}{ini} & \multicolumn{3}{l}{besar} & \multicolumn{13}{l}{ini}\\
& \multicolumn{3}{l}{\textsc{d.prox}} & \multicolumn{3}{l}{be.big} & \multicolumn{13}{l}{\textsc{d.prox}}\\
\lspbottomrule
\end{tabular}
\ea
\glt 
wow!, the pig was very big, I alone could not transport it, I thought, ‘oh no!, this pig, how am I going to transport (it), this (one) here is really big’
\z

\begin{tabular}{llllllllll}
\lsptoprule
0009 & tida & \multicolumn{2}{l}{lama} & \multicolumn{2}{l}{sa} & dengar & ada & swara, & orang,\\
& \textsc{neg} & \multicolumn{2}{l}{be.long} & \multicolumn{2}{l}{\textsc{1sg}} & hear & exist & voice & person\\
& \multicolumn{2}{l}{baru} & \multicolumn{2}{l}{saya} & \multicolumn{5}{l}{panggil}\\
& \multicolumn{2}{l}{and.then} & \multicolumn{2}{l}{\textsc{1sg}} & \multicolumn{5}{l}{call}\\
\lspbottomrule
\end{tabular}
\ea
\glt 
not long after that I heard there were voices, (there were) people, and then I called (them)
\z

\begin{tabular}{llllllllllllllllllllllllllll}
\lsptoprule
0010 & \multicolumn{2}{l}{mereka} & \multicolumn{3}{l}{ada} & \multicolumn{4}{l}{tiga} & \multicolumn{3}{l}{orang,} & \multicolumn{3}{l}{dorang} & \multicolumn{3}{l}{datang,} & \multicolumn{3}{l}{dengar} & \multicolumn{4}{l}{ini,} & \multicolumn{2}{l}{anjing}\\
& \multicolumn{2}{l}{\textsc{3pl}[SI]} & \multicolumn{3}{l}{exist} & \multicolumn{4}{l}{three} & \multicolumn{3}{l}{person} & \multicolumn{3}{l}{\textsc{3pl}} & \multicolumn{3}{l}{come} & \multicolumn{3}{l}{hear} & \multicolumn{4}{l}{\textsc{d.prox}} & \multicolumn{2}{l}{dog}\\
& \multicolumn{4}{l}{gong-gong} & \multicolumn{3}{l}{babi,} & \multicolumn{3}{l}{tapi} & \multicolumn{7}{l}{sementara,} & \multicolumn{3}{l}{karna} & \multicolumn{3}{l}{mereka} & \multicolumn{3}{l}{jaw,} & lari\\
& \multicolumn{4}{l}{bark(.at)} & \multicolumn{3}{l}{pig} & \multicolumn{3}{l}{but} & \multicolumn{7}{l}{in.meantime[SI]} & \multicolumn{3}{l}{because} & \multicolumn{3}{l}{\textsc{3pl}[SI]} & \multicolumn{3}{l}{far} & run\\
& mo & \multicolumn{7}{l}{pana} & \multicolumn{3}{l}{babi} & \multicolumn{2}{l}{bantu} & \multicolumn{3}{l}{sama} & \multicolumn{3}{l}{dengang} & \multicolumn{3}{l}{saya,} & \multicolumn{2}{l}{tapi} & \multicolumn{3}{l}{saya}\\
& want & \multicolumn{7}{l}{bow.shoot} & \multicolumn{3}{l}{pig} & \multicolumn{2}{l}{help} & \multicolumn{3}{l}{to} & \multicolumn{3}{l}{with} & \multicolumn{3}{l}{\textsc{1sg}} & \multicolumn{2}{l}{but} & \multicolumn{3}{l}{\textsc{1sg}}\\
& \multicolumn{3}{l}{suda} & \multicolumn{3}{l}{bunu,} & \multicolumn{6}{l}{pana} & \multicolumn{2}{l}{dia} & \multicolumn{13}{l}{kemuka}\\
& \multicolumn{3}{l}{already} & \multicolumn{3}{l}{kill} & \multicolumn{6}{l}{bow.shoot} & \multicolumn{2}{l}{\textsc{3sg}} & \multicolumn{13}{l}{first.before.others}\\
\lspbottomrule
\end{tabular}
\ea
\glt 
they were three people, they came (and) heard, what’s-its-name, the dogs barking at the pig, but in the meantime, because they were far away, (they) ran wanting to bow shoot the pig, to help me, but I had already killed (it), had bow shot (it) before the others
\z

\begin{tabular}{llllllllllllllll}
\lsptoprule
0011 & waktu & \multicolumn{3}{l}{mereka} & \multicolumn{2}{l}{sampe} & \multicolumn{2}{l}{dekat} & \multicolumn{2}{l}{saya,} & \multicolumn{2}{l}{babi} & suda & mati & jadi\\
& time & \multicolumn{3}{l}{\textsc{3pl}[SI]} & \multicolumn{2}{l}{reach} & \multicolumn{2}{l}{near} & \multicolumn{2}{l}{\textsc{1sg}} & \multicolumn{2}{l}{pig} & already & die & so\\
& \multicolumn{2}{l}{tinggal} & sa & \multicolumn{2}{l}{bilang} & \multicolumn{2}{l}{saja,} & \multicolumn{2}{l}{babi} & \multicolumn{2}{l}{suda} & \multicolumn{4}{l}{mati}\\
& \multicolumn{2}{l}{stay} & \textsc{1sg} & \multicolumn{2}{l}{say} & \multicolumn{2}{l}{just} & \multicolumn{2}{l}{pig} & \multicolumn{2}{l}{already} & \multicolumn{4}{l}{die}\\
\lspbottomrule
\end{tabular}
\ea
\glt 
when they arrived near me the pig was already dead, so it just remained for me to say, ‘the pig is already dead’
\z

\begin{tabular}{lllllllllllll}
\lsptoprule
0012 & jadi & \multicolumn{2}{l}{nanti} & \multicolumn{2}{l}{kitong} & \multicolumn{2}{l}{berusaha} & pikol & ke & \multicolumn{2}{l}{ruma} & kebung,\\
& so & \multicolumn{2}{l}{very.soon} & \multicolumn{2}{l}{\textsc{1pl}} & \multicolumn{2}{l}{attempt} & shoulder & to & \multicolumn{2}{l}{house} & garden\\
& \multicolumn{2}{l}{baru} & \multicolumn{2}{l}{nanti} & \multicolumn{2}{l}{kita} & potong, & baru & \multicolumn{2}{l}{nanti} & \multicolumn{2}{l}{bagi}\\
& \multicolumn{2}{l}{and.then} & \multicolumn{2}{l}{very.soon} & \multicolumn{2}{l}{\textsc{1pl}} & cut & and.then & \multicolumn{2}{l}{very.soon} & \multicolumn{2}{l}{divide}\\
\lspbottomrule
\end{tabular}
\ea
\glt 
so later we’ll try to carry the pig on our shoulders to the garden shelter, only then we’ll cut it up, and then we’ll distribute (it)
\z

\begin{tabular}{llllllllllll}
\lsptoprule
0013 & itu & juga, & a, & \multicolumn{2}{l}{tong} & \multicolumn{3}{l}{langsung} & ambil & itu, & pikol\\
& \textsc{d.dist} & also & ah! & \multicolumn{2}{l}{\textsc{1pl}} & \multicolumn{3}{l}{immediately} & fetch & \textsc{d.dist} & shoulder\\
& itu, & babi, & \multicolumn{2}{l}{bawa} & \multicolumn{2}{l}{ke} & ruma & \multicolumn{4}{l}{kebung}\\
& \textsc{d.dist} & pig & \multicolumn{2}{l}{bring} & \multicolumn{2}{l}{to} & house & \multicolumn{4}{l}{garden}\\
\lspbottomrule
\end{tabular}
\ea
\glt 
right after that, ah, we took it immediately, we shouldered it, the pig, (and) carried (it) to the garden shelter
\z

\begin{tabular}{llm{-9.4015896E-4in}llllllllllllllllllllll}
\lsptoprule
0014 & tong & \multicolumn{4}{l}{potong} & \multicolumn{2}{l}{hari} & \multicolumn{3}{l}{itu,} & \multicolumn{3}{l}{tong} & \multicolumn{3}{l}{bagi} & \multicolumn{3}{l}{buat} & \multicolumn{4}{l}{kitorang} & yang\\
& \textsc{1pl} & \multicolumn{4}{l}{cut} & \multicolumn{2}{l}{day} & \multicolumn{3}{l}{\textsc{d.dist}} & \multicolumn{3}{l}{\textsc{1pl}} & \multicolumn{3}{l}{divide} & \multicolumn{3}{l}{for} & \multicolumn{4}{l}{\textsc{1pl}} & \textsc{rel}\\
& \multicolumn{3}{l}{potong} & \multicolumn{3}{l}{itu} & \multicolumn{2}{l}{hari,} & \multicolumn{5}{l}{kemudiang} & \multicolumn{2}{l}{buat} & \multicolumn{6}{l}{sodara{\Tilde}sodara} & \multicolumn{3}{l}{yang}\\
& \multicolumn{3}{l}{cut} & \multicolumn{3}{l}{\textsc{d.dist}} & \multicolumn{2}{l}{day} & \multicolumn{5}{l}{then[SI]} & \multicolumn{2}{l}{for} & \multicolumn{6}{l}{\textsc{rdp}{\Tilde}sibling} & \multicolumn{3}{l}{\textsc{rel}}\\
& \multicolumn{2}{l}{tinggal} & \multicolumn{2}{l}{di} & \multicolumn{4}{l}{kampong,} & \multicolumn{3}{l}{kitong} & \multicolumn{3}{l}{hitung} & \multicolumn{4}{l}{ada} & \multicolumn{2}{l}{dua} & \multicolumn{2}{l}{pulu} & \multicolumn{2}{l}{satu}\\
& \multicolumn{2}{l}{stay} & \multicolumn{2}{l}{at} & \multicolumn{4}{l}{village} & \multicolumn{3}{l}{\textsc{1pl}} & \multicolumn{3}{l}{count} & \multicolumn{4}{l}{exist} & \multicolumn{2}{l}{two} & \multicolumn{2}{l}{tens} & \multicolumn{2}{l}{one}\\
& \multicolumn{6}{l}{KK} & di & \multicolumn{2}{l}{sa} & \multicolumn{3}{l}{punya} & \multicolumn{5}{l}{kampung} & \multicolumn{7}{l}{itu}\\
& \multicolumn{6}{l}{household.head} & at & \multicolumn{2}{l}{\textsc{1sg}} & \multicolumn{3}{l}{\textsc{poss}} & \multicolumn{5}{l}{village} & \multicolumn{7}{l}{\textsc{d.dist}}\\
\lspbottomrule
\end{tabular}
\ea
\glt 
we cut (it) up that day, we divided (it) for us who cut (it) up that day, (and) then for the relatives and friends who live in the village, we counted (them), there are twenty one heads of households in that village of mine
\z

\begin{tabular}{lllllllllll}
\lsptoprule
0015 & jadi, & waktu & \multicolumn{2}{l}{saya} & \multicolumn{2}{l}{potong} & babi & ini, & daging & saya\\
& so & time & \multicolumn{2}{l}{\textsc{1sg}} & \multicolumn{2}{l}{cut} & pig & \textsc{d.prox} & meat & \textsc{1sg}\\
& \multicolumn{3}{l}{memperkecil,} & \multicolumn{2}{l}{saya} & bagi & \multicolumn{4}{l}{juga}\\
& \multicolumn{3}{l}{make.smaller} & \multicolumn{2}{l}{\textsc{1sg}} & divide & \multicolumn{4}{l}{also}\\
\lspbottomrule
\end{tabular}
\ea
\glt 
so, when I cut up this pig, the meat, I cut (it) into small pieces, (and) I distributed them
\z

\begin{tabular}{llll}
\lsptoprule
0016 & Sarlota: & potong & kecil{\Tilde}kecil\\
&  & cut & \textsc{rdp}{\Tilde}be.small\\
\lspbottomrule
\end{tabular}
\ea
\glt 
Sarlota: (I) cut (it into) small (pieces)
\z

\begin{tabular}{llllllllllll}
\lsptoprule
0017 & Iskia: & \multicolumn{3}{l}{kecil{\Tilde}kecil,} & \multicolumn{2}{l}{baru} & saya & bagi & sampe & dua & pulu\\
&  & \multicolumn{3}{l}{\textsc{rdp}{\Tilde}be.small} & \multicolumn{2}{l}{and.then} & \textsc{1sg} & divide & reach & two & tens\\
& \multicolumn{2}{l}{bagi,} & dua & pulu & satu & \multicolumn{6}{l}{bagiang}\\
& \multicolumn{2}{l}{\textsc{tru}{}-part} & two & tens & one & \multicolumn{6}{l}{part}\\
\lspbottomrule
\end{tabular}
\ea
\glt 
Iskia: small (pieces), and then I divided (them) into twenty parts[\textsc{tru}], twenty one parts
\z

\begin{tabular}{llllllllllllllll}
\lsptoprule
0018 & \multicolumn{2}{l}{waktu} & kita & \multicolumn{3}{l}{pulang,} & ta & \multicolumn{2}{l}{p,} & \multicolumn{2}{l}{empat} & \multicolumn{3}{l}{orang} & itu,\\
& \multicolumn{2}{l}{time} & \textsc{1pl} & \multicolumn{3}{l}{go.home} & \textsc{1pl} & \multicolumn{2}{l}{\textsc{tru}{}-go.home} & \multicolumn{2}{l}{four} & \multicolumn{3}{l}{person} & \textsc{d.dist}\\
& kita & \multicolumn{3}{l}{suda} & bawa, & \multicolumn{3}{l}{masing-masing,} & \multicolumn{2}{l}{kita} & \multicolumn{2}{l}{suda} & baku & \multicolumn{2}{l}{bagi}\\
& \textsc{1pl} & \multicolumn{3}{l}{already} & bring & \multicolumn{3}{l}{each} & \multicolumn{2}{l}{\textsc{1pl}} & \multicolumn{2}{l}{already} & \textsc{recp} & \multicolumn{2}{l}{divide}\\
\lspbottomrule
\end{tabular}
\ea
\glt 
when we went home, (when) we went home[\textsc{tru}], those four people, we brought (the meat) already (having been divided up), each of us, we had already divided (the meat) with each other
\z

\begin{tabular}{llllllllllllllllllllllllll}
\lsptoprule
0019 & \multicolumn{4}{l}{nanti} & \multicolumn{3}{l}{ko} & \multicolumn{3}{l}{kasi} & \multicolumn{3}{l}{sodara} & \multicolumn{3}{l}{yang} & \multicolumn{3}{l}{laing,} & \multicolumn{2}{l}{saya} & juga & \multicolumn{3}{l}{nanti}\\
& \multicolumn{4}{l}{very.soon} & \multicolumn{3}{l}{\textsc{2sg}} & \multicolumn{3}{l}{give} & \multicolumn{3}{l}{sibling} & \multicolumn{3}{l}{\textsc{rel}} & \multicolumn{3}{l}{be.different} & \multicolumn{2}{l}{\textsc{1sg}} & also & \multicolumn{3}{l}{very.soon}\\
& \multicolumn{2}{l}{bagi} & \multicolumn{6}{l}{so} & \multicolumn{3}{l}{sodara} & \multicolumn{4}{l}{yang} & \multicolumn{3}{l}{laing} & \multicolumn{3}{l}{suda} & \multicolumn{2}{l}{punya} & \multicolumn{2}{l}{bagiang,}\\
& \multicolumn{2}{l}{divide} & \multicolumn{6}{l}{\textsc{tru}{}-sibling} & \multicolumn{3}{l}{sibling} & \multicolumn{4}{l}{\textsc{rel}} & \multicolumn{3}{l}{be.different} & \multicolumn{3}{l}{already} & \multicolumn{2}{l}{have} & \multicolumn{2}{l}{part}\\
& \multicolumn{3}{l}{tinggal} & \multicolumn{3}{l}{kita} & \multicolumn{3}{l}{bawa} & \multicolumn{3}{l}{sampe} & \multicolumn{2}{l}{di} & \multicolumn{3}{l}{ruma,} & \multicolumn{3}{l}{suda} & \multicolumn{4}{l}{sore} & hari,\\
& \multicolumn{3}{l}{stay} & \multicolumn{3}{l}{\textsc{1pl}} & \multicolumn{3}{l}{bring} & \multicolumn{3}{l}{reach} & \multicolumn{2}{l}{at} & \multicolumn{3}{l}{house} & \multicolumn{3}{l}{already} & \multicolumn{4}{l}{afternoon} & day\\
& kita & \multicolumn{4}{l}{bagi} & \multicolumn{20}{l}{malam}\\
& \textsc{1pl} & \multicolumn{4}{l}{divide} & \multicolumn{20}{l}{night}\\
\lspbottomrule
\end{tabular}
\ea
\glt 
later you give (the meat) to other friends and relatives, later I’ll also distribute (it to) other friends and relatives, (we) already have (our) share, it remains that we bring (our share home), having arrived home, it was already afternoon, we distributed (the meat until) the evening
\z

\begin{tabular}{lllllllllllll}
\lsptoprule
0020 & \multicolumn{3}{l}{sodara{\Tilde}sodara} & \multicolumn{2}{l}{dorang} & \multicolumn{2}{l}{mo} & \multicolumn{2}{l}{masak} & sayur, & liat & begini,\\
& \multicolumn{3}{l}{\textsc{rdp}{\Tilde}sibling} & \multicolumn{2}{l}{\textsc{3pl}} & \multicolumn{2}{l}{want} & \multicolumn{2}{l}{cook} & vegetable & see & like.this\\
& ta & bawa & \multicolumn{2}{l}{daging,} & \multicolumn{2}{l}{siapa} & \multicolumn{2}{l}{yang} & \multicolumn{4}{l}{dapat?}\\
& \textsc{1pl} & bring & \multicolumn{2}{l}{meat} & \multicolumn{2}{l}{who} & \multicolumn{2}{l}{\textsc{rel}} & \multicolumn{4}{l}{get}\\
\lspbottomrule
\end{tabular}
\ea
\glt 
the relatives and friends wanted to cook vegetables, as (they) saw that we brought (them) meat (they asked us), ‘who (is the one) who got (the pig)?’
\z

\begin{tabular}{llllllllllllllllllll}
\lsptoprule
0021 & \multicolumn{2}{l}{bilang,} & \multicolumn{2}{l}{saya} & \multicolumn{3}{l}{yang} & \multicolumn{3}{l}{tadi} & \multicolumn{2}{l}{pagi} & \multicolumn{2}{l}{berburu,} & \multicolumn{2}{l}{bawa} & \multicolumn{2}{l}{anjing,} & baru\\
& \multicolumn{2}{l}{say} & \multicolumn{2}{l}{\textsc{1sg}} & \multicolumn{3}{l}{\textsc{rel}} & \multicolumn{3}{l}{earlier} & \multicolumn{2}{l}{morning} & \multicolumn{2}{l}{hunt} & \multicolumn{2}{l}{bring} & \multicolumn{2}{l}{dog} & and.then\\
& dapat & \multicolumn{4}{l}{babi} & \multicolumn{3}{l}{ini,} & \multicolumn{3}{l}{betulang,} & \multicolumn{2}{l}{ini} & \multicolumn{2}{l}{daging} & \multicolumn{2}{l}{yang} & saya & bawa,\\
& get & \multicolumn{4}{l}{pig} & \multicolumn{3}{l}{\textsc{d.prox}} & \multicolumn{3}{l}{chance} & \multicolumn{2}{l}{\textsc{d.prox}} & \multicolumn{2}{l}{meat} & \multicolumn{2}{l}{\textsc{rel}} & \textsc{1sg} & bring\\
& \multicolumn{3}{l}{antar} & \multicolumn{3}{l}{buat} & \multicolumn{3}{l}{sodara} & \multicolumn{10}{l}{dorang}\\
& \multicolumn{3}{l}{deliver} & \multicolumn{3}{l}{for} & \multicolumn{3}{l}{sibling} & \multicolumn{10}{l}{\textsc{3pl}}\\
\lspbottomrule
\end{tabular}
\ea
\glt 
(I) said, ‘(it was) me who went hunting this morning, (who) took the dogs and then got this pig, coincidentally, this is the meat which I brought, (which I) delivered for (my) relatives’
\z

\begin{tabular}{lllllllllllllllllllllllll}
\lsptoprule
0022 & dong & \multicolumn{4}{l}{bilang,} & \multicolumn{4}{l}{trima-kasi,} & \multicolumn{3}{l}{tong} & \multicolumn{3}{l}{mo} & \multicolumn{3}{l}{makang} & \multicolumn{4}{l}{sayur} & \multicolumn{2}{l}{malam}\\
& \textsc{3pl} & \multicolumn{4}{l}{say} & \multicolumn{4}{l}{thank.you} & \multicolumn{3}{l}{\textsc{1pl}} & \multicolumn{3}{l}{want} & \multicolumn{3}{l}{eat} & \multicolumn{4}{l}{vegetable} & \multicolumn{2}{l}{night}\\
& \multicolumn{3}{l}{ini,} & tapi & \multicolumn{3}{l}{ya} & \multicolumn{3}{l}{sodara} & \multicolumn{3}{l}{ko} & \multicolumn{3}{l}{bawa} & \multicolumn{3}{l}{daging,} & \multicolumn{2}{l}{kitong} & \multicolumn{3}{l}{trima-kasi,}\\
& \multicolumn{3}{l}{\textsc{d.prox}} & but & \multicolumn{3}{l}{yes} & \multicolumn{3}{l}{sibling} & \multicolumn{3}{l}{\textsc{2sg}} & \multicolumn{3}{l}{bring} & \multicolumn{3}{l}{meat} & \multicolumn{2}{l}{\textsc{1pl}} & \multicolumn{3}{l}{thank.you}\\
& \multicolumn{2}{l}{karna} & \multicolumn{4}{l}{kitong} & \multicolumn{2}{l}{bisa} & \multicolumn{3}{l}{masak} & \multicolumn{3}{l}{daging,} & \multicolumn{3}{l}{sodara} & \multicolumn{3}{l}{berburu} & \multicolumn{3}{l}{daging,} & babi\\
& \multicolumn{2}{l}{because} & \multicolumn{4}{l}{\textsc{1pl}} & \multicolumn{2}{l}{be.able} & \multicolumn{3}{l}{cook} & \multicolumn{3}{l}{meat} & \multicolumn{3}{l}{sibling} & \multicolumn{3}{l}{hunt} & \multicolumn{3}{l}{meat} & pig\\
\lspbottomrule
\end{tabular}
\ea
\glt 
they said, ‘thank you!, we were going to eat vegetables tonight, but, yes, you brother brought (us) meat, we say thank you, because (now) we can cook meat, (you) brother hunted meat, a pig’
\z

\begin{tabular}{lllllllllllllll}
\lsptoprule
0023 & jadi & \multicolumn{3}{l}{ini} & \multicolumn{2}{l}{kehidupang} & \multicolumn{2}{l}{orang} & \multicolumn{2}{l}{Papua} & \multicolumn{2}{l}{ini} & sperti & begini,\\
& so & \multicolumn{3}{l}{\textsc{d.prox}} & \multicolumn{2}{l}{life} & \multicolumn{2}{l}{person} & \multicolumn{2}{l}{Papua} & \multicolumn{2}{l}{\textsc{d.prox}} & similar.to & like.this\\
& \multicolumn{2}{l}{kalo} & mo & \multicolumn{2}{l}{makang} & \multicolumn{2}{l}{babi,} & \multicolumn{2}{l}{harus} & \multicolumn{2}{l}{bawa} & \multicolumn{3}{l}{anjing}\\
& \multicolumn{2}{l}{if} & want & \multicolumn{2}{l}{eat} & \multicolumn{2}{l}{pig} & \multicolumn{2}{l}{have.to} & \multicolumn{2}{l}{bring} & \multicolumn{3}{l}{dog}\\
\lspbottomrule
\end{tabular}
\ea
\glt 
so, what’s-its-name, the life of (us) Papuan people here is like this: if (you) want to eat pig, (you) have to take dogs (with you)
\z

\begin{tabular}{lllllllllll}
\lsptoprule
0022 & \multicolumn{2}{l}{kemudiang,} & \multicolumn{2}{l}{itu} & ceritra & waktu & kita & berburu & pake & anjing,\\
& \multicolumn{2}{l}{then[SI]} & \multicolumn{2}{l}{\textsc{d.dist}} & tell & time & \textsc{1pl} & hunt & use & dog\\
& ya, & \multicolumn{2}{l}{sperti} & \multicolumn{7}{l}{itu}\\
& yes & \multicolumn{2}{l}{similar.to} & \multicolumn{7}{l}{\textsc{d.dist}}\\
\lspbottomrule
\end{tabular}
\ea
\glt
then, this was the story when we go hunting and use dogs, yes, it’s like that
\end{styleFreeTranslEngxvpt}

\subsection{Expository: Directions to a certain statue and tree\footnotemark{}}
\footnotetext{\\
\\
\\
\\
\\
\\
\\
\\
\\
\\
\\
\\
\\
\\
\\
\par After having recounted a story about a certain statue in Sarmi which was built close to a certain tree, a boy asked the narrator for directions to the statue and tree.}

\begin{tabular}{ll}
\lsptoprule
File name: & 080917-009-CvEx\\
Text type: & Conversation, spontaneous: Expository\\
Interlocutors: & 2 male child, 1 older female\\
Length (min.): & 0:50\\
\lspbottomrule
\end{tabular}
\begin{tabular}{llllllllllllll}
\lsptoprule
0001 & \multicolumn{2}{l}{Natalia:} & ... & \multicolumn{2}{l}{greja} & \multicolumn{2}{l}{sebla,} & \multicolumn{3}{l}{pokoknya} & \multicolumn{3}{l}{ruma}\\
& \multicolumn{2}{l}{} & ... & \multicolumn{2}{l}{church} & \multicolumn{2}{l}{side} & \multicolumn{3}{l}{the.main.thing.is} & \multicolumn{3}{l}{house}\\
& tingkat & \multicolumn{3}{l}{itu,} & \multicolumn{2}{l}{ruma-sakit} & \multicolumn{2}{l}{itu} & sebla & \multicolumn{2}{l}{itu} & ada & [\textsc{up}]\\
& floor & \multicolumn{3}{l}{\textsc{d.dist}} & \multicolumn{2}{l}{hospital} & \multicolumn{2}{l}{\textsc{d.dist}} & side & \multicolumn{2}{l}{\textsc{d.dist}} & exist & \\
\lspbottomrule
\end{tabular}
\ea
\glt 
[Reply about the directions to a certain statue:] Natalia: … next to the church, the main landmark is the multistoried house, that hospital next to it is [\textsc{up}]
\z

\begin{tabular}{llllllll}
\lsptoprule
0002 & Wili: & yang & Matias & de & ada & sakit & itu?\\
&  & \textsc{rel} & Matias & \textsc{3sg} & exist & be.sick & \textsc{d.dist}\\
\lspbottomrule
\end{tabular}
\ea
\glt 
Wili: where Matias was sick?
\z

\begin{tabular}{llllllllllllllllllllllllllll}
\lsptoprule
0003 & \multicolumn{2}{l}{Natalia:} & \multicolumn{4}{l}{yo,} & \multicolumn{3}{l}{Matias} & \multicolumn{3}{l}{ada} & \multicolumn{4}{l}{sakit} & \multicolumn{4}{l}{itu,} & \multicolumn{2}{l}{liat} & \multicolumn{2}{l}{sebla} & \multicolumn{2}{l}{laut} & itu\\
& \multicolumn{2}{l}{} & \multicolumn{4}{l}{yes} & \multicolumn{3}{l}{Matias} & \multicolumn{3}{l}{exist} & \multicolumn{4}{l}{be.sick} & \multicolumn{4}{l}{\textsc{d.dist}} & \multicolumn{2}{l}{see} & \multicolumn{2}{l}{side} & \multicolumn{2}{l}{sea} & \textsc{d.dist}\\
& dong & \multicolumn{4}{l}{ada} & \multicolumn{4}{l}{biking} & \multicolumn{5}{l}{begini,} & \multicolumn{3}{l}{besar} & \multicolumn{2}{l}{de} & \multicolumn{4}{l}{pu} & \multicolumn{4}{l}{tugu,}\\
& \textsc{3pl} & \multicolumn{4}{l}{exist} & \multicolumn{4}{l}{make} & \multicolumn{5}{l}{like.this} & \multicolumn{3}{l}{be.big} & \multicolumn{2}{l}{\textsc{3sg}} & \multicolumn{4}{l}{\textsc{poss}} & \multicolumn{4}{l}{monument}\\
& \multicolumn{3}{l}{baru,} & \multicolumn{4}{l}{a,} & \multicolumn{3}{l}{dong} & \multicolumn{3}{l}{biking} & \multicolumn{5}{l}{bagus,} & \multicolumn{3}{l}{smeng} & \multicolumn{4}{l}{bagus} & \multicolumn{2}{l}{skali,}\\
& \multicolumn{3}{l}{and.then} & \multicolumn{4}{l}{ah!} & \multicolumn{3}{l}{\textsc{3pl}} & \multicolumn{3}{l}{make} & \multicolumn{5}{l}{be.good} & \multicolumn{3}{l}{cement} & \multicolumn{4}{l}{be.good} & \multicolumn{2}{l}{very}\\
& \multicolumn{4}{l}{nanti} & \multicolumn{4}{l}{kalo} & \multicolumn{3}{l}{ko} & \multicolumn{4}{l}{blum} & \multicolumn{12}{l}{taw}\\
& \multicolumn{4}{l}{very.soon} & \multicolumn{4}{l}{if} & \multicolumn{3}{l}{\textsc{2sg}} & \multicolumn{4}{l}{not.yet} & \multicolumn{12}{l}{know}\\
\lspbottomrule
\end{tabular}
\ea
\glt 
Natalia: yes, Matias was sick there, look toward the ocean, they made (the statue) like this, big is its statue, and then, ah, they built it well, (they) cemented it very well, later (you’ll see it), if you don’t know (it) yet
\z

\begin{tabular}{lllllllllllll}
\lsptoprule
0004 & \multicolumn{2}{l}{nanti} & \multicolumn{2}{l}{tanya} & \multicolumn{2}{l}{Matias,} & \multicolumn{2}{l}{bilang,} & Matias & ko & bawa & sa\\
& \multicolumn{2}{l}{very.soon} & \multicolumn{2}{l}{ask} & \multicolumn{2}{l}{Matias} & \multicolumn{2}{l}{say} & Matias & \textsc{2sg} & bring & \textsc{1sg}\\
& pergi & \multicolumn{2}{l}{liat} & \multicolumn{2}{l}{tugu} & \multicolumn{2}{l}{itu} & \multicolumn{5}{l}{ka?}\\
& go & \multicolumn{2}{l}{see} & \multicolumn{2}{l}{monument} & \multicolumn{2}{l}{\textsc{d.dist}} & \multicolumn{5}{l}{or}\\
\lspbottomrule
\end{tabular}
\ea
\glt 
later ask Matias, say (to him), ‘will you Matias take me to go and see that statue?’
\z

\begin{tabular}{lllll}
\lsptoprule
0005 & Wili: & naik & ke & atas?\\
&  & ascend & to & top\\
\lspbottomrule
\end{tabular}
\ea
\glt 
Wili: (to get there one has to) climb up (the hill)?
\z

\begin{tabular}{lllllllll}
\lsptoprule
0006 & Natalia: & \multicolumn{2}{l}{tra} & naik, & di & dekat & puskesmas & itu,\\
&  & \multicolumn{2}{l}{\textsc{neg}} & ascend & at & near & government.clinic & \textsc{d.dist}\\
& \multicolumn{2}{l}{ruma-sakit} & \multicolumn{6}{l}{situ}\\
& \multicolumn{2}{l}{hospital} & \multicolumn{6}{l}{\textsc{l.med}}\\
\lspbottomrule
\end{tabular}
\ea
\glt 
Natalia: (you) don’t (have to) climb, (the statue) is close to that government clinic, the hospital there
\z

\begin{tabular}{llll}
\lsptoprule
0007 & Wili: & tra & liat\\
&  & \textsc{neg} & see\\
\lspbottomrule
\end{tabular}
\ea
\glt 
Wili: (I) didn’t see (it)
\z

\begin{tabular}{lllllllllllllllllllllllllllllllllll}
\lsptoprule
0008 & \multicolumn{4}{l}{Natalia:} & \multicolumn{3}{l}{ko} & \multicolumn{4}{l}{blum} & \multicolumn{3}{l}{liat,} & \multicolumn{3}{l}{a,} & \multicolumn{5}{l}{nanti} & \multicolumn{8}{l}{baru} & \multicolumn{3}{l}{Matias} & ka\\
& \multicolumn{4}{l}{} & \multicolumn{3}{l}{\textsc{2sg}} & \multicolumn{4}{l}{not.yet} & \multicolumn{3}{l}{see} & \multicolumn{3}{l}{ah!} & \multicolumn{5}{l}{very.soon} & \multicolumn{8}{l}{and.then} & \multicolumn{3}{l}{Matias} & or\\
& ato & \multicolumn{7}{l}{nanti} & \multicolumn{5}{l}{besok} & \multicolumn{3}{l}{ka,} & \multicolumn{3}{l}{deng} & \multicolumn{7}{l}{mama-ade} & \multicolumn{5}{l}{jalang,} & \multicolumn{3}{l}{baru}\\
& or & \multicolumn{7}{l}{very.soon} & \multicolumn{5}{l}{tomorrow} & \multicolumn{3}{l}{or} & \multicolumn{3}{l}{with} & \multicolumn{7}{l}{aunt} & \multicolumn{5}{l}{walk} & \multicolumn{3}{l}{and.then}\\
& \multicolumn{5}{l}{mama-ade} & \multicolumn{4}{l}{kas} & \multicolumn{3}{l}{tunjuk,} & \multicolumn{6}{l}{baru} & \multicolumn{2}{l}{sa} & \multicolumn{3}{l}{kas} & \multicolumn{6}{l}{tunjuk} & \multicolumn{4}{l}{pohong} & yang\\
& \multicolumn{5}{l}{aunt} & \multicolumn{4}{l}{give} & \multicolumn{3}{l}{show} & \multicolumn{6}{l}{and.then} & \multicolumn{2}{l}{\textsc{1sg}} & \multicolumn{3}{l}{give} & \multicolumn{6}{l}{show} & \multicolumn{4}{l}{tree} & \textsc{rel}\\
& \multicolumn{3}{l}{Matias} & \multicolumn{3}{l}{de} & \multicolumn{8}{l}{takut,} & \multicolumn{5}{l}{pohong} & \multicolumn{6}{l}{tagoyang,} & \multicolumn{3}{l}{e,} & \multicolumn{4}{l}{Ise,} & \multicolumn{2}{l}{Ise}\\
& \multicolumn{3}{l}{Matias} & \multicolumn{3}{l}{\textsc{3sg}} & \multicolumn{8}{l}{feel.afraid(.of)} & \multicolumn{5}{l}{tree} & \multicolumn{6}{l}{be.shaken} & \multicolumn{3}{l}{uh} & \multicolumn{4}{l}{Ise} & \multicolumn{2}{l}{Ise}\\
& \multicolumn{2}{l}{dia} & \multicolumn{8}{l}{takut} & \multicolumn{5}{l}{pohong} & \multicolumn{6}{l}{tagoyang,} & \multicolumn{3}{l}{jadi} & \multicolumn{3}{l}{de} & \multicolumn{7}{l}{menangis}\\
& \multicolumn{2}{l}{\textsc{3sg}} & \multicolumn{8}{l}{feel.afraid(.of)} & \multicolumn{5}{l}{tree} & \multicolumn{6}{l}{be.shaken} & \multicolumn{3}{l}{so} & \multicolumn{3}{l}{\textsc{3sg}} & \multicolumn{7}{l}{cry}\\
\lspbottomrule
\end{tabular}
\ea
\glt 
Natalia: you haven’t yet seen (the statue)?, ah, later on, maybe Matias or maybe tomorrow (you) walk (there) with me (‘aunt’), and then I (‘aunt’) will show, and then I’ll show (you) the tree which Matias was afraid of, the shaking tree, uh Ise, Ise was afraid of the shaking tree, so she cried
\z

\begin{tabular}{lllll}
\lsptoprule
0009 & Wili: & yang & dekat & ruma-sakit?\\
&  & \textsc{rel} & near & hospital\\
\lspbottomrule
\end{tabular}
\ea
\glt 
Wili: (the one) which is close to the hospital?
\z

\begin{tabular}{llll}
\lsptoprule
0010 & MC: & di & mana?\\
&  & at & where\\
\lspbottomrule
\end{tabular}
\ea
\glt 
MC: where?
\z

\begin{tabular}{llllllll}
\lsptoprule
0011 & Natalia: & di & dekat & ruma-sakit & sebla & laut & dulu\\
&  & at & near & hospital & side & sea & first\\
\lspbottomrule
\end{tabular}
\ea
\glt 
Natalia: (it) is close to the hospital toward the ocean, in the past
\z

\begin{tabular}{lllllll}
\lsptoprule
0012 & Wili: & yang & dekat & ada & ruma & to?\\
&  & \textsc{rel} & near & exist & house & right?\\
\lspbottomrule
\end{tabular}
\ea
\glt 
Wili: close by where the houses are, right?
\z

\begin{tabular}{llllllllll}
\lsptoprule
0013 & Natalia: & \multicolumn{2}{l}{mm-mm,} & ruma & di & pante, & jadi & luar & biasa\\
&  & \multicolumn{2}{l}{mhm} & house & at & coast & so & outside & be.usual\\
& \multicolumn{2}{l}{((pause))} & \multicolumn{7}{l}{sampe}\\
& \multicolumn{2}{l}{} & \multicolumn{7}{l}{until}\\
\lspbottomrule
\end{tabular}
\ea
\glt 
Natalia: mhm, the houses along the beach, so this has been magnificent\footnote{\\
\\
\\
\\
\\
\\
\\
\\
\\
\\
\\
\\
\\
\\
\\
\par This clause refers to the story about the statue and tree which the narrator had told before being asked for directions.} ((pause)) until
\z

\begin{tabular}{llll}
\lsptoprule
0014 & Wili: & skarang & ini\\
&  & now & \textsc{d.prox}\\
\lspbottomrule
\end{tabular}
\ea
\glt 
Wili: now!
\z

\begin{tabular}{llllllllllllllll}
\lsptoprule
0015 & Natalia: & \multicolumn{2}{l}{skarang,} & \multicolumn{2}{l}{say} & \multicolumn{2}{l}{kembali,} & \multicolumn{2}{l}{pulang} & \multicolumn{2}{l}{dari} & \multicolumn{2}{l}{skola,} & sa & di\\
&  & \multicolumn{2}{l}{now} & \multicolumn{2}{l}{\textsc{1sg}} & \multicolumn{2}{l}{return} & \multicolumn{2}{l}{go.home} & \multicolumn{2}{l}{from} & \multicolumn{2}{l}{school} & \textsc{1sg} & at\\
& \multicolumn{2}{l}{Pante-Timur,} & \multicolumn{2}{l}{Takar,} & \multicolumn{2}{l}{ke} & sini, & itu & \multicolumn{2}{l}{Tuhang} & \multicolumn{2}{l}{buat} & \multicolumn{3}{l}{luar}\\
& \multicolumn{2}{l}{Pante-Timur} & \multicolumn{2}{l}{Takar} & \multicolumn{2}{l}{to} & \textsc{l.prox} & \textsc{d.dist} & \multicolumn{2}{l}{God} & \multicolumn{2}{l}{make} & \multicolumn{3}{l}{outside}\\
& biasa & \multicolumn{14}{l}{itu}\\
& be.usual & \multicolumn{14}{l}{\textsc{d.dist}}\\
\lspbottomrule
\end{tabular}
\ea
\glt
Natalia: now (after this experience with the statue and the tree), I returned (to Jayapura), (I) went home (after having finished) college, (then) I (stayed) in Pante-Timur, (in) Takar, (then I came) here, all this, God made it wonderful
\end{styleFreeTranslEngxvpt}

\subsection{Expository: Sterility}

\begin{tabular}{ll}
\lsptoprule
File name: & 081006-030-CvEx\\
Text type: & Conversation with the author: Expository\\
Interlocutors: & 1 older female\\
Length (min.): & 1:56\\
\lspbottomrule
\end{tabular}
\begin{tabular}{lllllllllllllllllllllllllll}
\lsptoprule
0001 & \multicolumn{4}{l}{Natalia:} & jadi & \multicolumn{4}{l}{ada} & \multicolumn{2}{l}{dua,} & \multicolumn{2}{l}{kalo} & \multicolumn{7}{l}{misalnya,} & kita & \multicolumn{3}{l}{suda,} & \multicolumn{2}{l}{kitong}\\
& \multicolumn{4}{l}{} & so & \multicolumn{4}{l}{exist} & \multicolumn{2}{l}{two} & \multicolumn{2}{l}{if} & \multicolumn{7}{l}{for.example} & \textsc{1pl} & \multicolumn{3}{l}{already} & \multicolumn{2}{l}{\textsc{1pl}}\\
& \multicolumn{3}{l}{su} & \multicolumn{3}{l}{bayar} & \multicolumn{6}{l}{mas-kawing} & \multicolumn{3}{l}{to?,} & \multicolumn{4}{l}{baru} & \multicolumn{4}{l}{prempuang} & \multicolumn{2}{l}{itu} & de\\
& \multicolumn{3}{l}{already} & \multicolumn{3}{l}{pay} & \multicolumn{6}{l}{bride.price} & \multicolumn{3}{l}{right?} & \multicolumn{4}{l}{and.then} & \multicolumn{4}{l}{woman} & \multicolumn{2}{l}{\textsc{d.dist}} & \textsc{3sg}\\
& tra & \multicolumn{6}{l}{hamil,} & \multicolumn{3}{l}{na,} & \multicolumn{6}{l}{mungking} & \multicolumn{2}{l}{ada} & \multicolumn{3}{l}{masala} & \multicolumn{5}{l}{menge,}\\
& \textsc{neg} & \multicolumn{6}{l}{be.pregnant} & \multicolumn{3}{l}{well} & \multicolumn{6}{l}{maybe} & \multicolumn{2}{l}{exist} & \multicolumn{3}{l}{problem} & \multicolumn{5}{l}{\textsc{tru}{}-concern[SI]}\\
& \multicolumn{2}{l}{dari} & \multicolumn{6}{l}{kesehatang,} & \multicolumn{2}{l}{kita} & \multicolumn{4}{l}{bisa} & \multicolumn{3}{l}{liat} & \multicolumn{2}{l}{dari} & \multicolumn{3}{l}{kesehatang,} & \multicolumn{4}{l}{a}\\
& \multicolumn{2}{l}{from} & \multicolumn{6}{l}{health} & \multicolumn{2}{l}{\textsc{1pl}} & \multicolumn{4}{l}{be.able} & \multicolumn{3}{l}{see} & \multicolumn{2}{l}{from} & \multicolumn{3}{l}{health} & \multicolumn{4}{l}{ah!}\\
\lspbottomrule
\end{tabular}
\ea
\glt 
Natalia: so there’re two (issues related to sterility), if, for example, we’ve already, (if) we’ve already paid the bride price, right? and (if) that woman, (if) she doesn’t get pregnant, well, maybe there is a problem regarding[\textsc{tru}], due to health (issues), we can see (that the problem of sterility) is due to a health (problem), ah! (Lit. ‘from health’)
\z

\begin{tabular}{llllllllllllm{-2.4015456E-4in}lllll}
\lsptoprule
0002 & \multicolumn{3}{l}{bisa} & \multicolumn{2}{l}{juga} & \multicolumn{2}{l}{ada} & \multicolumn{2}{l}{pikirang} & \multicolumn{3}{l}{laki{\Tilde}laki} & itu & de & \multicolumn{3}{l}{mandul}\\
& \multicolumn{3}{l}{be.able} & \multicolumn{2}{l}{also} & \multicolumn{2}{l}{exist} & \multicolumn{2}{l}{thought} & \multicolumn{3}{l}{\textsc{rdp}{\Tilde}husband} & \textsc{d.dist} & \textsc{3sg} & \multicolumn{3}{l}{be.sterile}\\
& ato & \multicolumn{5}{l}{prempuang} & \multicolumn{2}{l}{itu} & \multicolumn{2}{l}{de} & mandul, & \multicolumn{4}{l}{makanya} & tida & ada\\
& or & \multicolumn{5}{l}{woman} & \multicolumn{2}{l}{\textsc{d.dist}} & \multicolumn{2}{l}{\textsc{3sg}} & be.sterile & \multicolumn{4}{l}{for.that.reason} & \textsc{neg} & exist\\
& \multicolumn{2}{l}{ana} & \multicolumn{2}{l}{sama} & \multicolumn{13}{l}{skali}\\
& \multicolumn{2}{l}{child} & \multicolumn{2}{l}{same} & \multicolumn{13}{l}{very}\\
\lspbottomrule
\end{tabular}
\ea
\glt 
it’s also possible that there is the thought, (that) that man, (that) he’s sterile or (that) that woman, (that) she’s sterile, for that reason there aren’t any children at all
\z

\begin{tabular}{lllllll}
\lsptoprule
0003 & a, & nanti & liat, & tinggal, & tinggal, & tinggal\\
& ah! & very.soon & see & stay & stay & stay\\
\lspbottomrule
\end{tabular}
\ea
\glt 
ah, later (we’ll) see, (we’ll) wait, wait, (and) wait
\z

\begin{tabular}{llllllllllllllllllm{-9.4015896E-4in}lllllllll}
\lsptoprule
0004 & \multicolumn{2}{l}{kalo} & \multicolumn{6}{l}{prempuang,} & \multicolumn{5}{l}{laki{\Tilde}laki} & \multicolumn{4}{l}{itu} & \multicolumn{2}{l}{dia} & \multicolumn{3}{l}{maw} & \multicolumn{4}{l}{turungang} & to?,\\
& \multicolumn{2}{l}{if} & \multicolumn{6}{l}{woman} & \multicolumn{5}{l}{\textsc{rdp}{\Tilde}husband} & \multicolumn{4}{l}{\textsc{d.dist}} & \multicolumn{2}{l}{\textsc{3sg}} & \multicolumn{3}{l}{want} & \multicolumn{4}{l}{descendant} & right?\\
& \multicolumn{3}{l}{dia} & \multicolumn{2}{l}{maw} & \multicolumn{4}{l}{ada} & \multicolumn{3}{l}{ana} & \multicolumn{2}{l}{lagi,} & \multicolumn{4}{l}{orang} & \multicolumn{2}{l}{Papua} & \multicolumn{3}{l}{punya} & \multicolumn{4}{l}{kebiasaang,}\\
& \multicolumn{3}{l}{\textsc{3sg}} & \multicolumn{2}{l}{want} & \multicolumn{4}{l}{exist} & \multicolumn{3}{l}{child} & \multicolumn{2}{l}{again} & \multicolumn{4}{l}{person} & \multicolumn{2}{l}{Papua} & \multicolumn{3}{l}{\textsc{poss}} & \multicolumn{4}{l}{habit}\\
& a, & \multicolumn{3}{l}{dia} & \multicolumn{2}{l}{e} & \multicolumn{7}{l}{kawing} & \multicolumn{5}{l}{ini} & \multicolumn{3}{l}{harus} & \multicolumn{2}{l}{ada} & ana & \multicolumn{3}{l}{karna}\\
& ah! & \multicolumn{3}{l}{\textsc{3sg}} & \multicolumn{2}{l}{uh} & \multicolumn{7}{l}{marry.unofficially} & \multicolumn{5}{l}{\textsc{d.prox}} & \multicolumn{3}{l}{have.to} & \multicolumn{2}{l}{exist} & child & \multicolumn{3}{l}{because}\\
& \multicolumn{2}{l}{dia} & \multicolumn{5}{l}{harus} & \multicolumn{3}{l}{ada} & \multicolumn{3}{l}{turungang,} & \multicolumn{2}{l}{a,} & \multicolumn{5}{l}{nanti} & \multicolumn{5}{l}{laki{\Tilde}laki} & \multicolumn{2}{l}{itu}\\
& \multicolumn{2}{l}{\textsc{3sg}} & \multicolumn{5}{l}{have.to} & \multicolumn{3}{l}{exist} & \multicolumn{3}{l}{descendant} & \multicolumn{2}{l}{ah!} & \multicolumn{5}{l}{very.soon} & \multicolumn{5}{l}{\textsc{rdp}{\Tilde}husband} & \multicolumn{2}{l}{\textsc{d.dist}}\\
& \multicolumn{3}{l}{dia} & \multicolumn{8}{l}{kawing} & \multicolumn{5}{l}{prempuang} & \multicolumn{11}{l}{laing}\\
& \multicolumn{3}{l}{\textsc{3sg}} & \multicolumn{8}{l}{marry.unofficially} & \multicolumn{5}{l}{woman} & \multicolumn{11}{l}{be.different}\\
\lspbottomrule
\end{tabular}
\ea
\glt 
if that woman (or) man, (if) he/she wants offsprings, right?, (if) he/she also wants to have children, the Papuan people’s habit, ah, (when) he/she here, what’s-its-name, marries (then) there have to be children because he/she has to have offsprings, ah, (otherwise) later that man, he’ll marry a different woman
\z

\begin{tabular}{lllllllllllllllllll}
\lsptoprule
0005 & a, & \multicolumn{2}{l}{de} & \multicolumn{5}{l}{kawing} & \multicolumn{4}{l}{prempuang} & \multicolumn{4}{l}{laing,} & \multicolumn{2}{l}{prempuang}\\
& ah! & \multicolumn{2}{l}{\textsc{3sg}} & \multicolumn{5}{l}{marry.unofficially} & \multicolumn{4}{l}{woman} & \multicolumn{4}{l}{be.different} & \multicolumn{2}{l}{woman}\\
& \multicolumn{2}{l}{itu} & \multicolumn{2}{l}{ada} & ana & \multicolumn{2}{l}{o,} & \multicolumn{3}{l}{kalo} & \multicolumn{2}{l}{begitu,} & \multicolumn{3}{l}{prempuang} & \multicolumn{2}{l}{ini} & yang\\
& \multicolumn{2}{l}{\textsc{d.dist}} & \multicolumn{2}{l}{exist} & child & \multicolumn{2}{l}{oh!} & \multicolumn{3}{l}{if} & \multicolumn{2}{l}{like.that} & \multicolumn{3}{l}{woman} & \multicolumn{2}{l}{\textsc{d.prox}} & \textsc{rel}\\
& \multicolumn{3}{l}{mandul,} & \multicolumn{3}{l}{prempuang} & \multicolumn{3}{l}{ini} & \multicolumn{2}{l}{tra} & \multicolumn{2}{l}{ada} & ana, & \multicolumn{4}{l}{begitu}\\
& \multicolumn{3}{l}{be.sterile} & \multicolumn{3}{l}{woman} & \multicolumn{3}{l}{\textsc{d.prox}} & \multicolumn{2}{l}{\textsc{neg}} & \multicolumn{2}{l}{exist} & child & \multicolumn{4}{l}{like.that}\\
\lspbottomrule
\end{tabular}
\ea
\glt 
ah, (when) he marries a different woman, (and when) that woman has children (we’ll know), ‘oh, in that case, (it’s) this (first) woman who’s sterile, this (first) woman doesn’t have children’, (it’s) like that
\z

\begin{tabular}{lllllllllllllllllllllll}
\lsptoprule
0006 & tapi & \multicolumn{3}{l}{kalo,} & \multicolumn{4}{l}{macang} & \multicolumn{5}{l}{prempuang} & \multicolumn{2}{l}{de} & \multicolumn{2}{l}{kasi} & \multicolumn{3}{l}{tinggal} & \multicolumn{2}{l}{laki{\Tilde}laki,}\\
& but & \multicolumn{3}{l}{if} & \multicolumn{4}{l}{variety} & \multicolumn{5}{l}{woman} & \multicolumn{2}{l}{\textsc{3sg}} & \multicolumn{2}{l}{give} & \multicolumn{3}{l}{stay} & \multicolumn{2}{l}{\textsc{rdp}{\Tilde}husband}\\
& \multicolumn{3}{l}{prempuang} & \multicolumn{3}{l}{de} & \multicolumn{7}{l}{kawing} & \multicolumn{2}{l}{deng} & \multicolumn{4}{l}{laki{\Tilde}laki} & \multicolumn{3}{l}{laing,}\\
& \multicolumn{3}{l}{woman} & \multicolumn{3}{l}{\textsc{3sg}} & \multicolumn{7}{l}{marry.unofficially} & \multicolumn{2}{l}{with} & \multicolumn{4}{l}{\textsc{rdp}{\Tilde}husband} & \multicolumn{3}{l}{be.different}\\
& \multicolumn{3}{l}{prempuang} & \multicolumn{4}{l}{itu} & \multicolumn{3}{l}{dapat} & \multicolumn{2}{l}{ana} & \multicolumn{2}{l}{o,} & \multicolumn{4}{l}{laki{\Tilde}laki} & \multicolumn{3}{l}{yang} & mandul,\\
& \multicolumn{3}{l}{woman} & \multicolumn{4}{l}{\textsc{d.dist}} & \multicolumn{3}{l}{get} & \multicolumn{2}{l}{child} & \multicolumn{2}{l}{oh!} & \multicolumn{4}{l}{\textsc{rdp}{\Tilde}husband} & \multicolumn{3}{l}{\textsc{rel}} & be.sterile\\
& \multicolumn{2}{l}{kalo} & \multicolumn{3}{l}{itu} & \multicolumn{4}{l}{memang,} & \multicolumn{2}{l}{e,} & \multicolumn{3}{l}{diliat} & \multicolumn{2}{l}{dari} & \multicolumn{6}{l}{kesehatang}\\
& \multicolumn{2}{l}{if} & \multicolumn{3}{l}{\textsc{d.dist}} & \multicolumn{4}{l}{indeed} & \multicolumn{2}{l}{uh} & \multicolumn{3}{l}{\textsc{uv}{}-see} & \multicolumn{2}{l}{from} & \multicolumn{6}{l}{health}\\
\lspbottomrule
\end{tabular}
\ea
\glt 
but if, for example, the woman leaves (her) husband (and if) the woman marries a different man (and if) that woman has children (we’ll know), ‘oh, (it’s) the (first) man who’s sterile’, if it’s like that indeed, umh, (the issue of sterility) is due to a health (problem)
\z

\begin{tabular}{lll}
\lsptoprule
0007 & Author: & yo\\
&  & yes\\
\lspbottomrule
\end{tabular}
\ea
\glt 
Author: yes
\z

\begin{tabular}{llllllllllll}
\lsptoprule
0008 & \multicolumn{2}{l}{Natalia:} & \multicolumn{3}{l}{begitu,} & tapi & kalo & kita & suda & bayar & mas-kawing,\\
& \multicolumn{2}{l}{} & \multicolumn{3}{l}{like.that} & but & if & \textsc{1pl} & already & pay & bride.price\\
& kalo & \multicolumn{2}{l}{kita} & pikir & \multicolumn{7}{l}{to?}\\
& if & \multicolumn{2}{l}{\textsc{1pl}} & think & \multicolumn{7}{l}{right?}\\
\lspbottomrule
\end{tabular}
\ea
\glt 
Natalia: (it’s) like that, but if we’ve already paid the bride price, if we think, right?
\z

\begin{tabular}{llllllllllm{-9.4015896E-4in}lllllllllllm{-2.4015456E-4in}ll}
\lsptoprule
0009 & o, & \multicolumn{3}{l}{mungking} & \multicolumn{3}{l}{kitong} & \multicolumn{4}{l}{blum} & \multicolumn{4}{l}{bayar} & \multicolumn{5}{l}{mas-kawing,} & de & \multicolumn{3}{l}{tra}\\
& oh! & \multicolumn{3}{l}{maybe} & \multicolumn{3}{l}{\textsc{1pl}} & \multicolumn{4}{l}{not.yet} & \multicolumn{4}{l}{pay} & \multicolumn{5}{l}{bride.price} & \textsc{3sg} & \multicolumn{3}{l}{\textsc{neg}}\\
& \multicolumn{3}{l}{hamil,} & \multicolumn{3}{l}{baru} & \multicolumn{4}{l}{kitong} & \multicolumn{3}{l}{bayar} & \multicolumn{5}{l}{mas-kawing,} & \multicolumn{4}{l}{tinggal,} & \multicolumn{2}{l}{tinggal,}\\
& \multicolumn{3}{l}{be.pregnant} & \multicolumn{3}{l}{and.then} & \multicolumn{4}{l}{\textsc{1pl}} & \multicolumn{3}{l}{pay} & \multicolumn{5}{l}{bride.price} & \multicolumn{4}{l}{stay} & \multicolumn{2}{l}{stay}\\
& \multicolumn{2}{l}{tinggal,} & \multicolumn{3}{l}{tinggal,} & \multicolumn{4}{l}{bereskang} & \multicolumn{3}{l}{smua} & \multicolumn{5}{l}{masala} & \multicolumn{2}{l}{apa,} & \multicolumn{4}{l}{prempuang} & tra\\
& \multicolumn{2}{l}{stay} & \multicolumn{3}{l}{stay} & \multicolumn{4}{l}{clean.up} & \multicolumn{3}{l}{all} & \multicolumn{5}{l}{problem} & \multicolumn{2}{l}{what} & \multicolumn{4}{l}{woman} & \textsc{neg}\\
& \multicolumn{3}{l}{hamil,} & \multicolumn{2}{l}{o,} & \multicolumn{3}{l}{ini} & \multicolumn{6}{l}{prempuang} & \multicolumn{2}{l}{de} & \multicolumn{8}{l}{mandul}\\
& \multicolumn{3}{l}{be.pregnant} & \multicolumn{2}{l}{oh!} & \multicolumn{3}{l}{\textsc{d.prox}} & \multicolumn{6}{l}{woman} & \multicolumn{2}{l}{\textsc{3sg}} & \multicolumn{8}{l}{be.sterile}\\
\lspbottomrule
\end{tabular}
\ea
\glt 
‘oh, maybe we haven’t yet paid the bride price, (and that’s the reason why) she’s not pregnant’, but then we pay the bride price, (and we) wait, wait, wait, (and) wait, (we) settle all problems what(ever they may be, and) the woman is (still) not pregnant, (then we’ll know,) ‘oh, this is because the woman is sterile’
\z

\begin{tabular}{llllllllllllllllllllllllllll}
\lsptoprule
0010 & kalo & \multicolumn{4}{l}{orang} & \multicolumn{4}{l}{yang} & \multicolumn{3}{l}{blum} & \multicolumn{4}{l}{bertobat,} & \multicolumn{3}{l}{bukang} & \multicolumn{3}{l}{hamba} & \multicolumn{3}{l}{Tuhang,} & \multicolumn{2}{l}{dia}\\
& if & \multicolumn{4}{l}{person} & \multicolumn{4}{l}{\textsc{rel}} & \multicolumn{3}{l}{not.yet} & \multicolumn{4}{l}{repent} & \multicolumn{3}{l}{\textsc{neg}} & \multicolumn{3}{l}{servant} & \multicolumn{3}{l}{God} & \multicolumn{2}{l}{\textsc{3{\textasciigrave}sg}}\\
& \multicolumn{7}{l}{kawing} & \multicolumn{4}{l}{satu} & lagi, & \multicolumn{2}{l}{de} & \multicolumn{7}{l}{kawing} & \multicolumn{3}{l}{satu,} & \multicolumn{3}{l}{prempuang}\\
& \multicolumn{7}{l}{marry.unofficially} & \multicolumn{4}{l}{one} & again & \multicolumn{2}{l}{\textsc{3sg}} & \multicolumn{7}{l}{marry.unofficially} & \multicolumn{3}{l}{one} & \multicolumn{3}{l}{woman}\\
& \multicolumn{2}{l}{itu} & \multicolumn{2}{l}{ada} & \multicolumn{4}{l}{ana,} & \multicolumn{7}{l}{kawing} & \multicolumn{3}{l}{satu,} & \multicolumn{5}{l}{prempuang} & \multicolumn{3}{l}{ada} & ana,\\
& \multicolumn{2}{l}{\textsc{d.dist}} & \multicolumn{2}{l}{exist} & \multicolumn{4}{l}{child} & \multicolumn{7}{l}{marry.unofficially} & \multicolumn{3}{l}{one} & \multicolumn{5}{l}{woman} & \multicolumn{3}{l}{exist} & child\\
& \multicolumn{3}{l}{baru} & \multicolumn{3}{l}{o,} & \multicolumn{4}{l}{kalo} & \multicolumn{3}{l}{begitu} & \multicolumn{4}{l}{prempuang} & \multicolumn{3}{l}{ini} & \multicolumn{7}{l}{mandul}\\
& \multicolumn{3}{l}{and.then} & \multicolumn{3}{l}{oh!} & \multicolumn{4}{l}{if} & \multicolumn{3}{l}{like.that} & \multicolumn{4}{l}{woman} & \multicolumn{3}{l}{\textsc{d.prox}} & \multicolumn{7}{l}{be.sterile}\\
\lspbottomrule
\end{tabular}
\ea
\glt 
if someone isn’t a Christian yet (and) is not a servant of God, (if) he marries another woman, (if) he marries another (woman and) that woman has children, (if) he marries another (woman and) the woman has children, then (we’ll know), ‘oh, if it’s like that, (then) this (first) woman is sterile’ (Lit. ‘if someone hasn’t yet repented’)
\z

\begin{tabular}{lllllllllllllllllllll}
\lsptoprule
0011 & de & \multicolumn{3}{l}{tida,} & \multicolumn{3}{l}{orang} & \multicolumn{2}{l}{Papua} & \multicolumn{2}{l}{bilang} & \multicolumn{2}{l}{[Is]} & \multicolumn{3}{l}{makanya} & \multicolumn{3}{l}{orang} & itu\\
& \textsc{3sg} & \multicolumn{3}{l}{\textsc{neg}} & \multicolumn{3}{l}{person} & \multicolumn{2}{l}{Papua} & \multicolumn{2}{l}{say} & \multicolumn{2}{l}{} & \multicolumn{3}{l}{for.that.reason} & \multicolumn{3}{l}{person} & \textsc{d.dist}\\
& \multicolumn{3}{l}{tida} & \multicolumn{2}{l}{ada} & \multicolumn{3}{l}{ana,} & \multicolumn{2}{l}{mandul,} & \multicolumn{2}{l}{jadi} & \multicolumn{2}{l}{de} & pu, & \multicolumn{2}{l}{tida} & ada & \multicolumn{2}{l}{ana}\\
& \multicolumn{3}{l}{\textsc{neg}} & \multicolumn{2}{l}{exist} & \multicolumn{3}{l}{child} & \multicolumn{2}{l}{be.sterile} & \multicolumn{2}{l}{so} & \multicolumn{2}{l}{\textsc{3sg}} & \textsc{poss} & \multicolumn{2}{l}{\textsc{neg}} & exist & \multicolumn{2}{l}{child}\\
& \multicolumn{2}{l}{jadi} & \multicolumn{4}{l}{mandul,} & \multicolumn{14}{l}{begitu}\\
& \multicolumn{2}{l}{so} & \multicolumn{4}{l}{be.sterile} & \multicolumn{14}{l}{like.that}\\
\lspbottomrule
\end{tabular}
\ea
\glt
he/she doesn’t, the Papuan people say ‘[Is]’, that is to say, that person doesn’t have children, (he/she’s) sterile, so, his/her, (he/she) doesn’t have children, so (he/she’s) sterile, (it’s) like that
\end{styleFreeTranslEngxvpt}

\subsection{Hortatory: Don’t get dirty!}

\begin{tabular}{ll}
\lsptoprule
File name: & 080917-004-CvHt\\
Text type: & Conversation, spontaneous: Hortatory\\
Interlocutors: & 2 male children,\footnotemark{} 1 older female\\
Length (min.): & 0:10\\
\lspbottomrule
\end{tabular}
\footnotetext{\\
\\
\\
\\
\\
\\
\\
\\
\\
\\
\\
\\
\\
\\
\\
The second male child did not participate in this exchange.}

\begin{tabular}{llllllllllllll}
\lsptoprule
0001 & \multicolumn{2}{l}{Wili:} & \multicolumn{2}{l}{Nofi} & \multicolumn{2}{l}{nanti} & ko & kejar & saya, & ko & liat, & ko & tunggu,\\
& \multicolumn{2}{l}{} & \multicolumn{2}{l}{Nofi} & \multicolumn{2}{l}{very.soon} & \textsc{2sg} & chase & \textsc{1sg} & \textsc{2sg} & see & \textsc{2sg} & wait\\
& tong & \multicolumn{2}{l}{dua} & \multicolumn{2}{l}{bla,} & \multicolumn{8}{l}{baru}\\
& \textsc{1pl} & \multicolumn{2}{l}{two} & \multicolumn{2}{l}{split} & \multicolumn{8}{l}{and.then}\\
\lspbottomrule
\end{tabular}
\ea
\glt 
Wili: Nofi, in a moment you chase (me down to the water), you observe (me), you wait, we two crack (the coconut) open, and then
\z

\begin{tabular}{lllllllllllllll}
\lsptoprule
0002 & Nofita: & \multicolumn{2}{l}{tida} & \multicolumn{2}{l}{usa,} & \multicolumn{3}{l}{kotor} & \multicolumn{2}{l}{dang} & \multicolumn{2}{l}{ko} & nanti & kena\\
&  & \multicolumn{2}{l}{\textsc{neg}} & \multicolumn{2}{l}{need.to} & \multicolumn{3}{l}{be.dirty} & \multicolumn{2}{l}{and} & \multicolumn{2}{l}{\textsc{2sg}} & very.soon & hit\\
& \multicolumn{2}{l}{picaang,} & \multicolumn{2}{l}{kam} & \multicolumn{2}{l}{dengar} & ato & \multicolumn{2}{l}{tida,} & \multicolumn{2}{l}{terlalu} & \multicolumn{3}{l}{nakal}\\
& \multicolumn{2}{l}{splinter} & \multicolumn{2}{l}{\textsc{2pl}} & \multicolumn{2}{l}{hear} & or & \multicolumn{2}{l}{\textsc{neg}} & \multicolumn{2}{l}{too} & \multicolumn{3}{l}{be.mischievous}\\
\lspbottomrule
\end{tabular}
\ea
\glt
Nofita: don’t (go down to the beach, it’s) dirty, and later you’ll run into broken glass and cans, are you listening or not?!, (you’re) too naughty!
\end{styleFreeTranslEngxvpt}

\subsection{Hortatory: Bathe in the ocean!}

\begin{tabular}{ll}
\lsptoprule
File name: & 080917-006-CvHt\\
Text type: & Conversation, spontaneous: Hortatory\\
Interlocutors: & 3 male children,\footnotemark{} 2 older females\\
Length (min.): & 1:00\\
\lspbottomrule
\end{tabular}
\footnotetext{\\
\\
\\
\\
\\
\\
\\
\\
\\
\\
\\
\\
\\
\\
\\
The third male child did not participate in this exchange.}

\begin{tabular}{llllllllllll}
\lsptoprule
0001 & \multicolumn{2}{l}{Nofita:} & kepala & \multicolumn{2}{l}{sakit} & sa & tra & \multicolumn{2}{l}{bisa} & bicara & banyak,\\
& \multicolumn{2}{l}{} & head & \multicolumn{2}{l}{be.sick} & \textsc{1sg} & \textsc{neg} & \multicolumn{2}{l}{be.able} & speak & many\\
& kam & \multicolumn{3}{l}{dengar{\Tilde}dengarang,} & \multicolumn{2}{l}{kam} & \multicolumn{2}{l}{cari} & \multicolumn{3}{l}{apa?}\\
& \textsc{2pl} & \multicolumn{3}{l}{\textsc{rdp}{\Tilde}hear\textsc{:pat}} & \multicolumn{2}{l}{\textsc{2pl}} & \multicolumn{2}{l}{search} & \multicolumn{3}{l}{what}\\
\lspbottomrule
\end{tabular}
\ea
\glt 
Nofita: (I have) a headache, I can’t talk much, you listen to me! what are you looking for?
\z

\begin{tabular}{lllllllll}
\lsptoprule
0002 & Wili: & a, & jangang, & Nofi & mana & kitong & pu & ikang{\Tilde}ikang\\
&  & ah! & \textsc{neg.imp} & Nofi & where & \textsc{1pl} & \textsc{poss} & \textsc{rdp}{\Tilde}fish\\
\lspbottomrule
\end{tabular}
\ea
\glt 
Wili: ah, don’t! Nofi, where are our fish?
\z

\begin{tabular}{llllllll}
\lsptoprule
0003 & Nofi: & sa & su & taru & di & ember & sini\\
&  & \textsc{1sg} & already & put & at & bucket & \textsc{l.prox}\\
\lspbottomrule
\end{tabular}
\ea
\glt 
Nofi: I already put (the fish) in the bucket here
\z

\begin{tabular}{llllllllll}
\lsptoprule
0004 & Nofita: & kam & dua & pi & spul & badang & di & laut & sana\\
&  & \textsc{2pl} & two & go & rinse & body & at & sea & \textsc{l.dist}\\
\lspbottomrule
\end{tabular}
\ea
\glt 
Nofita: you two go rinse (your) bodies in the ocean over there
\z

\begin{tabular}{llll}
\lsptoprule
0005 & Nofi: & ada & ni\\
&  & exist & \textsc{d.prox}\\
\lspbottomrule
\end{tabular}
\ea
\glt 
Nofi: (the fish) are here
\z

\begin{tabular}{llllllllllllllllllllllllll}
\lsptoprule
0006 & \multicolumn{3}{l}{Nofita:} & \multicolumn{4}{l}{spul} & \multicolumn{3}{l}{badang,} & \multicolumn{3}{l}{trus} & \multicolumn{3}{l}{celana} & \multicolumn{3}{l}{cuci} & \multicolumn{2}{l}{di} & \multicolumn{2}{l}{laut,} & \multicolumn{2}{l}{baru}\\
& \multicolumn{3}{l}{} & \multicolumn{4}{l}{rinse} & \multicolumn{3}{l}{body} & \multicolumn{3}{l}{next} & \multicolumn{3}{l}{trousers} & \multicolumn{3}{l}{wash} & \multicolumn{2}{l}{at} & \multicolumn{2}{l}{sea} & \multicolumn{2}{l}{and.then}\\
& pake & \multicolumn{4}{l}{ke} & \multicolumn{4}{l}{mari,} & \multicolumn{3}{l}{biking} & \multicolumn{3}{l}{kotor} & \multicolumn{2}{l}{saja,} & \multicolumn{3}{l}{saya} & \multicolumn{3}{l}{stenga} & mati & cuci,\\
& use & \multicolumn{4}{l}{to} & \multicolumn{4}{l}{hither} & \multicolumn{3}{l}{make} & \multicolumn{3}{l}{be.dirty} & \multicolumn{2}{l}{just} & \multicolumn{3}{l}{\textsc{1sg}} & \multicolumn{3}{l}{half} & die & wash\\
& \multicolumn{4}{l}{cape} & \multicolumn{4}{l}{cuci} & \multicolumn{3}{l}{pakeang} & \multicolumn{3}{l}{juga,} & \multicolumn{4}{l}{ana{\Tilde}ana} & \multicolumn{4}{l}{ini} & \multicolumn{3}{l}{kotor{\Tilde}kotor,}\\
& \multicolumn{4}{l}{be.tired} & \multicolumn{4}{l}{wash} & \multicolumn{3}{l}{clothes} & \multicolumn{3}{l}{also} & \multicolumn{4}{l}{\textsc{rdp}{\Tilde}child} & \multicolumn{4}{l}{\textsc{d.prox}} & \multicolumn{3}{l}{\textsc{rdp}{\Tilde}be.dirty}\\
& \multicolumn{2}{l}{dong} & \multicolumn{4}{l}{[\textsc{up}]} & \multicolumn{19}{l}{adu}\\
& \multicolumn{2}{l}{\textsc{3pl}} & \multicolumn{4}{l}{} & \multicolumn{19}{l}{oh.no!}\\
\lspbottomrule
\end{tabular}
\ea
\glt 
Nofita: rinse (your) bodies, then wash (your) trousers in the ocean, and then put them on (and) come here, (they) make (all their clothes) dirty, I’m half dead (from) washing, I’m also tired of washing clothes, these kids make (their trousers) dirty, they [\textsc{up}] oh no!
\z

\begin{tabular}{llllllllllllllllllllllll}
\lsptoprule
0007 & ey, & \multicolumn{2}{l}{kam} & \multicolumn{3}{l}{dua} & \multicolumn{2}{l}{pi} & \multicolumn{4}{l}{mandi} & di & \multicolumn{2}{l}{laut} & \multicolumn{3}{l}{suda,} & \multicolumn{2}{l}{trus} & \multicolumn{2}{l}{kam} & dua\\
& hey! & \multicolumn{2}{l}{\textsc{2pl}} & \multicolumn{3}{l}{two} & \multicolumn{2}{l}{go} & \multicolumn{4}{l}{bathe} & at & \multicolumn{2}{l}{sea} & \multicolumn{3}{l}{already} & \multicolumn{2}{l}{next} & \multicolumn{2}{l}{\textsc{2pl}} & two\\
& \multicolumn{2}{l}{cuci} & \multicolumn{3}{l}{celana} & \multicolumn{2}{l}{di} & \multicolumn{3}{l}{situ,} & \multicolumn{4}{l}{baru} & \multicolumn{3}{l}{pake,} & \multicolumn{2}{l}{naik,} & \multicolumn{2}{l}{tra} & \multicolumn{2}{l}{usa}\\
& \multicolumn{2}{l}{wash} & \multicolumn{3}{l}{trousers} & \multicolumn{2}{l}{at} & \multicolumn{3}{l}{\textsc{l.med}} & \multicolumn{4}{l}{and.then} & \multicolumn{3}{l}{use} & \multicolumn{2}{l}{ascend} & \multicolumn{2}{l}{\textsc{neg}} & \multicolumn{2}{l}{need.to}\\
& \multicolumn{4}{l}{loncat{\Tilde}loncat,} & \multicolumn{5}{l}{situ} & \multicolumn{2}{l}{ada} & \multicolumn{5}{l}{besi{\Tilde}besi} & \multicolumn{7}{l}{banyak}\\
& \multicolumn{4}{l}{\textsc{rdp}{\Tilde}jump} & \multicolumn{5}{l}{\textsc{l.med}} & \multicolumn{2}{l}{exist} & \multicolumn{5}{l}{\textsc{rdp}{\Tilde}metal} & \multicolumn{7}{l}{many}\\
\lspbottomrule
\end{tabular}
\ea
\glt 
hey, you two go bathe in the ocean already!, and then you two wash (your) trousers there, after that put (them) on (and) come up (to the house), don’t jump up and down, there are lots of metal pieces over there
\z

\begin{tabular}{llllll}
\lsptoprule
0008 & Anelia: & mm-mm, & picaang & juga & banyak\\
&  & mhm & splinter & also & many\\
\lspbottomrule
\end{tabular}
\ea
\glt 
Anelia: mhm, (at the beach there) are also lots of broken glass and cans (Lit. ‘the splinters are also many’)
\z

\begin{tabular}{llll}
\lsptoprule
0009 & Nofita: & picaang & banyak\\
&  & splinter & many\\
\lspbottomrule
\end{tabular}
\ea
\glt
Nofita: (there) are lots of broken glass and cans
\end{styleFreeTranslEngxvpt}

\subsection{Joke: Drawing a monkey}

\begin{tabular}{ll}
\lsptoprule
File name: & 081109-002-JR\\
Text type: & Joke (Elicited text)\\
Interlocutors: & 2 younger males\\
Length (min.): & 0:59\\
\lspbottomrule
\end{tabular}
\begin{tabular}{lllllllllllllllllllllll}
\lsptoprule
0001 & \multicolumn{2}{l}{skola} & \multicolumn{3}{l}{ini} & \multicolumn{4}{l}{ibu} & \multicolumn{3}{l}{mulay} & \multicolumn{3}{l}{suru} & \multicolumn{2}{l}{ana{\Tilde}ana} & \multicolumn{3}{l}{murit} & \multicolumn{2}{l}{mulay}\\
& \multicolumn{2}{l}{school} & \multicolumn{3}{l}{\textsc{d.prox}} & \multicolumn{4}{l}{woman} & \multicolumn{3}{l}{start} & \multicolumn{3}{l}{order} & \multicolumn{2}{l}{\textsc{rdp}{\Tilde}child} & \multicolumn{3}{l}{pupil} & \multicolumn{2}{l}{start}\\
& \multicolumn{3}{l}{gambar} & \multicolumn{3}{l}{monyet} & di & \multicolumn{3}{l}{atas} & \multicolumn{3}{l}{pohong} & \multicolumn{3}{l}{pisang,} & \multicolumn{3}{l}{suda,} & \multicolumn{3}{l}{ibu}\\
& \multicolumn{3}{l}{draw} & \multicolumn{3}{l}{monkey} & at & \multicolumn{3}{l}{top} & \multicolumn{3}{l}{tree} & \multicolumn{3}{l}{banana} & \multicolumn{3}{l}{already} & \multicolumn{3}{l}{woman}\\
& mulay & \multicolumn{3}{l}{suru} & \multicolumn{4}{l}{gambar,} & \multicolumn{3}{l}{suda} & \multicolumn{3}{l}{dong} & \multicolumn{2}{l}{mulay,} & \multicolumn{2}{l}{smua} & \multicolumn{3}{l}{dong} & gambar\\
& start & \multicolumn{3}{l}{order} & \multicolumn{4}{l}{draw} & \multicolumn{3}{l}{already} & \multicolumn{3}{l}{\textsc{3pl}} & \multicolumn{2}{l}{start} & \multicolumn{2}{l}{all} & \multicolumn{3}{l}{\textsc{3pl}} & draw\\
\lspbottomrule
\end{tabular}
\ea
\glt 
(in) this school, Ms. (Teacher) starts ordering the school kids to start drawing a monkey on a banana tree, well, Ms. Teacher orders (them to) draw, well, they start, they all draw (a picture)
\z

\begin{tabular}{lllllllllllllllllllllllllllll}
\lsptoprule
0002 & \multicolumn{3}{l}{baru} & \multicolumn{3}{l}{ana} & \multicolumn{3}{l}{kecil} & \multicolumn{3}{l}{satu} & \multicolumn{5}{l}{ini} & \multicolumn{4}{l}{de} & \multicolumn{2}{l}{tra} & \multicolumn{4}{l}{gambar,} & ana\\
& \multicolumn{3}{l}{and.then} & \multicolumn{3}{l}{child} & \multicolumn{3}{l}{be.small} & \multicolumn{3}{l}{one} & \multicolumn{5}{l}{\textsc{d.prox}} & \multicolumn{4}{l}{\textsc{3sg}} & \multicolumn{2}{l}{\textsc{neg}} & \multicolumn{4}{l}{draw} & child\\
& murit & \multicolumn{3}{l}{satu} & \multicolumn{4}{l}{ni} & \multicolumn{3}{l}{de} & \multicolumn{2}{l}{tra} & \multicolumn{5}{l}{gambar,} & \multicolumn{4}{l}{suda,} & \multicolumn{4}{l}{begini} & \multicolumn{2}{l}{de}\\
& pupil & \multicolumn{3}{l}{one} & \multicolumn{4}{l}{\textsc{d.prox}} & \multicolumn{3}{l}{\textsc{3sg}} & \multicolumn{2}{l}{\textsc{neg}} & \multicolumn{5}{l}{draw} & \multicolumn{4}{l}{already} & \multicolumn{4}{l}{like.this} & \multicolumn{2}{l}{\textsc{3sg}}\\
& \multicolumn{2}{l}{gambar} & \multicolumn{3}{l}{batu,} & \multicolumn{3}{l}{trus} & \multicolumn{3}{l}{de} & \multicolumn{4}{l}{gambar} & \multicolumn{5}{l}{monyet} & \multicolumn{4}{l}{ini} & di & \multicolumn{3}{l}{bawa}\\
& \multicolumn{2}{l}{draw} & \multicolumn{3}{l}{stone} & \multicolumn{3}{l}{next} & \multicolumn{3}{l}{\textsc{3sg}} & \multicolumn{4}{l}{draw} & \multicolumn{5}{l}{monkey} & \multicolumn{4}{l}{\textsc{d.prox}} & at & \multicolumn{3}{l}{bottom}\\
& \multicolumn{2}{l}{pohong} & \multicolumn{5}{l}{pisang,} & \multicolumn{3}{l}{begini} & \multicolumn{4}{l}{dong} & \multicolumn{2}{l}{bawa} & \multicolumn{3}{l}{ke} & \multicolumn{9}{l}{depang}\\
& \multicolumn{2}{l}{tree} & \multicolumn{5}{l}{banana} & \multicolumn{3}{l}{like.this} & \multicolumn{4}{l}{\textsc{3pl}} & \multicolumn{2}{l}{bring} & \multicolumn{3}{l}{to} & \multicolumn{9}{l}{front}\\
\lspbottomrule
\end{tabular}
\ea
\glt 
but then this particular small child, he doesn’t draw, this particular school kid, he doesn’t draw, well, he draws a stone (instead), and then he draws this monkey under the banana tree, it goes on like this (and) they bring (their drawings) to the front
\z

\begin{tabular}{llllllllllllllllllllllllllllll}
\lsptoprule
0003 & \multicolumn{3}{l}{ibu} & \multicolumn{5}{l}{bilang,} & \multicolumn{3}{l}{ibu} & \multicolumn{4}{l}{kalo} & \multicolumn{3}{l}{toki} & \multicolumn{4}{l}{meja} & \multicolumn{6}{l}{langsung} & kumpul\\
& \multicolumn{3}{l}{woman} & \multicolumn{5}{l}{say} & \multicolumn{3}{l}{woman} & \multicolumn{4}{l}{if} & \multicolumn{3}{l}{beat} & \multicolumn{4}{l}{table} & \multicolumn{6}{l}{immediately} & gather\\
& ke & \multicolumn{5}{l}{depang,} & \multicolumn{4}{l}{suda} & \multicolumn{3}{l}{pace} & \multicolumn{3}{l}{de} & \multicolumn{7}{l}{pikir{\Tilde}pikir} & \multicolumn{3}{l}{sampe} & \multicolumn{3}{l}{tra}\\
& to & \multicolumn{5}{l}{front} & \multicolumn{4}{l}{already} & \multicolumn{3}{l}{man} & \multicolumn{3}{l}{\textsc{3sg}} & \multicolumn{7}{l}{\textsc{rdp}{\Tilde}think} & \multicolumn{3}{l}{until} & \multicolumn{3}{l}{\textsc{neg}}\\
& \multicolumn{4}{l}{jadi,} & \multicolumn{5}{l}{suda} & \multicolumn{3}{l}{begini} & \multicolumn{8}{l}{langsung} & \multicolumn{6}{l}{i} & \multicolumn{3}{l}{ibu}\\
& \multicolumn{4}{l}{become} & \multicolumn{5}{l}{already} & \multicolumn{3}{l}{like.this} & \multicolumn{8}{l}{immediately} & \multicolumn{6}{l}{\textsc{tru}{}-woman} & \multicolumn{3}{l}{woman}\\
& \multicolumn{2}{l}{bagi} & \multicolumn{5}{l}{meja,} & \multicolumn{3}{l}{pak!,} & \multicolumn{7}{l}{langsung} & \multicolumn{4}{l}{pace} & \multicolumn{4}{l}{gambar} & \multicolumn{2}{l}{[\textsc{up}]} & \multicolumn{2}{l}{itu,}\\
& \multicolumn{2}{l}{divide} & \multicolumn{5}{l}{table} & \multicolumn{3}{l}{bang!} & \multicolumn{7}{l}{immediately} & \multicolumn{4}{l}{man} & \multicolumn{4}{l}{draw} & \multicolumn{2}{l}{} & \multicolumn{2}{l}{\textsc{d.dist}}\\
& \multicolumn{5}{l}{monyet} & di & \multicolumn{4}{l}{bawa} & \multicolumn{4}{l}{pohong} & \multicolumn{5}{l}{pisang,} & \multicolumn{5}{l}{bawa} & ke & \multicolumn{4}{l}{sana}\\
& \multicolumn{5}{l}{monkey} & at & \multicolumn{4}{l}{bottom} & \multicolumn{4}{l}{tree} & \multicolumn{5}{l}{banana} & \multicolumn{5}{l}{bring} & to & \multicolumn{4}{l}{\textsc{l.dist}}\\
\lspbottomrule
\end{tabular}
\ea
\glt 
Ms. (Teacher) says, ‘when I (‘Ms.’) knock (on) the table, (you) bring (your pictures) together to the front immediately’, then the guy thinks on and on (but) nothing happens, as it goes on like this immediately Ms.[\textsc{tru}], Ms. (Teacher) hits the table, ‘bang!’, immediately the guy draws [\textsc{up}], what’s-its-name, a monkey under a banana tree (and) brings it to the front
\z

\begin{tabular}{lllm{-9.4015896E-4in}llllllllllllllllllllllll}
\lsptoprule
0004 & \multicolumn{3}{l}{ibu} & \multicolumn{3}{l}{bilang,} & \multicolumn{2}{l}{e,} & \multicolumn{5}{l}{ibu} & \multicolumn{4}{l}{priksa} & \multicolumn{4}{l}{selesay,} & \multicolumn{4}{l}{ibu} & \multicolumn{2}{l}{tanya,}\\
& \multicolumn{3}{l}{woman} & \multicolumn{3}{l}{say} & \multicolumn{2}{l}{uh} & \multicolumn{5}{l}{woman} & \multicolumn{4}{l}{check} & \multicolumn{4}{l}{finish} & \multicolumn{4}{l}{woman} & \multicolumn{2}{l}{ask}\\
& \multicolumn{3}{l}{ini} & \multicolumn{2}{l}{siapa} & \multicolumn{5}{l}{punya?} & \multicolumn{2}{l}{de} & \multicolumn{5}{l}{bilang,} & \multicolumn{3}{l}{ibu,} & \multicolumn{3}{l}{sa} & \multicolumn{4}{l}{punya,}\\
& \multicolumn{3}{l}{\textsc{d.prox}} & \multicolumn{2}{l}{who} & \multicolumn{5}{l}{\textsc{poss}} & \multicolumn{2}{l}{\textsc{3sg}} & \multicolumn{5}{l}{say} & \multicolumn{3}{l}{woman} & \multicolumn{3}{l}{\textsc{1sg}} & \multicolumn{4}{l}{\textsc{poss}}\\
& de & \multicolumn{3}{l}{tanya,} & \multicolumn{3}{l}{pace} & \multicolumn{4}{l}{maju} & \multicolumn{4}{l}{ke} & \multicolumn{3}{l}{sana,} & \multicolumn{4}{l}{ibu} & \multicolumn{4}{l}{tanya} & dia,\\
& \textsc{3sg} & \multicolumn{3}{l}{ask} & \multicolumn{3}{l}{man} & \multicolumn{4}{l}{advance} & \multicolumn{4}{l}{to} & \multicolumn{3}{l}{\textsc{l.dist}} & \multicolumn{4}{l}{woman} & \multicolumn{4}{l}{ask} & \textsc{3sg}\\
& \multicolumn{2}{l}{knapa} & \multicolumn{2}{l}{ko} & \multicolumn{5}{l}{gambar} & \multicolumn{5}{l}{monyet} & \multicolumn{2}{l}{di} & \multicolumn{3}{l}{bawa} & \multicolumn{5}{l}{pohong} & \multicolumn{3}{l}{pisang?}\\
& \multicolumn{2}{l}{why} & \multicolumn{2}{l}{\textsc{2sg}} & \multicolumn{5}{l}{draw} & \multicolumn{5}{l}{monkey} & \multicolumn{2}{l}{at} & \multicolumn{3}{l}{bottom} & \multicolumn{5}{l}{tree} & \multicolumn{3}{l}{banana}\\
\lspbottomrule
\end{tabular}
\ea
\glt 
Ms. (Teacher) says, uh, after Ms. (Teacher) has finished checking (the pictures), Ms. (Teacher) asks (them), ‘this (picture here), whose is (it)?’, he says, ‘Madam, (it’s) mine’, she asks (him), the guy comes to the front, Ms. (Teacher) asks him, ‘why did you draw the monkey under the banana tree?’
\z

\begin{tabular}{lllllllllllllll}
\lsptoprule
0005 & de & \multicolumn{2}{l}{blang,} & \multicolumn{2}{l}{adu} & \multicolumn{3}{l}{ibu,} & \multicolumn{2}{l}{tadi} & \multicolumn{2}{l}{ibu} & toki & meja\\
& \textsc{3sg} & \multicolumn{2}{l}{say} & \multicolumn{2}{l}{oh.no!} & \multicolumn{3}{l}{woman} & \multicolumn{2}{l}{earlier} & \multicolumn{2}{l}{woman} & beat & table\\
& \multicolumn{2}{l}{itu} & \multicolumn{2}{l}{yang} & \multicolumn{2}{l}{monyet} & de & \multicolumn{2}{l}{jatu} & \multicolumn{2}{l}{dari} & \multicolumn{3}{l}{atas}\\
& \multicolumn{2}{l}{\textsc{d.dist}} & \multicolumn{2}{l}{\textsc{rel}} & \multicolumn{2}{l}{monkey} & \textsc{3sg} & \multicolumn{2}{l}{fall} & \multicolumn{2}{l}{from} & \multicolumn{3}{l}{top}\\
\lspbottomrule
\end{tabular}
\ea
\glt
he says, ‘oh no!, Madam!, a little bit earlier you (‘Madam’) knocked on the table, that’s why the monkey fell off from the top (of the banana plant)’
\end{styleFreeTranslEngxvpt}

\subsection{Joke: Dividing three fish}

\begin{tabular}{ll}
\lsptoprule
File name: & 081109-011-JR\\
Text type: & Joke (Elicited text)\\
Interlocutors: & 2 younger males\\
Length (min.): & 1:07\\
\lspbottomrule
\end{tabular}
\begin{tabular}{llllllllllllllllllllllllll}
\lsptoprule
0003 & pace & \multicolumn{5}{l}{orang} & \multicolumn{3}{l}{Biak} & \multicolumn{3}{l}{dong} & \multicolumn{2}{l}{dua} & \multicolumn{5}{l}{mancing,} & \multicolumn{3}{l}{dong} & \multicolumn{3}{l}{dua}\\
& man & \multicolumn{5}{l}{person} & \multicolumn{3}{l}{Biak} & \multicolumn{3}{l}{\textsc{3pl}} & \multicolumn{2}{l}{two} & \multicolumn{5}{l}{fish.with.rod} & \multicolumn{3}{l}{\textsc{3pl}} & \multicolumn{3}{l}{two}\\
& \multicolumn{5}{l}{mancing,} & \multicolumn{7}{l}{mancing} & \multicolumn{4}{l}{mancing} & \multicolumn{7}{l}{mancing,} & \multicolumn{2}{l}{suda}\\
& \multicolumn{5}{l}{fish.with.rod} & \multicolumn{7}{l}{fish.with.rod} & \multicolumn{4}{l}{fish.with.rod} & \multicolumn{7}{l}{fish.with.rod} & \multicolumn{2}{l}{already}\\
& \multicolumn{2}{l}{dong} & dua & \multicolumn{4}{l}{dapat} & \multicolumn{4}{l}{ikang} & \multicolumn{4}{l}{ini} & \multicolumn{2}{l}{tiga} & \multicolumn{4}{l}{ekor,} & \multicolumn{3}{l}{dapat} & ikang\\
& \multicolumn{2}{l}{\textsc{3pl}} & two & \multicolumn{4}{l}{get} & \multicolumn{4}{l}{fish} & \multicolumn{4}{l}{\textsc{d.prox}} & \multicolumn{2}{l}{three} & \multicolumn{4}{l}{tail} & \multicolumn{3}{l}{get} & fish\\
& \multicolumn{2}{l}{tiga} & \multicolumn{2}{l}{ekor,} & \multicolumn{4}{l}{dong} & \multicolumn{2}{l}{dua} & \multicolumn{3}{l}{mulay} & \multicolumn{5}{l}{mendarat} & \multicolumn{2}{l}{ke} & \multicolumn{5}{l}{darat}\\
& \multicolumn{2}{l}{three} & \multicolumn{2}{l}{tail} & \multicolumn{4}{l}{\textsc{3pl}} & \multicolumn{2}{l}{two} & \multicolumn{3}{l}{start} & \multicolumn{5}{l}{land} & \multicolumn{2}{l}{to} & \multicolumn{5}{l}{land}\\
\lspbottomrule
\end{tabular}
\ea
\glt 
the two Biak guys are fishing, they are fishing, fishing, fishing, fishing, eventually the two of them get these fish, three (of them), having gotten three fish, the two of them start landing on the shore
\z

\begin{tabular}{lllllllllllllllllllll}
\lsptoprule
0004 & \multicolumn{3}{l}{sampe} & di & \multicolumn{3}{l}{darat,} & \multicolumn{3}{l}{suda} & \multicolumn{2}{l}{dong} & \multicolumn{2}{l}{dua} & \multicolumn{2}{l}{mulay} & \multicolumn{2}{l}{bagi} & ikang & itu,\\
& \multicolumn{3}{l}{reach} & at & \multicolumn{3}{l}{land} & \multicolumn{3}{l}{already} & \multicolumn{2}{l}{\textsc{3pl}} & \multicolumn{2}{l}{two} & \multicolumn{2}{l}{start} & \multicolumn{2}{l}{divide} & fish & \textsc{d.dist}\\
& de & \multicolumn{4}{l}{mulay,} & \multicolumn{3}{l}{dep} & \multicolumn{3}{l}{kawang,} & \multicolumn{2}{l}{e,} & \multicolumn{2}{l}{mulay} & \multicolumn{2}{l}{bilang,} & \multicolumn{2}{l}{kawang} & ko\\
& \textsc{3sg} & \multicolumn{4}{l}{start} & \multicolumn{3}{l}{\textsc{3sg}:\textsc{poss}} & \multicolumn{3}{l}{friend} & \multicolumn{2}{l}{uh} & \multicolumn{2}{l}{start} & \multicolumn{2}{l}{say} & \multicolumn{2}{l}{friend} & \textsc{2sg}\\
& \multicolumn{2}{l}{bawa} & \multicolumn{4}{l}{satu,} & sa & \multicolumn{2}{l}{bawa} & \multicolumn{11}{l}{satu}\\
& \multicolumn{2}{l}{bring} & \multicolumn{4}{l}{one} & \textsc{1sg} & \multicolumn{2}{l}{bring} & \multicolumn{11}{l}{one}\\
\lspbottomrule
\end{tabular}
\ea
\glt 
having arrived on the shore, the two of them start dividing the fish, he starts to, his friend, uh, starts to say, ‘you friend take one (and) I take one’
\z

\begin{tabular}{lllllllllllllllllllll}
\lsptoprule
0005 & trus & \multicolumn{3}{l}{de} & \multicolumn{3}{l}{bilang,} & \multicolumn{2}{l}{i,} & \multicolumn{4}{l}{baru} & \multicolumn{2}{l}{yang} & \multicolumn{2}{l}{satu} & \multicolumn{2}{l}{ini,} & de\\
& next & \multicolumn{3}{l}{\textsc{3sg}} & \multicolumn{3}{l}{say} & \multicolumn{2}{l}{ugh!} & \multicolumn{4}{l}{and.then} & \multicolumn{2}{l}{\textsc{rel}} & \multicolumn{2}{l}{one} & \multicolumn{2}{l}{\textsc{d.prox}} & \textsc{3sg}\\
& \multicolumn{3}{l}{dep} & \multicolumn{3}{l}{temang} & \multicolumn{4}{l}{tu,} & \multicolumn{2}{l}{a,} & \multicolumn{2}{l}{ko} & \multicolumn{2}{l}{sala} & \multicolumn{2}{l}{bagi,} & \multicolumn{2}{l}{gabung}\\
& \multicolumn{3}{l}{\textsc{3sg}:\textsc{poss}} & \multicolumn{3}{l}{friend} & \multicolumn{4}{l}{\textsc{d.dist}} & \multicolumn{2}{l}{ah!} & \multicolumn{2}{l}{\textsc{2sg}} & \multicolumn{2}{l}{wrong} & \multicolumn{2}{l}{divide} & \multicolumn{2}{l}{join}\\
& \multicolumn{2}{l}{lagi,} & \multicolumn{3}{l}{temang} & \multicolumn{3}{l}{satu} & \multicolumn{3}{l}{bagi} & \multicolumn{9}{l}{lagi}\\
& \multicolumn{2}{l}{again} & \multicolumn{3}{l}{friend} & \multicolumn{3}{l}{one} & \multicolumn{3}{l}{divide} & \multicolumn{9}{l}{again}\\
\lspbottomrule
\end{tabular}
\ea
\glt 
then he says, ‘ugh, but what about this one?’, he, his friend says, ‘ah, you’ve divided (the fish) incorrectly’, (they) put (the fish back) together again, (that) one friend divides them again
\z

\begin{tabular}{lllllllllllllllll}
\lsptoprule
0006 & dapat & \multicolumn{2}{l}{satu,} & \multicolumn{2}{l}{sa} & \multicolumn{2}{l}{satu,} & \multicolumn{2}{l}{yang} & \multicolumn{2}{l}{ini?,} & \multicolumn{2}{l}{de} & pu & temang & tra\\
& get & \multicolumn{2}{l}{one} & \multicolumn{2}{l}{\textsc{1sg}} & \multicolumn{2}{l}{one} & \multicolumn{2}{l}{\textsc{rel}} & \multicolumn{2}{l}{\textsc{d.prox}} & \multicolumn{2}{l}{\textsc{3sg}} & \textsc{poss} & friend & \textsc{neg}\\
& \multicolumn{2}{l}{trima} & \multicolumn{2}{l}{baik} & \multicolumn{2}{l}{lagi,} & \multicolumn{2}{l}{de} & \multicolumn{2}{l}{gabung} & \multicolumn{2}{l}{lagi} & \multicolumn{4}{l}{((laughter))}\\
& \multicolumn{2}{l}{receive} & \multicolumn{2}{l}{good} & \multicolumn{2}{l}{again} & \multicolumn{2}{l}{\textsc{3sg}} & \multicolumn{2}{l}{join} & \multicolumn{2}{l}{again} & \multicolumn{4}{l}{}\\
\lspbottomrule
\end{tabular}
\ea
\glt 
‘(you) get one, I (get) one’, ‘(and) this one?’ again his friend doesn’t accept (the result of this dividing,) well, he puts (them back) together again ((laughter))
\z

\begin{tabular}{lllllllllllllllll}
\lsptoprule
0007 & \multicolumn{2}{l}{dong} & \multicolumn{2}{l}{dua} & \multicolumn{3}{l}{bagi} & \multicolumn{3}{l}{su} & \multicolumn{3}{l}{begitu} & \multicolumn{2}{l}{trus,} & suda\\
& \multicolumn{2}{l}{\textsc{3pl}} & \multicolumn{2}{l}{two} & \multicolumn{3}{l}{divide} & \multicolumn{3}{l}{already} & \multicolumn{3}{l}{like.that} & \multicolumn{2}{l}{be.continuous} & already\\
& \multicolumn{3}{l}{orang} & \multicolumn{3}{l}{Ayamaru} & \multicolumn{3}{l}{datang,} & \multicolumn{3}{l}{datang} & \multicolumn{2}{l}{de} & \multicolumn{2}{l}{bilang,}\\
& \multicolumn{3}{l}{person} & \multicolumn{3}{l}{Ayamaru} & \multicolumn{3}{l}{come} & \multicolumn{3}{l}{come} & \multicolumn{2}{l}{\textsc{3sg}} & \multicolumn{2}{l}{say}\\
& eh, & \multicolumn{3}{l}{kam} & dua & \multicolumn{3}{l}{baku} & \multicolumn{3}{l}{melawang} & \multicolumn{5}{l}{apa?}\\
& hey! & \multicolumn{3}{l}{\textsc{2pl}} & two & \multicolumn{3}{l}{\textsc{recp}} & \multicolumn{3}{l}{fight} & \multicolumn{5}{l}{what}\\
\lspbottomrule
\end{tabular}
\ea
\glt 
the two of them continue dividing (the fish) just like that, eventually an Ayamaru guy comes by, having come by, he says, ‘hey, about what are you two fighting with each other?’
\z

\begin{tabular}{llllllllllllllllllllll}
\lsptoprule
0008 & de & \multicolumn{3}{l}{bilang,} & \multicolumn{3}{l}{om,} & \multicolumn{3}{l}{ini,} & \multicolumn{3}{l}{kitong} & \multicolumn{3}{l}{dua} & \multicolumn{2}{l}{baku} & \multicolumn{3}{l}{melawang}\\
& \textsc{3sg} & \multicolumn{3}{l}{say} & \multicolumn{3}{l}{uncle} & \multicolumn{3}{l}{\textsc{d.prox}} & \multicolumn{3}{l}{\textsc{1pl}} & \multicolumn{3}{l}{two} & \multicolumn{2}{l}{\textsc{recp}} & \multicolumn{3}{l}{fight}\\
& \multicolumn{3}{l}{gara-gara} & \multicolumn{3}{l}{ikang,} & \multicolumn{3}{l}{kitong} & \multicolumn{2}{l}{dua} & \multicolumn{3}{l}{bagi,} & \multicolumn{3}{l}{de} & \multicolumn{2}{l}{satu,} & sa & satu\\
& \multicolumn{3}{l}{because} & \multicolumn{3}{l}{fish} & \multicolumn{3}{l}{\textsc{1pl}} & \multicolumn{2}{l}{two} & \multicolumn{3}{l}{divide} & \multicolumn{3}{l}{\textsc{3sg}} & \multicolumn{2}{l}{one} & \textsc{1sg} & one\\
& \multicolumn{2}{l}{baru} & \multicolumn{3}{l}{yang} & \multicolumn{3}{l}{ini} & \multicolumn{4}{l}{nanti} & \multicolumn{3}{l}{ke} & \multicolumn{6}{l}{mana?}\\
& \multicolumn{2}{l}{and.then} & \multicolumn{3}{l}{\textsc{rel}} & \multicolumn{3}{l}{\textsc{d.prox}} & \multicolumn{4}{l}{very.soon} & \multicolumn{3}{l}{to} & \multicolumn{6}{l}{where}\\
\lspbottomrule
\end{tabular}
\ea
\glt 
he says, ‘uncle, what’s-its-name, the two of us are fighting each other because of the fish, we two divide (it), he (has) one (fish), I (have) one (fish), but where does this one go?’
\z

\begin{tabular}{llllllllllllllll}
\lsptoprule
0009 & de & \multicolumn{3}{l}{bilang,} & \multicolumn{2}{l}{itu} & \multicolumn{2}{l}{yang} & masi & \multicolumn{3}{l}{tunggu} & saya, & de & pegang\\
& \textsc{3sg} & \multicolumn{3}{l}{say} & \multicolumn{2}{l}{\textsc{d.dist}} & \multicolumn{2}{l}{\textsc{3sg}} & still & \multicolumn{3}{l}{wait} & \textsc{1sg} & \textsc{3sg} & hold\\
& \multicolumn{2}{l}{dang} & dong & \multicolumn{2}{l}{jalang,} & \multicolumn{2}{l}{de} & \multicolumn{3}{l}{bilang,} & pas & \multicolumn{4}{l}{to?}\\
& \multicolumn{2}{l}{and} & \textsc{3pl} & \multicolumn{2}{l}{walk} & \multicolumn{2}{l}{\textsc{3sg}} & \multicolumn{3}{l}{say} & be.exact & \multicolumn{4}{l}{right?}\\
\lspbottomrule
\end{tabular}
\ea
\glt 
he says, ‘that (is one) which is still waiting for me’ he takes (it) and they walk (away), and he says, ‘that fits, right?’
\z

\begin{tabular}{lllllllllll}
\lsptoprule
0010 & de & bilang, & \multicolumn{2}{l}{itu} & \multicolumn{2}{l}{kawang} & sa & \multicolumn{2}{l}{su} & bilang,\\
& \textsc{3sg} & say & \multicolumn{2}{l}{\textsc{d.dist}} & \multicolumn{2}{l}{friend} & \textsc{1sg} & \multicolumn{2}{l}{already} & say\\
& \multicolumn{3}{l}{makanya} & \multicolumn{2}{l}{skola} & \multicolumn{3}{l}{baru} & \multicolumn{2}{l}{pintar}\\
& \multicolumn{3}{l}{for.that.reason} & \multicolumn{2}{l}{go.to.school} & \multicolumn{3}{l}{and.then} & \multicolumn{2}{l}{be.clever}\\
\lspbottomrule
\end{tabular}
\ea
\glt
he says, ‘friend, that’s (what) I already told (you), that’s why you should go to school, then you’ll be clever’
\end{styleFreeTranslEngxvpt}

%\setcounter{page}{1}\section{Overview of recorded corpus}
\label{bkm:Ref373929508}
This appendix gives an overview of the recorded 220 texts which form the basis for the present description of Papuan Malay (see also §1.11.4 and §1.11.5). For each text the following information is provided:


\begin{styleIvI}
File name: For each text the name of the respective WAV file and Toolbox record is given. This name specifies the date of its recording, a running number for all texts recorded during one day, and a code for the type of text recorded. This is illustrated with the record number 080919-007-CvNP: 080919 stands for “2008, September 19”; 007 stands for “recorded text \#7 of that day”; and CvNP stands for “Personal Narrative (NP) which occurred during a Conversation (Cv)”. The same record numbers are used for the examples given in this book (see ‘Conventions for examples‘, p. \pageref{bkm:Ref434428827}) and the transcribed texts presented in Appendix A.
\end{styleIvI}

\begin{styleIvI}
Text type: The meta data specify the genre of the recorded texts, such as conversation, narrative, expository, hortatory, or joke, and whether the recorded texts occurred spontaneously or were elicited.
\end{styleIvI}

\begin{styleIvI}
Interlocutors: The meta data give information about the gender and age group of the recorded interlocutors.
\end{styleIvI}

\begin{styleIvI}
Topics: For each recorded text the overall topic is given.
\end{styleIvI}

\begin{styleIvI}
Length in minutes
\end{styleIvI}


The following abbreviations are being used:


\begin{styleIvI}
File name: Cv = conversation, Ph = conversation over the phone, NP = narrative, personal experience, NF = narrative, folk story, Ex = expository, Ht = hortatory, Pr = procedural, JR = Joke/Riddle.
\end{styleIvI}

\begin{styleIvI}
Text type: CvSp = spontaneous conversation, Cv-w/auth = conversation with the author, Elicit = elicited text, Cas = casual conversation, ph. = conversation over the phone, Expos = expository, Hortat = hortatory, Proc = procedural, NarrP = narrative, personal experience, NarrF = narrative, folk story.
\end{styleIvI}

\begin{styleIiI}
Interlocutors: M = males, F = females, O = older adults in their thirties or older, Y = young adults in teens or twenties, C = children of about five to 13 years of age.
\end{styleIiI}

\tablehead{
 File name & Text type / Length & \arraybslash Topic / Interlocutors\\
}
\begin{tabular}{lll}
\lsptoprule
080916-001-CvNP & CvSp\_NarrP & A drunkard in the hospital at night\\
& 2:33:00 & 2 FO\\
080917-001-CvNP & CvSp\_NarrP & Woken up by a friend\\
& 0:40:00 & 1 MY, 1 MO\\
080917-002-Cv & CvSp\_Cas & Organizing a youth event got interrupted\\
& 1:25:00 & 1 MO, 1 MY, 1 FO\\
080917-003a-CvEx & CvSp\_Expos & Asking for a leave of absence 1\\
& 2:11:00 & 1 MC, 2 FY, 1 FO\\
080917-003b-CvEx & CvSp\_Expos & Asking for a leave of absence 2\\
& 2:27:00 & 1 MC, 2 FY, 1 FO\\
080917-004-CvHt & CvSp\_Hort & Don’t get dirty!\\
& 0:10:00 & 2 MC, 1 FO\\
080917-005-Cv & CvSp\_Cas & What is this?\\
& 0:02:00 & 1 MC\\
080917-006-CvHt & CvSp\_Hort & Bathe in the ocean!\\
& 1:00:00 & 3 MC, 2 FO\\
080917-007-CvHt & CvSp\_Hort & Get a Malaria blood test!\\
& 1:05:00 & 1 MY, 1 FO\\
080917-008-NP & Elicit\_NarrP & Deliverance for Sarmi\\
& 26:00:00 & 1 FO\\
080917-009-CvEx & CvSp\_Expos & Directions to a certain statue and tree\\
& 0:50:00 & 2 MC, 1 FO\\
080917-010-CvEx & CvSp\_Expos & Raising children well\\
& 16:00:00 & 1 FY, 1 FO\\
080918-001-CvNP & CvSp\_NarrP & Two sudden deaths\\
& 5:30:00 & 2 MY, 1 FO\\
080919-001-Cv & CvSp\_Cas & Candidates for local elections\\
& 8:30:00 & 1 MO\\
080919-002-Cv & CvSp\_Cas & Transport options to the regent’s office\\
& 0:45:00 & 2 FY, 2 FO\\
080919-003-NP & Elicit\_NarrP & Pig hunting with dogs\\
& 4:20:00 & 1 MO, 1 FO\\
080919-004-NP & Elicit\_NarrP & Pig hunting with bow and arrows\\
& 13:38:00 & 1 MO, 1 FO\\
080919-005-Cv & CvSp\_Cas & Wearing glasses\\
& 1:45:00 & 1 MY, 1 FY\\
080919-006-CvNP & CvSp\_NarrP & The speaker and her niece\\
& 4:30:00 & 1 MY, 2 FY\\
080919-007-CvNP & CvSp\_NarrP & A drunkard dies in hospital\\
& 6:00:00 & 1 MY, 1 FO, 2 FY\\
080919-008-CvNP & CvSp\_NarrP & A dying mother sees snow 1\\
& 2:40:00 & 1 MY, 1 FY\\
080921-001-CvNP & CvSp\_NarrP & Trip to Pante Timur 1\\
& 3:00:00 & 1 MO, 1 FY, 3 FO\\
080921-002-Cv & CvSp\_Cas & Coming to Sarmi\\
& 0:35:00 & 1 MY, 1 MO, 1 FY, 1 FO\\
080921-003-CvNP & CvSp\_NarrP & Trip to Pante Timur 2\\
& 0:55:00 & 1 MY, 1 MO, 1 FY, 1 FO\\
080921-004a-CvNP & CvSp\_NarrP & Trip to Pante Timur 3a\\
& 4:00:00 & 1 MY, 1 MO, 1 FY, 1 FO\\
080921-004b-CvNP & CvSp\_NarrP & Trip to Pante Timur 3b\\
& 2:00:00 & 1 MY, 1 MO, 1 FY, 1 FO\\
080921-005-CvNP & CvSp\_NarrP & A dying mother sees snow 2\\
& 1:00:00 & 1 MY, 1 FY\\
080921-006-CvNP & CvSp\_NarrP & A dying mother sees snow 3\\
& 1:20:00 & 1 MY, 1 FY\\
080921-007-CvNP & CvSp\_NarrP & Trip to Pante Timur 4\\
& 1:00:00 & 1 MY, 1 MO, 1 FY, 1 FO\\
080921-008-Cv & CvSp\_Cas & Swimming in the ocean\\
& 1:00:00 & 1 MY, 1 FY\\
080921-009-Cv & CvSp\_Cas & Feeling sleepy at night\\
& 0:30:00 & 1 MY, 2 FY\\
080921-010-Cv & CvSp\_Cas & A funny relative\\
& 0:45:00 & 1 MY, 1 FY\\
080921-011-Cv & CvSp\_Cas & Picnic at the beach\\
& 2:00:00 & 1 MY, 1 FY\\
080922-001a-CvPh & CvSp\_Cas (ph.) & Trip to Sorong 1\\
& 66:00:00 & 1 MY, 1 MO, 1 FY\\
080922-001b-CvPh & CvSp\_Cas (ph.) & Trip to Sorong 2\\
& 1:45:00 & 1 MY, 1 MO, 1 FY\\
080922-002-Cv & CvSp\_Cas & Various topics 1\\
& 7:54:00 & 1 MY, 1 FY\\
080922-003-Cv & CvSp\_Cas & Various topics 2\\
& 6:23:00 & 1 MY, 1 FY\\
080922-004-Cv & CvSp\_Cas & Meeting a certain woman\\
& 2:03:00 & 1 MY, 1 FY\\
080922-005-CvEx & CvSp\_Expos & Bride-exchange customs\\
& 1:52:00 & 1 MY, 1 FY\\
080922-006-CvEx & CvSp\_Expos & Children’s future\\
& 0:46:00 & 1 MY, 1 FY\\
080922-007-CvNP & CvSp\_NarrP & A child’s magic thinking\\
& 1:12:00 & 1 MY, 1 FY\\
080922-008-CvNP & CvSp\_NarrP & Various childhood experiences\\
& 6:12:00 & 1 MY, 1 FY\\
080922-009-CvNP & CvSp\_NarrP & Playing as children\\
& 2:57:00 & 1 MY, 1 FY\\
080922-010a-CvNF & CvSp\_NarrF & Origins of a certain clan 1\\
& 39:00:00 & 1 MC, 1 MY, 1 FY\\
080922-010b-CvNF & CvSp\_NarrF & Origins of a certain clan 2\\
& 0:45:00 & 1 MY, 1 FY\\
080923-001-CvNP & CvSp\_NarrP & A crying child; looking for someone\\
& 1:50:00 & 2 MO\\
080923-002-CvEx & CvSp\_Expos & Working moral\\
& 1:10:00 & 1 MY, 1 MO\\
080923-003-CvNP & CvSp\_NarrP & Getting up early\\
& 0:23:00 & 1 MO, 2 FY\\
080923-004-Cv & CvSp\_Cas & Cooking\\
& 2:30:00 & 1 MO, 1 FY\\
080923-005-Cv & CvSp\_Cas & A sudden death\\
& 0:19:00 & 1 MO, 1 FY\\
080923-006-CvNP & CvSp\_NarrP & An incident with a young person\\
& 1:25:00 & 1 MO, 1 FY\\
080923-007-Cv & CvSp\_Cas & A sparrow in the kitchen; bridges are needed\\
& 1:45:00 & 1 MO, 1 FY\\
080923-008-Cv & CvSp\_Cas & Reprimanding a young teacher\\
& 2:12:00 & 1 MO, 1 FY\\
080923-009-Cv & CvSp\_Cas & The speaker’s personal background\\
& 4:55:00 & 1 MO, 1 FY\\
080923-010-CvNP & CvSp\_NarrP & An accident with a motorbike taxi driver\\
& 2:16:00 & 1 MO, 1 FY\\
080923-011-Cv & CvSp\_Cas & A sudden death; about a sick relative\\
& 1:40:00 & 1 MO, 1 FO, 1 FY\\
080923-012-CvNP & CvSp\_NarrP & Being sick from eating grass-cutters\\
& 4:14:00 & 1 MY, 1 MO, 1 FO\\
080923-013-CvEx & CvSp\_Expos & How to find out whether a sick person will survive 1\\
& 3:45:00 & 1 MY, 1 MO\\
080923-014-CvEx & CvSp\_Expos & How to find out whether a sick person will survive 2\\
& 2:59:00 & 1 MY, 1 MO\\
080923-015-CvEx & CvSp\_Expos & Getting a sign about a dying person\\
& 5:20:00 & 1 MY, 1 MO\\
080923-016-CvNP & CvSp\_NarrP & Making a song book\\
& 1:48:00 & 1 MY, 1 MO\\
080924-001-Pr & Elicit\_Proc & Making sago\\
& 1:21:00 & 1 FO\\
080924-002-Pr & Elicit\_Proc & Working in the garden\\
& 1:57:00 & 1 FO\\
080925-001-CvEx & CvSp\_Expos & Children’s language proficiency\\
& 1:17:00 & 1 FY, 1 FO\\
080925-002-CvHt & CvSp\_Hort & Clean the water filter!\\
& 0:19:00 & 1 FO\\
080925-003-Cv & CvSp\_Cas & Construction work on the new church in Sawar 1\\
& 13:19:00 & 1 MO, 2 FO\\
080925-004-Cv & CvSp\_Cas & Evening tea\\
& 1:32:00 & 1 FO\\
080925-005-CvPh & CvSp\_Cas (ph.) & Talking on the phone with his children in Jayapura\\
& 1:15:00 & 1 MO\\
080927-001-Cv & CvSp\_Cas & Going on vacation\\
& 0:39:00 & 1 FO\\
080927-002-CvNP & CvSp\_NarrP & Eating food that is too spicy\\
& 1:33:00 & 1 FO\\
080927-003-Cv & CvSp\_Cas & Transporting goods by a Johnson motorboat\\
& 1:53:00 & 2 FO\\
080927-004-CvNP & Cv-w/auth\_NarrP & Construction work on the new church in Sawar 2a\\
& 1:00:00 & 1 FO\\
080927-005-CvNP & Cv-w/auth\_NarrP & Construction work on the new church in Sawar 2b\\
& 1:30:00 & 1 FO\\
080927-006-CvNP & Cv-w/auth\_NarrP & The speaker’s work in Sawar\\
& 4:02:00 & 1 FO\\
080927-007-CvNP & Cv-w/auth\_NarrP & The speaker’s family relations 1\\
& 2:06:00 & 1 FO\\
080927-008-CvEx & Cv-w/auth\_Expos & Metallic blue water color\\
& 1:04:00 & 1 FO\\
080927-009-CvNP & Cv-w/auth\_NarrP & The speaker’s family relations 2; life in Sarmi\\
& 6:47:00 & 1 FO\\
081002-001-CvNP & CvSp\_NarrP & Traveling from Webro to Sarmi\\
& 4:39:00 & 2 FY\\
081002-002-Cv & CvSp\_Cas & Eating and bathing\\
& 0:25:00 & 2 FY\\
081002-003-Cv & CvSp\_Cas & Eating betel nut\\
& 0:38:00 & 2 FY\\
081005-001-Cv & CvSp\_Cas & Visiting and calling government offices\\
& 1:43:00 & 2 MO\\
081006-001-Cv & CvSp\_Cas & Visiting a sick person\\
& 0:56:00 & 1 MY, 1 MO\\
081006-002-Cv & CvSp\_Cas & Returning from Sarmi to Jayapura\\
& 0:43:00 & 1 MO, 1 FO\\
081006-003-Cv & CvSp\_Cas & Traveling by low tide\\
& 0:32:00 & 1 MO, 1 FO\\
081006-004-Cv & CvSp\_Cas & Waiting for the village mayor\\
& 0:33:00 & 2 MO\\
081006-005-Cv & CvSp\_Cas & A proud person\\
& 0:22:00 & 2 FY\\
081006-006-Cv & CvSp\_Cas & What’s his/her name?\\
& 0:02:00 & 2 FY\\
081006-007-Cv & CvSp\_Cas & Traveling to Arso\\
& 0:20:00 & 2 FY\\
081006-008-Cv & CvSp\_Cas & Celebrating a birthday\\
& 1:35:00 & 2 FY\\
081006-009-Cv & CvSp\_Cas & An angry person\\
& 2:07:00 & 2 FY\\
081006-010-Cv & CvSp\_Cas & Returning from Webro\\
& 0:32:00 & 2 FY\\
081006-011-Cv & CvSp\_Cas & Going home\\
& 0:12:00 & 1 MO, 1 FY\\
081006-012-Cv & CvSp\_Cas & A church meeting\\
& 0:19:00 & 4 MO\\
081006-013-Cv & CvSp\_Cas & Taking a hot bath when sick with malaria\\
& 0:57:00 & 1 MO, 4 FO\\
081006-014-Cv & CvSp\_Cas & Playing football at the beach\\
& 3:43:00 & 5 FY\\
081006-015-Cv & CvSp\_Cas & The car outside and their owners\\
& 1:52:00 & 5 FY\\
081006-016-Cv & CvSp\_Cas & The upcoming youth retreat\\
& 2:17:00 & 5 FY\\
081006-017-Cv & CvSp\_Cas & Driving around Sarmi; preparing for Christmas\\
& 3:13:00 & 5 FY\\
081006-018-Cv & CvSp\_Cas & Playing music and irritating others\\
& 0:26:00 & 5 FY\\
081006-019-Cv & Cv-w/auth\_Cas & A trip to Webro and returning to Sarmi\\
& 0:50:00 & 1 FO\\
081006-020-Cv & Cv-w/auth\_Cas & A motorbike accident\\
& 2:46:00 & 1 MO, 1 FO\\
081006-021-CvHt & CvSp\_Hort & Picking flowers\\
& 1:14:00 & 1 FC, 1 FY\\
081006-022-CvEx & Cv-w/auth\_Expos & Evil spirits in humans\\
& 15:43:00 & 1 MO, 1 FO\\
081006-023-CvEx & Cv-w/auth\_Expos & What non-Christians believe; heaven and hell\\
& 12:36:00 & 1 MO, 1 FO\\
081006-024-CvEx & Cv-w/auth\_Expos & Exchanging children 1\\
& 17:39:00 & 1 FO\\
081006-025-CvEx & Cv-w/auth\_Expos & Exchanging children 2\\
& 4:07:00 & 1 FO\\
081006-026-CvEx & Cv-w/auth\_Expos & Exchanging children 3\\
& 2:06:00 & 1 FO\\
081006-027-CvEx & Cv-w/auth\_Expos & Exchanging children 4\\
& 2:21:00 & 1 FO\\
081006-028-CvEx & Cv-w/auth\_Expos & Exchanging children 5\\
& 2:44:00 & 1 FO\\
081006-029-CvEx & Cv-w/auth\_Expos & Bride price customs\\
& 5:29:00 & 1 FO\\
081006-030-CvEx & Cv-w/auth\_Expos & Sterility\\
& 1:56:00 & 1 FO\\
081006-031-Cv & CvSp\_Cas & Returning to Jayapura\\
& 0:48:00 & 2 MO, 1 FO\\
081006-032-Cv & CvSp\_Cas & An anniversary; Papuans and outsiders; elections\\
& 7:17:00 & 2 MO, 1 FO\\
081006-033-Cv & CvSp\_Cas & Building a road to Webro\\
& 5:01:00 & 2 MO, 2 FO\\
081006-034-CvEx & CvSp\_Expos & Spiritual warfare 1\\
& 6:50:00 & 1 MY, 1 FO\\
081006-035-CvEx & CvSp\_Expos & Spiritual warfare 2\\
& 11:40:00 & 1 MY, 1 FO\\
081006-036-CvEx & CvSp\_Expos & Spiritual warfare 3\\
& 2:41:00 & 1 MY, 1 FO\\
081008-001-Cv & CvSp\_Cas & A meeting of village mayors 1\\
& 4:43:00 & 2 MY, 3 MO\\
081008-002-Cv & CvSp\_Cas & A meeting of village mayors 2\\
& 0:30:00 & 2 MY, 2 MO\\
081008-003-Cv & CvSp\_Cas & Motorbike problems\\
& 5:17:00 & 3 MO\\
081010-001-Cv & CvSp\_Cas & A women’s meeting at the regent’s office\\
& 9:41:00 & 5 FO\\
081011-001-Cv & CvSp\_Cas & Development of Sarmi\\
& 15:08:00 & 2 MY, 2 MO\\
081011-002-Cv & CvSp\_Cas & Islamic service at school\\
& 0:03:00 & 1 FY\\
081011-003-Cv & CvSp\_Cas & Wanting bananas\\
& 0:35:00 & 1 MC, 2 FY, 2 FO\\
081011-004-Cv & CvSp\_Cas & Traveling to Jayapura\\
& 0:23:00 & 1 FY, 1 FO\\
081011-005-Cv & CvSp\_Cas & A youth retreat and youth meetings\\
& 4:22:00 & 2 FY\\
081011-006-Cv & CvSp\_Cas & Going to school\\
& 0:11:00 & 1 MC, 1 MO, 2 FO\\
081011-007-Cv & CvSp\_Cas & Songs in the vernacular\\
& 0:40:00 & 1 FO\\
081011-008-CvPh & CvSp\_Cas (ph.) & Looking for someone\\
& 1:17:00 & 1 MO, 1 FO\\
081011-009-Cv & CvSp\_Cas & A drunken youth\\
& 2:49:00 & 2 MO, 1 FY, 1 FO\\
081011-010-Cv & CvSp\_Cas & A small plane circling Sarmi\\
& 1:00:00 & 1 MO, 1 FO\\
081011-011-Cv & CvSp\_Cas & Obtaining a school certificate 1; a drunken youth\\
& 3:41:00 & 2 MO, 1 FO\\
081011-012-Cv & CvSp\_Cas & Obtaining a school certificate 2\\
& 1:37:00 & 1 FY, 1 FO\\
081011-013-Cv & CvSp\_Cas & Living in Wamena\\
& 0:52:00 & 1 MO, 1 FY\\
081011-014-Cv & CvSp\_Cas & About shortening something\\
& 1:03:00 & 1 FY, 1 FO\\
081011-015-Cv & CvSp\_Cas & Using different kinds of oil as medicine\\
& 0:34:00 & 1 FY, 1 FO\\
081011-016-Cv & CvSp\_Cas & A certain river\\
& 0:28:00 & 1 FO\\
081011-017-Cv & CvSp\_Cas & Problems with a muffler\\
& 0:12:00 & 1 MO, 1 FO\\
081011-018-Cv & CvSp\_Cas & Obtaining a school certificate 3\\
& 2:17:00 & 1 MY, 2 FO\\
081011-019-Cv & CvSp\_Cas & Electricity problems\\
& 1:30:00 & 2 FO\\
081011-020-Cv & CvSp\_Cas & Traveling to Pante Timur 1\\
& 10:59:00 & 1 MO, 1 FO\\
081011-021-Cv & CvSp\_Cas & Traveling to Pante Timur 2\\
& 1:36:00 & 1 FY, 1 FO\\
081011-022-Cv & CvSp\_Cas & Hosting many people; killing one of two twins\\
& 15:52:00 & 1 MO, 1 FY, 1 FO\\
081011-023-Cv & CvSp\_Cas & Obtaining a school certificate 4; rural school education\\
& 21:45:00 & 1 FY, 1 FO\\
081011-024-Cv & CvSp\_Cas & Rural school education and politics\\
& 9:13:00 & 4 MO\\
081012-001-CvPh & CvSp\_Cas (ph.) & Wanting to visit his friend\\
& 3:02:00 & 1 MO\\
081013-001-Cv & CvSp\_Cas & Obtaining a school certificate 5\\
& 0:39:00 & 1 MO, 1 FO\\
081013-002-Cv & CvSp\_Cas & The bridge at Muara\\
& 0:58:00 & 1 MO, 1 FO\\
081013-003-Cv & CvSp\_Cas & A motorbike accident 1\\
& 0:46:00 & 1 MO, 1 FO\\
081013-004-Cv & CvSp\_Cas & A motorbike accident 2\\
& 0:54:00 & 1 MO, 1 FO\\
081014-001-Cv & CvSp\_Cas & Feeling hot and cold\\
& 0:29:00 & 1 FO\\
081014-002-Cv & CvSp\_Cas & School problems\\
& 0:17:00 & 1 FO\\
081014-003-Cv & CvSp\_Cas & Motorbike problems\\
& 3:30:00 & 1 MO, 1 FO\\
081014-004-Cv & CvSp\_Cas & The upcoming local elections\\
& 4:15:00 & 1 FO\\
081014-005-Cv & CvSp\_Cas & Problems with a drunken person\\
& 4:10:00 & 1 FY\\
081014-006-CvPr & Cv-w/auth\_Proc & Making sago\\
& 8:30:00 & 1 FO\\
081014-007-CvEx & Cv-w/auth\_Expos & Women and men’s roles in Webro\\
& 15:18:00 & 1 FO\\
081014-008-CvNP & Cv-w/auth\_NarrP & A motorbike accident\\
& 3:30:00 & 1 MO, 1 FO\\
081014-009-CvEx & Cv-w/auth\_Expos & Honoring guests 1\\
& 3:00:00 & 1 FO\\
081014-010-CvEx & Cv-w/auth\_Expos & Honoring guests 2\\
& 2:06:00 & 1 MO, 1 FO\\
081014-011-CvEx & Cv-w/auth\_Expos & Honoring guests 3\\
& 7:02:00 & 1 MO, 1 FO\\
081014-012-NP & Elicit\_NarrP & A bicycle accident\\
& 1:16:00 & 1 FY\\
081014-013-NP & Elicit\_NarrP & A motorbike accident\\
& 1:06:00 & 1 FY\\
081014-014-NP & Elicit\_NarrP & Departed spirits\\
& 9:58:00 & 1 FY\\
081014-015-Cv & CvSp\_Cas & Making music\\
& 1:43:00 & 3 FO\\
081014-016-Cv & CvSp\_Cas & Cell phones\\
& 2:56:00 & 1 MY, 1 FO\\
081014-017-CvPr & CvSp\_Proc & Cooking pigs\\
& 6:10:00 & 1 MY, 1 MO, 1 FY, 2 FO\\
081015-001-Cv & CvSp\_Cas & A fight\\
& 2:31:00 & 1 FO\\
081015-002-Cv & CvSp\_Cas & On a plane\\
& 0:22:00 & 1 FO\\
081015-003-Cv & CvSp\_Cas & Traveling to Sarmi 1\\
& 1:38:00 & 1 FY, 1 FO\\
081015-004-Cv & CvSp\_Cas & Traveling to Sarmi 2\\
& 1:40:00 & 2 FY, 1 FO\\
081015-005-NP & Elicit\_NarrP & A motorbike accident\\
& 10:29:00 & 2 MO, 3 FO\\
081022-001-Cv & CvSp\_Cas & Returning from Takar\\
& 0:42:00 & 1 MC, 2 FY\\
081022-002-CvNP & CvSp\_NarrP & Retreat in Takar 1\\
& 1:40:00 & 1 MC, 1 FY\\
081022-003-Cv & CvSp\_Cas & Retreat in Takar 2\\
& 1:25:00 & 1 MC, 1 FY\\
081023-001-Cv & CvSp\_Cas & Playing volleyball; morning chores\\
& 4:52:00 & 1 MY, 2 FY\\
081023-002-Cv & CvSp\_Cas & The taekwondo team\\
& 0:46:00 & 1 MY, 2 FY\\
081023-003-Cv & CvSp\_Cas & Women’s sports\\
& 0:54:00 & 1 MY, 2 FY\\
081023-004-Cv & CvSp\_Cas & Retreat in Takar 3; school absences\\
& 3:16:00 & 1 MY, 2 FY\\
081025-001-CvHt & Cv-w/auth\_Hort & Put your trousers on!\\
& 0:46:00 & 1 FC, 1 FO\\
081025-002-Cv & Cv-w/auth\_Cas & Papeda and fish\\
& 0:13:00 & 1 FO\\
081025-003-Cv & Cv-w/auth\_Cas & Certain acquaintances; workshops in Sentani\\
& 16:53:00 & 1 MO, 1 FO\\
081025-004-Cv & Cv-w/auth\_Cas & Politics in Papua 1\\
& 12:35:00 & several MO, 1 FO\\
081025-005-Cv & Cv-w/auth\_Cas & Demonstrations\\
& 0:55:00 & several MO, 1 FO\\
081025-006-Cv & CvSp\_Cas & Retreat in Takar 4; spiritual warfare 1\\
& 24:49:00 & 3 FY, 1 FO\\
081025-007-Cv & CvSp\_Cas & Retreat in Takar 5; spiritual warfare 2\\
& 2:10:00 & 3 FY, 1 FO\\
081025-008-Cv & CvSp\_Cas & Retreat in Takar 6; spiritual warfare 3\\
& 11:14:00 & 3 FY, 1 FO\\
081025-009a-Cv & CvSp\_Cas & Retreat in Takar 7; spiritual warfare 4a\\
& 10:46:00 & 3 FY, 1 FO\\
081025-009b-Cv & CvSp\_Cas & Retreat in Takar 8; spiritual warfare 4b\\
& 7:04:00 & 3 FY, 1 FO\\
081029-001-Cv & Cv-w/auth\_Cas & Garden plants\\
& 1:06:00 & 1 MO\\
081029-002-Cv & Cv-w/auth\_Cas & Politics in Papua 2\\
& 5:44:00 & 1 MO\\
081029-003-Cv & Cv-w/auth\_Cas & Politics in Papua 3\\
& 1:00:00 & 1 MO\\
081029-004-Cv & Cv-w/auth\_Cas & Politics in Papua 4\\
& 12:16:00 & 1 MO\\
081029-005-Cv & Cv-w/auth\_Cas & Politics in Papua 5\\
& 19:43:00 & 1 MO\\
081106-001-CvPr & Cv-w/auth\_Proc & Killing dogs\\
& 0:49:00 & 1 MO, 1 FO\\
081108-001-JR & Elicit\_Joke & Joke: Killing a cow 1 (false start)\\
& 0:37:00 & 2 MY\\
081108-002-JR & Elicit\_Joke & Joke: Drawing an elephant\\
& 0:51:00 & 2 MY\\
081108-003-JR & Elicit\_Joke & Joke: Killing a cow 2\\
& 1:21:00 & 2 MY\\
081109-001-Cv & CvSp\_Cas & Doing sports\\
& 14:47:00 & 1 MY, 2 FY\\
081109-002-JR & Elicit\_Joke & Joke: Drawing a monkey\\
& 0:59:00 & 2 MY\\
081109-003-JR & Elicit\_Joke & Joke: Drawing a banana\\
& 1:06:00 & 2 MY\\
081109-004-JR & Elicit\_Joke & Joke: Taking singing lessons (false start)\\
& 0:29:00 & 2 MY\\
081109-005-JR & Elicit\_Joke & Joke: Grandfather and grandchild go fishing\\
& 1:15:00 & 2 MY\\
081109-006-JR & Elicit\_Joke & Joke: Taking singing lessons\\
& 0:44:00 & 2 MY\\
081109-007-JR & Elicit\_Joke & Joke: Twelve moons\\
& 1:24:00 & 2 MY\\
081109-008-JR & Elicit\_Joke & Joke: A ship arriving at the harbor\\
& 0:29:00 & 2 MY\\
081109-009-JR & Elicit\_Joke & Joke: A Javanese asking for papeda\\
& 1:13:00 & 2 MY\\
081109-010-JR & Elicit\_Joke & Joke: Seeing a turtle for the first time\\
& 0:42:00 & 2 MY\\
081109-011-JR & Elicit\_Joke & Joke: Dividing three fish\\
& 1:07:00 & 2 MY\\
081109-012-JR & Elicit\_Joke & Joke: Sleeping in church\\
& 0:55:00 & 2 MY\\
081110-001-Cv & Cv-w/auth\_Cas & Moving to Sawar\\
& 2:16:00 & 1 FO\\
081110-002-Cv & CvSp\_Cas & Buying soap; bringing gas to Webro\\
& 3:55:00 & 2 FO\\
081110-003-Cv & CvSp\_Cas & Forgetting chili peppers in Webro\\
& 1:17:00 & 2 FO\\
081110-004-Cv & Cv-w/auth\_Cas & Forgetting betel vine; getting ready to go out\\
& 0:48:00 & 1 FO\\
081110-005-CvPr & Cv-w/auth\_Proc & Wedding preparations 1\\
& 20:23:00 & 1 FO\\
081110-006-CvEx & Cv-w/auth\_Expos & Sexual relations and marriage\\
& 15:29:00 & 1 FO\\
081110-007-CvPr & Cv-w/auth\_Proc & Wedding preparations 2\\
& 2:32:00 & 1 FO\\
081110-008-CvNP & Cv-w/auth\_NarrP & Parents’ advice\\
& 29:35:00 & 1 FO\\
081115-001a-Cv & CvSp\_Cas & Problems with children 1\\
& 22:24:00 & 2 MO, 3 FO\\
081115-001b-Cv & CvSp\_Cas & Problems with children 2\\
& 4:51:00 & 2 MO, 3 FO\\
\lspbottomrule
\end{tabular}
%\setcounter{page}{1}\section{OLAC resources for the languages of the Sarmi regency}
\label{bkm:Ref374441356}
Table  ‎0 .1 and Table  ‎0 .2 give an overview of the resources available in and about the Papuan and Austronesian languages spoken in the Sarmi regency. The information is a summary of the information provided by OLAC, the Open Language Archives Community; OLAC is available at \url{http://www.language-archives.org} (8 January 2016).


\begin{stylecaption}
\label{bkm:Ref372910273}Table ‎0.\stepcounter{Table}{\theTable}:  OLAC resources available in and about the Papuan languages spoken in the Sarmi regency\footnote{\\
\\
\\
\\
\\
\\
\\
\\
\\
\\
\\
\\
\\
\\
\\
\par Abbreviations: Sur. = sociolinguistic survey; Lex. = lexical resources; Gr. = grammatical resources; Lit. = literacy resources; Ant. = anthropological resources; L1 = resources in the respective languages; NT = New Testament of the Bible in the respective languages.}
\end{stylecaption}

\tablehead{
\multicolumn{2}{l}{ Name \& ISO 639-3 code} & Sur. & Lex. & Gr. & Lit. & Ant. & L1 & \arraybslash NT\\
}
\begin{tabular}{lllllllll}
\lsptoprule
Aironan & [air] & X &  &  &  &  &  & \\
Beneraf & [bnv] &  & X &  &  &  &  & \\
Berik & [bkl] &  & X & X & X & X & X & \\
Dabe & [dbe] &  & X &  &  &  &  & \\
Isirawa & [srl] &  & X & X & X & X & X & \arraybslash X\\
Dineor & [mrx] &  & X &  &  &  &  & \\
Itik & [itx] &  & X &  &  &  &  & \\
Kwerba & [kwe] &  & X & X & X & X & X & \\
Kauwera & [xau] &  & X &  &  &  &  & \\
Kwesten & [kwt] &  & X &  &  &  &  & \\
Mander & [mqr] &  & X &  &  &  &  & \\
Massep & [mvs] & X &  &  &  &  &  & \\
Mawes & [mgk] &  & X &  &  &  &  & \\
Samarokena & [tmj] & X & X &  &  &  &  & \\
Bagusa & [bqb] &  &  &  &  &  &  & \\
Betaf & [bfe] &  &  &  &  &  &  & \\
Jofotek-Bromnya & [jbr] &  &  &  &  &  &  & \\
Keijar & [kdy] &  &  &  &  &  &  & \\
Kwerba Mamberamo & [xwr] &  &  &  &  &  &  & \\
Kwinsu & [kuc] &  &  &  &  &  &  & \\
Trimuris & [tip] &  &  &  &  &  &  & \\
Wares & [wai] &  &  &  &  &  &  & \\
Yoke & [yki] & X & X &  &  &  &  & \\
\lspbottomrule
\end{tabular}
\begin{styleCaptionxbefore}
\label{bkm:Ref372910275}Table ‎0.\stepcounter{Table}{\theTable}:  OLAC resources available in and about the Austronesian languages spoken in the Sarmi regency
\end{styleCaptionxbefore}

\tablehead{
\multicolumn{2}{l}{ Name \& ISO 639-3 code} & Sur. & Lex. & Gr. & Lit. & Ant. & L1 & \arraybslash NT\\
}
\begin{tabular}{lllllllll}
\lsptoprule
Anus & [auq] &  & X &  &  &  &  & \\
Bonggo & [bpg] &  & X &  &  &  &  & \\
Fedan & [pdn] &  &  &  &  &  &  & \\
Kaptiau & [kbi] &  &  &  &  &  &  & \\
Liki & [lio] &  &  &  &  &  &  & \\
Masimasi & [ism] &  &  &  &  &  &  & \\
Mo & [wkd] &  & X &  &  &  &  & \\
Sobei & [sob] &  & X & X &  & X & X & \\
Sunum & [ynm] &  &  &  &  &  &  & \\
Tarpia & [tpf] &  & X &  &  &  &  & \\
Yarsun & [yrs] &  &  &  &  &  &  & \\
\lspbottomrule
\end{tabular}
%\setcounter{page}{1}\section{Population totals for West Papua}
\label{bkm:Ref370212028}
Table  ‎0 .3 presents the population totals for Papua province and Papua Barat province by regencies (coastal versus interior) and ethnicity (Papuan versus non-Papuan). The figures are based on the 2010 census data ({Bidang Neraca Wilayah dan Analisis Statistik 2011b: 11–14; 2012b: 92}).\footnote{\\
\\
\\
\\
\\
\\
\\
\\
\\
\\
\\
\\
\\
\\
\\
\par Population totals for Papua province are also available at \url{http://papua.bps.go.id/yii/9400/index.php/post/552/Jumlah Penduduk Papua} (accessed 21 Oct 2013), and for Papua Barat province at \url{http://irjabar.bps.go.id/publikasi/2011/Statistik Daerah Provinsi Papua Barat 2011/baca_publikasi.php} (accessed 21 Oct 2013).} (For more details see §1.7.1).


\begin{stylecaption}
\label{bkm:Ref370210564}Table ‎0.\stepcounter{Table}{\theTable}:  Population totals by coastal and interior regencies and ethnicity
\end{stylecaption}

\tablehead{ &  & Regency & Papuan & non-Papuan & \arraybslash Total\\
}
\begin{tabular}{llllll}
\lsptoprule
\multicolumn{6}{l}{Papua province}\\
& \multicolumn{2}{l}{Coastal regencies} & \raggedleft 756,335 & \raggedleft 608,170 & \raggedleft\arraybslash 1,364,505\\
& \multicolumn{2}{l}{Interior regencies} & \raggedleft 1,394,041 & \raggedleft 51,462 & \raggedleft\arraybslash 1,445,503\\
& \multicolumn{2}{l}{Subtotal} & \raggedleft 2,150,376 & \raggedleft 659,632 & \raggedleft\arraybslash 2,810,008\\
\multicolumn{6}{l}{Papua Barat province}\\
& \multicolumn{2}{l}{Coastal regencies} & \raggedleft 373,302 & \raggedleft 354,039 & \raggedleft\arraybslash 727,341\\
& \multicolumn{2}{l}{Interior regencies} & \raggedleft 31,772 & \raggedleft 1,309 & \raggedleft\arraybslash 33,081\\
& \multicolumn{2}{l}{Subtotal} & \raggedleft 405,074 & \raggedleft 355,348 & \raggedleft\arraybslash 760,422\\
\multicolumn{3}{l}{Total} & \raggedleft 2,555,450 & \raggedleft 1,014,980 & \raggedleft\arraybslash 3,570,430\\
\lspbottomrule
\end{tabular}

The total of 3,570,430 in Table  ‎0 .3 more or less matches the total of 3,593,803 provided by {Bidang Neraca Wilayah dan Analisis \citet[92]{Statistik2012b}} (see §1.7.1). The difference of 23,373 is due to a mismatch between the overall population total of 2,833,381 for Papua province {(2012b: 92)} and the population details by regency, religious affiliation and ethnicity, which gives a total of 2,810,008;\footnote{\\
\\
\\
\\
\\
\\
\\
\\
\\
\\
\\
\\
\\
\\
\\
\par This data is available under the category \textstyleChItalic{Sosial Budaya} ‘Social (affairs) and Culture’ at \url{http://papua.bps.go.id/yii/9400/index.php/site/page?view=sp2010} (accessed 21 Oct 2013).} see also Footnote 40 in §1.7.1 (p. \pageref{bkm:Ref438387027}).
\end{styleBodyaftervbefore}


Table  ‎0 .4 lists the population totals by ethnicity for each of the regencies of Papua province and Papua Barat province.
\end{styleBodyvvafter}

\begin{stylecaption}
\label{bkm:Ref370210571}Table ‎0.\stepcounter{Table}{\theTable}:  Population details by regency and ethnicity
\end{stylecaption}

\tablehead{ &  & Regency & Papuan & non-Papuan & \arraybslash Total\\
}
\begin{tabular}{llllll}
\lsptoprule
\multicolumn{6}{l}{Papua province}\\
& \multicolumn{5}{l}{Costal regencies}\\
&  & Asmat & \raggedleft 68,598 & \raggedleft 7,943 & \raggedleft\arraybslash 76,541\\
&  & Biak Numfor & \raggedleft 93,340 & \raggedleft 33,136 & \raggedleft\arraybslash 126,476\\
&  & Jayapura & \raggedleft 68,116 & \raggedleft 42,812 & \raggedleft\arraybslash 110,928\\
&  & Jayapura Kota & \raggedleft 89,164 & \raggedleft 166,465 & \raggedleft\arraybslash 255,629\\
&  & Keerom & \raggedleft 19,698 & \raggedleft 27,873 & \raggedleft\arraybslash 47,571\\
&  & Kepulauan Yapen & \raggedleft 64,034 & \raggedleft 17,969 & \raggedleft\arraybslash 82,003\\
&  & Mamberamo Raya & \raggedleft 17,092 & \raggedleft 1,273 & \raggedleft\arraybslash 18,365\\
&  & Mappi & \raggedleft 72,134 & \raggedleft 9,261 & \raggedleft\arraybslash 81,395\\
&  & Merauke & \raggedleft 72,554 & \raggedleft 122,312 & \raggedleft\arraybslash 194,866\\
&  & Mimika & \raggedleft 71,672 & \raggedleft 96,855 & \raggedleft\arraybslash 168,527\\
&  & Nabire & \raggedleft 61,364 & \raggedleft 67,761 & \raggedleft\arraybslash 129,125\\
&  & Sarmi & \raggedleft 22,890 & \raggedleft 9,695 & \raggedleft\arraybslash 32,585\\
&  & Supiori & \raggedleft 15,297 & \raggedleft 558 & \raggedleft\arraybslash 15,855\\
&  & Waropen & \raggedleft 20,382 & \raggedleft 4,257 & \raggedleft\arraybslash 24,639\\
& \multicolumn{5}{l}{Interior regencies}\\
&  & Boven Digoel & \raggedleft 37,309 & \raggedleft 18,133 & \raggedleft\arraybslash 55,442\\
&  & Deiyai & \raggedleft 61,557 & \raggedleft 538 & \raggedleft\arraybslash 62,095\\
&  & Dogiyai & \raggedleft 83,400 & \raggedleft 830 & \raggedleft\arraybslash 84,230\\
&  & Intan Jaya & \raggedleft 40,413 & \raggedleft 77 & \raggedleft\arraybslash 40,490\\
&  & Jayawijaya & \raggedleft 177,581 & \raggedleft 18,093 & \raggedleft\arraybslash 195,674\\
&  & Lanny Jaya & \raggedleft 148,367 & \raggedleft 155 & \raggedleft\arraybslash 148,522\\
&  & Mamberamo Tengah & \raggedleft 39,329 & \raggedleft 208 & \raggedleft\arraybslash 39,537\\
&  & Nduga & \raggedleft 78,389 & \raggedleft 664 & \raggedleft\arraybslash 79,053\\
&  & Paniai & \raggedleft 147,680 & \raggedleft 3,389 & \raggedleft\arraybslash 151,069\\
&  & Pegunungan Bintang & \raggedleft 62,343 & \raggedleft 3,091 & \raggedleft\arraybslash 65,434\\
&  & Puncak & \raggedleft 92,532 & \raggedleft 686 & \raggedleft\arraybslash 93,218\\
&  & Puncak Jaya & \raggedleft 99,368 & \raggedleft 1,780 & \raggedleft\arraybslash 101,148\\
&  & Tolikara & \raggedleft 113,226 & \raggedleft 1,090 & \raggedleft\arraybslash 114,316\\
&  & Yahukimo & \raggedleft 162,192 & \raggedleft 2,320 & \raggedleft\arraybslash 164,512\\
&  & Yalimo & \raggedleft 50,355 & \raggedleft 408 & \raggedleft\arraybslash 50,763\\
\multicolumn{6}{l}{Papua Barat province}\\
& \multicolumn{5}{l}{Coastal regencies}\\
&  & Fakfak & \raggedleft 36,409 & \raggedleft 30,419 & \raggedleft\arraybslash 66,828\\
&  & Kaimana & \raggedleft 24,412 & \raggedleft 21,837 & \raggedleft\arraybslash 46,249\\
&  & Manokwari & \raggedleft 107,857 & \raggedleft 79,869 & \raggedleft\arraybslash 26,321\\
&  & Raja Ampat & \raggedleft 31,160 & \raggedleft 11,347 & \raggedleft\arraybslash 52,422\\
&  & Sorong & \raggedleft 26,400 & \raggedleft 44,219 & \raggedleft\arraybslash 187,726\\
&  & Sorong Kota & \raggedleft 62,070 & \raggedleft 128,555 & \raggedleft\arraybslash 37,900\\
&  & Sorong Selatan & \raggedleft 30,988 & \raggedleft 6,912 & \raggedleft\arraybslash 70,619\\
&  & Tambrauw & \raggedleft 5,878 & \raggedleft 266 & \raggedleft\arraybslash 42,507\\
&  & Teluk Bintuni & \raggedleft 27,947 & \raggedleft 24,475 & \raggedleft\arraybslash 6,144\\
&  & Teluk Wondama & \raggedleft 20,181 & \raggedleft 6,140 & \raggedleft\arraybslash 190,625\\
& \multicolumn{5}{l}{Interior regencies}\\
&  & Maybrat & \raggedleft 31,772 & \raggedleft 1,309 & \raggedleft\arraybslash 33,081\\
\lspbottomrule
\end{tabular}
%\setcounter{page}{1}\section{Affixation}
\label{bkm:Ref343337536}
The following sections present tables and figures which give the token frequencies by speakers, topics, and interlocutors for the affixes discussed in §3.1. The frequencies for prefix \textscItalBold{ter\-} are given in Appendix A.16, for suffix \textitbf{ang\-} in Appendix A.17, for prefix \textscItalBold{pe(n)\-} in AppendixA.18, for prefix \textscItalBold{ber\-} in AppendixA.19, for suffix \textitbf{\-nya} in AppendixA.20, and for circumfix \textitbf{ke}\textitbf{\-}/\textitbf{\-}\textitbf{ang} in Appendix A.21.
\end{styleBodyxvafter}

\subsection{Prefix \textscItalBold{ter\-}}
\label{bkm:Ref376624909}
The tables and figures give the token frequencies for \textscItalBold{ter\-}prefixed words with bi- and monovalent verbal bases.


\begin{stylecaption}
Table ‎0.\stepcounter{Table}{\theTable}:  Tokens for \textscItalBold{ter\-}prefixed words with bivalent verbal bases (38 items)
\end{stylecaption}

\tablehead{ & \multicolumn{4}{l}{ Topics (\textsc{top})} & \multicolumn{3}{l}{ Interlocutors (\textsc{ilct})} & \arraybslash Tokens\\
}
\begin{tabular}{lllllllll}
\lsptoprule
Speakers & \textsc{pol} & \textsc{edc} & \textsc{rel} & \textsc{low} & \textsc{+stat} & \textsc{\-stat} & \textsc{outsd} & \arraybslash Total\\
\textsc{+edc-spk} & \raggedleft 6 & \raggedleft 10 & \raggedleft 10 & \raggedleft 15 & \raggedleft {}-{}-{}- & \raggedleft {}-{}-{}- & \raggedleft 9 & \raggedleft\arraybslash 50\\
\textsc{\-edc-spk} & \raggedleft 2 & \raggedleft 1 & \raggedleft 26 & \raggedleft {}-{}-{}- & \raggedleft 45 & \raggedleft \textstyleChBold{23} & \raggedleft 6 & \raggedleft\arraybslash 103\\
\textstyleChBold{Total} & \raggedleft 8 & \raggedleft 11 & \raggedleft 36 & \raggedleft 15 & \raggedleft 45 & \raggedleft \textstyleChBold{23} & \raggedleft 15 & \raggedleft\arraybslash 153\\
\lspbottomrule
\end{tabular}
\begin{styleFigure}
  
%%please move the includegraphics inside the {figure} environment
%%\includegraphics[width=\textwidth]{kluge-img36.jpg}
 
\end{styleFigure}

\begin{styleCapFigure}
Figure ‎0.\stepcounter{Figure}{\theFigure}:  Tokens for \textscItalBold{ter\-}prefixed words with bivalent verbal bases
\end{styleCapFigure}

\begin{stylecaption}
Table ‎0.\stepcounter{Table}{\theTable}:  Tokens for \textscItalBold{ter\-}prefixed words with monovalent verbal bases (5 items)
\end{stylecaption}

\tablehead{ & \multicolumn{4}{l}{ Topics (\textsc{top})} & \multicolumn{3}{l}{ Interlocutors (\textsc{ilct})} & \arraybslash Tokens\\
}
\begin{tabular}{lllllllll}
\lsptoprule
Speakers & \textsc{pol} & \textsc{edc} & \textsc{rel} & \textsc{low} & \textsc{+stat} & \textsc{\-stat} & \textsc{outsd} & \arraybslash Total\\
\textsc{+edc-spk} & \raggedleft 0 & \raggedleft 0 & \raggedleft 1 & \raggedleft 4 & \raggedleft {}-{}-{}- & \raggedleft {}-{}-{}- & \raggedleft 0 & \raggedleft\arraybslash 5\\
\textsc{\-edc-spk} & \raggedleft 0 & \raggedleft 1 & \raggedleft 5 & \raggedleft {}-{}-{}- & \raggedleft 2 & \raggedleft \textstyleChBold{1} & \raggedleft 0 & \raggedleft\arraybslash 9\\
\textstyleChBold{Total} & \raggedleft 0 & \raggedleft 1 & \raggedleft 6 & \raggedleft 4 & \raggedleft 2 & \raggedleft \textstyleChBold{1} & \raggedleft 0 & \raggedleft\arraybslash 14\\
\lspbottomrule
\end{tabular}
\begin{styleFigure}
  
%%please move the includegraphics inside the {figure} environment
%%\includegraphics[width=\textwidth]{kluge-img37.jpg}
 
\end{styleFigure}

\begin{styleCapFigurexvafter}
Figure ‎0.\stepcounter{Figure}{\theFigure}:  Tokens for \textscItalBold{ter\-}prefixed words with monovalent verbal bases
\end{styleCapFigurexvafter}

\subsection{Suffix \textitbf{\-ang}}
\label{bkm:Ref376624910}
The tables and figures give the token frequencies for \textitbf{\-ang}{}-suffixed words with verbal, nominal, and quantifier bases.


\begin{stylecaption}
Table ‎0.\stepcounter{Table}{\theTable}:  Tokens for \textitbf{\-ang}{}-suffixed words with verbal bases (69 items)
\end{stylecaption}

\tablehead{ & \multicolumn{4}{l}{ Topics (\textsc{top})} & \multicolumn{3}{l}{ Interlocutors (\textsc{ilct})} & \arraybslash Tokens\\
}
\begin{tabular}{lllllllll}
\lsptoprule
Speakers & \textsc{pol} & \textsc{edc} & \textsc{rel} & \textsc{low} & \textsc{+stat} & \textsc{\-stat} & \textsc{outsd} & \arraybslash Total\\
\textsc{+edc-spk} & \raggedleft 30 & \raggedleft 26 & \raggedleft 15 & \raggedleft 46 & \raggedleft {}-{}-{}- & \raggedleft {}-{}-{}- & \raggedleft 75 & \raggedleft\arraybslash 192\\
\textsc{\-edc-spk} & \raggedleft 15 & \raggedleft 40 & \raggedleft 57 & \raggedleft {}-{}-{}- & \raggedleft 26 & \raggedleft \textstyleChBold{80} & \raggedleft 3 & \raggedleft\arraybslash 211\\
\textstyleChBold{Total} & \raggedleft 45 & \raggedleft 66 & \raggedleft 62 & \raggedleft 46 & \raggedleft 26 & \raggedleft \textstyleChBold{80} & \raggedleft 78 & \raggedleft\arraybslash 403\\
\lspbottomrule
\end{tabular}
\begin{styleFigure}
  
%%please move the includegraphics inside the {figure} environment
%%\includegraphics[width=\textwidth]{kluge-img38.jpg}
 
\end{styleFigure}

\begin{styleCapFigure}
Figure ‎0.\stepcounter{Figure}{\theFigure}:  Tokens for \textitbf{\-ang}{}-suffixed words with verbal bases
\end{styleCapFigure}

\begin{stylecaption}
Table ‎0.\stepcounter{Table}{\theTable}:  Tokens for \-\textitbf{ang}{}-suffixed words with nominal and quantifier bases (15 items)
\end{stylecaption}

\tablehead{ & \multicolumn{4}{l}{ Topics (\textsc{top})} & \multicolumn{3}{l}{ Interlocutors (\textsc{ilct})} & \arraybslash Tokens\\
}
\begin{tabular}{lllllllll}
\lsptoprule
Speakers & \textsc{pol} & \textsc{edc} & \textsc{rel} & \textsc{low} & \textsc{+stat} & \textsc{\-stat} & \textsc{outsd} & \arraybslash Total\\
\textsc{+edc-spk} & \raggedleft 4 & \raggedleft 1 & \raggedleft 0 & \raggedleft 4 & \raggedleft {}-{}-{}- & \raggedleft {}-{}-{}- & \raggedleft 8 & \raggedleft\arraybslash 17\\
\textsc{\-edc-spk} & \raggedleft 3 & \raggedleft 0 & \raggedleft 6 & \raggedleft {}-{}-{}- & \raggedleft 1 & \raggedleft \textstyleChBold{9} & \raggedleft 2 & \raggedleft\arraybslash 21\\
\textstyleChBold{Total} & \raggedleft 7 & \raggedleft 1 & \raggedleft 6 & \raggedleft 4 & \raggedleft 1 & \raggedleft \textstyleChBold{9} & \raggedleft 10 & \raggedleft\arraybslash 38\\
\lspbottomrule
\end{tabular}
\begin{styleFigure}
  
%%please move the includegraphics inside the {figure} environment
%%\includegraphics[width=\textwidth]{kluge-img39.jpg}
 
\end{styleFigure}

\begin{styleCapFigurexvafter}
Figure ‎0.\stepcounter{Figure}{\theFigure}:  Tokens for \-\textitbf{ang}{}-suffixed items derived from nominal bases
\end{styleCapFigurexvafter}

\subsection{Prefix \textscItalBold{pe(n)\-}}
\label{bkm:Ref376624911}
The tables and figures give the token frequencies for \textscItalBold{pe(n)\-}prefixed words with verbal and nominal bases.


\begin{stylecaption}
Table ‎0.\stepcounter{Table}{\theTable}:  Tokens for \textscItalBold{pe(n)\-}prefixed words with verbal bases (29 items)
\end{stylecaption}

\tablehead{ & \multicolumn{4}{l}{ Topics (\textsc{top})} & \multicolumn{3}{l}{ Interlocutors (\textsc{ilct})} & \arraybslash Tokens\\
}
\begin{tabular}{lllllllll}
\lsptoprule
Speakers & \textsc{pol} & \textsc{edc} & \textsc{rel} & \textsc{low} & \textsc{+stat} & \textsc{\-stat} & \textsc{outsd} & \arraybslash Total\\
\textsc{+edc-spk} & \raggedleft 37 & \raggedleft 6 & \raggedleft 3 & \raggedleft 19 & \raggedleft {}-{}-{}- & \raggedleft {}-{}-{}- & \raggedleft 11 & \raggedleft\arraybslash 76\\
\textsc{\-edc-spk} & \raggedleft 11 & \raggedleft 2 & \raggedleft 37 & \raggedleft {}-{}-{}- & \raggedleft 9 & \raggedleft \textstyleChBold{18} & \raggedleft 0 & \raggedleft\arraybslash 77\\
\textstyleChBold{Total} & \raggedleft 48 & \raggedleft 8 & \raggedleft 40 & \raggedleft 19 & \raggedleft 9 & \raggedleft \textstyleChBold{18} & \raggedleft 11 & \raggedleft\arraybslash 153\\
\lspbottomrule
\end{tabular}
\begin{styleFigure}
  
%%please move the includegraphics inside the {figure} environment
%%\includegraphics[width=\textwidth]{kluge-img40.jpg}
 
\end{styleFigure}

\begin{styleCapFigure}
Figure ‎0.\stepcounter{Figure}{\theFigure}:  Tokens for \textscItalBold{pe(n)\-}prefixed words with verbal bases
\end{styleCapFigure}

\begin{stylecaption}
Table ‎0.\stepcounter{Table}{\theTable}:  Tokens for \textscItalBold{pe(n)\-}prefixed words with nominal bases (5 items)
\end{stylecaption}

\tablehead{ & \multicolumn{4}{l}{ Topics (\textsc{top})} & \multicolumn{3}{l}{ Interlocutors (\textsc{ilct})} & \arraybslash Tokens\\
}
\begin{tabular}{lllllllll}
\lsptoprule
Speakers & \textsc{pol} & \textsc{edc} & \textsc{rel} & \textsc{low} & \textsc{+stat} & \textsc{\-stat} & \textsc{outsd} & \arraybslash Total\\
\textsc{+edc-spk} & \raggedleft 10 & \raggedleft 0 & \raggedleft 12 & \raggedleft 5 & \raggedleft {}-{}-{}- & \raggedleft {}-{}-{}- & \raggedleft 0 & \raggedleft\arraybslash 27\\
\textsc{\-edc-spk} & \raggedleft 1 & \raggedleft 2 & \raggedleft 2 & \raggedleft {}-{}-{}- & \raggedleft 0 & \raggedleft \textstyleChBold{1} & \raggedleft 0 & \raggedleft\arraybslash 6\\
\textstyleChBold{Total} & \raggedleft 11 & \raggedleft 2 & \raggedleft 14 & \raggedleft 5 & \raggedleft 0 & \raggedleft \textstyleChBold{1} & \raggedleft 0 & \raggedleft\arraybslash 33\\
\lspbottomrule
\end{tabular}
\begin{styleFigure}
  
%%please move the includegraphics inside the {figure} environment
%%\includegraphics[width=\textwidth]{kluge-img41.jpg}
 
\end{styleFigure}

\begin{styleCapFigurexvafter}
Figure ‎0.\stepcounter{Figure}{\theFigure}:  Tokens for \textscItalBold{pe(n)\-}prefixed words\textstyleCaptionxbeforeChar{ }with nominal bases
\end{styleCapFigurexvafter}

\subsection{Prefix \textscItalBold{ber\-}}
\label{bkm:Ref376624912}
The tables and figures give the token frequencies for \textscItalBold{ber\-}prefixed words with verbal, nominal, numeral, and quantifier bases.


\begin{stylecaption}
Table ‎0.\stepcounter{Table}{\theTable}:  Tokens for \textscItalBold{ber\-}prefixed words with verbal bases (27 items)
\end{stylecaption}

\tablehead{ & \multicolumn{4}{l}{ Topics (\textsc{top})} & \multicolumn{3}{l}{ Interlocutors (\textsc{ilct})} & \arraybslash Tokens\\
}
\begin{tabular}{lllllllll}
\lsptoprule
Speakers & \textsc{pol} & \textsc{edc} & \textsc{rel} & \textsc{low} & \textsc{+stat} & \textsc{\-stat} & \textsc{outsd} & \arraybslash Total\\
\textsc{+edc-spk} & \raggedleft 7 & \raggedleft 22 & \raggedleft 9 & \raggedleft 12 & \raggedleft {}-{}-{}- & \raggedleft {}-{}-{}- & \raggedleft 7 & \raggedleft\arraybslash 57\\
\textsc{\-edc-spk} & \raggedleft 3 & \raggedleft 7 & \raggedleft 5 & \raggedleft {}-{}-{}- & \raggedleft 7 & \raggedleft \textstyleChBold{8} & \raggedleft 7 & \raggedleft\arraybslash 37\\
\textstyleChBold{Total} & \raggedleft 10 & \raggedleft 29 & \raggedleft 14 & \raggedleft 12 & \raggedleft 7 & \raggedleft \textstyleChBold{8} & \raggedleft 14 & \raggedleft\arraybslash 94\\
\lspbottomrule
\end{tabular}
\begin{styleFigure}
  
%%please move the includegraphics inside the {figure} environment
%%\includegraphics[width=\textwidth]{kluge-img42.jpg}
 
\end{styleFigure}

\begin{styleCapFigure}
Figure ‎0.\stepcounter{Figure}{\theFigure}:  Tokens for \textscItalBold{ber\-}prefixed words with verbal bases
\end{styleCapFigure}

\begin{stylecaption}
Table ‎0.\stepcounter{Table}{\theTable}:  Tokens for \textscItalBold{ber\-}prefixed words with nominal, numeral, and quantifier bases (29 items)
\end{stylecaption}

\tablehead{ & \multicolumn{4}{l}{ Topics (\textsc{top})} & \multicolumn{3}{l}{ Interlocutors (\textsc{ilct})} & \arraybslash Tokens\\
}
\begin{tabular}{lllllllll}
\lsptoprule
Speakers & \textsc{pol} & \textsc{edc} & \textsc{rel} & \textsc{low} & \textsc{+stat} & \textsc{\-stat} & \textsc{outsd} & \arraybslash Total\\
\textsc{+edc-spk} & \raggedleft 13 & \raggedleft 9 & \raggedleft 11 & \raggedleft 6 & \raggedleft {}-{}-{}- & \raggedleft {}-{}-{}- & \raggedleft 7 & \raggedleft\arraybslash 46\\
\textsc{\-edc-spk} & \raggedleft 1 & \raggedleft 2 & \raggedleft 6 & \raggedleft {}-{}-{}- & \raggedleft 4 & \raggedleft \textstyleChBold{8} & \raggedleft 3 & \raggedleft\arraybslash 24\\
\textstyleChBold{Total} & \raggedleft 14 & \raggedleft 11 & \raggedleft 17 & \raggedleft 6 & \raggedleft 4 & \raggedleft \textstyleChBold{8} & \raggedleft 10 & \raggedleft\arraybslash 70\\
\lspbottomrule
\end{tabular}
\begin{styleFigure}
  
%%please move the includegraphics inside the {figure} environment
%%\includegraphics[width=\textwidth]{kluge-img43.jpg}
 
\end{styleFigure}

\begin{styleCapFigurexvafter}
Figure ‎0.\stepcounter{Figure}{\theFigure}:  Tokens for \textscItalBold{ber\-}prefixed words with nominal, numeral, and quantifier bases
\end{styleCapFigurexvafter}

\subsection{Suffix \textitbf{\-nya}}
\label{bkm:Ref376624913}
The tables and figures give the token frequencies for \textitbf{\-nya}{}-suffixed words with nominal, verbal, prepositional, and adverbial bases.


\begin{stylecaption}
Table ‎0.\stepcounter{Table}{\theTable}:  Tokens for \textitbf{\-nya}{}-suffixed words with nominal bases (81 items)
\end{stylecaption}

\tablehead{ & \multicolumn{4}{l}{ Topics (\textsc{top})} & \multicolumn{3}{l}{ Interlocutors (\textsc{ilct})} & \arraybslash Tokens\\
}
\begin{tabular}{lllllllll}
\lsptoprule
Speakers & \textsc{pol} & \textsc{edc} & \textsc{rel} & \textsc{low} & \textsc{+stat} & \textsc{\-stat} & \textsc{outsd} & \arraybslash Total\\
\textsc{+edc-spk} & \raggedleft 28 & \raggedleft 21 & \raggedleft 9 & \raggedleft 29 & \raggedleft {}-{}-{}- & \raggedleft {}-{}-{}- & \raggedleft 38 & \raggedleft\arraybslash 125\\
\textsc{\-edc-spk} & \raggedleft 12 & \raggedleft 5 & \raggedleft 14 & \raggedleft {}-{}-{}- & \raggedleft 20 & \raggedleft \textstyleChBold{16} & \raggedleft 23 & \raggedleft\arraybslash 90\\
\textstyleChBold{Total} & \raggedleft 40 & \raggedleft 26 & \raggedleft 23 & \raggedleft 29 & \raggedleft 20 & \raggedleft \textstyleChBold{16} & \raggedleft 61 & \raggedleft\arraybslash 215\\
\lspbottomrule
\end{tabular}
\begin{styleFigure}
  
%%please move the includegraphics inside the {figure} environment
%%\includegraphics[width=\textwidth]{kluge-img44.jpg}
 
\end{styleFigure}

\begin{styleCapFigure}
Figure ‎0.\stepcounter{Figure}{\theFigure}:  Tokens for \textitbf{\-nya}{}-suffixed words with nominal bases
\end{styleCapFigure}

\begin{stylecaption}
Table ‎0.\stepcounter{Table}{\theTable}:  Tokens for \textitbf{\-nya}{}-suffixed words with verbal bases (36 items)
\end{stylecaption}

\tablehead{ & \multicolumn{4}{l}{ Topics (\textsc{top})} & \multicolumn{3}{l}{ Interlocutors (\textsc{ilct})} & \arraybslash Tokens\\
}
\begin{tabular}{lllllllll}
\lsptoprule
Speakers & \textsc{pol} & \textsc{edc} & \textsc{rel} & \textsc{low} & \textsc{+stat} & \textsc{\-stat} & \textsc{outsd} & \arraybslash Total\\
\textsc{+edc-spk} & \raggedleft 5 & \raggedleft 14 & \raggedleft 3 & \raggedleft 11 & \raggedleft {}-{}-{}- & \raggedleft {}-{}-{}- & \raggedleft 16 & \raggedleft\arraybslash 49\\
\textsc{\-edc-spk} & \raggedleft 3 & \raggedleft 3 & \raggedleft 8 & \raggedleft {}-{}-{}- & \raggedleft 9 & \raggedleft \textstyleChBold{8} & \raggedleft 2 & \raggedleft\arraybslash 33\\
\textstyleChBold{Total} & \raggedleft 8 & \raggedleft 17 & \raggedleft 11 & \raggedleft 11 & \raggedleft 9 & \raggedleft \textstyleChBold{8} & \raggedleft 18 & \raggedleft\arraybslash 82\\
\lspbottomrule
\end{tabular}
\begin{styleFigure}
  
%%please move the includegraphics inside the {figure} environment
%%\includegraphics[width=\textwidth]{kluge-img45.jpg}
 
\end{styleFigure}

\begin{styleCapFigure}
Figure ‎0.\stepcounter{Figure}{\theFigure}:  Tokens for \textitbf{\-nya}{}-suffixed words with verbal bases
\end{styleCapFigure}

\begin{stylecaption}
Table ‎0.\stepcounter{Table}{\theTable}:  Tokens for \textitbf{\-nya}{}-suffixed words with prepositional and adverbial bases (5 items)
\end{stylecaption}

\tablehead{ & \multicolumn{4}{l}{ Topics (\textsc{top})} & \multicolumn{3}{l}{ Interlocutors (\textsc{ilct})} & \arraybslash Tokens\\
}
\begin{tabular}{lllllllll}
\lsptoprule
Speakers & \textsc{pol} & \textsc{edc} & \textsc{rel} & \textsc{low} & \textsc{+stat} & \textsc{\-stat} & \textsc{outsd} & \arraybslash Total\\
\textsc{+edc-spk} & \raggedleft 0 & \raggedleft 0 & \raggedleft 0 & \raggedleft 1 & \raggedleft {}-{}-{}- & \raggedleft {}-{}-{}- & \raggedleft 3 & \raggedleft\arraybslash 4\\
\textsc{\-edc-spk} & \raggedleft 1 & \raggedleft 3 & \raggedleft 6 & \raggedleft {}-{}-{}- & \raggedleft 1 & \raggedleft \textstyleChBold{2} & \raggedleft 3 & \raggedleft\arraybslash 16\\
\textstyleChBold{Total} & \raggedleft 1 & \raggedleft 3 & \raggedleft 6 & \raggedleft 1 & \raggedleft 1 & \raggedleft \textstyleChBold{2} & \raggedleft 6 & \raggedleft\arraybslash 20\\
\lspbottomrule
\end{tabular}
\begin{styleFigure}
  
%%please move the includegraphics inside the {figure} environment
%%\includegraphics[width=\textwidth]{kluge-img46.jpg}
 
\end{styleFigure}

\begin{styleCapFigurexvafter}
Figure ‎0.\stepcounter{Figure}{\theFigure}:  Tokens for \textitbf{\-nya}{}-suffixed words with prepositional and adverbial bases
\end{styleCapFigurexvafter}

\subsection{Circumfix \textitbf{ke}\textitbf{\-}/\textitbf{\-}\textitbf{ang}}
\label{bkm:Ref376624914}
The tables and figures give the token frequencies for \textitbf{ke}\textitbf{\-}/\textitbf{\-}\textitbf{ang}{}-circumfixed words with verbal, nominal, numeral, and quantifier bases.


\begin{stylecaption}
Table ‎0.\stepcounter{Table}{\theTable}:  Tokens for \textitbf{ke}\textitbf{\-}/\textitbf{\-}\textitbf{ang}{}-circumfixed words with verbal bases (57 items)
\end{stylecaption}

\tablehead{ & \multicolumn{4}{l}{ Topics (\textsc{top})} & \multicolumn{3}{l}{ Interlocutors (\textsc{ilct})} & \arraybslash Tokens\\
}
\begin{tabular}{lllllllll}
\lsptoprule
Speakers & \textsc{pol} & \textsc{edc} & \textsc{rel} & \textsc{low} & \textsc{+stat} & \textsc{\-stat} & \textsc{outsd} & \arraybslash Total\\
\textsc{+edc-spk} & \raggedleft 14 & \raggedleft 36 & \raggedleft 13 & \raggedleft 38 & \raggedleft {}-{}-{}- & \raggedleft {}-{}-{}- & \raggedleft 43 & \raggedleft\arraybslash 144\\
\textsc{\-edc-spk} & \raggedleft 5 & \raggedleft 19 & \raggedleft 16 & \raggedleft {}-{}-{}- & \raggedleft 35 & \raggedleft \textstyleChBold{12} & \raggedleft 8 & \raggedleft\arraybslash 95\\
\textstyleChBold{Total} & \raggedleft 19 & \raggedleft 55 & \raggedleft 29 & \raggedleft 38 & \raggedleft 35 & \raggedleft \textstyleChBold{12} & \raggedleft 51 & \raggedleft\arraybslash 239\\
\lspbottomrule
\end{tabular}
\begin{styleFigure}
  
%%please move the includegraphics inside the {figure} environment
%%\includegraphics[width=\textwidth]{kluge-img47.jpg}
 
\end{styleFigure}

\begin{styleCapFigure}
Figure ‎0.\stepcounter{Figure}{\theFigure}:  Tokens for \textitbf{ke}\textitbf{\-}/\textitbf{\-}\textitbf{ang}{}-circumfixed words with verbal bases
\end{styleCapFigure}

\begin{stylecaption}
Table ‎0.\stepcounter{Table}{\theTable}:  Tokens for \textitbf{ke}\textitbf{\-}/\textitbf{\-}\textitbf{ang}{}-circumfixed words with nominal, numeral, and quantifier bases (8 items)
\end{stylecaption}

\tablehead{ & \multicolumn{4}{l}{ Topics (\textsc{top})} & \multicolumn{3}{l}{ Interlocutors (\textsc{ilct})} & \arraybslash Tokens\\
}
\begin{tabular}{lllllllll}
\lsptoprule
Speakers & \textsc{pol} & \textsc{edc} & \textsc{rel} & \textsc{low} & \textsc{+stat} & \textsc{\-stat} & \textsc{outsd} & \arraybslash Total\\
\textsc{+edc-spk} & \raggedleft 2 & \raggedleft 10 & \raggedleft 1 & \raggedleft 0 & \raggedleft {}-{}-{}- & \raggedleft {}-{}-{}- & \raggedleft 2 & \raggedleft\arraybslash 15\\
\textsc{\-edc-spk} & \raggedleft 2 & \raggedleft 1 & \raggedleft 0 & \raggedleft {}-{}-{}- & \raggedleft 0 & \raggedleft \textstyleChBold{1} & \raggedleft 0 & \raggedleft\arraybslash 4\\
\textstyleChBold{Total} & \raggedleft 4 & \raggedleft 11 & \raggedleft 1 & \raggedleft 0 & \raggedleft 0 & \raggedleft \textstyleChBold{1} & \raggedleft 2 & \raggedleft\arraybslash 19\\
\lspbottomrule
\end{tabular}
\begin{styleFigure}
  
%%please move the includegraphics inside the {figure} environment
%%\includegraphics[width=\textwidth]{kluge-img48.jpg}
 
\end{styleFigure}

\begin{styleCapFigurexvafter}
Figure ‎0.\stepcounter{Figure}{\theFigure}:  Tokens for \textitbf{ke}\textitbf{\-}/\textitbf{\-}\textitbf{ang}{}-circumfixed words with nominal, numeral, and quantifier bases
\end{styleCapFigurexvafter}

%\setcounter{page}{1}\chapter[References]{References}
\begin{styleCitaviBibliographyEntry}
Abbot, Barbara. 2006. Definite and indefinite. In Brown, Keith (ed.), \textit{Encyclopedia of language and linguistics}, 2\textsuperscript{nd} edn.. Amsterdam: Elsevier Science Ltd. 392–399.
\end{styleCitaviBibliographyEntry}

\begin{styleCitaviBibliographyEntry}
Abrams, Meyer H. \& Harpham, Geoffrey G. 2009. \textit{A glossary of literary terms}, 9\textsuperscript{th} edn. Boston: Wadsworth Cengage Learning.
\end{styleCitaviBibliographyEntry}

\begin{styleCitaviBibliographyEntry}
Adelaar, K. A. 1992. \textit{Proto Malayic: The reconstruction of its phonology and parts of its lexicon and morphology} (Pacific Linguistics C-119). Canberra: Research School of Pacific Studies, The Australian National University.
\end{styleCitaviBibliographyEntry}

\begin{styleCitaviBibliographyEntry}
Adelaar, K. A. 2001. Malay: A short history. In Citro, Maria \& Maria, Luigi S. (eds.), \textit{Alam melayu il mondo maolese: Lingua, storia, cultura} (Oriente Moderno XIX.2). Roma: Istituto per l’Oriente C.A. Nallino, 225–242.
\end{styleCitaviBibliographyEntry}

\begin{styleCitaviBibliographyEntry}
Adelaar, K. A. 2005a. Malayo-Sumbawan. \textit{Oceanic Linguistics} 44(2): 357–388. Online URL: \url{http://www.jstor.org/stable/3623345} (Accessed 8 Jan 2016).
\end{styleCitaviBibliographyEntry}

\begin{styleCitaviBibliographyEntry}
Adelaar, K. A. 2005b. Structural diversity in the Malayic subgroup. In Adelaar, K. A. \& Himmelmann, Nikolaus P. (eds.), \textit{The Austronesian languages of Asia and Madagascar} (Routledge Language Family Series). London: Routledge, 202–226.
\end{styleCitaviBibliographyEntry}

\begin{styleCitaviBibliographyEntry}
Adelaar, K. A. 2005c. The Austronesian languages of Asia and Madagascar: A historical perspective. In Adelaar, K. A. \& Himmelmann, Nikolaus P. (eds.), \textit{The Austronesian languages of Asia and Madagascar} (Routledge Language Family Series). London: Routledge, 1–42.
\end{styleCitaviBibliographyEntry}

\begin{styleCitaviBibliographyEntry}
Adelaar, K. A. \& Prentice, David J. 1996. Malay: Its history, role and spread. In Wurm, Stephen A. \& Mühlhäusler, Peter \& Tryon, Darrell T. (eds.), \textit{Atlas of languages of intercultural communication in the Pacific, Asia, and the Americas} (Trends in Linguistics: Documentation 13). Berlin: Mouton de Gruyter, 673–693.
\end{styleCitaviBibliographyEntry}

\begin{styleCitaviBibliographyEntry}
Agheyisi, Rebecca \& Fishman, Joshua A. 1970. Language attitude studies: A brief survey of methodological approaches. \textit{Anthropological Linguistics} 12(5): 137–157.
\end{styleCitaviBibliographyEntry}

\begin{styleCitaviBibliographyEntry}
Aikhenvald, Alexandra Y. 2007. Typological distinctions in word-formation. In Shopen, Timothy (ed.), \textit{Language typology and syntactic description. Volume 3: Grammatical categories and the lexicon}, 2\textsuperscript{nd} edn.. Cambridge: Cambridge University Press, 1–65.
\end{styleCitaviBibliographyEntry}

\begin{styleCitaviBibliographyEntry}
Aikhenvald, Alexandra Y. \& Stebbins, Tonya. 2007. Languages of New Guinea. In Miyaoka, Osahito \& Sakiyama, Osamu \& Krauss, Michael E. (eds.), \textit{The vanishing languages of the Pacific rim} (Oxford Linguistics). Oxford: Oxford University Press, 239–266.
\end{styleCitaviBibliographyEntry}

\begin{styleCitaviBibliographyEntry}
Ajamiseba, Daniel C. 1984. Kebinekaan bahasa di Irian Jaya. In Koentjaraningrat (ed.), \textit{Irian Jaya: Membangun masyarakat majemuk} (Seri Etnografi Indonesia 5). Jakarta: Penerbit Djambatan, 119–135.
\end{styleCitaviBibliographyEntry}

\begin{styleCitaviBibliographyEntry}
Allerton, David J. 2006. Valency grammar. In Brown, Keith (ed.), \textit{Encyclopedia of language and linguistics}, 2\textsuperscript{nd} edn.. Amsterdam: Elsevier Science Ltd. 301–314.
\end{styleCitaviBibliographyEntry}

\begin{styleCitaviBibliographyEntry}
Alua, Agus A. 2006. \textit{Papua Barat dari pangkuan ke pangkuan: Suatu ikhtisar kronologis} (Seri Pendidikan Politik Papua 1). Jayapura: Sekretariat Presidium Dewan Papua dan Biro Penelitian STFT Fajar Timur.
\end{styleCitaviBibliographyEntry}

\begin{styleCitaviBibliographyEntry}
Ameka, Felix K. 2006. Interjections. In Brown, Keith (ed.), \textit{Encyclopedia of language and linguistics}, 2\textsuperscript{nd} edn.. Amsterdam: Elsevier Science Ltd. 743–746.
\end{styleCitaviBibliographyEntry}

\begin{styleCitaviBibliographyEntry}
Anceaux, Johannes C. no date. \textit{Nieuwguinees Maleis – Nederlands} (KITLV-inventaris 64. Or. 615: Collectie Anceaux, Johannes Cornelis Anceaux (1920-1988) 88). Leiden: KITLV Archives.
\end{styleCitaviBibliographyEntry}

\begin{styleCitaviBibliographyEntry}
Anceaux, Johannes C. \& Veldkamp, F. 1960. \textit{Woordenlijst Maleis-Nederlands-Dani: Naar gegevens van F. Veldkamp bewerkt door J.C. Anceaux} (Rapport: Kantoor voor Bevolkingszaken 140). Hollandia: Dienst van Binnenlandse Zaken, Kantoor voor Bevolkingszaken, Gouvernement van Nederlands-Nieuw-Guinea.
\end{styleCitaviBibliographyEntry}

\begin{styleCitaviBibliographyEntry}
Andaya, Leonard Y. 1993. \textit{The world of Maluku: Eastern Indonesia in the early modern period}. Honolulu: University of Hawai’i Press.
\end{styleCitaviBibliographyEntry}

\begin{styleCitaviBibliographyEntry}
Anderbeck, Karl R. 2007. ISO 639-3 Registration authority: Request for new language code element in ISO 639-3 (Papuan Malay) (ISO 639-3 - Change request documentation for: 2007-183). Dallas: SIL International. Online URL: \url{http://www.sil.org/iso639-3/cr_files/2007-183.pdf} (Accessed 8 Jan 2016).
\end{styleCitaviBibliographyEntry}

\begin{styleCitaviBibliographyEntry}
Anderson, Gregory D. 2011. The velar nasal. In Haspelmath, Martin \& Dryer, Matthew S. \& Gil, David \& Comrie, Bernard (eds.), \textit{The world atlas of language structures}. München: Max Planck Digital Library, 1–8. Online URL: \url{http://wals.info/chapter/9} (Accessed 8 Jan 2016).
\end{styleCitaviBibliographyEntry}

\begin{styleCitaviBibliographyEntry}
Anderson, Stephen R. \& Keenan, Edward L. 1985. Deixis. In Shopen, Timothy (ed.), \textit{Language typology and syntactic description. Volume 3: Grammatical categories and the lexicon}. Cambridge: Cambridge University Press, 259–308.
\end{styleCitaviBibliographyEntry}

\begin{styleCitaviBibliographyEntry}
Andrews, Avery D. 2007. The major functions of the noun phrase. In Shopen, Timothy (ed.), \textit{Language typology and syntactic description. Volume 1: Clause structure}, 2\textsuperscript{nd} edn.. Cambridge: Cambridge University Press, 132–223.
\end{styleCitaviBibliographyEntry}

\begin{styleCitaviBibliographyEntry}
Arakin, Vladimir D. 1963. \textit{Mal’ga\v{s}skij jazyk} (Jazyki zarube\v{z}nogo vostoka i Afriki). Moskva: Vosto\v{c}noj Literatury.
\end{styleCitaviBibliographyEntry}

\begin{styleCitaviBibliographyEntry}
Arakin, Vladimir D. 1981. \textit{Taitânskij âzyk} (\^{A}zyki Narodov Azii i Afriki). Moskva: Izdatel’stvo “Nauka”.
\end{styleCitaviBibliographyEntry}

\begin{styleCitaviBibliographyEntry}
Aronoff, Mark \& Schvaneveldt, Roger. 1978. Testing morphological productivity. \textit{Annals of the New York Academy of Sciences} 318: 106–114. Online URL: \url{http://dx.doi.org/10.1111/j.1749-6632.1978.tb16357.x} (Accessed 8 Jan 2016).
\end{styleCitaviBibliographyEntry}

\begin{styleCitaviBibliographyEntry}
Asher, Robert E. 1994. Glossary. In Asher, Robert E. (ed.), \textit{The Encyclopedia of language and linguistics}. Oxford: Pergamon Press, 5087–5188.
\end{styleCitaviBibliographyEntry}

\begin{styleCitaviBibliographyEntry}
Baayen, R. Harald. 1992. Quantitative aspects of morphological productivity. In Booij, Geert E. \& van Marle, Jaap (eds.), \textit{Yearbook of Morphology 1991}. Dordrecht: Kluwer Academic Publishers, 109–150.
\end{styleCitaviBibliographyEntry}

\begin{styleCitaviBibliographyEntry}
Baker, Colin. 1992. \textit{Attitudes and language} (Multilingual Matters 83). Clevedon: Multilingual Matters.
\end{styleCitaviBibliographyEntry}

\begin{styleCitaviBibliographyEntry}
Bao, Zhiming. 2009. One in Singapore English. \textit{Studies in Language} 33(2): 338–365. Online URL: \url{http://dx.doi.org/10.1075/sl.33.2.05bao} (Accessed 8 Jan 2016).
\end{styleCitaviBibliographyEntry}

\begin{styleCitaviBibliographyEntry}
Bauer, Laurie. 1983. \textit{English word-formation} (Cambridge Textbooks in Linguistics). Cambridge: Cambridge University Press.
\end{styleCitaviBibliographyEntry}

\begin{styleCitaviBibliographyEntry}
Bauer, Laurie. 2001. \textit{Morphological productivity}. Cambridge: Cambridge University Press.
\end{styleCitaviBibliographyEntry}

\begin{styleCitaviBibliographyEntry}
Bauer, Laurie. 2003. \textit{Introducing linguistic morphology}. Edinburgh: Edinburgh University Press.
\end{styleCitaviBibliographyEntry}

\begin{styleCitaviBibliographyEntry}
Bauer, Laurie. 2009. Typology of compounds. In Lieber, Rochelle \& Štekauer, Pavol (eds.), \textit{The Oxford handbook of compounding} (Oxford Handbooks in Linguistics). Oxford: Oxford University Press, 343–356.
\end{styleCitaviBibliographyEntry}

\begin{styleCitaviBibliographyEntry}
Besier, Dominik. 2012. Die Position und Bedeutung des Papua Malay in der Gesellschaft. Hamburg: University of Hamburg. (BA thesis.)
\end{styleCitaviBibliographyEntry}

\begin{styleCitaviBibliographyEntry}
Bhat, Darbhe N. S. 2007. \textit{Pronouns} (Oxford Studies in Typology and Linguistic Theory). Oxford: Oxford University Press.
\end{styleCitaviBibliographyEntry}

\begin{styleCitaviBibliographyEntry}
Bhat, Darbhe N. S. 2011. Third person pronouns and demonstratives. In Haspelmath, Martin \& Dryer, Matthew S. \& Gil, David \& Comrie, Bernard (eds.), \textit{The world atlas of language structures}. München: Max Planck Digital Library, Chapter 43. Online URL: \url{http://wals.info/chapter/43} (Accessed 8 Jan 2016).
\end{styleCitaviBibliographyEntry}

\begin{styleCitaviBibliographyEntry}
Biber, Douglas \& Conrad, Susan \& Leech, Geoffrey N. 2002. \textit{Longman student grammar of spoken and written English}. Harlow: Longman.
\end{styleCitaviBibliographyEntry}

\begin{styleCitaviBibliographyEntry}
Bickel, Balthasar \& Witzlack-Makarevich, Alena. 2008. Referential scales and case alignment: Reviewing the typological evidence. In Richards, Marc \& Malchukov, Andrej L. (eds.), \textit{Scales} (Linguistische Arbeitsberichte LAB 86). Leipzig: Institut für Linguistik, Universität Leipzig, 1–37.
\end{styleCitaviBibliographyEntry}

\begin{styleCitaviBibliographyEntry}
Bidang Neraca Wilayah dan Analisis Statistik. 2011a. \textit{Indikator pendidikan Provinsi Papua 2011}. Jayapura: Badan Pusat Statistik Provinsi Papua. Online URL: \url{http://papua.bps.go.id/arc/2012/indik2011/indik2011.html} (Accessed 22 Oct 2013).
\end{styleCitaviBibliographyEntry}

\begin{styleCitaviBibliographyEntry}
Bidang Neraca Wilayah dan Analisis Statistik. 2011b. \textit{Statistik daerah Provinsi Papua Barat 2011}. Manokwari: Badan Pusat Statistik Provinsi Papua Barat. Online URL: \url{http://papuabarat.bps.go.id/website/pdf_publikasi/Statistik-Daerah-Provinsi-Papua-Barat-2011.pdf} (Accessed 8 Jan 2016).
\end{styleCitaviBibliographyEntry}

\begin{styleCitaviBibliographyEntry}
Bidang Neraca Wilayah dan Analisis Statistik. 2012a. \textit{Papua dalam angka 2012 – Papua in \figref{fig:2012}}. Jayapura: Badan Pusat Statistik Provinsi Papua. Online URL: \url{http://papua.bps.go.id/arc/2012/dda2012/dda2012.html} (Accessed 22 Oct 2013).
\end{styleCitaviBibliographyEntry}

\begin{styleCitaviBibliographyEntry}
Bidang Neraca Wilayah dan Analisis Statistik. 2012b. \textit{Statistik daerah Provinsi Papua 2012}. Jayapura: Badan Pusat Statistik Provinsi Papua Barat. Online URL: \url{http://papua.bps.go.id/arc/2012/statda2012/statda2012.html} (Accessed 21 Oct 2013).
\end{styleCitaviBibliographyEntry}

\begin{styleCitaviBibliographyEntry}
Bink, G. L. 1894. Tocht van den zendeling Bink naar de Humboldts-baai (Met schetskaartje). \textit{Tijdschrift van het Koninklijk Nederlandsch Aardrijkskundig Genootschap} XI: 325–332. Online URL: \url{http://www.columbia.edu/cu/lweb/digital/collections/cul/texts/ldpd_10273444_001/ldpd_10273444_001.pdf} (Accessed 8 Jan 2016).
\end{styleCitaviBibliographyEntry}

\begin{styleCitaviBibliographyEntry}
Bisang, Walter. 2009. On the evolution of complexity: Sometimes less is more in East and mainland Southeast Asia. In Sampson, Geoffrey \& Gil, David \& Trudgill, Peter (eds.), \textit{Language complexity as an evolving variable} (Studies in the Evolution of Language 13). Oxford: Oxford University Press, 34–49.
\end{styleCitaviBibliographyEntry}

\begin{styleCitaviBibliographyEntry}
Blust, Robert A. 1994. The Austronesian settlement of mainland Southeast Asia. In Adams, Karen L. \& Hudak, Thomas J. \& Southeast Asian Linguistics Society (eds.), \textit{Papers from the Second Annual Meeting of the Southeast Asian Linguistics Society, 1992}. Tempe: Program for Southeast Asian Studies, Arizona State University, 25–83. Online URL: \url{http://sealang.net/sala/archives/pdf8/blust1994austronesian.pdf} (Accessed 8 Jan 2016).
\end{styleCitaviBibliographyEntry}

\begin{styleCitaviBibliographyEntry}
Blust, Robert A. 1999. Subgrouping, circularity and extinction: Some issues in Austronesian comparative linguistics. In Zeitoun, Elizabeth \& Li, Rengui (eds.), \textit{Selected papers from the Eighth International Conference on Austronesian Linguistics} (Symposium Series of the Institute of Linguistics (Preparatory Office), Academia Sinica 1). Taipei: Academia Sinica, 31–94.
\end{styleCitaviBibliographyEntry}

\begin{styleCitaviBibliographyEntry}
Blust, Robert A. 2010. The Greater North Borneo Hypothesis. \textit{Oceanic Linguistics} 49(1): 44–118. Online URL: \url{http://www.jstor.org/stable/40783586} (Accessed 8 Jan 2016).
\end{styleCitaviBibliographyEntry}

\begin{styleCitaviBibliographyEntry}
Blust, Robert A. 2012. One mark per word? Some patterns of dissimilation in Austronesian and Australian languages. \textit{Phonology} 29(3): 355 381. Online URL: \url{http://dx.doi.org/10.1017/S0952675712000206} (Accessed 8 Jan 2016).
\end{styleCitaviBibliographyEntry}

\begin{styleCitaviBibliographyEntry}
Blust, Robert A. 2013. \textit{The Austronesian languages} (Asia-Pacific Linguistics: Open Access Monographs A-PL 008), 2\textsuperscript{nd} edn. Canberra: Research School of Pacific and Asian Studies, The Australian National University. Online URL: \url{http://hdl.handle.net/1885/10191} (Accessed 8 Jan 2016).
\end{styleCitaviBibliographyEntry}

\begin{styleCitaviBibliographyEntry}
Booij, Geert E. 1986. Form and meaning in morphology: The case of Dutch ‘agent nouns’. \textit{Linguistics} 24: 503–517. Online URL: \url{http://dx.doi.org/10.1515/ling.1986.24.3.503} (Accessed 8 Jan 2016).
\end{styleCitaviBibliographyEntry}

\begin{styleCitaviBibliographyEntry}
Booij, Geert E. 2002. \textit{The morphology of Dutch} (Oxford Linguistics). Oxford: Oxford University Press.
\end{styleCitaviBibliographyEntry}

\begin{styleCitaviBibliographyEntry}
Booij, Geert E. 2007. \textit{The grammar of words: An introduction to linguistic morphology} (Oxford Textbooks in Linguistics), 2\textsuperscript{nd} edn. Oxford: Oxford University Press.
\end{styleCitaviBibliographyEntry}

\begin{styleCitaviBibliographyEntry}
Booij, Geert E. 2013. Morphology in construction grammar. In Hoffmann, Thomas \& Trousdale, Graeme (eds.), \textit{The Oxford handbook of construction grammar}. Oxford: Oxford University Press, 255–273. Online URL: \url{https://geertbooij.files.wordpress.com/2014/02/booij-2013-morphology-in-cxg.pdf} (Accessed 8 Jan 2016).
\end{styleCitaviBibliographyEntry}

\begin{styleCitaviBibliographyEntry}
Bosch, C. J. 1995. Memorie van overgave van het bestuur der Residentie Ternate door C. J. Bosch aftredende Resident aan P. van der Crab Assistent Resident ter beschikking van den Gouverneur der Moluksche Eilanden, belast met het beheer van even genoemd gewest. In Overweel, Jeroen A. (ed.), \textit{Topics relating to Netherlands New Guinea in Ternate Residency memoranda of transfer and other assorted documents} (Irian Jaya Source Materials 13). Leiden: DSALCUL/IRIS, 28–29.
\end{styleCitaviBibliographyEntry}

\begin{styleCitaviBibliographyEntry}
Bowen, J. D. \& Philippine Center for Language Study. 1965. \textit{Beginning Tagalog: A course for speakers of English}. Berkeley: University of California Press.
\end{styleCitaviBibliographyEntry}

\begin{styleCitaviBibliographyEntry}
Bublitz, Wolfram \& Bednarek, Monika. 2006. Reported speech: Pragmatic aspects. In Brown, Keith (ed.), \textit{Encyclopedia of language and linguistics}, 2\textsuperscript{nd} edn.. Amsterdam: Elsevier Science Ltd. 550–553.
\end{styleCitaviBibliographyEntry}

\begin{styleCitaviBibliographyEntry}
Bureau Cursussen en Vertalingen. 1950. \textit{Beknopte leergang Maleis voor Nieuw-Guinea}. Amsterdam: Koninklijk Instituut voor de Tropen.
\end{styleCitaviBibliographyEntry}

\begin{styleCitaviBibliographyEntry}
Burke, Edmund. 1831. \textit{The annual register, or a view of the history, politics, and literature of the year 1830} (The Annual Register 72). London: Dodsley. Online URL: \url{https://archive.org/details/annualregistero05burkgoog} (Accessed 8 Jan 2016).
\end{styleCitaviBibliographyEntry}

\begin{styleCitaviBibliographyEntry}
Burung, Willem. 2004. Comparisons in Melayu-Papua. Paper presented at the RCLT Workshop on Comparative Constructions La Trobe University. Melbourne, 6 October 2004.
\end{styleCitaviBibliographyEntry}

\begin{styleCitaviBibliographyEntry}
Burung, Willem. 2005. Discourse analysis. Kangaroo Ground: EQUIP Training.
\end{styleCitaviBibliographyEntry}

\begin{styleCitaviBibliographyEntry}
Burung, Willem. 2008a. Melayu Papua – A hidden treasure. Paper presented at the Second International Conference on Language Development, Language Revitalization and Multilingual Education in Ethnolinguistic Communities. Bangkok, 1-3 July 2008. Online URL: \url{http://www.seameo.org/_ld2008/doucments/Presentation_document/MicrosoftWord_Burung_MelayuPapuaAhidden_treasure_with_edits.pdf} (Accessed 8 Jan 2016).
\end{styleCitaviBibliographyEntry}

\begin{styleCitaviBibliographyEntry}
Burung, Willem. 2008b. The prime ‘FEEL’ in Melayu Papua: Cognition, emotion and body.
\end{styleCitaviBibliographyEntry}

\begin{styleCitaviBibliographyEntry}
Burung, Willem. 2009. Melayu Papua: Where have all its speakers gone. Jayapura: Universitas Cenderawasih.
\end{styleCitaviBibliographyEntry}

\begin{styleCitaviBibliographyEntry}
Burung, Willem \& Sawaki, Yusuf W. 2007. On syntactical paradigm of causative constructions in Melayu Papua. Paper presented at the Eleventh International Symposium on Malay/Indonesian Linguistics – ISMIL 11. Manokwari, 6-8 August 2007. Online URL: \url{http://email.eva.mpg.de/~gil/ismil/11/abstracts/BurungSawaki.pdf} (Accessed 8 Jan 2016).
\end{styleCitaviBibliographyEntry}

\begin{styleCitaviBibliographyEntry}
Bussmann, Hadumod. 1996. \textit{Routledge dictionary of language and linguistics}. London: Routledge.
\end{styleCitaviBibliographyEntry}

\begin{styleCitaviBibliographyEntry}
Bussmann, Hadumod. 2000. \textit{Routledge dictionary of language and linguistics} (Routledge Reference). Beijing: Foreign Language Teaching and Research Press.
\end{styleCitaviBibliographyEntry}

\begin{styleCitaviBibliographyEntry}
Butler, Christopher. 2003. \textit{Structure and function: A guide to three major structural-functional theories. Part II: From clause to discourse and beyond} (Studies in Language Companion Series 64). Amsterdam: John Benjamins Publishing Company.
\end{styleCitaviBibliographyEntry}

\begin{styleCitaviBibliographyEntry}
Bybee, Joan L. 2006. From usage to grammar. \textit{Language} 82(4): 711–733. Online URL: \url{http://dx.doi.org/10.1353/lan.2006.0186} (Accessed 8 Jan 2016).
\end{styleCitaviBibliographyEntry}

\begin{styleCitaviBibliographyEntry}
Campbell, George L. 2000a. \textit{Compendium of the world’s languages. Vo. I: Abaza to Kurdish}, 2\textsuperscript{nd} edn. London: Routledge.
\end{styleCitaviBibliographyEntry}

\begin{styleCitaviBibliographyEntry}
Campbell, George L. 2000b. \textit{Compendium of the world’s languages. Volume 2: Ladakhi to Zuni}, 2\textsuperscript{nd} edn. London: Routledge.
\end{styleCitaviBibliographyEntry}

\begin{styleCitaviBibliographyEntry}
Chauvel, Richard. 2002. Papua and Indonesia. Where contending nationalisms meet. In Kingsbury, Damien \& Aveling, Harry (eds.), \textit{Autonomy and disintegration Indonesia}. London: Routledge Curzon Press, 115–127.
\end{styleCitaviBibliographyEntry}

\begin{styleCitaviBibliographyEntry}
Churchward, C. M. 1953. \textit{Tongan grammar}. London: Oxford University Press.
\end{styleCitaviBibliographyEntry}

\begin{styleCitaviBibliographyEntry}
Clouse, Duane A. 2000. Papuan Malay – What is it! Paper presented at the SIL Malay Seminar. Yogyakarta, 8-10 January 2000.
\end{styleCitaviBibliographyEntry}

\begin{styleCitaviBibliographyEntry}
Clouse, Duane A. \& Donohue, Mark \& Ma, Felix. 2002. Survey report of the North Coast of Irian Jaya. \textit{SIL Electronic Survey Reports} 2002-078: 18 p. Online URL: \url{http://www.sil.org/silesr/2002/SILESR2002-078.pdf} (Accessed 8 Jan 2016).
\end{styleCitaviBibliographyEntry}

\begin{styleCitaviBibliographyEntry}
Collins, James T. 1980. \textit{Ambonese Malay and creolization theory} (Penerbitan Ilmiah 5). Kuala Lumpur: Dewan Bahasa dan Pustaka; Kementarian Pelajaran Malaysia.
\end{styleCitaviBibliographyEntry}

\begin{styleCitaviBibliographyEntry}
Collins, James T. 1998. \textit{Malay, world language: A short history}, 2\textsuperscript{nd} edn. Kuala Lumpur: Dewan Bahasa dan Pustaka.
\end{styleCitaviBibliographyEntry}

\begin{styleCitaviBibliographyEntry}
Commissie voor Nieuw Guinea \& van der Goes, H. D. A. \& Johan Hendrik Croockewit \& Commissie voor Nieuw Guinea, Netherlands. 1862. \textit{Nieuw Guinea, ethnographisch en natuurkundig onderzocht en beschreven in 1858}. Amsterdam: Frederik Muller. Online URL: \url{http://books.google.nl/books/download/Nieuw_Guinea_ethnographisch_en_natuurkun.pdf?id=CgAvAAAAYAAJ & output=pdf & sig=ACfU3U1Lv1gqcqbtPpihc6Yf8u8V_ipehQ} (Accessed 8 Jan 2016).
\end{styleCitaviBibliographyEntry}

\begin{styleCitaviBibliographyEntry}
Comrie, Bernard. 1989. \textit{Language universals and linguistic typology: Syntax and morphology}, 2\textsuperscript{nd} edn. Chicago: University of Chicago Press.
\end{styleCitaviBibliographyEntry}

\begin{styleCitaviBibliographyEntry}
Comrie, Bernard \& Thompson, Sandra A. 2007. Lexical nominalization. In Shopen, Timothy (ed.), \textit{Language typology and syntactic description. Volume 3: Grammatical categories and the lexicon}, 2\textsuperscript{nd} edn.. Cambridge: Cambridge University Press, 334–381.
\end{styleCitaviBibliographyEntry}

\begin{styleCitaviBibliographyEntry}
Cooper, Robert L. \& Fishman, Joshua A. 1974. The study of language attitudes. \textit{Linguistics} 136: 5–19.
\end{styleCitaviBibliographyEntry}

\begin{styleCitaviBibliographyEntry}
Cristofaro, Sonia. 2005. \textit{Subordination}. Oxford: Oxford University Press.
\end{styleCitaviBibliographyEntry}

\begin{styleCitaviBibliographyEntry}
Croft, William. 1991. \textit{Syntactic categories and grammatical relations: The cognitive organization of information}. Chicago: University of Chicago Press.
\end{styleCitaviBibliographyEntry}

\begin{styleCitaviBibliographyEntry}
Crystal, David. 2008. \textit{A dictionary of linguistics and phonetics} (The Language Library), 6\textsuperscript{th} edn. Malden: Basil Blackwell Publishers.
\end{styleCitaviBibliographyEntry}

\begin{styleCitaviBibliographyEntry}
Daniel, Michael. 2011. Plurality in independent personal pronouns. In Haspelmath, Martin \& Dryer, Matthew S. \& Gil, David \& Comrie, Bernard (eds.), \textit{The world atlas of language structures}. München: Max Planck Digital Library, Chapter 35. Online URL: \url{http://wals.info/chapter/35} (Accessed 8 Jan 2016).
\end{styleCitaviBibliographyEntry}

\begin{styleCitaviBibliographyEntry}
Daniel, Michael \& Moravcsik, Edith A. 2011. The associative plural. In Haspelmath, Martin \& Dryer, Matthew S. \& Gil, David \& Comrie, Bernard (eds.), \textit{The world atlas of language structures}. München: Max Planck Digital Library, Chapter 36. Online URL: \url{http://wals.info/chapter/36} (Accessed 8 Jan 2016).
\end{styleCitaviBibliographyEntry}

\begin{styleCitaviBibliographyEntry}
De Lacy, Paul V. 2006. \textit{Markedness: Reduction and preservation in phonology} (Cambridge Studies in Linguistics 112). Cambridge: Cambridge University Press.
\end{styleCitaviBibliographyEntry}

\begin{styleCitaviBibliographyEntry}
De Saussure, Ferdinand. 1959. \textit{Course in general linguistics (edited by Charles Bally and Albert Sechehaye in collaboration with Albert Reidlinger; translated from the French by Wade Baskin)}, 3\textsuperscript{rd} edn. New York: Philosophical Library.
\end{styleCitaviBibliographyEntry}

\begin{styleCitaviBibliographyEntry}
De Vries, Lourens. 2005. Towards a typology of tail–head linkage in Papuan languages. \textit{Studies in Language} 29(2): 363–384. Online URL: \url{http://dx.doi.org/10.1075/sl.29.2.04vri} (Accessed 8 Jan 2016).
\end{styleCitaviBibliographyEntry}

\begin{styleCitaviBibliographyEntry}
Diessel, Holger. 1999. \textit{Demonstratives: Form, function, and grammaticalization} (Typological Studies in Language 42). Amsterdam: John Benjamins Publishing Company.
\end{styleCitaviBibliographyEntry}

\begin{styleCitaviBibliographyEntry}
Diessel, Holger. 2006. Demonstratives. In Brown, Keith (ed.), \textit{Encyclopedia of language and linguistics}, 2\textsuperscript{nd} edn.. Amsterdam: Elsevier Science Ltd. 430–435.
\end{styleCitaviBibliographyEntry}

\begin{styleCitaviBibliographyEntry}
Dik, Simon C. \& Hengeveld, Kees. 1997. \textit{The theory of functional grammar. Part 2: Complex and derived constructions} (Functional Grammar Series 21), 2\textsuperscript{nd} edn. Berlin: Mouton de Gruyter.
\end{styleCitaviBibliographyEntry}

\begin{styleCitaviBibliographyEntry}
Dixon, Robert M. W. 1979. Ergativity. \textit{Language} 55: 59–138. Online URL: \url{http://www.jstor.org/stable/412519} (Accessed 8 Jan 2016).
\end{styleCitaviBibliographyEntry}

\begin{styleCitaviBibliographyEntry}
Dixon, Robert M. W. 1994. Adjectives. In Asher, Robert E. (ed.), \textit{The Encyclopedia of language and linguistics}. Oxford: Pergamon Press, 29–34.
\end{styleCitaviBibliographyEntry}

\begin{styleCitaviBibliographyEntry}
Dixon, Robert M. W. 2004. Adjective classes in typological perspective. In Dixon, Robert M. W. \& Aikhenvald, Alexandra Y. (eds.), \textit{Adjective classes: A cross-linguistic typology} (Explorations in Linguistic Typology 1). Oxford: Oxford University Press, 1–49.
\end{styleCitaviBibliographyEntry}

\begin{styleCitaviBibliographyEntry}
Dixon, Robert M. W. 2008. Comparative constructions: A cross-linguistic typology. \textit{Studies in Language} 32(4): 787–817. Online URL: \url{http://dx.doi.org/10.1075/sl.32.4.02dix} (Accessed 8 Jan 2016).
\end{styleCitaviBibliographyEntry}

\begin{styleCitaviBibliographyEntry}
Dixon, Robert M. W. 2010a. \textit{Basic linguistic theory. Volume 2: Grammatical topics}. Oxford: Oxford University Press.
\end{styleCitaviBibliographyEntry}

\begin{styleCitaviBibliographyEntry}
Dixon, Robert M. W. 2010b. \textit{Basic linguistic theory. Volume 3: Further grammatical topics}. Oxford: Oxford University Press.
\end{styleCitaviBibliographyEntry}

\begin{styleCitaviBibliographyEntry}
Dixon, Robert M. W. \& Aikhenvald, Alexandra Y. 2009. \textit{The semantics of clause linking: A cross-linguistic typology} (Explorations in Linguistic Typology 5). Oxford: Oxford University Press.
\end{styleCitaviBibliographyEntry}

\begin{styleCitaviBibliographyEntry}
Doetjes, Jenny. 2007. Adverbs and quantification: Degrees versus frequency. \textit{Lingua} 117: 685–720. Online URL: \url{http://dx.doi.org/10.1016/j.lingua.2006.04.003} (Accessed 8 Jan 2016).
\end{styleCitaviBibliographyEntry}

\begin{styleCitaviBibliographyEntry}
Donohue, Mark. 1997. Merauke Malay: Some observations on contact and change. Manchester: Department of Linguistics, University of Manchester.
\end{styleCitaviBibliographyEntry}

\begin{styleCitaviBibliographyEntry}
Donohue, Mark. 2003. Papuan Malay. Singapore: National University of Singapore. Online URL: \url{http://papuan.linguistics.anu.edu.au/Donohue/downloads/Donohue_2003_PapuanMalay.pdf} (Accessed 8 Jan 2016).
\end{styleCitaviBibliographyEntry}

\begin{styleCitaviBibliographyEntry}
Donohue, Mark. 2005a. Numerals and their position in Universal Grammar. \textit{Journal of Universal Language} 6(2): 1–37.
\end{styleCitaviBibliographyEntry}

\begin{styleCitaviBibliographyEntry}
Donohue, Mark. 2005b. Voice in some eastern Malay varieties. Paper presented at the Ninth International Symposium on Malay/Indonesian Linguistics – ISMIL 9. Ambun Pagi, 27-29 July 2005. Online URL: \url{http://email.eva.mpg.de/~gil/ismil/9/abstracts.zip} (Accessed 8 Jan 2016).
\end{styleCitaviBibliographyEntry}

\begin{styleCitaviBibliographyEntry}
Donohue, Mark. 2007a. Malay as a mirror of Austronesian: Voice development and voice variation. \textit{Lingua} 118(10): 1470–1499. Online URL: \url{http://dx.doi.org/10.1016/j.lingua.2007.08.007} (Accessed 8 Jan 2016).
\end{styleCitaviBibliographyEntry}

\begin{styleCitaviBibliographyEntry}
Donohue, Mark. 2007b. Variation in voice in Indonesian/Malay: Historical and synchronic perspectives. In Matsumoto, Yoshiko \& Oshima, David Y. \& Robinson, Orrin R. \& Sells, Peter (eds.), \textit{Diversity in language: Perspectives and implications} (CSLI Lecture Notes 176). Stanford: CSLI Publications, 71–129.
\end{styleCitaviBibliographyEntry}

\begin{styleCitaviBibliographyEntry}
Donohue, Mark. 2007c. Word order in Austronesian from north to south and west to east. \textit{Linguistic Typology} 11(2): 349–391. Online URL: \url{http://dx.doi.org/10.1515/LINGTY.2007.026} (Accessed 8 Jan 2016).
\end{styleCitaviBibliographyEntry}

\begin{styleCitaviBibliographyEntry}
Donohue, Mark. 2011. Papuan Malay of New Guinea: Melanesian influence on verb and clause structure. In Lefebvre, Claire (ed.), \textit{Creoles, their substrates, and language typology} (Typological Studies in Language 95). Amsterdam: John Benjamins Publishing Company, 413–435.
\end{styleCitaviBibliographyEntry}

\begin{styleCitaviBibliographyEntry}
Donohue, Mark \& Grimes, Charles E. 2008. Yet more on the position of the languages of Eastern Indonesia and East Timor. \textit{Oceanic Linguistics} 47(1): 114–158. Online URL: \url{http://www.jstor.org/stable/20172341} (Accessed 8 Jan 2016).
\end{styleCitaviBibliographyEntry}

\begin{styleCitaviBibliographyEntry}
Donohue, Mark \& Sawaki, Yusuf W. 2007. Papuan Malay pronominals: Forms and functions. \textit{Oceanic Linguistics} 46(1): 253–276. Online URL: \url{http://www.jstor.org/stable/4499988} (Accessed 8 Jan 2016).
\end{styleCitaviBibliographyEntry}

\begin{styleCitaviBibliographyEntry}
Donohue, Mark \& Smith, John C. 1998. What’s happened to us? Some developments in the Malay pronoun system. \textit{Oceanic Linguistics} 37(1): 65–84. Online URL: \url{http://www.jstor.org/stable/3623280} (Accessed 8 Jan 2016).
\end{styleCitaviBibliographyEntry}

\begin{styleCitaviBibliographyEntry}
Dooley, Robert A. \& Levinsohn, Stephen H. 2001. \textit{Analyzing discourse: A manual of basic concepts}. Dallas: SIL International.
\end{styleCitaviBibliographyEntry}

\begin{styleCitaviBibliographyEntry}
Dryer, Matthew S. 2007a. Clause types. In Shopen, Timothy (ed.), \textit{Language typology and syntactic description. Volume 1: Clause structure}, 2\textsuperscript{nd} edn.. Cambridge: Cambridge University Press, 224–275.
\end{styleCitaviBibliographyEntry}

\begin{styleCitaviBibliographyEntry}
Dryer, Matthew S. 2007b. Noun phrase structure. In Shopen, Timothy (ed.), \textit{Language typology and syntactic description. Volume 2: Complex constructions}, 2\textsuperscript{nd} edn.. Cambridge: Cambridge University Press, 151–205.
\end{styleCitaviBibliographyEntry}

\begin{styleCitaviBibliographyEntry}
Dryer, Matthew S. 2007c. Word order. In Shopen, Timothy (ed.), \textit{Language typology and syntactic description. Volume 1: Clause structure}, 2\textsuperscript{nd} edn.. Cambridge: Cambridge University Press, 61–131.
\end{styleCitaviBibliographyEntry}

\begin{styleCitaviBibliographyEntry}
Dumont d‘Urville, Jules-Sébastien-César. 1833. \textit{Voyage de d\'{e}couvertes autour du monde et \`{a} la recherche de La P\'{e}rouse: Ex\'{e}cut\'{e} sous son commandement et par ordre du gouvernement, sur la corvette l’Astrolabe, pendant les ann\'{e}es 1826, 1827, 1828 et 1829. Histoire du voyage, Tome Quatrième}. Paris: La Librairie Encyclopédique de Roret. Online URL: \url{http://books.google.nl/books/download/Voyage_de_découvertes_autour_du_monde_e.pdf?id=0eg_AAAAcAAJ & output=pdf & sig=ACfU3U1rV1mJTLlb0VUe2dkx1ftK0l7nNQ} (Accessed 8 Jan 2016).
\end{styleCitaviBibliographyEntry}

\begin{styleCitaviBibliographyEntry}
Embassy of the Republic of Indonesia in London. 2009. Issues and perspectives. London: Embassy of the Republic of Indonesia in London. Online URL: \url{http://www.indonesianembassy.org.uk/transmigration-1.htm} (Accessed 8 Jan 2016).
\end{styleCitaviBibliographyEntry}

\begin{styleCitaviBibliographyEntry}
Encyclopædia Britannica Inc. 2001a-. New Guinea (island, Malay Archipelago). In Encyclopædia Britannica Inc. (ed.), \textit{Encyclopædia Britannica online}. Chicago: Encyclopædia Britannica. Online URL: \url{http://www.britannica.com/EBchecked/topic/411548/New-Guinea} (Accessed 8 Jan 2016).
\end{styleCitaviBibliographyEntry}

\begin{styleCitaviBibliographyEntry}
Encyclopædia Britannica Inc. 2001b-. Papua (province, Indonesia). In Encyclopædia Britannica Inc. (ed.), \textit{Encyclopædia Britannica online}. Chicago: Encyclopædia Britannica. Online URL: \url{http://www.britannica.com/EBchecked/topic/293960/Papua} (Accessed 8 Jan 2016).
\end{styleCitaviBibliographyEntry}

\begin{styleCitaviBibliographyEntry}
Englebretson, Robert. 2003. \textit{Searching for structure: The problem of complementation in colloquial Indonesian conversation} (Studies in Discourse and Grammar 13). Amsterdam: John Benjamins Publishing Company.
\end{styleCitaviBibliographyEntry}

\begin{styleCitaviBibliographyEntry}
Englebretson, Robert. 2007. Grammatical resources for social purposes: Some aspects of stancetaking in colloquial Indonesian conversation. In Englebretson, Robert (ed.), \textit{Stancetaking in discourse: Subjectivity, evaluation, interaction} (Pragmatics \& Beyond 164). Amsterdam: John Benjamins Publishing Company, 69–110.
\end{styleCitaviBibliographyEntry}

\begin{styleCitaviBibliographyEntry}
Errington, Joseph. 2001. Colonial linguistics. \textit{Annual Review of Anthropology} 30: 19–39. Online URL: \url{http://dx.doi.org/10.1146/annurev.anthro.30.1.19} (Accessed 8 Jan 2016).
\end{styleCitaviBibliographyEntry}

\begin{styleCitaviBibliographyEntry}
Fasold, Ralph W. 1984. \textit{The sociolinguistics of society} (Language in Society 5). Oxford: Basil Blackwell Publishers.
\end{styleCitaviBibliographyEntry}

\begin{styleCitaviBibliographyEntry}
Fearnside, Philip M. 1997. Transmigration in Indonesia: Lessons from its environmental and social impacts. \textit{Environmental Management} 21(4): 553–570. Online URL: \url{https://www.academia.edu/1196557/Transmigration_in_Indonesia_Lessons_from_its_environmental_and_social_impacts} (Accessed 8 Jan 2016).
\end{styleCitaviBibliographyEntry}

\begin{styleCitaviBibliographyEntry}
Ferguson, Charles A. 1972. Diglossia. In Giglioli, Pier P. (ed.), \textit{Language and social context: Selected readings}. Harmondsworth: Penguin Press, 232–251 [Reprint of Word 15: 325-340 (1959)].
\end{styleCitaviBibliographyEntry}

\begin{styleCitaviBibliographyEntry}
Filimonova, Elena. 2005. \textit{Clusivity: Typology and case studies of inclusive-exclusive distinction} (Typological Studies in Language 63). Amsterdam: John Benjamins Publishing Company.
\end{styleCitaviBibliographyEntry}

\begin{styleCitaviBibliographyEntry}
Fishman, Joshua A. 1965. Who speaks what language to whom and when? \textit{La Linguistique} 1(2): 67–88.
\end{styleCitaviBibliographyEntry}

\begin{styleCitaviBibliographyEntry}
Foley, William A. 1986. \textit{The Papuan languages of New Guinea} (Cambridge Language Surveys). Cambridge: Cambridge University Press.
\end{styleCitaviBibliographyEntry}

\begin{styleCitaviBibliographyEntry}
Foley, William A. 2000. The languages of New Guinea. \textit{Annual Review of Anthropology} 29: 357–404. Online URL: \url{http://dx.doi.org/10.1146/annurev.anthro.29.1.357} (Accessed 8 Jan 2016).
\end{styleCitaviBibliographyEntry}

\begin{styleCitaviBibliographyEntry}
Gal, Susan \& Irvine, Judith T. 1995. The boundaries of languages and disciplines: How ideologies construct difference. \textit{Social Research} 62(4): 967–1001. Online URL: \url{http://www.jstor.org/stable/40971131} (Accessed 8 Jan 2016).
\end{styleCitaviBibliographyEntry}

\begin{styleCitaviBibliographyEntry}
Gil, David. 1994. The structure of Riau Indonesian. \textit{Nordic Journal of Linguistics} 17: 179–200.
\end{styleCitaviBibliographyEntry}

\begin{styleCitaviBibliographyEntry}
Gil, David. 1999. The grammaticalization of punya in Malay/Indonesian dialects. Paper presented at the Ninth Annual Meeting of the South-East Asian Linguistics Society. University of California, Berkeley, 22 May.
\end{styleCitaviBibliographyEntry}

\begin{styleCitaviBibliographyEntry}
Gil, David. 2001a. Commentary: Creoles, complexity, and Riau Indonesian. \textit{Linguistic Typology} 5(2-3): 325–371. Online URL: \url{http://dx.doi.org/10.1515/lity.2001.002} (Accessed 8 Jan 2016).
\end{styleCitaviBibliographyEntry}

\begin{styleCitaviBibliographyEntry}
Gil, David. 2001b. Quantifiers. In Haspelmath, Martin \& König, Ekkehard \& Oesterreicher, Wulf \& Raible, Wolfgang (eds.), \textit{Language typology and language universals: An international handbook. Volume 2} (Handbücher zur Sprach- und Kommunikationswissenschaft 20). Berlin: Walter de Gruyter, 1275–1294.
\end{styleCitaviBibliographyEntry}

\begin{styleCitaviBibliographyEntry}
Gil, David. 2009. Associative plurals and inclusories in Malay/Indonesian. Paper presented at the Thirteenth Symposium on Malay and Indonesian Linguistics – ISMIL 13. Senggigi, 6-7 June 2009. Online URL: \url{http://email.eva.mpg.de/~gil/ismil/13/abstracts/Gil abstract ISMIL 13.pdf} (Accessed 8 Jan 2016).
\end{styleCitaviBibliographyEntry}

\begin{styleCitaviBibliographyEntry}
Gil, David. 2011. Conjunctions and universal quantifiers. In Haspelmath, Martin \& Dryer, Matthew S. \& Gil, David \& Comrie, Bernard (eds.), \textit{The world atlas of language structures}. München: Max Planck Digital Library, 1–4. Online URL: \url{http://wals.info/chapter/56} (Accessed 8 Jan 2016).
\end{styleCitaviBibliographyEntry}

\begin{styleCitaviBibliographyEntry}
Gil, David. 2013. Riau Indonesian: A language without nouns and verbs. In Rijkhoff, Jan \& van Lier, Eva (eds.), \textit{Flexible word classes: Typological studies of underspecified parts of speech}. Oxford: Oxford University Press.
\end{styleCitaviBibliographyEntry}

\begin{styleCitaviBibliographyEntry}
Gil, David. 2014. Tanah Papua (Diversity Linguistics Comment - Language Structures throughout the World 655). Marseille: hypothesis. Online URL: \url{http://dlc.hypotheses.org/655} (Accessed 8 Jan 2016).
\end{styleCitaviBibliographyEntry}

\begin{styleCitaviBibliographyEntry}
Gil, David \& Tadmor, Uri. 1997. Towards a typology of Malay/Indonesian dialects. Paper presented at the First Symposium on Malay and Indonesian Linguistics – ISMIL 1. Penang, 14-15 January 1997. Online URL: \url{http://www.udel.edu/pcole/penang/abstracts.html#Towards a Typology of Malay/Ind} (Accessed 8 Jan 2016).
\end{styleCitaviBibliographyEntry}

\begin{styleCitaviBibliographyEntry}
Givón, Talmy. 2001. \textit{Syntax: An introduction. Volume 1}. Amsterdam: John Benjamins Publishing Company.
\end{styleCitaviBibliographyEntry}

\begin{styleCitaviBibliographyEntry}
Goodman, Thomas. 2002. The Rajas of Papua and East Seram during the early modern period (17\textsuperscript{th} – 18\textsuperscript{th} centuries): A bibliographic essay. \textit{Papuaweb’s Annotated Bibliographies}: 11 p. Online URL: \url{http://www.papuaweb.org/bib/abib/goodman.pdf} (Accessed 8 Jan 2016).
\end{styleCitaviBibliographyEntry}

\begin{styleCitaviBibliographyEntry}
Greenberg, Joseph H. 1978. Generalizations about numerals systems. In Greenberg, Joseph H. \& Ferguson, Charles A. \& Moravcsik, Andrew (eds.), \textit{Universals of human language. Volume 3: Word structure}. Stanford: Stanford University Press, 249–295.
\end{styleCitaviBibliographyEntry}

\begin{styleCitaviBibliographyEntry}
Grimes, Barbara D. 1991. The development and use of Ambonese Malay. In Steinhauer, Hein (ed.), \textit{Papers in Austronesian linguistics. Volume 1} (Pacific Linguistics A-81). Canberra: Research School of Pacific Studies, The Australian National University, 83–123.
\end{styleCitaviBibliographyEntry}

\begin{styleCitaviBibliographyEntry}
Grimes, Charles E. \& Jacob, June A. (eds.) 2008. \textit{Kupang Malay Online Dictionary}. Kupang: UBB-GMIT. Online URL: \href{http://e-kamus2.org/Kupang Malay Lexicon/lexicon/main.htm}{http://e-kamus2.org/Kupang Malay Lexicon/lexicon/main.htm} (Accessed 8 Jan 2016).
\end{styleCitaviBibliographyEntry}

\begin{styleCitaviBibliographyEntry}
Grimes, Joseph E. 1975. \textit{The thread of discourse} (Janua Linguarum: Series Minor 207). The Hague: Mouton de Gruyter.
\end{styleCitaviBibliographyEntry}

\begin{styleCitaviBibliographyEntry}
Grosjean, François. 1992. Another view of bilingualism. In Harris, Richard J. (ed.), \textit{Cognitive processing in bilinguals} (Advances in Psychology 83). Amsterdam: Elsevier Science Ltd. 51–62.
\end{styleCitaviBibliographyEntry}

\begin{styleCitaviBibliographyEntry}
Haga, A. 1884. \textit{Nederlandsch Nieuw Guinea en de Papoesche Eilanden: Historische Bijdrage. Eerste Deel ±1500-1817}. Batavia: W. Bruining \& Company. Online URL: \url{https://ia600300.us.archive.org/25/items/nederlandschnie00wetegoog/nederlandschnie00wetegoog.pdf} (Accessed 8 Jan 2016).
\end{styleCitaviBibliographyEntry}

\begin{styleCitaviBibliographyEntry}
Haga, A. 1885. Het rapport van H. Zwaardecroon en C. Chasteleijn betreffende de reis naar Nieuw Guinea in 1705 ondernomen door Jacob Weyland. \textit{Tijdschrift voor Indische Taal-, Land- en Volkenkunde} 30: 235–258.
\end{styleCitaviBibliographyEntry}

\begin{styleCitaviBibliographyEntry}
Hale, Kenneth L. 1973. Person marking in Walbiri. In Anderson, Stephen R. \& Kiparsky, Paul (eds.), \textit{A festschrift for Morris Halle}. New York: Holt, Rinehardt and Winston, 308–344.
\end{styleCitaviBibliographyEntry}

\begin{styleCitaviBibliographyEntry}
Hartanti. 2008. A sociolinguistics analysis on SMS texts on Papuan Malay: A case study of students SMS texts of semester VIII of faculty of letters. Manokwari: Universitas Negeri Papua.
\end{styleCitaviBibliographyEntry}

\begin{styleCitaviBibliographyEntry}
Haser, Verena \& Kortmann, Bernd. 2006. Adverbs. In Brown, Keith (ed.), \textit{Encyclopedia of language and linguistics}, 2\textsuperscript{nd} edn.. Amsterdam: Elsevier Science Ltd. 66–69.
\end{styleCitaviBibliographyEntry}

\begin{styleCitaviBibliographyEntry}
Haspelmath, Martin. 2004. Coordinating constructions: An overview. In Haspelmath, Martin (ed.), \textit{Coordinating constructions} (Typological Studies in Language 58). Amsterdam: John Benjamins Publishing Company, 3–39.
\end{styleCitaviBibliographyEntry}

\begin{styleCitaviBibliographyEntry}
Haspelmath, Martin. 2007a. Coordination. In Shopen, Timothy (ed.), \textit{Language typology and syntactic description. Volume 2: Complex constructions}, 2\textsuperscript{nd} edn.. Cambridge: Cambridge University Press, 1–51.
\end{styleCitaviBibliographyEntry}

\begin{styleCitaviBibliographyEntry}
Haspelmath, Martin. 2007b. Ditransitive alignment splits and inverse alignment. \textit{Functions of Language} 14(1): 79–102. Online URL: \url{http://dx.doi.org/10.1075/fol.14.1.06has} (Accessed 8 Jan 2016).
\end{styleCitaviBibliographyEntry}

\begin{styleCitaviBibliographyEntry}
Haspelmath, Martin. 2007c. Further remarks on reciprocal constructions. In Nedjalkov, Vladimir P. (ed.), \textit{Reciprocal constructions (Volume 4)} (Typological Studies in Language 71). Philadelphia: John Benjamins Publishing Company, 2087–2115.
\end{styleCitaviBibliographyEntry}

\begin{styleCitaviBibliographyEntry}
Hawkins, John A. 1983. \textit{Word order universals} (Quantitative Analyses of Linguistic Structure 3). New York: Academic Press.
\end{styleCitaviBibliographyEntry}

\begin{styleCitaviBibliographyEntry}
Hay, Jennifer. 2001. Lexical frequency in morphology: Is everything relative? \textit{Linguistics} 39(6): 1041–1070. Online URL: \url{http://dx.doi.org/10.1515/ling.2001.041} (Accessed 8 Jan 2016).
\end{styleCitaviBibliographyEntry}

\begin{styleCitaviBibliographyEntry}
Hay, Jennifer \& Baayen, R. Harald. 2002. Parsing and productivity. In Booij, Geert E. \& van Marle, Jaap (eds.), \textit{Yearbook of morphology 2001}. Dordrecht: Kluwer Academic Publishers, 203–235. Online URL: \url{http://www.sfs.uni-tuebingen.de/~hbaayen/publications/HayBaayenYoM2002.pdf} (Accessed 8 Jan 2016).
\end{styleCitaviBibliographyEntry}

\begin{styleCitaviBibliographyEntry}
Hayashi, Makoto \& Yoon, Kyung-Eun. 2006. A cross-linguistic exploration of demonstratives in interaction: With particular reference to the context of word-formulation trouble. \textit{Studies in Language} 30(3): 485–540. Online URL: \url{http://dx.doi.org/10.1075/sl.30.3.02hay} (Accessed 8 Jan 2016).
\end{styleCitaviBibliographyEntry}

\begin{styleCitaviBibliographyEntry}
Heine, Bernd \& Kuteva, Tania. 2002. \textit{World lexicon of grammaticalization}. Cambridge: Cambridge University Press.
\end{styleCitaviBibliographyEntry}

\begin{styleCitaviBibliographyEntry}
Helmbrecht, Johannes. 2004. Personal pronouns – Form, function, and grammaticalization. Erfurt: University of Erfurt. (Habitilationsschrift.)
\end{styleCitaviBibliographyEntry}

\begin{styleCitaviBibliographyEntry}
Helmbrecht, Johannes. 2011. Politeness distinctions in pronouns. In Haspelmath, Martin \& Dryer, Matthew S. \& Gil, David \& Comrie, Bernard (eds.), \textit{The world atlas of language structures}. München: Max Planck Digital Library, Chapter 45. Online URL: \url{http://wals.info/chapter/45} (Accessed 14 Dec 2013).
\end{styleCitaviBibliographyEntry}

\begin{styleCitaviBibliographyEntry}
Hengeveld, Kees. 1992. \textit{Non-verbal predication: Theory, typology, diachrony} (Functional Grammar Series 15). Berlin: Mouton de Gruyter.
\end{styleCitaviBibliographyEntry}

\begin{styleCitaviBibliographyEntry}
Himmelmann, Nikolaus P. 1991. \textit{The Philippine challenge to Universal Grammar} (Arbeitspapier 15). Köln: Institut für Sprachwissenschaft, Universität Köln. Online URL: \url{http://publikationen.ub.uni-frankfurt.de/files/24331/AP15NF-Himmelmann(1991).pdf} (Accessed 8 Jan 2016).
\end{styleCitaviBibliographyEntry}

\begin{styleCitaviBibliographyEntry}
Himmelmann, Nikolaus P. 1996. Demonstratives in narrative discourse: A taxonomy of universal uses. In Fox, Barbara A. (ed.), \textit{Studies in anaphora} (Typological Studies in Language 33). Amsterdam: John Benjamins Publishing Company, 205–255.
\end{styleCitaviBibliographyEntry}

\begin{styleCitaviBibliographyEntry}
Himmelmann, Nikolaus P. 2005. The Austronesian languages of Asia and Madagascar: Typological characteristics. In Adelaar, K. A. \& Himmelmann, Nikolaus P. (eds.), \textit{The Austronesian languages of Asia and Madagascar} (Routledge Language Family Series). London: Routledge, 110–181.
\end{styleCitaviBibliographyEntry}

\begin{styleCitaviBibliographyEntry}
Himmelmann, Nikolaus P. 2008. Lexical categories and voice in Tagalog. In Austin, Peter K. \& Musgrave, Simon (eds.), \textit{Voice and grammatical relations in Austronesian languages} (Studies in Constraint-Based Lexicalism). Stanford: Center for the Study of Language and Information, 247–293. Online URL: \url{http://www.uni-muenster.de/imperia/md/content/allgemeine_sprachwissenschaft/dozenten-unterlagen/himmelmann/himmelmann_lexical_categories_and_voice_in_tagalog.pdf} (Accessed 8 Jan 2016).
\end{styleCitaviBibliographyEntry}

\begin{styleCitaviBibliographyEntry}
Hoeksema, Jack \& Zwarts, Frans. 1991. Some remarks on focus adverbs. \textit{Journal of Semantics}(8): 51–70. Online URL: \url{http://dx.doi.org/10.1093/jos/8.1-2.51} (Accessed 8 Jan 2016).
\end{styleCitaviBibliographyEntry}

\begin{styleCitaviBibliographyEntry}
Huizinga, F. 1998. Relations between Tidore and the north coast of New Guinea in the nineteenth century. In Miedema, Jelle \& Ode, Cecilia \& Dam, Rien A. C. (eds.), \textit{Perspectives on the Bird’s Head of Irian Jaya, Indonesia: Proceedings of the Conference, Leiden, 13-17 October 1997}. Amsterdam: Rodopi, 385–419.
\end{styleCitaviBibliographyEntry}

\begin{styleCitaviBibliographyEntry}
Hymes, Dell H. 1974. \textit{Foundations in sociolinguistics: An ethnographic approach}. Philadelphia: University of Pennsylvania Press.
\end{styleCitaviBibliographyEntry}

\begin{styleCitaviBibliographyEntry}
Jacob, June A. \& Grimes, Barbara D. 2006. Developing a role for Kupang Malay: The contemporary politics of an eastern Indonesian creole. Paper presented at the Tenth International Conference on Austronesian Linguistics (10-ICAL). Puerto Princesa City, 17-20 January 2006. Online URL: \url{http://www.sil.org/asia/philippines/ical/papers/Jacob-Grimes Kupang Malay.pdf} (Accessed 8 Jan 2016).
\end{styleCitaviBibliographyEntry}

\begin{styleCitaviBibliographyEntry}
Jacob, June A. \& Grimes, Charles E. 2011. Aspect and directionality in Kupang Malay serial verb constructions: Calquing on the grammars of substrate languages. In Lefebvre, Claire (ed.), \textit{Creoles, their substrates, and language typology} (Typological Studies in Language 95). Amsterdam: John Benjamins Publishing Company, 337–366.
\end{styleCitaviBibliographyEntry}

\begin{styleCitaviBibliographyEntry}
Johannessen, Janne B. 2006. Just any pronoun anywhere? Pronouns and “new” demonstratives in Norwegian. In Solstad, Torgrim \& Grønn, Atle \& Dag, Haug (eds.), \textit{A festschrift for Kjell Johan Sæbø}. Oslo: University of Oslo, 91–106.
\end{styleCitaviBibliographyEntry}

\begin{styleCitaviBibliographyEntry}
Jones, Russell (ed.) 2007. \textit{Loan-words in Indonesian and Malay}. Leiden: KITLV Press. Online URL: \url{http://sealang.net/indonesia/lwim/} (Accessed 8 Jan 2016).
\end{styleCitaviBibliographyEntry}

\begin{styleCitaviBibliographyEntry}
Jurafsky, Dan. 1993. Universals in the semantics of the diminutive. In Guenter, Joshua S. \& Kaiser, Barbara A. \& Zoll, Cheryl C. \& Berkeley Linguistics Society (eds.), \textit{Proceedings of the nineteenth annual meeting of the Berkeley Linguistics Society, 12-15 February, 1993: General session and parasession on semantic typology and semantic universals}. Berkeley: Berkeley Linguistics Society, 423–436. Online URL: \url{http://dx.doi.org/10.3765/bls.v19i1.1531}; \url{http://journals.linguisticsociety.org/proceedings/index.php/BLS/article/view/1531/1314} (Accessed 8 Jan 2016).
\end{styleCitaviBibliographyEntry}

\begin{styleCitaviBibliographyEntry}
Kacandes, Irene. 1994. Narrative apostrophe: Reading, rhetoric, resistance in Michel Butor’s ‘La modification’ and Julio Cortazar’s “Graffiti” (Second-Person Narrative). \textit{Style} 28(3): 329–349. Online URL: \url{http://www.jstor.org/stable/42946255} (Accessed 8 Jan 2016).
\end{styleCitaviBibliographyEntry}

\begin{styleCitaviBibliographyEntry}
Karam, Francis X. 2000. Investigating mutual intelligibility and language coalescence. \textit{International Journal of the Sociology of Language} 146: 119–136.
\end{styleCitaviBibliographyEntry}

\begin{styleCitaviBibliographyEntry}
Kaufmann, Stefan. 2006. Conditionals. In Brown, Keith (ed.), \textit{Encyclopedia of language and linguistics}, 2\textsuperscript{nd} edn.. Amsterdam: Elsevier Science Ltd. 6–9.
\end{styleCitaviBibliographyEntry}

\begin{styleCitaviBibliographyEntry}
Keenan, Edward L. \& Comrie, Bernard. 1977. Noun phrase accessibility and Universal Grammar. \textit{Linguistic Inquiry} 8(1): 63–99. Online URL: \url{http://www.jstor.org/stable/4177973} (Accessed 8 Jan 2016).
\end{styleCitaviBibliographyEntry}

\begin{styleCitaviBibliographyEntry}
Kelman, Herbert C. 1971. Language as an aid and barrier to the involvement in the national system. In Rubin, Joan \& Jernudd, Björn H. (eds.), \textit{Can language be planned? Sociolinguistic theory and practice for developing nations.}. Honolulu: University of Hawai’i Press, 21–51.
\end{styleCitaviBibliographyEntry}

\begin{styleCitaviBibliographyEntry}
Kemmer, Suzanne. 1993. \textit{The middle voice}. Amsterdam: John Benjamins Publishing Company.
\end{styleCitaviBibliographyEntry}

\begin{styleCitaviBibliographyEntry}
Kennedy, Christopher. 2006. Comparatives, semantics. In Brown, Keith (ed.), \textit{Encyclopedia of language and linguistics}, 2\textsuperscript{nd} edn.. Amsterdam: Elsevier Science Ltd. 690–697.
\end{styleCitaviBibliographyEntry}

\begin{styleCitaviBibliographyEntry}
Kenstowicz, Michael J. 1994. \textit{Phonology in generative grammar}. Cambridge MA: Basil Blackwell Publishers.
\end{styleCitaviBibliographyEntry}

\begin{styleCitaviBibliographyEntry}
Kersten, J. P. F. 1948. \textit{Balische grammatica}. ‘s-Gravenhage: W. van Hoeve.
\end{styleCitaviBibliographyEntry}

\begin{styleCitaviBibliographyEntry}
Kilian-Hatz, Christa. 2006. Ideophones. In Brown, Keith (ed.), \textit{Encyclopedia of language and linguistics}, 2\textsuperscript{nd} edn.. Amsterdam: Elsevier Science Ltd. 508–512.
\end{styleCitaviBibliographyEntry}

\begin{styleCitaviBibliographyEntry}
Kim, Hyun \& Nussy, Christian G. \& Rumaropen, Ben E. W. \& Scott, Eleonora L. \& Scott, Graham R. 2007. A survey of Papuan Malay: An interim report. Paper presented at the Eleventh International Symposium on Malay/Indonesian Linguistics – ISMIL 11. Manokwari, 6-8 August 2007. Online URL: \url{http://email.eva.mpg.de/~gil/ismil/11/abstracts/KimShonRumaropen.pdf} (Accessed 8 Jan 2016).
\end{styleCitaviBibliographyEntry}

\begin{styleCitaviBibliographyEntry}
King, Peter. 2002. Morning Star Rising? Indonesia Raya and the New Papuan Nationalism. \textit{Indonesia} 73: 89–127. Online URL: \url{http://www.jstor.org/stable/3351470} (Accessed 8 Jan 2016).
\end{styleCitaviBibliographyEntry}

\begin{styleCitaviBibliographyEntry}
King, Peter. 2004. \textit{West Papua \& Indonesia since Suharto: Independence, autonomy or chaos? }Sydney: University of New South Wales Press.
\end{styleCitaviBibliographyEntry}

\begin{styleCitaviBibliographyEntry}
Kingsbury, Damien \& Aveling, Harry (eds.) 2002. \textit{Autonomy and disintegration Indonesia}. London: Routledge Curzon Press.
\end{styleCitaviBibliographyEntry}

\begin{styleCitaviBibliographyEntry}
Kiyomi, Setsuku. 2009. A new approach to reduplication: A semantic study of noun and verb reduplication in the Malayo-Polynesian languages. \textit{Linguistics} 33(6): 1145–1168. Online URL: \url{http://dx.doi.org/10.1515/ling.1995.33.6.1145} (Accessed 8 Jan 2016).
\end{styleCitaviBibliographyEntry}

\begin{styleCitaviBibliographyEntry}
Klamer, Marian. 2002. Typical features of Austronesian languages in Central/Eastern Indonesia. \textit{Oceanic Linguistics} 41(2): 363–383. Online URL: \url{http://dx.doi.org/10.1353/ol.2002.0007} (Accessed 8 Jan 2016).
\end{styleCitaviBibliographyEntry}

\begin{styleCitaviBibliographyEntry}
Klamer, Marian \& Ewing, Michael C. 2010. The languages of East Nusantara: An introduction. In Ewing, Michael C. \& Klamer, Marian (eds.), \textit{East Nusantara: Typological and areal analysis} (Pacific Linguistics 618). Canberra: College of Asia and the Pacific, School of Culture, History and Language, Pacific Linguistics, The Australian National University, 1–25. Online URL: \url{https://openaccess.leidenuniv.nl/bitstream/handle/1887/18306/Klamer & Ewing 2010 in Ewing & Klamer (eds).pdf?sequence=2} (Accessed 8 Jan 2016).
\end{styleCitaviBibliographyEntry}

\begin{styleCitaviBibliographyEntry}
Klamer, Marian \& Moro, Francesca R. 2013. ‘Give’-constructions in Heritage and Baseline Malay. Paper presented at the Workshop Structural Change in Heritage Languages. Noordwijkerhout, 23-25 January.
\end{styleCitaviBibliographyEntry}

\begin{styleCitaviBibliographyEntry}
Klamer, Marian \& Reesink, Gerard P. \& van Staden, Miriam. 2008. East Nusantara as a linguistic area. In Muysken, Pieter (ed.), \textit{From linguistic areas to areal linguistics} (Studies in Language Companion Series 90). Amsterdam: John Benjamins Publishing Company, 95–151.
\end{styleCitaviBibliographyEntry}

\begin{styleCitaviBibliographyEntry}
Kluge, Angela \& Rumaropen, Ben E. W. \& Aweta, Lodowik. 2014. \textit{Papuan Malay data – Word list}. Dallas: SIL International. Online URL: \url{http://www.sil.org/resources/archives/59649} (Accessed 8 Jan 2016).
\end{styleCitaviBibliographyEntry}

\begin{styleCitaviBibliographyEntry}
Krishnamurthy, Ramesh. 2006. Collocations. In Brown, Keith (ed.), \textit{Encyclopedia of language and linguistics}, 2\textsuperscript{nd} edn.. Amsterdam: Elsevier Science Ltd. 596–600.
\end{styleCitaviBibliographyEntry}

\begin{styleCitaviBibliographyEntry}
Kroeger, Paul R. 2005. \textit{Analyzing grammar: An introduction}. Cambridge: Cambridge University Press.
\end{styleCitaviBibliographyEntry}

\begin{styleCitaviBibliographyEntry}
Kroeger, Paul R. 2012. External vs. internal negation in Indonesian verbal clauses. Paper presented at the Twelfth International Conference on Austronesian Linguistics (12ICAL). Denpasar, 2-6 July 2012.
\end{styleCitaviBibliographyEntry}

\begin{styleCitaviBibliographyEntry}
Krupa, Viktor. 1967. \textit{Jazyk Maori}. Moskva: Izdatel’stvo “Nauka”.
\end{styleCitaviBibliographyEntry}

\begin{styleCitaviBibliographyEntry}
Kulikov, Leonid I. 2001. Causatives. In Haspelmath, Martin \& König, Ekkehard \& Oesterreicher, Wulf \& Raible, Wolfgang (eds.), \textit{Language typology and language universals: An international handbook. Volume 1} (Handbücher zur Sprach- und Kommunikationswissenschaft 20). Berlin: Walter de Gruyter, 886–898.
\end{styleCitaviBibliographyEntry}

\begin{styleCitaviBibliographyEntry}
Lakoff, Robin. 1974. Remarks on ‘this’ and ‘that’. \textit{Papers from the Regional Meetings, Chicago Linguistic Society} 10: 345–356.
\end{styleCitaviBibliographyEntry}

\begin{styleCitaviBibliographyEntry}
Lass, Roger. 1984. \textit{Phonology: An introduction to basic concepts} (Cambridge Textbooks in Linguistics). Cambridge: Cambridge University Press.
\end{styleCitaviBibliographyEntry}

\begin{styleCitaviBibliographyEntry}
Levinson, Stephen C. \& Wilkins, David P. (eds.) 2006. \textit{Grammars of space: Explorations in cognitive diversity} (Language, Culture and Cognition 6). Cambridge: Cambridge University Press.
\end{styleCitaviBibliographyEntry}

\begin{styleCitaviBibliographyEntry}
Lewis, M. Paul \& Simons, Gary F. \& Fennig, Charles D. (eds.) 2015a. \textit{Ethnologue: Languages of the world, Eighteenth edition}. Dallas: SIL International. Online URL: \url{http://www.ethnologue.com/} (Accessed 8 Jan 2016).
\end{styleCitaviBibliographyEntry}

\begin{styleCitaviBibliographyEntry}
Lewis, M. Paul \& Simons, Gary F. \& Fennig, Charles D. 2015b. The problem of language identification. In Lewis, M. Paul \& Simons, Gary F. \& Fennig, Charles D. (eds.), \textit{Ethnologue: Languages of the world, Eighteenth edition}. Dallas: SIL International. Online URL: \url{http://www.ethnologue.com/about/problem-language-identification} (Accessed 8 Jan 2016).
\end{styleCitaviBibliographyEntry}

\begin{styleCitaviBibliographyEntry}
Lichtenberk, Frantisek. 2000. Inclusory pronominals. \textit{Oceanic Linguistics} 39(1): 1–32. Online URL: \url{http://www.jstor.org/stable/3623215} (Accessed 8 Jan 2016).
\end{styleCitaviBibliographyEntry}

\begin{styleCitaviBibliographyEntry}
Lieber, Rochelle \& Štekauer, Pavol. 2009. Introduction: Status and definition of compounding. In Lieber, Rochelle \& Štekauer, Pavol (eds.), \textit{The Oxford handbook of compounding} (Oxford Handbooks in Linguistics). Oxford: Oxford University Press, 1–18.
\end{styleCitaviBibliographyEntry}

\begin{styleCitaviBibliographyEntry}
Lim, Sonny. 1988. Baba Malay: The language of the ‘Straits-born’ Chinese. In Lim, Sonny \& Soemarmo, Marmo P. K. \& Blust, Robert A. \& Kroeger, Paul R. (eds.), \textit{Papers in western Austronesian linguistics, No. 3} (Pacific Linguistics A-78). Canberra: Research School of Pacific Studies, The Australian National University.
\end{styleCitaviBibliographyEntry}

\begin{styleCitaviBibliographyEntry}
Litamahuputty, Betty. 1994. The use of biking and kasi in Ambonese Malay. \textit{Cakalele} 5: 11–31. Online URL: \url{http://scholarspace.manoa.hawaii.edu/bitstream/handle/10125/4137/UHM.CSEAS.Cakalele.v5.Litama.pdf} (Accessed 8 Jan 2016).
\end{styleCitaviBibliographyEntry}

\begin{styleCitaviBibliographyEntry}
Litamahuputty, Betty. 2012. \textit{Ternate Malay: Grammar and texts} (LOT Dissertation Series 307). Utrecht: LOT. Online URL: \url{http://www.lotpublications.nl/ternate-malay-ternate-malay-grammar-and-texts} (Accessed 8 Jan 2016).
\end{styleCitaviBibliographyEntry}

\begin{styleCitaviBibliographyEntry}
Loos, Eugene E. \& Anderson, Susan \& Day, Dwight H. \& Jordan, Paul C. \& Wingate, J. Douglas. 2003. \textit{Glossary of linguistic terms}. Dallas: SIL International Digital Resources. Online URL: \url{http://www.sil.org/linguistics/GlossaryOfLinguisticTerms/} (Accessed 8 Jan 2016).
\end{styleCitaviBibliographyEntry}

\begin{styleCitaviBibliographyEntry}
Lumi, Johnli H. 2007. The typology of plural personal pronouns in Papuan, Ambonese and Manado Malay: Malay varieties of Eastern Indonesian. Paper presented at the Eleventh International Symposium on Malay/Indonesian Linguistics – ISMIL 11. Manokwari, 6-8 August 2007. Online URL: \url{http://email.eva.mpg.de/~gil/ismil/11/abstracts/Lumi.pdf} (Accessed 8 Jan 2016).
\end{styleCitaviBibliographyEntry}

\begin{styleCitaviBibliographyEntry}
Lyons, Christopher. 1999. \textit{Definiteness} (Cambridge Textbooks in Linguistics). Cambridge: Cambridge University Press.
\end{styleCitaviBibliographyEntry}

\begin{styleCitaviBibliographyEntry}
Lyons, John. 1977. \textit{Semantics. Volume 2}. Cambridge: Cambridge University Press.
\end{styleCitaviBibliographyEntry}

\begin{styleCitaviBibliographyEntry}
MacDonald, R. R. 1976. \textit{Indonesian reference grammar}. Washington D.C.: Georgetown University Press.
\end{styleCitaviBibliographyEntry}

\begin{styleCitaviBibliographyEntry}
Maddieson, Ian. 2011a. Absence of common consonants. In Haspelmath, Martin \& Dryer, Matthew S. \& Gil, David \& Comrie, Bernard (eds.), \textit{The world atlas of language structures}. München: Max Planck Digital Library, Chapter 18. Online URL: \url{http://wals.info/chapter/18} (Accessed 8 Jan 2016).
\end{styleCitaviBibliographyEntry}

\begin{styleCitaviBibliographyEntry}
Maddieson, Ian. 2011b. Syllable structure. In Haspelmath, Martin \& Dryer, Matthew S. \& Gil, David \& Comrie, Bernard (eds.), \textit{The world atlas of language structures}. München: Max Planck Digital Library, Chapter 12. Online URL: \url{http://wals.info/chapter/12} (Accessed 8 Jan 2016).
\end{styleCitaviBibliographyEntry}

\begin{styleCitaviBibliographyEntry}
Malchukov, Andrej L. \& Haspelmath, Martin \& Comrie, Bernard. 2010. Ditransitive constructions: A typological overview. In Malchukov, Andrej L. \& Haspelmath, Martin \& Comrie, Bernard (eds.), \textit{Studies in ditransitive constructions: A comparative handbook}. Berlin: Mouton de Gruyter, 1–63.
\end{styleCitaviBibliographyEntry}

\begin{styleCitaviBibliographyEntry}
Marantz, Alec. 1982. Re reduplication. \textit{Linguistic Inquiry} 13(3): 435–482. Online URL: \url{http://www.jstor.org/stable/4178287} (Accessed 8 Jan 2016).
\end{styleCitaviBibliographyEntry}

\begin{styleCitaviBibliographyEntry}
Margetts, Anna \& Austin, Peter K. 2007. Three-participant events in the languages of the world: Towards a crosslinguistic typology. \textit{Linguistics} 45(3): 393–451. Online URL: \url{http://dx.doi.org/10.1515/LING.2007.014} (Accessed 8 Jan 2016).
\end{styleCitaviBibliographyEntry}

\begin{styleCitaviBibliographyEntry}
Masinambow, Eduard K. M. \& Haenen, Paul. 2002. \textit{Bahasa Indonesia dan bahasa daerah}. Jakarta: Yayasan Obor Indonesia.
\end{styleCitaviBibliographyEntry}

\begin{styleCitaviBibliographyEntry}
Mattes, Veronika. 2007. Types of reduplication: A case study of Bikol. Graz: Karl-Franzens-Universität Graz. (PhD dissertation.) Online URL: \url{http://reduplication.uni-graz.at/texte/Dissertation_gesamt.pdf} (Accessed 8 Jan 2016).
\end{styleCitaviBibliographyEntry}

\begin{styleCitaviBibliographyEntry}
Matushansky, Ora. 2008. On the linguistic complexity of proper names. \textit{Linguistics and Philosophy} 31(5): 573–627. Online URL: \url{http://dx.doi.org/10.1007/s10988-008-9050-1} (Accessed 8 Jan 2016).
\end{styleCitaviBibliographyEntry}

\begin{styleCitaviBibliographyEntry}
McWhorter, John H. 2001. The worlds simplest grammars are creole grammars. \textit{Linguistic Typology} 5(2-3): 125–166. Online URL: \url{http://dx.doi.org/10.1515/lity.2001.001} (Accessed 8 Jan 2016).
\end{styleCitaviBibliographyEntry}

\begin{styleCitaviBibliographyEntry}
McWhorter, John H. 2005. \textit{Defining creole}. New York: Oxford University Press.
\end{styleCitaviBibliographyEntry}

\begin{styleCitaviBibliographyEntry}
McWhorter, John H. 2007. \textit{Language interrupted: Signs of non-native acquisition in standard language grammars}. Oxford: Oxford University Press.
\end{styleCitaviBibliographyEntry}

\begin{styleCitaviBibliographyEntry}
Milner, George B. 1959. \textit{Fijian grammar}. Fiji: Government Press.
\end{styleCitaviBibliographyEntry}

\begin{styleCitaviBibliographyEntry}
Mintz, Malcolm W. 1994. \textit{A student’s grammar of Malay \& Indonesian}. Singapore: EPB Publishers.
\end{styleCitaviBibliographyEntry}

\begin{styleCitaviBibliographyEntry}
Mintz, Malcolm W. 2002. \textit{An Indonesian and Malay grammar for students} (Malay and Indonesian Language Collection), 2\textsuperscript{nd} edn. Perth: Indonesian/Malay Texts and Resources.
\end{styleCitaviBibliographyEntry}

\begin{styleCitaviBibliographyEntry}
Mithun, Marianne. 1988. The grammaticization of coordination. In Haiman, John \& Thompson, Sandra A. (eds.), \textit{Clause combining in grammar and discourse} (Typological Studies in Language 18). Amsterdam: John Benjamins Publishing Company, 331–359.
\end{styleCitaviBibliographyEntry}

\begin{styleCitaviBibliographyEntry}
Moeliono, Anton M. 1963. Ragam bahasa di Irian Barat. In Koentjaraningrat \& Bachtiar, Harsja W. (eds.), \textit{Penduduk Irian Barat} (Projek Penelitian Universitas Indonesia 102). Jakarta: Penerbitan Universitas, 28–38.
\end{styleCitaviBibliographyEntry}

\begin{styleCitaviBibliographyEntry}
Moravcsik, Edith A. 1971. Some cross-linguistic generalizations about yes-no questions and their answers. Stanford: Stanford University. (PhD dissertation.)
\end{styleCitaviBibliographyEntry}

\begin{styleCitaviBibliographyEntry}
Moravcsik, Edith A. 1978. Reduplicative constructions. In Greenberg, Joseph H. \& Ferguson, Charles A. \& Moravcsik, Andrew (eds.), \textit{Universals of human language. Volume 3: Word structure}. Stanford: Stanford University Press, 297–334.
\end{styleCitaviBibliographyEntry}

\begin{styleCitaviBibliographyEntry}
Moravcsik, Edith A. 2003. A semantic analysis of associative plurals. \textit{Studies in Language} 27(3): 469–503. Online URL: \url{http://dx.doi.org/10.1075/sl.27.3.02mor} (Accessed 8 Jan 2016).
\end{styleCitaviBibliographyEntry}

\begin{styleCitaviBibliographyEntry}
Moravcsik, Edith A. 2013. \textit{Introducing language typology} (Cambridge Introductions to Language and Linguistics). Cambridge: Cambridge University Press.
\end{styleCitaviBibliographyEntry}

\begin{styleCitaviBibliographyEntry}
Morley, G. D. 2000. \textit{Syntax in functional grammar: An introduction to lexicogrammar in systemic linguistics}. London: Continuum.
\end{styleCitaviBibliographyEntry}

\begin{styleCitaviBibliographyEntry}
Mosel, Ulrike. 2010. Ditransitive constructions and their alternatives in Teop. In Malchukov, Andrej L. \& Haspelmath, Martin \& Comrie, Bernard (eds.), \textit{Studies in ditransitive constructions: A comparative handbook}. Berlin: Mouton de Gruyter, 486–509.
\end{styleCitaviBibliographyEntry}

\begin{styleCitaviBibliographyEntry}
Moussay, Gérard. 1981. \textit{La langue Minangkebau}. Paris: Association Archipel.
\end{styleCitaviBibliographyEntry}

\begin{styleCitaviBibliographyEntry}
Mühlhäusler, Peter. 1996. \textit{Linguistic ecology: Language change and linguistic imperialism in the Pacific region}. London: Routledge.
\end{styleCitaviBibliographyEntry}

\begin{styleCitaviBibliographyEntry}
Mundhenk, A. N. 2002. \textit{Final particles in Melayu Papua}. Jayapura: Lembaga Alkitab Internasional.
\end{styleCitaviBibliographyEntry}

\begin{styleCitaviBibliographyEntry}
Nedjalkov, Vladimir P. 2007. Overview of the research: Definitions of terms, framework, and related issues. In Nedjalkov, Vladimir P. (ed.), \textit{Reciprocal constructions (Volume 1)} (Typological Studies in Language 71). Philadelphia: John Benjamins Publishing Company, 3–114.
\end{styleCitaviBibliographyEntry}

\begin{styleCitaviBibliographyEntry}
Nordhoff, Sebastian \& Hammarström, Harald \& Forkel, Robert \& Haspelmath, Martin (eds.) 2013. \textit{Glottolog 2.6}. Leipzig: Max Planck Institute for Evolutionary Anthropology. Online URL: \url{http://glottolog.org} (Accessed 8 Jan 2016).
\end{styleCitaviBibliographyEntry}

\begin{styleCitaviBibliographyEntry}
Nothofer, Bernd. 2009. Malay. In Brown, Keith \& Ogilvie, Sarah (eds.), \textit{Concise encyclopedia of languages of the world} (Concise Encyclopedias of Language and Linguistics). Amsterdam: Elsevier Science Ltd. 677–680.
\end{styleCitaviBibliographyEntry}

\begin{styleCitaviBibliographyEntry}
Overweel, Jeroen A. 1995. Appendix I: Sultans of Tidore and residents of Ternate, 1850-1909. In Overweel, Jeroen A. (ed.), \textit{Topics relating to Netherlands New Guinea in Ternate Residency memoranda of transfer and other assorted documents} (Irian Jaya Source Materials 13). Leiden: DSALCUL/IRIS, 137–138.
\end{styleCitaviBibliographyEntry}

\begin{styleCitaviBibliographyEntry}
Oxford University Press. 2000-. \textit{Oxford English dictionary online}. Oxford: Oxford University Press. Online URL: \url{http://www.oed.com/} (Accessed 8 Jan 2016).
\end{styleCitaviBibliographyEntry}

\begin{styleCitaviBibliographyEntry}
Paauw, Scott H. 2003. What is Bazaar Malay? Paper presented at the Seventh International Symposium on Malay/Indonesian Linguistics – ISMIL 7. Nijmegen, 27-29 June 2003. Online URL: \url{http://email.eva.mpg.de/~gil/ismil/7/abstracts/paauw.html} (Accessed 8 Jan 2016).
\end{styleCitaviBibliographyEntry}

\begin{styleCitaviBibliographyEntry}
Paauw, Scott H. 2005. Malay dialectology: A new analysis. Paper presented at the Niagara Linguistics Society. Buffalo, 1 October 2005.
\end{styleCitaviBibliographyEntry}

\begin{styleCitaviBibliographyEntry}
Paauw, Scott H. 2007. Malay contact varieties in eastern Indonesia. Paper presented at the Eleventh International Symposium on Malay/Indonesian Linguistics – ISMIL 11. Manokwari, 6-8 August 2007. Online URL: \url{http://email.eva.mpg.de/~gil/ismil/11/abstracts/Paauw.pdf} (Accessed 8 Jan 2016).
\end{styleCitaviBibliographyEntry}

\begin{styleCitaviBibliographyEntry}
Paauw, Scott H. 2009. The Malay contact varieties of eastern Indonesia: A typological comparison. Buffalo: State University of New York. (PhD dissertation.)
\end{styleCitaviBibliographyEntry}

\begin{styleCitaviBibliographyEntry}
Paauw, Scott H. 2013. The Malay varieties of eastern Indonesia - How, when and where they became isolating language varieties. Rochester: University of Rochester.
\end{styleCitaviBibliographyEntry}

\begin{styleCitaviBibliographyEntry}
Padgett, Jaye. 1994. Stricture and Nasal Place Assimilation. \textit{Natural Language and Linguistic Theory} 12(3): 465–513. Online URL: \url{http://www.jstor.org/stable/4047807} (Accessed 8 Jan 2016).
\end{styleCitaviBibliographyEntry}

\begin{styleCitaviBibliographyEntry}
Parker, Steve. 2008. Sound level protrusions as physical correlates of sonority. \textit{Journal of Phonetics} 36(1): 55–90. Online URL: \url{http://dx.doi.org/10.1016/j.wocn.2007.09.003} (Accessed 8 Jan 2016).
\end{styleCitaviBibliographyEntry}

\begin{styleCitaviBibliographyEntry}
Parkinson, Richard H. R. 1900. Die Berlinhafen-Section: Ein Beitrag zur Ethnographie der Neu-Guinea Küste. \textit{Internationales Archiv für Ethnographie} XIII: 18-54, Tafeln XV-XXII.
\end{styleCitaviBibliographyEntry}

\begin{styleCitaviBibliographyEntry}
Pawley, Andrew. 2005. The chequered career of the Trans New Guinea hypothesis. In Pawley, Andrew \& Attenborough, Robert \& Golson, Jack \& Hide, Robin (eds.), \textit{Papuan pasts: Cultural, linguistic and biological histories of Papuan-speaking peoples} (Pacific Linguistics 572). Canberra: Research School of Pacific and Asian Studies, The Australian National University, 67–107.
\end{styleCitaviBibliographyEntry}

\begin{styleCitaviBibliographyEntry}
Payne, Thomas E. 1997. \textit{Describing morphosyntax: A guide for field linguists}. Cambridge: Cambridge University Press.
\end{styleCitaviBibliographyEntry}

\begin{styleCitaviBibliographyEntry}
Pike, Kenneth L. 1967. \textit{Language in relation to a unified theory of the structure of human behavior} (Janua Linguarum: Series Maior 24), 2\textsuperscript{nd} edn. The Hague: Mouton de Gruyter.
\end{styleCitaviBibliographyEntry}

\begin{styleCitaviBibliographyEntry}
Plag, Ingo. 2006a. Productivity. In Brown, Keith (ed.), \textit{Encyclopedia of language and linguistics}, 2\textsuperscript{nd} edn.. Amsterdam: Elsevier Science Ltd. 121–128.
\end{styleCitaviBibliographyEntry}

\begin{styleCitaviBibliographyEntry}
Plag, Ingo. 2006b. Productivity. In Aarts, Bas \& McMahon, April (eds.), \textit{The handbook of English linguistics}. Malden: Basil Blackwell Publishers, 537–556.
\end{styleCitaviBibliographyEntry}

\begin{styleCitaviBibliographyEntry}
Podungge, Nurhayati. 2000. Slang in Papuan Malay (case study on the students at faculty of letters, the State University of Papua, Manokwari). Manokwari: Universitas Negeri Papua.
\end{styleCitaviBibliographyEntry}

\begin{styleCitaviBibliographyEntry}
Polinsky, Maria. 1998. A non-syntactic account of some asymmetries in the double-object construction. In Koenig, Jean-Pierre (ed.), \textit{Discourse and cognition: Bridging the gap}. Stanford: Center for the Study of Language and Information, 403–422.
\end{styleCitaviBibliographyEntry}

\begin{styleCitaviBibliographyEntry}
Prentice, David J. 1994. Manado Malay: Product and agent of language change. In Dutton, Thomas E. \& Tryon, Darrell T. (eds.), \textit{Language contact and change in the Austronesian world} (Trends in Linguistics: Studies and Monographs 77). Berlin: Mouton de Gruyter, 411–441.
\end{styleCitaviBibliographyEntry}

\begin{styleCitaviBibliographyEntry}
Quirk, Randolph \& Leech, Geoffrey N. \& Svartvik, Jan. 1972. \textit{A grammar of contemporary English}. Harlow: Longman.
\end{styleCitaviBibliographyEntry}

\begin{styleCitaviBibliographyEntry}
Ramos, Teresita V. 1971. \textit{Tagalog structures} (PALI Language Texts: Philippines 20). Honolulu: University of Hawai’i Press.
\end{styleCitaviBibliographyEntry}

\begin{styleCitaviBibliographyEntry}
Regier, Terry. 1994. \textit{A preliminary study of the semantics of reduplication} (Technical Report TR-94-019). Berkeley: International Computer Science Institute. Online URL: \url{ftp://ftp.icsi.berkeley.edu/pub/techreports/1994/tr-94-019.pdf} (Accessed 8 Jan 2016).
\end{styleCitaviBibliographyEntry}

\begin{styleCitaviBibliographyEntry}
Robidé van der Aa, Pieter J. B. C. 1879. \textit{Reizen naar Nederlandsch Nieuw-Guinea ondernomen op last der regeering van Nederlandsch-Indië in de jaren 1871, 1872, 1875-1876 door de heeren P. van der Crab en J. E. Teysmann, J. G. Coorengel en A. J. Langeveldt van Hemert en P. Swaan met geschied- en aardrijkskundige toelichtingen}. ‘s-Gravenhage: Martinus Nijhoff.
\end{styleCitaviBibliographyEntry}

\begin{styleCitaviBibliographyEntry}
Roehr, Dorian. 2005. Pronouns are determiners after all. In den Dikken, Marcel \& Tortora, Christina (eds.), \textit{The function of function words and functional categories} (Linguistik Aktuell / Linguistics Today 78). Philadelphia: John Benjamins Publishing Company, 251–285.
\end{styleCitaviBibliographyEntry}

\begin{styleCitaviBibliographyEntry}
Roosman, Raden S. 1982. Pidgin Malay as spoken in Irian Jaya. \textit{The Indonesian Quarterly} 10(2): 95–104.
\end{styleCitaviBibliographyEntry}

\begin{styleCitaviBibliographyEntry}
Ross, Malcolm D. 2001. Contact-induced change in Oceanic languages in North-West Melanesia. In Aikhenvald, Alexandra Y. \& Dixon, Robert M. W. (eds.), \textit{Areal diffusion and genetic inheritance: Problems in comparative linguistics}. Oxford: Oxford University Press, 134–166.
\end{styleCitaviBibliographyEntry}

\begin{styleCitaviBibliographyEntry}
Rowley, Charles D. 1972. \textit{The New Guinea villager: A retrospect from 1964}. Melbourne: F.W. Cheshire.
\end{styleCitaviBibliographyEntry}

\begin{styleCitaviBibliographyEntry}
Rubino, Carlo. 2011. Reduplication. In Haspelmath, Martin \& Dryer, Matthew S. \& Gil, David \& Comrie, Bernard (eds.), \textit{The world atlas of language structures}. München: Max Planck Digital Library, Chapter 27. Online URL: \url{http://wals.info/chapter/27} (Accessed 8 Jan 2016).
\end{styleCitaviBibliographyEntry}

\begin{styleCitaviBibliographyEntry}
Rudolph, Elisabeth. 1996. \textit{Contrast: Adversative and concessive relations and their expressions in English, German, Spanish, Portuguese on sentence and text level} (Research in Text Theory 23). Berlin: Walter de Gruyter.
\end{styleCitaviBibliographyEntry}

\begin{styleCitaviBibliographyEntry}
Rutherford, Danilyn. 2005. Frontiers of the lingua franca: Ideologies of the linguistic contact zone in Dutch New Guinea. \textit{Ethnos} 70(3): 387–412. Online URL: \url{http://dx.doi.org/10.1080/00141840500294490} (Accessed 8 Jan 2016).
\end{styleCitaviBibliographyEntry}

\begin{styleCitaviBibliographyEntry}
Sadock, Jerrold M. \& Zwicky, Arnold M. 1985. Speech act distinctions in syntax. In Shopen, Timothy (ed.), \textit{Language typology and syntactic description. Volume 1: Clause structure}. Cambridge: Cambridge University Press, 155–196.
\end{styleCitaviBibliographyEntry}

\begin{styleCitaviBibliographyEntry}
Samaun. 1979. The system of the contracted forms of the vernacular bahasa Indonesia in Jayapura, Irian Jaya: A project presented to the SEAMEO Regional Language Centre. Singapore: SEAMEO Regional Language Centre.
\end{styleCitaviBibliographyEntry}

\begin{styleCitaviBibliographyEntry}
Saragih, Chrisma F. 2012. The practical use of person reference in Papuan Malay. Nijmegen: Radboud University Nijmegen. (MA thesis.) Online URL: \url{http://www.ru.nl/publish/pages/518697/thesis_the_practical_use_of_person_reference_in_papuan_malay.docx} (Accessed 8 Jan 2016).
\end{styleCitaviBibliographyEntry}

\begin{styleCitaviBibliographyEntry}
Sawaki, Yusuf W. 2004. Some morpho-syntax notes on Melayu Papua. Manokwari: Universitas Negeri Papua.
\end{styleCitaviBibliographyEntry}

\begin{styleCitaviBibliographyEntry}
Sawaki, Yusuf W. 2005a. An agreement between the head of noun phrase and personal pronouns in Melayu-Papua syntax. Paper presented at the Ninth International Symposium on Malay/Indonesian Linguistics – ISMIL 9. Ambun Pagi, 27-29 July 2005.
\end{styleCitaviBibliographyEntry}

\begin{styleCitaviBibliographyEntry}
Sawaki, Yusuf W. 2005b. Melayu Papua: Tong pu bahasa. Manokwari: Universitas Negeri Papua.
\end{styleCitaviBibliographyEntry}

\begin{styleCitaviBibliographyEntry}
Sawaki, Yusuf W. 2007. Does passive exist in Melayu Papua? Paper presented at the Eleventh International Symposium on Malay/Indonesian Linguistics – ISMIL 11. Manokwari, 6-8 August 2007. Online URL: \url{http://email.eva.mpg.de/~gil/ismil/11/abstracts/Sawaki.pdf} (Accessed 8 Jan 2016).
\end{styleCitaviBibliographyEntry}

\begin{styleCitaviBibliographyEntry}
Schachter, Paul \& Shopen, Timothy. 2007. Parts-of-speech systems. In Shopen, Timothy (ed.), \textit{Language typology and syntactic description. Volume 1: Clause structure}, 2\textsuperscript{nd} edn.. Cambridge: Cambridge University Press, 1–60.
\end{styleCitaviBibliographyEntry}

\begin{styleCitaviBibliographyEntry}
Schütz, Albert J. \& Komaitai, Rusiate T. 1971. \textit{Spoken Fijian: an intensive course in Bauan Fijian, with grammatical notes and glossary} (PALI Language Texts: Melanesia 1). Honolulu: University of Hawai’i Press.
\end{styleCitaviBibliographyEntry}

\begin{styleCitaviBibliographyEntry}
Scott, Graham R. \& Kim, Hyun \& Rumaropen, Ben E. W. \& Scott, Eleonora L. \& Nussy, Christian G. \& Yumbi, Anita C. M. \& Cochran, Robert C. 2008. Tong pu bahasa: A preliminary report on some linguistic and sociolinguistic features of Papuan Malay. Sentani: SIL International Indonesia.
\end{styleCitaviBibliographyEntry}

\begin{styleCitaviBibliographyEntry}
Seiler, Walter. 1982. The spread of Malay to Kaiser-Wilhelmsland. In Kähler, Hans \& Carle, Rainer (eds.), \textit{GAVA*: Studies in Austronesian languages and cultures dedicated to Hans Kähler = Studien zu austronesischen Sprachen und Kulturen Hans Kähler gewidmet} (Veröffentlichungen des Seminars für Indonesische und Südseesprachen der Universität Hamburg 17). Berlin: Reimer, 67–85.
\end{styleCitaviBibliographyEntry}

\begin{styleCitaviBibliographyEntry}
Seiler, Walter. 1985. The Malay language of New Guinea. In Wurm, Stephen A. (ed.), \textit{Papers in Pidgin and Creole linguistics} (Pacific Linguistics A-72). Canberra: Research School of Pacific Studies, The Australian National University, 143–153.
\end{styleCitaviBibliographyEntry}

\begin{styleCitaviBibliographyEntry}
Shellabear, William G. 1904. \textit{A practical Malay grammar}, 2\textsuperscript{nd} edn. Singapore: The American Mission Press. Online URL: \url{http://ia600303.us.archive.org/18/items/practicalmalaygr00shelrich/practicalmalaygr00shelrich.pdf} (Accessed 8 Jan 2016).
\end{styleCitaviBibliographyEntry}

\begin{styleCitaviBibliographyEntry}
Siewierska, Anna. 2011. Gender distinctions in independent personal pronouns. In Haspelmath, Martin \& Dryer, Matthew S. \& Gil, David \& Comrie, Bernard (eds.), \textit{The world atlas of language structures}. München: Max Planck Digital Library, Chapter 44. Online URL: \url{http://wals.info/chapter/44} (Accessed 8 Jan 2016).
\end{styleCitaviBibliographyEntry}

\begin{styleCitaviBibliographyEntry}
Sigurðsson, Halldór Á. 2006. The Icelandic noun phrase: Central traits. \textit{Arkiv för Nordisk Filologi} 121: 193–236. Online URL: \url{http://lup.lub.lu.se/luur/download?func=downloadFile & recordOId=539373 & fileOId=625983} (Accessed 8 Jan 2016).
\end{styleCitaviBibliographyEntry}

\begin{styleCitaviBibliographyEntry}
SIL International. 1996–2008. IPA Help: A phonetics learning tool, Version 2.1 (SIL Computing). Dallas: SIL International. Online URL: \url{http://www-01.sil.org/computing/ipahelp/} (Accessed 8 Jan 2016).
\end{styleCitaviBibliographyEntry}

\begin{styleCitaviBibliographyEntry}
Silverstein, Michael. 1976. Hierarchy of features and ergativity. In Dixon, Robert M. W. (ed.), \textit{Grammatical categories in Australian languages} (Australian Institute of Aboriginal Studies Linguistic Series 22). Canberra: Humanities Press, 112–171. Online URL: \url{http://www.scribd.com/doc/36490729/silverstein-hierarchy-of-features-and-ergativity} (Accessed 8 Jan 2016).
\end{styleCitaviBibliographyEntry}

\begin{styleCitaviBibliographyEntry}
Silzer, Peter J. 1978. Notes on Irianese Indonesian. Jayapura: Summer Institute of Linguistics.
\end{styleCitaviBibliographyEntry}

\begin{styleCitaviBibliographyEntry}
Silzer, Peter J. 1979. Notes on Irianese Indonesian. Jayapura: Summer Institute of Linguistics; Universitas Cenderawasih.
\end{styleCitaviBibliographyEntry}

\begin{styleCitaviBibliographyEntry}
Slomanson, Peter. 2013. Sri Lankan Malay structure dataset. In Michaelis, Susanne M. \& Maurer, Philippe \& Haspelmath, Martin \& Huber, Magnus (eds.), \textit{The Atlas of Pidgin and Creole Language Structures (APiCS)}. Leipzig: Max Planck Institute for Evolutionary Anthropology. Online URL: \url{http://apics-online.info/contributions/66} (Accessed 8 Jan 2016).
\end{styleCitaviBibliographyEntry}

\begin{styleCitaviBibliographyEntry}
Smessaert, Hans \& ter Meulen, Alice G. B. 2004. Temporal reasoning with aspectual adverbs. \textit{Linguistics and Philosophy} 27(2): 209–261. Online URL: \url{http://dx.doi.org/10.1023/B:LING.0000016467.50422.63} (Accessed 8 Jan 2016).
\end{styleCitaviBibliographyEntry}

\begin{styleCitaviBibliographyEntry}
Sneddon, James N. 2003. \textit{The Indonesian language: Its history and role in modern society}. Sydney: UNSW Press.
\end{styleCitaviBibliographyEntry}

\begin{styleCitaviBibliographyEntry}
Sneddon, James N. 2006. \textit{Colloquial Jakartan Indonesian} (Pacific Linguistics 581). Canberra: Research School of Pacific and Asian Studies, The Australian National University.
\end{styleCitaviBibliographyEntry}

\begin{styleCitaviBibliographyEntry}
Sneddon, James N. 2010. \textit{Indonesian reference grammar}, 2\textsuperscript{nd} edn. Crows Nest: Allen and Unwin.
\end{styleCitaviBibliographyEntry}

\begin{styleCitaviBibliographyEntry}
Sommerfeldt, Karl-Ernst \& Schreiber, Herbert. 1983. \textit{Wörterbuch zur Valenz und Distribution der Substantive}. Leipzig: VEB Bibliographisches Institut.
\end{styleCitaviBibliographyEntry}

\begin{styleCitaviBibliographyEntry}
Song, Jae J. 2006. Causatives: Semantics. In Brown, Keith (ed.), \textit{Encyclopedia of language and linguistics}, 2\textsuperscript{nd} edn.. Amsterdam: Elsevier Science Ltd. 265–269.
\end{styleCitaviBibliographyEntry}

\begin{styleCitaviBibliographyEntry}
Song, Jae J. 2011. Nonperiphrastic causative constructions. In Haspelmath, Martin \& Dryer, Matthew S. \& Gil, David \& Comrie, Bernard (eds.), \textit{The world atlas of language structures}. München: Max Planck Digital Library, Chapter 111. Online URL: \url{http://wals.info/chapter/111} (Accessed 8 Jan 2016).
\end{styleCitaviBibliographyEntry}

\begin{styleCitaviBibliographyEntry}
Stassen, Leon. 2000. $AN$D-languages and WITH-languages. \textit{Linguistic Typology} 4: 1–54. Online URL: \url{http://dx.doi.org/10.1515/lity.2000.4.1.1} (Accessed 8 Jan 2016).
\end{styleCitaviBibliographyEntry}

\begin{styleCitaviBibliographyEntry}
Stassen, Leon. 2009. \textit{Predicative possession} (Oxford Studies in Typology and Linguistic Theory). Oxford: Oxford University Press.
\end{styleCitaviBibliographyEntry}

\begin{styleCitaviBibliographyEntry}
Stassen, Leon. 2011a. Noun phrase conjunction. In Haspelmath, Martin \& Dryer, Matthew S. \& Gil, David \& Comrie, Bernard (eds.), \textit{The world atlas of language structures}. München: Max Planck Digital Library, Chapter 63. Online URL: \url{http://wals.info/chapter/63} (Accessed 8 Jan 2016).
\end{styleCitaviBibliographyEntry}

\begin{styleCitaviBibliographyEntry}
Stassen, Leon. 2011b. Predicative possession. In Haspelmath, Martin \& Dryer, Matthew S. \& Gil, David \& Comrie, Bernard (eds.), \textit{The world atlas of language structures}. München: Max Planck Digital Library, Chapter 117. Online URL: \url{http://wals.info/chapter/117} (Accessed 8 Jan 2016).
\end{styleCitaviBibliographyEntry}

\begin{styleCitaviBibliographyEntry}
Steinhauer, Hein. 1983. Notes on the Malay of Kupang. In Collins, James T. (ed.), \textit{Studies in Malay dialects: Part II} (NUSA – Linguistic Studies of Indonesian and other Languages in Indonesia 17). Jakarta: Badan Penyelenggara Seri NUSA, Universitas Katolik Atma Jaya, 42–64.
\end{styleCitaviBibliographyEntry}

\begin{styleCitaviBibliographyEntry}
Steinhauer, Hein. 1991. Malay in east Indonesia: The case of Larantuka (Flores). In Steinhauer, Hein (ed.), \textit{Papers in Austronesian linguistics. Volume 1} (Pacific Linguistics A-81). Canberra: Research School of Pacific Studies, The Australian National University, 177–195.
\end{styleCitaviBibliographyEntry}

\begin{styleCitaviBibliographyEntry}
Stoel, Ruben. 2005. \textit{Focus in Manado Malay: Grammar, particles, and intonation}. Leiden: Department of Languages and Cultures of South-East Asia and Oceania, Leiden University.
\end{styleCitaviBibliographyEntry}

\begin{styleCitaviBibliographyEntry}
Suharno, Ignatius. 1979. Some notes on the teaching of Standard Indonesian to speakers of Irianese Indonesian. In Tarumingkeng, Rudolf C. (ed.), \textit{Irian, Bulletin of Irian Jaya Development} (8.1). Jayapura: Institute for Anthropology, Universitas Cenderawasih, 3–32. Online URL: \url{http://www.papuaweb.org/dlib/irian/8-1.pdf} (Accessed 8 Jan 2016).
\end{styleCitaviBibliographyEntry}

\begin{styleCitaviBibliographyEntry}
Suharno, Ignatius. 1981. The reductive system of an Indonesian dialect – A study of Irian Jaya Case. Paper presented at the Third International Conference on Austronesian Linguistics. National Center for Language Development, Ministry of Education and Culture, Denpasar, 19-24 January 1981.
\end{styleCitaviBibliographyEntry}

\begin{styleCitaviBibliographyEntry}
Sutri Narfafan \& Donohue, Mark. under review. Papuan Malay. \textit{Journal of the International Phonetic Association}.
\end{styleCitaviBibliographyEntry}

\begin{styleCitaviBibliographyEntry}
Tadmor, Uri. 2002. Language contact and the homeland of Malay. Paper presented at the Sixth International Symposium on Malay/Indonesian Linguistics – ISMIL 6. Bintan, 5 August 2002. Online URL: \url{http://lingweb.eva.mpg.de/jakarta/docs/homeland_handout.pdf} (Accessed 8 Jan 2016).
\end{styleCitaviBibliographyEntry}

\begin{styleCitaviBibliographyEntry}
Tadmor, Uri. 2009a. Indonesian vocabulary. In Haspelmath, Martin \& Tadmor, Uri (eds.), \textit{World Loanword Database (WOLD)}. München: Max Planck Digital Library. Online URL: \url{http://wold.clld.org/vocabulary/27} (Accessed 8 Jan 2016).
\end{styleCitaviBibliographyEntry}

\begin{styleCitaviBibliographyEntry}
Tadmor, Uri. 2009b. Malay-Indonesian. In Comrie, Bernard (ed.), \textit{The world’s major languages}, 2\textsuperscript{nd} edn.. London: Routledge, 791–818.
\end{styleCitaviBibliographyEntry}

\begin{styleCitaviBibliographyEntry}
Taylor, Paul M. 1983. North Moluccan Malay: Notes on a “substandard” dialect of Indonesian. In Collins, James T. (ed.), \textit{Studies in Malay dialects: Part II} (NUSA – Linguistic Studies of Indonesian and other Languages in Indonesia 17). Jakarta: Badan Penyelenggara Seri NUSA, Universitas Katolik Atma Jaya, 14–27.
\end{styleCitaviBibliographyEntry}

\begin{styleCitaviBibliographyEntry}
Tebay, Neles. 2005. \textit{West Papua: The struggle for peace with justice}. London: Catholic Institute for International Relations.
\end{styleCitaviBibliographyEntry}

\begin{styleCitaviBibliographyEntry}
Teeuw, Andries. 1961. \textit{A critical survey of studies on Malay and Bahasa Indonesia} (Koninklijk Instituut voor Taal-, Land- en Volkenkunde: Bibliographical Series 5). ‘S-Gravenhage: Martinus Nijhoff.
\end{styleCitaviBibliographyEntry}

\begin{styleCitaviBibliographyEntry}
Teutscher, Henk J. 1954. Wat gaat men straks op Nieuw-Guinea spreken? \textit{Wending – Manndblad voor Evangelie en Cultuur} 9: 113–129.
\end{styleCitaviBibliographyEntry}

\begin{styleCitaviBibliographyEntry}
The International Phonetic Association. 2005. IPA chart (available under a Creative Commons Attribution-Sharealike 3.0 Unported License). S.l.: The International Phonetic Association. Online URL: \url{https://www.internationalphoneticassociation.org/content/full-ipa-chart} (Accessed 8 Jan 2016).
\end{styleCitaviBibliographyEntry}

\begin{styleCitaviBibliographyEntry}
Thomason, Sarah G. \& Kaufman, Terrence. 1988. \textit{Language contact, creolization, and genetic linguistics}. Berkeley: University of California Press.
\end{styleCitaviBibliographyEntry}

\begin{styleCitaviBibliographyEntry}
Thompson, Sandra A. \& Longacre, Robert R. \& Hwang, Shin Ja J. 2007. Adverbial clauses. In Shopen, Timothy (ed.), \textit{Language typology and syntactic description. Volume 2: Complex constructions}, 2\textsuperscript{nd} edn.. Cambridge: Cambridge University Press, 237–300.
\end{styleCitaviBibliographyEntry}

\begin{styleCitaviBibliographyEntry}
Timmer, Jaap. 2002. A bibliographic essay on the southwestern Kepala Burung (Bird’s Head, Doberai) of Papua. \textit{Papuaweb’s Annotated Bibliographies}: 26 p. Online URL: \url{http://www.papuaweb.org/bib/abib/jt-kepala.pdf} (Accessed 8 Jan 2016).
\end{styleCitaviBibliographyEntry}

\begin{styleCitaviBibliographyEntry}
Topping, Donald M. \& Ogo, Pedro. 1960. \textit{Spoken Chamorro: An intensive language course with grammatical notes and glossary}. Honolulu: University of Hawai’i Press.
\end{styleCitaviBibliographyEntry}

\begin{styleCitaviBibliographyEntry}
Van der Eng, Pierre. 2004. Irian Jaya (West Irian). In Ooi Keat Gin (ed.), \textit{Southeast Asia: A historical encyclopedia, from Angkor Wat to East Timor}. Santa Barbara: ABC-CLIO, 663–665.
\end{styleCitaviBibliographyEntry}

\begin{styleCitaviBibliographyEntry}
Van Durme, Karen \& Institut for Sprog og Kommunikation. 1997. \textit{The valency of nouns} (Odense Working Papers in Language and Communication 15). Odense: Institute of Language and Communication, Odense University.
\end{styleCitaviBibliographyEntry}

\begin{styleCitaviBibliographyEntry}
Van Hasselt, Frans J. F. 1926. \textit{In het land van de Papoea’s}. Utrecht: Kemink \& Zoon.
\end{styleCitaviBibliographyEntry}

\begin{styleCitaviBibliographyEntry}
Van Hasselt, Frans J. F. 1936. Het Noemfoorsch als Eenheidstaal op het Noordwestelijk Deel van Nieuw Guinea. \textit{Tijdschrift Nieuw Guinea} 1: 114–117.
\end{styleCitaviBibliographyEntry}

\begin{styleCitaviBibliographyEntry}
Van Klinken, Catharina L. 1999. \textit{A grammar of the Fehan dialect of Tetun: An Austronesian language of West Timor} (Pacific Linguistics 155). Canberra: Research School of Pacific and Asian Studies, The Australian National University.
\end{styleCitaviBibliographyEntry}

\begin{styleCitaviBibliographyEntry}
Van Minde, Don. 1997. \textit{Malayu Ambong: Phonology, morphology, syntax}. Leiden: Department of Languages and Cultures of South-East Asia and Oceania, Leiden University.
\end{styleCitaviBibliographyEntry}

\begin{styleCitaviBibliographyEntry}
Van Oldenborgh, J. 1995. Memorie van overgave van het bestuur over de Residentie Ternate door den aftredenden Resident J. van Oldenborgh aan den optredenden \citet{Resident1895}. In Overweel, Jeroen A. (ed.), \textit{Topics relating to Netherlands New Guinea in Ternate Residency memoranda of transfer and other assorted documents} (Irian Jaya Source Materials 13). Leiden: DSALCUL/IRIS, 80–84.
\end{styleCitaviBibliographyEntry}

\begin{styleCitaviBibliographyEntry}
Van Valin, Robert D. 2001. \textit{An introduction to syntax}. Cambridge: Cambridge University Press.
\end{styleCitaviBibliographyEntry}

\begin{styleCitaviBibliographyEntry}
Van Valin, Robert D. 2005. \textit{Exploring the syntax-semantics interface}. Cambridge: Cambridge University Press.
\end{styleCitaviBibliographyEntry}

\begin{styleCitaviBibliographyEntry}
Van Valin, Robert D. \& LaPolla, Randy J. 1997. \textit{Syntax: Structure, meaning, and function}. Cambridge: Cambridge University Press.
\end{styleCitaviBibliographyEntry}

\begin{styleCitaviBibliographyEntry}
Van Velzen, Paul. 1995. Some notes on the variety of Malay used in Serui and vicinity. In Baak, Connie \& Bakker, Mary \& van der Meij, Dick (eds.), \textit{Tales from a concave world: Liber amicorum Bert Voorhoeve}. Leiden: Leiden University, 311-343 (265-296).
\end{styleCitaviBibliographyEntry}

\begin{styleCitaviBibliographyEntry}
Voorhoeve, Clemens L. 1983. Some observations on North-Moluccan Malay. In Collins, James T. (ed.), \textit{Studies in Malay dialects: Part II} (NUSA – Linguistic Studies of Indonesian and other Languages in Indonesia 17). Jakarta: Badan Penyelenggara Seri NUSA, Universitas Katolik Atma Jaya, 1–13.
\end{styleCitaviBibliographyEntry}

\begin{styleCitaviBibliographyEntry}
Walker, Roland W. 1982. Language use at Namatota: A sociolinguistic profile. In Halim, Amran \& Carrington, Lois \& Wurm, Stephen A. (eds.), \textit{Papers from the Third International Conference on Austronesian Linguistics} (Pacific Linguistics C-76). Canberra: Research School of Pacific Studies, The Australian National University, 79–94.
\end{styleCitaviBibliographyEntry}

\begin{styleCitaviBibliographyEntry}
Wallace, Alfred R. 1890. \textit{The Malay Archipelago and the land of the orang-utan and the bird of paradise: A narrative of travel with studies of man and nature}, 10\textsuperscript{th} edn. London: MacMillian and Co. Online URL: \url{https://archive.org/details/malayarchipelag04wallgoog} (Accessed 8 Jan 2016).
\end{styleCitaviBibliographyEntry}

\begin{styleCitaviBibliographyEntry}
Warami, Hugo. 2003. \textit{Wacana humor (mob) dialek Melayu Papua: Kumpulan pojok MOB “Warung Papeda” masyarakat Papua. (Surat Kabar Harian Cenderawasih Pos, 1994-200)}. Manokwari: Fakultas Sastra, Universitas Negeri Papua.
\end{styleCitaviBibliographyEntry}

\begin{styleCitaviBibliographyEntry}
Warami, Hugo. 2004. \textit{Bentuk dan peran humor (mob) dalam masyarakat Papua}. Manokwari: Fakultas Sastra, Universitas Negeri Papua.
\end{styleCitaviBibliographyEntry}

\begin{styleCitaviBibliographyEntry}
Warami, Hugo. 2005. Bentuk partikel bahasa Melayu Papua. \textit{Linguistika} 6(1): 87–112.
\end{styleCitaviBibliographyEntry}

\begin{styleCitaviBibliographyEntry}
Weinreich, Uriel. 1953. \textit{Languages in contact: Findings and problems} (Publications of the Linguistic Circle of New York 1). New York: Linguistic Circle of New York.
\end{styleCitaviBibliographyEntry}

\begin{styleCitaviBibliographyEntry}
Wichmann, Arthur. 1917. \textit{Bericht \"{u}ber eine im Jahre 1903 ausgef\"{u}hrte Reise nach Neu-Guinea} (Nova Guinea: R\'{e}sultats des exp\'{e}ditions scientifiques \`{a} la Nouvelle Guin\'{e}e en 1903 sous les auspices de Arthur Wichmann 4). Leiden: E. J. Brill.
\end{styleCitaviBibliographyEntry}

\begin{styleCitaviBibliographyEntry}
Wiltshire, Caroline \& Marantz, Alec. 1978. Reduplication. In Greenberg, Joseph H. \& Ferguson, Charles A. \& Moravcsik, Andrew (eds.), \textit{Universals of human language. Volume 3: Word structure}. Stanford: Stanford University Press, 557–567.
\end{styleCitaviBibliographyEntry}

\begin{styleCitaviBibliographyEntry}
Winstedt, Richard O. 1913. \textit{Malay grammar}. Oxford: Oxford University Press. Online URL: \url{http://ia600303.us.archive.org/18/items/malaygrammar00winsrich/malaygrammar00winsrich.pdf} (Accessed 8 Jan 2016).
\end{styleCitaviBibliographyEntry}

\begin{styleCitaviBibliographyEntry}
Winstedt, Richard O. 1938. \textit{Colloquial Malay: A simple grammar with conversations}, 4\textsuperscript{th} edn. Singapore: Kelly and Walsh, Limited.
\end{styleCitaviBibliographyEntry}

\begin{styleCitaviBibliographyEntry}
Wischer, Ilse. 2006. Grammaticalization. In Brown, Keith (ed.), \textit{Encyclopedia of language and linguistics}, 2\textsuperscript{nd} edn.. Amsterdam: Elsevier Science Ltd. 129–136.
\end{styleCitaviBibliographyEntry}

\begin{styleCitaviBibliographyEntry}
Wolff, John U. 1988. The contribution of Banjar Masin Malay to the reconstruction of Proto-Malay. In Ahmad, Mohammed T. \& Zain, Zaini M. (eds.), \textit{Rekonstruksi dan cabang-cabang bahasa Melayu Induk} (Siri Monograf Sejarah Bahasa Melayu). Kuala Lumpur: Dewan Bahasa dan Pustaka, 85–98.
\end{styleCitaviBibliographyEntry}

\begin{styleCitaviBibliographyEntry}
Woorden.org MMXI. 2010-. \textit{Woorden Nederlandse Taal}. Niebert: Woorden.org MMXI. Online URL: \url{http://www.woorden.org/woord/} (Accessed 8 Jan 2016).
\end{styleCitaviBibliographyEntry}

\begin{styleCitaviBibliographyEntry}
Yap, Foong H. 2007. On native and contact-induced grammaticalization: The case of Malay empunya. Hong Kong: Department of Linguistics and Modern Languages, Chinese University of Hong Kong. Online URL: \url{https://www.researchgate.net/publication/237259428_On_native_and_contact-induced_grammaticalization_The_case_of_Malay_empunya} (Accessed 8 Jan 2016).
\end{styleCitaviBibliographyEntry}

\begin{styleCitaviBibliographyEntry}
Yap, Foong H. \& Matthews, Stephen \& Horie, Kaoru. 2004. From pronominalizer to pragmatic marker – Implications for unidirectionality from a crosslinguistic perspective. In Fischer, Olga \& Norde, Muriel \& Perridon, Harry (eds.), \textit{Up and down the cline: The nature of grammaticalization} (Typological Studies in Language 59). Amsterdam: John Benjamins Publishing Company, 137–168.
\end{styleCitaviBibliographyEntry}

\begin{styleCitaviBibliographyEntry}
Yembise, Yohana S. 2011. Linguistic and cultural variations as barriers to the TEFL settings in Papua. \textit{TEFLIN Journal} 22(2): 201–225. Online URL: \url{http://journal.teflin.org/index.php/journal/article/download/27/28} (Accessed 8 Jan 2016).
\end{styleCitaviBibliographyEntry}

\begin{styleCitaviBibliographyEntry}
Zöller, Hugo. 1891. \textit{Deutsch-Neuguinea und meine Ersteigung des Finisterre-Gebirges: Eine Schilderung des ersten erfolgreichen Vordringens zu den Hochgebirgen Inner-Neuguineas, der Natur des Landes, der Sitten der Eingeborenen und des gegenwärtigen Standes der deutschen Kolonisationstätigkeit in Kaiser-Wilhelms-Land, Bismarck- und Salomo-Archipel, nebst einem Wortverzeichnis von 46 Papua-Sprachen}. Stuttgart: Union Deutsche Verlagsgesellschaft.
\end{styleCitaviBibliographyEntry}

\begin{styleCitaviBibliographyEntry}
Zsiga, Elizabeth. 2006. Assimilation. In Brown, Keith (ed.), \textit{Encyclopedia of language and linguistics}, 2\textsuperscript{nd} edn.. Amsterdam: Elsevier Science Ltd. 553–558.
\end{styleCitaviBibliographyEntry}

\begin{styleCitaviBibliographyEntry}
Zwanenburg, Wiecher. 2000. Correspondence between formal and semantic relations. In Booij, Geert E. \& Lehmann, Christian \& Mugdan, Joachim (eds.), \textit{Morphologie: Ein internationales Handbuch zur Flexion und Wortbildung = Morphology: An international handbook on inflection and word-formation. Volume 1} (Handbooks of Linguistics and Communication Science 17.1). Berlin: Mouton de Gruyter, 840–850.
\end{styleCitaviBibliographyEntry}
